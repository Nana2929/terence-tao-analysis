\chapter{Introduction}

This is a note of the textbook "Analysis I", 3rd edition by Terence Tao.
The note covers all axioms, theories, lemma, propositions and remark appear in the book.
Additionally, I also wrote proofs if there are none in the book.

\begin{note}\label{circularity}
\emph{circularity}: using an advanced fact to prove a more elementary fact, and then later using the elementary fact to prove the advanced fact.
When do a mathematics proofs, one should avoid \emph{circularity}.
\end{note}

\begin{note}
From a logical point of view, there is no difference between a lemma, proposition, theorem, or corollary - they are all claims waiting to be proved.
However, we use these terms to suggest different levels of importance and difficulty.
A lemma is an easily proved claim which is helpful for proving other propositions and theorems, but is usually not particularly interesting in its own right.
A proposition is a statement which is interesting in its own right, while a theorem is a more important statement than a proposition which says something definitive on the subject, and often takes more effort to prove than a proposition or lemma.
A corollary is a quick consequence of a proposition or theorem that was proven recently.
\end{note}