\section{Basic definitions}\label{sec 10}

\begin{note}
    We can now define derivatives analyti cally, using limits, in contrast to the geometric definition of derivatives, which uses tangents.
    The advantage of working analytically is that
    (a) we do not need to know the axioms of geometry, and
    (b) these definitions can be modified to handle functions of several variables, or functions whose values are vectors instead of scalar.
    Furthermore, one's geometric intuition becomes difficult to rely on once one has more than three dimensions in play.
    (Conversely, one can use one's experience in analytic rigour to extend one's geometric intuition to such abstract settings;
    as mentioned earlier, the two viewpoints complement rather than oppose each other.)
\end{note}

\begin{definition}[Differentiability at a point]\label{10.1.1}
    Let \(X\) be a subset of \(\mathbf{R}\), and let \(x_0 \in X\) be an element of \(X\) which is also a limit point of \(X\).
    Let \(f : X \to \mathbf{R}\) be a function.
    If the limit
    \[
        \lim_{x \to x_0 ; x \in X \setminus \{x_0\}} \frac{f(x) - f(x_0)}{x - x_0}
    \]
    converges to some real number \(L\), then we say that \(f\) is \emph{differentiable at \(x_0\) on \(X\) with derivative \(L\)}, and write \(f'(x_0) \coloneqq L\).
    If the limit does not exist, or if \(x_0\) is not an element of \(X\) or not a limit point of \(X\), we leave \(f'(x_0)\) undefined, and say that \(f\) is \emph{not differentiable at \(x_0\) on \(X\)}.
\end{definition}