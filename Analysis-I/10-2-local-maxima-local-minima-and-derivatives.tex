\section{Local maxima, local minima, and derivatives}\label{sec 10.2}

\begin{definition}[Local maxima and minima]\label{10.2.1}
    Let \(X\) be a subset of \(\mathbf{R}\), and let \(f : X \to \mathbf{R}\) be a function, and let \(x_0 \in X\).
    We say that \(f\) attains a \emph{local maximum} at \(x_0\) iff there exists a \(\delta > 0\) such that the restriction \(f|_{X \cap (x_0 - \delta, x_0 + \delta)}\) of \(f\) to \(X \cap (x_0 - \delta, x_0 + \delta)\) attains a maximum at \(x_0\).
    We say that \(f\) attains a \emph{local minimum} at \(x_0\) iff there exists a \(\delta > 0\) such that the restriction \(f|_{X \cap (x_0 - \delta, x_0 + \delta)}\) of \(f\) to \(X \cap (x_0 - \delta, x_0 + \delta)\) attains a minimum at \(x_0\).
\end{definition}