\section{Monotone functions and derivatives}\label{sec 10.3}

\begin{proposition}\label{10.3.1}
    Let \(X\) be a subset of \(\mathbf{R}\), let \(x_0 \in X\) be a limit point of \(X\), and let \(f : X \to \mathbf{R}\) be a function.
    If \(f\) is monotone increasing and \(f\) is differentiable at \(x_0\), then \(f'(x_0) \geq 0\).
    If f is monotone decreasing and \(f\) is differentiable at \(x_0\), then \(f'(x_0) \leq 0\).
\end{proposition}

\begin{proof}
    First suppose that \(f\) is monotone increasing.
    Since
    \begin{align*}
                 & \forall\ x \in X \setminus \{x_0\}, x \neq x_0                                      \\
        \implies & \begin{cases}
            x < x_0 \\
            x > x_0 \\
        \end{cases}                                                           \\
        \implies & \begin{cases}
            (x - x_0 < 0) \land \big(f(x) - f(x_0) \leq 0\big) \\
            (x - x_0 > 0) \land \big(f(x) - f(x_0) \geq 0\big) \\
        \end{cases}                      & \text{(by Definition \ref{9.8.1})} \\
        \implies & \frac{f(x) - f(x_0)}{x - x_0} \geq 0,
    \end{align*}
    by Proposition \ref{9.3.14} we have \(f'(x_0) \geq 0\).

    Now suppose that \(f\) is monotone decreasing.
    Since
    \begin{align*}
                 & \forall\ x \in X \setminus \{x_0\}, x \neq x_0                                      \\
        \implies & \begin{cases}
            x < x_0 \\
            x > x_0 \\
        \end{cases}                                                           \\
        \implies & \begin{cases}
            (x - x_0 < 0) \land \big(f(x) - f(x_0) \geq 0\big) \\
            (x - x_0 > 0) \land \big(f(x) - f(x_0) \leq 0\big) \\
        \end{cases}                      & \text{(by Definition \ref{9.8.1})} \\
        \implies & \frac{f(x) - f(x_0)}{x - x_0} \leq 0,
    \end{align*}
    by Proposition \ref{9.3.14} we have \(f'(x_0) \leq 0\).
\end{proof}

\begin{remark}\label{10.3.2}
    We have to assume that \(f\) is differentiable at \(x_0\);
    There exist monotone functions which are not always differentiable, and of course if \(f\) is not differentiable at \(x_0\) we cannot possibly conclude that \(f'(x_0) \geq 0\) or \(f'(x_0) \leq 0\).
\end{remark}

\begin{note}
    One might naively guess that if \(f\) were strictly monotone increasing, and \(f\) was differentiable at \(x_0\), then the derivative \(f'(x_0)\) would be strictly positive instead of merely non-negative.
    Unfortunately, this is not always the case.
\end{note}

\begin{proposition}\label{10.3.3}
    Let \(a < b\), and let \(f : [a, b] \to \mathbf{R}\) be a differentiable function.
    If \(f'(x) > 0\) for all \(x \in [a, b]\), then \(f\) is strictly monotone increasing.
    If \(f'(x) < 0\) for all \(x \in [a, b]\), then \(f\) is strictly monotone decreasing.
    If \(f'(x) = 0\) for all \(x \in [a, b]\), then \(f\) is a constant function.
\end{proposition}

\begin{proof}
    We first show that if \(f'(x) > 0\) for all \(x \in [a, b]\), then \(f\) is strictly monotone increasing.
    Let \(x_1, x_2 \in [a, b]\) and \(x_1 < x_2\).
    Then we know that \([x_1, x_2] \subseteq [a, b]\) and \((x_1, x_2) \subseteq (a, b)\).
    Since \(f\) is differentiable on \([a, b]\), by Exercise \ref{ex 10.1.1} we know that \(f\) is differentiable on \((x_1, x_2)\), and by Corollary \ref{10.1.12} \(f\) is continuous on \([x_1, x_2]\).
    By mean value theorem (Corollary \ref{10.2.9}) \(\exists\ c \in (x_1, x_2)\) such that \(f'(c) = \frac{f(x_2) - f(x_1)}{x_2 - x_1}\).
    Since \(c \in (x_1, x_2)\), we have \(c \in [a, b]\) and
    \begin{align*}
                 & f'(c) > 0                                           \\
        \implies & \frac{f(x_2) - f(x_1)}{x_2 - x_1} > 0               \\
        \implies & f(x_2) - f(x_1) > 0                   & (x_2 > x_1) \\
        \implies & f(x_2) > f(x_1).
    \end{align*}
    Thus by Definition \ref{9.8.1} \(f\) is strictly monotone increasing.

    Next we show that if \(f'(x) < 0\) for all \(x \in [a, b]\), then \(f\) is strictly monotone decreasing.
    Let \(x_1, x_2 \in [a, b]\) and \(x_1 < x_2\).
    By Theorem \ref{10.1.13}(e) we have \((-f)'(x) = -f'(x) > 0\).
    From proof above we know that \(-f\) is strictly monotone increasing and \(-f(x_1) < -f(x_2)\).
    Then we have \(f(x_1) > f(x_2)\) and by Definition \ref{9.8.1} \(f\) is strictly monotone decreasing.

    Finally we show that if \(f'(x) = 0\) for all \(x \in [a, b]\), then \(f\) is a constant function.
    Let \(x_1, x_2 \in [a, b]\) and \(x_1 \neq x_2\).
    Without the loss of generality suppose that \(x_1 < x_2\).
    Then we know that \([x_1, x_2] \subseteq [a, b]\) and \((x_1, x_2) \subseteq (a, b)\).
    Since \(f\) is differentiable on \([a, b]\), by Exercise \ref{ex 10.1.1} we know that \(f\) is differentiable on \((x_1, x_2)\), and by Corollary \ref{10.1.12} \(f\) is continuous on \([x_1, x_2]\).
    By mean value theorem (Corollary \ref{10.2.9}) \(\exists\ c \in (x_1, x_2)\) such that \(f'(c) = \frac{f(x_2) - f(x_1)}{x_2 - x_1}\).
    Since \(c \in (x_1, x_2)\), we have \(c \in [a, b]\) and
    \begin{align*}
                 & f'(c) = 0                                           \\
        \implies & \frac{f(x_2) - f(x_1)}{x_2 - x_1} = 0               \\
        \implies & f(x_2) - f(x_1) = 0                   & (x_2 > x_1) \\
        \implies & f(x_2) = f(x_1).
    \end{align*}
    Thus \(f\) is a constant function.
\end{proof}

\exercisesection

\begin{exercise}\label{ex 10.3.1}
    Prove Proposition \ref{10.3.1}.
\end{exercise}

\begin{proof}
    See Proposition \ref{10.3.1}.
\end{proof}

\begin{exercise}\label{ex 10.3.2}
    Give an example of a function \(f : (-1, 1) \to \mathbf{R}\) which is continuous and monotone increasing, but which is not differentiable at \(0\).
    Explain why this does not contradict Proposition \ref{10.3.1}.
\end{exercise}

\begin{proof}
    Define \(f\) as follow
    \[
        \forall\ x \in (-1, 1), f(x) = \begin{cases}
            x  & \text{if } x \in (-1, 0), \\
            2x & \text{if } x \in [0, 1).
        \end{cases}
    \]
    Then \(f\) is monotone increasing, \(f(0+) \geq 0\) and \(f(0-) < 0\).
    Since \(f(0+) \neq f(0-)\), by Additional Corollary \ref{ac 9.5.1} \(f\) is not continuous at \(0\), and by Proposition \ref{10.1.10} \(f\) is not differentiable at \(0\).
    This does not contradict to Proposition \ref{10.3.1} since \(0\) is not given to be differentiable.
\end{proof}

\begin{exercise}\label{ex 10.3.3}
    Give an example of a function \(f : \mathbf{R} \to \mathbf{R}\) which is strictly monotone increasing and differentiable, but whose derivative at \(0\) is zero.
    Explain why this does not contradict Proposition \ref{10.3.1} or Proposition \ref{10.3.3}.
\end{exercise}

\begin{proof}
    Let \(f(x) = x^3\).
    By Exercise \ref{ex 10.1.5} \(f\) is differentiable on \(\mathbf{R}\) and \(f'(x) = 3x^2\), thus \(f'(0) = 0\).
    If \(x, y \in \mathbf{R}\) and \(x < y\), then \(x^3 < y^3\), thus by Definition \ref{9.8.1} \(f\) is strictly monotone increasing.
    This does not contradict to Proposition \ref{10.3.1} since \(f'(0) = 0 \geq 0\).
    This does not contradict to Proposition \ref{10.3.3} since \(\forall\ x \in \mathbf{R}\), \(3x^2 \geq 0\).
\end{proof}

\begin{exercise}\label{ex 10.3.4}
    Prove Proposition \ref{10.3.3}.
\end{exercise}

\begin{proof}
    See Proposition \ref{10.3.3}.
\end{proof}

\begin{exercise}\label{ex 10.3.5}
    Give an example of a subset \(X \subseteq \mathbf{R}\) and a function \(f : X \to \mathbf{R}\) which is differentiable on \(X\), is such that \(f'(x) > 0\) for all \(x \in X\), but \(f\) is not strictly monotone increasing.
\end{exercise}

\begin{proof}
    Let \(X = [0, 0.5] \cup [1, 2]\) and let \(f : X \to \mathbf{R}\) be the following function
    \[
        \forall\ x \in X, f(x) = \begin{cases}
            2x & \text{if } x \in [0, 0.5], \\
            x  & \text{if } x \in [1, 2].
        \end{cases}
    \]
    By Exercise \ref{ex 10.1.5} we know that \(x\) and \(2x\) are differentiable and
    \[
        \forall\ x \in X, f'(x) = \begin{cases}
            2 & \text{if } x \in [0, 0.5], \\
            1 & \text{if } x \in [1, 2].
        \end{cases}
    \]
    So we have \(\forall\ x \in X\), \(f'(x) > 0\).
    Since \(0.5 < 1\) and \(1 = f(0.5) = f(1)\), by Definition \ref{9.8.1} \(f\) is not strictly monotone increasing.
\end{proof}