\section{Monotone functions and derivatives}\label{sec 10.3}

\begin{proposition}\label{10.3.1}
    Let \(X\) be a subset of \(\mathbf{R}\), let \(x_0 \in X\) be a limit point of \(X\), and let \(f : X \to \mathbf{R}\) be a function.
    If \(f\) is monotone increasing and \(f\) is differentiable at \(x_0\), then \(f'(x_0) \geq 0\).
    If f is monotone decreasing and \(f\) is differentiable at \(x_0\), then \(f'(x_0) \leq 0\).
\end{proposition}

\begin{proof}
    First suppose that \(f\) is monotone increasing.
    Since \(f\) is differentiable at \(x_0\), by Definition \ref{10.1.1} we have
    \[
        f'(x_0) = \lim_{x \to x_0 ; x \in X \setminus \{x_0\}} \frac{f(x) - f(x_0)}{x - x_0}.
    \]
    Since \(x, x_0 \in \mathbf{R}\) and \(x \in X \setminus \{x_0\}\), we have either \(x < x_0\) and \(x > x_0\).
    \begin{enumerate}
        \item If \(x < x_0\), then we have
              \begin{align*}
                           & (x - x_0 < 0) \land \big(f(x) - f(x_0) \leq 0\big) & \text{(by Definition \ref{9.8.1})} \\
                  \implies & \frac{f(x) - f(x_0)}{x - x_0} \geq 0.                                                   \\
              \end{align*}
        \item If \(x > x_0\), then we have
              \begin{align*}
                           & (x - x_0 > 0) \land \big(f(x) - f(x_0) \geq 0\big) & \text{(by Definition \ref{9.8.1})} \\
                  \implies & \frac{f(x) - f(x_0)}{x - x_0} \geq 0.                                                   \\
              \end{align*}
              From all cases above we have \(x \in X \setminus \{x_0\} \implies \frac{f(x) - f(x_0)}{x - x_0} \geq 0\).
              Thus by Proposition \ref{9.3.14} we have \(f'(x_0) \geq 0\).
    \end{enumerate}

    Now suppose that \(f\) is monotone decreasing.
    Since \(f\) is differentiable at \(x_0\), by Definition \ref{10.1.1} we have
    \[
        f'(x_0) = \lim_{x \to x_0 ; x \in X \setminus \{x_0\}} \frac{f(x) - f(x_0)}{x - x_0}.
    \]
    Since \(x, x_0 \in \mathbf{R}\) and \(x \in X \setminus \{x_0\}\), we have either \(x < x_0\) and \(x > x_0\).
    \begin{enumerate}
        \item If \(x < x_0\), then we have
              \begin{align*}
                           & (x - x_0 < 0) \land \big(f(x) - f(x_0) \geq 0\big) & \text{(by Definition \ref{9.8.1})} \\
                  \implies & \frac{f(x) - f(x_0)}{x - x_0} \leq 0.                                                   \\
              \end{align*}
        \item If \(x > x_0\), then we have
              \begin{align*}
                           & (x - x_0 > 0) \land \big(f(x) - f(x_0) \leq 0\big) & \text{(by Definition \ref{9.8.1})} \\
                  \implies & \frac{f(x) - f(x_0)}{x - x_0} \leq 0.                                                   \\
              \end{align*}
              From all cases above we have \(x \in X \setminus \{x_0\} \implies \frac{f(x) - f(x_0)}{x - x_0} \leq 0\).
              Thus by Proposition \ref{9.3.14} we have \(f'(x_0) \leq 0\).
    \end{enumerate}
\end{proof}

\begin{remark}\label{10.3.2}
    We have to assume that \(f\) is differentiable at \(x_0\);
    There exist monotone functions which are not always differentiable, and of course if \(f\) is not differentiable at \(x_0\) we cannot possibly conclude that \(f'(x_0) \geq 0\) or \(f'(x_0) \leq 0\).
\end{remark}

\begin{note}
    One might naively guess that if \(f\) were strictly monotone increasing, and \(f\) was differentiable at \(x_0\), then the derivative \(f'(x_0)\) would be strictly positive instead of merely non-negative.
    Unfortunately, this is not always the case.
\end{note}

\begin{proposition}\label{10.3.3}
    Let \(a < b\), and let \(f : [a, b] \to \mathbf{R}\) be a differentiable function.
    If \(f'(x) > 0\) for all \(x \in [a, b]\), then \(f\) is strictly monotone increasing.
    If \(f'(x) < 0\) for all \(x \in [a, b]\), then \(f\) is strictly monotone decreasing.
    If \(f'(x) = 0\) for all \(x \in [a, b]\), then \(f\) is a constant function.
\end{proposition}

\begin{proof}
    We first show that if \(f'(x) > 0\) for all \(x \in [a, b]\), then \(f\) is strictly monotone increasing.
    Let \(x_1, x_2 \in [a, b]\) and \(x_1 < x_2\).
    Then we know that \([x_1, x_2] \subseteq [a, b]\) and \((x_1, x_2) \subseteq (a, b)\).
    Since \(f\) is differentiable on \([a, b]\), by Exercise \ref{ex 10.1.1} we know that \(f\) is differentiable on \((x_1, x_2)\), and by Corollary \ref{10.1.12} \(f\) is continuous on \([x_1, x_2]\).
    By mean value theorem (Corollary \ref{10.2.9}) \(\exists\ c \in (x_1, x_2)\) such that \(f'(c) = \frac{f(x_2) - f(x_1)}{x_2 - x_1}\).
    Since \(c \in (x_1, x_2)\), we have \(c \in [a, b]\) and
    \begin{align*}
                 & f'(c) > 0                                           \\
        \implies & \frac{f(x_2) - f(x_1)}{x_2 - x_1} > 0               \\
        \implies & f(x_2) - f(x_1) > 0                   & (x_2 > x_1) \\
        \implies & f(x_2) > f(x_1).
    \end{align*}
    Thus by Definition \ref{9.8.1} \(f\) is strictly monotone increasing.

    Next we show that if \(f'(x) < 0\) for all \(x \in [a, b]\), then \(f\) is strictly monotone decreasing.
    Let \(x_1, x_2 \in [a, b]\) and \(x_1 < x_2\).
    By Theorem \ref{10.1.13}(e) we have \((-f)'(x) = -f'(x) > 0\).
    From proof above we know that \(-f\) is strictly monotone increasing and \(-f(x_1) < -f(x_2)\).
    Then we have \(f(x_1) > f(x_2)\) and by Definition \ref{9.8.1} \(f\) is strictly monotone decreasing.

    Finally we show that if \(f'(x) = 0\) for all \(x \in [a, b]\), then \(f\) is a constant function.
    Let \(x_1, x_2 \in [a, b]\) and \(x_1 \neq x_2\).
    Without the loss of generality suppose that \(x_1 < x_2\).
    Then we know that \([x_1, x_2] \subseteq [a, b]\) and \((x_1, x_2) \subseteq (a, b)\).
    Since \(f\) is differentiable on \([a, b]\), by Exercise \ref{ex 10.1.1} we know that \(f\) is differentiable on \((x_1, x_2)\), and by Corollary \ref{10.1.12} \(f\) is continuous on \([x_1, x_2]\).
    By mean value theorem (Corollary \ref{10.2.9}) \(\exists\ c \in (x_1, x_2)\) such that \(f'(c) = \frac{f(x_2) - f(x_1)}{x_2 - x_1}\).
    Since \(c \in (x_1, x_2)\), we have \(c \in [a, b]\) and
    \begin{align*}
                 & f'(c) = 0                                           \\
        \implies & \frac{f(x_2) - f(x_1)}{x_2 - x_1} = 0               \\
        \implies & f(x_2) - f(x_1) = 0                   & (x_2 > x_1) \\
        \implies & f(x_2) = f(x_1).
    \end{align*}
    Thus \(f\) is a constant function.
\end{proof}