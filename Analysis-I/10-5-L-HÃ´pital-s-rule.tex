\section{L'Hôpital's rule}\label{sec 10.5}

\begin{proposition}[L'Hôpital's rule I]\label{10.5.1}
    Let \(X\) be a subset of \(\mathbf{R}\), let \(f : X \to \mathbf{R}\) and \(g : X \to \mathbf{R}\) be functions, and let \(x_0 \in X\) be a limit point of \(X\).
    Suppose that \(f(x_0) = g(x_0) = 0\), that \(f\) and \(g\) are both differentiable at \(x_0\), but \(g'(x_0) \neq 0\).
    Then there exists a \(\delta > 0\) such that \(g(x) \neq 0\) for all \(x \in (X \cap (x_0 - \delta, x_0 + \delta)) \setminus \{x_0\}\), and
    \[
        \lim_{x \to x_0 ; x \in (X \cap (x_0 - \delta, x_0 + \delta)) \setminus \{x_0\}} \frac{f(x)}{g(x)} = \frac{f'(x_0)}{g'(x_0)}.
    \]
\end{proposition}

\begin{proof}
    Since \(g\) is differentiable at \(x_0\), by Newton's approximation (Proposition \ref{10.1.7}) we have \(\forall\ \varepsilon \in \mathbf{R}^+\), \(\exists\ \delta \in \mathbf{R}^+\) such that
    \begin{align*}
                 & \forall\ x \in X, \abs*{x - x_0} \leq \frac{\delta}{2} < \delta                                                         \\
        \implies & x \in X \cap (x_0 - \delta, x_0 + \delta)                                                                               \\
        \implies & \abs*{g(x) - (g(x_0) + g'(x_0)(x - x_0))} \leq \varepsilon \abs*{x - x_0}                                               \\
        \implies & \abs*{g(x) - g'(x_0)(x - x_0)} \leq \varepsilon \abs*{x - x_0}                                                          \\
        \implies & \abs*{g'(x_0)(x - x_0) - g(x)} \leq \varepsilon \abs*{x - x_0}                                                          \\
        \implies & \abs*{g'(x_0)(x - x_0)} \leq \abs*{g'(x_0)(x - x_0) - g(x)} + \abs*{g(x)} \leq \varepsilon \abs*{x - x_0} + \abs*{g(x)} \\
        \implies & \abs*{g'(x_0)(x - x_0)} - \varepsilon \abs*{x - x_0} \leq \abs*{g(x)}.
    \end{align*}
    Since \(g'(x_0) \neq 0\), we know \(\frac{\abs*{g'(x_0)}}{2} > 0\).
    By setting \(\varepsilon = \frac{\abs*{g'(x_0)}}{2}\) we know \(\exists\ \delta \in \mathbf{R}^+\) such that
    \begin{align*}
                 & \forall\ x \in X \cap (x_0 - \delta, x_0 + \delta)                    \\
        \implies & \abs*{g'(x_0)(x - x_0)} - \varepsilon \abs*{x - x_0} \leq \abs*{g(x)} \\
        \implies & 0 < \frac{\abs*{g'(x_0) (x - x_0)}}{2} \leq \abs*{g(x)}.
    \end{align*}
    The above statement also hold when \(x \in ((X \cap (x_0 - \delta, x_0 + \delta)) \setminus \{x_0\}\) since \((X \cap (x_0 - \delta, x_0 + \delta)) \setminus \{x_0\} \subseteq (X \cap (x_0 - \delta, x_0 + \delta)\).
    Thus we have
    \begin{align*}
          & \lim_{x \to x_0 ; x \in ((X \cap (x_0 - \delta, x_0 + \delta)) \setminus \{x_0\}} \frac{f(x)}{g(x)}                                                                                                                                                                           \\
        = & \lim_{x \to x_0 ; x \in ((X \cap (x_0 - \delta, x_0 + \delta)) \setminus \{x_0\}} \frac{f(x) - f(x_0)}{g(x) - g(x_0)}                                                                                                                   & (f(x_0) = g(x_0) = 0)               \\
        = & \lim_{x \to x_0 ; x \in ((X \cap (x_0 - \delta, x_0 + \delta)) \setminus \{x_0\}} \frac{f(x) - f(x_0)}{x - x_0} \frac{x - x_0}{g(x) - g(x_0)}                                                                                           & (x \neq x_0)                        \\
        = & \frac{\lim_{x \to x_0 ; x \in ((X \cap (x_0 - \delta, x_0 + \delta)) \setminus \{x_0\}} \frac{f(x) - f(x_0)}{x - x_0}}{\lim_{x \to x_0 ; x \in ((X \cap (x_0 - \delta, x_0 + \delta)) \setminus \{x_0\}} \frac{g(x) - g(x_0)}{x - x_0}} & \text{(by Theorem \ref{9.3.14})}    \\
        = & \frac{f'(x_0)}{g'(x_0)}.                                                                                                                                                                                                                & \text{(by Definition \ref{10.1.1})}
    \end{align*}
\end{proof}

\begin{note}
    The presence of the \(\delta\) here may seem somewhat strange, but is needed because \(g(x)\) might vanish at some points other than \(x_0\), which would imply that quotient \(\frac{f(x)}{g(x)}\) is not necessarily defined at all points in \(X \setminus \{x_0\}\).
\end{note}

\begin{proposition}[L'Hôpital's rule II]\label{10.5.2}
    Let \(a < b\) be real numbers, let \(f : [a, b] \to \mathbf{R}\) and \(g : [a, b] \to \mathbf{R}\) be functions which are differentiable on \([a, b]\).
    Suppose that \(f(a) = g(a) = 0\), that \(g'\) is non-zero on \([a, b]\) (i.e., \(g'(x) \neq 0\) for all \(x \in [a, b]\)), and \(\lim_{x \to a ; x \in (a, b]} \frac{f'(x)}{g'(x)}\) exists and equals \(L\).
    Then \(g(x) \neq 0\) for all \(x \in (a, b]\), and \(\lim_{x \to a ; x \in (a, b]} \frac{f(x)}{g(x)}\) exists and equals \(L\).
\end{proposition}

\begin{proof}
    We first show that \(g(x) \neq 0\) for all \(x \in (a, b]\).
    Suppose for sake of contradiction that \(g(x) = 0\) for some \(x \in (a, b]\).
    But since \(g(a)\) is also zero, we can apply Rolle's theorem (Theorem \ref{10.2.7}) to obtain \(g'(y) = 0\) for some \(a < y < x\), but this contradicts the hypothesis that \(g'\) is non-zero on \([a, b]\).

    Now we show that \(\lim_{x \to a ; x \in (a, b]} \frac{f(x)}{g(x)} = L\).
    By Proposition \ref{9.3.9}, it will suffice to show that
    \[
        \lim_{n \to \infty} \frac{f(x_n)}{g(x_n)} = L
    \]
    for any sequence \((x_n)_{n = 0}^\infty\) taking values in \((a, b]\) which converges to \(a\).

    Consider a single \(x_n\), and consider the function \(h_n : [a, x_n] \to \mathbf{R}\) defined by
    \[
        h_n(x) \coloneqq f(x) g(x_n) - g(x) f(x_n).
    \]
    Observe that \(h_n\) is continuous on \([a, x_n]\) and equals \(0\) at both \(a\) and \(x_n\), and is differentiable on \((a, x_n)\) with derivative \(h_n'(x) = f'(x) g(x_n) - g'(x) f(x_n)\).
    (Note that \(f(x_n)\) and \(g(x_n)\) are constants with respect to \(x\).)
    By Rolle's theorem (Theorem \ref{10.2.7}), we can thus find \(y_n \in (a, x_n)\) such that \(h_n'(y_n) = 0\), which implies that
    \[
        \frac{f(x_n)}{g(x_n)} = \frac{f'(y_n)}{g'(y_n)}.
    \]
    Since \(y_n \in (a, x_n)\) for all \(n\), and \(x_n\) converges to \(a\) as \(n \to \infty\), we see from the squeeze test (Corollary \ref{6.4.14}) that \(y_n\) also converges to \(a\) as \(n \to \infty\).
    Thus \(\frac{f'(y_n)}{g'(y_n)}\) converges to \(L\), and thus \(\frac{f(x_n)}{g(x_n)}\) also converges to \(L\), as desired.
\end{proof}

\begin{note}
    In Proposition \ref{10.5.2}, the hypothesis that \(f, g\) be differentiable on \([a, b]\) may be weakened to being continuous on \([a, b]\) and differentiable on \((a, b]\), with \(g'\) only assumed to be non-zero on \((a, b]\) rather than \([a, b]\).
\end{note}