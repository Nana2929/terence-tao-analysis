\section{Upper and lower Riemann integrals}\label{sec 11.3}

\begin{definition}[Majorization of functions]\label{11.3.1}
    Let \(f : I \to \mathbf{R}\) and \(g : I \to \mathbf{R}\).
    We say that \(g\) \emph{majorizes} \(f\) on \(I\) if we have \(g(x) \geq f(x)\) for all \(x \in I\), and that \(g\) \emph{minorizes} \(f\) on \(I\) if \(g(x) \leq f(x)\) for all \(x \in I\).
\end{definition}

\begin{definition}[Upper and lower Riemann integrals]\label{11.3.2}
    Let \(f : I \to \mathbf{R}\) be a bounded function defined on a bounded interval \(I\).
    We define the \emph{upper Riemann integral} \(\overline{\int}_I f\) by the formula
    \[
        \overline{\int}_I f \coloneqq \inf\{p.c. \int_I g : g \text{ is a piecewise constant function on \(I\) which majorizes } f\}
    \]
    and the \emph{lower Riemann integral} \(\underline{\int}_I f\) by the formula
    \[
        \underline{\int}_I f \coloneqq \sup\{p.c. \int_I g : g \text{ is a piecewise constant function on \(I\) which minorizes } f\}.
    \]
\end{definition}