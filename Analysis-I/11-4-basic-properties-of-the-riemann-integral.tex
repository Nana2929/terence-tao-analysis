\section{Basic properties of the Riemann integral}\label{sec 11.4}

\begin{theorem}[Laws of Riemann integration]\label{11.4.1}
    Let \(I\) be a bounded interval, and let \(f : I \to \mathbf{R}\) and \(g : I \to \mathbf{R}\) be Riemann integrable functions on \(I\).
    \begin{enumerate}
        \item The function \(f + g\) is Riemann integrable, and we have \(\int_I f + \int_I g\).
        \item For any real number \(c\), the function \(cf\) is Riemann integrable, and we have \(\int_I (cf) = c(\int_I f)\).
        \item The function \(f - g\) is Riemann integrable, and we have \(\int_I (f - g) = \int_I f - \int_I g\).
        \item If \(f(x) \geq 0\) for all \(x \in I\), then \(\int_I f \geq 0\).
        \item If \(f(x) \geq g(x)\) for all \(x \in I\), then \(\int_I f \geq \int_I g\).
        \item If \(f\) is the constant function \(f(x) = c\) for all \(x \in I\), then \(\int_I f = c \abs*{I}\).
        \item Let \(J\) be a bounded interval containing \(I\) (i.e., \(I \subseteq J\)), and let \(F : J \to \mathbf{R}\) be the function
              \[
                  F(x) \coloneqq \begin{cases}
                      f(x) & \text{if } x \in I    \\
                      0    & \text{if } x \notin I \\
                  \end{cases}
              \]
              Then \(F\) is Riemann integrable on \(J\), and \(\int_J F = \int_I f\).
        \item Suppose that \(\{J, K\}\) is a partition of \(I\) into two intervals \(J\) and \(K\).
              Then the functions \(f|_J : J \to \mathbf{R}\) and \(f|_K : K \to \mathbf{R}\) are Riemann integrable on \(J\) and \(K\) respectively, and we have
              \[
                  \int_I f = \int_J f|_J + \int_K f|_K.
              \]
    \end{enumerate}
\end{theorem}