\section{Basic properties of the Riemann integral}\label{sec 11.4}

\begin{theorem}[Laws of Riemann integration]\label{11.4.1}
    Let \(I\) be a bounded interval, and let \(f : I \to \mathbf{R}\) and \(g : I \to \mathbf{R}\) be Riemann integrable functions on \(I\).
    \begin{enumerate}
        \item The function \(f + g\) is Riemann integrable, and we have \(\int_I f + \int_I g\).
        \item For any real number \(c\), the function \(cf\) is Riemann integrable, and we have \(\int_I (cf) = c(\int_I f)\).
        \item The function \(f - g\) is Riemann integrable, and we have \(\int_I (f - g) = \int_I f - \int_I g\).
        \item If \(f(x) \geq 0\) for all \(x \in I\), then \(\int_I f \geq 0\).
        \item If \(f(x) \geq g(x)\) for all \(x \in I\), then \(\int_I f \geq \int_I g\).
        \item If \(f\) is the constant function \(f(x) = c\) for all \(x \in I\), then \(\int_I f = c \abs*{I}\).
        \item Let \(J\) be a bounded interval containing \(I\) (i.e., \(I \subseteq J\)), and let \(F : J \to \mathbf{R}\) be the function
              \[
                  F(x) \coloneqq \begin{cases}
                      f(x) & \text{if } x \in I    \\
                      0    & \text{if } x \notin I \\
                  \end{cases}
              \]
              Then \(F\) is Riemann integrable on \(J\), and \(\int_J F = \int_I f\).
        \item Suppose that \(\{J, K\}\) is a partition of \(I\) into two intervals \(J\) and \(K\).
              Then the functions \(f|_J : J \to \mathbf{R}\) and \(f|_K : K \to \mathbf{R}\) are Riemann integrable on \(J\) and \(K\) respectively, and we have
              \[
                  \int_I f = \int_J f|_J + \int_K f|_K.
              \]
    \end{enumerate}
\end{theorem}

\begin{proof}{(a)}
    Since \(f, g\) are Riemann integrable, by Definition \ref{11.3.4} we have
    \[
        \int_I f = \overline{\int}_I f = \underline{\int}_I f
    \]
    and
    \[
        \int_I g = \overline{\int}_I g = \underline{\int}_I g.
    \]
    Let \(f_U : I \to \mathbf{R}\) and \(g_U : I \to \mathbf{R}\) be piecewise constant functions on \(I\) which majorizes \(f\) and \(g\) respectively.
    Let \(f_L : I \to \mathbf{R}\) and \(g_L : I \to \mathbf{R}\) be piecewise constant functions on \(I\) which minorizes \(f\) and \(g\) respectively.
    \(f_U, g_U, f_L, g_L\) are well-defined since by Definition \ref{11.3.4} \(f, g\) are bounded functions on a bounded interval \(I\).
    By Definition \ref{11.3.2} we have
    \[
        p.c. \int_I f_L \leq \underline{\int}_I f = \int_I f = \overline{\int}_I f \leq p.c. \int_I f_U
    \]
    and
    \[
        p.c. \int_I g_L \leq \underline{\int}_I g = \int_I g = \overline{\int}_I g \leq p.c. \int_I g_U.
    \]
    By Definition \ref{11.3.4} both \(f, g\) are bounded functions, so \(f + g\) is bounded function, and \(\underline{\int}_I (f + g), \overline{\int}_I (f + g)\) are well-defined (by Definition \ref{11.3.2}).
    By Exercise \ref{ex 11.3.2} we know that \(f_U + g_U\) majorizes \(f + g_U\) and \(f + g_U\) majorizes \(f + g\), thus \(f_U + g_U\) majorizes \(f + g\).
    Similarly \(f_L + g_L\) minorizes \(f + g\).
    Then we have
    \begin{align*}
                 & \overline{\int}_I (f + g) \leq p.c. \int_I (f + g)                       & \text{(by Definition \ref{11.3.2})}  \\
        \implies & \overline{\int}_I (f + g) \leq p.c. \int_I f_U + p.c. \int_I g_U         & \text{(by Theorem \ref{11.2.16}(a))} \\
        \implies & \overline{\int}_I (f + g) - p.c. \int_I g_U \leq p.c. \int_I f_U                                                \\
        \implies & \overline{\int}_I (f + g) - p.c. \int_I g_U \leq \overline{\int}_I f     & \text{(by Definition \ref{11.3.2})}  \\
        \implies & \overline{\int}_I (f + g) - \overline{\int}_I f \leq p.c. \int_I g_U                                            \\
        \implies & \overline{\int}_I (f + g) - \overline{\int}_I f \leq \overline{\int}_I g & \text{(by Definition \ref{11.3.2})}  \\
        \implies & \overline{\int}_I (f + g) \leq \overline{\int}_I f + \overline{\int}_I g &                                      \\
        \implies & \overline{\int}_I (f + g) \leq \int_I f + \int_I g                       & \text{(by Definition \ref{11.3.4})}
    \end{align*}
    and
    \begin{align*}
                 & \underline{\int}_I (f + g) \geq p.c. \int_I (f + g)                         & \text{(by Definition \ref{11.3.2})}  \\
        \implies & \underline{\int}_I (f + g) \geq p.c. \int_I f_L + p.c. \int_I g_L           & \text{(by Theorem \ref{11.2.16}(a))} \\
        \implies & \underline{\int}_I (f + g) - p.c. \int_I g_L \geq p.c. \int_I f_L                                                  \\
        \implies & \underline{\int}_I (f + g) - p.c. \int_I g_L \geq \underline{\int}_I f      & \text{(by Definition \ref{11.3.2})}  \\
        \implies & \underline{\int}_I (f + g) - \underline{\int}_I f \geq p.c. \int_I g_L                                             \\
        \implies & \underline{\int}_I (f + g) - \underline{\int}_I f \geq \underline{\int}_I g & \text{(by Definition \ref{11.3.2})}  \\
        \implies & \underline{\int}_I (f + g) \geq \underline{\int}_I f + \underline{\int}_I g &                                      \\
        \implies & \underline{\int}_I (f + g) \geq \int_I f + \int_I g.                        & \text{(by Definition \ref{11.3.4})}
    \end{align*}
    By Lemma \ref{11.3.3} we have
    \[
        \int_I f + \int_I g \leq \underline{\int}_I (f + g) \leq \overline{\int}_I (f + g) \leq \int_I f + \int_I g
    \]
    and thus by Definition \ref{11.3.4} we have
    \[
        \int_I (f + g) = \underline{\int}_I (f + g) = \overline{\int}_I (f + g) = \int_I f + \int_I g.
    \]
\end{proof}

\begin{proof}{(b)}
    Since \(f\) is Riemann integrable, by Definition \ref{11.3.4} we have
    \[
        \int_I f = \overline{\int}_I f = \underline{\int}_I f.
    \]
    First suppose that \(c = 0\).
    Then we have \((cf)(x) = 0\) for all \(x \in 0\), thus we have
    \begin{align*}
        \int_I (cf) & = p.c. \int_I (cf) & \text{(by Lemma \ref{11.3.7})} \\
                    & = 0                                                 \\
                    & = c \int_I f.
    \end{align*}

    Next suppose that \(c > 0\).
    Let \(f_U : I \to \mathbf{R}\) be a piecewise constant function on \(I\) which majorizes \(f\).
    Let \(f_L : I \to \mathbf{R}\) be a piecewise constant function on \(I\) which minorizes \(f\).
    \(f_U, f_L\) are well-defined since by Definition \ref{11.3.4} \(f\) is bounded function on a bounded interval \(I\).
    Then by Definition \ref{11.3.2} we have
    \[
        p.c. \int_I f_L \leq \underline{\int}_I f = \int_I f = \overline{\int}_I f \leq p.c. \int_I f_U.
    \]
    Since \(f\) is bounded function, \(cf\) is also a bounded function, by Definition \ref{11.3.2} both \(\overline{\int}_I (cf), \underline{\int}_I (cf)\) are well-defined.
    Since \(c > 0\), by Definition \ref{11.3.1} we know that \(c f_U\) majorizes \(c f\) and \(c f_L\) minorizes \(c f\).
    Then we have
    \begin{align*}
                 & \overline{\int}_I (cf) \leq p.c. \int_I (c f_U)                         & \text{(by Definition \ref{11.3.2})}  \\
        \implies & \overline{\int}_I (cf) \leq c \bigg(p.c. \int_I f_U\bigg)               & \text{(by Theorem \ref{11.2.16}(b))} \\
        \implies & \frac{1}{c} \bigg(\overline{\int}_I (cf)\bigg) \leq p.c. \int_I f_U                                            \\
        \implies & \frac{1}{c} \bigg(\overline{\int}_I (cf)\bigg) \leq \overline{\int}_I f & \text{(by Definition \ref{11.3.2})}  \\
        \implies & \overline{\int}_I (cf) \leq c\bigg(\overline{\int}_I f\bigg)                                                   \\
        \implies & \overline{\int}_I (cf) \leq c\bigg(\int_I f\bigg)                       & \text{(by Definition \ref{11.3.4})}
    \end{align*}
    and
    \begin{align*}
                 & \underline{\int}_I (cf) \geq p.c. \int_I (c f_L)                          & \text{(by Definition \ref{11.3.2})}  \\
        \implies & \underline{\int}_I (cf) \geq c \bigg(p.c. \int_I f_L\bigg)                & \text{(by Theorem \ref{11.2.16}(b))} \\
        \implies & \frac{1}{c} \bigg(\underline{\int}_I (cf)\bigg) \geq p.c. \int_I f_L                                             \\
        \implies & \frac{1}{c} \bigg(\underline{\int}_I (cf)\bigg) \geq \underline{\int}_I f & \text{(by Definition \ref{11.3.2})}  \\
        \implies & \underline{\int}_I (cf) \geq c\bigg(\underline{\int}_I f\bigg)                                                   \\
        \implies & \underline{\int}_I (cf) \geq c\bigg(\int_I f\bigg).                       & \text{(by Definition \ref{11.3.4})}
    \end{align*}
    By Lemma \ref{11.3.3} we have
    \[
        c\bigg(\int_I f\bigg) \leq \underline{\int}_I (cf) \leq \overline{\int}_I (cf) \leq c\bigg(\int_I f\bigg)
    \]
    and thus by Definition \ref{11.3.4} we have
    \[
        \int_I (cf) = \underline{\int}_I (cf) = \overline{\int}_I (cf) = c\bigg(\int_I f\bigg).
    \]

    Next suppose that \(c = -1\).
    Using the same definition of \(f_U, f_L\) we have
    \begin{align*}
                 & \overline{\int}_I (cf) \leq p.c. \int_I (c f_U)                         & \text{(by Definition \ref{11.3.2})}  \\
                 & \overline{\int}_I (cf) \leq p.c. \int_I (c f_U)                         & \text{(by Lemma \ref{11.3.3})}       \\
        \implies & \overline{\int}_I (cf) \leq c \bigg(p.c. \int_I f_U\bigg)               & \text{(by Theorem \ref{11.2.16}(b))} \\
        \implies & \frac{1}{c} \bigg(\overline{\int}_I (cf)\bigg) \geq p.c. \int_I f_U                                            \\
        \implies & \frac{1}{c} \bigg(\overline{\int}_I (cf)\bigg) \geq \overline{\int}_I f & \text{(by Definition \ref{11.3.2})}  \\
        \implies & \overline{\int}_I (cf) \leq c\bigg(\overline{\int}_I f\bigg)                                                   \\
        \implies & \overline{\int}_I (cf) \leq c\bigg(\int_I f\bigg)                       & \text{(by Definition \ref{11.3.4})}
    \end{align*}
    and
    \begin{align*}
                 & \underline{\int}_I (cf) \geq p.c. \int_I (c f_L)                          & \text{(by Definition \ref{11.3.2})}  \\
                 & \underline{\int}_I (cf) \geq p.c. \int_I (c f_L)                          & \text{(by Lemma \ref{11.3.3})}       \\
        \implies & \underline{\int}_I (cf) \geq c \bigg(p.c. \int_I f_L\bigg)                & \text{(by Theorem \ref{11.2.16}(b))} \\
        \implies & \frac{1}{c} \bigg(\underline{\int}_I (cf)\bigg) \leq p.c. \int_I f_L                                             \\
        \implies & \frac{1}{c} \bigg(\underline{\int}_I (cf)\bigg) \leq \underline{\int}_I f & \text{(by Definition \ref{11.3.2})}  \\
        \implies & \underline{\int}_I (cf) \geq c\bigg(\underline{\int}_I f\bigg)                                                   \\
        \implies & \underline{\int}_I (cf) \geq c\bigg(\int_I f\bigg).                       & \text{(by Definition \ref{11.3.4})}
    \end{align*}
    Again we have
    \[
        \int_I (cf) = \underline{\int}_I (cf) = \overline{\int}_I (cf) = c\bigg(\int_I f\bigg).
    \]

    Finally suppose \(c < 0\).
    Then we have
    \begin{align*}
        \int_I (cf) & = \int_I ((-1)(-c)f)              & \text{(by Definition \ref{9.2.1})} \\
                    & = (-1) \bigg(\int_I ((-c)f)\bigg)                                      \\
                    & = (-1)(-c) \bigg(\int_I f\bigg)   & (-c > 0)                           \\
                    & = c \bigg(\int_I f\bigg).
    \end{align*}
    We conclude that \(\forall\ c \in \mathbf{R}\), \(\int_I (cf) = c (\int_I f)\).
\end{proof}