\section{Riemann integrability of continuous functions}\label{sec 11.5}

\begin{theorem}\label{11.5.1}
    Let \(I\) be a bounded interval, and let \(f\) be a function which is uniformly continuous on \(I\).
    Then \(f\) is Riemann integrable.
\end{theorem}

\begin{proof}
    From Proposition \ref{9.9.15} we see that \(f\) is bounded.
    Now we have to show that \(\underline{\int}_I f = \overline{\int}_I f\).

    If \(I\) is a point or the empty set then the theorem is trivial, so let us assume that \(I\) is one of the four intervals \([a, b]\), \((a, b)\), \((a, b]\), or \([a, b)\) for some real numbers \(a < b\).

            Let \(\varepsilon > 0\) be arbitrary.
            By uniform continuity, there exists a \(\delta > 0\) such that \(\abs*{f(x) - f(y)} < \varepsilon\) whenever \(x, y \in I\) are such that \(\abs*{x - y} < \delta\).
            By the Archimedean principle, there exists an integer \(N > 0\) such that \((b - a) / N < \delta\).

            Note that we can partition \(I\) into \(N\) intervals \(J_1, \dots, J_N\), each of length \((b - a) / N\).
            (How? One has to treat each of the cases \([a, b]\), \((a, b)\), \((a, b]\), \([a, b)\) slightly differently.)
    By Proposition \ref{11.3.12}, we thus have
    \[
        \overline{\int}_I f \leq \sum_{k = 1}^N (\sup_{x \in J_k} f(x)) \abs*{J_k}
    \]
    and
    \[
        \underline{\int}_I f \geq \sum_{k = 1}^N (\inf_{x \in J_k} f(x)) \abs*{J_k}
    \]
    so in particular
    \[
        \overline{\int}_I f - \underline{\int}_I f \leq \sum_{k = 1}^N (\sup_{x \in J_k} f(x) - \inf_{x \in J_k} f(x)) \abs*{J_k}.
    \]
    However, we have \(\abs*{f(x) - f(y)} < \varepsilon\) for all \(x, y \in J_k\), since \(\abs*{J_k} = (b - a) / N < \delta\).
    In particular we have
    \[
        f(x) < f(y) + \varepsilon \text{ for all } x, y \in J_k.
    \]
    Taking suprema in \(x\), we obtain
    \[
        \sup_{x \in J_k} f(x) \leq f(y) + \varepsilon \text{ for all } y \in J_k,
    \]
    and then taking infima in \(y\) we obtain
    \[
        \sup_{x \in J_k} f(x) \leq \inf_{y \in J_k} f(y) + \varepsilon.
    \]
    Inserting this bound into our previous inequality, we obtain
    \[
        \overline{\int}_I f - \underline{\int}_I f \leq \sum_{k = 1}^N \varepsilon \abs*{J_k},
    \]
    but by Theorem \ref{11.1.13} we thus have
    \[
        \overline{\int}_I f - \underline{\int}_I f \leq \varepsilon (b - a).
    \]
    But \(\varepsilon > 0\) was arbitrary, while \((b - a)\) is fixed.
    Thus \(\overline{\int}_I f - \underline{\int}_I f\) cannot be positive.
    By Lemma \ref{11.3.3} and the definition of Riemann integrability we thus have that \(f\) is Riemann integrable.
\end{proof}