\section{Riemann integrability of continuous functions}\label{sec 11.5}

\begin{theorem}\label{11.5.1}
    Let \(I\) be a bounded interval, and let \(f\) be a function which is uniformly continuous on \(I\).
    Then \(f\) is Riemann integrable.
\end{theorem}

\begin{proof}
    From Proposition \ref{9.9.15} we see that \(f\) is bounded.
    Now we have to show that \(\underline{\int}_I f = \overline{\int}_I f\).

    If \(I\) is a point or the empty set then the theorem is trivial, so let us assume that \(I\) is one of the four intervals \([a, b]\), \((a, b)\), \((a, b]\), or \([a, b)\) for some real numbers \(a < b\).

            Let \(\varepsilon > 0\) be arbitrary.
            By uniform continuity, there exists a \(\delta > 0\) such that \(\abs{f(x) - f(y)} < \varepsilon\) whenever \(x, y \in I\) are such that \(\abs{x - y} < \delta\).
            By the Archimedean principle, there exists an integer \(N > 0\) such that \((b - a) / N < \delta\).

            Note that we can partition \(I\) into \(N\) intervals \(J_1, \dots, J_N\), each of length \((b - a) / N\).
            (How? One has to treat each of the cases \([a, b]\), \((a, b)\), \((a, b]\), \([a, b)\) slightly differently.)
    By Proposition \ref{11.3.12}, we thus have
    \[
        \overline{\int}_I f \leq \sum_{k = 1}^N \big(\sup_{x \in J_k} f(x)\big) \abs{J_k}
    \]
    and
    \[
        \underline{\int}_I f \geq \sum_{k = 1}^N \big(\inf_{x \in J_k} f(x)\big) \abs{J_k}
    \]
    so in particular
    \[
        \overline{\int}_I f - \underline{\int}_I f \leq \sum_{k = 1}^N \big(\sup_{x \in J_k} f(x) - \inf_{x \in J_k} f(x)\big) \abs{J_k}.
    \]
    However, we have \(\abs{f(x) - f(y)} < \varepsilon\) for all \(x, y \in J_k\), since \(\abs{J_k} = (b - a) / N < \delta\).
    In particular we have
    \[
        f(x) < f(y) + \varepsilon \text{ for all } x, y \in J_k.
    \]
    Taking suprema in \(x\), we obtain
    \[
        \sup_{x \in J_k} f(x) \leq f(y) + \varepsilon \text{ for all } y \in J_k,
    \]
    and then taking infima in \(y\) we obtain
    \[
        \sup_{x \in J_k} f(x) \leq \inf_{y \in J_k} f(y) + \varepsilon.
    \]
    Inserting this bound into our previous inequality, we obtain
    \[
        \overline{\int}_I f - \underline{\int}_I f \leq \sum_{k = 1}^N \varepsilon \abs{J_k},
    \]
    but by Theorem \ref{11.1.13} we thus have
    \[
        \overline{\int}_I f - \underline{\int}_I f \leq \varepsilon (b - a).
    \]
    But \(\varepsilon > 0\) was arbitrary, while \((b - a)\) is fixed.
    Thus \(\overline{\int}_I f - \underline{\int}_I f\) cannot be positive.
    By Lemma \ref{11.3.3} and the definition of Riemann integrability we thus have that \(f\) is Riemann integrable.
\end{proof}

\begin{corollary}\label{11.5.2}
    Let \([a, b]\) be a closed interval, and let \(f : [a, b] \to \R\) be continuous.
    Then \(f\) is Riemann integrable.
\end{corollary}

\begin{proof}
    Combining Theorem \ref{11.5.1} with Theorem \ref{9.9.16} we are done.
\end{proof}

\begin{note}
    Note that Corollary \ref{11.5.2} is not true if \([a, b]\) is replaced by any other sort of interval, since it is not even guaranteed then that continuous functions are bounded.
    For instance, the function \(f : (0, 1) \to \R\) defined by \(f(x) \coloneqq 1 / x\) is continuous but not Riemann integrable.
    However, if we assume that a function is both continuous \emph{and} bounded, we can recover Riemann integrability (see Proposition \ref{11.5.3}).
\end{note}

\begin{proposition}\label{11.5.3}
    Let \(I\) be a bounded interval, and let \(f : I \to \R\) be both continuous and bounded.
    Then \(f\) is Riemann integrable on \(I\).
\end{proposition}

\begin{proof}
    If \(I\) is a point or an empty set then the claim is trivial;
    if \(I\) is a closed interval the claim follows from Corollary \ref{11.5.2}.
    So let us assume that \(I\) is of the form \((a, b]\), \((a, b)\), or \([a, b)\) for some \(a < b\).

    We have a bound \(M\) for \(f\), so that \(-M \leq f(x) \leq M\) for all \(x \in I\).
    Now let \(0 < \varepsilon < (b - a) / 2\) be a small number.
    The function \(f\) when restricted to the interval \([a + \varepsilon, b - \varepsilon]\) is continuous, and hence Riemann integrable by Corollary \ref{11.5.2}.
    In particular, we can find a piecewise constant function \(h : [a + \varepsilon, b - \varepsilon] \to \R\) which majorizes \(f\) on \([a + \varepsilon, b - \varepsilon]\) such that
    \[
        \int_{[a + \varepsilon, b - \varepsilon]} h \leq \int_{[a + \varepsilon, b - \varepsilon]} f + \varepsilon.
    \]
    Define \(\tilde{h} : I \to \R\) by
    \[
        \tilde{h}(x) \coloneqq \begin{cases}
            h(x) & \text{if } x \in [a + \varepsilon, b - \varepsilon]             \\
            M    & \text{if } x \in I \setminus [a + \varepsilon, b - \varepsilon]
        \end{cases}
    \]
    Clearly \(\tilde{h}\) is piecewise constant on \(I\) and majorizes \(f\);
    by Theorem \ref{11.2.16} we have
    \[
        \int_I \tilde{h} = \varepsilon M + \int_{[a + \varepsilon, b - \varepsilon]} h + \varepsilon M \leq \int_{[a + \varepsilon, b - \varepsilon]} f + (2M + 1) \varepsilon.
    \]
    In particular we have
    \[
        \overline{\int}_I f \leq \int_{[a + \varepsilon, b - \varepsilon]} f + (2M + 1) \varepsilon.
    \]
    This is true since \(\tilde{h}\) majorize \(f\).
    A similar argument gives
    \[
        \underline{\int}_I f \geq \int_{[a + \varepsilon, b - \varepsilon]} f - (2M + 1) \varepsilon.
    \]
    and hence
    \[
        \overline{\int}_I f - \underline{\int}_I f \leq (4M + 2) \varepsilon.
    \]
    But \(\varepsilon\) is arbitrary, and so we can argue as in the proof of Theorem \ref{11.5.1} to conclude Riemann integrability.
\end{proof}

\begin{note}
    From Theorem \ref{11.5.1}, Corollary \ref{11.5.2} and Proposition \ref{11.5.3} we see that if we can show a function \(f\) being \emph{uniformly continuous} (not just continuous) on some bounded interval \(I\), then \(f\) is Riemann integrable on \(I\).
\end{note}

\begin{definition}\label{11.5.4}
    Let \(I\) be a bounded interval, and let \(f : I \to \R\).
    We say that \(f\) is \emph{piecewise continuous on \(I\)} iff there exists a partition \(\mathbf{P}\) of \(I\) such that \(f|_J\) is continuous on \(J\) for all \(J \in \mathbf{P}\).
\end{definition}

\setcounter{theorem}{5}
\begin{proposition}\label{11.5.6}
    Let \(I\) be a bounded interval, and let \(f : I \to \R\) be both piecewise continuous and bounded.
    Then \(f\) is Riemann integrable.
\end{proposition}

\begin{proof}
    Since \(f\) is piecewise continuous on \(I\), by Definition \ref{11.5.4} \(\exists\ \mathbf{P}\) such that \(\mathbf{P}\) is a partition of \(I\) and \(f|_J\) is continuous on \(J\) for all \(J \in \mathbf{P}\).
    Since \(f\) is bounded, we know that \(f|_J\) is bounded for all \(J \in \mathbf{P}\).
    Thus by Proposition \ref{11.5.3} \(f|_J\) is Riemann integrable on \(J\) for all \(J \in \mathbf{P}\).
    For each \(J \in \mathbf{P}\), we define \(F_J : I \to \R\) to be the function
    \[
        F_J(x) = \begin{cases}
            f|_J(x) & \text{if } x \in J    \\
            0       & \text{if } x \notin J
        \end{cases}
    \]
    Then by Theorem \ref{11.4.1}(g) \(F|_J\) is Riemann integrable for all \(J \in \mathbf{P}\) and
    \begin{align*}
        \sum_{J \in \mathbf{P}} \int_I F_J & = \sum_{J \in \mathbf{P}} \int_J f|_J & \text{(by Theorem \ref{11.4.1}(g))}  \\
                                           & = \int_I f.                           & \text{(by Exercise \ref{ex 11.4.3})}
    \end{align*}
    Thus \(f\) is Riemann integrable on \(I\).
\end{proof}

\exercisesection

\begin{exercise}\label{ex 11.5.1}
    Prove Proposition \ref{11.5.6}.
\end{exercise}

\begin{proof}
    See Proposition \ref{11.5.6}.
\end{proof}