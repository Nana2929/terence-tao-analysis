\section{The Riemann-Stieltjes integral}\label{sec 11.8}

\begin{note}
    Let \(I\) be a bounded interval, let \(\alpha : I \to \mathbf{R}\) be a monotone increasing function, and let \(f : I \to \mathbf{R}\) be a function.
    Then there is a generalization of the Riemann integral, known as the \emph{Riemann-Stieltjes integral}.
    This integral is defined just like the Riemann integral, but with one twist:
    instead of taking the length \(\abs*{J}\) of intervals \(J\), we take the \(\alpha\)-length \(\alpha[J]\), defined as follows.
    If \(J\) is a point or the empty set, then \(\alpha[J] \coloneqq 0\).
    If \(J\) is an interval of the form \([a, b]\), \((a, b)\), \((a, b]\), or \([a, b)\), then \(\alpha[J] \coloneqq \alpha(b) - \alpha(a)\).
    Note that in the special case where \(\alpha\) is the identity function \(\alpha(x) \coloneqq x\), then \(\alpha[J]\) is just the same as \(\abs*{J}\).
    However, for more general monotone functions \(\alpha\), the \(\alpha\)-length \(\alpha[J]\) is a different quantity from \(\abs*{J}\).
    Nevertheless, it turns out one can still do much of the above theory, but replacing \(\abs*{J}\) by \(\alpha[J]\) throughout.
\end{note}

\begin{definition}[\(\alpha\)-length]\label{11.8.1}
    Let \(I\) be a bounded interval, and let \(\alpha : X \to \mathbf{R}\) be a function defined on some domain \(X\) which contains \(I\).
    Then we define the \emph{\(\alpha\)-length} \(\alpha[I]\) of \(I\) as follows.
    If \(I\) is a point or the empty set, we set \(\alpha[I] = 0\).
    If \(I\) is an interval of the form \([a, b]\), \([a, b)\), \((a, b]\), or \((a, b)\) for some \(b > a\), then we set \(\alpha[I] = \alpha(b) - \alpha(a)\).
\end{definition}