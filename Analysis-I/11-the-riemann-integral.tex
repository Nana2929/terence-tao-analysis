\chapter{The Riemann integral}\label{ch 11}

\begin{note}
    In Chapter \ref{ch 10} we reviewed \emph{differentiation} - one of the two pillars of single variable calculus.
    The other pillar is, of course, \emph{integration}, which is the focus of the current chapter.
    More precisely, we will turn to the \emph{definite integral}, the integral of a function on a fixed interval, as opposed to the \emph{indefinite integral}, otherwise known as the \emph{antiderivative}.
    These two are of course linked by the \emph{Fundamental theorem of calculus}.
\end{note}

\begin{note}
    To actually \emph{define} this integral \(\int_I f\) is somewhat delicate (especially if one does not want to assume any axioms concerning geometric notions such as area), and not all functions \(f\) are integrable.
    It turns out that there are at least two ways to define this integral:
    the \emph{Riemann integral}, named after Georg Riemann (1826 -- 1866), which suffices for most applications, and the \emph{Lebesgue integral}, named after Henri Lebesgue (1875 -- 1941), which supercedes the Riemann integral and works for a much larger class of functions.
    There is also the \emph{Riemann-Stieltjes integral} \(\int_I f(x) d \alpha(x)\), a generalization of the Riemann integral due to Thomas Stieltjes (1856 -- 1894).
\end{note}

\section{Partitions}\label{sec 11.1}

\begin{definition}\label{11.1.1}
    Let \(X\) be a subset of \(\mathbf{R}\).
    We say that \(X\) is \emph{connected} iff \(X\) is nonempty and the following property is true:
    whenever \(x, y\) are elements in \(X\) such that \(x < y\), the bounded interval \([x, y]\) is a subset of \(X\)
    (i.e., every number between \(x\) and \(y\) is also in \(X\)).
\end{definition}

\setcounter{theorem}{3}
\begin{lemma}\label{11.1.4}
    Let \(X\) be a subset of the real line.
    Then the following two statements are logically equivalent:
    \begin{enumerate}
        \item \(X\) is bounded and either connected or empty.
        \item \(X\) is a bounded interval.
    \end{enumerate}
\end{lemma}

\begin{proof}
    Both statements are logically equivalent when \(X = \emptyset\) (which is vacuously true).
    So suppose that \(X \neq \emptyset\).

    We first show that \(X\) is bounded and connected implies \(X\) is a bounded interval.
    Since \(X\) is bounded, by Theorem \ref{5.5.9} we know that \(\inf(X), \sup(X) \in \mathbf{R}\).
    Thus \(X \subseteq [\inf(X), \sup(X)]\).
    Now we split into four cases:
    \begin{itemize}
        \item If \(\sup(X) \in X\) and \(\inf(X) \in X\), then by Definition \ref{11.1.1} \(X\) is connected implies \([\inf(X), \sup(X)] \subseteq X\).
              Thus by Proposition \ref{3.1.18} we have \(X = [\inf(X), \sup(X)]\).
        \item If \(\sup(X) \in X\) and \(\inf(X) \notin X\), then we claim that \(\big(\inf(X), \sup(X)] \subseteq X\).
              This is true since \(X\) is connected and by Definition \ref{11.1.1} we have \(\big(a, \sup(X)] \subseteq X\) for every \(a \in X\).
        \item If \(\sup(X) \notin X\) and \(\inf(X) \in X\), then we claim that \([\inf(X), \sup(X)\big) \subseteq X\).
              This is true since \(X\) is connected and by Definition \ref{11.1.1} we have \([\inf(X), b\big) \subseteq X\) for every \(b \in X\).
        \item If \(\sup(X) \notin X\) and \(\inf(X) \notin X\), then we claim that \(\big(\inf(X), \sup(X)\big) \subseteq X\).
              This is true since \(X\) is connected and by Definition \ref{11.1.1} we have \((a, b) \subseteq X\) for every \(a, b \in X\) and \(a < b\).
    \end{itemize}
    From all cases above we conclude that \(X\) is a bounded interval.

    Now we show that \(X\) is a bounded interval implies \(X\) is bounded and connected.
    Obviously \(X\) is bounded.
    Let \(a, b \in \mathbf{R}\).
    Then \(X\) can be one of \((a, b), [a, b], (a, b], [a, b)\), and by Definition \ref{11.1.1} all of which are connected.
\end{proof}

\begin{remark}\label{11.1.5}
    Recall that intervals are allowed to be singleton points, or even the empty set.
\end{remark}

\begin{corollary}\label{11.1.6}
    If \(I\) and \(J\) are bounded intervals, then the intersection \(I \cap J\) is also a bounded interval.
\end{corollary}

\begin{proof}
    If \(I \cap J = \emptyset\), then \(I \cap J\) is bounded interval.
    So suppose that \(I \cap J \neq \emptyset\).
    Since \(I, J\) are bounded intervals, by Lemma \ref{11.1.4} we know that \(I, J\) are bounded and connected.
    Since \(I, J\) are bounded, \(\exists\ M_1, M_2 \in \mathbf{R}\) such that \(I \subseteq [-M_1, M_1]\) and \(J \subseteq [-M_2, M_2]\).
    Let \(M = \min(M_1, M_2)\).
    Then we have \(I \cap J \subseteq [-M, M]\) and thus \(I \cap J\) is bounded.
    Let \(x, y \in I \cap J\) and \(x < y\).
    Since \(I\) is connected and \(I \cap J \subseteq I\), we have \([x, y] \subseteq I\).
    Similarly since \(J\) is connected and \(I \cap J \subseteq J\), we have \([x, y] \subseteq J\).
    Thus \([x, y] \subseteq I \cap J\) and by Definition \ref{11.1.1} \(I \cap J\) is connected.
    Since \(I \cap J\) is bounded and connected, by Lemma \ref{11.1.4} \(I \cap J\) is bounded interval.
\end{proof}

\setcounter{theorem}{7}
\begin{definition}[Length of intervals]\label{11.1.8}
    If \(I\) is a bounded interval, we define the \emph{length} of \(I\), denoted \(\abs*{I}\) as follows.
    If \(I\) is one of the intervals \([a, b]\), \((a, b)\), \([a, b)\), or \((a, b]\) for some real numbers \(a < b\), then we define \(\abs*{I} \coloneqq b - a\).
    Otherwise, if \(I\) is a point or the empty set, we define \(\abs*{I} = 0\).
\end{definition}

\setcounter{theorem}{9}
\begin{definition}[Partitions]\label{11.1.10}
    Let \(I\) be a bounded interval.
    A \emph{partition} of \(I\) is a finite set \(\mathbf{P}\) of bounded intervals contained in \(I\), such that every \(x\) in \(I\) lies in exactly one of the bounded intervals \(J\) in \(\mathbf{P}\).
\end{definition}

\begin{remark}\label{11.1.11}
    Note that a partition is a set of intervals, while each interval is itself a set of real numbers.
    Thus a partition is a set consisting of other sets.
\end{remark}

\setcounter{theorem}{12}
\begin{theorem}[Length is finitely additive]\label{11.1.13}
    Let \(I\) be a bounded interval, \(n\) be a natural number, and let \(\mathbf{P}\) be a partition of \(I\) of cardinality \(n\).
    Then
    \[
        \abs*{I} = \sum_{J \in \mathbf{P}} \abs*{J}.
    \]
\end{theorem}

\begin{proof}
    We prove this by induction on \(n\).
    More precisely, we let \(P(n)\) be the property that whenever \(I\) is a bounded interval, and whenever \(\mathbf{P}\) is a partition of \(I\) with cardinality \(n\), that \(\abs*{I} = \sum_{J \in \mathbf{P}} \abs*{J}\).

    The base case \(P(0)\) is trivial;
    the only way that \(I\) can be partitioned into an empty partition is if \(I\) is itself empty, at which point the claim is easy.
    The case \(P(1)\) is also very easy;
    the only way that \(I\) can be partitioned into a singleton set \(\{J\}\) is if \(J = I\), at which point the claim is again very easy.

    Now suppose inductively that \(P(n)\) is true for some \(n \geq 1\), and now we prove \(P(n + 1)\).
    Let \(I\) be a bounded interval, and let \(\mathbf{P}\) be a partition of \(I\) of cardinality \(n + 1\).

    If \(I\) is the empty set or a point, then all the intervals in \(\mathbf{P}\) must also be either the empty set or a point, and so every interval has length zero and the claim is trivial.
    Thus we will assume that \(I\) is an interval of the form \((a, b)\), \((a, b]\), \([a, b)\), or \([a, b]\).

            Let us first suppose that \(b \in I\), i.e., \(I\) is either \((a, b]\) or \([a, b]\).
    Since \(b \in I\), we know that one of the intervals \(K\) in \(\mathbf{P}\) contains \(b\).
    Since \(K\) is contained in \(I\), it must therefore be of the form \((c, b]\), \([c, b]\), or \(\{b\}\) for some real number \(c\), with \(a \leq c \leq b\) (in the latter case of \(K = \{b\}\), we set \(c \coloneqq b\)).
    In particular, this means that the set \(I \setminus K\) is also an interval of the form \([a, c]\), \((a, c)\), \((a, c]\), \([a, c)\) when \(c > a\), or a point or empty set when \(a = c\).
    Either way, we easily see that
    \[
        \abs*{I} = \abs*{K} + \abs*{I \setminus K}.
    \]
    On the other hand, since \(\mathbf{P}\) forms a partition of \(I\), we see that \(\mathbf{P} \setminus \{K\}\) forms a partition of \(I \setminus K\).
    By the induction hypothesis, we thus have
    \[
        \abs*{I \setminus K} = \sum_{J \in \mathbf{P} \setminus \{K\}} \abs*{J}.
    \]
    Combining these two identities (and using the laws of addition for finite sets, see Proposition \ref{7.1.11}(e)) we obtain
    \[
        \abs*{I} = \sum_{J \in \mathbf{P}} \abs*{J}
    \]
    as desired.

    Now suppose that \(b \notin I\), i.e., \(I\) is either \((a, b)\) or \([a, b)\).
    Then one of the intervals \(K\) also is of the form \((c, b)\) or \([c, b)\) (see Exercise \ref{ex 11.1.3}).
            In particular, this means that the set \(I \setminus K\) is also an interval of the form \([a, c]\), \((a, c)\), \((a, c]\), \([a, c)\) when \(c > a\), or a point or empty set when \(a = c\).
    The rest of the argument then proceeds as above.
\end{proof}

\begin{definition}[Finer and coarser partitions]\label{11.1.14}
    Let \(I\) be a bounded interval, and let \(\mathbf{P}\) and \(\mathbf{P}'\) be two partitions of \(I\).
    We say that \(\mathbf{P}'\) is \emph{finer} than \(\mathbf{P}\) (or equivalently, that \(\mathbf{P}\) is \emph{coarser} than \(\mathbf{P}'\)) if for every \(J\) in \(\mathbf{P}'\), there exists a \(K\) in \(\mathbf{P}\) such that \(J \subseteq K\).
\end{definition}

\begin{note}
    There is no such thing as a ``finest'' partition of some interval \(I\).
    (recall all partitions are assumed to be finite.)
    We do not compare partitions of different intervals.
\end{note}

\setcounter{theorem}{15}
\begin{definition}[Common refinement]\label{11.1.16}
    Let \(I\) be a bounded interval, and let \(\mathbf{P}\) and \(\mathbf{P}'\) be two partitions of \(I\).
    We define the \emph{common refinement} \(\mathbf{P} \# \mathbf{P}'\) of \(\mathbf{P}\) and \(\mathbf{P}'\) to be the set
    \[
        \mathbf{P} \# \mathbf{P}' \coloneqq \{K \cap J : K \in \mathbf{P} \land J \in \mathbf{P}'\}.
    \]
\end{definition}

\setcounter{theorem}{17}
\begin{lemma}\label{11.1.18}
    Let \(I\) be a bounded interval, and let \(\mathbf{P}\) and \(\mathbf{P}'\) be two partitions of \(I\).
    Then \(\mathbf{P} \# \mathbf{P}'\) is also a partition of \(I\), and is both finer than \(\mathbf{P}\) and finer than \(\mathbf{P}'\).
\end{lemma}

\begin{proof}
    Let \(x \in I\).
    Then by Definition \ref{11.1.10} we know that \(\exists!\ K \in \mathbf{P}\) such that \(x \in K\).
    Similarly \(\exists!\ J \in \mathbf{P}'\) such that \(x \in J\), thus \(x \in K \cap J\).
    By Definition \ref{11.1.16} we know that \(K \cap J \in \mathbf{P} \# \mathbf{P}'\).
    Thus we have
    \[
        I \subseteq \bigcup \big(\mathbf{P} \# \mathbf{P}'\big).
    \]

    Let \(S \in \mathbf{P} \# \mathbf{P}'\).
    By Definition \ref{11.1.16} we know that \(\exists\ K \in \mathbf{P}\) and \(\exists\ J \in \mathbf{P}'\) such that \(S = K \cap J\).
    Since \(S = K \cap J\), we have \(S \subseteq I\), thus
    \[
        \bigcup \big(\mathbf{P} \# \mathbf{P}'\big) \subseteq I.
    \]
    Combining the result above we have
    \[
        I = \bigcup \big(\mathbf{P} \# \mathbf{P}'\big).
    \]
    Since \(K, J\) are bounded interval, by Corollary \ref{11.1.6} we know that \(S = K \cap J\) is a bounded interval.

    We now claim that \(\forall\ x \in I\), \(\exists!\ S \in \mathbf{P} \# \mathbf{P}'\) such that \(x \in S\).
    So suppose for sake of contradiction that \(\exists\ S_1, S_2 \in \mathbf{P} \# \mathbf{P}'\) such that \(S_1 \neq S_2\), \(x \in S_1\) and \(x \in S_2\).
    By Definition \ref{11.1.16} we know that \(\exists\ K_1 \in \mathbf{P}\) and \(\exists\ J_1 \in \mathbf{P}'\) such that \(S_1 = K_1 \cap J_1\).
    Similarly \(\exists\ K_2 \in \mathbf{P}\) and \(\exists\ J_2 \in \mathbf{P}'\) such that \(S_2 = K_2 \cap J_2\).
    Since \(x \in S_1\), \(x \in K_1\).
    Since \(x \in S_2\), \(x \in K_2\).
    But by Definition \ref{11.1.10} we know that \(K_1 = K_2\), similar argument holds for \(J_1 = J_2\).
    Thus we must have \(S_1 = S_2\), a contradiction.

    We already show that \(\mathbf{P} \# \mathbf{P}' = I\);
    \(\forall\ S \in \mathbf{P} \# \mathbf{P}'\), \(S\) is a bounded interval;
    \(\forall\ x \in I\), \(\exists!\ S \in \mathbf{P} \# \mathbf{P}'\) such that \(x \in I\).
    To show that \(\mathbf{P} \# \mathbf{P}'\) is a partition of \(I\), by Definition \ref{11.1.10} we also need to show that \(\mathbf{P} \# \mathbf{P}'\) is a finite set.
    Let \(f : \mathbf{P} \times \mathbf{P}' \to \mathbf{P} \# \mathbf{P}'\) be a function where
    \[
        \forall\ K \in \mathbf{P} \land \forall\ J \in \mathbf{P}', f(K, J) = K \cap J.
    \]
    From the proof above we can see that \(f\) is surjective.
    By Definition \ref{11.1.10} we have \(\#(\mathbf{P}), \#(\mathbf{P}')\) are finite.
    Thus by Proposition \ref{3.6.14}(e) and Exercise \ref{ex 8.4.3} we have
    \[
        \#(\mathbf{P} \times \mathbf{P}') = \#(\mathbf{P}) \times \#(\mathbf{P}') \geq \#(\mathbf{P} \# \mathbf{P}')
    \]
    and \(\mathbf{P} \# \mathbf{P}'\) is finite, and we conclude that \(\mathbf{P} \# \mathbf{P}'\) is a partition of \(I\).

    From the proof above we have \(\forall\ S \in \mathbf{P} \# \mathbf{P}'\), \(\exists!\ K \in \mathbf{P}\) and \(\exists!\ J \in \mathbf{P}'\) such that \(S = K \cap J\).
    Then we have \(S \subseteq K\) and \(S \subseteq J\).
    Thus by Definition \ref{11.1.14} \(\mathbf{P} \# \mathbf{P}'\) is both finer than \(\mathbf{P}\) and finer than \(\mathbf{P}'\)
\end{proof}

\exercisesection

\begin{exercise}\label{ex 11.1.1}
    Prove Lemma \ref{11.1.4}.
\end{exercise}

\begin{proof}
    See Lemma \ref{11.1.4}.
\end{proof}

\begin{exercise}\label{ex 11.1.2}
    Prove Corollary \ref{11.1.6}.
\end{exercise}

\begin{proof}
    Prove Corollary \ref{11.1.6}.
\end{proof}

\begin{exercise}\label{ex 11.1.3}
    Let \(I\) be a bounded interval of the form \(I = (a, b)\) or \(I = [a, b)\) for some real numbers \(a < b\).
    Let \(I_1, \dots, I_n\) be a partition of \(I\).
    Prove that one of the intervals \(I_j\) in this partition is of the form \(I_j = (c, b)\) or \(I_j = [c, b)\) for some \(a \leq c \leq b\).
\end{exercise}

\begin{proof}
    Let \(\mathbf{P} = \{I_1, \dots, I_n\}\).
    If \(c = b\), then \((c, b) = \emptyset\), and thus by Definition \ref{11.1.10} \(\mathbf{P} \cup \emptyset\) is a partition of \(I\).
    So we only need to proof the cases where \(a \leq c < b\).
    Suppose for sake of contradiction that every interval \(I_j\) in the partition is not of the form \((c, b)\) or \([c, b)\).
    By Definition \ref{11.1.10} this means \(\forall\ x \in I_j\), we have \(x \geq b\) or \(x < c\).
    Since \(I = (a, b)\) or \(I = [a, b)\), we know that \(x < b\), thus we must have \(x < c\).
    This means \(\sup(I_j) \leq c < b\).
    But we know that \(\sup(I) = b\).
    So we have \(\forall\ x \in I\), \(\exists!\ I_j \in \mathbf{P}\) such that \(x \in I_j\), and \(x \leq \sup(I_j) < b = \sup(I)\).
    Thus \(\sup(I) \neq b\), a contradiction.
\end{proof}

\begin{exercise}\label{ex 11.1.4}
    Prove Lemma \ref{11.1.18}.
\end{exercise}

\begin{proof}
    Prove Lemma \ref{11.1.18}.
\end{proof}
\section{Piecewise constant functions}\label{sec 11.2}

\begin{definition}[Constant functions]\label{11.2.1}
    Let \(X\) be a subset of \(\mathbf{R}\), and let \(f : X \to \mathbf{R}\) be a function.
    We say that \(f\) is \emph{constant} iff there exists a real number \(c\) such that \(f(x) = c\) for all \(x \in X\).
    If \(E\) is a subset of \(X\), we say that \(f\) is \emph{constant on} \(E\) if the restriction \(f|_E\) of \(f\) to \(E\) is constant, in other words there exists a real number \(c\) such that \(f(x) = c\) for all \(x \in E\).
    We refer to \(c\) as the \emph{constant value} of \(f\) on \(E\).
\end{definition}

\begin{remark}\label{11.2.2}
    If \(E\) is a non-empty set, then a function \(f\) which is constant on \(E\) can have only one constant value;
    However, if \(E\) is empty, every real number \(c\) is a constant value for \(f\) on \(E\).
\end{remark}

\begin{definition}[Piecewise constant functions I]\label{11.2.3}
    Let \(I\) be a bounded interval, let \(f : I \to \mathbf{R}\) be a function, and let \(\mathbf{P}\) be a partition of \(I\).
    We say that \(f\) is \emph{piecewise constant with respect to \(\mathbf{P}\)} if for every \(J \in \mathbf{P}\), \(f\) is constant on \(J\).
\end{definition}

\setcounter{theorem}{4}
\begin{definition}[Piecewise constant functions II]\label{11.2.5}
    Let \(I\) be a bounded interval, and let \(f : I \to \mathbf{R}\) be a function.
    We say that \(f\) is \emph{piecewise constant on \(I\)} if there exists a partition \(\mathbf{P}\) of \(I\) such that \(f\) is piecewise constant with respect to \(\mathbf{P}\).
\end{definition}

\setcounter{theorem}{6}
\begin{lemma}\label{11.2.7}
    Let \(I\) be a bounded interval, let \(\mathbf{P}\) be a partition of \(I\), and let \(f : I \to \mathbf{R}\) be a function which is piecewise constant with respect to \(\mathbf{P}\).
    Let \(\mathbf{P}'\) be a partition of \(I\) which is finer than \(\mathbf{P}\).
    Then \(f\) is also piecewise constant with respect to \(\mathbf{P}'\).
\end{lemma}

\begin{proof}
    Let \(K \in \mathbf{P}'\).
    Since \(\mathbf{P}'\) is finer than \(\mathbf{P}\), by Definition \ref{11.1.14} \(\exists\ J \in \mathbf{P}\) such that \(K \subseteq J\).
    Since \(f\) is piecewise constant with respect to \(\mathbf{P}\), by Definition \ref{11.2.3} we know that \(\forall\ x \in J\), \(f(x)\) is constant.
    Thus \(\forall\ x \in K\), \(x \in J\) and \(f(x)\) is constant.
    By Definition \ref{11.2.3} \(f\) is piecewise constant with respect to \(\mathbf{P}'\).
\end{proof}

\begin{lemma}\label{11.2.8}
    Let \(I\) be a bounded interval, and let \(f : I \to \mathbf{R}\) and \(g : I \to \mathbf{R}\) be piecewise constant functions on \(I\).
    Then the functions \(f + g\), \(f - g\), \(\max(f, g)\), \(\min(f, g)\) and \(fg\) are also piecewise constant functions on \(I\).
    Here of course \(\max(f, g) : I \to \mathbf{R}\) is the function \(\max(f, g)(x) \coloneqq \max(f(x), g(x))\).
    If \(g\) does not vanish anywhere on \(I\) (i.e., \(g(x) \neq 0\) for all \(x \in I\)) then \(f / g\) is also a piecewise constant function on \(I\).
\end{lemma}

\begin{proof}
    Since \(f\) is piecewise constant function on \(I\), by Definition \ref{11.2.5} \(\exists\ \mathbf{P}\) such that \(\mathbf{P}\) is a partition of \(I\) and \(f\) is piecewise constant with respect to \(\mathbf{P}\).
    Similarly since \(g\) is piecewise constant function on \(I\), by Definition \ref{11.2.5} \(\exists\ \mathbf{P}'\) such that \(\mathbf{P}'\) is a partition of \(I\) and \(g\) is piecewise constant with respect to \(\mathbf{P}'\).
    By Lemma \ref{11.1.18} we know that \(\mathbf{P} \# \mathbf{P}'\) is also a partition of \(I\) and \(\mathbf{P} \# \mathbf{P}'\) is both finer than \(\mathbf{P}\) and finer than \(\mathbf{P}'\).
    By Lemma \ref{11.2.7} we know that both \(f\) and \(g\) are piecewise constant with respect to \(\mathbf{P} \# \mathbf{P}'\).

    Now we show that \(f, g\) remain piecewise constant functions on \(I\) after algebraic operation.
    Since \(\forall\ J \in \mathbf{P} \# \mathbf{P}'\), we have \(\forall\ x \in J\), \(f(x)\) is constant and \(g(x)\) is constant.
    Thus we know that \(f(x) + g(x)\), \(f(x) - g(x)\), \(\max(f(x), g(x))\), \(\min(f(x), g(x))\) and \(f(x) g(x)\) are constant.
    If \(g(x) \neq 0\), then we also have \(f(x) / g(x)\) is constant.
    Thus by Definition \ref{11.2.3} \(f + g\), \(f - g\), \(\max(f, g)\), \(\min(f, g)\), \(fg\) is piecewise constant with respect to \(\mathbf{P} \# \mathbf{P}'\), and when \(g(x) \neq 0\) we have \(f / g\) is piecewise constant with respect to \(\mathbf{P} \# \mathbf{P}'\).
    By Definition \ref{11.2.5} \(f + g\), \(f - g\), \(\max(f, g)\), \(\min(f, g)\), \(fg\) is piecewise constant on \(I\), and when \(g(x) \neq 0\) we have \(f / g\) is piecewise constant on \(I\).
\end{proof}

\begin{definition}[Piecewise constant integral I]\label{11.2.9}
    Let \(I\) be a bounded interval, let \(\mathbf{P}\) be a partition of \(I\).
    Let \(f : I \to \mathbf{R}\) be a function which is piecewise constant with respect to \(\mathbf{P}\).
    Then we define the \emph{piecewise constant integral} \(p.c. \int_{[\mathbf{P}]} f\) of \(f\) with respect to the partition \(\mathbf{P}\) by the formula
    \[
        p.c. \int_{[\mathbf{P}]} f \coloneqq \sum_{J \in \mathbf{P}} c_J \abs*{J},
    \]
    where for each \(J\) in \(\mathbf{P}\), we let \(c_J\) be the constant value of \(f\) on \(J\).
\end{definition}

\begin{remark}\label{11.2.10}
    This definition seems like it could be ill-defined, because if \(J\) is empty then every number \(c_J\) can be the constant value of \(f\) on \(J\), but fortunately in such cases \(\abs*{J}\) is zero and so the choice of \(c_J\) is irrelevant.
    The notation \(p.c. \int_{[\mathbf{P}]} f\) is rather artificial, but we shall only need it temporarily, en route to a more useful definition.
    Note that since \(\mathbf{P}\) is finite, the sum \(\sum_{J \in \mathbf{P}} c_J \abs*{J}\) is always well-defined
    (it is never divergent or infinite).
\end{remark}

\begin{remark}\label{11.2.11}
    The piecewise constant integral corresponds intuitively to one's notion of area, given that the area of a rectangle ought to be the product of the lengths of the sides.
    (Of course, if \(f\) is negative somewhere, then the ``area'' \(c_J \abs*{J}\) would also be negative.)
\end{remark}

\setcounter{theorem}{12}
\begin{proposition}[Piecewise constant integral is independent of partition]\label{11.2.13}
    Let \(I\) be a bounded interval, and let \(f : I \to \mathbf{R}\) be a function.
    Suppose that \(\mathbf{P}\) and \(\mathbf{P}'\) are partitions of \(I\) such that \(f\) is piecewise constant both with respect to \(\mathbf{P}\) and with respect to \(\mathbf{P}'\).
    Then \(p.c. \int_{[\mathbf{P}]} f = p.c. \int_{[\mathbf{P}']} f\).
\end{proposition}

\begin{proof}
    By Lemma \ref{11.1.18} we know that \(\mathbf{P} \# \mathbf{P}'\) is a partition of \(I\) and is both finer than \(\mathbf{P}\) and finer than \(\mathbf{P}'\), thus by Definition \ref{11.2.9} we have
    \[
        p.c. \int_{[\mathbf{P} \# \mathbf{P}']} f = \sum_{J \in \mathbf{P} \# \mathbf{P}'} c_J \abs*{J}.
    \]
    By Theorem \ref{11.1.13}, we know that
    \[
        \abs*{I} = \sum_{J \in \mathbf{P}} \abs*{J} = \sum_{J \in \mathbf{P} \# \mathbf{P}'} \abs*{J}.
    \]
    By Definition \ref{11.1.14}, \(\forall\ S \in \mathbf{P} \# \mathbf{P}'\), \(\exists\ K \in \mathbf{P}\) such that \(S \subseteq K\).
    Now we fix such \(K\) and let \(\mathbf{P}_K\) be the set
    \[
        \mathbf{P}_K = \{S \in \mathbf{P} \# \mathbf{P}' : S \subseteq K\}.
    \]
    We claim that \(\mathbf{P}_K\) is a partition of \(K\).
    We know that \(\mathbf{P}_K\) is finite since \(\mathbf{P}_K \subseteq \mathbf{P} \# \mathbf{P}'\) and \(\mathbf{P} \# \mathbf{P}'\) is finite.
    Since \(\mathbf{P} \# \mathbf{P}'\) is a partition of \(I\), by Definition \ref{11.1.10} \(\forall\ S \in \mathbf{P}_K\), \(S\) is a bounded interval, and if \(S' \in \mathbf{P}_K\) and \(S \neq S'\) then \(S \cap S' = \emptyset\).
    Let \(x \in K\).
    By Definition \ref{11.1.10} we must have \(x \in I\), and \(\exists!\ K' \in \mathbf{P}'\) such that \(x \in K'\).
    By Definition \ref{11.1.16} we know that \(K \cap K' \in \mathbf{P} \# \mathbf{P}'\).
    Since \(K \cap K' \subseteq K\), we have \(x \in \bigcup \mathbf{P}_K\), so \(K \subseteq \bigcup \mathbf{P}_K\).
    By the definition of \(\mathbf{P}_K\) we know that \(\bigcup \mathbf{P}_K \subseteq K\), thus by Proposition \ref{3.1.18} we have \(K = \bigcup \mathbf{P}_K\) and by Definition \ref{11.1.10} \(\mathbf{P}_K\) is a partition of \(K\).

    Now we show that \(\bigcup_{K \in \mathbf{P}} \mathbf{P}_K = \mathbf{P} \# \mathbf{P}'\).
    By the definition of \(\mathbf{P}_K\) we have \(\bigcup_{K \in \mathbf{P}} \mathbf{P}_K \subseteq \mathbf{P} \# \mathbf{P}'\).
    Let \(S \in \mathbf{P} \# \mathbf{P}'\).
    Since \(\mathbf{P} \# \mathbf{P}'\) is finer than \(\mathbf{P}\), by Definition \ref{11.1.14} \(\exists\ K \in \mathbf{P}\) such that \(S \subseteq K\).
    Thus \(S \in \mathbf{P}_K\) and \(\mathbf{P} \# \mathbf{P}' \subseteq \bigcup_{K \in \mathbf{P}} \mathbf{P}_K\).
    Again by Proposition \ref{3.1.18} we have \(\bigcup_{K \in \mathbf{P}} \mathbf{P}_K = \mathbf{P} \# \mathbf{P}'\).

    Since \(f\) is piecewise constant with respect to \(\mathbf{P}\), by Lemma \ref{11.2.7} we know that \(f\) is piecewise constant with respect to \(\mathbf{P} \# \mathbf{P}'\).
    So we have
    \begin{align*}
        p.c. \int_{[\mathbf{P} \# \mathbf{P}']} f & = \sum_{J \in \mathbf{P} \# \mathbf{P}'} c_J \abs*{J}                        & \text{(by Definition \ref{11.2.9})}     \\
                                                  & = \sum_{J \in \bigcup_{K \in \mathbf{P}} \mathbf{P}_K} c_J \abs*{J}                                                    \\
                                                  & = \sum_{K \in \mathbf{P}} \sum_{J \in \mathbf{P}_K} c_J \abs*{J}             & \text{(by Proposition \ref{7.1.11}(e))} \\
                                                  & = \sum_{K \in \mathbf{P}} \sum_{J \in \mathbf{P}_K} c_K \abs*{J}             & (J \subseteq K)                         \\
                                                  & = \sum_{K \in \mathbf{P}} c_K \bigg(\sum_{J \in \mathbf{P}_K} \abs*{J}\bigg)                                           \\
                                                  & = \sum_{K \in \mathbf{P}} c_K \abs*{K}                                       & \text{(by Theorem \ref{11.1.13})}       \\
                                                  & = p.c. \int_{[\mathbf{P}]} f.                                                & \text{(by Definition \ref{11.2.9})}
    \end{align*}
    Using similar arguments we can show that \(p.c. \int_{[\mathbf{P}']} f = p.c. \int_{[\mathbf{P} \# \mathbf{P}']} f\).
    Thus we have \(p.c. \int_{[\mathbf{P}]} f = p.c. \int_{[\mathbf{P}']} f\).
\end{proof}

\begin{definition}[Piecewise constant integral II]\label{11.2.14}
    Let \(I\) be a bounded interval, and let \(f : I \to \mathbf{R}\) be a piecewise constant function on \(I\).
    We define the \emph{piecewise constant integral} \(p.c. \int_I f\) by the formula
    \[
        p.c. \int_I f \coloneqq p.c. \int_{[\mathbf{P}]} f,
    \]
    where \(\mathbf{P}\) is any partition of \(I\) with respect to which \(f\) is piecewise constant.
    (Note that Proposition \ref{11.2.13} tells us that the precise choice of this partition is irrelevant.)
\end{definition}

\setcounter{theorem}{15}
\begin{theorem}[Laws of integration]\label{11.2.16}
    Let \(I\) be a bounded interval, and let \(f : I \to \mathbf{R}\) and \(g : I \to \mathbf{R}\) be piecewise constant functions on \(I\).
    \begin{enumerate}
        \item We have \(p.c. \int_I (f + g) = p.c. \int_I f + p.c. \int_I g\).
        \item For any real number \(c\), we have \(p.c. \int_I (cf) = c (p.c. \int_I f)\).
        \item We have \(p.c. \int_I (f - g) = p.c. \int_I f - p.c. \int_I g\).
        \item If \(f(x) \geq 0\) for all \(x \in I\), then \(p.c. \int_I f \geq 0\).
        \item If \(f(x) \geq g(x)\) for all \(x \in I\), then \(p.c. \int_I f \geq p.c. \int_I g\).
        \item If \(f\) is the constant function \(f(x) = c\) for all \(x \in I\), then \(p.c. \int_I f = c \abs*{I}\).
        \item Let \(J\) be a bounded interval containing \(I\) (i.e., \(I \subseteq J\)), and let \(F : J \to \mathbf{R}\) be the function
              \[
                  F(x) \coloneqq \begin{cases}
                      f(x) & \text{if } x \in I    \\
                      0    & \text{if } x \notin I
                  \end{cases}
              \]
              Then \(F\) is piecewise constant on \(J\), and \(p.c. \int_I F = p.c. \int_I f\).
        \item Suppose that \(\{J, K\}\) is a partition of \(I\) into two intervals \(J\) and \(K\).
              Then the function \(f|_J : J \to \mathbf{R}\) and \(f|_K : K \to \mathbf{R}\) are piecewise constant on \(J\) and \(K\) respectively, and we have
              \[
                  p.c. \int_I f = p.c. \int_I f|_J + p.c. \int_I f|_K.
              \]
    \end{enumerate}
\end{theorem}

\begin{proof}{(a)}
    Since \(f, g\) are both piecewise constant on \(I\), by Lemma \ref{11.2.8} \(f + g\) is also piecewise constant on \(I\).
    By Definition \ref{11.2.3}, \(\exists\ \mathbf{P}_f, \mathbf{P}_g\) such that \(\mathbf{P}_f, \mathbf{P}_g\) are partitions of \(I\), \(f\) is piecewise constant with respect to \(\mathbf{P}_f\) and \(g\) is piecewise constant with respect to \(\mathbf{P}_g\).
    Let \(\mathbf{P} = \mathbf{P}_f \# \mathbf{P}_g\).
    Then by Lemma \ref{11.1.18} we know that \(\mathbf{P}\) is also a partition of \(I\) and by Lemma \ref{11.2.7} \(f, g\) are piecewise constant with respect to \(\mathbf{P}\).
    Now let \(J \in \mathbf{P}\), let \(f_J \in \mathbf{R}\) be the constant value of \(f\) on \(J\) and \(g_J \in \mathbf{R}\) be the constant value of \(g\) on \(J\).
    Then by Definition \ref{11.2.1} \(f_J + g_J\) is also a constant of \(f + g\) on \(J\).
    Thus we have \(f + g\) is piecewise constant with respect to \(\mathbf{P}\) and
    \begin{align*}
        p.c. \int_I f + p.c. \int_I g & = p.c. \int_{[\mathbf{P}]} f + p.c. \int_{[\mathbf{P}]} g                     & \text{(by Definition \ref{11.2.14})}    \\
                                      & = \sum_{J \in \mathbf{P}} f_J \abs*{J} + \sum_{J \in \mathbf{P}} g_J \abs*{J} & \text{(by Definition \ref{11.2.9})}     \\
                                      & = \sum_{J \in \mathbf{P}} (f_J + g_J) \abs*{J}                                & \text{(by Proposition \ref{7.1.11}(f))} \\
                                      & = p.c. \int_{[\mathbf{P}]} (f_J + g_J)                                        & \text{(by Definition \ref{11.2.9})}     \\
                                      & = p.c. \int_I (f_J + g_J).                                                    & \text{(by Definition \ref{11.2.14})}
    \end{align*}
\end{proof}

\begin{proof}{(b)}
    Since \(f\) is piecewise constant on \(I\), by Lemma \ref{11.2.8} \(cf\) is also piecewise constant on \(I\) (since \(c\) is constant on \(I\)).
    By Definition \ref{11.2.3}, \(\exists\ \mathbf{P}\) such that \(\mathbf{P}\) is a partition of \(I\) and \(f\) is piecewise constant with respect to \(\mathbf{P}\).
    Now let \(J \in \mathbf{P}\) and let \(f_J \in \mathbf{R}\) be the constant value of \(f\) on \(J\).
    Then by Definition \ref{11.2.1} \(c f_J\) is also a constant of \(cf\) on \(J\).
    Thus we have \(cf\) is piecewise constant with respect to \(\mathbf{P}\) and
    \begin{align*}
        c (p.c. \int_I f) & = c (p.c. \int_{[\mathbf{P}]} f)           & \text{(by Definition \ref{11.2.14})}    \\
                          & = c (\sum_{J \in \mathbf{P}} f_J \abs*{J}) & \text{(by Definition \ref{11.2.9})}     \\
                          & = \sum_{J \in \mathbf{P}} c f_J \abs*{J}   & \text{(by Proposition \ref{7.1.11}(g))} \\
                          & = p.c. \int_{[\mathbf{P}]} (c f)           & \text{(by Definition \ref{11.2.9})}     \\
                          & = p.c. \int_I (c f).                       & \text{(by Definition \ref{11.2.14})}
    \end{align*}
\end{proof}

\begin{proof}{(c)}
    We have
    \begin{align*}
        p.c. \int_I f - p.c. \int_I g & = p.c. \int_I f + (-1) p.c. \int_I g                                        \\
                                      & = p.c. \int_I f + p.c. \int_I (-g)   & \text{(by Theorem \ref{11.2.16}(b))} \\
                                      & = p.c. \int_I (f + (-g))             & \text{(by Theorem \ref{11.2.16}(a))} \\
                                      & = p.c. \int_I (f - g).               & \text{(by Definition \ref{9.2.1})}
    \end{align*}
\end{proof}

\begin{proof}{(d)}
    By Definition \ref{11.2.3}, \(\exists\ \mathbf{P}\) such that \(\mathbf{P}\) is a partition of \(I\) and \(f\) is piecewise constant with respect to \(\mathbf{P}\).
    Let \(J \in \mathbf{P}\) and let \(f_J \in \mathbf{R}\) be the constant value of \(f\) on \(J\).
    Since \(\forall\ x \in I\), \(f(x) \geq 0\), we then have \(f_J \geq 0\) and \(f_J \abs*{J} \geq 0\).
    Thus
    \begin{align*}
        p.c. \int_I f & = p.c. \int_{[\mathbf{P}]} f           & \text{(by Definition \ref{11.2.14})}    \\
                      & = \sum_{J \in \mathbf{P}} f_J \abs*{J} & \text{(by Definition \ref{11.2.9})}     \\
                      & \geq \sum_{J \in \mathbf{P}} 0         & \text{(by Proposition \ref{7.1.11}(h))} \\
                      & = 0.
    \end{align*}
\end{proof}

\begin{proof}{(e)}
    Since \(f(x) \geq g(x)\) for all \(x \in I\), we have \(f(x) - g(x) \geq 0\) and
    \begin{align*}
        p.c. \int_I f - p.c. \int_I g & = p.c. \int_I (f - g) & \text{(by Theorem \ref{11.2.16}(c))} \\
                                      & \geq 0.               & \text{(by Theorem \ref{11.2.16}(d))}
    \end{align*}
    Thus
    \[
        p.c. \int_I f \geq p.c. \int_I g.
    \]
\end{proof}

\begin{proof}{(f)}
    Since \(I\) is a partition of \(I\), we have
    \begin{align*}
        p.c. \int_I f & = p.c. \int_{[I]} f         & \text{(by Definition \ref{11.2.14})}    \\
                      & = \sum_{J \in I} c \abs*{J} & \text{(by Definition \ref{11.2.9})}     \\
                      & = c \sum_{J \in I} \abs*{J} & \text{(by Proposition \ref{7.1.11}(g))} \\
                      & = c \abs*{I}.               & \text{(by Theorem \ref{11.1.13})}
    \end{align*}
\end{proof}

\begin{proof}{(g)}
    Let \(I_1, I_2\) be the sets
    \[
        I_1 = \{x \in J, x \leq \inf(I) \land x \notin I\}
    \]
    and
    \[
        I_2 = \{x \in J, x \geq \sup(I) \land x \notin I \cup I_1\}.
    \]
    Then we know that \(I \cap I_1 = I \cap I_2 = I_1 \cap I_2 = \emptyset\).
    Let \(\mathbf{P} = \{I_1, I, I_2\}\).
    We know claim that \(\mathbf{P}\) is a partition of \(J\).
    Since \(J\) is a bounded interval, we know that \(\inf(J) = \inf(I_1)\) and \(\sup(J) = \sup(I_2)\).
    Then we have \(I_1 \subseteq [\inf(J), \inf(I)]\) and \(I_2 \subseteq [\sup(I), \sup(J)]\).
    If \(\inf(J) \in J\), then we know that \(I_1 = [\inf(J), \inf(I)]\) or \(I_1 = [\inf(J), \inf(I))\), which depends on whether \(\inf(I) \in I\).
    Otherwise we have \(I_1 = (\inf(J), \inf(I)]\) or \(I_1 = (\inf(J), \inf(I))\), which again depends on whether \(\inf(I) \in I\).
    Using similar arguments we know thtat \(I_2\) can be one of \((\sup(I), \sup(J))\), \([\sup(I), \sup(J))\), \((\sup(I), \sup(J)]\) or \([\sup(I), \sup(J)]\).
    Thus \(I_1, I_2\) are bounded intervals.
    By the definition of \(I_1, I_2\), we know that \(\bigcup \mathbf{P} \subseteq J\).
    To show that \(\bigcup \mathbf{P} = J\), by Proposition \ref{3.1.18} we need to show that \(J \subseteq \bigcup \mathbf{P}\).
    Let \(x \in J\).
    If \(x \in I\), we have \(x \in \bigcup \mathbf{P}\).
    If \(x \notin I\), we then have \(x \leq \inf(I) \lor x \geq \sup(I)\), thus \(x \in I_1 \lor x \in I_2\), and again \(x \in \bigcup \mathbf{P}\).
    Since \(x\) is arbitrary, we have \(J \subseteq \mathbf{P}\).
    Since \(\bigcup \mathbf{P} = J\) and \(\mathbf{P}\) is finite (\(\abs*{P} = 3\)), by Definition \ref{11.1.10} \(\mathbf{P}\) is a partition of \(J\).

    Now we show that \(F\) is piecewise constant on \(J\).
    Since \(f\) is piecewise constant on \(I\), by Definition \ref{11.2.5} \(\exists\ \mathbf{P}_I\) such that \(\mathbf{P}_I\) be the partition of \(I\) and \(f\) is piecewise constant with respect to \(\mathbf{P}_I\).
    By hypothesis we know that \(\forall\ K \in \mathbf{P}_I\), \(F\) is piecewise constant on \(K\) with constant value \(F(x) = f(x)\) for all \(x \in K\), thus by Definition \ref{11.2.5} \(F\) is piecewise constant on \(I\).
    Since \(\forall\ x \in I_1\), \(x \notin I\), by hypothesis we know that \(F(x) = 0\), thus \(F\) is piecewise constant on \(I_1\).
    Similar arguments show that \(F\) is piecewise constant on \(I_2\).
    Thus \(F\) is piecewise constant on \(\mathbf{P}\), and we have
    \begin{align*}
        p.c. \int_J F & = p.c. \int_{[\mathbf{P}]} F                                                       & \text{(by Definition \ref{11.2.14})}    \\
                      & = \sum_{K \in \mathbf{P}} c_K \abs*{K}                                             & \text{(by Definition \ref{11.2.9})}     \\
                      & = c_{I_1} \abs*{I_1} + \sum_{K \in \mathbf{P}_I} c_K \abs*{K} + c_{I_2} \abs*{I_2} & \text{(by Proposition \ref{7.1.11}(e))} \\
                      & = 0 \abs*{I_1} + \sum_{K \in \mathbf{P}_I} c_K \abs*{K} + 0 \abs*{I_2}             & \text{(by hypothesis)}                  \\
                      & = \sum_{K \in \mathbf{P}_I} c_K \abs*{K}                                                                                     \\
                      & = p.c. \int_{[\mathbf{P}_I]} f                                                     & \text{(by Definition \ref{11.2.9})}     \\
                      & = p.c. \int_I f.                                                                   & \text{(by Definition \ref{11.2.14})}
    \end{align*}
\end{proof}

\begin{proof}{(h)}
    We first show that \(f|_J\) is piecewise constant on \(J\) and \(f|_K\) is piecewise constant on \(K\).
    Since \(f\) is a piecewise constant function on \(I\), by Definition \ref{11.2.5} \(\exists\ \mathbf{P}\) such that \(\mathbf{P}\) is a partition of \(I\) and \(f\) is piecewise constant with respect to \(\mathbf{P}\).
    Let \(\mathbf{P}_J, \mathbf{P}_K\) be the sets
    \[
        \mathbf{P}_J = \{S \cap J : S \in \mathbf{P}\}
    \]
    and
    \[
        \mathbf{P}_K = \{S \cap K : S \in \mathbf{P}\}.
    \]
    Since \(\forall\ S_J \in \mathbf{P}_J\), \(S_J \in \mathbf{P}\), by Definition \ref{11.2.3} we know that \(f\) is constant on \(S_J\),
    By Definition \ref{11.1.10} \(S_J\) is a bounded interval, and \(S_{J'} \in \mathbf{P}_J\) and \(S_{J'} \neq S_J \implies S_{J'} \cap S_J = \emptyset\).
    By the definition of \(\mathbf{P}_J\) we know that \(\bigcup \mathbf{P}_J \subseteq J\).
    Let \(x \in J\).
    Since \(x \in J\), \(x \in I\), by Definition \ref{11.1.10} \(\exists!\ S_J \in \mathbf{P}_J\) such that \(x \in S_J\).
    Then we have \(x \in J \cap S_J\) and \(x \in \bigcup \mathbf{P}_J\), thus \(J \subseteq \bigcup \mathbf{P}_J\) and by Proposition \ref{3.1.18} \(J = \bigcup \mathbf{P}_J\).
    Since \(\mathbf{P}_J \subseteq \mathbf{P}\) and \(\mathbf{P}\) is finite, we know that \(\mathbf{P}_J\) is finite.
    Thus by Definition \ref{11.1.10} \(\mathbf{P}_J\) is a partition of \(J\).
    Using similar arguments we can show that \(\mathbf{P}_K\) is a partition of \(K\).
    Since \(\forall\ S_J \in \mathbf{P}_J\), \(S_J \in \mathbf{P}\) and \(f\) is piecewise constant on \(S_J\), by Definition \ref{11.2.3} \(f|_J\) is piecewise constant with respect to \(\mathbf{P}_J\).
    Using similar arguments we know that \(f|_K\) is piecewise constant with respect to \(\mathbf{P}_K\).
    Thus by Definition \ref{11.2.5} \(f|_J\) is piecewise constant on \(J\) and \(f|_K\) is piecewise constant on \(K\).

    Now we show that \(\mathbf{P} = \mathbf{P}_J \cup \mathbf{P}_K\).
    By the definition of \(\mathbf{P}_J\) and \(\mathbf{P}_K\) we know that \(\mathbf{P}_J \cup \mathbf{P}_J \subseteq \mathbf{P}\).
    Let \(S \in \mathbf{P}\).
    If \(S = \emptyset\), then \(S \subseteq \mathbf{P}_J \cup \mathbf{P}_K\).
    If \(S \neq \emptyset\), since \(S \subseteq I\) and \(\{J, K\}\) is a partition of \(I\), we know that \(S \cap (J \cup K) \neq \emptyset\).
    Thus we have \(S \in \mathbf{P}_J\) or \(S \in \mathbf{P}_K\), which means \(\mathbf{P} \subseteq \mathbf{P}_J \cup \mathbf{P}_K\).
    By Proposition \ref{3.1.18} we have \(\mathbf{P} = \mathbf{P}_J \cup \mathbf{P}_K\).
    Thus we have
    \begin{align*}
        p.c. \int_J f|_J + p.c. \int_K f|_K & = p.c. \int_{[\mathbf{P}_J]} f|_J + p.c. \int_{[\mathbf{P}_K]} f|_K               & \text{(by Definition \ref{11.2.14})}    \\
                                            & = \sum_{S \in \mathbf{P}_J} c_S \abs*{S} + \sum_{S \in \mathbf{P}_K} c_S \abs*{S} & \text{(by Proposition \ref{7.1.11}(e))} \\
                                            & = \sum_{S \in \mathbf{P}_J \cup \mathbf{P}_K} c_S \abs*{S}                        & \text{(by Definition \ref{11.2.9})}     \\
                                            & = \sum_{S \in \mathbf{P}} c_S \abs*{S}                                                                                      \\
                                            & = p.c. \int_{[\mathbf{P}]} f                                                      & \text{(by Definition \ref{11.2.9})}     \\
                                            & = p.c. \int_I f.                                                                  & \text{(by Definition \ref{11.2.14})}
    \end{align*}
\end{proof}
\section{Upper and lower Riemann integrals}\label{sec 11.3}

\begin{definition}[Majorization of functions]\label{11.3.1}
    Let \(f : I \to \mathbf{R}\) and \(g : I \to \mathbf{R}\).
    We say that \(g\) \emph{majorizes} \(f\) on \(I\) if we have \(g(x) \geq f(x)\) for all \(x \in I\), and that \(g\) \emph{minorizes} \(f\) on \(I\) if \(g(x) \leq f(x)\) for all \(x \in I\).
\end{definition}

\begin{definition}[Upper and lower Riemann integrals]\label{11.3.2}
    Let \(f : I \to \mathbf{R}\) be a bounded function defined on a bounded interval \(I\).
    We define the \emph{upper Riemann integral} \(\overline{\int}_I f\) by the formula
    \[
        \overline{\int}_I f \coloneqq \inf\{p.c. \int_I g : g \text{ is a piecewise constant function on \(I\) which majorizes } f\}
    \]
    and the \emph{lower Riemann integral} \(\underline{\int}_I f\) by the formula
    \[
        \underline{\int}_I f \coloneqq \sup\{p.c. \int_I g : g \text{ is a piecewise constant function on \(I\) which minorizes } f\}.
    \]
\end{definition}

\begin{lemma}\label{11.3.3}
    Let \(f : I \to \mathbf{R}\) be a function on a bounded interval \(I\) which is bounded by some real number \(M\), i.e., \(-M \leq f(x) \leq M\) for all \(x \in I\).
    Then we have
    \[
        -M \abs*{I} \leq \underline{\int}_I f \leq \overline{\int}_I f \leq M \abs*{I}.
    \]
    in particular, both the lower and upper Riemann integrals are real numbers (i.e., they are not infinite).
\end{lemma}

\begin{proof}
    The function \(g : I \to \mathbf{R}\) defined by \(g(x) = M\) is constant, hence piecewise constant, and majorizes \(f\);
    thus \(\overline{\int}_I f \leq p.c. \int_I g = M \abs*{I}\) by definition of the upper Riemann integral.
    A similar argument gives \(-M \abs*{I} \leq \underline{\int}_I f\).
    Finally, we have to show that \(\underline{\int}_I f \leq \overline{\int}_I f\).
    Let \(g\) be any piecewise constant function majorizing \(f\), and let \(h\) be any piecewise constant function minorizing \(f\).
    Then \(g\) majorizes \(h\), and hence \(p.c. \int_I h \leq p.c. \int_I g\).
    Taking suprema in \(h\), we obtain that \(\underline{\int}_I f \leq p.c. \int_I g\).
    Taking infima in \(g\), we thus obtain \(\underline{\int}_I f \leq \overline{\int}_I f\), as desired.
\end{proof}

\begin{definition}[Riemann integral]\label{11.3.4}
    Let \(f : I \to \mathbf{R}\) be a bounded function on a bounded interval \(I\).
    If \(\underline{\int}_I f = \overline{\int}_I f\), then we say that \(f\) is \emph{Riemann integrable on \(I\)} and define
    \[
        \int_I f \coloneqq \underline{\int}_I f = \overline{\int}_I f.
    \]
    If the upper and lower Riemann integrals are unequal, we say that \(f\) is not Riemann integrable.
\end{definition}

\begin{remark}\label{11.3.5}
    Compare this definition to the relationship between the \(\limsup\), \(liminf\), and limit of a sequence \(a_n\) that was established in Proposition \ref{6.4.12}(f);
    the \(\limsup\) is always greater than or equal to the \(\liminf\), but they are only equal when the sequence converges, and in this case they are both equal to the limit of the sequence.
    The definition given above may differ from the definition you may have encountered in your calculus courses, based on Riemann sums.
    However, the two definitions turn out to be equivalent.
\end{remark}

\begin{remark}\label{11.3.6}
    Note that we do not consider unbounded functions to be Riemann integrable;
    an integral involving such functions is known as an \emph{improper integral}.
    It is possible to still evaluate such integrals using more sophisticated integration methods (such as the Lebesgue integral).
\end{remark}

\begin{lemma}\label{11.3.7}
    Let \(f : I \to \mathbf{R}\) be a piecewise constant function on a bounded interval \(I\).
    Then \(f\) is Riemann integrable, and \(\int_I f = p.c. \int_I f\).
\end{lemma}

\begin{proof}
    Since \(f(x) \leq f(x)\), by Definition \ref{11.3.2} we have
    \[
        \overline{\int}_I f \leq p.c. \int_I f
    \]
    and
    \[
        p.c. \int_I f \leq \underline{\int}_I f.
    \]
    By Lemma \ref{11.3.3} we know that
    \[
        p.c. \int_I f \leq \underline{\int}_I f \leq \overline{\int}_I f \leq p.c. \int_I f.
    \]
    Thus by Definition \ref{11.3.4} we have
    \[
        \int_I f = \underline{\int}_I f = \overline{\int}_I f = p.c. \int_I f.
    \]
\end{proof}

\begin{remark}\label{11.3.8}
    Because of this lemma, we will not refer to the piecewise constant integral \(p.c. \int_I\) again, and just use the Riemann integral \(\int_I\) throughout
    (until this integral is itself superceded by the Lebesgue integral).
    We observe one special case of Lemma \ref{11.3.7}:
    if \(I\) is a point or the empty set, then \(\int_I f = 0\) for all functions \(f : I \to \mathbf{R}\).
    (Note that all such functions are automatically constant.)
\end{remark}

\begin{definition}[Riemann sums]\label{11.3.9}
    Let \(f : I \to \mathbf{R}\) be a bounded function on a bounded interval \(I\), and let \(\mathbf{P}\) be a partition of \(I\).
    We define the \emph{upper Riemann sum} \(U(f, \mathbf{P})\) and the \emph{lower Riemann sum} \(L(f, \mathbf{P})\) by
    \[
        U(f, \mathbf{P}) \coloneqq \sum_{J \in \mathbf{P} : J \neq \emptyset} (\sup_{x \in J} f(x)) \abs*{J}
    \]
    and
    \[
        L(f, \mathbf{P}) \coloneqq \sum_{J \in \mathbf{P} : J \neq \emptyset} (\inf_{x \in J} f(x)) \abs*{J}
    \]
\end{definition}

\begin{remark}\label{11.3.10}
    The restriction \(J \neq \emptyset\) is required because the quantities \(\inf_{x \in J} f(x)\) and \(\sup_{x \in J} f(x)\) are infinite (or negative infinite) if \(J\) is empty.
\end{remark}
\section{Basic properties of the Riemann integral}\label{sec 11.4}

\begin{theorem}[Laws of Riemann integration]\label{11.4.1}
    Let \(I\) be a bounded interval, and let \(f : I \to \mathbf{R}\) and \(g : I \to \mathbf{R}\) be Riemann integrable functions on \(I\).
    \begin{enumerate}
        \item The function \(f + g\) is Riemann integrable, and we have \(\int_I (f + g) = \int_I f + \int_I g\).
        \item For any real number \(c\), the function \(cf\) is Riemann integrable, and we have \(\int_I (cf) = c(\int_I f)\).
        \item The function \(f - g\) is Riemann integrable, and we have \(\int_I (f - g) = \int_I f - \int_I g\).
        \item If \(f(x) \geq 0\) for all \(x \in I\), then \(\int_I f \geq 0\).
        \item If \(f(x) \geq g(x)\) for all \(x \in I\), then \(\int_I f \geq \int_I g\).
        \item If \(f\) is the constant function \(f(x) = c\) for all \(x \in I\), then \(\int_I f = c \abs*{I}\).
        \item Let \(J\) be a bounded interval containing \(I\) (i.e., \(I \subseteq J\)), and let \(F : J \to \mathbf{R}\) be the function
              \[
                  F(x) \coloneqq \begin{cases}
                      f(x) & \text{if } x \in I    \\
                      0    & \text{if } x \notin I \\
                  \end{cases}
              \]
              Then \(F\) is Riemann integrable on \(J\), and \(\int_J F = \int_I f\).
        \item Suppose that \(\{J, K\}\) is a partition of \(I\) into two intervals \(J\) and \(K\).
              Then the functions \(f|_J : J \to \mathbf{R}\) and \(f|_K : K \to \mathbf{R}\) are Riemann integrable on \(J\) and \(K\) respectively, and we have
              \[
                  \int_I f = \int_J f|_J + \int_K f|_K.
              \]
    \end{enumerate}
\end{theorem}

\begin{proof}{(a)}
    Since \(f, g\) are Riemann integrable on \(I\), by Definition \ref{11.3.4} we have
    \[
        \int_I f = \overline{\int}_I f = \underline{\int}_I f
    \]
    and
    \[
        \int_I g = \overline{\int}_I g = \underline{\int}_I g.
    \]
    Let \(f_U : I \to \mathbf{R}\) and \(g_U : I \to \mathbf{R}\) be piecewise constant functions on \(I\) which majorizes \(f\) and \(g\), respectively.
    Let \(f_L : I \to \mathbf{R}\) and \(g_L : I \to \mathbf{R}\) be piecewise constant functions on \(I\) which minorizes \(f\) and \(g\), respectively.
    \(f_U, g_U, f_L, g_L\) are well-defined since by Definition \ref{11.3.4} \(f, g\) are bounded functions on a bounded interval \(I\).
    By Definition \ref{11.3.2} we have
    \[
        p.c. \int_I f_L \leq \underline{\int}_I f = \int_I f = \overline{\int}_I f \leq p.c. \int_I f_U
    \]
    and
    \[
        p.c. \int_I g_L \leq \underline{\int}_I g = \int_I g = \overline{\int}_I g \leq p.c. \int_I g_U.
    \]
    By Definition \ref{11.3.4} both \(f, g\) are bounded functions, so \(f + g\) is bounded function, and \(\underline{\int}_I (f + g), \overline{\int}_I (f + g)\) are well-defined (by Definition \ref{11.3.2}).
    By Exercise \ref{ex 11.3.2} we know that \(f_U + g_U\) majorizes \(f + g_U\) and \(f + g_U\) majorizes \(f + g\), thus \(f_U + g_U\) majorizes \(f + g\).
    Similarly \(f_L + g_L\) minorizes \(f + g\).
    Then we have
    \begin{align*}
                 & \overline{\int}_I (f + g) \leq p.c. \int_I (f_U + g_U)                   & \text{(by Definition \ref{11.3.2})}     \\
        \implies & \overline{\int}_I (f + g) \leq p.c. \int_I f_U + p.c. \int_I g_U         & \text{(by Theorem \ref{11.2.16}(a))}    \\
        \implies & \overline{\int}_I (f + g) - p.c. \int_I g_U \leq p.c. \int_I f_U         & \text{(note that \(f_U\) is arbitrary)} \\
        \implies & \overline{\int}_I (f + g) - p.c. \int_I g_U \leq \overline{\int}_I f     & \text{(by Definition \ref{11.3.2})}     \\
        \implies & \overline{\int}_I (f + g) - \overline{\int}_I f \leq p.c. \int_I g_U     & \text{(note that \(g_U\) is arbitrary)} \\
        \implies & \overline{\int}_I (f + g) - \overline{\int}_I f \leq \overline{\int}_I g & \text{(by Definition \ref{11.3.2})}     \\
        \implies & \overline{\int}_I (f + g) \leq \overline{\int}_I f + \overline{\int}_I g &                                         \\
        \implies & \overline{\int}_I (f + g) \leq \int_I f + \int_I g                       & \text{(by Definition \ref{11.3.4})}
    \end{align*}
    and
    \begin{align*}
                 & \underline{\int}_I (f + g) \geq p.c. \int_I (f_L + g_L)                     & \text{(by Definition \ref{11.3.2})}     \\
        \implies & \underline{\int}_I (f + g) \geq p.c. \int_I f_L + p.c. \int_I g_L           & \text{(by Theorem \ref{11.2.16}(a))}    \\
        \implies & \underline{\int}_I (f + g) - p.c. \int_I g_L \geq p.c. \int_I f_L           & \text{(note that \(f_L\) is arbitrary)} \\
        \implies & \underline{\int}_I (f + g) - p.c. \int_I g_L \geq \underline{\int}_I f      & \text{(by Definition \ref{11.3.2})}     \\
        \implies & \underline{\int}_I (f + g) - \underline{\int}_I f \geq p.c. \int_I g_L      & \text{(note that \(g_L\) is arbitrary)} \\
        \implies & \underline{\int}_I (f + g) - \underline{\int}_I f \geq \underline{\int}_I g & \text{(by Definition \ref{11.3.2})}     \\
        \implies & \underline{\int}_I (f + g) \geq \underline{\int}_I f + \underline{\int}_I g &                                         \\
        \implies & \underline{\int}_I (f + g) \geq \int_I f + \int_I g.                        & \text{(by Definition \ref{11.3.4})}
    \end{align*}
    By Lemma \ref{11.3.3} we have
    \[
        \int_I f + \int_I g \leq \underline{\int}_I (f + g) \leq \overline{\int}_I (f + g) \leq \int_I f + \int_I g
    \]
    and thus by Definition \ref{11.3.4} we have
    \[
        \int_I (f + g) = \underline{\int}_I (f + g) = \overline{\int}_I (f + g) = \int_I f + \int_I g.
    \]
\end{proof}

\begin{proof}{(b)}
    Since \(f\) is Riemann integrable on \(I\), by Definition \ref{11.3.4} we have
    \[
        \int_I f = \overline{\int}_I f = \underline{\int}_I f.
    \]
    First suppose that \(c = 0\).
    Then we have \((cf)(x) = 0\) for all \(x \in 0\), thus we have
    \begin{align*}
        \int_I (cf) & = p.c. \int_I (cf) & \text{(by Lemma \ref{11.3.7})} \\
                    & = 0                                                 \\
                    & = c \int_I f.
    \end{align*}

    Next suppose that \(c > 0\).
    Let \(f_U : I \to \mathbf{R}\) be a piecewise constant function on \(I\) which majorizes \(f\).
    Let \(f_L : I \to \mathbf{R}\) be a piecewise constant function on \(I\) which minorizes \(f\).
    \(f_U, f_L\) are well-defined since by Definition \ref{11.3.4} \(f\) is a bounded function on a bounded interval \(I\).
    Then by Definition \ref{11.3.2} we have
    \[
        p.c. \int_I f_L \leq \underline{\int}_I f = \int_I f = \overline{\int}_I f \leq p.c. \int_I f_U.
    \]
    Since \(f\) is a bounded function, \(cf\) is also a bounded function, by Definition \ref{11.3.2} both \(\overline{\int}_I (cf), \underline{\int}_I (cf)\) are well-defined.
    Since \(c > 0\), by Definition \ref{11.3.1} we know that \(c f_U\) majorizes \(c f\) and \(c f_L\) minorizes \(c f\).
    Then we have
    \begin{align*}
                 & \overline{\int}_I (cf) \leq p.c. \int_I (c f_U)                         & \text{(by Definition \ref{11.3.2})}     \\
        \implies & \overline{\int}_I (cf) \leq c \bigg(p.c. \int_I f_U\bigg)               & \text{(by Theorem \ref{11.2.16}(b))}    \\
        \implies & \frac{1}{c} \bigg(\overline{\int}_I (cf)\bigg) \leq p.c. \int_I f_U     & \text{(note that \(f_U\) is arbitrary)} \\
        \implies & \frac{1}{c} \bigg(\overline{\int}_I (cf)\bigg) \leq \overline{\int}_I f & \text{(by Definition \ref{11.3.2})}     \\
        \implies & \overline{\int}_I (cf) \leq c\bigg(\overline{\int}_I f\bigg)                                                      \\
        \implies & \overline{\int}_I (cf) \leq c\bigg(\int_I f\bigg)                       & \text{(by Definition \ref{11.3.4})}
    \end{align*}
    and
    \begin{align*}
                 & \underline{\int}_I (cf) \geq p.c. \int_I (c f_L)                          & \text{(by Definition \ref{11.3.2})}     \\
        \implies & \underline{\int}_I (cf) \geq c \bigg(p.c. \int_I f_L\bigg)                & \text{(by Theorem \ref{11.2.16}(b))}    \\
        \implies & \frac{1}{c} \bigg(\underline{\int}_I (cf)\bigg) \geq p.c. \int_I f_L      & \text{(note that \(f_L\) is arbitrary)} \\
        \implies & \frac{1}{c} \bigg(\underline{\int}_I (cf)\bigg) \geq \underline{\int}_I f & \text{(by Definition \ref{11.3.2})}     \\
        \implies & \underline{\int}_I (cf) \geq c\bigg(\underline{\int}_I f\bigg)                                                      \\
        \implies & \underline{\int}_I (cf) \geq c\bigg(\int_I f\bigg).                       & \text{(by Definition \ref{11.3.4})}
    \end{align*}
    By Lemma \ref{11.3.3} we have
    \[
        c\bigg(\int_I f\bigg) \leq \underline{\int}_I (cf) \leq \overline{\int}_I (cf) \leq c\bigg(\int_I f\bigg)
    \]
    and thus by Definition \ref{11.3.4} we have
    \[
        \int_I (cf) = \underline{\int}_I (cf) = \overline{\int}_I (cf) = c\bigg(\int_I f\bigg).
    \]

    Finally suppose that \(c < 0\).
    Using the same definition of \(f_U, f_L\) we have
    \begin{align*}
                 & \overline{\int}_I (cf) \leq p.c. \int_I (c f_U)                                              & \text{(by Definition \ref{11.3.2})}  \\
        \implies & \overline{\int}_I (cf) \leq c \bigg(p.c. \int_I f_U\bigg)                                    & \text{(by Theorem \ref{11.2.16}(b))} \\
        \implies & \frac{1}{c} \bigg(\overline{\int}_I (cf)\bigg) \geq p.c. \int_I f_U                                                                 \\
        \implies & \frac{1}{c} \bigg(\overline{\int}_I (cf)\bigg) \geq p.c. \int_I f_U \geq \overline{\int}_I f & \text{(by Definition \ref{11.3.2})}  \\
        \implies & \overline{\int}_I (cf) \leq c\bigg(\overline{\int}_I f\bigg)                                                                        \\
        \implies & \overline{\int}_I (cf) \leq c\bigg(\int_I f\bigg)                                            & \text{(by Definition \ref{11.3.4})}
    \end{align*}
    and
    \begin{align*}
                 & \underline{\int}_I (cf) \geq p.c. \int_I (c f_L)                                               & \text{(by Definition \ref{11.3.2})}  \\
        \implies & \underline{\int}_I (cf) \geq c \bigg(p.c. \int_I f_L\bigg)                                     & \text{(by Theorem \ref{11.2.16}(b))} \\
        \implies & \frac{1}{c} \bigg(\underline{\int}_I (cf)\bigg) \leq p.c. \int_I f_L                                                                  \\
        \implies & \frac{1}{c} \bigg(\underline{\int}_I (cf)\bigg) \leq p.c. \int_I f_L \leq \underline{\int}_I f & \text{(by Definition \ref{11.3.2})}  \\
        \implies & \underline{\int}_I (cf) \geq c\bigg(\underline{\int}_I f\bigg)                                                                        \\
        \implies & \underline{\int}_I (cf) \geq c\bigg(\int_I f\bigg).                                            & \text{(by Definition \ref{11.3.4})}
    \end{align*}
    By Lemma \ref{11.3.3} we have
    \[
        c\bigg(\int_I f\bigg) \leq \underline{\int}_I (cf) \leq \overline{\int}_I (cf) \leq c\bigg(\int_I f\bigg)
    \]
    and thus by Definition \ref{11.3.4} we have
    \[
        \int_I (cf) = \underline{\int}_I (cf) = \overline{\int}_I (cf) = c\bigg(\int_I f\bigg).
    \]
    We conclude that \(\forall c \in \mathbf{R}\), \(\int_I (cf) = c (\int_I f)\).
\end{proof}

\begin{proof}{(c)}
    We have
    \begin{align*}
        \int_I f - \int_I g & = \int_I f + \int_I (-g)    & \text{(by Theorem \ref{11.4.1}(b))} \\
                            & = \int_I \big(f + (-g)\big) & \text{(by Theorem \ref{11.4.1}(a))} \\
                            & = \int_I (f - g).           & \text{(by Definition \ref{9.2.1})}
    \end{align*}
\end{proof}

\begin{proof}{(d)}
    Let \(f_U : I \to \mathbf{R}\) be a piecewise constant function on \(I\) which majorizes \(f\).
    \(f_U\) is well-defined since by Definition \ref{11.3.4} \(f\) is a bounded function on a bounded interval \(I\).
    Since \(0 \leq f(x) \leq f_U(x)\) for every \(x \in I\), we have
    \begin{align*}
                 & 0 \leq p.c. \int_I f_U     & \text{(by Theorem \ref{11.2.16}(d))} \\
        \implies & 0 \leq \overline{\int}_I f & \text{(by Definition \ref{11.3.2})}  \\
        \implies & 0 \leq \int_I f.           & \text{(by Definition \ref{11.3.4})}
    \end{align*}
\end{proof}

\begin{proof}{(e)}
    We have \(f(x) - g(x) \geq 0\) for every \(x \in I\) and by Theorem \ref{11.4.1}(c) \(f - g\) is Riemann integrable on \(I\).
    Thus
    \begin{align*}
                 & \int_I (f - g) \geq 0      & \text{(by Theorem \ref{11.4.1}(d))} \\
        \implies & \int_I f - \int_I g \geq 0 & \text{(by Theorem \ref{11.4.1}(c))} \\
        \implies & \int_I f \geq \int_I g.
    \end{align*}
\end{proof}

\begin{proof}{(f)}
    We have
    \begin{align*}
        \int_I f & = p.c. \int_I f & \text{(by Lemma \ref{11.3.7})}       \\
                 & = c \abs*{I}.   & \text{(by Theorem \ref{11.2.16}(f))}
    \end{align*}
\end{proof}

\begin{proof}{(g)}
    Let \(f_U : I \to \mathbf{R}\) be a piecewise constant function on \(I\) which majorizes \(f\).
    Let \(f_L : I \to \mathbf{R}\) be a piecewise constant function on \(I\) which minorizes \(f\).
    \(f_U, f_L\) are well-defined since by Definition \ref{11.3.4} \(f\) is a bounded function on a bounded interval \(I\).
    Then by Definition \ref{11.3.2} we have
    \[
        p.c. \int_I f_L \leq \underline{\int}_I f = \int_I f = \overline{\int}_I f \leq p.c. \int_I f_U.
    \]
    Let \(F_U : J \to \mathbf{R}\) be the function
    \[
        F_U(x) = \begin{cases}
            f_U(x) & \text{if } x \in I    \\
            0      & \text{if } x \notin I
        \end{cases}
    \]
    and let \(F_L : J \to \mathbf{R}\) be the function
    \[
        F_L(x) = \begin{cases}
            f_L(x) & \text{if } x \in I     \\
            0      & \text{if } x \notin I.
        \end{cases}
    \]
    We know that \(F_U\) majorizes \(F\) and \(F_L\) minorizes \(F\), and by Theorem \ref{11.2.16}(g) we have \(p.c. \int_J F_U = p.c. \int_I f_U\) and \(p.c. \int_J F_L = p.c. \int_I f_L\).
    Thus \(F\) is a bounded function on a bounded interval \(I\), and we have
    \begin{align*}
                 & \overline{\int}_J F \leq p.c. \int_J F_U     & \text{(by Definition \ref{11.3.2})}                          \\
        \implies & \overline{\int}_J F \leq p.c. \int_I f_U     & \text{(by Theorem \ref{11.2.16}(g))}                         \\
        \implies & \overline{\int}_J F \leq \overline{\int}_I f & \text{(by Definition \ref{11.3.2} and \(f_U\) is arbitrary)} \\
        \implies & \overline{\int}_J F \leq \int_I f            & \text{(by Definition \ref{11.3.4})}
    \end{align*}
    and
    \begin{align*}
                 & \underline{\int}_J F \geq p.c. \int_J F_L      & \text{(by Definition \ref{11.3.2})}                          \\
        \implies & \underline{\int}_J F \geq p.c. \int_I f_L      & \text{(by Theorem \ref{11.2.16}(g))}                         \\
        \implies & \underline{\int}_J F \geq \underline{\int}_I f & \text{(by Definition \ref{11.3.2} and \(f_L\) is arbitrary)} \\
        \implies & \underline{\int}_J F \geq \int_I f.            & \text{(by Definition \ref{11.3.4})}
    \end{align*}
    By Lemma \ref{11.3.3} we have
    \[
        \int_I f \leq \underline{\int}_J F \leq \overline{\int}_J F \leq \int_I f
    \]
    and thus by Definition \ref{11.3.4} we have
    \[
        \int_J F = \underline{\int}_J F = \overline{\int}_J F = \int_I f.
    \]
\end{proof}

\begin{proof}{(h)}
    Let \(f_U : I \to \mathbf{R}\) be a piecewise constant function on \(I\) which majorizes \(f\).
    Let \(f_L : I \to \mathbf{R}\) be a piecewise constant function on \(I\) which minorizes \(f\).
    \(f_U, f_L\) are well-defined since by Definition \ref{11.3.4} \(f\) is a bounded function on a bounded interval \(I\).
    Then by Definition \ref{11.3.2} we have
    \[
        p.c. \int_I f_L \leq \underline{\int}_I f = \int_I f = \overline{\int}_I f \leq p.c. \int_I f_U.
    \]
    By Theorem \ref{11.2.16}(h) we know that \(f_U|_J : J \to \mathbf{R}\), \(f_L|_J : J \to \mathbf{R}\) are piecewise constant function on \(J\) and \(f_U|_K : K \to \mathbf{R}\), \(f_L|_K : K \to \mathbf{R}\) are piecewise constant functions on \(K\).
    By Definition \ref{11.3.1} we know that \(f_U|_J\) majorizes \(f|_J\) and \(f_L|_J\) minorizes \(f|_J\), similarly \(f_U|_K\) majorizes \(f|_K\) and \(f_L|_K\) minorizes \(f|_K\).
    Thus \(f|_J\), \(f|_K\) are bounded functions on bounded intervals \(J, K\), respectively.
    So \(\overline{\int}_J f|_J\), \(\overline{\int}_K f|_K\), \(\underline{\int}_J f|_J\), \(\underline{\int}_K f|_K\) are well-defined.
    Then we have
    \begin{align*}
                 & \overline{\int}_J f|_J + \overline{\int}_K f|_K \leq p.c. \int_J f_U|_J + p.c. \int_K f_U|_K & \text{(by Definition \ref{11.3.2})}  \\
        \implies & \overline{\int}_J f|_J + \overline{\int}_K f|_K \leq p.c. \int_I f_U                         & \text{(by Theorem \ref{11.2.16}(h))} \\
        \implies & \overline{\int}_J f|_J + \overline{\int}_K f|_K \leq \overline{\int}_I f                     & \text{(by Definition \ref{11.3.2})}  \\
        \implies & \overline{\int}_J f|_J + \overline{\int}_K f|_K \leq \int_I f                                & \text{(by Definition \ref{11.3.4})}
    \end{align*}
    and
    \begin{align*}
                 & \underline{\int}_J f|_J + \underline{\int}_K f|_K \geq p.c. \int_J f_L|_J + p.c. \int_K f_L|_K & \text{(by Definition \ref{11.3.2})}  \\
        \implies & \underline{\int}_J f|_J + \underline{\int}_K f|_K \geq p.c. \int_I f_L                         & \text{(by Theorem \ref{11.2.16}(h))} \\
        \implies & \underline{\int}_J f|_J + \underline{\int}_K f|_K \geq \underline{\int}_I f                    & \text{(by Definition \ref{11.3.2})}  \\
        \implies & \underline{\int}_J f|_J + \underline{\int}_K f|_K \geq \int_I f.                               & \text{(by Definition \ref{11.3.4})}
    \end{align*}
    By Lemma \ref{11.3.3} we have
    \[
        \int_I f \leq \underline{\int}_J f|_J + \underline{\int}_K f|_K \leq \overline{\int}_J f|_J + \overline{\int}_K f|_K \leq \int_I f
    \]
    and thus we have
    \[
        \underline{\int}_J f|_J + \underline{\int}_K f|_K = \overline{\int}_J f|_J + \overline{\int}_J f|_K = \int_I f.
    \]
    Since
    \begin{align*}
                 & \underline{\int}_J f|_J + \underline{\int}_K f|_K = \overline{\int}_J f|_J + \overline{\int}_J f|_K                                                \\
        \implies & 0 \geq \underline{\int}_J f|_J - \overline{\int}_J f|_J = \overline{\int}_J f|_K - \underline{\int}_K f|_K \geq 0 & \text{(by Lemma \ref{11.3.3})} \\
        \implies & \underline{\int}_J f|_J - \overline{\int}_J f|_J = \overline{\int}_J f|_K - \underline{\int}_K f|_K = 0,
    \end{align*}
    by Definition \ref{11.3.4} we have
    \begin{align*}
         & \int_J f|_J = \underline{\int}_J f|_J = \overline{\int}_J f|_J, \\
         & \int_K f|_K = \underline{\int}_K f|_K = \overline{\int}_K f|_K, \\
         & \int_J f|_J + \int_K f|_K = \int_I f.
    \end{align*}
\end{proof}

\begin{remark}\label{11.4.2}
    We often abbreviate \(\int_J f|_J\) as \(\int_J f\) even though \(f\) is really defined on a larger domain than just \(J\).
    We also observe from Theorem \ref{11.4.1}(h) and Remark \ref{11.3.8} that if \(f : [a, b] \to \mathbf{R}\) is Riemann integrable on a closed interval \([a, b]\), then \(\int_{[a, b]} f = \int_{(a, b]} f = \int_{[a, b)} f = \int_{(a, b)} f\).
\end{remark}

\begin{theorem}[Max and min preserve integrability]\label{11.4.3}
    Let \(I\) be a bounded interval, and let \(f : I \to \mathbf{R}\) and \(g : I \to \mathbf{R}\) be a Riemann integrable function.
    Then the functions \(\max(f, g) : I \to \mathbf{R}\) and \(\min(f, g) : I \to \mathbf{R}\) defined by \(\max(f, g)(x) \coloneqq \max\big(f(x), g(x)\big)\) and \(\min(f, g)(x) \coloneqq \min\big(f(x), g(x)\big)\) are also Riemann integrable.
\end{theorem}

\begin{proof}
    We shall just prove the claim for \(\max(f, g)\), the case of \(\min(f, g)\) being similar.
    First note that since \(f\) and \(g\) are bounded, then \(\max(f, g)\) is also bounded.

    Let \(\varepsilon > 0\).
    Since \(\int_I f = \underline{\int}_I f\), there exists a piecewise constant function \(\underline{f} : I \to \mathbf{R}\) which minorizes \(f\) on \(I\) such that
    \[
        \int_I \underline{f} \geq \int_I f - \varepsilon.
    \]
    Similarly we can find a piecewise constant \(g : I \to \mathbf{R}\) which minorizes \(g\) on \(I\) such that
    \[
        \int_I \underline{g} \geq \int_I g - \varepsilon,
    \]
    and we can find piecewise functions \(\overline{f}, \overline{g}\) which majorize \(f, g\) respectively on \(I\) such that
    \[
        \int_I \overline{f} \leq \int_I f + \varepsilon
    \]
    and
    \[
        \int_I \overline{g} \leq \int_I g + \varepsilon.
    \]
    In particular, if \(h : I \to \mathbf{R}\) denotes the function
    \[
        h \coloneqq (\overline{f} - \underline{f}) + (\overline{g} - \underline{g})
    \]
    we have
    \[
        \int_I h \leq 4 \varepsilon.
    \]
    On the other hand, \(\max(\underline{f}, \underline{g})\) is a piecewise constant function on \(I\) which minorizes \(\max(f, g)\), while \(\max(\overline{f}, \overline{g})\) is similarly a piecewise constant function on \(I\) which majorizes \(\max(f, g)\).
    Thus
    \[
        \int_I \max(\underline{f}, \underline{g}) \leq \underline{\int}_I \max(f, g) \leq \overline{\int}_I \max(f, g) \leq \int_I \max(\overline{f}, \overline{g}),
    \]
    and so
    \[
        0 \leq \overline{\int}_I \max(f, g) - \underline{\int}_I \max(f, g) \leq \int_I \max(\overline{f}, \overline{g}) - \max(\underline{f}, \underline{g}).
    \]
    But we have
    \[
        \overline{f}(x) = \underline{f}(x) + (\overline{f} - \underline{f})(x) \leq \underline{f}(x) + h(x)
    \]
    and similarly
    \[
        \overline{g}(x) = \underline{g}(x) + (\overline{g} - \underline{g})(x) \leq \underline{g}(x) + h(x)
    \]
    and thus
    \[
        \max\big(\overline{f}(x), \overline{g}(x)\big) \leq \max\big(\underline{f}(x), \underline{g}(x)\big) + h(x).
    \]
    Inserting this into the previous inequality, we obtain
    \[
        0 \leq \overline{\int}_I \max(f, g) - \underline{\int}_I \max(f, g) \leq \int_I h \leq 4 \varepsilon.
    \]
    To summarize, we have shown that
    \[
        0 \leq \overline{\int}_I \max(f, g) - \underline{\int}_I \max(f, g) \leq 4 \varepsilon
    \]
    for every \(\varepsilon\).
    Since \(\overline{\int}_I \max(f, g) - \underline{\int}_I \max(f, g)\) does not depend on \(\varepsilon\), we thus see that
    \[
        \overline{\int}_I \max(f, g) - \underline{\int}_I \max(f, g) = 0
    \]
    and hence that \(\max(f, g)\) is Riemann integrable.
\end{proof}

\begin{corollary}[Absolute values preserve Riemann integrability]\label{11.4.4}
    \quad
    Let \(I\) be a bounded interval.
    If \(f : I \to \mathbf{R}\) is a Riemann integrable function, then the positive part \(f_+ \coloneqq \max(f, 0)\) and the negative part \(f_- \coloneqq \min(f, 0)\) are also Riemann integrable on \(I\).
    Also, the absolute value \(\abs*{f}\), defined by \(\abs*{f}(x) = \abs*{f(x)}\) is also Riemann integrable on \(I\).
    (observe that \(\abs*{f} = f_+ - f_-\))
\end{corollary}

\begin{proof}
    By Theorem \ref{11.4.3} we know that \(f_+, f_-\) are Riemann integrable.
    Since \(\abs*{f} = f_+ - f_-\), by Theorem \ref{11.4.1}(a) we know that \(\abs*{f}\) is Riemann integrable.
\end{proof}

\begin{theorem}[Products preserve Riemann integrability]\label{11.4.5}
    Let \(I\) be a bounded interval.
    If \(f : I \to \mathbf{R}\) and \(g : I \to \mathbf{R}\) are Riemann integrable, then \(fg : I \to \mathbf{R}\) is also Riemann integrable.
\end{theorem}

\begin{proof}
    We split \(f = f_+ + f_-\) and \(g = g_+ + g_-\) into positive and negative parts;
    by Corollary \ref{11.4.4}, the functions \(f_+\), \(f_-\), \(g_+\), \(g_-\) are Riemann integrable.
    Since
    \[
        fg = f_+ g_+ + f_+ g_- + f_- g_+ + f_- g_-
    \]
    then it suffices to show that the functions \(f_+ g_+\), \(f_+ g_-\), \(f_- g_+\), \(f_- g_-\) are individually Riemann integrable.
    We will just show this for \(f_+ g_+\);
    the other three are similar.

    Since \(f_+\) and \(g_+\) are bounded and positive, there are \(M_1, M_2 > 0\) such that
    \[
        0 \leq f_+(x) \leq M_1 \text{ and } 0 \leq g_+(x) \leq M_2
    \]
    for all \(x \in I\).
    Now let \(\varepsilon > 0\) be arbitrary.
    Then, as in the proof of Theorem \ref{11.4.3}, we can find a piecewise constant function \(\underline{f_+}\) minorizing \(f_+\) on \(I\), and a piecewise constant function \(\overline{f_+}\) majorizing \(f_+\) on \(I\), such that
    \[
        \int_I \overline{f_+} \leq \int_I f_+ + \varepsilon
    \]
    and
    \[
        \int_I \underline{f_+} \geq \int_I f_+ - \varepsilon.
    \]
    Note that \(\underline{f_+}\) may be negative at places, but we can fix this by replacing \(\underline{f_+}\) by \(\max(\underline{f_+}, 0)\), since this still minorizes \(f_+\) and still has integral greater than or equal to \(\int_I f_+ - \varepsilon\).
    So without loss of generality we may assume that \(\underline{f_+}(x) \geq 0\) for all \(x \in I\).
    Similarly we may assume that \(\overline{f_+}(x) \leq M_1\) for all \(x \in I\);
    thus
    \[
        0 \leq \underline{f_+}(x) \leq f_+(x) \leq \overline{f_+}(x) \leq M_1
    \]
    for all \(x \in I\).

    Similar reasoning allows us to find piecewise constant \(\underline{g_+}\) minorizing \(g_+\), and \(\overline{g_+}\) majorizing \(g_+\), such that
    \[
        \int_I \overline{g_+} \leq \int_I g_+ + \varepsilon
    \]
    and
    \[
        \int_I \underline{g_+} \geq \int_I g_+ - \varepsilon,
    \]
    and
    \[
        0 \leq \underline{g_+}(x) \leq g_+(x) \leq \overline{g_+}(x) \leq M_2
    \]
    for all \(x \in I\).

    Notice that \(\underline{f_+} \underline{g_+}\) is piecewise constant and minorizes \(f_+ g_+\), while \(\overline{f_+} \overline{g_+}\) is piecewise constant and majorizes \(f_+ g_+\).
    Thus
    \[
        0 \leq \overline{\int}_I f_+ g_+ - \underline{\int}_I f_+ g_+ \leq \int_I \overline{f}_+ \overline{g_+} - \underline{f_+} \underline{g_+}.
    \]
    However, we have
    \begin{align*}
        \overline{f_+}(x) \overline{g_+}(x) - \underline{f_+}(x) \underline{g_+}(x) & = \overline{f_+}(x) (\overline{g_+} - \underline{g_+})(x) + \underline{g_+}(x) (\overline{f_+} - \underline{f_+})(x) \\
                                                                                    & \leq M_1 (\overline{g_+} - \underline{g_+})(x) + M_2 (\overline{f_+} - \underline{f_+})(x)
    \end{align*}
    for all \(x \in I\), and thus
    \begin{align*}
        0 \leq \overline{\int}_I f_+ g_+ - \underline{\int}_I f_+ g_+ & \leq M_1 \int_I (\overline{g_+} - \underline{g_+}) + M_2 \int_I (\overline{f_+} - \underline{f_+}) \\
                                                                      & \leq M_1 (2\varepsilon) + M_2 (2\varepsilon).
    \end{align*}
    Again, since \(\varepsilon\) was arbitrary, we can conclude that \(f_+ g_+\) is Riemann integrable, as before.
    Similar argument show that \(f_+ g_-\), \(f_- g_+\), \(f_- g_-\) are Riemann integrable;
    combining them we obtain that \(fg\) is Riemann integrable.
\end{proof}

\exercisesection

\begin{exercise}\label{ex 11.4.1}
    Prove Theorem \ref{11.4.1}.
\end{exercise}

\begin{proof}
    See Theorem \ref{11.4.1}.
\end{proof}

\begin{exercise}\label{ex 11.4.2}
    Let \(a < b\) be real numbers, and let \(f : [a, b] \to \mathbf{R}\) be a continuous, non-negative function
    (so \(f(x) \geq 0\) for all \(x \in [a, b]\)).
    Suppose that \(\int_{[a, b]} f = 0\).
    Show that \(f(x) = 0\) for all \(x \in [a, b]\).
\end{exercise}

\begin{proof}
    Suppose for sake of contradiction that \(\exists\ x_0 \in [a, b]\) such that \(f(x_0) > 0\).
    Since \(f\) is continuous, by Proposition \ref{9.4.7} we have
    \[
        \forall \varepsilon \in \mathbf{R}^+, \exists\ \delta \in \mathbf{R}^+ : \big(\forall x \in [a, b], \abs*{x - x_0} < \delta \implies \abs*{f(x) - f(x_0)} \leq \varepsilon\big),
    \]
    or equivalently
    \[
        \forall \varepsilon \in \mathbf{R}^+, \exists\ \delta \in \mathbf{R}^+ : \big(\forall x \in [a, b] \cap (x_0 - \delta, x_0 + \delta) \implies \abs*{f(x) - f(x_0)} \leq \varepsilon\big).
    \]
    In particular, we have
    \[
        \exists\ \delta \in \mathbf{R}^+ : \bigg(\forall x \in [a, b] \cap (x_0 - \delta, x_0 + \delta) \implies \abs*{f(x) - f(x_0)} \leq \frac{f(x_0)}{2}\bigg),
    \]
    or equivalently
    \[
        \exists\ \delta \in \mathbf{R}^+ : \bigg(\forall x \in [a, b] \cap (x_0 - \delta, x_0 + \delta) \implies \frac{f(x_0)}{2} \leq f(x) \leq \frac{3 f(x_0)}{2}\bigg).
    \]
    Since \(\delta \neq 0\), we know that \([a, b] \cap (x_0 - \delta, x_0 + \delta) \neq \emptyset\).
    Since \(a \neq b\), we know that
    \[
        \sup\big([a, b] \cap (x_0 - \delta, x_0 + \delta)\big) \neq \inf\big([a, b] \cap (x_0 - \delta, x_0 + \delta)\big).
    \]
    Thus by Definition \ref{11.1.8} we have \(\abs*{[a, b] \cap (x_0 - \delta, x_0 + \delta)} > 0\).
    By Corollary \ref{11.1.6} we know that \([a, b] \cap (x_0 - \delta, x_0 + \delta)\) is a bounded interval.
    Let \(f_L : [a, b] \to \mathbf{R}\) be the function
    \[
        f_L(x) = \begin{cases}
            \frac{f(x_0)}{2} & \text{if } x \in [a, b] \cap (x_0 - \delta, x_0 + \delta)    \\
            0                & \text{if } x \notin [a, b] \cap (x_0 - \delta, x_0 + \delta)
        \end{cases}
    \]
    Since \(f(x) \geq 0\) for all \(x \in [a, b]\), we know that \(f_L\) minorizes \(f\).
    By Theorem \ref{11.2.16}(g) we know that \(f_L\) is a piecewise constant function.
    By Lemma \ref{11.3.7} we have
    \[
        \int_{[a, b]} f_L = p.c. \int_{[a, b]} f_L = \frac{f(x_0)}{2}\abs*{[a, b] \cap (x_0 - \delta, x_0 + \delta)} > 0.
    \]
    But by Definition \ref{11.3.2} and Definition \ref{11.3.4} we have
    \[
        0 < \int_{[a, b]} f_L \leq \underline{\int}_{[a, b]} f = \int_{[a, b]} f = 0,
    \]
    a contradiction.
    Thus we must have \(f(x) = 0\) for all \(x \in [a, b]\).
\end{proof}

\begin{exercise}\label{ex 11.4.3}
    Let \(I\) be a bounded interval, let \(f : I \to \mathbf{R}\) be a Riemann integrable function, and let \(\mathbf{P}\) be a partition of \(I\).
    Show that
    \[
        \int_I f = \sum_{J \in \mathbf{P}} \int_J f|_J.
    \]
\end{exercise}

\begin{proof}
    Let \(P(n)\) be the statement ``\(\#(\mathbf{P}) = n\) and \(\int_I f = \sum_{J \in \mathbf{P}} \int_J f|_J\)''.
    We use induction on \(n\) to show that \(P(n)\) is true \(\forall n \in \mathbf{N}\).
    For \(n = 0\), we have \(\mathbf{P} = \emptyset\) and \(I = \emptyset\).
    Thus
    \begin{align*}
        p.c. \int_{[\emptyset]} f & = \sum_{J \in \emptyset} c_J \abs*{J} & \text{(by Definition \ref{11.2.9})}     \\
                                  & = 0                                   & \text{(by Proposition \ref{7.1.11}(a))} \\
                                  & = p.c. \int_{\emptyset} f             & \text{(by Definition \ref{11.2.14})}    \\
                                  & = \int_{\emptyset} f                  & \text{(by Lemma \ref{11.3.7})}          \\
                                  & = \sum_{J \in \emptyset} \int_J f|_J  & \text{(by Proposition \ref{7.1.11}(a))}
    \end{align*}
    and the base case holds.
    Suppose inductively that \(P(n)\) is true for some \(n \geq 0\).
    Then we need to show that \(P(n + 1)\) is true.
    Let \(K \in \mathbf{P}\) such that \(x < y\) for every \(x \in K\) and \(y \in I \setminus K\).
    Then \(\big\{K, \bigcup (\mathbf{P} \setminus \{K\})\big\}\) is a partition of \(I\), and
    \begin{align*}
        \int_I f & = \int_K f|_K + \int_{\bigcup (\mathbf{P} \setminus \{K\})} f|_{\bigcup (\mathbf{P} \setminus \{K\})} & \text{(by Theorem \ref{11.4.1}(h))}     \\
                 & = \int_K f|_K + \sum_{J \in \mathbf{P} \setminus \{K\}} \int_J f|_J                                   & \text{(by induction hypothesis)}        \\
                 & = \sum_{J \in \mathbf{P}} \int_J f|_J.                                                                & \text{(by Proposition \ref{7.1.11}(e))}
    \end{align*}
    This closes the induction.
\end{proof}

\begin{exercise}\label{ex 11.4.4}
    Without repeating all the computations in the above proofs, give a short explanation as to why the remaining cases of Theorem \ref{11.4.3} and Theorem \ref{11.4.5} follow automatically from the cases presented in the text.
\end{exercise}

\begin{proof}
    We first show that the remaining case of Theorem \ref{11.4.3} is true.
    By Theorem \ref{11.4.1}(b) \(-f\) and \(-g\) are Riemann integrable on \(I\).
    Since \(\max(-f, -g)\) is Riemann integrable and \(\min(f, g) = -\max(-f, -g)\), by Theorem \ref{11.4.1}(b) we know that \(\min(f, g)\) is Riemann integrable.

    Now we show that the remaining cases of Theorem \ref{11.4.5} are true.
    By Corollary \ref{11.4.4} \((-f)_+\) and \((-g)_+\) are Riemann integrable on \(I\).
    Since for any Riemann integrable functions \(p\) and \(q\), \(p_+ q_+\) are Riemann integrable (which is showed in the proof of Theorem \ref{11.4.5}), we have
    \begin{align*}
        f_+ g_- & = f_+ \cdot \big(\min(g, 0)\big)                      & \text{(by Corollary \ref{11.4.4})} \\
                & = f_+ \cdot \big(-\max(-g, 0)\big)                                                         \\
                & = f_+ \cdot \big(-(-g)_+\big)                         & \text{(by Corollary \ref{11.4.4})} \\
                & = -\big(f_+ \cdot (-g)_+\big)                         & \text{(by Definition \ref{9.2.1})} \\
        f_- g_+ & = \big(\min(f, 0)\big) \cdot g_+                      & \text{(by Corollary \ref{11.4.4})} \\
                & = \big(-\max(-f, 0)\big) \cdot g_+                                                         \\
                & = \big(-(-f)_+\big) \cdot g_+                         & \text{(by Corollary \ref{11.4.4})} \\
                & = -\big((-f)_+ \cdot g_+\big)                         & \text{(by Definition \ref{9.2.1})} \\
        f_- g_- & = \big(\min(f, 0)\big) \cdot \big(\min(g, 0)\big)     & \text{(by Corollary \ref{11.4.4})} \\
                & = \big(-\max(-f, 0)\big) \cdot \big(-\max(-g, 0)\big)                                      \\
                & = \big(-(-f)_+\big) \big(-(-g)_+\big)                 & \text{(by Corollary \ref{11.4.4})} \\
                & = (-f)_+ \cdot (-g)_+                                 & \text{(by Definition \ref{9.2.1})}
    \end{align*}
    and thus \(f_+ g_-\), \(f_- g_+\), \(f_- g_-\) are Riemann integrable.
\end{proof}
\section{Riemann integrability of continuous functions}\label{sec 11.5}

\begin{theorem}\label{11.5.1}
  Let \(I\) be a bounded interval, and let \(f\) be a function which is uniformly continuous on \(I\).
  Then \(f\) is Riemann integrable.
\end{theorem}

\begin{proof}
  From \cref{9.9.15} we see that \(f\) is bounded.
  Now we have to show that \(\underline{\int}_I f = \overline{\int}_I f\).

  If \(I\) is a point or the empty set then the theorem is trivial, so let us assume that \(I\) is one of the four intervals \([a, b]\), \((a, b)\), \((a, b]\), or \([a, b)\) for some real numbers \(a < b\).

      Let \(\varepsilon > 0\) be arbitrary.
      By uniform continuity, there exists a \(\delta > 0\) such that \(\abs{f(x) - f(y)} < \varepsilon\) whenever \(x, y \in I\) are such that \(\abs{x - y} < \delta\).
      By the Archimedean principle, there exists an integer \(N > 0\) such that \((b - a) / N < \delta\).

      Note that we can partition \(I\) into \(N\) intervals \(J_1, \dots, J_N\), each of length \((b - a) / N\).
      (How? One has to treat each of the cases \([a, b]\), \((a, b)\), \((a, b]\), \([a, b)\) slightly differently.)
  By \cref{11.3.12}, we thus have
  \[
    \overline{\int}_I f \leq \sum_{k = 1}^N \big(\sup_{x \in J_k} f(x)\big) \abs{J_k}
  \]
  and
  \[
    \underline{\int}_I f \geq \sum_{k = 1}^N \big(\inf_{x \in J_k} f(x)\big) \abs{J_k}
  \]
  so in particular
  \[
    \overline{\int}_I f - \underline{\int}_I f \leq \sum_{k = 1}^N \big(\sup_{x \in J_k} f(x) - \inf_{x \in J_k} f(x)\big) \abs{J_k}.
  \]
  However, we have \(\abs{f(x) - f(y)} < \varepsilon\) for all \(x, y \in J_k\), since \(\abs{J_k} = (b - a) / N < \delta\).
  In particular we have
  \[
    f(x) < f(y) + \varepsilon \text{ for all } x, y \in J_k.
  \]
  Taking suprema in \(x\), we obtain
  \[
    \sup_{x \in J_k} f(x) \leq f(y) + \varepsilon \text{ for all } y \in J_k,
  \]
  and then taking infima in \(y\) we obtain
  \[
    \sup_{x \in J_k} f(x) \leq \inf_{y \in J_k} f(y) + \varepsilon.
  \]
  Inserting this bound into our previous inequality, we obtain
  \[
    \overline{\int}_I f - \underline{\int}_I f \leq \sum_{k = 1}^N \varepsilon \abs{J_k},
  \]
  but by \cref{11.1.13} we thus have
  \[
    \overline{\int}_I f - \underline{\int}_I f \leq \varepsilon (b - a).
  \]
  But \(\varepsilon > 0\) was arbitrary, while \((b - a)\) is fixed.
  Thus \(\overline{\int}_I f - \underline{\int}_I f\) cannot be positive.
  By \cref{11.3.3} and the definition of Riemann integrability we thus have that \(f\) is Riemann integrable.
\end{proof}

\begin{corollary}\label{11.5.2}
  Let \([a, b]\) be a closed interval, and let \(f : [a, b] \to \R\) be continuous.
  Then \(f\) is Riemann integrable.
\end{corollary}

\begin{proof}
  Combining \cref{11.5.1} with \cref{9.9.16} we are done.
\end{proof}

\begin{note}
  Note that \cref{11.5.2} is not true if \([a, b]\) is replaced by any other sort of interval, since it is not even guaranteed then that continuous functions are bounded.
  For instance, the function \(f : (0, 1) \to \R\) defined by \(f(x) \coloneqq 1 / x\) is continuous but not Riemann integrable.
  However, if we assume that a function is both continuous \emph{and} bounded, we can recover Riemann integrability (see \cref{11.5.3}).
\end{note}

\begin{proposition}\label{11.5.3}
  Let \(I\) be a bounded interval, and let \(f : I \to \R\) be both continuous and bounded.
  Then \(f\) is Riemann integrable on \(I\).
\end{proposition}

\begin{proof}
  If \(I\) is a point or an empty set then the claim is trivial;
  if \(I\) is a closed interval the claim follows from \cref{11.5.2}.
  So let us assume that \(I\) is of the form \((a, b]\), \((a, b)\), or \([a, b)\) for some \(a < b\).

  We have a bound \(M\) for \(f\), so that \(-M \leq f(x) \leq M\) for all \(x \in I\).
  Now let \(0 < \varepsilon < (b - a) / 2\) be a small number.
  The function \(f\) when restricted to the interval \([a + \varepsilon, b - \varepsilon]\) is continuous, and hence Riemann integrable by \cref{11.5.2}.
  In particular, we can find a piecewise constant function \(h : [a + \varepsilon, b - \varepsilon] \to \R\) which majorizes \(f\) on \([a + \varepsilon, b - \varepsilon]\) such that
  \[
    \int_{[a + \varepsilon, b - \varepsilon]} h \leq \int_{[a + \varepsilon, b - \varepsilon]} f + \varepsilon.
  \]
  Define \(\tilde{h} : I \to \R\) by
  \[
    \tilde{h}(x) \coloneqq \begin{cases}
      h(x) & \text{if } x \in [a + \varepsilon, b - \varepsilon]             \\
      M    & \text{if } x \in I \setminus [a + \varepsilon, b - \varepsilon]
    \end{cases}
  \]
  Clearly \(\tilde{h}\) is piecewise constant on \(I\) and majorizes \(f\);
  by \cref{11.2.16} we have
  \[
    \int_I \tilde{h} = \varepsilon M + \int_{[a + \varepsilon, b - \varepsilon]} h + \varepsilon M \leq \int_{[a + \varepsilon, b - \varepsilon]} f + (2M + 1) \varepsilon.
  \]
  In particular we have
  \[
    \overline{\int}_I f \leq \int_{[a + \varepsilon, b - \varepsilon]} f + (2M + 1) \varepsilon.
  \]
  This is true since \(\tilde{h}\) majorize \(f\).
  A similar argument gives
  \[
    \underline{\int}_I f \geq \int_{[a + \varepsilon, b - \varepsilon]} f - (2M + 1) \varepsilon.
  \]
  and hence
  \[
    \overline{\int}_I f - \underline{\int}_I f \leq (4M + 2) \varepsilon.
  \]
  But \(\varepsilon\) is arbitrary, and so we can argue as in the proof of \cref{11.5.1} to conclude Riemann integrability.
\end{proof}

\begin{note}
  From \cref{11.5.1}, \cref{11.5.2} and \cref{11.5.3} we see that if we can show a function \(f\) being \emph{uniformly continuous} (not just continuous) on some bounded interval \(I\), then \(f\) is Riemann integrable on \(I\).
\end{note}

\begin{definition}\label{11.5.4}
  Let \(I\) be a bounded interval, and let \(f : I \to \R\).
  We say that \(f\) is \emph{piecewise continuous on \(I\)} iff there exists a partition \(\mathbf{P}\) of \(I\) such that \(f|_J\) is continuous on \(J\) for all \(J \in \mathbf{P}\).
\end{definition}

\setcounter{theorem}{5}
\begin{proposition}\label{11.5.6}
  Let \(I\) be a bounded interval, and let \(f : I \to \R\) be both piecewise continuous and bounded.
  Then \(f\) is Riemann integrable.
\end{proposition}

\begin{proof}
  Since \(f\) is piecewise continuous on \(I\), by \cref{11.5.4} \(\exists\ \mathbf{P}\) such that \(\mathbf{P}\) is a partition of \(I\) and \(f|_J\) is continuous on \(J\) for all \(J \in \mathbf{P}\).
  Since \(f\) is bounded, we know that \(f|_J\) is bounded for all \(J \in \mathbf{P}\).
  Thus by \cref{11.5.3} \(f|_J\) is Riemann integrable on \(J\) for all \(J \in \mathbf{P}\).
  For each \(J \in \mathbf{P}\), we define \(F_J : I \to \R\) to be the function
  \[
    F_J(x) = \begin{cases}
      f|_J(x) & \text{if } x \in J    \\
      0       & \text{if } x \notin J
    \end{cases}
  \]
  Then by \cref{11.4.1}(g) \(F|_J\) is Riemann integrable for all \(J \in \mathbf{P}\) and
  \begin{align*}
    \sum_{J \in \mathbf{P}} \int_I F_J & = \sum_{J \in \mathbf{P}} \int_J f|_J & \text{(by \cref{11.4.1}(g))} \\
                                       & = \int_I f.                           & \text{(by \cref{ex 11.4.3})}
  \end{align*}
  Thus \(f\) is Riemann integrable on \(I\).
\end{proof}

\exercisesection

\begin{exercise}\label{ex 11.5.1}
  Prove \cref{11.5.6}.
\end{exercise}

\begin{proof}
  See \cref{11.5.6}.
\end{proof}
\section{Riemann integrability of monotone functions}\label{sec 11.6}

\begin{proposition}\label{11.6.1}
    Let \([a, b]\) be a closed and bounded interval and let \(f : [a, b] \to \mathbf{R}\) be a monotone function.
    Then \(f\) is Riemann integrable on \([a, b]\).
\end{proposition}

\begin{proof}
    Without loss of generality we may take \(f\) to be monotone increasing (instead of monotone decreasing).
    From Exercise \ref{ex 9.8.1} we know that \(f\) is bounded.
    Now let \(N > 0\) be an integer, and partition \([a, b]\) into \(N\) half-open intervals \(\{[a + \frac{b - a}{N} j, a + \frac{b - a}{N} (j + 1)) : 0 \leq j \leq N - 1\}\) of length \((b - a) / N\), together with the point \(\{b\}\).
    Then by Proposition \ref{11.3.12} we have
    \[
        \overline{\int}_I f \leq \sum_{j = 0}^{N - 1} \Bigg(\sup_{x \in [a + \frac{b - a}{N} j, a + \frac{b - a}{N} (j + 1))} f(x)\Bigg) \frac{b - a}{N},
    \]
    (the point \(\{b\}\) clearly giving only a zero contribution).
    Since \(f\) is monotone increasing, we thus have
    \[
        \overline{\int}_I f \leq \sum_{j = 0}^{N - 1} f\bigg(a + \frac{b - a}{N} (j + 1)\bigg) \frac{b - a}{N}.
    \]
    Similarly we have
    \[
        \underline{\int}_I f \geq \sum_{j = 0}^{N - 1} f\bigg(a + \frac{b - a}{N} j\bigg) \frac{b - a}{N}.
    \]
    Thus we have
    \[
        \overline{\int}_I f - \underline{\int}_I f \leq \sum_{j = 0}^{N - 1} \Bigg(f\bigg(a + \frac{b - a}{N} (j + 1)\bigg) - f\bigg(a + \frac{b - a}{N} j\bigg)\Bigg) \frac{b - a}{N}.
    \]
    Using telescoping series (Lemma \ref{7.2.15}) we thus have
    \begin{align*}
        \overline{\int}_I f - \underline{\int}_I f & \leq \Bigg(f\bigg(a + \frac{b - a}{N} N\bigg) - f\bigg(a + \frac{b - a}{N} 0\bigg)\Bigg) \frac{b - a}{N} \\
                                                   & = (f(b) - f(a)) \frac{b - a}{N}.
    \end{align*}
    But \(N\) was arbitrary, so we can conclude as in the proof of Theorem \ref{11.5.1} that \(f\) is Riemann integrable.
\end{proof}

\begin{remark}\label{11.6.2}
    From Exercise \ref{ex 9.8.5} we know that there exist monotone functions which are not piecewise continuous, so Proposition \ref{11.6.1} is not subsumed by Proposition \ref{11.5.6}.
\end{remark}

\begin{corollary}\label{11.6.3}
    Let \(I\) be a bounded interval, and let \(f : I \to \mathbf{R}\) be both monotone and bounded.
    Then \(f\) is Riemann integrable on \(I\).
\end{corollary}

\begin{proof}
    If \(I\) is a point or an empty set then the claim is trivial;
    if \(I\) is a closed interval the claim follows from Proposition \ref{11.6.1}.
    So let us assume that \(I\) is of the form \((a, b]\), \((a, b)\), or \([a, b)\) for some \(a < b\).

    We have a bound \(M\) for \(f\), so that \(-M \leq f(x) \leq M\) for all \(x \in I\).
    Now let \(0 < \varepsilon < (b - a) / 2\) be a small number.
    The function \(f\) when restricted to the interval \([a + \varepsilon, b - \varepsilon]\) is monotone, and hence Riemann integrable by Proposition \ref{11.6.1}.
    In particular, we can find a piecewise constant function \(h : [a + \varepsilon, b - \varepsilon] \to \mathbf{R}\) which majorizes \(f\) on \([a + \varepsilon, b - \varepsilon]\) such that
    \[
        \int_{[a + \varepsilon, b - \varepsilon]} h \leq \int_{[a + \varepsilon, b - \varepsilon]} f + \varepsilon.
    \]
    Define \(\tilde{h} : I \to \mathbf{R}\) by
    \[
        \tilde{h}(x) \coloneqq \begin{cases}
            h(x) & \text{if } x \in [a + \varepsilon, b - \varepsilon]             \\
            M    & \text{if } x \in I \setminus [a + \varepsilon, b - \varepsilon]
        \end{cases}
    \]
    Clearly \(\tilde{h}\) is piecewise constant on \(I\) and majorizes \(f\);
    by Theorem \ref{11.2.16} we have
    \[
        \int_I \tilde{h} = \varepsilon M + \int_{[a + \varepsilon, b - \varepsilon]} h + \varepsilon M \leq \int_{[a + \varepsilon, b - \varepsilon]} f + (2M + 1) \varepsilon.
    \]
    In particular we have
    \[
        \overline{\int}_I f \leq \int_{[a + \varepsilon, b - \varepsilon]} f + (2M + 1) \varepsilon
    \]
    (since \(\overline{\int}_{I \cap [a, a + \varepsilon]} f \leq M \varepsilon\) and \(\overline{\int}_{I \cap [b, b - \varepsilon]} f \leq M \varepsilon\)).
    A similar argument gives
    \[
        \underline{\int}_I f \geq \int_{[a + \varepsilon, b - \varepsilon]} f - (2M + 1) \varepsilon.
    \]
    and hence
    \[
        \overline{\int}_I f - \underline{\int}_I f \leq (4M + 2) \varepsilon.
    \]
    But \(\varepsilon\) is arbitrary, and so we can argue as in the proof of Theorem \ref{11.5.1} to conclude Riemann integrability.
\end{proof}

\begin{proposition}[Integral test]\label{11.6.4}
    Let \(f : [0, \infty) \to \mathbf{R}\) be a monotone decreasing function which is non-negative
    (i.e., \(f(x) \geq 0\) for all \(x \geq 0\)).
    Then the sum \(\sum_{n = 0}^\infty f(n)\) is convergent if and only if \(\sup_{N > 0} \int_{[0, N]} f\) is finite.
\end{proposition}

\begin{proof}
    Let \(N \in \mathbf{N}\) and \(N > 0\).
    Since \(f\) is monotone decreasing, by Proposition \ref{11.6.1} we know that \(f\) is Riemann integrable on both \([0, N]\) and every interval \([a, b] \subseteq [0, N]\).
    Then we have
    \begin{align*}
        \int_{[0, N]} f & = \sum_{n = 0}^{N - 1} \int_{[n, n + 1)} f|_{[n, n + 1)} + \int_{[N, N]} f|_{[N, N]} & \text{(by Exercise \ref{ex 11.4.3})}      \\
                        & = \sum_{n = 0}^{N - 1} \int_{[n, n + 1)} f|_{[n, n + 1)}                             & \text{(by Definition \ref{11.1.8})}       \\
                        & \leq \sum_{n = 0}^{N - 1} \int_{[n, n + 1)} f(n)                                     & \text{(by Theorem \ref{11.4.1}(e))}       \\
                        & = \sum_{n = 0}^{N - 1} f(n) \abs*{n + 1 - n}                                         & \text{(by Definition \ref{11.2.9})}       \\
                        & = \sum_{n = 0}^{N - 1} f(n)                                                                                                      \\
                        & \leq \sum_{n = 0}^N f(n)                                                             & (\forall\ x \in [0, \infty), f(x) \geq 0)
    \end{align*}
    and
    \begin{align*}
        \int_{[0, N]} f & = \sum_{n = 0}^{N - 1} \int_{[n, n + 1)} f|_{[n, n + 1)} + \int_{[N, N]} f|_{[N, N]} & \text{(by Exercise \ref{ex 11.4.3})} \\
                        & = \sum_{n = 0}^{N - 1} \int_{[n, n + 1)} f|_{[n, n + 1)}                             & \text{(by Definition \ref{11.1.8})}  \\
                        & \geq \sum_{n = 0}^{N - 1} \int_{[n, n + 1)} f(n + 1)                                 & \text{(by Theorem \ref{11.4.1}(e))}  \\
                        & = \sum_{n = 0}^{N - 1} f(n + 1) \abs*{n + 1 - n}                                     & \text{(by Definition \ref{11.2.9})}  \\
                        & = \sum_{n = 0}^{N - 1} f(n + 1)                                                                                             \\
                        & = \sum_{n = 1}^N f(n).                                                               & \text{(by Lemma \ref{7.1.4}(b))}
    \end{align*}

    Suppose that \(\sum_{n = 0}^\infty f(n)\) is convergent.
    Then by Definition \ref{7.2.2} we know that
    \[
        \sum_{n = 0}^\infty f(n) = \lim_{m \to \infty} \sum_{n = 0}^m f(n)
    \]
    and by Proposition \ref{6.1.12} \((\sum_{n = 0}^m f(n))_{m = 0}^\infty\) is a Cauchy sequence.
    By Lemma \ref{5.1.15} we know that \((\sum_{n = 0}^m f(n))_{m = 0}^\infty\) is bounded by some \(M \in \mathbf{R}\).
    By Comparison principle (Lemma \ref{6.4.13}) we have
    \[
        \sup \bigg(\int_{[0, N] f}\bigg)_{N = 1}^\infty \leq \sup \bigg(\sum_{n = 0}^N f(n)\bigg)_{N = 1}^\infty \leq M
    \]
    and thus \(\sup_{N > 0} \int_{[0, N]} f\) is finite.

    Now suppose that \(\sup_{N > 0} \int_{[0, N]} f\) is finite.
    By Comparison principle (Lemma \ref{6.4.13}) we have
    \[
        \sup \bigg(\sum_{n = 1}^N f(n)\bigg)_{N = 1}^\infty \leq \sup \bigg(\int_{[0, N] f}\bigg)_{N = 1}^\infty
    \]
    By Proposition \ref{7.3.1} we thus have \(\sum_{n = 0}^\infty f(n)\) is convergent.
\end{proof}
\section{A non-Riemann integrable function}\label{sec 11.7}
\section{The Riemann-Stieltjes integral}\label{sec 11.8}

\begin{definition}[\(\alpha\)-length]\label{11.8.1}
    Let \(I\) be a bounded interval, and let \(\alpha : X \to \mathbf{R}\) be a monotone increasing function defined on some interval \(X\) which contains \(I\).
    Then we define the \emph{\(\alpha\)-length} \(\alpha[I]\) of \(I\) as follows.
    If \(I\) is the empty set, we set \(\alpha[I] = 0\).
    If \(I\) is a point of the form \(\{a\}\) a point for some real number \(a\), we set
    \[
        \alpha[\{\alpha\}] \coloneqq \lim_{x \to a^+ ; x \in X} \alpha(x) - \lim_{x \to a^- ; x \in X} \alpha(x),
    \]
    with the convention that \(\lim_{x \to a^+ ; x \in X} \alpha(x)\) (resp. \(\lim_{x \to a^- ; x \in X} \alpha(x)\)) is \(\alpha(a)\) when \(a\) is the right (resp. left) endpoint of \(X\).
    If \(I\) is an interval of the form \((a, b)\) for some real numbers \(b > a\), set
    \[
        \alpha[(a, b)] \coloneqq \lim_{x \to b^- ; x \in X} \alpha(x) - \lim_{x \to a^+ ; x \in X} \alpha(x).
    \]
    If \(I\) is an interval of the form \([a, b)\), \((a, b]\), or \([a, b]\) for some real numbers \(b > a\), then we set
    \[
        \alpha[I] = \begin{cases}
            \alpha(\{a\}) + \alpha((a, b))                 & \text{if } I = [a, b) \\
            \alpha((a, b)) + \alpha(\{b\})                 & \text{if } I = (a, b] \\
            \alpha(\{a\}) + \alpha((a, b)) + \alpha(\{b\}) & \text{if } I = [a, b]
        \end{cases}
    \]
\end{definition}

\begin{note}
    In the special case when \(\alpha\) is continuous, the definition of \(\alpha[I]\) where \(I\) is of the form \((a, b)\), \([a, b)\), \((a, b]\), or \([a, b]\) simplifies to \(\alpha[I] = \alpha(b) - \alpha(a)\).
\end{note}

\begin{note}
    We sometimes write \(\alpha\big|_a^b\) or \(\alpha(x)\big|_{x = a}^{x = b}\) instead of \(\alpha[[a, b]]\).
\end{note}

\setcounter{theorem}{3}
\begin{lemma}\label{11.8.4}
    Let \(I\) be a bounded interval, let \(\alpha : X \to \mathbf{R}\) be a monotone increasing function defined on some interval \(X\) which contains \(I\), and let \(\mathbf{P}\) be a partition of \(I\).
    Then we have
    \[
        \alpha[I] = \sum_{J \in \mathbf{P}} \alpha[J].
    \]
\end{lemma}

\begin{proof}
    We prove this by induction on \(n\).
    More precisely, we let \(P(n)\) be the property that whenever \(I\) is a bounded interval, and whenever \(\mathbf{P}\) is a partition of \(I\) with cardinality \(n\), that \(\alpha[I] = \sum_{J \in \mathbf{P}} \alpha[J]\).

    The base case \(P(0)\) is trivial;
    the only way that \(I\) can be partitioned into an empty partition is if \(I\) is itself empty, so by Definition \ref{11.8.1} \(\alpha[I] = 0\).
    The case \(P(1)\) is also very easy;
    the only way that \(I\) can be partitioned into a singleton set \(\{J\}\) is if \(J = I\), at which point the claim is again very easy.

    Now suppose inductively that \(P(n)\) is true for some \(n \geq 1\), and now we prove \(P(n + 1)\).
    Let \(I\) be a bounded interval, and let \(\mathbf{P}\) be a partition of \(I\) of cardinality \(n + 1\).

    If \(I\) is the empty set or a point, then all the intervals in \(\mathbf{P}\) must also be either the empty set or a point, and by Definition \ref{11.8.1} every interval either has \(\alpha\)-length zero or
    \[
        \alpha[\{a\}] = \lim_{x \to a^+ ; x \in X} \alpha(x) - \lim_{x \to a^- ; x \in X} \alpha(x),
    \]
    and the claim is trivial.
    Thus we will assume that \(I\) is an interval of the form \((a, b)\), \((a, b]\), \([a, b)\), or \([a, b]\).

            Let us first suppose that \(b \in I\), i.e., \(I\) is either \((a, b]\) or \([a, b]\).
    Since \(b \in I\), we know that one of the intervals \(K\) in \(\mathbf{P}\) contains \(b\).
    Since \(K\) is contained in \(I\), it must therefore be of the form \((c, b]\), \([c, b]\), or \(\{b\}\) for some real number \(c\), with \(a \leq c \leq b\) (in the latter case of \(K = \{b\}\), we set \(c \coloneqq b\)).
    In particular, this means that the set \(I \setminus K\) is also an interval of the form \([a, c]\), \((a, c)\), \((a, c]\), \([a, c)\) when \(c > a\), or a point or empty set when \(a = c\).
            Either way, by Definition \ref{11.8.1} we see that
            \begin{align*}
                \alpha[I] & = \alpha[(a, b)] + \alpha(\{b\})                                                              \\
                          & = \lim_{x \to b^- ; x \in X} \alpha(x) - \lim_{x \to a^+ ; x \in X} \alpha(x) + \alpha(\{b\}) \\
                          & = \lim_{x \to b^- ; x \in X} \alpha(x) - \lim_{x \to c^+ ; x \in X} \alpha(x)                 \\
                          & + \lim_{x \to c^+ ; x \in X} \alpha(x) - \lim_{x \to c^- ; x \in X} \alpha(x)                 \\
                          & + \lim_{x \to c^- ; x \in X} \alpha(x) - \lim_{x \to a^+ ; x \in X} \alpha(x) + \alpha(\{b\}) \\
                          & = \alpha[(c, b)] + \alpha[\{c\}] + \alpha[(a, c)] + \alpha(\{b\})                             \\
                          & = \alpha[(a, c)] + \alpha[[c, b]]                                                             \\
                          & = \alpha[K] + \alpha[I \setminus K]
            \end{align*}
            when \(I = (a, b]\) and
    \begin{align*}
        \alpha[I] & = \alpha(\{a\}) + \alpha[(a, b)] + \alpha(\{b\})                                                              \\
                  & = \alpha(\{a\}) + \lim_{x \to b^- ; x \in X} \alpha(x) - \lim_{x \to a^+ ; x \in X} \alpha(x) + \alpha(\{b\}) \\
                  & = \alpha(\{a\}) + \lim_{x \to b^- ; x \in X} \alpha(x) - \lim_{x \to c^+ ; x \in X} \alpha(x)                 \\
                  & + \lim_{x \to c^+ ; x \in X} \alpha(x) - \lim_{x \to c^- ; x \in X} \alpha(x)                                 \\
                  & + \lim_{x \to c^- ; x \in X} \alpha(x) - \lim_{x \to a^+ ; x \in X} \alpha(x) + \alpha(\{b\})                 \\
                  & = \alpha(\{a\}) + \alpha[(c, b)] + \alpha[\{c\}] + \alpha[(a, c)] + \alpha(\{b\})                             \\
                  & = \alpha[[a, c)] + \alpha[[c, b]]                                                                             \\
                  & = \alpha[K] + \alpha[I \setminus K].
    \end{align*}
    when \(I = [a, b]\).
    On the other hand, since \(\mathbf{P}\) forms a partition of \(I\), we see that \(\mathbf{P} \setminus \{K\}\) forms a partition of \(I \setminus K\).
    By the induction hypothesis, we thus have
    \[
        \alpha[I \setminus K] = \sum_{J \in \mathbf{P} \setminus \{K\}} \alpha[J].
    \]
    Combining these two identities (and using the laws of addition for finite sets, see Proposition \ref{7.1.11}(e)) we obtain
    \[
        \alpha[I] = \sum_{J \in \mathbf{P}} \alpha[J]
    \]
    as desired.

    Now suppose that \(b \notin I\), i.e., \(I\) is either \((a, b)\) or \([a, b)\).
    Then one of the intervals \(K\) also is of the form \((c, b)\) or \([c, b)\) (see Exercise \ref{ex 11.1.3}).
            In particular, this means that the set \(I \setminus K\) is also an interval of the form \([a, c]\), \((a, c)\), \((a, c]\), \([a, c)\) when \(c > a\), or a point or empty set when \(a = c\).
    By Definition \ref{11.8.1} we see that
    \begin{align*}
        \alpha[I] & = \alpha[(a, b)]                                                              \\
                  & = \lim_{x \to b^- ; x \in X} \alpha(x) - \lim_{x \to a^+ ; x \in X} \alpha(x) \\
                  & = \lim_{x \to b^- ; x \in X} \alpha(x) - \lim_{x \to c^+ ; x \in X} \alpha(x) \\
                  & + \lim_{x \to c^+ ; x \in X} \alpha(x) - \lim_{x \to c^- ; x \in X} \alpha(x) \\
                  & + \lim_{x \to c^- ; x \in X} \alpha(x) - \lim_{x \to a^+ ; x \in X} \alpha(x) \\
                  & = \alpha[(c, b)] + \alpha[\{c\}] + \alpha[(a, c)]                             \\
                  & = \alpha[(a, c)] + \alpha[[c, b)]                                             \\
                  & = \alpha[K] + \alpha[I \setminus K]
    \end{align*}
    when \(I = (a, b)\) and
    \begin{align*}
        \alpha[I] & = \alpha(\{a\}) + \alpha[(a, b)]                                                              \\
                  & = \alpha(\{a\}) + \lim_{x \to b^- ; x \in X} \alpha(x) - \lim_{x \to a^+ ; x \in X} \alpha(x) \\
                  & = \alpha(\{a\}) + \lim_{x \to b^- ; x \in X} \alpha(x) - \lim_{x \to c^+ ; x \in X} \alpha(x) \\
                  & + \lim_{x \to c^+ ; x \in X} \alpha(x) - \lim_{x \to c^- ; x \in X} \alpha(x)                 \\
                  & + \lim_{x \to c^- ; x \in X} \alpha(x) - \lim_{x \to a^+ ; x \in X} \alpha(x)                 \\
                  & = \alpha(\{a\}) + \alpha[(c, b)] + \alpha[\{c\}] + \alpha[(a, c)]                             \\
                  & = \alpha[[a, c)] + \alpha[[c, b)]                                                             \\
                  & = \alpha[K] + \alpha[I \setminus K].
    \end{align*}
    when \(I = [a, b)\).
    The rest of the argument then proceeds as above.
\end{proof}

\begin{definition}[piecewise constant Riemann-Stieltjes integral]\label{11.8.5}
    Let \(I\) be a bounded interval, and let \(\mathbf{P}\) be a partition of \(I\).
    Let \(\alpha : X \to \mathbf{R}\) be a monotone increasing function defined on some interval \(X\) which contains \(I\), and let \(f : I \to \mathbf{R}\) be a function which is piecewise constant with respect to \(\mathbf{P}\).
    Then we define
    \[
        p.c. \int_{[\mathbf{P}]} f \; d \alpha \coloneqq \sum_{J \in \mathbf{P}} c_J \alpha[J]
    \]
    where \(c_J\) is the constant value of \(f\) on \(J\).
\end{definition}

\setcounter{theorem}{6}
\begin{example}\label{11.8.7}
    Let \(\alpha : \mathbf{R} \to \mathbf{R}\) be the identity function \(\alpha(x) \coloneqq x\).
    Then for any bounded interval \(I\), any partition \(\mathbf{P}\) of \(I\), and any function \(f\) that is piecewise constant with respect to \(P\), we have \(p.c. \int_{[\mathbf{P}]} f \; d \alpha = p.c. \int_{[\mathbf{P}]} f\).
\end{example}

\begin{proposition}\label{11.8.8}
    Let \(I\) be a bounded interval, let \(\alpha : X \to \mathbf{R}\) be a monotone increasing function defined on some interval \(X\) which contains \(I\), and let \(f : I \to \mathbf{R}\) be a function.
    Suppose that \(\mathbf{P}\) and \(\mathbf{P}'\) are partitions of \(I\) such that \(f\) is piecewise constant both with respect to \(\mathbf{P}\) and with respect to \(\mathbf{P}'\).
    Also suppose that both \(p.c. \int_{[\mathbf{P}]} f \; d \alpha\) and \(p.c. \int_{[\mathbf{P}']} f \; d \alpha\) are well-defined.
    Then \(p.c. \int_{[\mathbf{P}]} f \; d \alpha = p.c. \int_{[\mathbf{P}']} f \; d \alpha\).
\end{proposition}

\begin{proof}
    By Lemma \ref{11.1.18} we know that \(\mathbf{P} \# \mathbf{P}'\) is a partition of \(I\) and is both finer than \(\mathbf{P}\) and finer than \(\mathbf{P}'\), thus by Definition \ref{11.8.5} we have
    \[
        p.c. \int_{[\mathbf{P} \# \mathbf{P}']} f \; d \alpha = \sum_{J \in \mathbf{P} \# \mathbf{P}'} c_J \alpha[J].
    \]
    By Lemma \ref{11.8.4}, we know that
    \[
        \alpha[I] = \sum_{J \in \mathbf{P}} \alpha[J] = \sum_{J \in \mathbf{P} \# \mathbf{P}'} \alpha[J].
    \]
    By Definition \ref{11.1.14}, \(\forall\ S \in \mathbf{P} \# \mathbf{P}'\), \(\exists\ K \in \mathbf{P}\) such that \(S \subseteq K\).
    Now we fix such \(K\) and let \(\mathbf{P}_K\) be the set
    \[
        \mathbf{P}_K = \{S \in \mathbf{P} \# \mathbf{P}' : S \subseteq K\}.
    \]
    We know that \(\mathbf{P}_K\) is a partition of \(K\) and \(\bigcup_{K \in \mathbf{P}} \mathbf{P}_K = \mathbf{P} \# \mathbf{P}'\).
    (See the proof of Proposition \ref{11.2.13})
    Since \(f\) is piecewise constant with respect to \(\mathbf{P}\), by Lemma \ref{11.2.7} we know that \(f\) is piecewise constant with respect to \(\mathbf{P} \# \mathbf{P}'\).
    So we have
    \begin{align*}
        p.c. \int_{[\mathbf{P} \# \mathbf{P}']} f \; d \alpha & = \sum_{J \in \mathbf{P} \# \mathbf{P}'} c_J \alpha[J]                        & \text{(by Definition \ref{11.8.5})}     \\
                                                              & = \sum_{J \in \bigcup_{K \in \mathbf{P}} \mathbf{P}_K} c_J \alpha[J]                                                    \\
                                                              & = \sum_{K \in \mathbf{P}} \sum_{J \in \mathbf{P}_K} c_J \alpha[J]             & \text{(by Proposition \ref{7.1.11}(e))} \\
                                                              & = \sum_{K \in \mathbf{P}} \sum_{J \in \mathbf{P}_K} c_K \alpha[J]             & (J \subseteq K)                         \\
                                                              & = \sum_{K \in \mathbf{P}} c_K \bigg(\sum_{J \in \mathbf{P}_K} \alpha[J]\bigg)                                           \\
                                                              & = \sum_{K \in \mathbf{P}} c_K \alpha[K]                                       & \text{(by Lemma \ref{11.8.4})}          \\
                                                              & = p.c. \int_{[\mathbf{P}]} f \; d \alpha.                                     & \text{(by Definition \ref{11.8.5})}
    \end{align*}
    Using similar arguments we can show that \(p.c. \int_{[\mathbf{P}']} f \; d \alpha = p.c. \int_{[\mathbf{P} \# \mathbf{P}']} f \; d \alpha\).
    Thus we have \(p.c. \int_{[\mathbf{P}]} f \; d \alpha = p.c. \int_{[\mathbf{P}']} f \; d \alpha\).
\end{proof}

\begin{definition}\label{11.8.9}
    Let \(I\) be a bounded interval, let \(\alpha : X \to \mathbf{R}\) be a monotone increasing function defined on some interval \(X\) which contains \(I\), and let \(f : I \to \mathbf{R}\) be a piecewise constant function on \(I\).
    Then we define
    \[
        p.c. \int_I f \; d \alpha \coloneqq p.c. \int_{[\mathbf{P}]} f \; d \alpha,
    \]
    where \(\mathbf{P}\) is any partition of \(I\) with respect to which \(f\) is piecewise constant.
    (Note that Proposition \ref{11.8.8} tells us that the precise choice of this partition is irrelevant.)
\end{definition}

\begin{theorem}\label{11.8.10}
    Let \(I\) be a bounded interval, let \(\alpha : X \to \mathbf{R}\) be a monotone increasing function defined on some interval \(X\) which contains \(I\), and let \(f : I \to \mathbf{R}\) and \(g : I \to \mathbf{R}\) be piecewise constant functions on \(I\) such that both \(p.c. \int_I f \; d \alpha\) and \(p.c. \int_I g \; d \alpha\) are well-defined.
    \begin{enumerate}
        \item We have \(p.c. \int_I (f + g) \; d \alpha = p.c. \int_I f \; d \alpha + p.c. \int_I g \; d \alpha\).
        \item For any real number \(c\), we have \(p.c. \int_I (cf) \; d \alpha = c (p.c. \int_I f \; d \alpha)\).
        \item We have \(p.c. \int_I (f - g) \; d \alpha = p.c. \int_I f \; d \alpha - p.c. \int_I g \; d \alpha\).
        \item If \(f(x) \geq 0\) for all \(x \in I\), then \(p.c. \int_I f \; d \alpha \geq 0\).
        \item If \(f(x) \geq g(x)\) for all \(x \in I\), then \(p.c. \int_I f \; d \alpha \geq p.c. \int_I g \; d \alpha\).
        \item If \(f\) is the constant function \(f(x) = c\) for all \(x \in I\), then \(p.c. \int_I f \; d \alpha = c \alpha[I]\).
        \item Let \(J\) be a bounded interval containing \(I\) (i.e., \(I \subseteq J\)), and let \(F : J \to \mathbf{R}\) be the function
              \[
                  F(x) \coloneqq \begin{cases}
                      f(x) & \text{if } x \in I    \\
                      0    & \text{if } x \notin I
                  \end{cases}
              \]
              Then \(F\) is piecewise constant on \(J\), and \(p.c. \int_J F \; d \alpha = p.c. \int_I f \; d \alpha\).
        \item Suppose that \(\{J, K\}\) is a partition of \(I\) into two intervals \(J\) and \(K\).
              Then the function \(f|_J : J \to \mathbf{R}\) and \(f|_K : K \to \mathbf{R}\) are piecewise constant on \(J\) and \(K\) respectively, and we have
              \[
                  p.c. \int_I f \; d \alpha = p.c. \int_J f|_J \; d \alpha + p.c. \int_K f|_K \; d \alpha.
              \]
    \end{enumerate}
\end{theorem}

\begin{proof}{(a)}
    Since \(f, g\) are both piecewise constant on \(I\), by Lemma \ref{11.2.8} \(f + g\) is also piecewise constant on \(I\).
    By Definition \ref{11.2.3}, \(\exists\ \mathbf{P}_f, \mathbf{P}_g\) such that \(\mathbf{P}_f, \mathbf{P}_g\) are partitions of \(I\), \(f\) is piecewise constant with respect to \(\mathbf{P}_f\) and \(g\) is piecewise constant with respect to \(\mathbf{P}_g\).
    Let \(\mathbf{P} = \mathbf{P}_f \# \mathbf{P}_g\).
    Then by Lemma \ref{11.1.18} we know that \(\mathbf{P}\) is also a partition of \(I\) and by Lemma \ref{11.2.7} \(f, g\) are piecewise constant with respect to \(\mathbf{P}\).
    Now let \(J \in \mathbf{P}\), let \(f_J \in \mathbf{R}\) be the constant value of \(f\) on \(J\) and \(g_J \in \mathbf{R}\) be the constant value of \(g\) on \(J\).
    Then by Definition \ref{11.2.1} \(f_J + g_J\) is also a constant of \(f + g\) on \(J\).
    Thus we have \(f + g\) is piecewise constant with respect to \(\mathbf{P}\) and
    \begin{align*}
        p.c. \int_I f \; d \alpha + p.c. \int_I g \; d \alpha & = p.c. \int_{[\mathbf{P}]} f \; d \alpha + p.c. \int_{[\mathbf{P}]} g \; d \alpha & \text{(by Definition \ref{11.8.9})}     \\
                                                              & = \sum_{J \in \mathbf{P}} f_J \alpha[J] + \sum_{J \in \mathbf{P}} g_J \alpha[J]   & \text{(by Definition \ref{11.8.5})}     \\
                                                              & = \sum_{J \in \mathbf{P}} (f_J + g_J) \alpha[J]                                   & \text{(by Proposition \ref{7.1.11}(f))} \\
                                                              & = p.c. \int_{[\mathbf{P}]} (f_J + g_J) \; d \alpha                                & \text{(by Definition \ref{11.8.5})}     \\
                                                              & = p.c. \int_I (f_J + g_J) \; d \alpha.                                            & \text{(by Definition \ref{11.8.9})}
    \end{align*}
\end{proof}

\begin{proof}{(b)}
    Since \(f\) is piecewise constant on \(I\), by Lemma \ref{11.2.8} \(cf\) is also piecewise constant on \(I\) (since \(c\) is constant on \(I\)).
    By Definition \ref{11.2.3}, \(\exists\ \mathbf{P}\) such that \(\mathbf{P}\) is a partition of \(I\) and \(f\) is piecewise constant with respect to \(\mathbf{P}\).
    Now let \(J \in \mathbf{P}\) and let \(f_J \in \mathbf{R}\) be the constant value of \(f\) on \(J\).
    Then by Definition \ref{11.2.1} \(c f_J\) is also a constant of \(cf\) on \(J\).
    Thus we have \(cf\) is piecewise constant with respect to \(\mathbf{P}\) and
    \begin{align*}
        c (p.c. \int_I f \; d \alpha) & = c (p.c. \int_{[\mathbf{P}]} f \; d \alpha) & \text{(by Definition \ref{11.8.9})}     \\
                                      & = c (\sum_{J \in \mathbf{P}} f_J \alpha[J])  & \text{(by Definition \ref{11.8.5})}     \\
                                      & = \sum_{J \in \mathbf{P}} c f_J \alpha[J]    & \text{(by Proposition \ref{7.1.11}(g))} \\
                                      & = p.c. \int_{[\mathbf{P}]} (c f) \; d \alpha & \text{(by Definition \ref{11.8.5})}     \\
                                      & = p.c. \int_I (c f) \; d \alpha.             & \text{(by Definition \ref{11.8.9})}
    \end{align*}
\end{proof}

\begin{proof}{(c)}
    We have
    \begin{align*}
        p.c. \int_I f \; d \alpha - p.c. \int_I g \; d \alpha & = p.c. \int_I f \; d \alpha + (-1) p.c. \int_I g \; d \alpha                                        \\
                                                              & = p.c. \int_I f \; d \alpha + p.c. \int_I (-g) \; d \alpha   & \text{(by Theorem \ref{11.8.10}(b))} \\
                                                              & = p.c. \int_I (f + (-g)) \; d \alpha                         & \text{(by Theorem \ref{11.8.10}(a))} \\
                                                              & = p.c. \int_I (f - g) \; d \alpha.                           & \text{(by Definition \ref{9.2.1})}
    \end{align*}
\end{proof}

\begin{proof}{(d)}
    By Definition \ref{11.2.3}, \(\exists\ \mathbf{P}\) such that \(\mathbf{P}\) is a partition of \(I\) and \(f\) is piecewise constant with respect to \(\mathbf{P}\).
    Let \(J \in \mathbf{P}\) and let \(f_J \in \mathbf{R}\) be the constant value of \(f\) on \(J\).
    Since \(\alpha\) is monotone increasing and \(p.c. \int_I f \; d \alpha\) is well-defined, we know that \(\alpha[J]\) is well-defined.
    So for any \(a, b \in J\) and \(a \leq b\) we have
    \begin{align*}
        \alpha[J] & = \alpha[(a, b)]                                                                                                                                             \\
                  & = \lim_{x \to b^- ; x \in (a, b)} \alpha(x) - \lim_{x \to a^+ ; x \in (a, b)} \alpha(x)                                & \text{(by Definition \ref{11.8.1})} \\
                  & = \lim_{x \to b ; x \in (a, b) \cap (-\infty, b)} \alpha(x) - \lim_{x \to a ; x \in (a, b) \cap (a, \infty)} \alpha(x) & \text{(by Definition \ref{9.5.1})}  \\
                  & = \lim_{x \to b ; x \in (a, b)} \alpha(x) - \lim_{x \to a ; x \in (a, b)} \alpha(x)                                                                          \\
                  & \geq 0                                                                                                                 & (a \leq b)
    \end{align*}
    when \(J\) is a point or a open interval, and \(\alpha[J] = 0\) when \(J = \emptyset\).
    Similar arguments work for the cases \([a, b), (a, b], [a, b]\), and we conclude that \(\alpha[J] \geq 0\).
    Since \(\forall\ x \in I\), \(f(x) \geq 0\), we then have \(f_J \geq 0\) and \(f_J \alpha[J] \geq 0\).
    Thus
    \begin{align*}
        p.c. \int_I f \; d \alpha & = p.c. \int_{[\mathbf{P}]} f \; d \alpha & \text{(by Definition \ref{11.8.9})}     \\
                                  & = \sum_{J \in \mathbf{P}} f_J \alpha[J]  & \text{(by Definition \ref{11.8.5})}     \\
                                  & \geq \sum_{J \in \mathbf{P}} 0           & \text{(by Proposition \ref{7.1.11}(h))} \\
                                  & = 0.
    \end{align*}
\end{proof}

\begin{proof}{(e)}
    Since \(f(x) \geq g(x)\) for all \(x \in I\), we have \(f(x) - g(x) \geq 0\) and
    \begin{align*}
        p.c. \int_I f \; d \alpha - p.c. \int_I g \; d \alpha & = p.c. \int_I (f - g) \; d \alpha & \text{(by Theorem \ref{11.8.10}(c))} \\
                                                              & \geq 0.                           & \text{(by Theorem \ref{11.8.10}(d))}
    \end{align*}
    Thus
    \[
        p.c. \int_I f \; d \alpha \geq p.c. \int_I g \; d \alpha.
    \]
\end{proof}

\begin{proof}{(f)}
    Since \(I\) is a partition of \(I\), we have
    \begin{align*}
        p.c. \int_I f \; d \alpha & = p.c. \int_{[I]} f \; d \alpha & \text{(by Definition \ref{11.8.9})}     \\
                                  & = \sum_{J \in I} c \alpha[J]    & \text{(by Definition \ref{11.8.5})}     \\
                                  & = c \sum_{J \in I} \alpha[J]    & \text{(by Proposition \ref{7.1.11}(g))} \\
                                  & = c \alpha[I].                  & \text{(by Lemma \ref{11.8.4})}
    \end{align*}
\end{proof}

\begin{proof}{(g)}
    Let \(I_1, I_2\) be the sets
    \[
        I_1 = \{x \in J, x \leq \inf(I) \land x \notin I\}
    \]
    and
    \[
        I_2 = \{x \in J, x \geq \sup(I) \land x \notin I \cup I_1\}.
    \]
    Then we know that \(I \cap I_1 = I \cap I_2 = I_1 \cap I_2 = \emptyset\).
    Let \(\mathbf{P} = \{I_1, I, I_2\}\).
    We know that \(\mathbf{P}\) is a partition of \(J\) and \(F\) is piecewise constant on \(J\)
    (See the proof of Theorem \ref{11.2.16}(g)).
    Since \(f\) is piecewise constant on \(I\), by Definition \ref{11.2.5} \(\exists\ \mathbf{P}_I\) such that \(\mathbf{P}_I\) be the partition of \(I\) and \(f\) is piecewise constant with respect to \(\mathbf{P}_I\).
    Thus we have
    \begin{align*}
        p.c. \int_J F \; d \alpha & = p.c. \int_{[\mathbf{P}]} F \; d \alpha                                              & \text{(by Definition \ref{11.8.9})}     \\
                                  & = \sum_{K \in \mathbf{P}} c_K \alpha[K]                                               & \text{(by Definition \ref{11.8.5})}     \\
                                  & = c_{I_1} \alpha[I_1] + \sum_{K \in \mathbf{P}_I} c_K \alpha[K] + c_{I_2} \alpha[I_2] & \text{(by Proposition \ref{7.1.11}(e))} \\
                                  & = 0 \alpha[I_1] + \sum_{K \in \mathbf{P}_I} c_K \alpha[K] + 0 \alpha[I_2]             & \text{(by hypothesis)}                  \\
                                  & = \sum_{K \in \mathbf{P}_I} c_K \alpha[K]                                                                                       \\
                                  & = p.c. \int_{[\mathbf{P}_I]} f \; d \alpha                                            & \text{(by Definition \ref{11.8.5})}     \\
                                  & = p.c. \int_I f \; d \alpha.                                                          & \text{(by Definition \ref{11.8.9})}
    \end{align*}
\end{proof}

\begin{proof}{(h)}
    By Theorem \ref{11.2.16}(h) we know that \(f|_J\) is piecewise constant on \(J\) and \(f|_K\) is piecewise constant on \(K\).
    Since \(f\) is a piecewise constant function on \(I\), by Definition \ref{11.2.5} \(\exists\ \mathbf{P}\) such that \(\mathbf{P}\) is a partition of \(I\) and \(f\) is piecewise constant with respect to \(\mathbf{P}\).
    Let \(\mathbf{P}_J, \mathbf{P}_K\) be the sets
    \[
        \mathbf{P}_J = \{S \cap J : S \in \mathbf{P}\}
    \]
    and
    \[
        \mathbf{P}_K = \{S \cap K : S \in \mathbf{P}\}.
    \]
    We know that \(\mathbf{P}_J\) is a partition of \(J\), \(\mathbf{P}_K\) is a partition of \(K\) and \(\mathbf{P} = \mathbf{P}_J \cup \mathbf{P}_K\)
    (See the proof of Theorem \ref{11.2.16}(h)).
    Thus we have
    \begin{align*}
          & p.c. \int_J f|_J \; d \alpha + p.c. \int_K f|_K \; d \alpha                                                                         \\
        = & p.c. \int_{[\mathbf{P}_J]} f|_J \; d \alpha + p.c. \int_{[\mathbf{P}_K]} f|_K \; d \alpha & \text{(by Definition \ref{11.8.9})}     \\
        = & \sum_{S \in \mathbf{P}_J} c_S \alpha[S] + \sum_{S \in \mathbf{P}_K} c_S \alpha[S]         & \text{(by Proposition \ref{7.1.11}(e))} \\
        = & \sum_{S \in \mathbf{P}_J \cup \mathbf{P}_K} c_S \alpha[S]                                 & \text{(by Definition \ref{11.8.5})}     \\
        = & \sum_{S \in \mathbf{P}} c_S \alpha[S]                                                                                               \\
        = & p.c. \int_{[\mathbf{P}]} f \; d \alpha                                                    & \text{(by Definition \ref{11.8.5})}     \\
        = & p.c. \int_I f \; d \alpha.                                                                & \text{(by Definition \ref{11.8.9})}
    \end{align*}
\end{proof}

\begin{definition}\label{11.8.11}
    Let \(I\) be a bounded interval, let \(\alpha : X \to \mathbf{R}\) be a monotone increasing function defined on some interval \(X\) which contains \(I\), and let \(f : I \to \mathbf{R}\) be a bounded function.
    We define the \emph{upper Riemann-Stieltjes integral} \(\overline{\int}_I f\) by the formula
    \[
        \overline{\int}_I f \; d \alpha \coloneqq \inf\{p.c. \int_I g \; d \alpha : g \text{ is a p.c. function on \(I\) which majorizes } f\}
    \]
    and the \emph{lower Riemann-Stieltjes integral} \(\underline{\int}_I f\) by the formula
    \[
        \underline{\int}_I f \; d \alpha \coloneqq \sup\{p.c. \int_I g \; d \alpha : g \text{ is a p.c. function on \(I\) which minorizes } f\}.
    \]
    If \(\underline{\int}_I f \; d \alpha = \overline{\int}_I f \; d \alpha\), then we say that \(f\) is \emph{Riemann-Stieltjes integrable on \(I\) with respect to \(\alpha\)} and define
    \[
        \int_I f \; d \alpha \coloneqq \underline{\int}_I f \; d \alpha = \overline{\int}_I f \; d \alpha.
    \]
    If the upper and lower Riemann-Stieltjes integrals are unequal, we say that \(f\) is not Riemann-Stieltjes integrable.
\end{definition}

\begin{note}
    When \(\alpha\) is the identity function \(\alpha(x) \coloneqq x\) then the Riemann-Stieltjes integral is identical to the Riemann integral;
    thus the Riemann-Stieltjes integral is a generalization of the Riemann integral.
    We sometimes write \(\int_I f\) as \(\int_I f \; dx\) or \(\int_I f(x) \; dx\).
\end{note}

\begin{theorem}\label{11.8.12}
    Let \(I\) be a bounded interval, let \(\alpha : X \to \mathbf{R}\) be a monotone increasing function defined on some interval \(X\) which contains \(I\), and let \(f\) be a function which is uniformly continuous on \(I\).
    Then \(f\) is Riemann-Stieltjes integrable.
\end{theorem}

\begin{proof}
    From Proposition \ref{9.9.15} we see that \(f\) is bounded.
    Now we have to show that \(\underline{\int}_I f \; d \alpha = \overline{\int}_I f \; d \alpha\).

    If \(I\) is a point or the empty set then the theorem is trivial, so let us assume that \(I\) is one of the four intervals \([a, b]\), \((a, b)\), \((a, b]\), or \([a, b)\) for some real numbers \(a < b\).

            Let \(\varepsilon > 0\) be arbitrary.
            By uniform continuity, there exists a \(\delta > 0\) such that \(\abs*{f(x) - f(y)} < \varepsilon\) whenever \(x, y \in I\) are such that \(\abs*{x - y} < \delta\).
            By the Archimedean principle, there exists an integer \(N > 0\) such that \((b - a) / N < \delta\) and \((\alpha(b) - \alpha(a)) / N < \delta\).

            Note that we can partition \(I\) into \(N\) intervals \(J_1, \dots, J_N\), each of length \((b - a) / N\).
            (How? One has to treat each of the cases \([a, b]\), \((a, b)\), \((a, b]\), \([a, b)\) slightly differently.)
    By Proposition \ref{11.3.12}, we thus have
    \[
        \overline{\int}_I f \; d \alpha \leq \sum_{k = 1}^N (\sup_{x \in J_k} f(x)) \alpha[J_k]
    \]
    and
    \[
        \underline{\int}_I f \; d \alpha \geq \sum_{k = 1}^N (\inf_{x \in J_k} f(x)) \alpha[J_k]
    \]
    so in particular
    \[
        \overline{\int}_I f \; d \alpha - \underline{\int}_I f \; d \alpha \leq \sum_{k = 1}^N (\sup_{x \in J_k} f(x) - \inf_{x \in J_k} f(x)) \alpha[J_k].
    \]
    However, we have \(\abs*{f(x) - f(y)} < \varepsilon\) for all \(x, y \in J_k\), since \(\alpha[J_k] = (\alpha(b) - \alpha(a)) / N < \delta\).
    In particular we have
    \[
        f(x) < f(y) + \varepsilon \text{ for all } x, y \in J_k.
    \]
    Taking suprema in \(x\), we obtain
    \[
        \sup_{x \in J_k} f(x) \leq f(y) + \varepsilon \text{ for all } y \in J_k,
    \]
    and then taking infima in \(y\) we obtain
    \[
        \sup_{x \in J_k} f(x) \leq \inf_{y \in J_k} f(y) + \varepsilon.
    \]
    Inserting this bound into our previous inequality, we obtain
    \[
        \overline{\int}_I f \; d \alpha - \underline{\int}_I f \; d \alpha \leq \sum_{k = 1}^N \varepsilon \alpha[J_k],
    \]
    but by Lemma \ref{11.8.4} we thus have
    \[
        \overline{\int}_I f - \underline{\int}_I f \leq \varepsilon \alpha[I].
    \]
    But \(\varepsilon > 0\) was arbitrary, while \(\alpha[I]\) is fixed.
    Thus \(\overline{\int}_I f - \underline{\int}_I f\) cannot be positive.
    By Definition \ref{11.8.11} we thus have that \(f\) is Riemann-Stieltjes integrable.
\end{proof}

\exercisesection

\begin{exercise}\label{ex 11.8.1}
    Prove Lemma \ref{11.8.4}.
\end{exercise}

\begin{proof}
    See Lemma \ref{11.8.4}.
\end{proof}

\begin{exercise}\label{ex 11.8.2}
    State and prove a version of Proposition \ref{11.2.13} for the Riemann-Stieltjes integral.
\end{exercise}

\begin{proof}
    See Proposition \ref{11.8.8}.
\end{proof}

\begin{exercise}\label{ex 11.8.3}
    State and prove a version of Theorem \ref{11.2.16} for the Riemann-Stieltjes integral.
\end{exercise}

\begin{proof}
    See Theorem \ref{11.8.10}.
\end{proof}

\begin{exercise}\label{ex 11.8.4}
    State and prove a version of Theorem \ref{11.5.1} for the Riemann-Stieltjes integral.
\end{exercise}

\begin{proof}
    See Theorem \ref{11.8.12}.
\end{proof}

\begin{exercise}\label{ex 11.8.5}
    Let \(\text{sgn} : \mathbf{R} \to \mathbf{R}\) be the signum function
    \[
        \text{sgn}(x) = \begin{cases}
            1  & \text{when } x > 0  \\
            0  & \text{when } x = 0  \\
            -1 & \text{when } x < 0.
        \end{cases}
    \]
    Let \(f : [-1, 1] \to \mathbf{R}\) be a continuous function.
    Show that \(f\) is Riemann-Stieltjes integrable with respect to \(\text{sgn}\), and that
    \[
        \int_{[-1, 1]} f \; d \, \text{sgn} = 2f(0).
    \]
\end{exercise}

\begin{proof}
    We first show that \(f\) is Riemann-Stieltjes integrable on \([-1, 1]\) with respect to \(\text{sgn}\).
    By Theorem \ref{9.9.16} \(f\) is uniformly continuous, and thus by Proposition \ref{9.9.15} \(f\) is bounded.
    Since \(\text{sgn}\) is monoton increasing, by Theorem \ref{11.8.12} we know that \(f\) is Riemann-Stieltjes integrable on \([-1, 1]\) with respect to \(\text{sgn}\).

    Now we show that \(\int_{[-1, 1]} f \; d \, \text{sgn} = 2f(0)\).
    Since \(f\) is continuous, by Proposition \ref{9.4.7} \(\forall\ \varepsilon \in \mathbf{R}^+\), \(\exists\ \delta \in \mathbf{R}^+\) such that
    \begin{align*}
        \forall\ x \in [-1, 1], \abs*{x - 0} \leq \delta \implies & \abs*{f(x) - f(0)} \leq \varepsilon                   \\
        \implies                                                  & f(0) - \varepsilon \leq f(x) \leq f(0) + \varepsilon.
    \end{align*}
    Since \(f\) is bounded, \(\exists\ M \in \mathbf{R}\) such that \(\abs*{f(x)} \leq M\) for all \(x \in [-1, 1]\).
    Let \(f_U : [-1, 1] \to \mathbf{R}\) be the function
    \[
        f_U(x) = \begin{cases}
            f(0) + \varepsilon & \text{if } x \in [-\delta, \delta]                   \\
            M                  & \text{if } x \in [-1, 1] \setminus [-\delta, \delta]
        \end{cases}
    \]
    and let \(f_L : [-1, 1] \to \mathbf{R}\) be the function
    \[
        f_L(x) = \begin{cases}
            f(0) - \varepsilon & \text{if } x \in [-\delta, \delta]                   \\
            -M                 & \text{if } x \in [-1, 1] \setminus [-\delta, \delta]
        \end{cases}
    \]
    Clearly \(f_U\) majorizes \(f\) and \(f_L\) minorizes \(f\).
    Then we have
    \begin{align*}
        \overline{\int}_{[-1, 1]} f \; d \text{sgn} & \leq p.c. \int f_U \; d \text{sgn}                                & \text{(by Definition \ref{11.8.11})} \\
                                                    & = M (\text{sgn}(-\delta) - \text{sgn}(-1))                        & \text{(by Definition \ref{11.8.5})}  \\
                                                    & + (f(0) + \varepsilon) (\text{sgn}(\delta) - \text{sgn}(-\delta))                                        \\
                                                    & + M (\text{sgn}(1) - \text{sgn}(\delta))                                                                 \\
                                                    & = 2(f(0) + \varepsilon)
    \end{align*}
    and
    \begin{align*}
        \underline{\int}_{[-1, 1]} f \; d \text{sgn} & \geq p.c. \int f_L \; d \text{sgn}                                & \text{(by Definition \ref{11.8.11})} \\
                                                     & = M (\text{sgn}(-\delta) - \text{sgn}(-1))                        & \text{(by Definition \ref{11.8.5})}  \\
                                                     & + (f(0) - \varepsilon) (\text{sgn}(\delta) - \text{sgn}(-\delta))                                        \\
                                                     & + M (\text{sgn}(1) - \text{sgn}(\delta))                                                                 \\
                                                     & = 2(f(0) - \varepsilon).
    \end{align*}
    Combining results above we have
    \[
        2(f(0) - \varepsilon) \leq \underline{\int}_{[-1, 1]} f \; d \text{sgn} = \int_{[-1, 1]} f \; d \text{sgn} = \overline{\int}_{[-1, 1]} f \; d \text{sgn} \leq 2(f(0) + \varepsilon).
    \]
    Since \(\varepsilon\) is arbitrary, we thus have \(\int_{[-1, 1]} f \; d \text{sgn} = 2f(0)\).
\end{proof}
\section{The two fundamental theorems of calculus}\label{sec 11.9}

\begin{theorem}[First Fundamental Theorem of Calculus]\label{11.9.1}
    Let \(a < b\) be real numbers, and let \(f : [a, b] \to \mathbf{R}\) be a Riemann integrable function.
    Let \(F : [a, b] \to \mathbf{R}\) be the function
    \[
        F(x) \coloneqq \int_{[a, x]} f.
    \]
    Then \(F\) is continuous.
    Furthermore, if \(x_0 \in [a, b]\) and \(f\) is continuous at \(x_0\), then \(F\) is differentiable at \(x_0\), and \(F'(x_0) = f(x_0)\).
\end{theorem}

\begin{proof}
    Since \(f\) is Riemann integrable, it is bounded (by Definition \ref{11.3.4}).
    Thus we have some real number \(M\) such that \(-M \leq f(x) \leq M\) for all \(x \in [a, b]\).

    Now let \(x < y\) be two elements of \([a, b]\).
    Then notice that
    \[
        F(y) - F(x) = \int_{[a, y]} f - \int_{[a, x]} f = \int_{[x, y]} f
    \]
    by Theorem \ref{11.4.1}(h).
    By Theorem \ref{11.4.1}(e) we thus have
    \[
        \int_{[x, y]} f \leq \int_{[x, y]} M = p.c. \int_{[x, y]} M = M(y - x)
    \]
    and
    \[
        \int_{[x, y]} f \geq \int_{[x, y]} -M = p.c. \int_{[x, y]} -M = -M(y - x)
    \]
    and thus
    \[
        \abs*{F(y) - F(x)} \leq M(y - x).
    \]
    This is for \(y > x\).
    By interchanging \(x\) and \(y\) we thus see that
    \[
        \abs*{F(y) - F(x)} \leq M(x - y)
    \]
    when \(x > y\).
    Also, we have \(F(y) - F(x) = 0\) when \(x = y\).
    Thus in all
    three cases we have
    \[
        \abs*{F(y) - F(x)} \leq M \abs*{x - y}.
    \]
    Now let \(z \in [a, b]\), and let \((z_n)_{n = 0}^\infty\) be any sequence in \([a, b]\) converging to \(z\).
    Then we have
    \[
        -M \abs*{z_n - z} \leq F(z_n) - F(z) \leq M \abs*{z_n - z}
    \]
    for each \(n\).
    But \(-M \abs*{z_n - z}\) and \(M \abs*{z_n - z}\) both converge to \(0\) as \(n \to \infty\), so by the squeeze test \(F(z_n) - F(z)\) converges to \(0\) as \(n \to \infty\), and thus \(\lim_{n \to \infty} F(z_n) = F(z)\).
    Since this is true for all sequences \(z_n \in [a, b]\) converging to \(z\), we thus see that \(F\) is continuous at \(z\) (by Proposition \ref{9.4.7}).
    Since \(z\) was an arbitrary element of \([a, b]\), we thus see that \(F\) is continuous
    (The above proof also show that when \(F\) is Lipschitz continuous, \(F\) is also continuous, see Exercise \ref{ex 10.2.6}).

    Now suppose that \(x_0 \in [a, b]\), and \(f\) is continuous at \(x_0\).
    Choose any \(\varepsilon > 0\).
    Then by continuity, we can find a \(\delta > 0\) such that \(\abs*{f(x) - f(x_0)} \leq \varepsilon\) for all \(x\) in the interval \(I \coloneqq [x_0 - \delta, x_0 + \delta] \cap [a, b]\), or in other words
    \[
        f(x_0) - \varepsilon \leq f(x) \leq f(x_0) + \varepsilon \text{ for all } x \in I.
    \]
    We now show that
    \[
        \abs*{F(y) - F(x_0) - f(x_0)(y - x_0)} \leq \varepsilon \abs*{y - x_0}
    \]
    for all \(y \in I\), since Proposition \ref{10.1.7} will then imply that \(F\) is differentiable at \(x_0\) with derivative \(F'(x_0) = f(x_0)\) as desired.

    Now fix \(y \in I\).
    There are three cases.
    If \(y = x_0\), then \(F(y) - F(x_0) - f(x_0)(y - x_0) = 0\) and so the claim is obvious.
    If \(y > x_0\), then
    \[
        F(y) - F(x_0) = \int_{[x_0, y]} f.
    \]
    Since \(x_0\), \(y \in I\), and \(I\) is a connected set (by Corollary \ref{11.1.6}), then \([x_0, y]\) is a subset of \(I\), and thus we have
    \[
        f(x_0) - \varepsilon \leq f(x) \leq f(x_0) + \varepsilon \text{ for all } x \in [x_0, y],
    \]
    and thus by Theorem \ref{11.4.1}(e)
    \[
        (f(x_0) - \varepsilon) (y - x_0) \leq \int_{[x_0, y]} f \leq (f(x_0) + \varepsilon) (y - x_0)
    \]
    and so in particular
    \[
        \abs*{F(y) - F(x_0) - f(x_0)(y - x_0)} \leq \varepsilon \abs*{y - x_0}
    \]
    as desired.
    If \(y < x_0\), then
    \[
        F(y) - F(x_0) = - (F(x_0) - F(y)) = -\int_{[y, x_0]} f.
    \]
    Since \(x_0\), \(y \in I\), and \(I\) is a connected set (by Corollary \ref{11.1.6}), then \([y, x_0]\) is a subset of \(I\), and thus we have
    \[
        f(x_0) - \varepsilon \leq f(x) \leq f(x_0) + \varepsilon \text{ for all } x \in [y, x_0],
    \]
    and thus by Theorem \ref{11.4.1}(e)
    \begin{align*}
                 & (f(x_0) - \varepsilon) (x_0 - y) \leq \int_{[y, x_0]} f \leq (f(x_0) + \varepsilon) (x_0 - y)   \\
        \implies & -(f(x_0) - \varepsilon) (y - x_0) \leq \int_{[y, x_0]} f \leq -(f(x_0) + \varepsilon) (y - x_0) \\
        \implies & -\varepsilon (y - x_0) \leq \int_{[y, x_0]} f + f(x_0)(y - x_0) \leq \varepsilon (y - x_0)      \\
        \implies & -\varepsilon (y - x_0) \leq F(x_0) - F(y) + f(x_0)(y - x_0) \leq \varepsilon (y - x_0)          \\
        \implies & -\varepsilon (y - x_0) \leq F(y) - F(x_0) - f(x_0)(y - x_0) \leq \varepsilon (y - x_0)          \\
        \implies & \abs*{F(y) - F(x_0) - f(x_0)(y - x_0)} \leq \varepsilon (y - x_0)
    \end{align*}
    as desired.
\end{proof}

\begin{note}
    Informally, the first fundamental theorem of calculus asserts that
    \[
        \bigg(\int_{[a, x]} f\bigg)'(x) = f(x)
    \]
    given a certain number of assumptions on \(f\).
    Roughly, this means that the derivative of an integral recovers the original function.
\end{note}

\setcounter{theorem}{2}
\begin{definition}[Antiderivatives]\label{11.9.3}
    Let \(I\) be a bounded interval, and let \(f : I \to \mathbf{R}\) be a function.
    We say that a function \(F : I \to \mathbf{R}\) is an \emph{antiderivative} of \(f\) if \(F\) is differentiable on \(I\) and \(F'(x) = f(x)\) for all limit points \(x\) of \(I\).
\end{definition}

\begin{theorem}[Second Fundamental Theorem of Calculus]\label{11.9.4}
    Let \(a < b\) be real numbers, and let \(f : [a, b] \to \mathbf{R}\) be a Riemann integrable function.
    If \(F : [a, b] \to \mathbf{R}\) is an antiderivative of \(f\), then
    \[
        \int_{[a, b]} f = F(b) - F(a).
    \]
\end{theorem}

\begin{proof}
    The claim is trivial when \(b = a\), so assume \(b > a\), so in particular all points of \([a, b]\) are limit points.
    We will use Riemann sums.
    The idea is to show that
    \[
        U(f, \mathbf{P}) \geq F(b) - F(a) \geq L(f, \mathbf{P})
    \]
    for every partition \(\mathbf{P}\) of \([a, b]\).
    The left inequality asserts that \(F(b) - F(a)\) is a lower bound for \(\{U(f, \mathbf{P}) : \mathbf{P} \text{ is a partition of } [a, b]\}\), while the right inequality asserts that \(F(b) - F(a)\) is an upper bound for \(\{L(f, \mathbf{P}) : \mathbf{P} \text{ is a partition of } [a, b]\}\).
    But by Proposition \ref{11.3.12}, this means that
    \[
        \overline{\int}_{[a, b]} f \geq F(b) - F(a) \geq \underline{\int}_{[a, b]} f,
    \]
    but since \(f\) is assumed to be Riemann integrable, both the upper and lower Riemann integral equal \(\int_{[a, b]} f\).
    The claim follows.

    We have to show the bound \(U(f, \mathbf{P}) \geq F(b) - F(a) \geq L(f, \mathbf{P})\).
    We shall just show the first inequality \(U(f, \mathbf{P}) \geq F(b) - F(a)\);
    the other inequality is similar.

    Let \(\mathbf{P}\) be a partition of \([a, b]\).
    From Lemma \ref{11.8.4} we have
    \[
        F(b) - F(a) = \sum_{J \in \mathbf{P}} F[J] = \sum_{J \in \mathbf{P} : J \neq \emptyset} F[J],
    \]
    while from definition we have
    \[
        U(f, \mathbf{P}) = \sum_{J \in \mathbf{P} : J \neq \emptyset} \sup_{x \in J} f(x) \abs*{J}.
    \]
    Thus it will suffice to show that
    \[
        F[J] \leq \sup_{x \in J} f(x) \abs*{J}
    \]
    for all \(J \in \mathbf{P}\)
    (other than the empty set).

    When \(J\) is a point then the claim is clear, since both sides are zero.
    Now suppose that \(J = [c, d], (c, d], [c, d)\), or \((c, d)\) for some \(c < d\).
    Then the left-hand side is \(F[J] = F(d) - F(c)\).
    Note that \(F\), being differentiable, is continuous, so we may use the simplified formula for the \(F\)-length as opposed to the more complicated one in Definition \ref{11.8.1}.
    By the mean-value theorem (Corollary \ref{10.2.9}), this is equal to \((d - c) F'(e)\) for some \(e \in J\).
    But since \(F'(e) = f(e)\), we thus have
    \[
        F[J] = (d - c) f(e) = f(e) \abs*{J} \leq \sup_{x \in J} f(x) \abs*{J}
    \]
    as desired.
\end{proof}

\begin{note}
    One can use the second fundamental theorem of calculus to compute integrals relatively easily provided that you can find an anti-derivative of the integrand \(f\).
    the first fundamental theorem of calculus ensures that every \emph{continuous} Riemann integrable function has an anti-derivative.
    For discontinuous functions, the situation is more complicated.
    Also, not every function with an anti-derivative is Riemann integrable.
\end{note}

\begin{lemma}\label{11.9.5}
    Let \(I\) be a bounded interval, and let \(f : I \to \mathbf{R}\) be a function.
    Let \(F : I \to \mathbf{R}\) and \(G : I \to \mathbf{R}\) be two antiderivatives of \(f\).
    Then there exists a real number \(C\) such that \(F(x) = G(x) + C\) for all \(x \in I\).
\end{lemma}

\begin{proof}
    Since \(I\) is a bounded interval, \(\forall\ x \in \overline{I}\), \(x\) is a limit point, thus \(\forall\ x \in I\), \(x\) is also a limit point.
    Since \(F, G\) are antiderivatives of \(f\), by Definition \ref{11.9.3} we know that \(F, G\) are differentiable on \(I\).
    By Theorem \ref{10.1.13}(f) we know that \(F - G\) is differentiable on \(I\), and thus by Corollary \ref{10.1.12} we know that \(F - G\) are continuous on \(I\).
    Let \(x, y \in I\) and \(x < y\).
    Since \(I\) is a bounded interval, by Lemma \ref{11.1.10} we know that \(I\) is connected, and thus by Definition \ref{11.1.1} \([x, y] \subseteq I\).
    By the mean-value theorem (Corollary \ref{10.2.9}) we know that
    \[
        \exists\ c \in I : \frac{(F - G)(x) - (F - G)(y)}{x - y} = (F - G)'(c).
    \]
    Thus we have
    \begin{align*}
                 & \frac{(F - G)(x) - (F - G)(y)}{x - y} = (F - G)'(c)                                          \\
        \implies & \frac{(F - G)(x) - (F - G)(y)}{x - y} = F'(c) - G'(c) & \text{(by Theorem \ref{10.1.13}(f))} \\
        \implies & \frac{(F - G)(x) - (F - G)(y)}{x - y} = 0             & \text{(by hypothesis)}               \\
        \implies & (F - G)(x) = (F - G)(y)                                                                      \\
        \implies & F(x) - G(x) = F(y) - G(y)                             & \text{(by Definition \ref{9.2.1})}   \\
        \implies & F(x) = G(x) + F(y) - G(y).
    \end{align*}
    By setting \(C = F(y) - G(y)\) we are done.
\end{proof}

\begin{proposition}\label{11.9.6}
    Let \(a < b\) be real numbers, let \(x \in [a, b]\) and let \(f : [a, b] \to \mathbf{R}\) be a monotone function.
    If \(f\) is monotone increasing, then we have
    \[
        f(x_0+) = \inf_{x \in [a, b] \cap (x_0, \infty)} f(x)
    \]
    and
    \[
        f(x_0-) = \sup_{x \in [a, b] \cap (-\infty, x_0)} f(x).
    \]
    If \(f\) is monotone decreasing, then we have
    \[
        f(x_0+) = \sup_{x \in [a, b] \cap (x_0, \infty)} f(x)
    \]
    and
    \[
        f(x_0-) = \inf_{x \in [a, b] \cap (-\infty, x_0)} f(x).
    \]
\end{proposition}

\begin{proof}
    First suppose that \(f\) is monotone increasing.
    Since \(f\) is monotone increasing, we have
    \[
        \sup_{x \in [a, b] \cap (x_0, \infty)} f(x) \leq \sup_{x \in [a, b]} f(x)
    \]
    and
    \[
        \inf_{x \in [a, b] \cap (x_0, \infty)} f(x) \geq \inf_{x \in [a, b]} f(x)
    \]
    and thus \(f(x_0+)\) and \(f(x_0-)\) are finite.
    Let \(U = \inf_{x \in [a, b] \cap (x_0, \infty)} f(x)\).
    By the definition of \(U\), \(\forall\ \varepsilon \in \mathbf{R}^+\), we can always find an \(x \in [a, b] \cap (x_0, \infty)\) such that \(0 \leq f(x) - U \leq \varepsilon\).
    We also know that for all \(y \in [a, b] \cap (x_0, \infty)\) and \(y \leq x\), we have \(0 \leq f(y) - U \leq f(x) - U \leq \varepsilon\).
    Thus by setting \(\delta = x - x_0\) we have
    \[
        \forall\ y \in [a, b] \cap (x_0, \infty), \abs*{y - x_0} < \delta \implies \abs*{f(y) - U} \leq \varepsilon
    \]
    and by Definition \ref{9.3.6} and Definition \ref{9.5.1} we have \(f(x_0+) = U\).
    Similarly let \(L = \sup_{x \in [a, b] \cap (x_0, \infty)} f(x)\).
    By the definition of \(L\), \(\forall\ \varepsilon \in \mathbf{R}^+\), we can always find an \(x \in [a, b] \cap (-\infty, x_0)\) such that \(0 \leq L - f(x) \leq \varepsilon\).
    We also know that for all \(y \in [a, b] \cap (-\infty, x_0)\) and \(y \geq x\), we have \(0 \leq L - f(y) \leq L - f(x) \leq \varepsilon\).
    Thus by setting \(\delta = x_0 - x\) we have
    \[
        \forall\ y \in [a, b] \cap (-\infty, x_0), \abs*{y - x_0} < \delta \implies \abs*{f(y) - L} \leq \varepsilon
    \]
    and by Definition \ref{9.3.6} and Definition \ref{9.5.1} we have \(f(x_0-) = L\).

    Now suppose that \(f\) is monotone decreasing.
    Since \(f\) is monotone decreasing, we have
    \[
        \sup_{x \in [a, b] \cap (x_0, \infty)} f(x) \leq \sup_{x \in [a, b]} f(x)
    \]
    and
    \[
        \inf_{x \in [a, b] \cap (x_0, \infty)} f(x) \geq \inf_{x \in [a, b]} f(x)
    \]
    and thus \(f(x_0+)\) and \(f(x_0-)\) are finite.
    Let \(U = \sup_{x \in [a, b] \cap (x_0, \infty)} f(x)\).
    By the definition of \(U\), \(\forall\ \varepsilon \in \mathbf{R}^+\), we can always find an \(x \in [a, b] \cap (x_0, \infty)\) such that \(0 \leq U - f(x) \leq \varepsilon\).
    We also know that for all \(y \in [a, b] \cap (x_0, \infty)\) and \(y \leq x\), we have \(0 \leq U - f(y) \leq U - f(x) \leq \varepsilon\).
    Thus by setting \(\delta = x - x_0\) we have
    \[
        \forall\ y \in [a, b] \cap (x_0, \infty), \abs*{y - x_0} < \delta \implies \abs*{f(y) - U} \leq \varepsilon
    \]
    and by Definition \ref{9.3.6} and Definition \ref{9.5.1} we have \(f(x_0+) = U\).
    Similarly let \(L = \inf_{x \in [a, b] \cap (x_0, \infty)} f(x)\).
    By the definition of \(L\), \(\forall\ \varepsilon \in \mathbf{R}^+\), we can always find an \(x \in [a, b] \cap (-\infty, x_0)\) such that \(0 \leq f(x) - L \leq \varepsilon\).
    We also know that for all \(y \in [a, b] \cap (-\infty, x_0)\) and \(y \geq x\), we have \(0 \leq f(y) - L \leq f(x) - L \leq \varepsilon\).
    Thus by setting \(\delta = x_0 - x\) we have
    \[
        \forall\ y \in [a, b] \cap (-\infty, x_0), \abs*{y - x_0} < \delta \implies \abs*{f(y) - L} \leq \varepsilon
    \]
    and by Definition \ref{9.3.6} and Definition \ref{9.5.1} we have \(f(x_0-) = L\).
\end{proof}

\exercisesection

\begin{exercise}\label{ex 11.9.1}
    Let \(f : [0, 1] \to \mathbf{R}\) be the function in Exercise \ref{ex 9.8.5}.
    Show that for every rational number \(q \in \mathbf{Q} \cap (0, 1)\), the function \(F : [0, 1] \to \mathbf{R}\) defined by the formula \(F(x) \coloneqq \int_{[0, x]} f\) is not differentiable at \(q\).
\end{exercise}

\begin{proof}
    By Exercise \ref{ex 9.8.5} we know that \(f\) is strictly monotone increasing, thus by Proposition \ref{11.6.1} we know that \(f\) is Riemann integrable and \(F\) is well-defined.
    Suppose for sake of contradiction that \(F\) is differentiable at \(q \in \mathbf{Q} \cap (0, 1)\).
    Then by Definition \ref{10.1.1} we have
    \[
        F'(q) = \lim_{x \to q ; x \in Q \cap (0, 1) \setminus \{q\}} \frac{F(x) - F(q)}{x - q} = \lim_{x \to q ; x \in Q \cap (0, 1) \setminus \{q\}} \frac{\int_{[0, x]} f - \int_{[0, q]} f}{x - q}.
    \]
    If \(x \in (0, q)\), then we have
    \[
        \frac{\int_{[0, x]} f - \int_{[0, q]} f}{x - q} = \frac{-\int_{[x, q]} f}{x - q} = \frac{\int_{[x, q]} f}{q - x} \leq \frac{\int_{[x, q]} f(q)}{q - x} = f(q).
    \]
    If \(x \in (q, 1)\), then we have
    \[
        \frac{\int_{[0, x]} f - \int_{[0, q]} f}{x - q} = \frac{\int_{[q, x]} f}{x - q} \geq \frac{\int_{[q, x]} f(q)}{x - q} = f(q).
    \]
    Thus by Definition \ref{9.5.1} we have
    \[
        F'(q-) = \lim_{x \to q^- ; x \in Q \cap (0, 1) \setminus \{q\}} \frac{\int_{[0, x]} f - \int_{[0, q]} f}{x - q} \leq f(q)
    \]
    and
    \[
        F'(q+) = \lim_{x \to q^+ ; x \in Q \cap (0, 1) \setminus \{q\}} \frac{\int_{[0, x]} f - \int_{[0, q]} f}{x - q} \geq f(q)
    \]
    Don't know where to go now.
\end{proof}

\begin{exercise}\label{ex 11.9.2}
    Prove Lemma \ref{11.9.5}.
\end{exercise}

\begin{proof}
    See Lemma \ref{11.9.5}.
\end{proof}

\begin{exercise}\label{ex 11.9.3}
    Let \(a < b\) be real numbers, and let \(f : [a, b] \to \mathbf{R}\) be a monotone increasing function.
    Let \(F : [a, b] \to \mathbf{R}\) be the function \(F(x) \coloneqq \int_{[a, x]} f\).
    Let \(x_0\) be an element of \((a, b)\).
    Show that \(F\) is differentiable at \(x_0\) if and only if \(f\) is continuous at \(x_0\).
\end{exercise}

\begin{proof}
    Since \(f\) is monotone increasing, by Proposition \ref{11.6.1} we know that \(f\) is Riemann integrable and \(F\) is well-defined.
    If \(f\) is continuous at \(x_0\) then by Theorem \ref{11.9.1} we know that \(F\) is differentiable at \(x_0\).
    So we only need to show that if \(F\) is differentiable at \(x_0\) then \(f\) is continuous at \(x_0\).

    Suppose that \(F\) is differentiable at \(x_0\).
    Suppose for sake of contradiction that \(f\) is not continuous at \(x_0\).
    Since \(f\) is monotone increasing, by Proposition \ref{11.9.6} we know that both \(f(x_0+)\) and \(f(x_0-)\) exist.
    Since \(f\) is not continuous at \(x_0\), we have \(f(x_0) \neq f(x_0-)\) or \(f(x_0) \neq f(x_0+)\).

    First suppose that \(f(x_0) \neq f(x_0-)\).
    Then by Proposition \ref{11.9.6} we have \(f(x_0-) < f(x_0)\).
    If \(x \in [a, b] \cap (-\infty, x_0)\), then we have
    \[
        \frac{F(x) - F(x_0)}{x - x_0} = \frac{-\int_{[x, x_0]} f}{x - x_0} = \frac{\int_{[x, x_0]} f}{x_0 - x} \leq \frac{\int_{[x, x_0]} f(x_0-)}{x_0 - x} = f(x_0-).
    \]
    If \(x \in [a, b] \cap (x_0, \infty)\), then we have
    \[
        \frac{F(x) - F(x_0)}{x - x_0} = \frac{\int_{[x_0, x]} f}{x - x_0} \geq \frac{\int_{[x_0, x]} f(x_0)}{x - x_0} = f(x_0).
    \]
    Thus we have
    \[
        F'(x_0-) \leq f(x_0-) < f(x_0) \leq F'(x_0+)
    \]
    and \(F'(x_0)\) does not exists, a contradiction.

    Now suppose that \(f(x_0) \neq f(x_0+)\).
    Then by Proposition \ref{11.9.6} we have \(f(x_0) < f(x_0+)\).
    If \(x \in [a, b] \cap (-\infty, x_0)\), then we have
    \[
        \frac{F(x) - F(x_0)}{x - x_0} = \frac{-\int_{[x, x_0]} f}{x - x_0} = \frac{\int_{[x, x_0]} f}{x_0 - x} \leq \frac{\int_{[x, x_0]} f(x_0)}{x_0 - x} = f(x_0).
    \]
    If \(x \in [a, b] \cap (x_0, \infty)\), then we have
    \[
        \frac{F(x) - F(x_0)}{x - x_0} = \frac{\int_{[x_0, x]} f}{x - x_0} \geq \frac{\int_{[x_0, x]} f(x_0+)}{x - x_0} = f(x_0+).
    \]
    Thus we have
    \[
        F'(x_0-) \leq f(x_0) < f(x_0+) \leq F'(x_0+)
    \]
    and \(F'(x_0)\) does not exists, a contradiction.
    So we conclude that \(f\) is continuous at \(x_0\).
\end{proof}
\section{Consequences of the fundamental theorems}\label{sec 11.10}

\begin{proposition}[Integration by parts formula]\label{11.10.1}
    Let \(I = [a, b]\), and let \(F : [a, b] \to \mathbf{R}\) and \(G : [a, b] \to \mathbf{R}\) be differentiable functions on \([a, b]\) such that \(F'\) and \(G'\) are Riemann integrable on \(I\).
    Then we have
    \[
        \int_{[a, b]} F G' = F(b) G(b) - F(a) G(a) - \int_{[a, b]} F' G.
    \]
\end{proposition}

\begin{proof}
    Since \(F\) is an antiderivative of \(F'\) and \(F'\) is Riemann integrable on \([a, b]\), by Theorem \ref{11.9.1} we know that \(F\) is continuous on \([a, b]\).
    Similarly \(G\) is continuous on \([a, b]\).
    By Corollarly \ref{11.5.2} we know that \(F\) and \(G\) are Riemann integrable on \([a, b]\).
    By Theorem \ref{11.4.5} we know that \(F G'\) and \(F' G\) are Riemann integrable on \([a, b]\).
    By Theorem \ref{10.1.13}(d) we have \((FG)' = F' G + F G'\).
    Thus by Theorem \ref{11.4.1}(a) \((FG)'\) is Riemann integrable on \([a, b]\) and
    \begin{align*}
        \int_{[a, b]} (F G') & = \int_{[a, b]} ((FG)' - F' G)                                                        \\
                             & = \int_{[a, b]} ((FG)') - \int_{[a, b]} (F' G)  & \text{(by Theorem \ref{11.4.1}(c))} \\
                             & = F(b) G(b) - F(a) G(a) - \int_{[a, b]} (F' G). & \text{(by Theorem \ref{11.9.4})}
    \end{align*}
\end{proof}

\begin{theorem}\label{11.10.2}
    Let \(\alpha : [a, b] \to \mathbf{R}\) be a monotone increasing function, and suppose that \(\alpha\) is also differentiable on \([a, b]\), with \(\alpha'\) being Riemann integrable.
    Let \(f : [a, b] \to \mathbf{R}\) be a piecewise constant function on \([a, b]\).
    Then \(f \alpha'\) is Riemann integrable on \([a, b]\), and
    \[
        \int_{[a, b]} f \; d \alpha = \int_{[a, b]} f \alpha'.
    \]
\end{theorem}

\begin{proof}
    Since \(f\) is piecewise constant, it is Riemann integrable, and since \(\alpha'\) is also Riemann integrable, then \(f \alpha'\) is Riemann integrable by Theorem \ref{11.4.5}.

    Suppose that \(f\) is piecewise constant with respect to some partition \(\mathbf{P}\) of \([a, b]\);
    without loss of generality we may assume that \(\mathbf{P}\) does not contain the empty set.
    Then we have
    \[
        \int_{[a, b]} f \; d \alpha = p.c. \int_{[\mathbf{P}]} f \; d \alpha = \sum_{J \in \mathbf{P}} c_J \alpha[J]
    \]
    where \(c_J\) is the constant value of \(f\) on \(J\).
    On the other hand, from Theorem \ref{11.4.1}(h) (and Exercise \ref{ex 11.4.3}) we have
    \[
        \int_{[a, b]} f \alpha' = \sum_{J \in \mathbf{P}} \int_J f \alpha' = \sum_{J \in \mathbf{P}} \int_J c_J \alpha' = \sum_{J \in \mathbf{P}} c_J \int_J \alpha'.
    \]
    But by the second fundamental theorem of calculus (Theorem \ref{11.9.4}), \(\int_J \alpha' = \alpha[J]\), and the claim follows.
\end{proof}

\begin{corollary}\label{11.10.3}
    Let \(\alpha : [a, b] \to \mathbf{R}\) be a monotone increasing function, and suppose that \(\alpha\) is also differentiable on \([a, b]\), with \(\alpha'\) being Riemann integrable.
    Let \(f : [a, b] \to \mathbf{R}\) be a function which is Riemann-Stieltjes integrable with respect to \(\alpha\) on \([a, b]\).
    Then \(f \alpha'\) is Riemann integrable on \([a, b]\), and
    \[
        \int_{[a, b]} f \; d \alpha = \int_{[a, b]} f \alpha'.
    \]
\end{corollary}

\begin{proof}
    Note that since \(f\) and \(\alpha'\) are bounded, then \(f \alpha'\) must also be bounded.
    Also, since \(\alpha\) is monotone increasing and differentable, \(\alpha'\) is non-negative (by Proposition \ref{10.3.1}).

    Let \(\varepsilon > 0\).
    Then, we can find a piecewise constant function \(\overline{f}\) majorizing \(f\) on \([a, b]\), and a piecewise constant function \(\underline{f}\) minorizing \(f\) on \([a, b]\), such that
    \[
        \int_{[a, b]} f \; d \alpha - \varepsilon \leq \int_{[a, b]} \underline{f} \; d \alpha \leq \int_{[a, b]} \overline{f} \; d \alpha \leq \int_{[a, b]} f \; d \alpha + \varepsilon.
    \]
    Applying Theorem \ref{11.10.2}, we obtain
    \[
        \int_{[a, b]} f \; d \alpha - \varepsilon \leq \int_{[a, b]} \underline{f} \alpha' \leq \int_{[a, b]} \overline{f} \alpha' \leq \int_{[a, b]} f \; d \alpha + \varepsilon.
    \]

    Since \(\alpha'\) is non-negative and \(\underline{f}\) minorizes \(f\), then \(\underline{f} \alpha'\) minorizes \(f \alpha'\).
    Thus \(\int_{[a, b]} \underline{f} \alpha' \leq \underline{\int}_{[a, b]} f \alpha'\).
    Thus
    \[
        \int_{[a, b]} f \; d \alpha - \varepsilon \leq \underline{\int}_{[a, b]} f \alpha'.
    \]
    Similarly we have
    \[
        \overline{\int}_{[a, b]} f \alpha' \leq \int_{[a, b]} f \; d \alpha + \varepsilon.
    \]
    Since these statements are true for any \(\varepsilon > 0\), we must have
    \[
        \int_{[a, b]} f \; d \alpha \leq \underline{\int}_{[a, b]} f \alpha' \leq \overline{\int}_{[a, b]} f \alpha' \leq \int_{[a, b]} f \; d \alpha
    \]
    and the claim follows.
\end{proof}