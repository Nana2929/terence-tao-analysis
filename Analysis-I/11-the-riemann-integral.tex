\chapter{The Riemann integral}\label{ch 11}

\begin{note}
    In Chapter \ref{ch 10} we reviewed \emph{differentiation} - one of the two pillars of single variable calculus.
    The other pillar is, of course, \emph{integration}, which is the focus of the current chapter.
    More precisely, we will turn to the \emph{definite integral}, the integral of a function on a fixed interval, as opposed to the \emph{indefinite integral}, otherwise known as the \emph{antiderivative}.
    These two are of course linked by the \emph{Fundamental theorem of calculus}.
\end{note}

\begin{note}
    To actually \emph{define} this integral \(\int_I f\) is somewhat delicate (especially if one does not want to assume any axioms concerning geometric notions such as area), and not all functions \(f\) are integrable.
    It turns out that there are at least two ways to define this integral:
    the \emph{Riemann integral}, named after Georg Riemann (1826 -- 1866), which suffices for most applications, and the \emph{Lebesgue integral}, named after Henri Lebesgue (1875 -- 1941), which supercedes the Riemann integral and works for a much larger class of functions.
    There is also the \emph{Riemann-Steiltjes integral} \(\int_I f(x) d \alpha(x)\), a generalization of the Riemann integral due to Thomas Stieltjes (1856 -- 1894).
\end{note}

\section{Partitions}\label{sec 11.1}

\begin{definition}\label{11.1.1}
    Let \(X\) be a subset of \(\mathbf{R}\).
    We say that \(X\) is \emph{connected} iff \(X\) is nonempty and the following property is true:
    whenever \(x, y\) are elements in \(X\) such that \(x < y\), the bounded interval \([x, y]\) is a subset of \(X\)
    (i.e., every number between \(x\) and \(y\) is also in \(X\)).
\end{definition}

\setcounter{theorem}{3}
\begin{lemma}\label{11.1.4}
    Let \(X\) be a subset of the real line.
    Then the following two statements are logically equivalent:
    \begin{enumerate}
        \item \(X\) is bounded and either connected or empty.
        \item \(X\) is a bounded interval.
    \end{enumerate}
\end{lemma}

\begin{proof}
    Both statements are logically equivalent when \(X = \emptyset\) (which is vacuously true).
    So suppose that \(X \neq \emptyset\).

    We first show that \(X\) is bounded and connected implies \(X\) is a bounded interval.
    Since \(X\) is bounded, by Theorem \ref{5.5.9} we know that \(\inf(X), \sup(X) \in \mathbf{R}\).
    Thus \(X \subseteq [\inf(X), \sup(X)]\).
    Now we split into four cases:
    \begin{itemize}
        \item If \(\sup(X) \in X\) and \(\inf(X) \in X\), then by Definition \ref{11.1.1} \(X\) is connected implies \([\inf(X), \sup(X)] \subseteq X\).
              Thus by Proposition \ref{3.1.18} we have \(X = [\inf(X), \sup(X)]\).
        \item If \(\sup(X) \in X\) and \(\inf(X) \notin X\), then we claim that \(\big(\inf(X), \sup(X)] \subseteq X\).
              This is true since \(X\) is connected and by Definition \ref{11.1.1} we have \(\big(a, \sup(X)] \subseteq X\) for every \(a \in X\).
        \item If \(\sup(X) \notin X\) and \(\inf(X) \in X\), then we claim that \([\inf(X), \sup(X)\big) \subseteq X\).
              This is true since \(X\) is connected and by Definition \ref{11.1.1} we have \([\inf(X), b\big) \subseteq X\) for every \(b \in X\).
        \item If \(\sup(X) \notin X\) and \(\inf(X) \notin X\), then we claim that \(\big(\inf(X), \sup(X)\big) \subseteq X\).
              This is true since \(X\) is connected and by Definition \ref{11.1.1} we have \((a, b) \subseteq X\) for every \(a, b \in X\) and \(a < b\).
    \end{itemize}
    From all cases above we conclude that \(X\) is a bounded interval.

    Now we show that \(X\) is a bounded interval implies \(X\) is bounded and connected.
    Obviously \(X\) is bounded.
    Let \(a, b \in \mathbf{R}\).
    Then \(X\) can be one of \((a, b), [a, b], (a, b], [a, b)\), and by Definition \ref{11.1.1} all of which are connected.
\end{proof}

\begin{remark}\label{11.1.5}
    Recall that intervals are allowed to be singleton points, or even the empty set.
\end{remark}

\begin{corollary}\label{11.1.6}
    If \(I\) and \(J\) are bounded intervals, then the intersection \(I \cap J\) is also a bounded interval.
\end{corollary}

\begin{proof}
    If \(I \cap J = \emptyset\), then \(I \cap J\) is bounded interval.
    So suppose that \(I \cap J \neq \emptyset\).
    Since \(I, J\) are bounded intervals, by Lemma \ref{11.1.4} we know that \(I, J\) are bounded and connected.
    Since \(I, J\) are bounded, \(\exists\ M_1, M_2 \in \mathbf{R}\) such that \(I \subseteq [-M_1, M_1]\) and \(J \subseteq [-M_2, M_2]\).
    Let \(M = \min(M_1, M_2)\).
    Then we have \(I \cap J \subseteq [-M, M]\) and thus \(I \cap J\) is bounded.
    Let \(x, y \in I \cap J\) and \(x < y\).
    Since \(I\) is connected and \(I \cap J \subseteq I\), we have \([x, y] \subseteq I\).
    Similarly since \(J\) is connected and \(I \cap J \subseteq J\), we have \([x, y] \subseteq J\).
    Thus \([x, y] \subseteq I \cap J\) and by Definition \ref{11.1.1} \(I \cap J\) is connected.
    Since \(I \cap J\) is bounded and connected, by Lemma \ref{11.1.4} \(I \cap J\) is bounded interval.
\end{proof}

\setcounter{theorem}{7}
\begin{definition}[Length of intervals]\label{11.1.8}
    If \(I\) is a bounded interval, we define the \emph{length} of \(I\), denoted \(\abs*{I}\) as follows.
    If \(I\) is one of the intervals \([a, b]\), \((a, b)\), \([a, b)\), or \((a, b]\) for some real numbers \(a < b\), then we define \(\abs*{I} \coloneqq b - a\).
    Otherwise, if \(I\) is a point or the empty set, we define \(\abs*{I} = 0\).
\end{definition}

\setcounter{theorem}{9}
\begin{definition}[Partitions]\label{11.1.10}
    Let \(I\) be a bounded interval.
    A \emph{partition} of \(I\) is a finite set \(\mathbf{P}\) of bounded intervals contained in \(I\), such that every \(x\) in \(I\) lies in exactly one of the bounded intervals \(J\) in \(\mathbf{P}\).
\end{definition}

\begin{remark}\label{11.1.11}
    Note that a partition is a set of intervals, while each interval is itself a set of real numbers.
    Thus a partition is a set consisting of other sets.
\end{remark}

\setcounter{theorem}{12}
\begin{theorem}[Length is finitely additive]\label{11.1.13}
    Let \(I\) be a bounded interval, \(n\) be a natural number, and let \(\mathbf{P}\) be a partition of \(I\) of cardinality \(n\).
    Then
    \[
        \abs*{I} = \sum_{J \in \mathbf{P}} \abs*{J}.
    \]
\end{theorem}

\begin{proof}
    We prove this by induction on \(n\).
    More precisely, we let \(P(n)\) be the property that whenever \(I\) is a bounded interval, and whenever \(\mathbf{P}\) is a partition of \(I\) with cardinality \(n\), that \(\abs*{I} = \sum_{J \in \mathbf{P}} \abs*{J}\).

    The base case \(P(0)\) is trivial;
    the only way that \(I\) can be partitioned into an empty partition is if \(I\) is itself empty, at which point the claim is easy.
    The case \(P(1)\) is also very easy;
    the only way that \(I\) can be partitioned into a singleton set \(\{J\}\) is if \(J = I\), at which point the claim is again very easy.

    Now suppose inductively that \(P(n)\) is true for some \(n \geq 1\), and now we prove \(P(n + 1)\).
    Let \(I\) be a bounded interval, and let \(\mathbf{P}\) be a partition of \(I\) of cardinality \(n + 1\).

    If \(I\) is the empty set or a point, then all the intervals in \(\mathbf{P}\) must also be either the empty set or a point, and so every interval has length zero and the claim is trivial.
    Thus we will assume that \(I\) is an interval of the form \((a, b)\), \((a, b]\), \([a, b)\), or \([a, b]\).

            Let us first suppose that \(b \in I\), i.e., \(I\) is either \((a, b]\) or \([a, b]\).
    Since \(b \in I\), we know that one of the intervals \(K\) in \(\mathbf{P}\) contains \(b\).
    Since \(K\) is contained in \(I\), it must therefore be of the form \((c, b]\), \([c, b]\), or \(\{b\}\) for some real number \(c\), with \(a \leq c \leq b\) (in the latter case of \(K = \{b\}\), we set \(c \coloneqq b\)).
    In particular, this means that the set \(I \setminus K\) is also an interval of the form \([a, c]\), \((a, c)\), \((a, c]\), \([a, c)\) when \(c > a\), or a point or empty set when \(a = c\).
    Either way, we easily see that
    \[
        \abs*{I} = \abs*{K} + \abs*{I \setminus K}.
    \]
    On the other hand, since \(\mathbf{P}\) forms a partition of \(I\), we see that \(\mathbf{P} \setminus \{K\}\) forms a partition of \(I \setminus K\).
    By the induction hypothesis, we thus have
    \[
        \abs*{I \setminus K} = \sum_{J \in \mathbf{P} \setminus \{K\}} \abs*{J}.
    \]
    Combining these two identities (and using the laws of addition for finite sets, see Proposition \ref{7.1.11}(e)) we obtain
    \[
        \abs*{I} = \sum_{J \in \mathbf{P}} \abs*{J}
    \]
    as desired.

    Now suppose that \(b \notin I\), i.e., \(I\) is either \((a, b)\) or \([a, b)\).
    Then one of the intervals \(K\) also is of the form \((c, b)\) or \([c, b)\) (see Exercise \ref{ex 11.1.3}).
            In particular, this means that the set \(I \setminus K\) is also an interval of the form \([a, c]\), \((a, c)\), \((a, c]\), \([a, c)\) when \(c > a\), or a point or empty set when \(a = c\).
    The rest of the argument then proceeds as above.
\end{proof}

\begin{definition}[Finer and coarser partitions]\label{11.1.14}
    Let \(I\) be a bounded interval, and let \(\mathbf{P}\) and \(\mathbf{P}'\) be two partitions of \(I\).
    We say that \(\mathbf{P}'\) is \emph{finer} than \(\mathbf{P}\) (or equivalently, that \(\mathbf{P}\) is \emph{coarser} than \(\mathbf{P}'\)) if for every \(J\) in \(\mathbf{P}'\), there exists a \(K\) in \(\mathbf{P}\) such that \(J \subseteq K\).
\end{definition}

\begin{note}
    There is no such thing as a ``finest'' partition of some interval \(I\).
    (recall all partitions are assumed to be finite.)
    We do not compare partitions of different intervals.
\end{note}

\setcounter{theorem}{15}
\begin{definition}[Common refinement]\label{11.1.16}
    Let \(I\) be a bounded interval, and let \(\mathbf{P}\) and \(\mathbf{P}'\) be two partitions of \(I\).
    We define the \emph{common refinement} \(\mathbf{P} \# \mathbf{P}'\) of \(\mathbf{P}\) and \(\mathbf{P}'\) to be the set
    \[
        \mathbf{P} \# \mathbf{P}' \coloneqq \{K \cap J : K \in \mathbf{P} \land J \in \mathbf{P}'\}.
    \]
\end{definition}

\setcounter{theorem}{17}
\begin{lemma}\label{11.1.18}
    Let \(I\) be a bounded interval, and let \(\mathbf{P}\) and \(\mathbf{P}'\) be two partitions of \(I\).
    Then \(\mathbf{P} \# \mathbf{P}'\) is also a partition of \(I\), and is both finer than \(\mathbf{P}\) and finer than \(\mathbf{P}'\).
\end{lemma}

\begin{proof}
    Let \(x \in I\).
    Then by Definition \ref{11.1.10} we know that \(\exists!\ K \in \mathbf{P}\) such that \(x \in K\).
    Similarly \(\exists!\ J \in \mathbf{P}'\) such that \(x \in J\), thus \(x \in K \cap J\).
    By Definition \ref{11.1.16} we know that \(K \cap J \in \mathbf{P} \# \mathbf{P}'\).
    Thus we have
    \[
        I \subseteq \bigcup \big(\mathbf{P} \# \mathbf{P}'\big).
    \]

    Let \(S \in \mathbf{P} \# \mathbf{P}'\).
    By Definition \ref{11.1.16} we know that \(\exists\ K \in \mathbf{P}\) and \(\exists\ J \in \mathbf{P}'\) such that \(S = K \cap J\).
    Since \(S = K \cap J\), we have \(S \subseteq I\), thus
    \[
        \bigcup \big(\mathbf{P} \# \mathbf{P}'\big) \subseteq I.
    \]
    Combining the result above we have
    \[
        I = \bigcup \big(\mathbf{P} \# \mathbf{P}'\big).
    \]
    Since \(K, J\) are bounded interval, by Corollary \ref{11.1.6} we know that \(S = K \cap J\) is a bounded interval.

    We now claim that \(\forall\ x \in I\), \(\exists!\ S \in \mathbf{P} \# \mathbf{P}'\) such that \(x \in S\).
    So suppose for sake of contradiction that \(\exists\ S_1, S_2 \in \mathbf{P} \# \mathbf{P}'\) such that \(S_1 \neq S_2\), \(x \in S_1\) and \(x \in S_2\).
    By Definition \ref{11.1.16} we know that \(\exists\ K_1 \in \mathbf{P}\) and \(\exists\ J_1 \in \mathbf{P}'\) such that \(S_1 = K_1 \cap J_1\).
    Similarly \(\exists\ K_2 \in \mathbf{P}\) and \(\exists\ J_2 \in \mathbf{P}'\) such that \(S_2 = K_2 \cap J_2\).
    Since \(x \in S_1\), \(x \in K_1\).
    Since \(x \in S_2\), \(x \in K_2\).
    But by Definition \ref{11.1.10} we know that \(K_1 = K_2\), similar argument holds for \(J_1 = J_2\).
    Thus we must have \(S_1 = S_2\), a contradiction.

    We already show that \(\mathbf{P} \# \mathbf{P}' = I\);
    \(\forall\ S \in \mathbf{P} \# \mathbf{P}'\), \(S\) is a bounded interval;
    \(\forall\ x \in I\), \(\exists!\ S \in \mathbf{P} \# \mathbf{P}'\) such that \(x \in I\).
    To show that \(\mathbf{P} \# \mathbf{P}'\) is a partition of \(I\), by Definition \ref{11.1.10} we also need to show that \(\mathbf{P} \# \mathbf{P}'\) is a finite set.
    Let \(f : \mathbf{P} \times \mathbf{P}' \to \mathbf{P} \# \mathbf{P}'\) be a function where
    \[
        \forall\ K \in \mathbf{P} \land \forall\ J \in \mathbf{P}', f(K, J) = K \cap J.
    \]
    From the proof above we can see that \(f\) is surjective.
    By Definition \ref{11.1.10} we have \(\#(\mathbf{P}), \#(\mathbf{P}')\) are finite.
    Thus by Proposition \ref{3.6.14}(e) and Exercise \ref{ex 8.4.3} we have
    \[
        \#(\mathbf{P} \times \mathbf{P}') = \#(\mathbf{P}) \times \#(\mathbf{P}') \geq \#(\mathbf{P} \# \mathbf{P}')
    \]
    and \(\mathbf{P} \# \mathbf{P}'\) is finite, and we conclude that \(\mathbf{P} \# \mathbf{P}'\) is a partition of \(I\).

    From the proof above we have \(\forall\ S \in \mathbf{P} \# \mathbf{P}'\), \(\exists!\ K \in \mathbf{P}\) and \(\exists!\ J \in \mathbf{P}'\) such that \(S = K \cap J\).
    Then we have \(S \subseteq K\) and \(S \subseteq J\).
    Thus by Definition \ref{11.1.14} \(\mathbf{P} \# \mathbf{P}'\) is both finer than \(\mathbf{P}\) and finer than \(\mathbf{P}'\)
\end{proof}

\exercisesection

\begin{exercise}\label{ex 11.1.1}
    Prove Lemma \ref{11.1.4}.
\end{exercise}

\begin{proof}
    See Lemma \ref{11.1.4}.
\end{proof}

\begin{exercise}\label{ex 11.1.2}
    Prove Corollary \ref{11.1.6}.
\end{exercise}

\begin{proof}
    Prove Corollary \ref{11.1.6}.
\end{proof}

\begin{exercise}\label{ex 11.1.3}
    Let \(I\) be a bounded interval of the form \(I = (a, b)\) or \(I = [a, b)\) for some real numbers \(a < b\).
    Let \(I_1, \dots, I_n\) be a partition of \(I\).
    Prove that one of the intervals \(I_j\) in this partition is of the form \(I_j = (c, b)\) or \(I_j = [c, b)\) for some \(a \leq c \leq b\).
\end{exercise}

\begin{proof}
    Let \(\mathbf{P} = \{I_1, \dots, I_n\}\).
    If \(c = b\), then \((c, b) = \emptyset\), and thus by Definition \ref{11.1.10} \(\mathbf{P} \cup \emptyset\) is a partition of \(I\).
    So we only need to proof the cases where \(a \leq c < b\).
    Suppose for sake of contradiction that every interval \(I_j\) in the partition is not of the form \((c, b)\) or \([c, b)\).
    By Definition \ref{11.1.10} this means \(\forall\ x \in I_j\), we have \(x \geq b\) or \(x < c\).
    Since \(I = (a, b)\) or \(I = [a, b)\), we know that \(x < b\), thus we must have \(x < c\).
    This means \(\sup(I_j) \leq c < b\).
    But we know that \(\sup(I) = b\).
    So we have \(\forall\ x \in I\), \(\exists!\ I_j \in \mathbf{P}\) such that \(x \in I_j\), and \(x \leq \sup(I_j) < b = \sup(I)\).
    Thus \(\sup(I) \neq b\), a contradiction.
\end{proof}

\begin{exercise}\label{ex 11.1.4}
    Prove Lemma \ref{11.1.18}.
\end{exercise}

\begin{proof}
    Prove Lemma \ref{11.1.18}.
\end{proof}
\section{Piecewise constant functions}\label{sec 11.2}

\begin{definition}[Constant functions]\label{11.2.1}
    Let \(X\) be a subset of \(\mathbf{R}\), and let \(f : X \to \mathbf{R}\) be a function.
    We say that \(f\) is \emph{constant} iff there exists a real number \(c\) such that \(f(x) = c\) for all \(x \in X\).
    If \(E\) is a subset of \(X\), we say that \(f\) is \emph{constant on} \(E\) if the restriction \(f|_E\) of \(f\) to \(E\) is constant, in other words there exists a real number \(c\) such that \(f(x) = c\) for all \(x \in E\).
    We refer to \(c\) as the \emph{constant value} of \(f\) on \(E\).
\end{definition}

\begin{remark}\label{11.2.2}
    If \(E\) is a non-empty set, then a function \(f\) which is constant on \(E\) can have only one constant value;
    However, if \(E\) is empty, every real number \(c\) is a constant value for \(f\) on \(E\).
\end{remark}

\begin{definition}[Piecewise constant functions I]\label{11.2.3}
    Let \(I\) be a bounded interval, let \(f : I \to \mathbf{R}\) be a function, and let \(\mathbf{P}\) be a partition of \(I\).
    We say that \(f\) is \emph{piecewise constant with respect to \(\mathbf{P}\)} if for every \(J \in \mathbf{P}\), \(f\) is constant on \(J\).
\end{definition}

\setcounter{theorem}{4}
\begin{definition}[Piecewise constant functions II]\label{11.2.5}
    Let \(I\) be a bounded interval, and let \(f : I \to \mathbf{R}\) be a function.
    We say that \(f\) is \emph{piecewise constant on \(I\)} if there exists a partition \(\mathbf{P}\) of \(I\) such that \(f\) is piecewise constant with respect to \(\mathbf{P}\).
\end{definition}

\setcounter{theorem}{6}
\begin{lemma}\label{11.2.7}
    Let \(I\) be a bounded interval, let \(\mathbf{P}\) be a partition of \(I\), and let \(f : I \to \mathbf{R}\) be a function which is piecewise constant with respect to \(\mathbf{P}\).
    Let \(\mathbf{P}'\) be a partition of \(I\) which is finer than \(\mathbf{P}\).
    Then \(f\) is also piecewise constant with respect to \(\mathbf{P}'\).
\end{lemma}

\begin{proof}
    Let \(K \in \mathbf{P}'\).
    Since \(\mathbf{P}'\) is finer than \(\mathbf{P}\), by Definition \ref{11.1.14} \(\exists\ J \in \mathbf{P}\) such that \(K \subseteq J\).
    Since \(f\) is piecewise constant with respect to \(\mathbf{P}\), by Definition \ref{11.2.3} we know that \(\forall\ x \in J\), \(f(x)\) is constant.
    Thus \(\forall\ x \in K\), \(x \in J\) and \(f(x)\) is constant.
    By Definition \ref{11.2.3} \(f\) is piecewise constant with respect to \(\mathbf{P}'\).
\end{proof}

\begin{lemma}\label{11.2.8}
    Let \(I\) be a bounded interval, and let \(f : I \to \mathbf{R}\) and \(g : I \to \mathbf{R}\) be piecewise constant functions on \(I\).
    Then the functions \(f + g\), \(f - g\), \(\max(f, g)\), \(\min(f, g)\) and \(fg\) are also piecewise constant functions on \(I\).
    Here of course \(\max(f, g) : I \to \mathbf{R}\) is the function \(\max(f, g)(x) \coloneqq \max(f(x), g(x))\).
    If \(g\) does not vanish anywhere on \(I\) (i.e., \(g(x) \neq 0\) for all \(x \in I\)) then \(f / g\) is also a piecewise constant function on \(I\).
\end{lemma}

\begin{proof}
    Since \(f\) is piecewise constant function on \(I\), by Definition \ref{11.2.5} \(\exists\ \mathbf{P}\) such that \(\mathbf{P}\) is a partition of \(I\) and \(f\) is piecewise constant with respect to \(\mathbf{P}\).
    Similarly since \(g\) is piecewise constant function on \(I\), by Definition \ref{11.2.5} \(\exists\ \mathbf{P}'\) such that \(\mathbf{P}'\) is a partition of \(I\) and \(g\) is piecewise constant with respect to \(\mathbf{P}'\).
    By Lemma \ref{11.1.18} we know that \(\mathbf{P} \# \mathbf{P}'\) is also a partition of \(I\) and \(\mathbf{P} \# \mathbf{P}'\) is both finer than \(\mathbf{P}\) and finer than \(\mathbf{P}'\).
    By Lemma \ref{11.2.7} we know that both \(f\) and \(g\) are piecewise constant with respect to \(\mathbf{P} \# \mathbf{P}'\).

    Now we show that \(f, g\) remain piecewise constant functions on \(I\) after algebraic operation.
    Since \(\forall\ J \in \mathbf{P} \# \mathbf{P}'\), we have \(\forall\ x \in J\), \(f(x)\) is constant and \(g(x)\) is constant.
    Thus we know that \(f(x) + g(x)\), \(f(x) - g(x)\), \(\max(f(x), g(x))\), \(\min(f(x), g(x))\) and \(f(x) g(x)\) are constant.
    If \(g(x) \neq 0\), then we also have \(f(x) / g(x)\) is constant.
    Thus by Definition \ref{11.2.3} \(f + g\), \(f - g\), \(\max(f, g)\), \(\min(f, g)\), \(fg\) is piecewise constant with respect to \(\mathbf{P} \# \mathbf{P}'\), and when \(g(x) \neq 0\) we have \(f / g\) is piecewise constant with respect to \(\mathbf{P} \# \mathbf{P}'\).
    By Definition \ref{11.2.5} \(f + g\), \(f - g\), \(\max(f, g)\), \(\min(f, g)\), \(fg\) is piecewise constant on \(I\), and when \(g(x) \neq 0\) we have \(f / g\) is piecewise constant on \(I\).
\end{proof}

\begin{definition}[Piecewise constant integral I]\label{11.2.9}
    Let \(I\) be a bounded interval, let \(\mathbf{P}\) be a partition of \(I\).
    Let \(f : I \to \mathbf{R}\) be a function which is piecewise constant with respect to \(\mathbf{P}\).
    Then we define the \emph{piecewise constant integral} \(p.c. \int_{[\mathbf{P}]} f\) of \(f\) with respect to the partition \(\mathbf{P}\) by the formula
    \[
        p.c. \int_{[\mathbf{P}]} f \coloneqq \sum_{J \in \mathbf{P}} c_J \abs*{J},
    \]
    where for each \(J\) in \(\mathbf{P}\), we let \(c_J\) be the constant value of \(f\) on \(J\).
\end{definition}

\begin{remark}\label{11.2.10}
    This definition seems like it could be ill-defined, because if \(J\) is empty then every number \(c_J\) can be the constant value of \(f\) on \(J\), but fortunately in such cases \(\abs*{J}\) is zero and so the choice of \(c_J\) is irrelevant.
    The notation \(p.c. \int_{[\mathbf{P}]} f\) is rather artificial, but we shall only need it temporarily, en route to a more useful definition.
    Note that since \(\mathbf{P}\) is finite, the sum \(\sum_{J \in \mathbf{P}} c_J \abs*{J}\) is always well-defined
    (it is never divergent or infinite).
\end{remark}

\begin{remark}\label{11.2.11}
    The piecewise constant integral corresponds intuitively to one's notion of area, given that the area of a rectangle ought to be the product of the lengths of the sides.
    (Of course, if \(f\) is negative somewhere, then the ``area'' \(c_J \abs*{J}\) would also be negative.)
\end{remark}

\setcounter{theorem}{12}
\begin{proposition}[Piecewise constant integral is independent of partition]\label{11.2.13}
    Let \(I\) be a bounded interval, and let \(f : I \to \mathbf{R}\) be a function.
    Suppose that \(\mathbf{P}\) and \(\mathbf{P}'\) are partitions of \(I\) such that \(f\) is piecewise constant both with respect to \(\mathbf{P}\) and with respect to \(\mathbf{P}'\).
    Then \(p.c. \int_{[\mathbf{P}]} f = p.c. \int_{[\mathbf{P}']} f\).
\end{proposition}

\begin{proof}
    By Lemma \ref{11.1.18} we know that \(\mathbf{P} \# \mathbf{P}'\) is a partition of \(I\) and is both finer than \(\mathbf{P}\) and finer than \(\mathbf{P}'\), thus by Definition \ref{11.2.9} we have
    \[
        p.c. \int_{[\mathbf{P} \# \mathbf{P}']} f = \sum_{J \in \mathbf{P} \# \mathbf{P}'} c_J \abs*{J}.
    \]
    By Theorem \ref{11.1.13}, we know that
    \[
        \abs*{I} = \sum_{J \in \mathbf{P}} \abs*{J} = \sum_{J \in \mathbf{P} \# \mathbf{P}'} \abs*{J}.
    \]
    By Definition \ref{11.1.14}, \(\forall\ S \in \mathbf{P} \# \mathbf{P}'\), \(\exists\ K \in \mathbf{P}\) such that \(S \subseteq K\).
    Now we fix such \(K\) and let \(\mathbf{P}_K\) be the set
    \[
        \mathbf{P}_K = \{S \in \mathbf{P} \# \mathbf{P}' : S \subseteq K\}.
    \]
    We claim that \(\mathbf{P}_K\) is a partition of \(K\).
    We know that \(\mathbf{P}_K\) is finite since \(\mathbf{P}_K \subseteq \mathbf{P} \# \mathbf{P}'\) and \(\mathbf{P} \# \mathbf{P}'\) is finite.
    Since \(\mathbf{P} \# \mathbf{P}'\) is a partition of \(I\), by Definition \ref{11.1.10} \(\forall\ S \in \mathbf{P}_K\), \(S\) is a bounded interval, and if \(S' \in \mathbf{P}_K\) and \(S \neq S'\) then \(S \cap S' = \emptyset\).
    Let \(x \in K\).
    By Definition \ref{11.1.10} we must have \(x \in I\), and \(\exists!\ K' \in \mathbf{P}'\) such that \(x \in K'\).
    By Definition \ref{11.1.16} we know that \(K \cap K' \in \mathbf{P} \# \mathbf{P}'\).
    Since \(K \cap K' \subseteq K\), we have \(x \in \bigcup \mathbf{P}_K\), so \(K \subseteq \bigcup \mathbf{P}_K\).
    By the definition of \(\mathbf{P}_K\) we know that \(\bigcup \mathbf{P}_K \subseteq K\), thus by Proposition \ref{3.1.18} we have \(K = \bigcup \mathbf{P}_K\) and by Definition \ref{11.1.10} \(\mathbf{P}_K\) is a partition of \(K\).

    Now we show that \(\bigcup_{K \in \mathbf{P}} \mathbf{P}_K = \mathbf{P} \# \mathbf{P}'\).
    By the definition of \(\mathbf{P}_K\) we have \(\bigcup_{K \in \mathbf{P}} \mathbf{P}_K \subseteq \mathbf{P} \# \mathbf{P}'\).
    Let \(S \in \mathbf{P} \# \mathbf{P}'\).
    Since \(\mathbf{P} \# \mathbf{P}'\) is finer than \(\mathbf{P}\), by Definition \ref{11.1.14} \(\exists\ K \in \mathbf{P}\) such that \(S \subseteq K\).
    Thus \(S \in \mathbf{P}_K\) and \(\mathbf{P} \# \mathbf{P}' \subseteq \bigcup_{K \in \mathbf{P}} \mathbf{P}_K\).
    Again by Proposition \ref{3.1.18} we have \(\bigcup_{K \in \mathbf{P}} \mathbf{P}_K = \mathbf{P} \# \mathbf{P}'\).

    Since \(f\) is piecewise constant with respect to \(\mathbf{P}\), by Lemma \ref{11.2.7} we know that \(f\) is piecewise constant with respect to \(\mathbf{P} \# \mathbf{P}'\).
    So we have
    \begin{align*}
        p.c. \int_{[\mathbf{P} \# \mathbf{P}']} f & = \sum_{J \in \mathbf{P} \# \mathbf{P}'} c_J \abs*{J}                        & \text{(by Definition \ref{11.2.9})}     \\
                                                  & = \sum_{J \in \bigcup_{K \in \mathbf{P}} \mathbf{P}_K} c_J \abs*{J}                                                    \\
                                                  & = \sum_{K \in \mathbf{P}} \sum_{J \in \mathbf{P}_K} c_J \abs*{J}             & \text{(by Proposition \ref{7.1.11}(e))} \\
                                                  & = \sum_{K \in \mathbf{P}} \sum_{J \in \mathbf{P}_K} c_K \abs*{J}             & (J \subseteq K)                         \\
                                                  & = \sum_{K \in \mathbf{P}} c_K \bigg(\sum_{J \in \mathbf{P}_K} \abs*{J}\bigg)                                           \\
                                                  & = \sum_{K \in \mathbf{P}} c_K \abs*{K}                                       & \text{(by Theorem \ref{11.1.13})}       \\
                                                  & = p.c. \int_{[\mathbf{P}]} f.                                                & \text{(by Definition \ref{11.2.9})}
    \end{align*}
    Using similar arguments we can show that \(p.c. \int_{[\mathbf{P}']} f = p.c. \int_{[\mathbf{P} \# \mathbf{P}']} f\).
    Thus we have \(p.c. \int_{[\mathbf{P}]} f = p.c. \int_{[\mathbf{P}']} f\).
\end{proof}

\begin{definition}[Piecewise constant integral II]\label{11.2.14}
    Let \(I\) be a bounded interval, and let \(f : I \to \mathbf{R}\) be a piecewise constant function on \(I\).
    We define the \emph{piecewise constant integral} \(p.c. \int_I f\) by the formula
    \[
        p.c. \int_I f \coloneqq p.c. \int_{[\mathbf{P}]} f,
    \]
    where \(\mathbf{P}\) is any partition of \(I\) with respect to which \(f\) is piecewise constant.
    (Note that Proposition \ref{11.2.13} tells us that the precise choice of this partition is irrelevant.)
\end{definition}

\setcounter{theorem}{15}
\begin{theorem}[Laws of integration]\label{11.2.16}
    Let \(I\) be a bounded interval, and let \(f : I \to \mathbf{R}\) and \(g : I \to \mathbf{R}\) be piecewise constant functions on \(I\).
    \begin{enumerate}
        \item We have \(p.c. \int_I (f + g) = p.c. \int_I f + p.c. \int_I g\).
        \item For any real number \(c\), we have \(p.c. \int_I (cf) = c (p.c. \int_I f)\).
        \item We have \(p.c. \int_I (f - g) = p.c. \int_I f - p.c. \int_I g\).
        \item If \(f(x) \geq 0\) for all \(x \in I\), then \(p.c. \int_I f \geq 0\).
        \item If \(f(x) \geq g(x)\) for all \(x \in I\), then \(p.c. \int_I f \geq p.c. \int_I g\).
        \item If \(f\) is the constant function \(f(x) = c\) for all \(x \in I\), then \(p.c. \int_I f = c \abs*{I}\).
        \item Let \(J\) be a bounded interval containing \(I\) (i.e., \(I \subseteq J\)), and let \(F : J \to \mathbf{R}\) be the function
              \[
                  F(x) \coloneqq \begin{cases}
                      f(x) & \text{if } x \in I    \\
                      0    & \text{if } x \notin I
                  \end{cases}
              \]
              Then \(F\) is piecewise constant on \(J\), and \(p.c. \int_I F = p.c. \int_I f\).
        \item Suppose that \(\{J, K\}\) is a partition of \(I\) into two intervals \(J\) and \(K\).
              Then the function \(f|_J : J \to \mathbf{R}\) and \(f|_K : K \to \mathbf{R}\) are piecewise constant on \(J\) and \(K\) respectively, and we have
              \[
                  p.c. \int_I f = p.c. \int_I f|_J + p.c. \int_I f|_K.
              \]
    \end{enumerate}
\end{theorem}

\begin{proof}{(a)}
    Since \(f, g\) are both piecewise constant on \(I\), by Lemma \ref{11.2.8} \(f + g\) is also piecewise constant on \(I\).
    By Definition \ref{11.2.3}, \(\exists\ \mathbf{P}_f, \mathbf{P}_g\) such that \(\mathbf{P}_f, \mathbf{P}_g\) are partitions of \(I\), \(f\) is piecewise constant with respect to \(\mathbf{P}_f\) and \(g\) is piecewise constant with respect to \(\mathbf{P}_g\).
    Let \(\mathbf{P} = \mathbf{P}_f \# \mathbf{P}_g\).
    Then by Lemma \ref{11.1.18} we know that \(\mathbf{P}\) is also a partition of \(I\) and by Lemma \ref{11.2.7} \(f, g\) are piecewise constant with respect to \(\mathbf{P}\).
    Now let \(J \in \mathbf{P}\), let \(f_J \in \mathbf{R}\) be the constant value of \(f\) on \(J\) and \(g_J \in \mathbf{R}\) be the constant value of \(g\) on \(J\).
    Then by Definition \ref{11.2.1} \(f_J + g_J\) is also a constant of \(f + g\) on \(J\).
    Thus we have \(f + g\) is piecewise constant with respect to \(\mathbf{P}\) and
    \begin{align*}
        p.c. \int_I f + p.c. \int_I g & = p.c. \int_{[\mathbf{P}]} f + p.c. \int_{[\mathbf{P}]} g                     & \text{(by Definition \ref{11.2.14})}    \\
                                      & = \sum_{J \in \mathbf{P}} f_J \abs*{J} + \sum_{J \in \mathbf{P}} g_J \abs*{J} & \text{(by Definition \ref{11.2.9})}     \\
                                      & = \sum_{J \in \mathbf{P}} (f_J + g_J) \abs*{J}                                & \text{(by Proposition \ref{7.1.11}(f))} \\
                                      & = p.c. \int_{[\mathbf{P}]} (f_J + g_J)                                        & \text{(by Definition \ref{11.2.9})}     \\
                                      & = p.c. \int_I (f_J + g_J).                                                    & \text{(by Definition \ref{11.2.14})}
    \end{align*}
\end{proof}

\begin{proof}{(b)}
    Since \(f\) is piecewise constant on \(I\), by Lemma \ref{11.2.8} \(cf\) is also piecewise constant on \(I\) (since \(c\) is constant on \(I\)).
    By Definition \ref{11.2.3}, \(\exists\ \mathbf{P}\) such that \(\mathbf{P}\) is a partition of \(I\) and \(f\) is piecewise constant with respect to \(\mathbf{P}\).
    Now let \(J \in \mathbf{P}\) and let \(f_J \in \mathbf{R}\) be the constant value of \(f\) on \(J\).
    Then by Definition \ref{11.2.1} \(c f_J\) is also a constant of \(cf\) on \(J\).
    Thus we have \(cf\) is piecewise constant with respect to \(\mathbf{P}\) and
    \begin{align*}
        c (p.c. \int_I f) & = c (p.c. \int_{[\mathbf{P}]} f)           & \text{(by Definition \ref{11.2.14})}    \\
                          & = c (\sum_{J \in \mathbf{P}} f_J \abs*{J}) & \text{(by Definition \ref{11.2.9})}     \\
                          & = \sum_{J \in \mathbf{P}} c f_J \abs*{J}   & \text{(by Proposition \ref{7.1.11}(g))} \\
                          & = p.c. \int_{[\mathbf{P}]} (c f)           & \text{(by Definition \ref{11.2.9})}     \\
                          & = p.c. \int_I (c f).                       & \text{(by Definition \ref{11.2.14})}
    \end{align*}
\end{proof}

\begin{proof}{(c)}
    We have
    \begin{align*}
        p.c. \int_I f - p.c. \int_I g & = p.c. \int_I f + (-1) p.c. \int_I g                                        \\
                                      & = p.c. \int_I f + p.c. \int_I (-g)   & \text{(by Theorem \ref{11.2.16}(b))} \\
                                      & = p.c. \int_I (f + (-g))             & \text{(by Theorem \ref{11.2.16}(a))} \\
                                      & = p.c. \int_I (f - g).               & \text{(by Definition \ref{9.2.1})}
    \end{align*}
\end{proof}

\begin{proof}{(d)}
    By Definition \ref{11.2.3}, \(\exists\ \mathbf{P}\) such that \(\mathbf{P}\) is a partition of \(I\) and \(f\) is piecewise constant with respect to \(\mathbf{P}\).
    Let \(J \in \mathbf{P}\) and let \(f_J \in \mathbf{R}\) be the constant value of \(f\) on \(J\).
    Since \(\forall\ x \in I\), \(f(x) \geq 0\), we then have \(f_J \geq 0\) and \(f_J \abs*{J} \geq 0\).
    Thus
    \begin{align*}
        p.c. \int_I f & = p.c. \int_{[\mathbf{P}]} f           & \text{(by Definition \ref{11.2.14})}    \\
                      & = \sum_{J \in \mathbf{P}} f_J \abs*{J} & \text{(by Definition \ref{11.2.9})}     \\
                      & \geq \sum_{J \in \mathbf{P}} 0         & \text{(by Proposition \ref{7.1.11}(h))} \\
                      & = 0.
    \end{align*}
\end{proof}

\begin{proof}{(e)}
    Since \(f(x) \geq g(x)\) for all \(x \in I\), we have \(f(x) - g(x) \geq 0\) and
    \begin{align*}
        p.c. \int_I f - p.c. \int_I g & = p.c. \int_I (f - g) & \text{(by Theorem \ref{11.2.16}(c))} \\
                                      & \geq 0.               & \text{(by Theorem \ref{11.2.16}(d))}
    \end{align*}
    Thus
    \[
        p.c. \int_I f \geq p.c. \int_I g.
    \]
\end{proof}

\begin{proof}{(f)}
    Since \(I\) is a partition of \(I\), we have
    \begin{align*}
        p.c. \int_I f & = p.c. \int_{[I]} f         & \text{(by Definition \ref{11.2.14})}    \\
                      & = \sum_{J \in I} c \abs*{J} & \text{(by Definition \ref{11.2.9})}     \\
                      & = c \sum_{J \in I} \abs*{J} & \text{(by Proposition \ref{7.1.11}(g))} \\
                      & = c \abs*{I}.               & \text{(by Theorem \ref{11.1.13})}
    \end{align*}
\end{proof}

\begin{proof}{(g)}
    Let \(I_1, I_2\) be the sets
    \[
        I_1 = \{x \in J, x \leq \inf(I) \land x \notin I\}
    \]
    and
    \[
        I_2 = \{x \in J, x \geq \sup(I) \land x \notin I \cup I_1\}.
    \]
    Then we know that \(I \cap I_1 = I \cap I_2 = I_1 \cap I_2 = \emptyset\).
    Let \(\mathbf{P} = \{I_1, I, I_2\}\).
    We know claim that \(\mathbf{P}\) is a partition of \(J\).
    Since \(J\) is a bounded interval, we know that \(\inf(J) = \inf(I_1)\) and \(\sup(J) = \sup(I_2)\).
    Then we have \(I_1 \subseteq [\inf(J), \inf(I)]\) and \(I_2 \subseteq [\sup(I), \sup(J)]\).
    If \(\inf(J) \in J\), then we know that \(I_1 = [\inf(J), \inf(I)]\) or \(I_1 = [\inf(J), \inf(I))\), which depends on whether \(\inf(I) \in I\).
    Otherwise we have \(I_1 = (\inf(J), \inf(I)]\) or \(I_1 = (\inf(J), \inf(I))\), which again depends on whether \(\inf(I) \in I\).
    Using similar arguments we know thtat \(I_2\) can be one of \((\sup(I), \sup(J))\), \([\sup(I), \sup(J))\), \((\sup(I), \sup(J)]\) or \([\sup(I), \sup(J)]\).
    Thus \(I_1, I_2\) are bounded intervals.
    By the definition of \(I_1, I_2\), we know that \(\bigcup \mathbf{P} \subseteq J\).
    To show that \(\bigcup \mathbf{P} = J\), by Proposition \ref{3.1.18} we need to show that \(J \subseteq \bigcup \mathbf{P}\).
    Let \(x \in J\).
    If \(x \in I\), we have \(x \in \bigcup \mathbf{P}\).
    If \(x \notin I\), we then have \(x \leq \inf(I) \lor x \geq \sup(I)\), thus \(x \in I_1 \lor x \in I_2\), and again \(x \in \bigcup \mathbf{P}\).
    Since \(x\) is arbitrary, we have \(J \subseteq \mathbf{P}\).
    Since \(\bigcup \mathbf{P} = J\) and \(\mathbf{P}\) is finite (\(\abs*{P} = 3\)), by Definition \ref{11.1.10} \(\mathbf{P}\) is a partition of \(J\).

    Now we show that \(F\) is piecewise constant on \(J\).
    Since \(f\) is piecewise constant on \(I\), by Definition \ref{11.2.5} \(\exists\ \mathbf{P}_I\) such that \(\mathbf{P}_I\) be the partition of \(I\) and \(f\) is piecewise constant with respect to \(\mathbf{P}_I\).
    By hypothesis we know that \(\forall\ K \in \mathbf{P}_I\), \(F\) is piecewise constant on \(K\) with constant value \(F(x) = f(x)\) for all \(x \in K\), thus by Definition \ref{11.2.5} \(F\) is piecewise constant on \(I\).
    Since \(\forall\ x \in I_1\), \(x \notin I\), by hypothesis we know that \(F(x) = 0\), thus \(F\) is piecewise constant on \(I_1\).
    Similar arguments show that \(F\) is piecewise constant on \(I_2\).
    Thus \(F\) is piecewise constant on \(\mathbf{P}\), and we have
    \begin{align*}
        p.c. \int_J F & = p.c. \int_{[\mathbf{P}]} F                                                       & \text{(by Definition \ref{11.2.14})}    \\
                      & = \sum_{K \in \mathbf{P}} c_K \abs*{K}                                             & \text{(by Definition \ref{11.2.9})}     \\
                      & = c_{I_1} \abs*{I_1} + \sum_{K \in \mathbf{P}_I} c_K \abs*{K} + c_{I_2} \abs*{I_2} & \text{(by Proposition \ref{7.1.11}(e))} \\
                      & = 0 \abs*{I_1} + \sum_{K \in \mathbf{P}_I} c_K \abs*{K} + 0 \abs*{I_2}             & \text{(by hypothesis)}                  \\
                      & = \sum_{K \in \mathbf{P}_I} c_K \abs*{K}                                                                                     \\
                      & = p.c. \int_{[\mathbf{P}_I]} f                                                     & \text{(by Definition \ref{11.2.9})}     \\
                      & = p.c. \int_I f.                                                                   & \text{(by Definition \ref{11.2.14})}
    \end{align*}
\end{proof}

\begin{proof}{(h)}
    We first show that \(f|_J\) is piecewise constant on \(J\) and \(f|_K\) is piecewise constant on \(K\).
    Since \(f\) is a piecewise constant function on \(I\), by Definition \ref{11.2.5} \(\exists\ \mathbf{P}\) such that \(\mathbf{P}\) is a partition of \(I\) and \(f\) is piecewise constant with respect to \(\mathbf{P}\).
    Let \(\mathbf{P}_J, \mathbf{P}_K\) be the sets
    \[
        \mathbf{P}_J = \{S \cap J : S \in \mathbf{P}\}
    \]
    and
    \[
        \mathbf{P}_K = \{S \cap K : S \in \mathbf{P}\}.
    \]
    Since \(\forall\ S_J \in \mathbf{P}_J\), \(S_J \in \mathbf{P}\), by Definition \ref{11.2.3} we know that \(f\) is constant on \(S_J\),
    By Definition \ref{11.1.10} \(S_J\) is a bounded interval, and \(S_{J'} \in \mathbf{P}_J\) and \(S_{J'} \neq S_J \implies S_{J'} \cap S_J = \emptyset\).
    By the definition of \(\mathbf{P}_J\) we know that \(\bigcup \mathbf{P}_J \subseteq J\).
    Let \(x \in J\).
    Since \(x \in J\), \(x \in I\), by Definition \ref{11.1.10} \(\exists!\ S_J \in \mathbf{P}_J\) such that \(x \in S_J\).
    Then we have \(x \in J \cap S_J\) and \(x \in \bigcup \mathbf{P}_J\), thus \(J \subseteq \bigcup \mathbf{P}_J\) and by Proposition \ref{3.1.18} \(J = \bigcup \mathbf{P}_J\).
    Since \(\mathbf{P}_J \subseteq \mathbf{P}\) and \(\mathbf{P}\) is finite, we know that \(\mathbf{P}_J\) is finite.
    Thus by Definition \ref{11.1.10} \(\mathbf{P}_J\) is a partition of \(J\).
    Using similar arguments we can show that \(\mathbf{P}_K\) is a partition of \(K\).
    Since \(\forall\ S_J \in \mathbf{P}_J\), \(S_J \in \mathbf{P}\) and \(f\) is piecewise constant on \(S_J\), by Definition \ref{11.2.3} \(f|_J\) is piecewise constant with respect to \(\mathbf{P}_J\).
    Using similar arguments we know that \(f|_K\) is piecewise constant with respect to \(\mathbf{P}_K\).
    Thus by Definition \ref{11.2.5} \(f|_J\) is piecewise constant on \(J\) and \(f|_K\) is piecewise constant on \(K\).

    Now we show that \(\mathbf{P} = \mathbf{P}_J \cup \mathbf{P}_K\).
    By the definition of \(\mathbf{P}_J\) and \(\mathbf{P}_K\) we know that \(\mathbf{P}_J \cup \mathbf{P}_J \subseteq \mathbf{P}\).
    Let \(S \in \mathbf{P}\).
    If \(S = \emptyset\), then \(S \subseteq \mathbf{P}_J \cup \mathbf{P}_K\).
    If \(S \neq \emptyset\), since \(S \subseteq I\) and \(\{J, K\}\) is a partition of \(I\), we know that \(S \cap (J \cup K) \neq \emptyset\).
    Thus we have \(S \in \mathbf{P}_J\) or \(S \in \mathbf{P}_K\), which means \(\mathbf{P} \subseteq \mathbf{P}_J \cup \mathbf{P}_K\).
    By Proposition \ref{3.1.18} we have \(\mathbf{P} = \mathbf{P}_J \cup \mathbf{P}_K\).
    Thus we have
    \begin{align*}
        p.c. \int_J f|_J + p.c. \int_K f|_K & = p.c. \int_{[\mathbf{P}_J]} f|_J + p.c. \int_{[\mathbf{P}_K]} f|_K               & \text{(by Definition \ref{11.2.14})}    \\
                                            & = \sum_{S \in \mathbf{P}_J} c_S \abs*{S} + \sum_{S \in \mathbf{P}_K} c_S \abs*{S} & \text{(by Proposition \ref{7.1.11}(e))} \\
                                            & = \sum_{S \in \mathbf{P}_J \cup \mathbf{P}_K} c_S \abs*{S}                        & \text{(by Definition \ref{11.2.9})}     \\
                                            & = \sum_{S \in \mathbf{P}} c_S \abs*{S}                                                                                      \\
                                            & = p.c. \int_{[\mathbf{P}]} f                                                      & \text{(by Definition \ref{11.2.9})}     \\
                                            & = p.c. \int_I f.                                                                  & \text{(by Definition \ref{11.2.14})}
    \end{align*}
\end{proof}
\section{Upper and lower Riemann integrals}\label{sec 11.3}

\begin{definition}[Majorization of functions]\label{11.3.1}
    Let \(f : I \to \mathbf{R}\) and \(g : I \to \mathbf{R}\).
    We say that \(g\) \emph{majorizes} \(f\) on \(I\) if we have \(g(x) \geq f(x)\) for all \(x \in I\), and that \(g\) \emph{minorizes} \(f\) on \(I\) if \(g(x) \leq f(x)\) for all \(x \in I\).
\end{definition}

\begin{definition}[Upper and lower Riemann integrals]\label{11.3.2}
    Let \(f : I \to \mathbf{R}\) be a bounded function defined on a bounded interval \(I\).
    We define the \emph{upper Riemann integral} \(\overline{\int}_I f\) by the formula
    \[
        \overline{\int}_I f \coloneqq \inf\{p.c. \int_I g : g \text{ is a piecewise constant function on \(I\) which majorizes } f\}
    \]
    and the \emph{lower Riemann integral} \(\underline{\int}_I f\) by the formula
    \[
        \underline{\int}_I f \coloneqq \sup\{p.c. \int_I g : g \text{ is a piecewise constant function on \(I\) which minorizes } f\}.
    \]
\end{definition}

\begin{lemma}\label{11.3.3}
    Let \(f : I \to \mathbf{R}\) be a function on a bounded interval \(I\) which is bounded by some real number \(M\), i.e., \(-M \leq f(x) \leq M\) for all \(x \in I\).
    Then we have
    \[
        -M \abs*{I} \leq \underline{\int}_I f \leq \overline{\int}_I f \leq M \abs*{I}.
    \]
    in particular, both the lower and upper Riemann integrals are real numbers (i.e., they are not infinite).
\end{lemma}

\begin{proof}
    The function \(g : I \to \mathbf{R}\) defined by \(g(x) = M\) is constant, hence piecewise constant, and majorizes \(f\);
    thus \(\overline{\int}_I f \leq p.c. \int_I g = M \abs*{I}\) by definition of the upper Riemann integral.
    A similar argument gives \(-M \abs*{I} \leq \underline{\int}_I f\).
    Finally, we have to show that \(\underline{\int}_I f \leq \overline{\int}_I f\).
    Let \(g\) be any piecewise constant function majorizing \(f\), and let \(h\) be any piecewise constant function minorizing \(f\).
    Then \(g\) majorizes \(h\), and hence \(p.c. \int_I h \leq p.c. \int_I g\).
    Taking suprema in \(h\), we obtain that \(\underline{\int}_I f \leq p.c. \int_I g\).
    Taking infima in \(g\), we thus obtain \(\underline{\int}_I f \leq \overline{\int}_I f\), as desired.
\end{proof}

\begin{definition}[Riemann integral]\label{11.3.4}
    Let \(f : I \to \mathbf{R}\) be a bounded function on a bounded interval \(I\).
    If \(\underline{\int}_I f = \overline{\int}_I f\), then we say that \(f\) is \emph{Riemann integrable on \(I\)} and define
    \[
        \int_I f \coloneqq \underline{\int}_I f = \overline{\int}_I f.
    \]
    If the upper and lower Riemann integrals are unequal, we say that \(f\) is not Riemann integrable.
\end{definition}

\begin{remark}\label{11.3.5}
    Compare this definition to the relationship between the \(\limsup\), \(liminf\), and limit of a sequence \(a_n\) that was established in Proposition \ref{6.4.12}(f);
    the \(\limsup\) is always greater than or equal to the \(\liminf\), but they are only equal when the sequence converges, and in this case they are both equal to the limit of the sequence.
    The definition given above may differ from the definition you may have encountered in your calculus courses, based on Riemann sums.
    However, the two definitions turn out to be equivalent.
\end{remark}

\begin{remark}\label{11.3.6}
    Note that we do not consider unbounded functions to be Riemann integrable;
    an integral involving such functions is known as an \emph{improper integral}.
    It is possible to still evaluate such integrals using more sophisticated integration methods (such as the Lebesgue integral).
\end{remark}

\begin{lemma}\label{11.3.7}
    Let \(f : I \to \mathbf{R}\) be a piecewise constant function on a bounded interval \(I\).
    Then \(f\) is Riemann integrable, and \(\int_I f = p.c. \int_I f\).
\end{lemma}

\begin{proof}
    Since \(f(x) \leq f(x)\), by Definition \ref{11.3.2} we have
    \[
        \overline{\int}_I f \leq p.c. \int_I f
    \]
    and
    \[
        p.c. \int_I f \leq \underline{\int}_I f.
    \]
    By Lemma \ref{11.3.3} we know that
    \[
        p.c. \int_I f \leq \underline{\int}_I f \leq \overline{\int}_I f \leq p.c. \int_I f.
    \]
    Thus by Definition \ref{11.3.4} we have
    \[
        \int_I f = \underline{\int}_I f = \overline{\int}_I f = p.c. \int_I f.
    \]
\end{proof}

\begin{remark}\label{11.3.8}
    Because of this lemma, we will not refer to the piecewise constant integral \(p.c. \int_I\) again, and just use the Riemann integral \(\int_I\) throughout
    (until this integral is itself superceded by the Lebesgue integral).
    We observe one special case of Lemma \ref{11.3.7}:
    if \(I\) is a point or the empty set, then \(\int_I f = 0\) for all functions \(f : I \to \mathbf{R}\).
    (Note that all such functions are automatically constant.)
\end{remark}

\begin{definition}[Riemann sums]\label{11.3.9}
    Let \(f : I \to \mathbf{R}\) be a bounded function on a bounded interval \(I\), and let \(\mathbf{P}\) be a partition of \(I\).
    We define the \emph{upper Riemann sum} \(U(f, \mathbf{P})\) and the \emph{lower Riemann sum} \(L(f, \mathbf{P})\) by
    \[
        U(f, \mathbf{P}) \coloneqq \sum_{J \in \mathbf{P} : J \neq \emptyset} (\sup_{x \in J} f(x)) \abs*{J}
    \]
    and
    \[
        L(f, \mathbf{P}) \coloneqq \sum_{J \in \mathbf{P} : J \neq \emptyset} (\inf_{x \in J} f(x)) \abs*{J}
    \]
\end{definition}

\begin{remark}\label{11.3.10}
    The restriction \(J \neq \emptyset\) is required because the quantities \(\inf_{x \in J} f(x)\) and \(\sup_{x \in J} f(x)\) are infinite (or negative infinite) if \(J\) is empty.
\end{remark}