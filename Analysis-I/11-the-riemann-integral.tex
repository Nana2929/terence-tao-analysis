\chapter{The Riemann integral}\label{ch 11}

\begin{note}
    In Chapter \ref{ch 10} we reviewed \emph{differentiation} - one of the two pillars of single variable calculus.
    The other pillar is, of course, \emph{integration}, which is the focus of the current chapter.
    More precisely, we will turn to the \emph{definite integral}, the integral of a function on a fixed interval, as opposed to the \emph{indefinite integral}, otherwise known as the \emph{antiderivative}.
    These two are of course linked by the \emph{Fundamental theorem of calculus}.
\end{note}

\begin{note}
    To actually \emph{define} this integral \(\int_I f\) is somewhat delicate (especially if one does not want to assume any axioms concerning geometric notions such as area), and not all functions \(f\) are integrable.
    It turns out that there are at least two ways to define this integral:
    the \emph{Riemann integral}, named after Georg Riemann (1826 -- 1866), which suffices for most applications, and the \emph{Lebesgue integral}, named after Henri Lebesgue (1875 -- 1941), which supercedes the Riemann integral and works for a much larger class of functions.
    There is also the \emph{Riemann-Steiltjes integral} \(\int_I f(x) d \alpha(x)\), a generalization of the Riemann integral due to Thomas Stieltjes (1856 -- 1894).
\end{note}