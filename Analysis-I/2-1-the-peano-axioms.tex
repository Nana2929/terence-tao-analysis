\section{The Peano axioms}\label{sec 2.1}

\begin{note}
We now present one standard way to define the natural numbers, in terms of the \emph{Peano axioms}, which were first laid out by Giuseppe Peano (1858–1932).
This is not the only way to define the natural numbers.
For instance, another approach is to talk about the cardinality of finite sets, for instance one could take a set of five elements and define \(5\) to be the number of elements in that set.
\end{note}

\begin{note}
In some texts the natural numbers start at \(1\) instead of \(0\), but this is a matter of notational convention more than anything else.
In this text we shall refer to the set \(\{1, 2, 3,...\}\) as the positive integers \(\mathbf{Z}^+\) rather than the natural numbers.
Natural numbers are sometimes also known as \emph{whole numbers}.
\end{note}

\begin{note}
In mathematics we try not to define a variable more than once in any given setting, as it can often lead to confusion;
many of the statements which were true for the old value of the variable can now become false, and vice versa.
\end{note}

\begin{axiom}\label{2.1}
\(0\) is a natural number.
\end{axiom}

\begin{axiom}\label{2.2}
If \(n\) is a natural number, then \(n++\) is also a natural number.
\end{axiom}

\begin{axiom}\label{2.3}
\(0\) is not the successor of any natural number;
i.e., we have \(n++ \neq 0\) for every natural number \(n\).
\end{axiom}

\begin{axiom}\label{2.4}
Different natural numbers must have different successors;
i.e., if \(n\), \(m\) are natural numbers and \(n \neq m\), then \(n++ \neq m++\).
Equivalently, if \(n++ = m++\), then we must have \(n = m\).
\end{axiom}

\begin{axiom}[Principle of mathematical induction]\label{2.5}
Let \(P(n)\) be any property pertaining to a natural number \(n\).
Suppose that \(P(0)\) is true, and suppose that whenever \(P(n)\) is true, \(P(n++)\) is also true.
Then \(P(n)\) is true for every natural number \(n\).
\end{axiom}

\begin{note}
Axioms \ref{2.1}-\ref{2.5} are known as the Peano axioms for the natural numbers.
\end{note}