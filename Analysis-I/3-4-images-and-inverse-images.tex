\section{Images and inverse images}\label{sec 3.4}

\begin{definition}[Images of sets]\label{3.4.1}
If \(f : X \to Y\) is a function from \(X\) to \(Y\), and \(S\) is a set in \(X\), we define \(f(S)\) to be the set
\[
    f(S) \coloneqq \{f(x) : x \in S\};
\]
this set is a subset of \(Y\), and is sometimes called the \emph{image} of \(S\) under the map \(f\).
We sometimes call \(f(S)\) the \emph{forward image} of \(S\) to distinguish it from the concept of the \emph{inverse image} \(f^{-1}(S)\) of \(S\).
\end{definition}

\setcounter{theorem}{3}
\begin{definition}[Inverse images]\label{3.4.4}
If \(U\) is a subset of \(Y\), we define the set \(f^{-1}(U)\) to be the set
\[
    f^{-1}(U) \coloneqq \{x \in X : f(x) \in U\}.
\]
In other words, \(f^{-1}(U)\) consists of all the elements of \(X\) which map into \(U\):
\[
    f(x) \in U \iff x \in f^{-1}(U).
\]
We call \(f^{-1}(U)\) the \emph{inverse image} of \(U\).
\end{definition}

\setcounter{theorem}{6}
\begin{remark}\label{3.4.6}
If \(f\) is a bijective function, then we have defined \(f^{-1}\) in two slightly different ways, but this is not an issue because both definitions are equivalent.
\end{remark}

\begin{axiom}[Power set axiom]\label{3.10}
Let \(X\) and \(Y\) be sets.
Then there exists a set, denoted \(Y^X\), which consists of all the functions from \(X\) to \(Y\), thus
\[
    f \in Y^X \iff (f \text{ is a function with domain } X \text{ and range } Y).
\]
\end{axiom}

\begin{note}
The reason we use the notation \(Y^X\) to denote this set is that if \(Y\) has \(n\) elements and \(X\) has \(m\) elements, then one can show that \(Y^X\) has \(n^m\) elements.
\end{note}

\setcounter{theorem}{8}
\begin{lemma}\label{3.4.9}
Let \(X\) be a set.
Then the set
\[
    \{Y : Y \text{ is a subset of } X\}
\]
is a set.
\end{lemma}

\begin{proof}
By Axiom \ref{3.10}, there exists a set \(\{0, 1\}^X\) which consists of all the functions from \(X\) to \(\{0, 1\}\).
Also by Axiom \ref{3.6}, we can replace each \(f \in \{0, 1\}^X\) with \(f^{-1}(\{1\})\), i.e., there exists a set \(S = \{f^{-1}(\{1\}) : f \in \{0, 1\}^X\}\).
Now we only have to show that \(\forall\ f^{-1}(\{1\}) \in S\), \(f^{-1}(\{1\}) \subseteq X\).
By Definition \ref{3.4.4}, \(f^{-1}(\{1\}) = \{x \in X : f(x) \in \{1\}\}\).
Thus \(f^{-1}(\{1\}) \subseteq X\) is true because some element in \(X\) can be map to \(0\).
And we conclude that \(S = \{Y : Y \text{ is a subset of } X\}\) exists.
\end{proof}

\begin{remark}\label{3.4.10}
The set \(\{Y : Y \text{ is a subset of } X\}\) is know as the \emph{power set} of \(X\) and is denoted \(2^X\).
\end{remark}

\begin{axiom}[Union]\label{3.11}
Let \(A\) be a set, all of whose elements are themselves sets.
Then there exists a set \(\bigcup A\) whose elements are precisely those objects which are elements of the elements of \(A\), thus for all objects \(x\)
\[
    x \in \bigcup A \iff (x \in S \text{ for some } S \in A)
\]
\end{axiom}

\begin{note}
The axiom of union, combined with the axiom of pair set, implies the axiom of pairwise union.
Another important consequence of this axiom is that if one has some set \(I\), and for every element \(\alpha \in I\) we have some set \(A_{\alpha}\), then we can form the union set \(\bigcup_{\alpha \in I} A_{\alpha}\) by defining
\[
    \bigcup_{\alpha \in I} A_{\alpha} \coloneqq \bigcup \{A_{\alpha} : \alpha \in I\},
\]
which is a set thanks to the axiom of replacement and the axiom of union.
More generally, we see that for any object \(y\),
\[
    y \in \bigcup_{\alpha \in I} A_{\alpha} \iff (y \in A_{\alpha} \text{ for some } \alpha \in I).
\]
In situations like this, we often refer to \(I\) as an \emph{index set}, and the elements \(\alpha\) of this index set as \emph{labels};
the sets \(A_{\alpha}\) are then called a \emph{family of sets}, and are \emph{indexed} by the labels \(\alpha \in I\).
Note that if \(I\) was empty, then \(\bigcup_{\alpha \in I} A_{\alpha}\) would automatically also be empty.
\end{note}

\begin{note}
We can similarly form intersections of families of sets, as long as the index set is non-empty.
More specifically, given any non-empty set \(I\), and given an assignment of a set \(A_{\alpha}\) to each \(\alpha \in I\), we can define the intersection \(\bigcap_{\alpha \in I} A_{\alpha}\) by first choosing some element \(\beta\) of \(I\) (which we can do since \(I\) is non-empty), and setting
\[
    \bigcap_{\alpha \in I} A_{\alpha} \coloneqq \{x \in A_{\beta} : x \in A_{\alpha} \text{ for all } \alpha \in I\},
\]
which is a set by the axiom of specification.
This definition may look like it depends on the choice of \(\beta\), but it does not.
Observe that for any object \(y\),
\[
    y \in \bigcap_{\alpha \in I} A_{\alpha} \iff (y \in A_{\alpha} \text{ for all } \alpha \in I).
\]
\end{note}

\setcounter{theorem}{11}
\begin{remark}\label{3.4.12}
The axioms of set theory that we have introduced (Axioms \ref{3.1}-\ref{3.11}, excluding the dangerous Axiom \ref{3.8}) are known as the \emph{Zermelo-Fraenkel axioms of set theory}, after Ernest Zermelo (1871--1953) and Abraham Fraenkel (1891--1965).
There is one further axiom we will eventually need, the famous \emph{axiom of choice}, giving rise to the \emph{Zermelo-Fraenkel-Choice (ZFC) axioms of set theory}, but we will not need this axiom for some time.
\end{remark}

\exercisesection

\begin{exercise}\label{ex 3.4.1}
Let \(f : X \to Y\) be a bijective function, and let \(f^{-1} : Y \to X\) be its inverse.
Let \(V\) be any subset of \(Y\).
Prove that the forward image of \(V\) under \(f^{-1}\) is the same set as the inverse image of \(V\) under \(f\);
thus the fact that both sets are denoted by \(f^{-1}(V)\) will not lead to any inconsistency.
\end{exercise}

\begin{proof}
Because \(f\) is bijective by the given condition, \(\forall\ x \in f^{-1}(V) \iff f(x) \in V \iff x\) is in the inverse image of \(V\) under \(f\).
\end{proof}

\begin{exercise}\label{ex 3.4.2}
Let \(f : X \to Y\) be a function from one set \(X\) to another set \(Y\), let \(S\) be a subset of \(X\), and let \(U\) be a subset of \(Y\).
What, in general, can one say about \(f^{-1}(f(S))\) and \(S\)?
What about \(f(f^{-1}(U))\) and \(U\)?
\end{exercise}

\begin{proof}
\(\forall\ x \in S\), \(x \in f^{-1}(f(S))\) is true because \(f(x) \in f(S)\).
So we can say \(S \subseteq f^{-1}(f(S))\) is true.
\(\forall\ x \in f^{-1}(U)\), \(\exists\ y \in U\) such that \(f(x) = y\).
This means \(\forall\ y \in f(f^{-1}(U))\), \(\exists\ y \in U\).
So we can say \(f(f^{-1}(U)) \subseteq U\) is true.
\end{proof}

\begin{exercise}\label{ex 3.4.3}
Let \(A\), \(B\) be two subsets of a set \(X\), and let \(f : X \to Y\) be a function.
Show that \(f(A \cap B) \subseteq f(A) \cap f(B)\), that \(f(A) \setminus f(B) \subseteq f(A \setminus B)\), \(f(A \cup B) = f(A) \cup f(B)\).
For the first two statements, is it true that the \(\subseteq\) relation can be imporved to \(=\)?
\end{exercise}

\begin{proof}
We first prove the intersection part.
\(\forall\ x \in A \cap B\), \(x \in A\) is true, and \(f(x) \in f(A)\) is true.
Similarly, \(x \in B\) is true, and \(f(x) \in f(B)\) is true.
Thus \(f(x) \in f(A) \cap f(B)\) is true, or equivalently \(f(A \cap B) \subseteq f(A) \cap f(B)\).
If \(A \cap B = \emptyset\), we can not derived \(f(A) \cap f(B) = \emptyset\), thus \(f(A) \cap f(B) \subsetneq f(A \cap B)\).

Next we prove the difference part.
\(\forall\ x \in A\), if \(f(x) \in f(A) \setminus f(B)\), then \(x \notin B\), otherwise \(f(x) \in f(B)\), a contradiction.
So \(x \in A \setminus B\), thus \(f(x) \in f(A \setminus B)\), or equivalently \(f(A) \setminus f(B) \subseteq f(A \setminus B)\).
If \(A \setminus B = A\), then \(f(A \setminus B) = f(A)\), but we can not derive \(f(A) \setminus f(B) = f(A)\), so \(f(A \setminus B) \subsetneq f(A) \setminus f(B)\).

Now we prove the union part.
\(\forall\ x \in A \cup B\), \(x \in A\) or \(x \in B\) is true.
If \(x \in A\) is true, then \(f(x) \in f(A)\) is true.
Similarly, if \(x \in B\) is true, then \(f(x) \in f(B)\) is true.
So \(\forall\ y \in f(A \cup B)\), \(y \in f(A) \cup f(B)\) is true, and we conclude that \(f(A \cup B) \subseteq f(A) \cup f(B)\).
\(\forall\ y \in f(A) \cup f(B)\), \(y \in f(A)\) or \(y \in f(B)\) is true.
If \(y \in f(A)\) is true, then \(\exists\ x \in A\) such that \(f(x) = y\), and because \(x \in A \implies x \in A \cup B\), so \(f(x) \in f(A \cup B)\).
Similarly, if \(y \in f(B)\) is true, then \(\exists\ x \in B\) such that \(f(x) = y\), and because \(x \in B \implies x \in A \cup B\), so \(f(x) \in f(A \cup B)\).
Thus \(f(A) \cup f(B) \subseteq f(A \cup B)\), and combined with previous result we conclude that \(f(A \cup B) = f(A) \cup f(B)\).
\end{proof}

\begin{exercise}\label{ex 3.4.4}
Let \(f : X \to Y\) be a function from one set \(X\) to another set \(Y\), and let \(U\), \(V\) be subsets of \(Y\). Show that \(f^{-1}(U \cup V) = f^{-1}(U) \cup f^{-1}(V)\), that
\(f^{-1}(U \cap V) = f^{-1}(U) \cap f^{-1}(V)\), and that \(f^{-1}(U \setminus V) = f^{-1}(U) \setminus f^{-1}(V)\).
\end{exercise}

\begin{proof}
We first prove the union part.
\(\forall\ x \in f^{-1}(U \cup V)\), \(f(x) \in U \cup V\).
If \(f(x) \in U\), then \(x \in f^{-1}(U)\).
Similarly, if \(f(x) \in V\), then \(x \in f^{-1}(V)\).
Thus \(f^{-1}(U \cup V) \subseteq f^{-1}(U) \cup f^{-1}(V)\).
\(\forall\ x \in f^{-1}(U) \cup f^{-1}(V)\), \(x \in f^{-1}(U)\) or \(x \in f^{-1}(V)\) is true.
If \(x \in f^{-1}(U)\), then \(f(x) \in U\) is true, and \(f(x) \in U \cup V\) is true, so \(x \in f^{-1}(U \cup V)\) is true.
Similarly, if \(x \in f^{-1}(V)\), then \(f(x) \in V\) is true, and \(f(x) \in U \cup V\) is true, so \(x \in f^{-1}(U \cup V)\) is true.
Thus \(f^{-1}(U) \cup f^{-1}(V) \subseteq f^{-1}(U \cup V)\).
We conclude that \(f^{-1}(U \cup V) = f^{-1}(U) \cup f^{-1}(V)\).

Next we prove the intersection part.
\(\forall\ x \in f^{-1}(U \cap V)\), \(f(x) \in U \cap V\).
Because \(f(x) \in U\), so \(x \in f^{-1}(U)\);
similar argument show that \(x \in f^{-1}(V)\), thus \(x \in f^{-1}(U) \cap f^{-1}(V)\), and we conclude that \(f^{-1}(U \cap V) \subseteq f^{-1}(U) \cap f^{-1}(V)\).
\(\forall\ x \in f^{-1}(U) \cap f^{-1}(V)\), \(x \in f^{-1}(U)\) is true, so \(f(x) \in U\) is true;
similar argument show that \(f(x) \in V\) is true, so \(f(x) \in U \cap V\) is true, and \(x \in f^{-1}(U \cap V)\) is also true, we conclude that \(f^{-1}(U) \cap f^{-1}(V) \subseteq f^{-1}(U \cap V)\).
Combine both result we get \(f^{-1}(U \cap V) = f^{-1}(U) \cap f^{-1}(V)\).

Now we prove the difference part.
\(\forall\ x \in f^{-1}(U \setminus V)\), \(f(x) \in U \setminus V\).
Because \(f(x) \in U\), \(x \in f^{-1}(U)\) is true.
But \(f(x) \notin V\), \(x \notin f^{-1}(V)\) is true.
So \(x \in f^{-1}(U) \setminus f^{-1}(V)\) is true, thus \(f^{-1}(U \setminus V) \subseteq f^{-1}(U) \setminus f^{-1}(V)\).
\(\forall\ x \in f^{-1}(U) \setminus f^{-1}(V)\), \(x \in f^{-1}(U) \implies f(x) \in U\) and \(x \notin f^{-1}(x) \implies f(x) \notin V\).
So \(f(x) \in U \setminus V\), and \(x \in f^{-1}(U \setminus V)\) is true, thus \( f^{-1}(U) \setminus f^{-1}(V) \subseteq f^{-1}(U \setminus V)\).
Combine both result we get \(f^{-1}(U \setminus V) = f^{-1}(U) \setminus f^{-1}(V)\).
\end{proof}

\begin{exercise}\label{ex 3.4.5}
Let \(f : X \to Y\) be a function from one set \(X\) to another set \(Y\).
Show that \(f(f^{-1}(S)) = S\) for every \(S \subseteq Y\) if and only if \(f\) is surjective.
Show that \(f^{-1}(f(S)) = S\) for every \(S \subseteq X\) if and only if \(f\) is injective.
\end{exercise}

\begin{proof}
We first prove the surjective part.
We begin with the necessary condition.
By the given condition \(f(f^{-1}(S)) = S\), \(\forall\ y \in S\), \(\exists\ x \in f^{-1}(S)\) such that \(f(x) = y\).
Since \(S \subseteq Y\), by replace \(S\) with \(Y\) we get the statement \(\forall\ y \in Y\), \(\exists\ x \in f^{-1}(Y)\) such that \(f(x) = y\), which is the definition of surjective.
Thus the necessary condition is true, all we left is the sufficient condition.
By the given condition \(f\) is surjective, \(\forall\ y \in Y\), \(\exists\ x \in X\) such that \(f(x) = y\).
Because \(S \subseteq Y\) and \(f\) is surjective, \(\forall\ y \in S\), \(\exists\ x \in X\) such that \(f(x) = y\).
Then \(x \in f^{-1}(S)\) is true, so \(f(f^{-1}(S)) = S\) is also true, otherwise some \(y \in S\) cannot be map under \(f\), which is a contradition with \(f\) is surjective.
So the sufficient condition is true, we conclude that the if and only if statement is true.

Now we prove the injective part.
We begin with the necessary condition.
Let \(x, x' \in S\) such that \(f(x) = f(x')\).
By the given condition \(f^{-1}(f(S)) = S\), \(f^{-1}(f(x)) = x\) and \(f^{-1}(f(x')) = x'\) are true.
But \(f(x) = f(x')\), so \(x = f^{-1}(f(x)) = f^{-1}(f(x')) = x'\), thus \(f\) is injective, and the necessary condition is true.
All we left is sufficient condition.
\(\forall\ x \in S\), \(x \in f^{-1}(f(S))\) is true, so \(S \subseteq f^{-1}(f(S))\).
And \(\forall\ x \in f^{-1}(f(S))\), \(\exists\ x' \in S\) such that \(f^{-1}(f(x')) = x\), then \(x = x'\) because \(f\) is injective, so \(x \in S\) is true, and \(f^{-1}(f(S)) \subseteq S\).
Combine the results above we get \(f\) is injective implies \(f^{-1}(f(S)) = S\), and the if and only if statement is true.
\end{proof}

\begin{exercise}\label{ex 3.4.6}
Prove Lemma \ref{3.4.9}.
\end{exercise}

\begin{proof}
See Lemma \ref{3.4.9}.
\end{proof}

\begin{exercise}\label{ex 3.4.7}
Let \(X\), \(Y\) be sets.
Define a \emph{partial function} from \(X\) to \(Y\) to be any function \(f : X' \to Y'\) whose domain \(X'\) is a subset of \(X\), and whose range \(Y'\) is a subset of \(Y\).
Show that the collection of all partial functions from \(X\) to \(Y\) is itself a set.
\end{exercise}

\begin{proof}
By Lemma \ref{3.4.9}, both \(\hat{X} = \{X' : X' \subseteq X\}\) and \(\hat{Y} = \{Y' : Y' \subseteq Y\}\) are sets.
Let \(x \in \hat{X}\) and \(y \in \hat{Y}\).
By Axiom \ref{3.10}, exists a set \(y^{x}\) which is consist of functions with domain \(x\) and range \(y\).
By Axiom \ref{3.6} and Axiom \ref{3.11}, exists a set \(\bigcup_{x \in \hat{X} \land y \in \hat{Y}} \{f : f \in y^{x}\}\}\).
Thus the collection of all partial functions from \(X\) to \(Y\) is itself a set.
\end{proof}

\begin{exercise}\label{ex 3.4.8}
Show that Axiom \ref{3.4} can be deduced from Axiom \ref{3.1}, Axiom \ref{3.3} and Axiom \ref{3.11}.
\end{exercise}

\begin{proof}
By Axiom \ref{3.1}, \(A\) is a set and \(B\) is a set.
And if \(x\) is a object, we can say \(x \in A\) or \(x \in B\).
By Axiom \ref{3.3}, there exists a set \(\{A, B\}\) whose only elements are \(A\) and \(B\).
By Axiom \ref{3.11}, \(x \in \bigcup \{A, B\} \iff x \in A \lor x \in B\).
By defining \(A \cup B \coloneqq \bigcup \{A, B\}\), we show that \(x \in A \cup B \iff x \in A \lor x \in B\) is true.
\end{proof}

\begin{exercise}\label{ex 3.4.9}
Show that if \(\beta\) and \(\beta'\) are two elements of a set \(I\), and to each \(\alpha \in I\) we assign a set \(A_{\alpha}\), then
\[
    \{x \in A_{\beta} : x \in A_{\alpha} \text{ for all } \alpha \in I\} = \{x \in A_{\beta'} : x \in A_{\alpha} \text{ for all } \alpha \in I\},
\]
and so the definition of \(\bigcap_{\alpha \in I} A_{\alpha}\) does not depend on \(\beta\).
\end{exercise}

\begin{proof}
\(\forall\ \alpha \in I \implies x \in A_{\alpha}\).
Because \(x \in A_{\beta}\) and \(\beta' \in I\), so \(x \in A_{\beta'}\).
Similar argument shows that \(x \in A_{\beta'} \implies x \in A_{\beta}\).
Thus the definition of \(\bigcap_{\alpha \in I} A_{\alpha}\) does not depend on \(\beta\).
\end{proof}

\begin{exercise}\label{ex 3.4.10}
Suppose that \(I\) and \(J\) are two sets, and for all \(\alpha \in I \cup J\) let \(A_{\alpha}\) be a set.
Show that \((\bigcup_{\alpha \in I} A_{\alpha}) \cup (\bigcup_{\alpha \in J} A_{\alpha}) = \bigcup_{\alpha \in I \cup J} A_{\alpha}\).
If \(I\) and \(J\) are non-empty, show that \((\bigcap_{\alpha \in I} A_{\alpha}) \cap (\bigcap_{\alpha \in J} A_{\alpha}) = \bigcap_{\alpha \in I \cup J} A_{\alpha}\).
\end{exercise}

\begin{proof}
We first prove the union part.
\(\forall\ x \in (\bigcup_{\alpha \in I} A_{\alpha}) \cup (\bigcup_{\alpha \in J} A_{\alpha})\), \(x \in \bigcup_{\alpha \in I} A_{\alpha} \lor x \in \bigcup_{\alpha \in J} A_{\alpha}\) is true.
If \(x \in \bigcup_{\alpha \in I} A_{\alpha}\), then \(x \in A_{\alpha}\) for some \(\alpha \in I\), so \(x \in A_{\alpha}\) for some \(\alpha \in I \cup J\) is true, or equivalently \(x \in \bigcup_{\alpha \in I \cup J} A_{\alpha}\) is true.
Similarly if \(x \in \bigcup_{\alpha \in J} A_{\alpha}\), then \(x \in A_{\alpha}\) for some \(\alpha \in J\), so \(x \in A_{\alpha}\) for some \(\alpha \in I \cup J\) is true, or equivalently \(x \in \bigcup_{\alpha \in I \cup J} A_{\alpha}\) is true.
Thus \(x \in (\bigcup_{\alpha \in I} A_{\alpha}) \cup (\bigcup_{\alpha \in J} A_{\alpha}) \implies x \in \bigcup_{\alpha \in I \cup J} A_{\alpha}\), or equivalently \((\bigcup_{\alpha \in I} A_{\alpha}) \cup (\bigcup_{\alpha \in J} A_{\alpha}) \subseteq \bigcup_{\alpha \in I \cup J} A_{\alpha}\).
\(\forall\ y \in \bigcup_{\alpha \in I \cup J} A_{\alpha}\), \(y \in A_{\alpha}\) for some \(\alpha \in I \cup J\).
If \(\alpha \in I\), then \(y \in A_{\alpha}\) for some \(\alpha \in I\) is true, or equivalently \(y \in \bigcup_{\alpha \in I} A_{\alpha}\) is true.
Similarly if \(\alpha \in J\), then \(y \in A_{\alpha}\) for some \(\alpha \in J\) is true, or equivalently \(y \in \bigcup_{\alpha \in J} A_{\alpha}\) is true.
Combine both results we get \(\alpha \in I \cup J \implies y \in (\bigcup_{\alpha \in I} A_{\alpha}) \cup (\bigcup_{\alpha \in J} A_{\alpha})\) is true.
Thus \(y \in \bigcup_{\alpha \in I \cup J} A_{\alpha} \implies y \in (\bigcup_{\alpha \in I} A_{\alpha}) \cup (\bigcup_{\alpha \in J} A_{\alpha})\), or equivalently \(\bigcup_{\alpha \in I \cup J} A_{\alpha} \subseteq (\bigcup_{\alpha \in I} A_{\alpha}) \cup (\bigcup_{\alpha \in J} A_{\alpha})\).
We conclude that \((\bigcup_{\alpha \in I} A_{\alpha}) \cup (\bigcup_{\alpha \in J} A_{\alpha}) = \bigcup_{\alpha \in I \cup J} A_{\alpha}\).

Now we prove the intersection part.
\(\forall\ x \in (\bigcap_{\alpha \in I} A_{\alpha}) \cap (\bigcap_{\alpha \in J} A_{\alpha})\), \(x \in A_{\alpha}\) for all \(\alpha \in I\) is true, and \(x \in A_{\alpha}\) for all \(\alpha \in J\) is true, so \(x \in A_{\alpha}\) for all \(\alpha \in I \cup J\) is true, or equivalently \(x \in \bigcap_{\alpha \in I \cup J} A_{\alpha}\) is true.
Thus \(x \in (\bigcap_{\alpha \in I} A_{\alpha}) \cap (\bigcap_{\alpha \in J} A_{\alpha}) \implies x \in \bigcap_{\alpha \in I \cup J} A_{\alpha}\), or equivalently \((\bigcap_{\alpha \in I} A_{\alpha}) \cap (\bigcap_{\alpha \in J} A_{\alpha}) \subseteq \bigcap_{\alpha \in I \cup J} A_{\alpha}\).
\(\forall\ y \in \bigcap_{\alpha \in I \cup J} A_{\alpha}\), \(y \in A_{\alpha}\) for all \(\alpha \in I \cup J\).
Because \(I \neq \emptyset\), \(\alpha \in I\) is true, \(y \in A_{\alpha}\) for all \(\alpha \in I\) is true, or equivalently \(y \in \bigcap_{\alpha \in I} A_{\alpha}\) is true.
Similarly \(J \neq \emptyset\), \(\alpha \in J\) is true, \(y \in A_{\alpha}\) for all \(\alpha \in J\) is true, or equivalently \(y \in \bigcap_{\alpha \in J} A_{\alpha}\) is true.
Since \(y \in \bigcap_{\alpha \in I} A_{\alpha} \land y \in \bigcap_{\alpha \in J} A_{\alpha}\), so \(y \in (\bigcap_{\alpha \in I} A_{\alpha}) \cap (\bigcap_{\alpha \in J} A_{\alpha})\) is true.
Thus \(y \in \bigcap_{\alpha \in I \cup J} A_{\alpha} \implies y \in (\bigcap_{\alpha \in I} A_{\alpha}) \cap (\bigcap_{\alpha \in J} A_{\alpha})\), or equivalently \(\bigcap_{\alpha \in I \cup J} A_{\alpha} \subseteq (\bigcap_{\alpha \in I} A_{\alpha}) \cap (\bigcap_{\alpha \in J} A_{\alpha})\).
We conclude that \((\bigcap_{\alpha \in I} A_{\alpha}) \cap (\bigcap_{\alpha \in J} A_{\alpha}) = \bigcap_{\alpha \in I \cup J} A_{\alpha}\).
\end{proof}

\begin{exercise}\label{ex 3.4.11}
Let \(X\) be a set, let \(I\) be a non-empty set, and for all \(\alpha \in I\) let \(A_{\alpha}\) be a subset of \(X\).
Show that
\[
    X \setminus \bigcup_{\alpha \in I} A_{\alpha} = \bigcap_{\alpha \in I} (X \setminus A_{\alpha})
\]
and
\[
    X \setminus \bigcap_{\alpha \in I} A_{\alpha} = \bigcup_{\alpha \in I} (X \setminus A_{\alpha}).
\]
This should be compared with de Morgan’s laws in Proposition \ref{3.1.28}
(although one cannot derive the above identities directly from de Morgan’s laws, as \(I\) could be infinite).
\end{exercise}

\begin{proof}
We first prove the union part.
\(\forall\ x \in X \setminus \bigcup_{\alpha \in I} A_{\alpha}\), \(x \in X \land x \notin \bigcup_{\alpha \in I} A_{\alpha} \iff x \in X \land \nexists\ \alpha \in I\) such that \(x \in A_{\alpha} \iff x \in X \land x \notin A_{\alpha}\) for all \(\alpha \in I \iff x \in X \setminus A_{\alpha}\) for all \(\alpha \in I \iff x \in \bigcap_{\alpha \in I} (X \setminus A_{\alpha})\).

Now we prove the intersection part.
\(\forall\ x \in X \setminus \bigcap_{\alpha \in I} A_{\alpha}\), \(x \in X \land x \notin \bigcap_{\alpha \in I} A_{\alpha} \iff x \in X \land\) not all \(\alpha \in I\) such that \(x \in A_{\alpha} \iff x \in X \land x \notin A_{\alpha}\) for some \(\alpha \in I \iff x \in X \setminus A_{\alpha}\) for some \(\alpha \in I \iff x \in \bigcup_{\alpha \in I} (X \setminus A_{\alpha})\).
\end{proof}

\begin{exercise}\label{ex 3.4.12}
Show the following statement is true:
\[
    y \in \bigcap_{\alpha \in I} A_{\alpha} \iff \forall\ \alpha \in I, y \in A_{\alpha}.
\]
\end{exercise}

\begin{proof}
\(y \in \bigcap_{\alpha \in I} A_{\alpha} \iff y \in \{x \in A_{\beta} : x \in A_{\alpha} \text{ for all } \alpha \in I\} \iff \forall\ \alpha \in I, y \in A_{\alpha}\).
\end{proof}