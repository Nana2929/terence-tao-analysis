\section{Images and inverse images}\label{sec 3.4}

\begin{definition}[Images of sets]\label{3.4.1}
    If \(f : X \to Y\) is a function from \(X\) to \(Y\), and \(S\) is a set in \(X\), we define \(f(S)\) to be the set
    \[
        f(S) \coloneqq \{f(x) : x \in S\};
    \]
    this set is a subset of \(Y\), and is sometimes called the \emph{image} of \(S\) under the map \(f\).
    We sometimes call \(f(S)\) the \emph{forward image} of \(S\) to distinguish it from the concept of the \emph{inverse image} \(f^{-1}(S)\) of \(S\).
\end{definition}

\setcounter{theorem}{3}
\begin{definition}[Inverse images]\label{3.4.4}
    If \(U\) is a subset of \(Y\), we define the set \(f^{-1}(U)\) to be the set
    \[
        f^{-1}(U) \coloneqq \{x \in X : f(x) \in U\}.
    \]
    In other words, \(f^{-1}(U)\) consists of all the elements of \(X\) which map into \(U\):
    \[
        f(x) \in U \iff x \in f^{-1}(U).
    \]
    We call \(f^{-1}(U)\) the \emph{inverse image} of \(U\).
\end{definition}

\setcounter{theorem}{6}
\begin{remark}\label{3.4.6}
    If \(f\) is a bijective function, then we have defined \(f^{-1}\) in two slightly different ways, but this is not an issue because both definitions are equivalent.
\end{remark}

\begin{axiom}[Power set axiom]\label{3.10}
    Let \(X\) and \(Y\) be sets.
    Then there exists a set, denoted \(Y^X\), which consists of all the functions from \(X\) to \(Y\), thus
    \[
        f \in Y^X \iff (f \text{ is a function with domain } X \text{ and range } Y).
    \]
\end{axiom}

\begin{note}
    The reason we use the notation \(Y^X\) to denote this set is that if \(Y\) has \(n\) elements and \(X\) has \(m\) elements, then one can show that \(Y^X\) has \(n^m\) elements.
\end{note}

\setcounter{theorem}{8}
\begin{lemma}\label{3.4.9}
    Let \(X\) be a set.
    Then the set
    \[
        \{Y : Y \text{ is a subset of } X\}
    \]
    is a set.
\end{lemma}

\begin{proof}
    Suppose that \(X\) is a set.
    By Axiom \ref{3.10}, there exists a set \(\{0, 1\}^X\) which consists of all the functions from \(X\) to \(\{0, 1\}\).
    \[
        f \in \{0, 1\}^X \iff (f \text{ is a function with domain } X \text{ and range } \{0, 1\}).
    \]
    By Axiom \ref{3.6}, we can replace each \(f \in \{0, 1\}^X\) with \(f^{-1}(\{1\})\), i.e., there exists a set
    \[
        S = \{f^{-1}(\{1\}) : f \in \{0, 1\}^X\}.
    \]
    By Definition \ref{3.4.4} we have
    \begin{align*}
                 & \forall Y \in S                                                              \\
        \implies & \exists\ f \in \{0, 1\}^X : Y = f^{-1}(\{1\}) = \{x \in X : f(x) \in \{1\}\} \\
        \implies & Y \subseteq X.
    \end{align*}
    Now \(\forall Y' \subseteq X\) we can define the following sets:
    \begin{align*}
        A_0 & = \{0 : x \in X \setminus Y'\}. & \text{(by Definition \ref{3.1.27} and Axiom \ref{3.6})} \\
        A_1 & = \{1 : x \in Y'\}.             & \text{(by Axiom \ref{3.6})}
    \end{align*}
    So we have
    \begin{align*}
                 & \forall Y' \subseteq X                                               \\
        \implies & \exists\ A_0, A_1                                                    \\
        \implies & \exists\ f : X \to A_0 \cup A_1 & \text{(by Axiom \ref{3.6})}        \\
        \implies & f \in \{0, 1\}^X                & \text{(by Axiom \ref{3.10})}       \\
        \implies & f^{-1}(A_1) = Y'                & \text{(by Definition \ref{3.4.4})} \\
        \implies & Y' \in S.
    \end{align*}
    Since \(\forall Y : Y \in S \iff Y \subseteq X\), we have show that \(S = \{Y : Y \subseteq X\}\) exists.
\end{proof}

\begin{remark}\label{3.4.10}
    The set \(\{Y : Y \text{ is a subset of } X\}\) is know as the \emph{power set} of \(X\) and is denoted \(2^X\).
\end{remark}

\begin{axiom}[Union]\label{3.11}
    Let \(A\) be a set, all of whose elements are themselves sets.
    Then there exists a set \(\bigcup A\) whose elements are precisely those objects which are elements of the elements of \(A\), thus for all objects \(x\)
    \[
        x \in \bigcup A \iff (x \in S \text{ for some } S \in A)
    \]
\end{axiom}

\begin{note}
    The axiom of union (Axiom \ref{3.11}), combined with the axiom of pair set (Axiom \ref{3.3}), implies the axiom of pairwise union (Axiom \ref{3.4}).
    Another important consequence of Axiom \ref{3.11} is that if one has some set \(I\), and for every element \(\alpha \in I\) we have some set \(A_{\alpha}\), then we can form the union set \(\bigcup_{\alpha \in I} A_{\alpha}\) by defining
    \[
        \bigcup_{\alpha \in I} A_{\alpha} \coloneqq \bigcup \{A_{\alpha} : \alpha \in I\},
    \]
    which is a set thanks to the axiom of replacement (Axiom \ref{3.6}) and the axiom of union (Axiom \ref{3.11}).
    More generally, we see that for any object \(y\),
    \[
        y \in \bigcup_{\alpha \in I} A_{\alpha} \iff (y \in A_{\alpha} \text{ for some } \alpha \in I).
    \]
    In situations like this, we often refer to \(I\) as an \emph{index set}, and the elements \(\alpha\) of this index set as \emph{labels};
    the sets \(A_{\alpha}\) are then called a \emph{family of sets}, and are \emph{indexed} by the labels \(\alpha \in I\).
    Note that if \(I\) was empty, then \(\bigcup_{\alpha \in I} A_{\alpha}\) would automatically also be empty.
\end{note}

\begin{note}
    We can similarly form intersections of families of sets, as long as the index set is non-empty.
    More specifically, given any non-empty set \(I\), and given an assignment of a set \(A_{\alpha}\) to each \(\alpha \in I\), we can define the intersection \(\bigcap_{\alpha \in I} A_{\alpha}\) by first choosing some element \(\beta\) of \(I\) (which we can do since \(I\) is non-empty), and setting
    \[
        \bigcap_{\alpha \in I} A_{\alpha} \coloneqq \{x \in A_{\beta} : x \in A_{\alpha} \text{ for all } \alpha \in I\},
    \]
    which is a set by the axiom of specification (Axiom \ref{3.5}).
    This definition may look like it depends on the choice of \(\beta\), but it does not.
    Observe that for any object \(y\),
    \[
        y \in \bigcap_{\alpha \in I} A_{\alpha} \iff (y \in A_{\alpha} \text{ for all } \alpha \in I).
    \]
\end{note}

\setcounter{theorem}{11}
\begin{remark}\label{3.4.12}
    The axioms of set theory that we have introduced (Axioms \ref{3.1}-\ref{3.11}, excluding the dangerous Axiom \ref{3.8}) are known as the \emph{Zermelo-Fraenkel axioms of set theory}, after Ernst Zermelo (1871 -- 1953) and Abraham Fraenkel (1891 -- 1965).
    There is one further axiom we will eventually need, the famous \emph{axiom of choice}, giving rise to the \emph{Zermelo-Fraenkel-Choice (ZFC) axioms of set theory}, but we will not need this axiom for some time.
\end{remark}

\exercisesection

\begin{exercise}\label{ex 3.4.1}
    Let \(f : X \to Y\) be a bijective function, and let \(f^{-1} : Y \to X\) be its inverse.
    Let \(V\) be any subset of \(Y\).
    Prove that the forward image of \(V\) under \(f^{-1}\) is the same set as the inverse image of \(V\) under \(f\);
    thus the fact that both sets are denoted by \(f^{-1}(V)\) will not lead to any inconsistency.
\end{exercise}

\begin{proof}
    Suppose that \(X, Y, V\) are sets and \(f : X \to Y\) is a function such that \(V \subseteq Y\) and \(f\) is bijective.
    Let \(f^{-1} : Y \to X\) be the inverse of \(f\).
    Let \(A\) be the set of the forward image of \(V\) under \(f^{-1}\).
    Let \(B\) be the set of the inverse image of \(V\) under \(f\).
    Then we have
    \begin{align*}
        \forall x \in A \iff & \exists\ v \in V : f^{-1}(v) = x & \text{(by Definition \ref{3.4.1})}  \\
        \iff                 & f(f^{-1}(v)) = v = f(x)          & \text{(by Definition \ref{3.3.20})} \\
        \iff                 & x \in B.                         & \text{(by Definition \ref{3.4.4})}
    \end{align*}
    Thus by Definition \ref{3.1.4} we have \(A = B\).
\end{proof}

\begin{exercise}\label{ex 3.4.2}
    Let \(f : X \to Y\) be a function from one set \(X\) to another set \(Y\), let \(S\) be a subset of \(X\), and let \(U\) be a subset of \(Y\).
    What, in general, can one say about \(f^{-1}(f(S))\) and \(S\)?
    What about \(f(f^{-1}(U))\) and \(U\)?
\end{exercise}

\begin{proof}
    We first show that \(S \subseteq f^{-1}(f(S))\).
    Suppose that \(X, Y, S\) are sets such that \(S \subseteq X\) and \(f : X \to Y\) is a function.
    Then we have
    \begin{align*}
        \forall x \in S \implies & f(x) \in f(S)       & \text{(by Definition \ref{3.4.1})} \\
        \implies                 & x \in f^{-1}(f(S)). & \text{(by Definition \ref{3.4.4})}
    \end{align*}
    Thus by Definition \ref{3.1.15} we have \(S \subseteq f^{-1}(f(S))\).

    Now we show that \(f(f^{-1}(U)) \subseteq U\).
    Suppose that \(X, Y, U\) are sets such that \(U \subseteq Y\) and \(f : X \to Y\) is a function.
    Then we have
    \begin{align*}
        \forall y \in f(f^{-1}(U)) \implies & \exists\ x \in f^{-1}(U) : f(x) = y & \text{(by Definition \ref{3.4.1})} \\
        \implies                            & y \in U.                            & \text{(by Definition \ref{3.4.4})}
    \end{align*}
    Thus by Definition \ref{3.1.15} we have \(f(f^{-1}(U)) \subseteq U\).
\end{proof}

\begin{exercise}\label{ex 3.4.3}
    Let \(A\), \(B\) be two subsets of a set \(X\), and let \(f : X \to Y\) be a function.
    Show that \(f(A \cap B) \subseteq f(A) \cap f(B)\), that \(f(A) \setminus f(B) \subseteq f(A \setminus B)\), \(f(A \cup B) = f(A) \cup f(B)\).
    For the first two statements, is it true that the \(\subseteq\) relation can be imporved to \(=\)?
\end{exercise}

\begin{proof}
    We first show that \(f(A \cap B) \subseteq f(A) \cap f(B)\).
    Suppose that \(A, B, X, Y\) are sets such that \(A \subseteq X\) and \(B \subseteq X\).
    Suppose that \(f : X \to Y\) is a function.
    Then we have
    \begin{align*}
                 & \forall y \in f(A \cap B)                                                                     \\
        \implies & y \in \{f(x) : x \in A \cap B\}                         & \text{(by Definition \ref{3.4.1})}  \\
        \implies & y \in \{f(x) : x \in A \land x \in B\}                  & \text{(by Definition \ref{3.1.23})} \\
        \implies & y \in \{f(x) : x \in A\} \land y \in \{f(x) : x \in B\}                                       \\
        \implies & y \in f(A) \land y \in f(B)                             & \text{(by Definition \ref{3.4.1})}  \\
        \implies & y \in f(A) \cap f(B).                                   & \text{(by Definition \ref{3.1.23})}
    \end{align*}
    Thus by Definition \ref{3.1.15} we have \(f(A \cap B) \subseteq f(A) \cap f(B)\).

    Next we show that \(f(A) \setminus f(B) \subseteq f(A \setminus B)\).
    Suppose that \(A, B, X, Y\) are sets such that \(A \subseteq X\) and \(B \subseteq X\).
    Suppose that \(f : X \to Y\) is a function.
    Then we have
    \begin{align*}
                 & \forall y \in f(A) \setminus f(B)                                                                \\
        \implies & y \in f(A) \land y \notin f(B)                             & \text{(by Definition \ref{3.1.27})} \\
        \implies & y \in \{f(x) : x \in A\} \land y \notin \{f(x) : x \in B\} & \text{(by Definition \ref{3.4.1})}  \\
        \implies & y \in \{f(x) : x \in A \land x \notin B\}                                                        \\
        \implies & y \in \{f(x) : x \in A \setminus B\}                       & \text{(by Definition \ref{3.1.27})} \\
        \implies & y \in f(A \setminus B).                                    & \text{(by Definition \ref{3.4.1})}
    \end{align*}
    Thus by Definition \ref{3.1.15} we have \(f(A) \setminus f(B) \subseteq f(A \setminus B)\).

    Finally we show that \(f(A \cup B) = f(A) \cup f(B)\).
    Suppose that \(A, B, X, Y\) are sets such that \(A \subseteq X\) and \(B \subseteq X\).
    Suppose that \(f : X \to Y\) is a function.
    Then we have
    \begin{align*}
             & \forall y \in f(A \cup B)                                                                   \\
        \iff & y \in \{f(x) : x \in A \cup B\}                        & \text{(by Definition \ref{3.4.1})} \\
        \iff & y \in \{f(x) : x \in A \lor x \in B\}                  & \text{(by Axiom \ref{3.4})}        \\
        \iff & y \in \{f(x) : x \in A\} \lor y \in \{f(x) : x \in B\}                                      \\
        \iff & y \in f(A) \lor y \in f(B)                             & \text{(by Definition \ref{3.4.1})} \\
        \iff & y \in f(A) \cup f(B).                                  & \text{(by Axiom \ref{3.4})}
    \end{align*}
    Thus by Definition \ref{3.1.4} we have \(f(A \cup B) = f(A) \cup f(B)\).
\end{proof}

\begin{exercise}\label{ex 3.4.4}
    Let \(f : X \to Y\) be a function from one set \(X\) to another set \(Y\), and let \(U\), \(V\) be subsets of \(Y\). Show that \(f^{-1}(U \cup V) = f^{-1}(U) \cup f^{-1}(V)\), that
    \(f^{-1}(U \cap V) = f^{-1}(U) \cap f^{-1}(V)\), and that \(f^{-1}(U \setminus V) = f^{-1}(U) \setminus f^{-1}(V)\).
\end{exercise}

\begin{proof}
    We first show that \(f^{-1}(U \cup V) = f^{-1}(U) \cup f^{-1}(V)\).
    Suppose that \(U, V, X, Y\) are sets such that \(U \subseteq Y\) and \(V \subseteq Y\).
    Suppose that \(f : X \to Y\) is a function.
    Then we have
    \begin{align*}
             & \forall x \in f^{-1}(U \cup V)                                            \\
        \iff & f(x) \in U \cup V                    & \text{(by Definition \ref{3.4.4})} \\
        \iff & f(x) \in U \lor f(x) \in V           & \text{(by Axiom \ref{3.4})}        \\
        \iff & x \in f^{-1}(U) \lor x \in f^{-1}(V) & \text{(by Definition \ref{3.4.4})} \\
        \iff & x \in f^{-1}(U) \cup f^{-1}(V).      & \text{(by Axiom \ref{3.4})}
    \end{align*}
    Thus by Definition \ref{3.1.4} we have \(f^{-1}(U \cup V) = f^{-1}(U) \cup f^{-1}(V)\).

    Next we show that \(f^{-1}(U \cap V) = f^{-1}(U) \cap f^{-1}(V)\).
    Suppose that \(U, V, X, Y\) are sets such that \(U \subseteq Y\) and \(V \subseteq Y\).
    Suppose that \(f : X \to Y\) is a function.
    Then we have
    \begin{align*}
             & \forall x \in f^{-1}(U \cap V)                                              \\
        \iff & f(x) \in U \cap V                     & \text{(by Definition \ref{3.4.4})}  \\
        \iff & f(x) \in U \land f(x) \in V           & \text{(by Definition \ref{3.1.23})} \\
        \iff & x \in f^{-1}(U) \land x \in f^{-1}(V) & \text{(by Definition \ref{3.4.4})}  \\
        \iff & x \in f^{-1}(U) \cap f^{-1}(V).       & \text{(by Definition \ref{3.1.23})}
    \end{align*}
    Thus by Definition \ref{3.1.4} we have \(f^{-1}(U \cap V) = f^{-1}(U) \cap f^{-1}(V)\).

    Finally we show that \(f^{-1}(U \setminus V) = f^{-1}(U) \setminus f^{-1}(V)\).
    Suppose that \(U, V, X, Y\) are sets such that \(U \subseteq Y\) and \(V \subseteq Y\).
    Suppose that \(f : X \to Y\) is a function.
    Then we have
    \begin{align*}
             & \forall x \in f^{-1}(U \setminus V)                                            \\
        \iff & f(x) \in U \setminus V                   & \text{(by Definition \ref{3.4.4})}  \\
        \iff & f(x) \in U \land f(x) \notin V           & \text{(by Definition \ref{3.1.23})} \\
        \iff & x \in f^{-1}(U) \land x \notin f^{-1}(V) & \text{(by Definition \ref{3.4.4})}  \\
        \iff & x \in f^{-1}(U) \setminus f^{-1}(V).     & \text{(by Definition \ref{3.1.23})}
    \end{align*}
    Thus by Definition \ref{3.1.4} we have \(f^{-1}(U \setminus V) = f^{-1}(U) \setminus f^{-1}(V)\).
\end{proof}

\begin{exercise}\label{ex 3.4.5}
    Let \(f : X \to Y\) be a function from one set \(X\) to another set \(Y\).
    Show that \(f(f^{-1}(S)) = S\) for every \(S \subseteq Y\) if and only if \(f\) is surjective.
    Show that \(f^{-1}(f(S)) = S\) for every \(S \subseteq X\) if and only if \(f\) is injective.
\end{exercise}

\begin{proof}
    We first show that \(\forall S \subseteq Y : f(f^{-1}(S)) = S \iff f\) is surjective.
    Suppose that \(X, Y, S\) are sets such that \(S \subseteq Y\) and \(f : X \to Y\) is a function.
    Then we have
    \begin{align*}
             & f \text{ is surjective}                                                                                                  \\
        \iff & (\forall S \subseteq Y : y \in S \implies \exists\ x \in X : f(x) = y)            & \text{(by Definition \ref{3.3.17})}  \\
        \iff & (\forall S \subseteq Y : y \in S \implies \exists\ x \in f^{-1}(S) : f(x) = y)    & \text{(by Definition \ref{3.4.4})}   \\
        \iff & (\forall S \subseteq Y : y \in S \implies y \in f(f^{-1}(S)))                     & \text{(by Definition \ref{3.4.1})}   \\
        \iff & (\forall S \subseteq Y : S \subseteq f(f^{-1}(S)))                                & \text{(by Definition \ref{3.1.15})}  \\
        \iff & (\forall S \subseteq Y : S \subseteq f(f^{-1}(S)) \land f(f^{-1}(S)) \subseteq S) & \text{(by Exercise \ref{ex 3.4.2})}  \\
        \iff & (\forall S \subseteq Y : S = f(f^{-1}(S))).                                       & \text{(by Proposition \ref{3.1.18})}
    \end{align*}

    Now we show that \(\forall S \subseteq X : f^{-1}(f(S)) = S \iff f\) is injective.
    Suppose that \(X, Y, S\) are sets such that \(S \subseteq X\) and \(f : X \to Y\) is a function.
    If \(f\) is injective, then \(\forall S \subseteq X\) we have
    \begin{align*}
                 & x \in f^{-1}(f(S))                                                                       \\
        \implies & f(x) \in f(S)                                      & \text{(by Definition \ref{3.4.4})}  \\
        \implies & \exists\ x' \in S : (f(x) = f(x') \implies x = x') & \text{(by Definition \ref{3.3.14})} \\
        \implies & x \in S.
    \end{align*}
    Thus \(f^{-1}(f(S)) \subseteq S\).
    By Exercise \ref{ex 3.4.2} we have \(S \subseteq f^{-1}(f(S))\), thus by Proposition \ref{3.1.18} we have \(S = f^{-1}(f(S))\).
    On the other hand, if \(\forall S \subseteq X : f^{-1}(f(S)) = S\), then we have
    \begin{align*}
                 & \forall x, x' \in S : f(x) = f(x')                                                        \\
        \implies & x \in f^{-1}(f(\{x\})) = f^{-1}(f(\{x'\})) = \{x'\}                                       \\
        \implies & x = x'                                                                                    \\
        \implies & f \text{ is injective}.                             & \text{(by Definition \ref{3.3.14})}
    \end{align*}
    Thus we conclude that \(\forall S \subseteq X : f^{-1}(f(S)) = S \iff f\) is injective.
\end{proof}

\begin{exercise}\label{ex 3.4.6}
    Prove Lemma \ref{3.4.9}.
\end{exercise}

\begin{proof}
    See Lemma \ref{3.4.9}.
\end{proof}

\begin{exercise}\label{ex 3.4.7}
    Let \(X\), \(Y\) be sets.
    Define a \emph{partial function} from \(X\) to \(Y\) to be any function \(f : X' \to Y'\) whose domain \(X'\) is a subset of \(X\), and whose range \(Y'\) is a subset of \(Y\).
    Show that the collection of all partial functions from \(X\) to \(Y\) is itself a set.
\end{exercise}

\begin{proof}
    Suppose that \(X, Y\) are sets.
    Then by Lemma \ref{3.4.9}, the set \(A = \{X' : X' \subseteq X\}\) exists, so does \(B = \{Y' : Y' \subseteq Y\}\).
    Now we have
    \begin{align*}
        C_1 & = \{Y'^{X'} : X' \in A \land Y' \in B\}             & \text{(by Axiom \ref{3.10})} \\
        C_2 & = \bigcup C_1 = \{f \in Y'^{X'} : Y'^{X'} \in C_1\} & \text{(by Axiom \ref{3.11})} \\
    \end{align*}
    If \(f : X' \to Y'\) is a partial function whose domain \(X' \subseteq X\) and whose range \(Y' \subseteq Y\), then we have \(Y'^{X'} \in C_1\), and thus \(f \in C_2\).
\end{proof}

\begin{exercise}\label{ex 3.4.8}
    Show that Axiom \ref{3.4} can be deduced from Axiom \ref{3.1}, Axiom \ref{3.3} and Axiom \ref{3.11}.
\end{exercise}

\begin{proof}
    By Axiom \ref{3.1}, \(A\) is a set and \(B\) is a set.
    And if \(x\) is a object, we can say \(x \in A\) or \(x \in B\).
    By Axiom \ref{3.3}, there exists a set \(\{A, B\}\) whose only elements are \(A\) and \(B\).
    By Axiom \ref{3.11}, \(x \in \bigcup \{A, B\} \iff x \in A \lor x \in B\).
    By defining \(A \cup B \coloneqq \bigcup \{A, B\}\), we show that \(x \in A \cup B \iff x \in A \lor x \in B\) is true.
\end{proof}

\begin{exercise}\label{ex 3.4.9}
    Show that if \(\beta\) and \(\beta'\) are two elements of a set \(I\), and to each \(\alpha \in I\) we assign a set \(A_{\alpha}\), then
    \[
        \{x \in A_{\beta} : x \in A_{\alpha} \ \forall \alpha \in I\} = \{x \in A_{\beta'} : x \in A_{\alpha} \ \forall \alpha \in I\},
    \]
    and so the definition of \(\bigcap_{\alpha \in I} A_{\alpha}\) does not depend on \(\beta\).
\end{exercise}

\begin{proof}
    Suppose that \(I\) is a set and \(\forall \alpha \in I : A_{\alpha}\) is a set.
    Let \(\beta, \beta' \in I\) and \(B, B'\) be sets
    \begin{align*}
        B  & = \{x \in A_{\beta} : x \in A_{\alpha} \ \forall \alpha \in I\}   \\
        B' & = \{x \in A_{\beta'} : x \in A_{\alpha} \ \forall \alpha \in I\}.
    \end{align*}
    We now show that \(B = B'\).
    \begin{align*}
             & \forall x \in B                                                                 \\
        \iff & x \in A_{\beta} \land x \in A_{\alpha} \ \forall \alpha \in I                   \\
        \iff & x \in A_{\alpha} \ \forall \alpha \in I                        & (\beta \in I)  \\
        \iff & x \in A_{\beta'} \land x \in A_{\alpha} \ \forall \alpha \in I & (\beta' \in I) \\
        \iff & x \in B'.
    \end{align*}
    Thus by Definition \ref{3.1.4} we have \(B = B'\).
\end{proof}

\begin{exercise}\label{ex 3.4.10}
    Suppose that \(I\) and \(J\) are two sets, and for all \(\alpha \in I \cup J\) let \(A_{\alpha}\) be a set.
    Show that \((\bigcup_{\alpha \in I} A_{\alpha}) \cup (\bigcup_{\alpha \in J} A_{\alpha}) = \bigcup_{\alpha \in I \cup J} A_{\alpha}\).
    If \(I\) and \(J\) are non-empty, show that \((\bigcap_{\alpha \in I} A_{\alpha}) \cap (\bigcap_{\alpha \in J} A_{\alpha}) = \bigcap_{\alpha \in I \cup J} A_{\alpha}\).
\end{exercise}

\begin{proof}
    We first show that \((\bigcup_{\alpha \in I} A_{\alpha}) \cup (\bigcup_{\alpha \in J} A_{\alpha}) = \bigcup_{\alpha \in I \cup J} A_{\alpha}\).
    Suppose that \(I\) and \(J\) are two sets, and \(\forall \alpha \in I \cup J : A_{\alpha}\) be a set.
    Then we have
    \begin{align*}
             & \forall x \in (\bigcup_{\alpha \in I} A_{\alpha}) \cup (\bigcup_{\alpha \in J} A_{\alpha})                                \\
        \iff & x \in \bigcup_{\alpha \in I} A_{\alpha} \lor x \in \bigcup_{\alpha \in J} A_{\alpha}       & \text{(by Axiom \ref{3.4})}  \\
        \iff & (\exists\ \alpha \in I : x \in A_{\alpha}) \lor (\exists\ \alpha \in J : x \in A_{\alpha}) & \text{(by Axiom \ref{3.11})} \\
        \iff & \exists\ \alpha \in I \lor \alpha \in J : x \in A_{\alpha}                                                                \\
        \iff & \exists\ \alpha \in I \cup J : x \in A_{\alpha}                                            & \text{(by Axiom \ref{3.4})}  \\
        \iff & x \in \bigcup_{\alpha \in I \cup J} A_{\alpha}.                                            & \text{(by Axiom \ref{3.11})}
    \end{align*}
    Thus by Definition \ref{3.1.4} we have \((\bigcup_{\alpha \in I} A_{\alpha}) \cup (\bigcup_{\alpha \in J} A_{\alpha}) = \bigcup_{\alpha \in I \cup J} A_{\alpha}\).

    Now we show that \(I \neq \emptyset \land J \neq \emptyset \implies (\bigcap_{\alpha \in I} A_{\alpha}) \cap (\bigcap_{\alpha \in J} A_{\alpha}) = \bigcap_{\alpha \in I \cup J} A_{\alpha}\).
    Suppose that \(I\) and \(J\) are two non-empty sets, and \(\forall \alpha \in I \cup J : A_{\alpha}\) be a set.
    Then we have
    \begin{align*}
             & \forall x \in (\bigcap_{\alpha \in I} A_{\alpha}) \cap (\bigcap_{\alpha \in J} A_{\alpha})                                       \\
        \iff & x \in \bigcap_{\alpha \in I} A_{\alpha} \land x \in \bigcap_{\alpha \in J} A_{\alpha}      & \text{(by Definition \ref{3.1.23})} \\
        \iff & (\forall \alpha \in I : x \in A_{\alpha}) \land (\forall \alpha \in J : x \in A_{\alpha})  & \text{(by Exercise \ref{ex 3.4.9})} \\
        \iff & \forall \alpha \in I \lor \alpha \in J : x \in A_{\alpha}                                                                        \\
        \iff & \forall \alpha \in I \cup J : x \in A_{\alpha}                                             & \text{(by Axiom \ref{3.4})}         \\
        \iff & x \in \bigcap_{\alpha \in I \cup J} A_{\alpha}.                                            & \text{(by Exercise \ref{ex 3.4.9})}
    \end{align*}
    Thus by Definition \ref{3.1.4} we have \((\bigcap_{\alpha \in I} A_{\alpha}) \cap (\bigcap_{\alpha \in J} A_{\alpha}) = \bigcap_{\alpha \in I \cup J} A_{\alpha}\).
\end{proof}

\begin{exercise}\label{ex 3.4.11}
    Let \(X\) be a set, let \(I\) be a non-empty set, and for all \(\alpha \in I\) let \(A_{\alpha}\) be a subset of \(X\).
    Show that
    \[
        X \setminus \bigcup_{\alpha \in I} A_{\alpha} = \bigcap_{\alpha \in I} (X \setminus A_{\alpha})
    \]
    and
    \[
        X \setminus \bigcap_{\alpha \in I} A_{\alpha} = \bigcup_{\alpha \in I} (X \setminus A_{\alpha}).
    \]
    This should be compared with de Morgan's laws in Proposition \ref{3.1.28}
    (although one cannot derive the above identities directly from de Morgan's laws, as \(I\) could be infinite).
\end{exercise}

\begin{proof}
    We first show that \(X \setminus \bigcup_{\alpha \in I} A_{\alpha} = \bigcap_{\alpha \in I} (X \setminus A_{\alpha})\).
    Suppose that \(X, I\) are sets, \(I \neq \emptyset\), \(\forall \alpha \in I : A_{\alpha}\) is a set and \(A_{\alpha} \subseteq X\).
    Then we have
    \begin{align*}
             & \forall x \in X \setminus \bigcup_{\alpha \in I} A_{\alpha}                                         \\
        \iff & x \in X \land x \notin \bigcup_{\alpha \in I} A_{\alpha}      & \text{(by Definition \ref{3.1.27})} \\
        \iff & x \in X \land \lnot(\exists\ \alpha \in I : x \in A_{\alpha}) & \text{(by Axiom \ref{3.11})}        \\
        \iff & x \in X \land (\forall \alpha \in I : x \notin A_{\alpha})                                          \\
        \iff & \forall \alpha \in I : x \in X \land x \notin A_{\alpha}                                            \\
        \iff & \forall \alpha \in I : x \in X \setminus A_{\alpha}           & \text{(by Definition \ref{3.1.27})} \\
        \iff & x \in \bigcap_{\alpha \in I} (X \setminus A_{\alpha})         & \text{(by Exercise \ref{ex 3.4.9})} \\
    \end{align*}
    Thus by Definition \ref{3.1.4} we have \(X \setminus \bigcup_{\alpha \in I} A_{\alpha} = \bigcap_{\alpha \in I} (X \setminus A_{\alpha})\).

    Now we show that \(X \setminus \bigcap_{\alpha \in I} A_{\alpha} = \bigcup_{\alpha \in I} (X \setminus A_{\alpha})\).
    Suppose that \(X, I\) are sets, \(I \neq \emptyset\), \(\forall \alpha \in I : A_{\alpha}\) is a set and \(A_{\alpha} \subseteq X\).
    Then we have
    \begin{align*}
             & \forall x \in X \setminus \bigcap_{\alpha \in I} A_{\alpha}                                        \\
        \iff & x \in X \land x \notin \bigcap_{\alpha \in I} A_{\alpha}     & \text{(by Definition \ref{3.1.27})} \\
        \iff & x \in X \land \lnot(\forall \alpha \in I : x \in A_{\alpha}) & \text{(by Exercise \ref{ex 3.4.9})} \\
        \iff & x \in X \land (\exists\ \alpha \in I : x \notin A_{\alpha})                                        \\
        \iff & \exists\ \alpha \in I : x \in X \land x \notin A_{\alpha}                                          \\
        \iff & \exists\ \alpha \in I : x \in X \setminus A_{\alpha}         & \text{(by Definition \ref{3.1.27})} \\
        \iff & x \in \bigcup_{\alpha \in I} (X \setminus A_{\alpha})        & \text{(by Axiom \ref{3.11})}        \\
    \end{align*}
    Thus by Definition \ref{3.1.4} we have \(X \setminus \bigcap_{\alpha \in I} A_{\alpha} = \bigcup_{\alpha \in I} (X \setminus A_{\alpha})\).
\end{proof}