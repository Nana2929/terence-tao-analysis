\section{Cardinality of sets}\label{sec 3.6}

\begin{definition}[Equal cardinality]\label{3.6.1}
We say that two sets \(X\) and \(Y\) have \emph{equal cardinality} iff there exists a bijection \(f : X \to Y\) from \(X\) to \(Y\).
\end{definition}

\setcounter{theorem}{2}
\begin{remark}\label{3.6.3}
The fact that two sets have equal cardinality does not preclude one of the sets from containing the other.
For instance, if \(X\) is the set of natural numbers and \(Y\) is the set of even natural numbers, then the map \(f : X \to Y\) defined by \(f(n) \coloneqq 2n\) is a bijection from \(X\) to \(Y\), and so \(X\) and \(Y\) have equal cardinality, despite \(Y\) being a subset of \(X\) and seeming intuitively as if it should only have ``half'' of the elements of \(X\).
\end{remark}

\begin{proposition}\label{3.6.4}
Let \(X\), \(Y\), \(Z\) be sets.
Then \(X\) has equal cardinality with \(X\).
If \(X\) has equal cardinality with \(Y\), then \(Y\) has equal cardinality with \(X\).
If \(X\) has equal cardinality with \(Y\) and \(Y\) has equal cardinality with \(Z\), then \(X\) has equal cardinality with \(Z\).
\end{proposition}

\begin{proof}
We first show that Definition \ref{3.6.1} is reflexive.
Suppose that \(X\) is a set.
Let \(f : X \to X\) be a function where \(f = x \mapsto x\).
By Axiom \ref{3.6} \(f\) is well-defined.
Such \(f\) is injective since \(\forall\ x, x' \in X\), \(f(x) = f(x') \implies x = x'\), and \(f\) is also surjective since \(\forall\ x \in X\), \(\exists\ x \in X\) such that \(f(x) = x\).
Thus \(f\) is bijective, and by Definition \ref{3.6.1} \(X\) has equal cardinality with \(X\).

Next we show that Definition \ref{3.6.1} is symmetric.
Suppose that \(X, Y\) are sets such that \(X\) has equal cardinality with \(Y\).
Then by Definition \ref{3.6.1} there exists a function \(f : X \to Y\) such that \(f\) is bijective.
Since \(f\) is bijective, by Exercise \ref{ex 3.3.6} \(f^{-1} : Y \to X\) is also bijective.
Thus by Definition \ref{3.6.1} \(Y\) has equal cardinality with \(X\).

Finally we show that Definition \ref{3.6.1} is transitive.
Suppose that \(X, Y, Z\) are sets such that \(X\) has equal cardinality with \(Y\) and \(Y\) has equal cardinality with \(Z\).
Then by Definition \ref{3.6.1} there exist two functions \(f : X \to Y\) and \(g : Y \to Z\) such that \(f\) and \(g\) are bijective.
Since \(f\) and \(g\) are bijective, by Exercise \ref{ex 3.3.7} \(g \circ f : X \to Z\) is also bijective.
Thus by Definition \ref{3.6.1} \(X\) has equal cardinality with \(Z\).
\end{proof}

\begin{definition}\label{3.6.5}
Let \(n\) be a natural number.
A set \(X\) is said to have \emph{cardinality} \(n\), iff it has equal cardinality with \(\{i \in \mathbf{N} : 1 \leq i \leq n\}\).
We also say that \(X\) \emph{has \(n\) elements} iff it has cardinality \(n\).
\end{definition}

\begin{remark}\label{3.6.6}
One can use the set \(\{i \in \mathbf{N} : i < n\}\) instead of \(\{i \in \mathbf{N} : 1 \leq i \leq n\}\), since these two sets clearly have equal cardinality.
\end{remark}

\setcounter{theorem}{7}
\begin{proposition}[Uniqueness of cardinality]\label{3.6.8}
Let \(X\) be a set with some cardinality \(n\).
Then \(X\) cannot have any other cardinality, i.e., \(X\) cannot have cardinality \(m\) for any \(m \neq n\).
\end{proposition}

\begin{proof}
We induct on \(n\).
First suppose that \(n = 0\).
Then \(X\) must be empty, and so \(X\) cannot have any non-zero cardinality.
Now suppose that the proposition is already proven for some \(n\);
we now prove it for \(n++\).
Let \(X\) have cardinality \(n++\);
and suppose that \(X\) also has some other cardinality \(m \neq n++\).
By Lemma \ref{3.6.9}, \(X\) is non-empty, and if \(x\) is any element of \(X\), then \(X \setminus \{x\}\) has cardinality \(n\) and also has cardinality \(p\), where \(p++ = m\), by Lemma \ref{3.6.9}.
By induction hypothesis, this means that \(n = p\), which implies that \(p++ = m = n++\), a contradiction.
This closes the induction.
\end{proof}

\begin{lemma}\label{3.6.9}
Suppose that \(n \geq 1\), and \(X\) has cardinality \(n\).
Then \(X\) is non-empty, and if \(x\) is any element of \(X\), then the set \(X \setminus \{x\}\) (i.e., \(X\) with the element \(x\) removed) has cardinality \(m\), where \(m++ = n\).
\end{lemma}

\begin{proof}
If \(X\) is empty then it clearly cannot have the same cardinality as the non-empty set \(\{i \in \mathbf{N} : 1 \leq i \leq n\}\), as there is no bijection from the empty set to a non-empty set.
Now let \(x\) be an element of \(X\).
Since \(X\) has the same cardinality as \(\{i \in \mathbf{N} : 1 \leq i \leq n\}\), we thus have a bijection \(f\) from \(X\) to \(\{i \in \mathbf{N} : 1 \leq i \leq n\}\).
In particular, \(f(x)\) is a natural number between \(1\) and \(n\).
Now define the function \(g : X \setminus \{x\} \to \{i \in \mathbf{N} : 1 \leq i \leq m\}\) by the following rule: for any \(y \in X \setminus \{x\}\), we define \(g(y) \coloneqq f(y)\) if \(f(y) < f(x)\), and define \(g(y)++ \coloneqq f(y)\) if \(f(y) > f(x)\).
(Note that \(f(y)\) cannot equal \(f(x)\) since \(y \neq x\) and \(f\) is a bijection.)
It is easy to check that this map is also a bijection, and so \(X \setminus \{x\}\) has equal cardinality with \(\{i \in \mathbf{N} : 1 \leq i \leq m\}\).
In particular \(X \setminus \{x\}\) has cardinality \(m\), as desired.
\end{proof}

\begin{definition}[Finite sets]\label{3.6.10}
A set is \emph{finite} iff it has cardinality \(n\) for some natural number \(n\);
otherwise, the set is called \emph{infinite}.
If \(X\) is a finite set, we use \(\#(X)\) to denote the cardinality of \(X\).
\end{definition}

\setcounter{theorem}{11}
\begin{theorem}\label{3.6.12}
The set of natural numbers \(\mathbf{N}\) is infinite.
\end{theorem}

\begin{proof}
Suppose for sake of contradiction that the set of natural numbers \(\mathbf{N}\) was finite, so it had some cardinality \(\#(\mathbf{N}) = n\).
Then there is a bijection \(f\) from \(\{i \in \mathbf{N} : 1 \leq i \leq n\}\) to \(\mathbf{N}\).
One can show that the sequence \(f(1), f(2), \dots, f(n)\) is bounded, or more precisely that there exists a natural number \(M\) such that \(f(i) \leq M\) for all \(1 \leq i \leq n\) (Exercise \ref{ex 3.6.3}).
But then the natural number \(M+1\) is not equal to any of the \(f(i)\), contradicting the hypothesis that \(f\) is a bijection.
\end{proof}

\begin{remark}\label{3.6.13}
One can also use similar arguments to show that any unbounded set is infinite;
for instance the rationals \(\mathbf{Q}\) and the reals \(\mathbf{R}\) are infinite.
However, it is possible for some sets to be ``more'' infinite than others.
\end{remark}

\begin{proposition}[Cardinal arithmetic]\label{3.6.14}
\leavevmode
\begin{enumerate}
    \item Let \(X\) be a finite set, and let \(x\) be an object which is not an element of \(X\).
    Then \(X \cup \{x\}\) is finite and \(\#(X \cup \{x\}) = \#(X) + 1\).
    \item Let \(X\) and \(Y\) be finite sets.
    Then \(X \cup Y\) is finite and \(\#(X \cup Y) \leq \#(X) + \#(Y)\).
    If in addition \(X\) and \(Y\) are disjoint (i.e., \(X \cap Y = \emptyset\)), then \(\#(X \cup Y) = \#(X) + \#(Y)\).
    \item Let \(X\) be a finite set, and let \(Y\) be a subset of \(X\).
    Then \(Y\) is finite, and \(\#(Y) \leq \#(X)\).
    If in addition \(Y \neq X\) (i.e., \(Y\) is a proper subset of \(X\)), then we have \(\#(Y) < \#(X)\).
    \item If \(X\) is a finite set, and \(f : X \to Y\) is a function, then \(f(X)\) is a finite set with \(\#(f(X)) \leq \#(X)\).
    If in addition \(f\) is one-to-one, then \(\#(f(X)) = \#(X)\).
    \item Let \(X\) and \(Y\) be finite sets.
    Then Cartesian product \(X \times Y\) is finite and \(\#(X \times Y) = \#(X) \times \#(Y)\).
    \item Let \(X\) and \(Y\) be finite sets.
    Then the set \(Y^X\) (defined in Axiom \ref{3.10}) is finite and \(\#(Y^X) = \#(Y)^{\#(X)}\).
\end{enumerate}
\end{proposition}

\begin{proof}{(a)}
Suppose that \(X\) is a finite set and \(x \notin X\).
By Definition \ref{3.6.10} \(\exists\ n \in \mathbf{N}\) such that \(\#(X) = n\).
By Definition \ref{3.6.5} \(\exists\ f : X \to \{i \in \mathbf{N} : 1 \leq i \leq n\}\) such that \(f\) is bijective.
Now we define a function \(g : X \cup \{x\} \to \{i \in \mathbf{N} : 1 \leq i \leq n + 1\}\) as follow:
\[
    g(y) = \begin{cases}
        f(y) & \text{if } y \in X \\
        n + 1 & \text{otherwise}
    \end{cases}
\]

Now we need to show that \(g\) is bijective.
Since \(f\) is bijective, \(\forall\ i \in \{i \in \mathbf{N} : 1 \leq i \leq n\}, \exists\ y \in X\) such that \(f(y) = i\).
With that and \(g(x) = n + 1\) we thus have \(g\) is surjective.
For any \(y, y' \in X \cup \{x\}\), if \(g(y) = g(y')\), then we have two cases:
\begin{enumerate}[label=(\roman*)]
    \item If \(g(y) \in X\), then \(y = y'\) since \(f\) is bijective.
    \item If \(g(y) = x\), then \(y = y' = x\) by definition of \(g\).
\end{enumerate}
For all cases above we have \(g(y) = g(y') \implies y = y'\), thus \(g\) is injective.
Since \(g\) is bijective, we have \(\#(X \cup \{x\}) = n + 1 = \#(X) + 1\).
\end{proof}

\begin{proof}{(b)}
Suppose that \(X, Y\) are finite sets.
By Definition \ref{3.6.10} \(\exists\ n \in \mathbf{N}\) such that \(\#(X) = n\).
We use induction on \(n\) to show that \(X \cup Y\) is finite and \(\#(X \cup Y) \leq \#(X) + \#(Y)\).
For \(n = 0\), we have
\begin{align*}
\#(X \cup Y) &= \#(\emptyset \cup Y) & \text{(by Definition \ref{3.6.5})} \\
&= \#(Y) & \text{(by Lemma \ref{3.1.13})} \\
&\leq \#(Y) & \text{(by Proposition \ref{2.2.12})} \\
&= 0 + \#(Y) & \text{(by Definition \ref{2.2.1})} \\
&= \#(X) + \#(Y). & \text{(by Definition \ref{3.6.5})}
\end{align*}
Thus \(X \cup Y\) is finite and the base case holds.
Suppose inductively that the statement is true for some \(\#(X) = n\).
Then for \(\#(X) = n++\), we must have \(X \neq \emptyset\) by Lemma \ref{3.6.9} since \(n++ \neq 0\).
Let \(x \in X\).
Then we have
\begin{align*}
\#(X \cup Y) &= \#((X \setminus \{x\}) \cup \{x\} \cup Y) & \text{(by Proposition \ref{3.1.28}(g))} \\
&= \#((X \setminus \{x\}) \cup Y \cup \{x\}) & \text{(by Proposition \ref{3.1.28}(d)(e))} \\
&= \#((X \setminus \{x\}) \cup Y) + 1 & \text{(by Proposition \ref{3.6.14}(a))} \\
&\leq \#(X \setminus \{x\}) + \#(Y) + 1 & \text{(by induction hypothesis)} \\
&= \#((X \setminus \{x\}) \cup \{x\}) + \#(Y) & \text{(by Proposition \ref{3.6.14}(a))} \\
&= \#(X) + \#(Y). & \text{(by Proposition \ref{3.1.28}(g))}
\end{align*}
Thus \(X \cup Y\) is finite and this close the induction.

Now suppose that \(X, Y\) are finite sets and \(X \cap Y = \emptyset\).
By Definition \ref{3.6.10} \(\exists\ n \in \mathbf{N}\) such that \(\#(X) = n\).
We use induction on \(n\) to show that \(\#(X \cup Y) = \#(X) + \#(Y)\).
For \(n = 0\), we have
\begin{align*}
\#(X \cup Y) &= \#(\emptyset \cup Y) & \text{(by Definition \ref{3.6.5})} \\
&= \#(Y) & \text{(by Lemma \ref{3.1.13})} \\
&= 0 + \#(Y) & \text{(by Definition \ref{2.2.1})} \\
&= \#(X) + \#(Y). & \text{(by Definition \ref{3.6.5})}
\end{align*}
Thus \(X \cup Y\) is finite and the base case holds.
Suppose inductively that the statement is true for some \(\#(X) = n\).
Then for \(\#(X) = n++\), we must have \(X \neq \emptyset\) by Lemma \ref{3.6.9} since \(n++ \neq 0\).
Let \(x \in X\).
Then we have
\begin{align*}
\#(X \cup Y) &= \#((X \setminus \{x\}) \cup \{x\} \cup Y) & \text{(by Proposition \ref{3.1.28}(g))} \\
&= \#((X \setminus \{x\}) \cup Y \cup \{x\}) & \text{(by Proposition \ref{3.1.28}(d)(e))} \\
&= \#((X \setminus \{x\}) \cup Y) + 1 & \text{(by Proposition \ref{3.6.14}(a))} \\
&= \#(X \setminus \{x\}) + \#(Y) + 1 & \text{(by induction hypothesis)} \\
&= \#((X \setminus \{x\}) \cup \{x\}) + \#(Y) & \text{(by Proposition \ref{3.6.14}(a))} \\
&= \#(X) + \#(Y). & \text{(by Proposition \ref{3.1.28}(g))}
\end{align*}
Thus \(X \cup Y\) is finite and this close the induction.
\end{proof}

\begin{proof}{(c)}
Let \(\#(X) = n\).
We use induction on \(n\).
For \(n = 0\), if \(Y \subseteq X = \emptyset\), then \(Y = \emptyset\).
So \(\#(Y) = 0 = \#(X) \leq \#(X)\), and \(Y\) is finite.
Thus the base case holds.
Suppose inductively that when \(\#(X) = n\), \(\forall\ Y \subseteq X \implies \#(Y) \leq \#(X)\) and \(Y\) is finite.
Then for \(\#(X) = n++\), let \(Y \subseteq X\).
If \(Y = \emptyset\), then \(\#(Y) = 0 \leq n++ = \#(X)\).
If \(Y \neq \emptyset\), then let \(y \in Y\), so \(y \in X\).
Because \(Y \setminus \{y\}  \subseteq X \setminus \{y\}\), by induction hypothesis, \(\#(Y \setminus \{y\}) \leq \#(X \setminus \{y\}) = n\).
So \(\#(Y) = \#(Y \setminus \{y\}) + 1 \leq \#(X \setminus \{y\}) + 1 = n + 1 = \#(X)\).
Thus \(\#(Y) = m\) where \(m \in \mathbf{N}\) and \(m \leq n++\), so \(Y\) is finite.
This close the induction.

Now we show that if \((Y \subseteq X) \land (Y \neq X)\), then \(\#(Y) < \#(X)\).
Because \(Y \neq X\), there exists a \(x\) such that \((x \in X) \land (x \notin Y)\).
So \(Y \cup \{x\} \subseteq X\), and \(\#(Y) + 1 = \#(Y \cup \{x\}) \leq \#(X)\).
By Proposition \ref{2.2.12}, \(1\) is positive, so \(\#(Y) < \#(X)\).
\end{proof}

\begin{proof}{(d)}
Let \(\#(X) = n\).
We use induction on \(n\).
For \(n = 0\), if \(f : X \to Y\) is a function, then \(f(X) = \emptyset\).
Thus \(\#(f(X)) = 0 \leq 0 = \#(X)\).
So the base case holds.
Suppose inductively that when \(\#(X) = n\), for any \(f : X \to Y\) we can derive \(\#(f(X)) \leq \#(X)\).
Then for \(\#(X) = n++\), for any \(f : X \to Y\), we can rewrite the domain of \(f\) as \(X' \cup x\), where \(x \notin X'\).
So \(\#(X') = n\), and by induction hypothesis \(\#(f(X')) \leq \#(X')\).
Because \(f(X) = f(X') \cup \{f(x)\}\), so \(\#(f(X)) = \#(f(X') \cup \{f(x)\}) = \#(f(X')) + 1 \leq \#(X') + 1 = \#(X' \cup \{x\}) = \#(X)\).
This close the induction.

Now we prove the one-to-one part.
Let \(f : X \to Y\) be a function which is one-to-one and \(X\) be finite.
We define \(g : X \to f(X)\) where \(\forall\ x \in X\), \(g(x) = f(x)\).
Because \(f\) is one-to-one, so \(g\) is also one-to-one.
And \(\forall\ y \in f(X)\), \(\exists\ x \in X\) such that \(g(x) = y\), \(g\) is surjective.
So \(g\) is bijective, thus \(\#(f(X)) = \#(X)\).
\end{proof}

\begin{proof}{(e)}
Let \(\#(X) = n\).
We use induction on \(n\).
For \(n = 0\), \(X \times Y = \emptyset\), so \(\#(X \times Y) = 0 = 0 \times \#(Y) = \#(X) \times \#(Y)\).
Thus the base case holds.
Suppose inductively that when \(\#(X) = n\), \(\#(X \times Y) = \#(X) \times \#(Y)\).
Then for \(\#(X) = n++\), let \(X = X' \cup \{x\}\), where \(x \notin X'\), so \(\#(X') = n\) and \(\#(\{x\}) = 1\).
Since \(\#(\{x\}) = 1\), let \(f : \{x\} \times Y \to Y\) be a function such that \(f(x, y) = y\), where \(y \in Y\).
Then such \(f\) is bijective, so \(\#(\{x\} \times Y) = \#(Y) = 1 \times \#(Y) = \#(\{x\}) \times \#(Y)\).
By induction hypothesis, \(\#(X' \times Y) = \#(X') \times \#(Y)\).
By Exercise \ref{ex 3.5.4}, \(\#(X \times Y) = \#((X' \cup \{x\}) \times Y) = \#((X' \times Y) \cup (\{x\} \times Y)) = \#(X' \times Y) + \#(\{x\} \times Y) = \#(X') \times \#(Y) + \#(\{x\}) \times \#(Y) = (\#(X') + \#(\{x\})) \times \#(Y) = \#(X) \times \#(Y)\).
This close the induction.
\end{proof}

\begin{proof}{(f)}
We first show that \(\#(Y^{\emptyset}) = 1\).
Because \(\forall\ f, g \in Y^{\emptyset}\), \(f\) and \(g\) have same domain and range, and both are empty functions, so \(\forall\ x \in \emptyset\), \(f(x) = g(x)\) is vacuously true, which means \(f = g\), and \(\#(Y^X) = \#(Y^{\emptyset}) = 1\).

Next we show that \(\#(\{y\}^{\{x\}}) = 1\).
Because \(\{y\}^{\{x\}} = \{f \mid f : \{x\} \to \{y\}\}\), \(\forall\ f, g \in \{y\}^{\{x\}}\), \(f\) and \(g\) have same domain and range, and there is only one element in both domain and range, so \(f(x) = y = g(x)\) is true, and \(f = g\) is true.
Thus there is only one element in \(\{y\}^{\{x\}}\), so \(\#(\{y\}^{\{x\}}) = 1\).

Next we show that \(Y^{\{x\}} = \#(Y)\).
Let \(\#(Y) = n\).
We use induction on \(n\).
For \(n = 0\), \(Y^{\{x\}} = \emptyset^{\{x\}} = \{f \mid f : \{x\} \to \emptyset\}\).
By Definition \ref{3.3.1}, there is no function whose range equals to \(\emptyset\), so \(\emptyset^{\{x\}} = \emptyset\).
Thus \(\#(\emptyset^{\{x\}}) = 0 = 0^1 = \#(\emptyset)^{\#(\{x\})}\), and the base case holds.
Suppose inductively that when \(\#(Y) = n\), \(\#(Y^{\{x\}}) = \#(Y)\).
Then for \(\#(Y) = n++\), let \(Y = Y' \cup \{y\}\), where \(y \notin Y'\).
So \(\#(Y) = n\) and \(\#(\{y\}) = 1\), and \(Y^{\{x\}} = (Y' \cup \{y\})^{\{x\}} = \{f \mid f : \{x\} \to Y' \cup \{y\}\}\).
For any \(f \in (Y' \cup \{y\})^{\{x\}}\), there is only one element in domain of \(f\), i.e., \(x\).
So \(f(x) \in Y'\) or \(f(x) \in \{y\}\) is true, and \(\{f \mid f : \{x\} \to Y' \cup \{y\}\} = \{f \mid f : \{x\} \to Y'\} \cup \{f \mid f : \{x\} \to \{y\}\} = Y'^{\{x\}} \cup \{y\}^{\{x\}}\) is true.
By induction hypothesis, \(\#(Y'^{\{x\}}) = \#(Y')\), and by previous prove \(\#(\{y\}^{\{x\}}) = 1\).
So \(\#(Y^{\{x\}}) = \#((Y' \cup \{y\})^{\{x\}}) = \#(Y'^{\{x\}} \cup \{y\}^{\{x\}}) = \#(Y'^{\{x\}}) + \#(\{y\}^{\{x\}}) = \#(Y') + 1 = \#(Y)\).
This close the induction.

Finally we prove that \(\#(Y^X) = \#(Y)^{\#(X)}\).
Let \(\#(X) = n\).
We use induction on \(n\).
For \(n = 0\), by previous prove \(\#(Y^X) = \#(Y^{\emptyset}) = 1 = \#(Y)^0 = \#(Y)^{\#(X)}\).
Thus the base case holds.
Suppose inductively that when \(\#(X) = n\), \(\#(Y^X) = \#(Y)^{\#(X)}\).
Then for \(\#(X) = n++\), let \(X = X' \cup \{x\}\), where \(x \notin X'\), so \(\#(X') = n\) and \(\#(\{x\}) = 1\).
We claim that there exists a bijection \(F\) from \(Y^X\) to \(Y^{X'} \times Y^{\{x\}}\).
By setting \(F(f_1) = (f_2, f_3)\) where \(f_1(a) = f_2(a)\) if \(a \in X'\) and \(f_1(a) = f_3(a)\) if \(a \in \{x\}\), we can show that \(F\) is bijective.
Let \(f_1, f_1' \in Y^X\) where \(F(f_1) = F(f_1') = (f_2, f_3)\).
Since \(f_1\) and \(f_1'\) have same domain and range, and \(\forall\ a \in X' \cup \{x\}\), \(f_1(a) = f_1'(a)\), so \(f_1 = f_1'\) is true, and \(F\) is injective.
And \(\forall\ (f_2, f_3) \in Y^{X'} \times Y^{\{x\}}\), we can always have a \(f_1 \in Y^X\) where \(f_1(a) = f_2(a)\) if \(a \in X'\) and \(f_1(a) = f_3(a)\) if \(a \in \{x'\}\), so \(F\) is surjective.
Because \(F\) is both injective and surjective, \(F\) is bijective, we can derive that \(\#(Y^X) = \#(Y^{X'} \times Y^{\{x\}})\).
By induction hypothesis, \(\#(Y^{X'}) = \#(Y)^{\#(X')}\), and by previous prove \(\#(Y^{\{x\}}) = \#(Y)\).
Thus \(\#(Y^X) = \#(Y^{X'} \times Y^{\{x\}}) = \#(Y^{X'}) \times \#(Y^{\{x\}}) = \#(Y)^{\#(X')} \times \#(Y) = \#(Y)^{\#(X') + 1} = \#(Y)^{\#(X)}\).
This close the induction.
\end{proof}

\begin{remark}\label{3.6.15}
Proposition \ref{3.6.14} suggests that there is another way to define the arithmetic operations of natural numbers;
not defined recursively as in Definitions \ref{2.2.1}, \ref{2.3.1}, \ref{2.3.11}, but instead using the notions of union, Cartesian product, and power set.
This is the basis of \emph{cardinal arithmetic}, which is an alternative foundation to arithmetic than the Peano arithmetic we have developed here.
\end{remark}

\exercisesection

\begin{exercise}\label{ex 3.6.1}
Prove Proposition \ref{3.6.4}.
\end{exercise}

\begin{proof}
See Proposition \ref{3.6.4}.
\end{proof}

\begin{exercise}\label{ex 3.6.2}
Show that a set \(X\) has cardinality \(0\) if and only if \(X\) is the empty set.
\end{exercise}

\begin{proof}
We first prove the necessary condition.
By Definition \ref{3.6.5}, if \(\#(X) = 0\), then there exist a bijective function \(f : X \to \{i \in \mathbf{N} : 1 \leq i \leq 0\}\).
By Axiom \ref{2.3}, there does not exist a natural number \(n\) where \(n++ = 0\).
So \(\{i \in \mathbf{N} : 1 \leq i \leq 0\} = \emptyset\).
By Definition \ref{3.3.1}, if \(X\) is non-empty, then there exist a \(x \in X\) such that \(f(x) \in \emptyset\), a contradiction.
So \(X = \emptyset\) (and \(\forall\ x \in \emptyset, f(x) \in \emptyset\) is vacuously true).

Now we prove the sufficient condition.
If \(X = \emptyset\), let \(f : \emptyset \to \emptyset\).
Because \(\forall\ x_1, x_2 \in \emptyset\), \(f(x_1) = f(x_2) \implies x_1 = x_2\) is vacuously true, so \(f\) is injective.
Also because \(\forall\ y \in \emptyset\), it is vacuously true that there exist \(x \in \emptyset\) such that \(f(x) = y\), so \(f\) is surjective.
Since \(f\) is both injective and surjective, \(f\) is bijective.
And the range of \(f\) is \(\emptyset\) which equals to \(\{i \in \mathbf{N} : 1 \leq i \leq 0\}\), so \(\#(\emptyset) = 0\).

Since both necessary and sufficient conditions are proved, we conclude that \(X\) has cardinality \(0\) if and only if \(X\) is the empty set.
\end{proof}

\begin{exercise}\label{ex 3.6.3}
Let \(n\) be a natural number, and let \(f : \{i \in \mathbf{N} : 1 \leq i \leq n\} \to \mathbf{N}\) be a function.
Show that there exists a natural number \(M\) such that \(f(i) \leq M\) for all \(1 \leq i \leq n\).
Thus finite subsets of the natural numbers are bounded.
\end{exercise}

\begin{proof}
We use induction on \(n\).
For \(n = 0\), \(\{i \in \mathbf{N} : 1 \leq i \leq 0\} = \emptyset\), so \(\forall\ x \in \emptyset\), \(f(x) \leq M\) is vacuously true for any \(M \in \mathbf{N}\).
Thus the base case holds.
Suppose inductively that for some \(n \in \mathbf{N}\), for any function \(f : \{i \in \mathbf{N} : 1 \leq i \leq n\} \to \mathbf{N}\) there exists a natural number \(M\) such that \(f(i) \leq M\) for all \(1 \leq i \leq n\).
Then for \(n++\), let \(f\) be a function, \(f : \{i \in \mathbf{N} : 1 \leq i \leq n++\} \to \mathbf{N}\).
By induction hypothesis, there exists \(M' \in \mathbf{N}\), \(f(i) \leq M'\) for all \(1 \leq i \leq n\).
Let \(M = M' + f(n++)\), then \(f(i) \leq M' \leq M\) for all \(1 \leq i \leq n\).
And \(f(n++) \leq f(n++) = M\).
So \(f(i) \leq M\) for all \(1 \leq i \leq n++\).
This close the induction.
\end{proof}

\begin{exercise}\label{ex 3.6.4}
Prove Proposition \ref{3.6.14}.
\end{exercise}

\begin{proof}
See Proposition \ref{3.6.14}.
\end{proof}

\begin{exercise}\label{ex 3.6.5}
Let \(A\) and \(B\) be sets.
Show that \(A \times B\) and \(B \times A\) have equal cardinality by constructing an explicit bijection between the two sets.
Then use Proposition \ref{3.6.14} to conclude an alternate proof of Lemma \ref{2.3.2}.
\end{exercise}

\begin{proof}
Let \(f : A \times B \to B \times A\) and setting \(f((a, b)) = (b, a)\), where \((a \in A) \land (b \in B)\).
We want to show that \(f\) is bijective.
Let \((a_1, b_1), (a_2, b_2) \in A \times B\) where \(f(a_1, b_1) = f(a_2, b_2)\).
Then \(f(a_1, b_1) = (b_1, a_1) = (b_2, a_2) = f(a_2, b_2)\), which means \(a_1 = a_2\) and \(b_1 = b_2\), so \((a_1, b_1) = (a_2, b_2)\), and \(f\) is injective.
Let \((b, a) \in B \times A\), there exists a order pair \((a, b) \in A \times B\) where \(f((a, b)) = (b, a)\), so \(f\) is surjective.
Since \(f\) is both injective and surjective, \(f\) is bijective, and \(\#(A \times B) = \#(B \times A)\).

By Proposition \ref{3.6.14}, \(\#(A \times B) = \#(A) \times \#(B)\) and \(\#(B \times A) = \#(B) \times \#(A)\).
Since \(\#(A), \#(B) \in \mathbf{N}\) and \(\#(A \times B) = \#(B \times A)\), \(\#(A) \times \#(B) = \#(B) \times \#(A)\).
This conclude the Lemma \ref{2.3.2}.
\end{proof}

\begin{exercise}\label{ex 3.6.6}
Let \(A, B, C\) be sets.
Show that the sets \((A^B)^C\) and \(A^{B \times C}\) have equal cardinality by constructing an explicit bijection between the two sets.
Conclude that \((a^b)^c = a^{bc}\) for any natural numbers \(a, b, c\).
Use a similar argument to also conclude \(a^b \times a^c = a^{b+c}\).
\end{exercise}

\begin{proof}
We first prove that \(\#((A^B)^C) = \#(A^{B \times C})\).
By Axiom \ref{3.10}, \((A^B)^C = \{f \mid f : C \to A^B\}\), \(A^B = \{f \mid f : B \to A\}\) and \(A^{B \times C} = \{f \mid f : B \times C \to A\}\).
We define a function \(F : (A^B)^C \to A^{B \times C}\) by setting \(F(f)(b, c) = (f(c))(b) \ \forall\ f \in (A^B)^C, b \in B\) and \(c \in C\).
We want to show that \(F\) is bijective.
Let \(f, g \in (A^B)^C\) and \(f \neq g\), then \(\forall\ c \in C\), \(f(c) \neq g(c)\), which means there exists \(b \in B\) such that \((f(c))(b) \neq (g(c))(b)\).
By the above definition, \(F(f)(b, c) = (f(c))(b) \neq (g(c))(b) = F(g)(b, c)\), so \(F(f) \neq F(g)\), and \(F\) is injective.
\(\forall\ h \in A^{B \times C}\), we can define \(h' \in (A^B)^C\) such that \((h'(c))(b) = h(b, c) \ \forall\ (b \in B) \land (c \in C)\).
Since such \(h'\) is well-defined and \(F(h') = h\), \(F\) is surjective.
Thus \(F\) is both injective and surjective, or equivalently \(F\) is bijective, and \(\#(A^{B \times C}) = \#((A^B)^C)\).

By Proposition \ref{3.6.14}, \(\#((A^B)^C) = \#(A^B)^{\#(C)} = (\#(A)^{\#(B)})^{\#(C)}\).
Also by Proposition \ref{3.6.14}, \(\#(A^{B \times C}) = \#(A)^{\#(B \times C)} = \#(A)^{(\#(B) \times \#(C))}\).
From previous prove that \(\#(A^{B \times C}) = \#((A^B)^C)\) and \(\#(A), \#(B), \#(C) \in \mathbf{N}\), we conclude that \((\#(A)^{\#(B)})^{\#(C)} = \#(A)^{(\#(B) \times \#(C))}\).

Now we prove that \(\#(A^B \times A^C) = \#(A^{B \cup C})\).
We define a function \(F : A^{B \cup C} \to A^B \times A^C\) by setting \(F(f) = (g, h)\) if \(f(b) = g(b)\) and \(f(c) = h(c)\), \(\forall\ f \in A^{B \cup C}, (g, h) \in A^B \times A^C, b \in B\) and \(c \in C\).
We want to show that \(F\) is bijective.
If \(f, f' \in A^{B \cup C}\) and \(F(f) = F(f') = (g, h) \in A^B \times A^C\), then \(\forall\ b \in B, c \in C\), \(f(b) = g(b) = f'(b)\) and \(f(c) = h(c) = f'(c)\), which means \(f = f'\), so \(F\) is injective.
\(\forall\ (p, q) \in A^B \times A^C\), we can define a function \(r \in A^{B \cup C}\) such that \(r(b) = p(b)\) and \(r(c) = q(c)\), \(\forall\ (b \in B) \land (c \in C)\).
Since such function \(r\) is well-defined and \(F(r) = (p, q)\), \(F\) is surjective.
Thus \(F\) is both injective and surjective, or equivalently \(F\) is bijective, and \(\#(A^B \times A^C) = \#(A^{B \cup C})\).

By Proposition \ref{3.6.14}, \(\#(A^B \times A^C) = \#(A^B) \times \#(A^C) = \#(A)^{\#(B)} \times \#(A)^{\#(C)}\).
Also by Proposition \ref{3.6.14}, \(\#(A^{B \cup C}) = \#(A)^{\#(B \cup C)} = \#(A)^{(\#(B) + \#(C))}\).
From previous prove that \(\#(A^B \times A^C) = \#(A^{B \cup C})\) and \(\#(A), \#(B), \#(C) \in \mathbf{N}\), we conclude that \(\#(A)^{\#(B)} \times \#(A)^{\#(C)} = \#(A)^{(\#(B) + \#(C))}\).
\end{proof}

\begin{exercise}\label{ex 3.6.7}
Let \(A\) and \(B\) be sets.
Let us say that \(A\) has \emph{lesser or equal} cardinality to \(B\) if there exists an injection \(f : A \to B\) from \(A\) to \(B\).
Show that if \(A\) and \(B\) are finite sets, then \(A\) has lesser or equal cardinality to \(B\) if and only if \(\#(A) \leq \#(B)\).
\end{exercise}

\begin{proof}
We first prove the necessary condition.
Let \(f : A \to B\) be a injective function.
By Proposition \ref{3.6.14}, \(\#(A) = \#(f(A))\).
Also by Proposition \ref{3.6.14}, \(f(A) \subseteq B \implies \#(f(A)) \leq \#(B)\).
So \(\#(A) = \#(f(A)) \leq \#(B)\).

Now we prove the sufficient condition.
Because \(\#(A) \leq \#(B)\), either \(\#(A) = \#(B)\) or \(\#(A) < \#(B)\) is true.
If \(\#(A) = \#(B)\), then \(\exists\ f \in B^A\) such that \(f\) is bijective, so \(f\) is injective.
If \(\#(A) < \#(B)\), then \(\forall\ f \in B^A\), \(f\) is not bijective.
Since \(f\) is not bijective, \(f\) is not injective or not surjective, and we choose \(f\) is not surjective.
So we can define  \(f : A \to B\) by setting \(x = x'\) if \(f(x) = f(x') \ \forall\ x, x' \in A\).
We can also set that \(\exists\ y \in B\) such that \(\forall\ x \in A\), \(f(x) \neq y\).
Thus \(f \in B^A\), \(f\) is not surjective and \(f\) is injective.

Since both the necessary and sufficient conditions are proved, we conclude that \(A\) has lesser or equal cardinality to \(B\) if and only if \(\#(A) \leq \#(B)\).
\end{proof}

\begin{exercise}\label{ex 3.6.8}
Let \(A\) and \(B\) be sets and \(A \neq \emptyset\) such that there exists an injection \(f : A \to B\) from \(A\) to \(B\) (i.e., \(A\) has lesser or equal cardinality to \(B\)).
Show that there exists a surjection \(g : B \to A\) from \(B\) to \(A\).
\end{exercise}

\begin{proof}
Because \(f\) is injective, so \(\#(A) \leq \#(B)\), and we divide into two cases.
If \(\#(A) = \#(B)\), then \(\exists\ g \in A^B\) where \(g\) is bijective, so \(g\) is surjective.
If \(\#(A) < \#(B)\), then \(\forall\ g \in A^B\) where \(g\) is not bijective.
Since \(g\) is not bijective, \(g\) is either not injective or not surjective, and we choose \(g\) is not injective.
So we can define \(g : B \to A\) by setting \(\exists\ x, x' \in B\), \(g(x) = g(x')\) and \(x \neq x'\).
We can also set that \(\forall\ y \in A\), there exists a \(x \in B\) such that \(g(x) = y\).
Thus \(g \in A^B\), \(g\) is not injective and \(g\) is surjective.
\end{proof}

\begin{exercise}\label{ex 3.6.9}
Let \(A\) and \(B\) be finite sets.
Show that \(A \cup B\) and \(A \cap B\) are also finite sets, and that \(\#(A) + \#(B) = \#(A \cup B) + \#(A \cap B)\).
\end{exercise}

\begin{proof}
Let \(\#(A) = n\).
We use induction on \(n\).
For \(n = 0\), \(A = \emptyset\), so \(\#(A) + \#(B) = \#(\emptyset) + \#(B) = 0 + \#(B) = \#(B) = \#(\emptyset \cup B) = \#(A \cup B) = \#(A \cup B) + 0 = \#(A \cup B) + \#(\emptyset) = \#(A \cup B) + \#(\emptyset \cap B) = \#(A \cup B) + \#(A \cap B)\).
Thus \(\#(A \cup B) = \#(B)\) and \(\#(A \cap B) = \#(\emptyset) = 0\), so \(A \cup B\) and \(A \cap B\) are finite, and the base case holds.
Suppose inductively that when \(\#(A) = n\), \(A \cup B\) and \(A \cap B\) are finite, and \(\#(A) + \#(B) = \#(A \cup B) + \#(A \cap B)\).
Then when \(\#(A) = n++\), let \(A = A' \cup \{a\}\), where \(a \notin A'\), so \(\#(A') = n\) and \(\#(\{a\}) = 1\).
By Proposition \ref{3.6.14}, \(\#(A) = \#(A') + \#(\{a\}) = \#(A') + 1\).
And by induction hypothesis, \(A' \cup B\) and \(A' \cap B\) are finite, and \(\#(A') + \#(B) = \#(A' \cup B) + \#(A' \cap B)\).
So \(\#(A) + \#(B) = \#(A') + 1 + \#(B) = \#(A' \cup B) + \#(A' \cap B) + 1\).
If \(a \in B\), then \(A' \cup B = A' \cup B \cup \{a\} = A \cup B\), which means \(\#(A' \cup B) = \#(A \cup B)\), so \(A \cup B\) is finite.
Also \((A' \cap B) \cup \{a\} = (A' \cup \{a\}) \cap (B \cup \{a\}) = A \cap B\), and by Proposition \ref{3.6.14}, \(a \notin A' \cap B \implies (A' \cap B) \cap \{a\} = \emptyset \implies \#(A' \cap B) + \#(\{a\}) = \#((A' \cap B) \cup \{a\}) = \#(A \cap B)\), so \(A \cap B\) is finite.
And we derive \(\#(A) + \#(B) = \#(A' \cup B) + \#(A' \cap B) + 1 = \#(A \cup B) + \#(A \cap B)\).
If \(a \notin B\), then \(a \notin A' \cup B \implies A' \cap B = (A' \cap B) \cup \emptyset = (A' \cap B) \cup (\{a\} \cap B) = (A' \cup \{a\}) \cap B = A \cap B\), which means \(\#(A' \cap B) = \#(A \cap B)\), so \(A \cap B\) is finite.
Also by Proposition \ref{3.6.14}, \((A' \cup B) \cap \{a\} = \emptyset \implies \#(A' \cup B) + \#(\{a\}) = \#(A' \cup B \cup \{a\}) = \#(A \cup B)\), so \(A \cup B\) is finite.
And we derive \(\#(A) + \#(B) = \#(A' \cup B) + \#(A' \cap B) + 1 = \#(A \cup B) + \#(A \cap B)\).
This close the induction.
\end{proof}

\begin{exercise}\label{ex 3.6.10}
Let \(A_1, \dots, A_n\) be finite sets such that \(\#(\bigcup_{i \in \{1, \dots, n\}} A_i) > n\).
Show that there exists \(i \in \{1, \dots, n\}\) such that \(\#(A_i) \geq 2\).
(This is known as the \emph{pigeonhole principle}.)
\end{exercise}

\begin{proof}
We use induction on \(n\).
We start from \(n = 1\) because for \(n = 0\), the statement is vacuously true.
For \(n = 1\), let \(\#(\bigcup_{i \in \{1\}} A_i) > 1\).
Because \(\bigcup_{i \in \{1\}} A_i = A_1\), so \(\#(A_1) > 1\), or equivalently, \(\#(A_1) \geq 2\).
Thus the base case holds.
Suppose inductively that for some \(n \in \mathbf{N}\), if \(\#(\bigcup_{i \in \{1, \dots, n\}} A_i) > n\), then \(\exists\ i \in \{1, \dots, n\}\) such that \(\#(A_i) \geq 2\).
Then for \(n++\), let \(\#(\bigcup_{i \in \{1, \dots, n++\}} A_i) > n++\), and we can divide into three cases.

If \(\#(A_{n++}) = 0\), then \(\bigcup_{i \in \{1, \dots, n++\}} A_i = \bigcup_{i \in \{1, \dots, n\}} A_i\).
So \(\#(\bigcup_{i \in \{1, \dots, n\}} A_i) > n++ > n\), and by induction hypothesis \(\exists\ i \in \{1, \dots, n\}\) such that \(\#(A_i) \geq 2\), which means \(\exists\ i \in \{1, \dots, n++\}\) such that \(\#(A_i) \geq 2\).
Thus we are done in case \(\#(A_{n++}) = 0\).

If \(\#(A_{n++}) = 1\), let \(x \in A_{n++}\).
If \(x \in \bigcup_{i \in \{1, \dots, n\}} A_i\), then \(\bigcup_{i \in \{1, \dots, n++\}} A_i = \bigcup_{i \in \{1, \dots, n\}} A_i\), so \(\#(\bigcup_{i \in \{1, \dots, n\}} A_i) > n++ > n\), and by induction hypothesis \(\exists\ i \in \{1, \dots, n\}\) such that \(\#(A_i) \geq 2\), which means \(\exists\ i \in \{1, \dots, n++\}\) such that \(\#(A_i) \geq 2\).
If \(x \notin \bigcup_{i \in \{1, \dots, n\}} A_i\), then by Proposition \ref{3.6.14}, \((\bigcup_{i \in \{1, \dots, n\}} A_i) \cap A_{n++} = \emptyset \implies \#(\bigcup_{i \in \{1, \dots, n\}} A_i) + \#(A_{n++}) = \#((\bigcup_{i \in \{1, \dots, n\}} A_i) \cup A_{n++}) = \#(\bigcup_{i \in \{1, \dots, n++\}} A_i)\).
And because \(\#(\bigcup_{i \in \{1, \dots, n++\}} A_i) > n++\), and \(\#(\bigcup_{i \in \{1, \dots, n\}} A_i) + \#(A_{n++}) = \#(\bigcup_{i \in \{1, \dots, n\}} A_i) + 1 > n++\), so \(\#(\bigcup_{i \in \{1, \dots, n\}} A_i) > n\).
By induction hypothesis \(\exists\ i \in \{1, \dots, n\}\) such that \(\#(A_i) \geq 2\), which means \(\exists\ i \in \{1, \dots, n++\}\) such that \(\#(A_i) \geq 2\).
Thus we are done in case \(\#(A_{n++}) = 1\).

If \(\#(A_{n++}) \geq 2\), then \(\exists i \in \{1, \dots, n++\}\) such that \(\#(A_i) \geq 2\).
Thus we are done in case \(\#(A_{n++}) = 2\).
This close the induction.
\end{proof}