\section{The rationals}

\begin{definition}\label{4.2.1}
A \emph{rational number} is an expression of the form \(a // b\), where \(a\) and \(b\) are integers and \(b\) is non-zero;
\(a // 0\) is not considered to be a rational number.
Two rational numbers are considered to be equal, \(a // b = c // d\), if and only if \(ad = cb\).
The set of all rational numbers is denoted \(\mathds{Q}\).
\end{definition}

\begin{note}
There is no reasonable way we can divide by zero, since one cannot have both the identities \((a / b) \times b = a\) and \(c \times 0 = 0\) hold simultaneously if \(b\) is allowed to be zero and \(a\) is non-zero.
Similarly, the identities \(a / a = 1\) and \(2 \times (a / a) = (2 \times a) / a\) cannot hold simultaneously if \(0 / 0\) is defined.
However, we can eventually get a reasonable notion of dividing by a quantity which approaches zero
- think of L'H\^opital's rule (see Section 10.5), which suffices for doing things like defining differentiation.
\end{note}

\begin{additional corollary}\label{ac 4.2.1}
The definition of equality for the rational numbers is reflexive, symmetric and transitive.
\end{additional corollary}

\begin{proof}
We first prove the reflexivity of the rational numbers.
Let \(a \in \mathds{Q}\) and \(a = a_1 // a_2\), where \(a_1, a_2 \in \mathds{Z}\) and \(a_2 \neq 0\).
By Lemma \ref{4.1.3}, multiplication of integers is well-defined, so \(a_1a_2 = a_1a_2\).
By Definition \ref{4.2.1}, \(a_1a_2 = a_1a_2 \implies a_1 // a_2 = a_1 // a_2\).

Next we prove the symmetry of the rational numbers.
Let \(a, b \in \mathds{Q}\), \(a = a_1 // a_2\) and \(b = b_1 // b_2\), where \(a_1, a_2, b_1, b_2 \in \mathds{Z}\), \(a_2 \neq 0\) and \(b_2 \neq 0\).
So if \(a = b\), then
\begin{align*}
& a = b \\
\implies & a_1 // a_2 = b_1 // b_2 & \text{(by Definition \ref{4.2.1})} \\
\implies & a_1b_2 = b_1a_2 & \text{(by Definition \ref{4.2.1})} \\
\implies & b_1a_2 = a_1b_2 \\
\implies & b_1 // b_2 = a_1 // a_2 & \text{(by Definition \ref{4.2.1})} \\
\implies & b = a. & \text{(by Definition \ref{4.2.1})}
\end{align*}

Finally we prove the transitivity of the rational numbers.
Let \(a, b, c \in \mathds{Q}\), \(a = a_1 // a_2\), \(b = b_1 // b_2\) and \(c = c_1 // c_2\), where \(a_1, a_2, b_1, b_2, c_1, c_2 \in \mathds{Z}\), \(a_2 \neq 0\), \(b_2 \neq 0\) and \(c_2 \neq 0\).
So if \(a = b\) and \(b = c\), then
\begin{align*}
& a = b \\
\implies & a_1 // a_2 = b_1 // b_2 & \text{(by Definition \ref{4.2.1})} \\
\implies & a_1b_2 = b_1a_2. & \text{(by Definition \ref{4.2.1})} \\
& b = c \\
\implies & b_1 // b_2 = c_1 // c_2 & \text{(by Definition \ref{4.2.1})} \\
\implies & b_1c_2 = c_1b_2 & \text{(by Definition \ref{4.2.1})} \\
\implies & a_2(b_1c_2) = a_2(c_1b_2) & \text{(by Lemma \ref{4.1.3})} \\
\implies & (a_2b_1)c_2 = a_2(c_1b_2) & \text{(by Proposition \ref{4.1.6})} \\
\implies & (b_1a_2)c_2 = a_2(c_1b_2) & \text{(by Proposition \ref{4.1.6})} \\
\implies & (a_1b_2)c_2 = a_2(c_1b_2) & \text{(by the given condition)} \\
\implies & a_1(b_2c_2) = a_2(c_1b_2) & \text{(by Proposition \ref{4.1.6})} \\
\implies & a_1(c_2b_2) = a_2(c_1b_2) & \text{(by Proposition \ref{4.1.6})} \\
\implies & (a_1c_2)b_2 = (a_2c_1)b_2 & \text{(by Proposition \ref{4.1.6})} \\
\implies & a_1c_2 = a_2c_1 & \text{(by Corollary \ref{4.1.9})} \\
\implies & a_1 // a_2 = c_1 // c_2 & \text{(by Definition \ref{4.2.1})} \\
\implies & a = c. & \text{(by Definition \ref{4.2.1})}
\end{align*}
\end{proof}

\begin{definition}\label{4.2.2}
If \(a // b\) and \(c // d\) are rational numbers, we define their sum
\[
    (a // b) + (c // d) \coloneqq (ad + bc) // (bd)
\]
their product
\[
    (a // b) \times (c // d) \coloneqq (ac) // (bd)
\]
and the negation
\[
    -(a // b) \coloneqq (-a) // b.
\]
\end{definition}

\begin{note}
If \(b\) and \(d\) are non-zero, then \(bd\) is also non-zero, by Proposition \ref{4.1.8}, so the sum or product of two rational numbers remains a rational number.
\end{note}

\begin{lemma}\label{4.2.3}
The sum, product, and negation operations on rational numbers are well-defined, in the sense that if one replaces \(a // b\) with another rational number \(a' // b'\) which is equal to \(a // b\), then the output of the above operations remains unchanged, and similarly for \(c // d\).
\end{lemma}

\begin{proof}
We first prove the addition on rationals numbers is well-defined.
Suppose \(a // b = a' // b'\), so that \(b\) and \(b'\) are non-zero and \(ab' = a'b\).
We now show that \((a // b) + (c // d) = (a' // b') + (c // d)\).
By Definition \ref{4.2.2}, the left-hand side is \((ad + bc) // bd\) and the right-hand side is \((a'd + b'c) // b'd\).
So by Definition \ref{4.2.1}, we have to show that
\[
    (ad + bc)b'd = (a'd + b'c)bd,
\]
which expands to
\[
    ab'd^2 + bb'cd = a'bd^2 + bb'cd.
\]
But since \(ab' = a'b\), the claim follows.
Similarly suppose \(c // d = c' // d'\), so that \(d\) and \(d'\) are non-zero and \(cd' = c'd\).
We now show that \((a // b) + (c // d) = (a // b) + (c' // d')\).
By Definition \ref{4.2.2}, the left-hand side is \((ad + bc) // bd\) and the right-hand side is \((ad' + bc') // bd'\).
So by Definition \ref{4.2.1}, we have to show that
\[
    (ad + bc)bd' = (ad' + bc')bd,
\]
which expands to
\[
    abdd' + b^2cd' = abdd' + b^2c'd.
\]
But since \(cd' = c'd\), the claim follows.

Next we prove the multiplication on rationals numbers is well-defined.
Suppose \(a // b = a' // b'\), so that \(b\) and \(b'\) are non-zero and \(ab' = a'b\).
We now show that \((a // b) \times (c // d) = (a' // b') \times (c // d)\).
By Definition \ref{4.2.2}, the left-hand side is \((ac) // (bd)\) and the right-hand side is \((a'c) // (b'd)\).
So by Definition \ref{4.2.1}, we have to show that
\[
    (ac)(b'd) = (a'c)(bd),
\]
which is equivalent to
\[
    ab'cd = a'bcd.
\]
But since \(ab' = a'b\), the claim follows.
Similary suppose \(c // d = c' // d'\), so that \(d\) and \(d'\) are non-zero and \(cd' = c'd\).
We now show that \((a // b) \times (c // d) = (a // b) \times (c' // d')\).
By Definition \ref{4.2.2}, the left-hand side is \((ac) // (bd)\) and the right-hand side is \((ac') // (bd')\).
So by Definition \ref{4.2.1}, we have to show that
\[
    (ac)(bd') = (ac')(bd),
\]
which is equivalent to
\[
    abcd' = abc'd.
\]
But since \(cd' = c'd\), the claim follows.

Finally we prove the negation on rationals numbers is well-defined.
Suppose \(a // b = a' // b'\), so that \(b\) and \(b'\) are non-zero and \(ab' = a'b\).
We now show that \(-(a // b) = -(a' // b')\).
By Definition \ref{4.2.2}, the left-hand side is \((-a) // b\) and the right-hand side is \((-a') // b'\).
So by Definition \ref{4.2.1}, we have to show that
\[
    (-a)b' = (-a')b,
\]
which by Exercise \ref{ex 4.1.3} is equivalent to
\[
    (-1)ab' = (-1)a'b.
\]
But since \(ab' = a'b\), the claim follows.
\end{proof}

\begin{note}
The rational numbers \(a // 1\) behave in a manner identical to the integers \(a\):
\begin{align*}
    (a // 1) + (b // 1) &= (a + b) // 1; \\
    (a // 1) \times (b // 1) &= (ab // 1); \\
    -(a // 1) &= (-a) // 1.
\end{align*}
Also, \(a // 1\) and \(b // 1\) are only equal when \(a\) and \(b\) are equal.
Because of this, we will identify \(a\) with \(a // 1\) for each integer \(a\): \(a \equiv a // 1\);
the above identities then guarantee that the arithmetic of the integers is consistent with the arithmetic of the rationals.
Thus just as we embedded the natural numbers inside the integers, we embed the integers inside the rational numbers.
In particular, all natural numbers are rational numbers, for instance \(0\) is equal to \(0 // 1\) and \(1\) is equal to \(1 // 1\).
\end{note}

\begin{note}
Observe that a rational number \(a // b\) is equal to \(0 = 0 // 1\) if and only if \(a \times 1 = b \times 0\), i.e., if the numerator \(a\) is equal to \(0\).
Thus if \(a\) and \(b\) are non-zero then so is \(a // b\).
\end{note}

\begin{note}
We now define a new operation on the rationals: reciprocal.
If \(x = a // b\) is a non-zero rational (so that \(a, b \neq 0\)) then we define the \emph{reciprocal} \(x^{-1}\) of \(x\) to be the rational number \(x^{-1} \coloneqq b // a\).
\end{note}

\begin{additional corollary}\label{ac 4.2.2}
The reciprocal operation on rational numbers is consistent with Definition \ref{4.2.1}:
if two rational numbers \(a // b\), \(a' // b'\) are equal, then their reciprocals are also equal.
We however leave the reciprocal of \(0\) undefined.
\end{additional corollary}

\begin{proof}
By Definition \ref{4.2.1}, \(b \neq 0\) and \(b' \neq 0\).
By the definition of reciprocal, \(a \neq 0\) and \(a' \neq 0\).
\begin{align*}
& a // b = a' // b' \\
\implies & ab' = a'b & \text{(by Definition \ref{4.2.1})} \\
\implies & b'a = ba' & \text{(by Proposition \ref{4.1.6})} \\
\implies & b' // a' = b // a. & \text{(by Definition \ref{4.2.1})}
\end{align*}
\end{proof}

\begin{note}
In contrast to reciprocal, an operation such as ``numerator'' is not well-defined:
the rationals \(3 // 4\) and \(6 // 8\) are equal, but have unequal numerators, so we have to be careful when referring to such terms as ``the numerator of \(x\)''.
\end{note}

\begin{proposition}[Laws of algebra for rationals]\label{4.2.4}
Let \(x\), \(y\), \(z\) be rationals.
Then the following laws of algebra hold:
\begin{align*}
x + y &= y + x \\
(x + y) + z &= x + (y + z) \\
x + 0 = 0 + x &= x \\
x + (-x) = (-x) + x &= 0 \\
xy &= yx \\
(xy)z &= x(yz) \\
x1 = 1x &= x \\
x(y + z) &= xy + xz \\
(y + z)x &= yx + zx.
\end{align*}
If \(x\) is non-zero, we also have
\[
    xx^{-1} = x^{-1}x = 1.
\]
\end{proposition}

\begin{proof}
To prove this identity, one writes \(x = a // b\), \(y = c // d\), \(z = e // f\) for some integers \(a\), \(c\), \(e\) and non-zero integers \(b\), \(d\), \(f\), and verifies each identity in turn using the algebra of the integers.

First we prove \(x + y = y + x\).
\begin{align*}
x + y &= (a // b) + (c // d) & \text{(by the given condition)} \\
&= (ad + bc) // bd & \text{(by Definition \ref{4.2.2})} \\
&= (bc + ad) // bd & \text{(by Proposition \ref{4.1.6})} \\
&= (cb + da) // db & \text{(by Proposition \ref{4.1.6})} \\
&= (c // d) + (a // b) & \text{(by Definition \ref{4.2.2})} \\
&= y + x. & \text{(by the given condition)}
\end{align*}

Next we prove \((x + y) + z = x + (y + z)\).
\begin{align*}
(x + y) + z &= ((a // b) + (c // d)) + (e // f) & \text{(by the given condition)} \\
&= ((ad + bc) // bd) + (e // f) & \text{(by Definition \ref{4.2.2})} \\
&= ((ad + bc)f + (bd)e) // (bd)f & \text{(by Definition \ref{4.2.2})} \\
&= ((ad)f + (bc)f + (bd)e) // (bd)f & \text{(by Proposition \ref{4.1.6})} \\
&= (a(df) + b(cf) + b(de)) // b(df) & \text{(by Proposition \ref{4.1.6})} \\
&= (a(df) + b(cf + de)) // b(df) & \text{(by Proposition \ref{4.1.6})} \\
&= (a // b) + ((cf + de) // df) & \text{(by Definition \ref{4.2.2})} \\
&= (a // b) + ((c // d) + (e // f)) & \text{(by Definition \ref{4.2.2})} \\
&= x + (y + z). & \text{(by the given condition)}
\end{align*}

Next we prove \(x + 0 = 0 + x = x\).
Because we already prove that \(x + y = y + x\), so \(x + 0 = 0 + x\).
Thus we only need to show that \(x + 0 = x\).
\begin{align*}
x + 0 &= (a // b) + (0 // 1) & \text{(by the given condition)} \\
&= (a1 + b0) // b1 & \text{(by Definition \ref{4.2.2})} \\
&= (a + 0) // b & \text{(by Proposition \ref{4.1.6})} \\
&= a // b & \text{(by Proposition \ref{4.1.6})} \\
&= x. & \text{(by the given condition)}
\end{align*}

Next we prove \(x + (-x) = (-x) + x = 0\).
Because we already prove that \(x + y = y + x\), so \(x + (-x) = (-x) + x\).
Thus we only need to show that \(x + (-x) = 0\).
\begin{align*}
x + (-x) &= (a // b) + (-(a // b)) & \text{(by the given condition)} \\
&= (a // b) + ((-a) // b) & \text{(by Definition \ref{4.2.2})} \\
&= (ab + b(-a)) // b^2 & \text{(by Definition \ref{4.2.2})} \\
&= (ab + (-a)b) // b^2 & \text{(by Proposition \ref{4.1.6})} \\
&= (ab + ((-1)a)b) // b^2 & \text{(by Exercise \ref{ex 4.1.3})} \\
&= (ab + (-1)(ab)) // b^2 & \text{(by Proposition \ref{4.1.6})} \\
&= (ab + (-(ab)) // b^2 & \text{(by Exercise \ref{ex 4.1.3})} \\
&= 0 // b^2 & \text{(by Proposition \ref{4.1.6})} \\
&= 0.
\end{align*}

Next we prove \(xy = yx\).
\begin{align*}
xy &= (a // b) \times (c // d) & \text{(by the given condition)} \\
&= ac // bd & \text{(by Definition \ref{4.2.2})} \\
&= ca // db & \text{(by Proposition \ref{4.1.6})} \\
&= (c // d) \times (a // b) & \text{(by Definition \ref{4.2.2})} \\
&= yx. & \text{(by the given condition)}
\end{align*}

Next we prove \((xy)z = x(yz)\).
\begin{align*}
(xy)z &= ((a // b) \times (c // d)) \times (e // f) & \text{(by the given condition)} \\
&= (ac // bd) \times (e // f) & \text{(by Definition \ref{4.2.2})} \\
&= (ac)e // (bd)f & \text{(by Definition \ref{4.2.2})} \\
&= a(ce) // b(df) & \text{(by Proposition \ref{4.1.6})} \\
&= (a // b) \times (ce // df) & \text{(by Definition \ref{4.2.2})} \\
&= (a // b) \times ((c // d) \times (e // f)) & \text{(by Definition \ref{4.2.2})} \\
&= x(yz). & \text{(by the given condition)}
\end{align*}

Next we prove \(x1 = 1x = x\).
Because we already prove that \(xy = yx\), so \(x1 = 1x\).
Thus we only need to show that \(x1 = x\).
\begin{align*}
x1 &= (a // b) \times (1 // 1) & \text{(by the given condition)} \\
&= a1 // b1 & \text{(by Definition \ref{4.2.2})} \\
&= a // b & \text{(by Proposition \ref{4.1.6})} \\
&= x. & \text{(by the given condition)}
\end{align*}

Next we prove \(x(y + z) = xy + xz\).
\begin{align*}
x(y + z) &= (a // b) \times ((c // d) + (e // f)) & \text{(by the given condition)} \\
&= (a // b) \times ((cf + de) // df) & \text{(by Definition \ref{4.2.2})} \\
&= a(cf + de) // b(df) & \text{(by Definition \ref{4.2.2})} \\
&= (a(cf) + a(de)) // b(df) & \text{(by Proposition \ref{4.1.6})} \\
&= b(a(cf) + a(de)) // b^2(df) & \text{(by Proposition \ref{4.2.1})} \\
&= (b(a(cf)) + b(a(de))) // b^2(df) & \text{(by Proposition \ref{4.1.6})} \\
&= ((ba)(cf) + (ba)(de)) // b^2(df) & \text{(by Proposition \ref{4.1.6})} \\
&= ((b(ac))f + (ba)(de)) // (b(bd))f & \text{(by Proposition \ref{4.1.6})} \\
&= (((ac)b)f + (ba)(ed)) // ((bd)b)f & \text{(by Proposition \ref{4.1.6})} \\
&= ((ac)(bf) + (b(ae))d) // (bd)(bf) & \text{(by Proposition \ref{4.1.6})} \\
&= ((ac)(bf) + b((ae)d)) // (bd)(bf) & \text{(by Proposition \ref{4.1.6})} \\
&= ((ac)(bf) + b(d(ae))) // (bd)(bf) & \text{(by Proposition \ref{4.1.6})} \\
&= ((ac)(bf) + (bd)(ae)) // (bd)(bf) & \text{(by Proposition \ref{4.1.6})} \\
&= ((ac)(bf) + (bd)(ae)) // (bd)(bf) & \text{(by Definition \ref{4.2.1})} \\
&= (ac // bd) + (ae // bf) & \text{(by Definition \ref{4.2.2})} \\
&= ((a // b) \times (c // d)) + ((a // b) \times (e // f)) & \text{(by Definition \ref{4.2.2})} \\
&= xy + xz. & \text{(by the given condition)}
\end{align*}

Next we prove \((y + z)x = yx + zx\).
Because we already prove \(pq = qp\) and \(p(q + r) = pq + pr\), where \(p, q, r \in \mathds{Q}\),
so
\begin{align*}
(y + z)x &= x(y + z) & (\because pq = qp) \\
&= xy + xz & (\because p(q + r) = pq + pr) \\
&= yx + zx. & (\because pq = qp)
\end{align*}

Finally we prove \(xx^{-1} = x^{-1}x = 1\).
Because we already prove that \(xy = yx\), so \(xx^{-1} = x^{-1}x\).
Thus we only need to show that \(xx^{-1} = 1\).
\begin{align*}
xx^{-1} &= (a // b) \times (b // a) & \text{(by the given condition)} \\
&= ab // ba & \text{(by Definition \ref{4.2.2})} \\
&= ab // ab & \text{(by Proposition \ref{4.1.6})} \\
&= 1 // 1  & \text{(by Definition \ref{4.2.1})} \\
&= 1.
\end{align*}
\end{proof}

\begin{remark}\label{4.2.5}
The above set (Proposition \ref{4.2.4}) of ten identities have a name;
they are asserting that the rationals \(\mathds{Q}\) form a \emph{field}.
This is better than being a commutative ring because of the tenth identity \(xx^{-1} = x^{-1}x = 1\).
Note that Proposition \ref{4.2.4} supercedes Proposition \ref{4.1.6}.
\end{remark}

\begin{note}
We can now define the \emph{quotient} \(x / y\) of two rational numbers \(x\) and \(y\), \emph{provided that} \(y\) is non-zero, by the formula
\[
    x / y \coloneqq x \times y^{-1}.
\]
\end{note}

\begin{note}
Using the above formula, it is easy to see that \(a / b = a // b\) for every integer \(a\) and every non-zero integer \(b\).
Thus we can now discard the \(//\) notation, and use the more customary \(a / b\) instead of \(a // b\).
\end{note}

\begin{note}
In a similar spirit, we define subtraction on the rationals by the formula
\[
    x - y \coloneqq x + (-y),
\]
just as we did with the integers.
\end{note}

\begin{definition}\label{4.2.6}
A rational number \(x\) is said to be \emph{positive} iff we have \(x = a / b\) for some positive integers \(a\) and \(b\).
It is said to be \emph{negative} iff we have \(x = -y\) for some positive rational \(y\)
(i.e., \(x = (-a) / b\) for some positive integers \(a\) and \(b\)).
\end{definition}

\begin{note}
Thus for instance, every positive integer is a positive rational number, and every negative integer is a negative rational number, so our new definition is consistent with our old one.
\end{note}

\begin{additional corollary}\label{ac 4.2.3}
Let \(x = a / b\) be a rational number where \(a, b \in \mathds{Z}\) and \(b \neq 0\).
Then
\[
    -x = (-a) / b = a / (-b) = (-1)(a / b) = (-1)x.
\]
\end{additional corollary}

\begin{proof}
\begin{align*}
-x &= -(a / b) & \text{(by given condition)} \\
&= (-a) / b & \text{(by Definition \ref{4.2.2})} \\
&= ((-1)a) / b & \text{(by Exercise \ref{ex 4.1.3})} \\
&= ((-1)a) / 1b & \text{(by Proposition \ref{4.1.6})} \\
&= ((-1) / 1) \times (a / b) & \text{(by Definition \ref{4.2.2})} \\
&= (-1)(a / b) \\
&= (-1)x \\
&= (1 / (-1)) \times (a / b) & \text{(by Definition \ref{4.2.1})} \\
&= 1a / (-1)b & \text{(by Definition \ref{4.2.2})} \\
&= a / (-1)b & \text{(by Proposition \ref{4.1.6})} \\
&= a / (-b). & \text{(by Exercise \ref{ex 4.1.3})}
\end{align*}
\end{proof}

\begin{lemma}[Trichotomy of rationals]\label{4.2.7}
Let \(x\) be a rational number.
Then exactly one of the following three statements is true:
\begin{enumerate*}
    \item \(x\) is equal to \(0\).
    \item \(x\) is a positive rational number.
    \item \(x\) is a negative rational number.
\end{enumerate*}
\end{lemma}

\begin{proof}
We first show that at least one of (a), (b), (c) is true.
Let \(x = a / b\), where \(a, b \in \mathds{Z}\) and \(b \neq 0\).
By Lemma \ref{4.1.11}, \(a\) can only satisified one of the following three statements:
\(a = 0\), \(a > 0\) and \(a < 0\).
Similarly, \(b\) can only satisified one of the following two statements:
\(b > 0\) and \(b < 0\).
So
\begin{enumerate}[label=(\Roman*)]
    \item If \(a = 0\), then \(x = 0 / b = 0\).
    \item If \(a > 0\), then we need to consider \(b\).
    \begin{enumerate}[label=(\roman*)]
        \item If \(b > 0\), then by Definition \ref{4.2.6}, \(x\) is positive.
        \item If \(b < 0\), then \(b = -c\) by Definition \ref{4.1.4}, where \(c \in \mathds{Z}^+\).
        So \(a / b = a / (-c) = (-a) / c\) by Additional Corollary \ref{ac 4.2.3}, which means \(x\) is negative by Definition \ref{4.2.6}.
    \end{enumerate}
    \item If \(a < 0\), then \(a = -c\) by Definition \ref{4.1.4}, where \(c \in \mathds{Z}^+\).
    Now we consider \(b\).
    \begin{enumerate}[label=(\roman*)]
        \item If \(b > 0\), then \(a / b = (-c) / b\), which means \(x\) is negative by Definition \ref{4.2.6}.
        \item If \(b < 0\), then \(b = -d\) by Definition \ref{4.1.4}, where \(d \in \mathds{Z}^+\).
        So \(a / b = (-c) / (-d) = (-1)((-c) / d) = (-1)(-1)(c / d) = c / d\) by Additional Corollary \ref{ac 4.2.3}, which means \(x\) is positive by Definition \ref{4.2.6}.
    \end{enumerate}
\end{enumerate}

Now we show that at most one of (a), (b), (c) is true.
\begin{enumerate}[label=(\Roman*)]
    \item If \(x\) is both positive and \(0\), then by Definition \ref{4.2.6} and \ref{4.2.1}, \(x = a / b = 0 / 1\), where \(a, b \in \mathds{Z}^+\).
    But \(a / b = 0 / 1\) means \(a = 0\), contradicted to \(a\) is positive.
    \item If \(x\) is both negative and \(0\), then by Definition \ref{4.2.6} and \ref{4.2.1}, \(x = (-a) / b = 0 / 1\), where \(a, b \in \mathds{Z}^+\).
    But \((-a) / b = 0 / 1\) means \(-a = 0\), contradicted to \(a\) is positive.
    \item If \(x\) is both positive and negative, then by Definition \ref{4.2.6} and \ref{4.2.1}, \(x = a / b = (-c) / d\), where \(a, b, c, d \in \mathds{Z}^+\).
    But \(a / b = (-c) / d\) means \(ad = b(-c) = b((-1)c) = (b(-1))c = ((-1)b)c = (-1)(bc)\).
    By Lemma \ref{2.3.3}, \((-1)(bc)\) is positive, a contradiction.
\end{enumerate}
So no more than one of (a), (b), (c) is true at the same time.
\end{proof}

\begin{definition}[Ordering of the rationals]\label{4.2.8}
Let \(x\) and \(y\) be rational numbers.
We say that \(x > y\) iff \(x - y\) is a positive rational number, and \(x < y\) iff \(x - y\) is a negative rational number.
We write \(x \geq y\) iff either \(x > y\) or \(x = y\), and similarly define \(x \leq y\) iff either \(x < y\) or \(x = y\).
\end{definition}

\begin{additional corollary}\label{ac 4.2.4}
If \(x\) and \(y\) are two positive rationals, then \(x + y\) is also a positive rational number.
\end{additional corollary}

\begin{proof}
By Definition \ref{4.2.6}, \(x = a / b\) and \(y = c / d\), where \(a, b, c, d \in \mathds{Z}^+\).
Then by Definition \ref{4.2.2}, \(x + y = (ad + bc) / bd\).
By Lemma \ref{2.3.2}, \(ad, bc, bd \in \mathds{Z}^+\).
And by Proposition \ref{2.2.8}, \(ad + bc \in \mathds{Z}^+\).
Thus by Definition \ref{4.2.6}, \(x + y\) is a positive rational number.
\end{proof}

\begin{additional corollary}\label{ac 4.2.5}
If \(x\) and \(y\) are two positive rationals, then \(xy\) is also a positive rational number.
\end{additional corollary}

\begin{proof}
By Definition \ref{4.2.6}, \(x = a / b\) and \(y = c / d\), where \(a, b, c, d \in \mathds{Z}^+\).
Then by Definition \ref{4.2.2}, \(xy = ac / bd\).
By Lemma \ref{2.3.2}, \(ac, bd \in \mathds{Z}^+\), thus by Definition \ref{4.2.6}, \(xy\) is a positive rational number.
\end{proof}

\begin{proposition}[Basic properties of order on the rationals]\label{4.2.9}
Let \(x\), \(y\), \(z\) be rational numbers.
Then the following properties hold.
\begin{enumerate}
    \item (Order trichotomy)
    Exactly one of the three statements \(x = y\), \(x < y\), or \(x > y\) is true.
    \item (Order is anti-symmetric)
    One has \(x < y\) if and only if \(y > x\).
    \item (Order is transitive)
    If \(x < y\) and \(y < z\), then \(x < z\).
    \item (Addition preserves order)
    If \(x < y\), then \(x + z < y + z\).
    \item (Positive multiplication preserves order)
    If \(x < y\) and \(z\) is positive, then \(xz < yz\).
\end{enumerate}
\end{proposition}

\begin{proof}{(a)}
We first show that at least one of the three statements \(x = y\), \(x > y\) or \(x < y\) is true.
By Lemma \ref{4.2.7}, \(x - y\) satisify exactly one of the three statements \(x - y = 0\), \(x - y\) is positive, or \(x - y\) is negative.
\begin{enumerate}[label=(\roman*)]
    \item If \(x - y = 0\), then \(x - y = 0 \implies x - y + y = y \implies x = y\) by Proposition \ref{4.2.4}.
    \item If \(x - y\) is positive, then by Definition \ref{4.2.8} \(x > y\).
    \item If \(x - y\) is negative, then by Definition \ref{4.2.8} \(x < y\).
\end{enumerate}
Thus at least one of the three statements \(x = y\), \(x > y\) or \(x < y\) is true.

Now we show that at most one of the three statements \(x = y\), \(x > y\) or \(x < y\) is true.
\begin{enumerate}[label=(\roman*)]
    \item If \(x = y\) and \(x > y\), then \(x - y\) is both \(0\) and positive, contradict to Lemma \ref{4.2.7}.
    \item If \(x = y\) and \(x < y\), then \(x - y\) is both \(0\) and negative, contradict to Lemma \ref{4.2.7}.
    \item If \(x > y\) and \(x < y\), then \(x - y\) is both positive and negative, contradict to Lemma \ref{4.2.7}.
\end{enumerate}
Thus at most one of the three statements \(x = y\), \(x > y\) or \(x < y\) is true.
\end{proof}

\begin{proof}{(b)}
\begin{align*}
x < y & \iff x - y \text{ is negative} & \text{(by Definition \ref{4.2.8})} \\
& \iff x - y = (-a) / b, \text{ where } a, b \in \mathds{Z}^+ & \text{(by Definition \ref{4.2.6})} \\
& \iff (x - y) + y = (-a) / b + y & \text{(by Proposition \ref{4.2.4})} \\
& \iff x + (-y + y) = (-a) / b + y & \text{(by Proposition \ref{4.2.4})} \\
& \iff x + 0 = (-a) / b + y & \text{(by Proposition \ref{4.2.4})} \\
& \iff x = (-a) / b + y & \text{(by Proposition \ref{4.2.4})} \\
& \iff x + (-x) = ((-a) / b + y) + (-x) & \text{(by Proposition \ref{4.2.4})} \\
& \iff 0 = ((-a) / b + y) + (-x) & \text{(by Proposition \ref{4.2.4})} \\
& \iff a / b + 0 = a / b + (((-a) / b + y) + (-x)) & \text{(by Proposition \ref{4.2.4})} \\
& \iff a / b = a / b + (((-a) / b + y) + (-x)) & \text{(by Proposition \ref{4.2.4})} \\
& \iff a / b = (a / b + ((-a) / b + y)) + (-x) & \text{(by Proposition \ref{4.2.4})} \\
& \iff a / b = ((a / b + (-a) / b) + y) + (-x) & \text{(by Proposition \ref{4.2.4})} \\
& \iff a / b = (0 + y) + (-x) & \text{(by Proposition \ref{4.2.4})} \\
& \iff a / b = y + (-x) & \text{(by Proposition \ref{4.2.4})} \\
& \iff a / b = y - x \\
& \iff y - x \text{ is positive} & \text{(by Definition \ref{4.2.6})} \\
& \iff y > x. & \text{(by Definition \ref{4.2.8})}
\end{align*}
\end{proof}

\begin{proof}{(c)}
By the given conditions and Definition \ref{4.2.8}, \(x < y \implies x - y\) is negative and \(y < z \implies y - z\) is negative.
Let \(x - y = -p\) and \(y - z = -q\), where \(p, q\) are positive rationals.
Then
\begin{align*}
& x - y = -p \\
\implies & (x - y) + y = (-p) + y & \text{(by Proposition \ref{4.2.4})} \\
\implies & x + (-y + y) = (-p) + y & \text{(by Proposition \ref{4.2.4})} \\
\implies & x + 0 = (-p) + y & \text{(by Proposition \ref{4.2.4})} \\
\implies & x = (-p) + y & \text{(by Proposition \ref{4.2.4})} \\
\implies & p + x = p + ((-p) + y) & \text{(by Proposition \ref{4.2.4})} \\
\implies & p + x = (p + (-p)) + y & \text{(by Proposition \ref{4.2.4})} \\
\implies & p + x = 0 + y & \text{(by Proposition \ref{4.2.4})} \\
\implies & p + x = y. & \text{(by Proposition \ref{4.2.4})} \\
& y - z = -q \\
\implies & (p + x) - z = -q \\
\implies & p + (x - z) = -q & \text{(by Proposition \ref{4.2.4})} \\
\implies & (-p) + (p + (x - z)) = (-p) + (-q) & \text{(by Proposition \ref{4.2.4})} \\
\implies & ((-p) + p) + (x - z) = (-p) + (-q) & \text{(by Proposition \ref{4.2.4})} \\
\implies & 0 + (x - z) = (-p) + (-q) & \text{(by Proposition \ref{4.2.4})} \\
\implies & x - z = (-p) + (-q) & \text{(by Proposition \ref{4.2.4})} \\
\implies & x - z = (-1)p + (-1)q & \text{(by Additional Corollary \ref{ac 4.2.3})} \\
\implies & x - z = (-1)(p + q) & \text{(by Proposition \ref{4.2.4})} \\
\implies & x - z = -(p + q). & \text{(by Additional Corollary \ref{ac 4.2.3})}
\end{align*}
Because by Additional Corollary \ref{ac 4.2.4}, \(p + q\) is a positive rational number, so by Definition \ref{4.2.6}, \(-(p + q)\) is a negative rational number.
Since \(x - z = -(p + q)\), \(x - z\) is also a negative rational number, and by Definition \ref{4.2.8}, \(x < z\).
\end{proof}

\begin{proof}{(d)}
By the given condition and Definition \ref{4.2.8}, \(x < y \implies x - y\) is negative.
Let \(x - y = -a\), where \(a\) is a positive rational number.
So
\begin{align*}
& x - y = -a \\
\implies & (x + (-y)) = -a \\
\implies & (x + (-y)) + 0 = -a & \text{(by Proposition \ref{4.2.4})} \\
\implies & (x + (-y)) + (z + (-z)) = -a & \text{(by Proposition \ref{4.2.4})} \\
\implies & (x + (-y)) + ((-z) + z) = -a & \text{(by Proposition \ref{4.2.4})} \\
\implies & x + (((-y) + (-z)) + z) = -a & \text{(by Proposition \ref{4.2.4})} \\
\implies & x + (z + ((-y) + (-z))) = -a & \text{(by Proposition \ref{4.2.4})} \\
\implies & (x + z) + ((-y) + (-z)) = -a & \text{(by Proposition \ref{4.2.4})} \\
\implies & (x + z) + ((-1)y + (-1)z) = -a & \text{(by Additional Corollary \ref{ac 4.2.3})} \\
\implies & (x + z) + (-1)(y + z) = -a & \text{(by Proposition \ref{4.2.4})} \\
\implies & (x + z) + (-(y + z)) = -a & \text{(by Additional Corollary \ref{ac 4.2.3})} \\
\implies & (x + z) - (y + z) = -a \\
\implies & x + z < y + z. & \text{(by Definition \ref{4.2.8})} \\
\end{align*}
\end{proof}

\begin{proof}{(e)}
By the given condition and Definition \ref{4.2.8}, \(x < y \implies x - y\) is negative.
Let \(x - y = -a\), where \(a\) is a positive rational number.
So
\begin{align*}
& x - y = -a \\
\implies & (x + (-y)) = -a \\
\implies & (x + (-y))z = (-a)z & \text{(by Proposition \ref{4.2.4})} \\
\implies & xz + (-y)z = (-a)z & \text{(by Proposition \ref{4.2.4})} \\
\implies & xz + ((-1)y)z = ((-1)a)z & \text{(by Additional Corollary \ref{ac 4.2.3})} \\
\implies & xz + (-1)(yz) = (-1)(az) & \text{(by Proposition \ref{4.2.4})} \\
\implies & xz + (-(yz)) = -(az) & \text{(by Additional Corollary \ref{ac 4.2.3})} \\
\implies & xz - yz = -(az).
\end{align*}
Because by Additional Corollary \ref{ac 4.2.5}, \(az\) is a positive rational number, so by Definition \ref{4.2.6}, \(-(az)\) is a negative rational number.
Since \(xz - yz = -(az)\), \(xz - yz\) is also a negative rational number, and by Definition \ref{4.2.8}, \(xz < yz\).
\end{proof}

\begin{remark}\label{4.2.10}
The above five properties in Proposition \ref{4.2.9}, combined with the field axioms in Proposition \ref{4.2.4}, have a name:
they assert that the rationals \(\mathds{Q}\) form an \emph{ordered field}.
It is important to keep in mind that Proposition \ref{4.2.9}(e) only works when \(z\) is positive.
\end{remark}

\exercisesection

\begin{exercise}\label{ex 4.2.1}
Show that the definition of equality for the rational numbers is reflexive, symmetric, and transitive.
\end{exercise}

\begin{proof}
See Additional Corollary \ref{ac 4.2.1}.
\end{proof}

\begin{exercise}\label{ex 4.2.2}
Prove the remaining components of Lemma \ref{4.2.3}.
\end{exercise}

\begin{proof}
See Lemma \ref{4.2.3}.
\end{proof}

\begin{exercise}\label{ex 4.2.3}
Prove the remaining components of Proposition \ref{4.2.4}.
\end{exercise}

\begin{proof}
See Proposition \ref{4.2.4}.
\end{proof}

\begin{exercise}\label{ex 4.2.4}
Prove Lemma \ref{4.2.7}.
\end{exercise}

\begin{proof}
See Lemma \ref{4.2.7}.
\end{proof}

\begin{exercise}\label{ex 4.2.5}
Prove Proposition \ref{4.2.9}.
\end{exercise}

\begin{proof}
See Proposition \ref{4.2.9}.
\end{proof}

\begin{exercise}\label{ex 4.2.6}
Show that if \(x\), \(y\), \(z\) are rational numbers such that \(x < y\) and \(z\) is negative, then \(xz > yz\).
\end{exercise}

\begin{proof}
By the given condition and Definition \ref{4.2.8}, \(x < y \implies x - y\) is negative.
Let \(x - y = -a\) and \(z = -b\), where \(a, b\) is a positive rational number.
So
\begin{align*}
& x - y = -a \\
\implies & (x + (-y)) = -a \\
\implies & (x + (-y))z = (-a)z & \text{(by Proposition \ref{4.2.4})} \\
\implies & xz + (-y)z = (-a)z & \text{(by Proposition \ref{4.2.4})} \\
\implies & xz + ((-1)y)z = ((-1)a)z & \text{(by Additional Corollary \ref{ac 4.2.3})} \\
\implies & xz + (-1)(yz) = (-1)(az) & \text{(by Proposition \ref{4.2.4})} \\
\implies & xz + (-(yz)) = (-1)(az) & \text{(by Additional Corollary \ref{ac 4.2.3})} \\
\implies & xz - yz = (-1)(az) \\
\implies & xz - yz = (-1)(a(-b)) & \text{(by the given conditions)} \\
\implies & xz - yz = (-1)(a((-1)b)) & \text{(by Additional Corollary \ref{ac 4.2.3})} \\
\implies & xz - yz = (-1)((a(-1))b) & \text{(by Proposition \ref{4.2.4})} \\
\implies & xz - yz = (-1)(((-1)a)b) & \text{(by Proposition \ref{4.2.4})} \\
\implies & xz - yz = (-1)((-1)(ab)) & \text{(by Proposition \ref{4.2.4})} \\
\implies & xz - yz = ((-1)(-1))(ab) & \text{(by Proposition \ref{4.2.4})} \\
\implies & xz - yz = 1(ab) & \text{(by Definition \ref{4.2.1})} \\
\implies & xz - yz = ab & \text{(by Proposition \ref{4.2.4})} \\
\end{align*}
Because by Additional Corollary \ref{ac 4.2.5}, \(ab\) is a positive rational number.
Since \(xz - yz = az\), \(xz - yz\) is also a positive rational number, and by Definition \ref{4.2.8}, \(xz > yz\).
\end{proof}