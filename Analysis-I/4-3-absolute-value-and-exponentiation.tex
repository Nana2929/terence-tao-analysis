\section{Absolute value and exponentiation}

\begin{definition}[Absolute value]\label{4.3.1}
If \(x\) is a rational number, the \emph{absolute value} \(\abs*{x}\) of \(x\) is defined as follows.
If \(x\) is positive, then \(\abs*{x} \coloneqq x\).
If \(x\) is negative, then \(\abs*{x} \coloneqq -x\).
If \(x\) is zero, then \(\abs*{x} \coloneqq 0\).
\end{definition}

\begin{definition}[Distance]\label{4.3.2}
Let \(x\) and \(y\) be rational numbers.
The quantity \(\abs*{x - y}\) is called the \emph{distance between \(x\) and \(y\)} and is sometimes denoted \(d(x, y)\), thus \(d(x, y) \coloneqq \abs*{x - y}\).
\end{definition}

\begin{proposition}[Basic properties of absolute value and distance]\label{4.3.3}
Let \(x\), \(y\), \(z\) be rational numbers.
\begin{enumerate}
    \item (Non-degeneracy of absolute value)
    We have \(\abs*{x} \geq 0\).
    Also, \(\abs*{x} = 0\) if and only if \(x\) is \(0\).
    \item (Triangle inequality for absolute value)
    We have \(\abs*{x + y} \leq \abs*{x} + \abs*{y}\).
    \item We have the inequalities \(-y \leq x \leq y\) if and only if \(y \geq \abs*{x}\).
    In particular, we have \(-\abs*{x} \leq x \leq \abs*{x}\).
    \item (Multiplicativity of absolute value)
    We have \(\abs*{xy} = \abs*{x} \abs*{y}\).
    In particular, \(\abs*{-x} = \abs*{x}\).
    \item (Non-degeneracy of distance)
    We have \(d(x, y) \geq 0\).
    Also, \(d(x, y) = 0\) if and only if \(x = y\).
    \item (Symmetry of distance)
    \(d(x, y) = d(y, x)\).
    \item (Triangle inequality for distance)
    \(d(x, z) \leq d(x, y) + d(y, z)\).
\end{enumerate}
\end{proposition}

\begin{proof}{(a)}
By Lemma \ref{4.2.7}, exactly one of the three statements is true:
\begin{enumerate}[label=(\roman*)]
    \item \(x = 0\).
    Then by Definition \ref{4.3.1}, \(\abs*{x} = 0\).
    \item \(x\) is a positive rational number.
    Then by Definition \ref{4.3.1}, \(\abs*{x} = x\), which is a positive rational number.
    \item \(x\) is a negative rational number.
    Then by Definition \ref{4.3.1}, \(\abs*{x} = -x\).
    By Additional Corollary \ref{ac 4.2.5}, \(-x = (-1)x\) is a positive rational number.
\end{enumerate}
So \(\abs*{x}\) is either \(0\) or a positive rational number, which by Definition \ref{4.2.8}, \(\abs*{x} \geq 0\).

Now we proof that \(\abs*{x} = 0 \iff x = 0\).
By Definition \ref{4.3.1}, \(x = 0 \implies \abs*{x} = 0\), so we only need to show that \(\abs*{x} = 0 \implies x = 0\).
By Lemma \ref{4.2.7}, exactly one of the following is true:
\(x = 0\), \(x\) is positive rational number or \(x\) is negative rational number.
If \(x\) is positive rational number, then \(\abs*{x} = x \neq 0\).
If \(x\) is negative rational number, then \(\abs*{x} = -x \neq 0\).
So \(x\) can only be \(0\), which means \(\abs*{x} = 0 \implies x = 0\).
\end{proof}

\begin{proof}{(b)}
By Lemma \ref{4.2.7}, \(x\) and \(y\) can both have three different cases.
\begin{enumerate}[label=(\Roman*)]
    \item For \(x = 0\) and
    \begin{enumerate}[label=(\roman*)]
        \item \(y = 0\).
        By Proposition \ref{4.2.4} and Definition \ref{4.3.1}, \(\abs*{0 + 0} = \abs*{0} = 0\), and \(\abs*{0} + \abs*{0} = 0 + 0 = 0\).
        Thus \(\abs*{x + y} = 0 = \abs*{x} + \abs*{y}\).
        \item \(y\) is positive.
        By Proposition \ref{4.2.4} and Definition \ref{4.3.1}, \(\abs*{0 + y} = \abs*{y} = y\), and \(\abs*{0} + \abs*{y} = 0 + y = y\).
        Thus \(\abs*{x + y} = y = \abs*{x} + \abs*{y}\).
        \item \(y\) is negative.
        By Proposition \ref{4.2.4} and Definition \ref{4.3.1}, \(\abs*{0 + y} = \abs*{y} = -y\), and \(\abs*{0} + \abs*{y} = 0 + (-y) = -y\).
        Thus \(\abs*{x + y} = -y = \abs*{x} + \abs*{y}\).
    \end{enumerate}
    \item For \(x\) is positive and
    \begin{enumerate}[label=(\roman*)]
        \item \(y = 0\).
        Which is just equivalent to the case \(x = 0\) and \(y\) is positive.
        \item \(y\) is positive.
        By Additional Corollary \ref{ac 4.2.4} and Definition \ref{4.3.1}, \(\abs*{x + y} = x + y\), and \(\abs*{x} + \abs*{y} = x + y\).
        Thus \(\abs*{x + y} = x + y = \abs*{x} + \abs*{y}\).
        \item \(y\) is negative.
        Let \(y = -a\), where \(a\) is a positive rational number.
        By Proposition \ref{4.2.9}, exactly one of the following three statements is true:
        \begin{enumerate}[label=(\arabic*)]
            \item \(x = a\).
            By Proposition \ref{4.2.4} and Definition \ref{4.3.1}, \((\abs*{x} + \abs*{-a}) - \abs*{x - a} = (\abs*{x} + \abs*{-a}) - \abs*{0} = (x + a) - 0 = x + a\).
            By Additional Corollary \ref{ac 4.2.4}, \(x + a\) is a positive rational number.
            Thus by Definition \ref{4.2.8}, \(\abs*{x + y} = 0 < x + a = \abs*{x} + \abs*{y}\).
            \item \(x > a\).
            By Definition \ref{4.2.8}, \(x - a\) is a positive rational number, so by Definition \ref{4.3.1}, \(\abs*{x - a} = x - a\).
            By Definition \ref{4.3.1}, \(\abs*{x} + \abs*{-a} = x + a\).
            By Proposition \ref{4.2.4} and Additional Corollary \ref{ac 4.2.3}, \((\abs*{x} + \abs*{y}) - \abs*{x + y} = (x + a) - (x - a) = 2a\).
            By Additional Corollary \ref{ac 4.2.5}, \(2a\) is a positive rational number.
            Thus by Definition \ref{4.2.8}, \(\abs*{x + y} = x - a < x + a = \abs*{x} + \abs*{y}\).
            \item \(x < a\).
            By Definition \ref{4.2.8}, \(x - a\) is a negative rational number, so by Definition \ref{4.3.1}, \(\abs*{x - a} = -(x - a) = a - x\).
            By Definition \ref{4.3.1}, \(\abs*{x} + \abs*{-a} = x + a\).
            By Proposition \ref{4.2.4} and Additional Corollary \ref{ac 4.2.3}, \((\abs*{x} + \abs*{y}) - \abs*{x + y} = (x + a) - (a - x) = 2x\).
            By Additional Corollary \ref{ac 4.2.5}, \(2x\) is a positive rational number.
            Thus by Definition \ref{4.2.8}, \(\abs*{x + y} = a - x < x + a = \abs*{x} + \abs*{y}\).
        \end{enumerate}
    \end{enumerate}
    \item For \(x\) is negative and
    \begin{enumerate}[label=(\roman*)]
        \item \(y = 0\).
        Which is just equivalent to the case \(x = 0\) and \(y\) is negative.
        \item \(y\) is positive.
        Which is just equivalent to the case \(x\) is positive and \(y\) is negative.
        \item \(y\) is negative.
        By Additional Corollary \ref{ac 4.2.4} and Definition \ref{4.3.1}, \(\abs*{x + y} = -(x + y)\).
        By Definition \ref{4.3.1}, Additional Corollary \ref{ac 4.2.3} and Proposition \ref{4.2.4}, \(\abs*{x} + \abs*{y} = (-x) + (-y) = (-1)x + (-1)y = (-1)(x + y) = -(x + y)\).
        Thus \(\abs*{x + y} = -(x + y) = \abs*{x} + \abs*{y}\).
    \end{enumerate}
\end{enumerate}
For all cases above, we get either \(\abs*{x + y} = \abs*{x} + \abs*{y}\) or \(\abs*{x + y} < \abs*{x} + \abs*{y}\).
So by Definition \ref{4.2.8}, \(\abs*{x + y} \leq \abs*{x} + \abs*{y}\).
\end{proof}

\begin{proof}{(c)}
We first prove that \(-y \leq x \leq y\) implies \(y \geq \abs*{x}\).
Because \(-y \leq y\), so \(y \geq 0\) by Proposition \ref{4.2.9} and \(-y \leq 0\) by Exercise \ref{ex 4.2.6}.
Again by Proposition \ref{4.2.9}, exactly one of the three statements is true:
\begin{enumerate}[label=(\Roman*)]
    \item \(x < 0\).
    Then we have
    \begin{align*}
        & -y \leq x < 0 \leq y \\
        \implies & (-y \leq x) \land (x < 0) \\
        \implies & (-x \leq y) \land (x < 0) & \text{(by Exercise \ref{ex 4.2.6})} \\
        \implies & (-x \leq y) \land (x \text{ is negative}) & \text{(by Addtitional Corollary \ref{ac 4.2.7})} \\
        \implies & (-x \leq y) \land (\abs*{x} = -x) & \text{(by Definition \ref{4.3.1})} \\
        \implies & \abs*{x} \leq y & \text{(by Addtitional Corollary \ref{ac 4.2.1})} \\
        \implies & y \geq \abs*{x}. & \text{(by Proposition \ref{4.2.9})}
    \end{align*}
    \item \(x = 0\).
    Then we have
    \begin{align*}
        & -y \leq x = 0 \leq y \\
        \implies & (-y \leq 0) \land (x = 0) \\
        \implies & (0 \leq y) \land (x = 0) & \text{(by Exercise \ref{ex 4.2.6})} \\
        \implies & (0 \leq y) \land (\abs*{x} = 0) & \text{(by Definition \ref{4.3.1})} \\
        \implies & \abs*{x} \leq y & \text{(by Addtitional Corollary \ref{ac 4.2.1})} \\
        \implies & y \geq \abs*{x}. & \text{(by Proposition \ref{4.2.9})}
    \end{align*}
    \item \(x > 0\).
    Then we have
    \begin{align*}
        & -y \leq 0 < x \leq y \\
        \implies & (0 < x) \land (x \leq y) \\
        \implies & (x \text{ is positive}) \land (x \leq y) & \text{(by Addtitional Corollary \ref{ac 4.2.7})} \\
        \implies & (\abs*{x} = x) \land (x \leq y) & \text{(by Definition \ref{4.3.1})} \\
        \implies & \abs*{x} \leq y & \text{(by Addtitional Corollary \ref{ac 4.2.1})} \\
        \implies & y \geq \abs*{x}. & \text{(by Proposition \ref{4.2.9})}
    \end{align*}
\end{enumerate}
For all cases above, we have \(y \geq \abs*{x}\).
Thus we conclude that \(-y \leq x \leq y \implies y \geq \abs*{x}\).

Now we prove that \(y \geq \abs*{x} \implies -y \leq x \leq y\).
By Proposition \ref{4.3.3} (a), \(\abs*{x} \geq 0\).
So we have \(y \geq \abs*{x} \geq 0\).
By Exercise \ref{ex 4.2.6}, \(y \geq 0 \implies 0 \geq -y\).
Then we have \(y \geq \abs*{x} \geq -y\) by Proposition \ref{4.2.9}.
By Lemma \ref{4.2.7}, exactly one of the three statements is true:
\begin{enumerate}
    \item \(x\) is positive.
    Then by Definition \ref{4.3.1}, \(\abs*{x} = x\).
    Thus we have \(y \geq x \geq -y\), or \(-y \leq x \leq y\) by Proposition \ref{4.2.9}.
    \item \(x = 0\).
    Then by Definition \ref{4.3.1}, \(\abs*{x} = 0\).
    Thus we have \(y \geq 0 \geq -y\), or \(-y \leq 0 \leq y\) by Proposition \ref{4.2.9}.
    \item \(x\) is negative.
    Then by Definition \ref{4.3.1}, \(\abs*{x} = -x\).
    Thus we have \(y \geq -x \geq -y\), or \(-y \leq x \leq y\) by Exercise \ref{ex 4.2.6}.
\end{enumerate}
For all cases above, we have \(-y \leq x \leq y\).
Thus we conclude that \(y \geq \abs*{x} \implies -y \leq x \leq y\).
Combine with proof above, we have \(-y \leq x \leq y \iff y \geq \abs*{x}\).
In particular, by replacing \(y\) with \(\abs*{x}\), we have \(-\abs*{x} \leq x \leq \abs*{x} \iff \abs*{x} \geq \abs*{x}\).
\end{proof}

\begin{proof}{(d)}
By Lemma \ref{4.2.7}, exactly one of the following three statements is true:
\begin{enumerate}[label=(\Roman*)]
    \item \(x = 0\).
    By Proposition \ref{4.2.4} and Definition \ref{4.3.1}, \(\abs*{xy} = \abs*{0y} = \abs*{0} = 0\) and \(\abs*{x}\abs*{y} = \abs*{0}\abs*{y} = 0\abs*{y} = 0\).
    So \(\abs*{xy} = 0 = \abs*{x}\abs*{y}\).
    \item \(x\) is a positive rational number.
    By Lemma \ref{4.2.7}, exactly one of the following three statements is true:
    \begin{enumerate}[label=(\roman*)]
        \item \(y = 0\).
        Which is just the same case as \(x = 0\).
        \item \(y\) is a positive rational number.
        By Additional Corollary \ref{ac 4.2.5}, \(xy\) is a positive rational number.
        Then by Definition \ref{4.3.1}, \(\abs*{xy} = xy\) and \(\abs*{x}\abs*{y} = xy\).
        So \(\abs*{xy} = xy = \abs*{x}\abs*{y}\).
        \item \(y\) is a negative rational number.
        By Additional Corollary \ref{ac 4.2.6}, \(xy\) is a negative rational number.
        Then by Definition \ref{4.3.1}, \(\abs*{xy} = -(xy)\).
        By Definition \ref{4.3.1}, Additional Corollary \ref{ac 4.2.3} and Proposition \ref{4.2.4}, \(\abs*{x}\abs*{y} = x(-y) = x((-1)y) = (x(-1))y = ((-1)x)y = (-1)(xy) = -(xy)\).
        So \(\abs*{xy} = -(xy) = \abs*{x}\abs*{y}\).
    \end{enumerate}
    \item \(x\) is a negative rational number.
    By Lemma \ref{4.2.7}, exactly one of the following three statements is true:
    \begin{enumerate}[label=(\roman*)]
        \item \(y = 0\).
        Which is just the same case as \(x = 0\).
        \item \(y\) is a positive rational number.
        Which is just the same case as \(x\) is a positive rational number and \(y\) is a negative rational number.
        \item \(y\) is a negative rational number.
        By Additional Corollary \ref{ac 4.2.5}, \(xy\) is a positive rational number.
        Then by Definition \ref{4.3.1}, \(\abs*{xy} = xy\).
        By Definition \ref{4.3.1}, Additional Corollary \ref{ac 4.2.3} and Proposition \ref{4.2.4}, \(\abs*{x}\abs*{y} = (-x)(-y) = ((-1)x)((-1)y) = (x(-1))((-1)y) = (x((-1)(-1)))y = (x1)y = xy\).
        So \(\abs*{xy} = xy = \abs*{x}\abs*{y}\).
    \end{enumerate}
\end{enumerate}
From all cases above, we can see that \(\abs*{xy} = \abs*{x}\abs*{y}\).

Now we show that \(\abs*{-x} = \abs*{x}\).
Using previous proof, let \(y = -1\), we get \(\abs*{x(-1)} = \abs*{x}\abs*{-1}\).
By Proposition \ref{4.2.4}, Definition \ref{4.3.1} and Additional Corollary \ref{4.2.3}, \(\abs*{x(-1)} = \abs*{(-1)x} = \abs*{-x}\).
Again by Proposition \ref{4.2.4}, Definition \ref{4.3.1} and Additional Corollary \ref{4.2.3}, \(\abs*{x}\abs*{-1} = \abs*{x}(-(-1)) = \abs*{x}((-1)(-1)) = \abs*{x}1 = \abs*{x}\).
Thus \(\abs*{-x} = \abs*{x}\).
\end{proof}

\begin{proof}{(e)}
We first show that \(d(x, y) \geq 0\).
By Definition \ref{4.3.2}, \(d(x, y) = \abs*{x - y}\).
By Proposition \ref{4.3.3}(a), \(\abs*{x - y} \geq 0\).
So \(d(x, y) = \abs*{x - y} \geq 0\).

Now we show that \(d(x, y) = 0 \iff x = y\).
\begin{align*}
& d(x, y) = 0 & \text{(by the given condition)} \\
\iff & \abs*{x - y} = 0 & \text{(by Definition \ref{4.3.2})} \\
\iff & x - y = 0 & \text{(by Proposition \ref{4.3.3}(a))} \\
\iff & (x - y) + y = 0 + y & \text{(by Proposition \ref{4.2.4})} \\
\iff & (x - y) + y = y & \text{(by Proposition \ref{4.2.4})} \\
\iff & (x + (-y)) + y = y \\
\iff & x + ((-y) + y) = y & \text{(by Proposition \ref{4.2.4})} \\
\iff & x + 0 = y & \text{(by Proposition \ref{4.2.4})} \\
\iff & x = y. & \text{(by Proposition \ref{4.2.4})}
\end{align*}
\end{proof}

\begin{proof}{(f)}
\begin{align*}
d(x, y) &= \abs*{x - y} & \text{(by Definition \ref{4.3.2})} \\
&= \abs*{x + (-y)} \\
&= \abs*{-(x + (-y))} & \text{(by Proposition \ref{4.3.3}(d))} \\
&= \abs*{(-1)(x + (-1)y)} & \text{(by Additional Corollary \ref{ac 4.2.3})} \\
&= \abs*{(-1)x + (-1)((-1)y)} & \text{(by Proposition \ref{4.2.4})} \\
&= \abs*{(-1)x + ((-1)(-1))y} & \text{(by Proposition \ref{4.2.4})} \\
&= \abs*{(-1)x + 1y} & \text{(by Proposition \ref{4.2.4})} \\
&= \abs*{(-1)x + y} & \text{(by Proposition \ref{4.2.4})} \\
&= \abs*{y + (-1)x} & \text{(by Proposition \ref{4.2.4})} \\
&= \abs*{y + -x} & \text{(by Additional Corollary \ref{ac 4.2.3})} \\
&= \abs*{y - x} \\
&= d(y, x). & \text{(by Definition \ref{4.3.2})}
\end{align*}
\end{proof}

\begin{proof}{(g)}
\begin{align*}
d(x, z) &= \abs*{x - z} & \text{(by Definition \ref{4.3.2})} \\
&= \abs*{x + (-z)} \\
&= \abs*{(x + (-z)) + 0} & \text{(by Proposition \ref{4.2.4})} \\
&= \abs*{(x + (-z)) + ((-y) + y)} & \text{(by Proposition \ref{4.2.4})} \\
&= \abs*{(x + (((-z) + (-y)) + y)} & \text{(by Proposition \ref{4.2.4})} \\
&= \abs*{(x + (((-y) + (-z)) + y)} & \text{(by Proposition \ref{4.2.4})} \\
&= \abs*{(x + (-y)) + ((-z) + y)} & \text{(by Proposition \ref{4.2.4})} \\
&= \abs*{(x + (-y)) + (y + (-z))} & \text{(by Proposition \ref{4.2.4})} \\
&= \abs*{(x - y) + (y - z)} \\
&\leq \abs*{x - y} + \abs*{y - z} & \text{(by Proposition \ref{4.3.3}(b))} \\
&= d(x, y) + d(y, z). & \text{(by Definition \ref{4.3.2})} \\
\end{align*}
\end{proof}

\begin{additional corollary}\label{ac 4.3.1}
Let \(x, y\) be rational numbers.
Then \(\abs*{x} - \abs*{y} \leq \abs*{x + y}\).
\end{additional corollary}

\begin{proof}
\begin{align*}
& \abs*{(x + y) + (-y)} \leq \abs*{x + y} + \abs*{-y} & \text{(by Proposition \ref{4.3.3})} \\
\implies & \abs*{x + (y + (-y))} \leq \abs*{x + y} + \abs*{-y}& \text{(by Proposition \ref{4.2.4})} \\
\implies & \abs*{x + 0} \leq \abs*{x + y} + \abs*{-y}& \text{(by Proposition \ref{4.2.4})} \\
\implies & \abs*{x} \leq \abs*{x + y} + \abs*{-y}& \text{(by Proposition \ref{4.2.4})} \\
\implies & \abs*{x} \leq \abs*{x + y} + \abs*{y}& \text{(by Proposition \ref{4.3.3})} \\
\implies & \abs*{x} + (-\abs*{y}) \leq (\abs*{x + y} + \abs*{y}) + (-\abs*{y}) & \text{(by Proposition \ref{4.2.9})} \\
\implies & \abs*{x} + (-\abs*{y}) \leq \abs*{x + y} + (\abs*{y} + (-\abs*{y})) & \text{(by Proposition \ref{4.2.4})} \\
\implies & \abs*{x} + (-\abs*{y}) \leq \abs*{x + y} + 0 & \text{(by Proposition \ref{4.2.4})} \\
\implies & \abs*{x} + (-\abs*{y}) \leq \abs*{x + y} & \text{(by Proposition \ref{4.2.4})} \\
\implies & \abs*{x} - \abs*{y} \leq \abs*{x + y}.
\end{align*}
\end{proof}

\begin{definition}[\(\varepsilon\)-closeness]\label{4.3.4}
Let \(\varepsilon > 0\) be a rational number, and let \(x\), \(y\) be rational numbers.
We say that \(y\) is \emph{\(\varepsilon\)-close} to \(x\) iff we have \(d(y, x) \leq \varepsilon\).
\end{definition}

\begin{remark}\label{4.3.5}
This definition is not standard in mathematics textbooks;
we will use it as ``scaffolding'' to construct the more important notions of limits (and of Cauchy sequences) later on, and once we have those more advanced notions we will discard the notion of \(\varepsilon\)-close.
\end{remark}

\begin{note}
We do not bother defining a notion of \(\varepsilon\)-close when \(\varepsilon\) is zero or negative, because if \(\varepsilon\) is zero then \(x\) and \(y\) are only \(\varepsilon\)-close when they are equal, and when \(\varepsilon\) is negative then \(x\) and \(y\) are never \(\varepsilon\)-close.
\end{note}

\begin{note}
In any event it is a long-standing tradition in analysis that the Greek letters \(\varepsilon\), \(\delta\) should only denote small positive numbers.
\end{note}

\setcounter{theorem}{6}
\begin{proposition}\label{4.3.7}
Let \(x, y, z, w\) be rational numbers.
(extended to cover the \(0\)-close case)
\begin{enumerate}
    \item If \(x = y\), then \(x\) is \(\varepsilon\)-close to \(y\) for every \(\varepsilon > 0\).
    Conversely, if \(x\) is \(\varepsilon\)-close to \(y\) for every \(\varepsilon > 0\), then we have \(x = y\).
    \item Let \(\varepsilon > 0\).
    If \(x\) is \(\varepsilon\)-close to \(y\), then \(y\) is \(\varepsilon\)-close to \(x\).
    \item Let \(\varepsilon, \delta > 0\).
    If \(x\) is \(\varepsilon\)-close to \(y\), and \(y\) is \(\delta\)-close to \(z\), then \(x\) and \(z\) are \((\varepsilon + \delta)\)-close.
    \item Let \(\varepsilon, \delta > 0\).
    If \(x\) and \(y\) are \(\varepsilon\)-close, and \(z\) and \(w\) are \(\delta\)-close, then \(x + z\) and \(y + w\) are \((\varepsilon + \delta)\)-close, and \(x - z\) and \(y - w\) are also \((\varepsilon + \delta)\)-close.
    \item Let \(\varepsilon > 0\).
    If \(x\) and \(y\) are \(\varepsilon\)-close, they are also \(\varepsilon'\)-close for every \(\varepsilon' > \varepsilon\).
    \item Let \(\varepsilon > 0\).
    If \(y\) and \(z\) are both \(\varepsilon\)-close to \(x\), and \(w\) is between \(y\) and \(z\) (i.e., \(y \leq w \leq z\) or \(z \leq w \leq y\)), then \(w\) is also \(\varepsilon\)-close to \(x\).
    \item Let \(\varepsilon > 0\).
    If \(x\) and \(y\) are \(\varepsilon\)-close, and \(z\) is non-zero, then \(xz\) and \(yz\) are \(\varepsilon\abs*{z}\)-close.
    \item Let \(\varepsilon, \delta > 0\).
    If \(x\) and \(y\) are \(\varepsilon\)-close, and \(z\) and \(w\) are \(\delta\)-close, then \(xz\) and \(yw\) are \((\varepsilon\abs*{z} + \delta\abs*{x} + \varepsilon\delta)\)-close.
\end{enumerate}
\end{proposition}

\begin{proof}{(a)}
We first prove that \(x = y\) implies \(x\) is \(\varepsilon\)-close to \(y\), \(\forall\ \varepsilon > 0\).
By Proposition \ref{4.3.3}, \(x = y \iff d(x, y) = 0\).
Then by the given condition, \(\forall\ \varepsilon > 0 = d(x, y)\), which means \(d(x, y) < \varepsilon\) by Proposition \ref{4.2.9}.
By Definition \ref{4.2.8}, \(d(x, y) < \varepsilon \implies d(x, y) \leq \varepsilon\).
Thus by Definition \ref{4.3.4}, \(x\) is \(\varepsilon\)-close to \(y\).

Now we prove that \(\forall\ \varepsilon > 0\), \(x\) is \(\varepsilon\)-close to \(y\) implies \(x = y\).
By the given condition, \(d(x, y) \leq \varepsilon\).
Suppose for sake of contradiction that \(x \neq y\), then \(d(x, y) > 0\) by Proposition \ref{4.3.3}.
Since \(d(x, y) > 0\), \(d(x, y) / 2 > 0\).
But by the given condition, \(\forall\ \varepsilon > 0\), \(d(x, y) \leq \varepsilon\).
Then we get \(d(x, y) \leq d(x, y) / 2\), and because \(d(x, y) \neq 0\), \(d(x, y) < d(x, y) / 2\), a contradiction.
So \(x = y\).
\end{proof}

\begin{proof}{(b)}
\begin{align*}
& x \text{ is \(\varepsilon\)-close to } y \\
\implies & d(x, y) \leq \varepsilon & \text{(by Definition \ref{4.3.4})} \\
\implies & d(y, x) \leq \varepsilon & \text{(by Proposition \ref{4.3.3}, \(d(x, y) = d(y, x)\))} \\
\implies & y \text{ is \(\varepsilon\)-close to } x. & \text{(by Definition \ref{4.3.4})}
\end{align*}
\end{proof}

\begin{proof}{(c)}
By Definition \ref{4.3.4}, \(x\) is \(\varepsilon\)-close to \(y\) implies \(d(x, y) \leq \varepsilon\), and \(y\) is \(\delta\)-close to \(z\) implies \(d(y, z) \leq \delta\).
So
\begin{align*}
& d(y, z) \leq \delta \\
\implies & \varepsilon + d(y, z) \leq \varepsilon + \delta. & \text{(by Proposition \ref{4.3.3})} \\
& d(x, y) \leq \varepsilon \\
\implies & d(x, y) + d(y, z) \leq \varepsilon + d(y, z) & \text{(by Proposition \ref{4.3.3})} \\
\implies & d(x, y) + d(y, z) \leq \varepsilon + \delta & \text{(by Proposition \ref{4.2.9})} \\
\implies & d(x, z) \leq d(x, y) + d(y, z) \leq \varepsilon + \delta & \text{(by Proposition \ref{4.3.3})} \\
\implies & x \text{ is \((\varepsilon + \delta)\)-close to } z. & \text{(by Definition \ref{4.3.4})}
\end{align*}
\end{proof}

\begin{proof}{(d)}
We first prove that If \(x\) and \(y\) are \(\varepsilon\)-close, and \(z\) and \(w\) are \(\delta\)-close, then \(x + z\) and \(y + w\) are \((\varepsilon + \delta)\)-close.
By Definition \ref{4.3.4}, \(x\) is \(\varepsilon\)-close to \(y\) implies \(d(x, y) \leq \varepsilon\), and \(z\) is \(\delta\)-close to \(w\) implies \(d(z, w) \leq \delta\).
Because
\begin{align*}
& d(z, w) \leq \delta \\
\implies & \varepsilon + d(z, w) \leq \varepsilon + \delta. & \text{(by Proposition \ref{4.2.9})} \\
& d(x, y) \leq \varepsilon \\
\implies & d(x, y) + d(z, w) \leq \varepsilon + d(z, w) & \text{(by Proposition \ref{4.2.9})} \\
\implies & d(x, y) + d(z, w) \leq \varepsilon + \delta. & \text{(by Proposition \ref{4.2.9})}
\end{align*}
So
\begin{align*}
& d(x + z, y + w) \\
&= \abs*{(x + z) - (y + w)} & \text{(by Definition \ref{4.3.2})} \\
&= \abs*{(x + z) + (-(y + w))} \\
&= \abs*{(x + z) + (-1)(y + w)} & \text{(by Additional Corollary \ref{ac 4.2.3})} \\
&= \abs*{(x + z) + ((-1)y + (-1)w)} & \text{(by Proposition \ref{4.2.4})} \\
&= \abs*{(x + (z + (-1)y)) + (-1)w} & \text{(by Proposition \ref{4.2.4})} \\
&= \abs*{(x + ((-1)y + z)) + (-1)w} & \text{(by Proposition \ref{4.2.4})} \\
&= \abs*{(x + (-1)y) + (z + (-1)w)} & \text{(by Proposition \ref{4.2.4})} \\
&= \abs*{(x + (-y)) + (z + (-w))} & \text{(by Additional Corollary \ref{ac 4.2.3})} \\
&= \abs*{(x - y) + (z - w)} \\
&\leq \abs*{x - y} + \abs*{z - w} & \text{(by Proposition \ref{4.3.3})} \\
&= d(x, y) + d(z, w) & \text{(by Definition \ref{4.3.2})} \\
&\leq \varepsilon + \delta. & \text{(by the given conditions)}
\end{align*}
Thus by Definition \ref{4.3.4}, \(x + z\) and \(y + w\) are \((\varepsilon + \delta)\)-close.

Now we prove that If \(x\) and \(y\) are \(\varepsilon\)-close, and \(z\) and \(w\) are \(\delta\)-close, then \(x - z\) and \(y - w\) are \((\varepsilon + \delta)\)-close.
By Definition \ref{4.3.4}, \(x\) is \(\varepsilon\)-close to \(y\) implies \(d(x, y) \leq \varepsilon\), and \(z\) is \(\delta\)-close to \(w\) implies \(d(z, w) \leq \delta\).
Again because
\begin{align*}
& d(z, w) \leq \delta \\
\implies & \varepsilon + d(z, w) \leq \varepsilon + \delta. & \text{(by Proposition \ref{4.2.9})} \\
& d(x, y) \leq \varepsilon \\
\implies & d(x, y) + d(z, w) \leq \varepsilon + d(z, w) & \text{(by Proposition \ref{4.2.9})} \\
\implies & d(x, y) + d(z, w) \leq \varepsilon + \delta. & \text{(by Proposition \ref{4.2.9})}
\end{align*}
So
\begin{align*}
& d(x - z, y - w) \\
&= \abs*{(x - z) - (y - w)} & \text{(by Definition \ref{4.3.2})} \\
&= \abs*{(x + (-z)) + (-(y + (-w)))} \\
&= \abs*{(x + (-1)z) + (-1)(y + (-1)w)} & \text{(by Additional Corollary \ref{ac 4.2.3})} \\
&= \abs*{(x + (-1)z) + ((-1)y + (-1)(-1)w)} & \text{(by Proposition \ref{4.2.4})} \\
&= \abs*{(x + (-1)z) + ((-1)y + 1w)} & \text{(by Proposition \ref{4.2.4})} \\
&= \abs*{(x + (-1)z) + ((-1)y + w)} & \text{(by Proposition \ref{4.2.4})} \\
&= \abs*{(x + (-z)) + ((-y) + w)} & \text{(by Additional Corollary \ref{ac 4.2.3})} \\
&= \abs*{(x + ((-z) + (-y))) + w} & \text{(by Proposition \ref{4.2.4})} \\
&= \abs*{(x + ((-y) + (-z))) + w} & \text{(by Proposition \ref{4.2.4})} \\
&= \abs*{(x + (-y)) + ((-z) + w)} & \text{(by Proposition \ref{4.2.4})} \\
&= \abs*{(x + (-y)) + (w + (-z))} & \text{(by Proposition \ref{4.2.4})} \\
&= \abs*{(x - y) + (w - z)} \\
&\leq \abs*{x - y} + \abs*{w - z} & \text{(by Proposition \ref{4.3.3})} \\
&= \abs*{x - y} + \abs*{-(w - z)} & \text{(by Proposition \ref{4.3.3})} \\
&= \abs*{x - y} + \abs*{-(w + (-z))} \\
&= \abs*{x - y} + \abs*{(-1)(w + (-1)z)} & \text{(by Additional Corollary \ref{ac 4.2.3})} \\
&= \abs*{x - y} + \abs*{(-1)w + (-1)((-1)z)} & \text{(by Proposition \ref{4.2.4})} \\
&= \abs*{x - y} + \abs*{(-1)w + ((-1)(-1))z} & \text{(by Proposition \ref{4.2.4})} \\
&= \abs*{x - y} + \abs*{(-1)w + 1z} & \text{(by Proposition \ref{4.2.4})} \\
&= \abs*{x - y} + \abs*{(-1)w + z} & \text{(by Proposition \ref{4.2.4})} \\
&= \abs*{x - y} + \abs*{z + (-1)w} & \text{(by Proposition \ref{4.2.4})} \\
&= \abs*{x - y} + \abs*{z + (-w)} & \text{(by Additional Corollary \ref{ac 4.2.3})} \\
&= \abs*{x - y} + \abs*{z - w} \\
&= d(x, y) + d(z, w) & \text{(by Definition \ref{4.3.2})} \\
&\leq \varepsilon + \delta. & \text{(by the given conditions)}
\end{align*}
Thus by Definition \ref{4.3.4}, \(x - z\) and \(y - w\) are \((\varepsilon + \delta)\)-close.
\end{proof}

\begin{proof}{(e)}
By the given condition, \(\forall\ \varepsilon'\), \(\varepsilon < \varepsilon'\).
And by Definition \ref{4.3.4}, \(x\) is \(\varepsilon\)-close to \(y\) implies \(d(x, y) \leq \varepsilon\).
If \(d(x, y) = \varepsilon\), then \(d(x, y) < \varepsilon'\).
If \(d(x, y) < \varepsilon\), then by Proposition \ref{4.2.9}, \(d(x, y) < \varepsilon'\).
Thus \(d(x, y) < \varepsilon'\), by Definition \ref{4.2.8}, \(d(x, y) \leq \varepsilon'\), which means \(x\) is \(\varepsilon'\)-close to \(y\) by Definition \ref{4.3.4}.
\end{proof}

\begin{proof}{(f)}
\begin{align*}
& y \text{ is } \varepsilon\text{-close to } x \\
\implies & d(y, x) \leq \varepsilon & \text{(by Definition \ref{4.3.4})} \\
\implies & \abs*{y - x} \leq \varepsilon & \text{(by Definition \ref{4.3.2})} \\
\implies & (-1)\abs*{y - x} \geq (-1)\varepsilon & \text{(by Exercise \ref{ex 4.2.6})} \\
\implies & -\abs*{y - x} \geq -\varepsilon. \\
& z \text{ is } \varepsilon\text{-close to } x \\
\implies & d(z, x) \leq \varepsilon & \text{(by Definition \ref{4.3.4})} \\
\implies & \abs*{z - x} \leq \varepsilon & \text{(by Definition \ref{4.3.2})} \\
\implies & (-1)\abs*{z - x} \geq (-1)\varepsilon & \text{(by Exercise \ref{ex 4.2.6})} \\
\implies & -\abs*{z - x} \geq -\varepsilon. \\
& y \leq w \leq z \\
\implies & y + (-x) \leq w + (-x) \leq z + (-x) & \text{(by Lemma \ref{4.2.9})} \\
\implies & y - x \leq w - x \leq z - x \\
\implies & -\abs*{y - x} \leq y - x \leq w - x \leq z - x \leq \abs*{z - x} & \text{(by Proposition \ref{4.3.3})} \\
\implies & -\abs*{y - x} \leq w - x \leq \abs*{z - x} \\
\implies & -\varepsilon \leq -\abs*{y - x} \leq w - x \leq \abs*{z - x} \leq \varepsilon & \text{(by the given conditions)} \\
\implies & -\varepsilon \leq w - x \leq \varepsilon \\
\implies & \varepsilon \geq \abs*{w - x} & \text{(by Proposition \ref{4.3.3})} \\
\implies & \abs*{w - x} \leq \varepsilon & \text{(by Proposition \ref{4.2.9})} \\
\implies & d(w, x) \leq \varepsilon & \text{(by Definition \ref{4.3.2})} \\
\implies & w \text{ is } \varepsilon\text{-close to } x. \\
& z \leq w \leq y \\
\implies & z + (-x) \leq w + (-x) \leq y + (-x) & \text{(by Lemma \ref{4.2.9})} \\
\implies & z - x \leq w - x \leq y - x \\
\implies & -\abs*{z - x} \leq z - x \leq w - x \leq y - x \leq \abs*{y - x} & \text{(by Proposition \ref{4.3.3})} \\
\implies & -\abs*{z - x} \leq w - x \leq \abs*{y - x} \\
\implies & -\varepsilon \leq -\abs*{z - x} \leq w - x \leq \abs*{y - x} \leq \varepsilon & \text{(by the given conditions)} \\
\implies & -\varepsilon \leq w - x \leq \varepsilon \\
\implies & \varepsilon \geq \abs*{w - x} & \text{(by Proposition \ref{4.3.3})} \\
\implies & \abs*{w - x} \leq \varepsilon & \text{(by Proposition \ref{4.2.9})} \\
\implies & d(w, x) \leq \varepsilon & \text{(by Definition \ref{4.3.2})} \\
\implies & w \text{ is } \varepsilon\text{-close to } x.
\end{align*}
\end{proof}

\begin{proof}{(g)}
By Proposition \ref{4.3.3}, \(\abs*{z} \geq 0\).
So
\begin{align*}
& x \text{ is } \varepsilon\text{-close to } y \\
\implies & d(x, y) \leq \varepsilon & \text{(by Definition \ref{4.3.4})} \\
\implies & \abs*{x - y} \leq \varepsilon & \text{(by Definition \ref{4.3.2})} \\
\implies & \abs*{x - y}\abs*{z} \leq \varepsilon\abs*{z} & \text{(by Proposition \ref{4.2.9})} \\
\implies & \abs*{(x - y)z} \leq \varepsilon\abs*{z} & \text{(by Proposition \ref{4.3.3})} \\
\implies & \abs*{(x + (-y))z} \leq \varepsilon\abs*{z} \\
\implies & \abs*{xz + (-y)z} \leq \varepsilon\abs*{z} & \text{(by Proposition \ref{4.2.4})} \\
\implies & \abs*{xz + ((-1)y)z} \leq \varepsilon\abs*{z} & \text{(by Additional Corollary \ref{ac 4.2.3})} \\
\implies & \abs*{xz + (-1)(yz)} \leq \varepsilon\abs*{z} & \text{(by Proposition \ref{4.2.4})} \\
\implies & \abs*{xz + (-(yz))} \leq \varepsilon\abs*{z} & \text{(by Additional Corollary \ref{ac 4.2.3})} \\
\implies & \abs*{xz - yz} \leq \varepsilon\abs*{z} \\
\implies & d(xz, yz) \leq \varepsilon\abs*{z} & \text{(by Definition \ref{4.3.2})} \\
\implies & xz \text{ is } \varepsilon\abs*{z}\text{-close to } yz. & \text{(by Definition \ref{4.3.4})} \\
\end{align*}
\end{proof}

\begin{proof}{(h)}
Let \(\varepsilon, \delta > 0\), and suppose that \(x\) and \(y\) are \(\varepsilon\)-close.
If we write \(a \coloneqq y - x\), then we have \(y = x + a\) and that \(\abs*{a} \leq \varepsilon\).
Similarly, if \(z\) and \(w\) are \(\delta\)-close, and we define \(b \coloneqq w - z\), then \(w = z + b\) and \(\abs*{b} \leq \delta\).

Since \(y = x + a\) and \(w = z + b\), we have
\[
    yw = (x + a)(z + b) = xz + az + xb + ab.
\]
Thus
\[
    \abs*{yw - xz} = \abs*{az + bx + ab} \leq \abs*{az} + \abs*{bx} + \abs*{ab} = \abs*{a}\abs*{z} + \abs*{b}\abs*{x} + \abs*{a}\abs*{b}.
\]
Since \(\abs*{a} \leq \varepsilon\) and \(\abs*{b} \leq \delta\), we thus have
\[
    \abs*{yw - xz} \leq \varepsilon\abs*{z} + \delta\abs*{x} + \varepsilon\delta
\]
and thus that \(yw\) and \(xz\) are \((\varepsilon\abs*{z} + \delta\abs*{x} + \varepsilon\delta)\)-close.
\end{proof}

\begin{remark}\label{4.3.8}
One should compare statements (a)-(c) of Proposition \ref{4.3.7} with the reflexive, symmetric, and transitive axioms of equality.
It is often useful to think of the notion of ``\(\varepsilon\)-close'' as an approximate substitute for that of equality in analysis.
\end{remark}

\begin{definition}[Exponentiation to a natural number]\label{4.3.9}
Let \(x\) be a rational number.
To raise \(x\) to the power \(0\), we define \(x^0 \coloneqq 1\);
in particular we define \(0^0 \coloneqq 1\).
Now suppose inductively that \(x^n\) has been defined for some natural number \(n\), then we define \(x^{n+1} \coloneqq x^n \times x\).
\end{definition}

\begin{proposition}[Properties of exponentiation, I]\label{4.3.10}
Let \(x\), \(y\) be rational numbers, and let \(n\), \(m\) be natural numbers.
\begin{enumerate}
    \item We have \(x^n x^m = x^{n + m}\), \((x^n)^m = x^{nm}\), and \((xy)^n = x^n y^n\).
    \item Suppose \(n > 0\).
    Then we have \(x^n = 0\) if and only if \(x = 0\).
    \item If \(x \geq y \geq 0\), then \(x^n \geq y^n \geq 0\).
    If \(x > y \geq 0\) and \(n > 0\), then \(x^n > y^n \geq 0\).
    \item We have \(\abs*{x^n} = \abs*{x}^n\).
\end{enumerate}
\end{proposition}

\begin{proof}{(a)}
We first prove that \(x^n x^m = x^{n + m}\).
We use induction on \(n\).
For \(n = 0\), \(x^0 x^m = 1 x^m = x^m\) by Definition \ref{4.3.9} and Proposition \ref{4.2.4}.
And \(x^{0 + m} = x^m\) by Proposition \ref{4.2.4}.
So \(x^0 x^m = x^{0 + m}\), and the base case holds.
Suppose inductively that for some \(n\), \(x^n x^m = x^{n + m}\).
Then for \(n++\), \(x^{n++} x^m = (x^n x) x^m = x^n (x x^m) = x^n (x^m x) = (x^n x^m)x\) by Definition \ref{4.3.9} and Proposition \ref{4.2.4}.
And \(x^{(n++) + m} = x^{(n + m)++} = x^{n + m} x = (x^n x^m)x\) by Definition \ref{2.2.1} and induction hypothesis.
So \(x^{n++} x^m = x^{(n++) + m}\), and this close the induction.

Next we prove that \((x^n)^m = x^{nm}\).
We use induction on \(m\).
For \(m = 0\), \((x^n)^0 = 1\) by Definition \ref{4.3.9}.
And \(x^{n0} = x^0 = 1\) by Additional Corollary \ref{ac 2.3.2} and Definition \ref{4.3.9}.
So \((x^n)^0 = x^{n0}\), and the base case holds.
Suppose inductively that for some \(m\), \((x^n)^m = x^{nm}\).
Then for \(m++\), \((x^n)^{m++} = (x^n)^m x^n = x^{nm} x^n\) by Definition \ref{4.3.9} and induction hypothesis.
And \(x^{n(m++)} = x^{nm + n} = x^{nm} x^n\) by Additional Corollary \ref{ac 2.3.3} and previous prove.
So \((x^n)^{m++} = x^{n(m++)}\), and this close the induction.

Finally we prove that \((xy)^n = x^n y^n\).
We use induction on \(n\).
For \(n = 0\), \((xy)^0 = 1\) by Definition \ref{4.3.9}.
And \(x^0 y^0 = 1 \times 1 = 1\) by Definition \ref{4.3.9}.
So \((xy)^0 = x^0 y^0\), and the base case holds.
Suppose inductively that for some \(n\), \((xy)^n = x^n y^n\).
Then for \(n++\), \((xy)^{n++} = (xy)^n (xy) = (x^n y^n)(xy)\) by Definition \ref{4.3.9} and induction hypothesis.
And \(x^{n++} y^{n++} = (x^n x)(y^n y) = (x^n (x y^n))y = (x^n (y^n x))y = (x^n y^n)(xy)\) by Definition \ref{4.3.9} and Proposition \ref{4.2.4}.
So \((xy)^{n++} = x^{n++} y^{n++}\), and this close the induction.
\end{proof}

\begin{proof}{(b)}
We first prove that \(x^n = 0\) implies \(x = 0\).
We use induction on \(n\) and begin with \(n = 1\).
For \(n = 1\), \(x^1 = x^0 x = 1x = x = 0\) by Definition \ref{4.3.9} and Proposition \ref{4.2.4}.
So \(x^1 = 0 \implies x = 0\), and the base case holds.
Suppose inductively that for some \(n\), \(x^n = 0 \implies x = 0\).
Then for \(n++\), \(x^{n++} = x^n x = 0\) by Definition \ref{4.3.9}.
Suppose for sake of contradiction that \(x \neq 0\).
Then \(x^n x = 0 \implies x^n = 0 / x = 0\).
But by induction hypothesis, \(x^n = 0 \implies x = 0\), a contradiction.
Thus \(x = 0\), which means \(x^{n++} = 0 \implies x = 0\), and this close the induction.

Now we prove that \(x = 0\) implies \(x^n = 0\).
Since \(n > 0\), \(x^n = x^{(n - 1) + 1} = x^{n - 1} x = x^{n - 1} 0 = 0\) by Proposition \ref{4.3.10}(a).
Thus \(x = 0 \implies x^n = 0\).
\end{proof}

\begin{proof}{(c)}
We first prove that \(x \geq y \geq 0\) implies \(x^n \geq y^n \geq 0\).
We use induction on \(n\).
For \(n = 0\), \(x^0 = 1\) and \(y^0 = 1\) by Definition \ref{4.3.9}, and \(1 \geq 1 \geq 0\), so the base case holds.
Suppose inductively that for some \(n\), \(x^n \geq y^n \geq 0\).
Then for \(n++\), \(x^{n++} = x^n x \geq y^n x\) and \(y^n x \geq y^n y = y^{n++}\) by Definition \ref{4.3.9}, induction hypothesis, Proposition \ref{4.2.9} and the given conditions.
So \(x^{n++} \geq y^{n++}\) by Proposition \ref{4.2.4}.
And \(y^{n++} = y^n y \geq y^n 0 = 0\) by Definition \ref{4.3.9} and the given conditions.
So \(y^{n++} \geq 0\).
Thus \(x^{n++} \geq y^{n++} \geq 0\), and this close the induction.

Now we prove that \(x > y \geq 0\) and \(n > 0\), then \(x^n > y^n \geq 0\).
We use induction on \(n\) and begin with \(n = 1\).
For \(n = 1\), \(x^1 = x^0 x = 1x = x\) and \(y^1 = y^0 y = 1y = y\) by Definition \ref{4.3.9} and Proposition \ref{4.2.4}.
So \(x^1 > y^1 \geq 0\) by the given conditions, and the base case holds.
Suppose inductively that for some \(n\), \(x^n > y^n \geq 0\).
Then for \(n++\), \(x^{n++} = x^n x > y^n x\) and \(y^n x > y^n y = y^{n++}\) by Definition \ref{4.3.9}, induction hypothesis, Proposition \ref{4.2.9} and the given conditions.
So \(x^{n++} > y^{n++}\) by Proposition \ref{4.2.4}.
And \(y^{n++} = y^n y \geq y^n 0 = 0\) by Definition \ref{4.3.9} and the given conditions.
So \(y^{n++} \geq 0\).
Thus \(x^{n++} > y^{n++} \geq 0\), and this close the induction.
\end{proof}

\begin{proof}{(d)}
We use induction on \(n\).
For \(n = 0\), \(\abs*{x^0} = \abs*{1} = 1\) and \(\abs*{x}^0 = 1\) by Definition \ref{4.3.9} and Definition \ref{4.3.1}.
So \(\abs*{x^0} = \abs*{x}^0\), and the base case holds.
Suppose inductively that for some \(n\), \(\abs*{x^n} = \abs*{x}^n\).
Then for \(n++\), \(\abs*{x^{n++}} = \abs*{x^n x} = \abs*{x^n}\abs*{x} = \abs*{x}^n \abs*{x}\) and \(\abs*{x}^{n++} = \abs*{x}^n \abs*{x}\) by Definition \ref{4.3.9}, Proposition \ref{4.3.3} and induction hypothesis.
So \(\abs*{x^{n++}} = \abs*{x}^{n++}\), and this close the induction.
\end{proof}

\begin{definition}[Exponentiation to a negative number]\label{4.3.11}
Let \(x\) be a non-zero rational number.
Then for any negative integer \(-n\), we define \(x^{-n} \coloneqq 1 / x^n\).
\end{definition}

\begin{proposition}[Properties of exponentiation, II]\label{4.3.12}
Let \(x\), \(y\) be nonzero rational numbers, and let \(n\), \(m\) be integers.
\begin{enumerate}
    \item We have \(x^n x^m = x^{n + m}\), \((x^n)^m = x^{nm}\), and \((xy)^n = x^n y^n\).
    \item If \(x \geq y > 0\), then \(x^n \geq y^n > 0\) if \(n\) is positive, and \(0 < x^n \leq y^n\) if \(n\) is negative.
    \item If \(x, y > 0\), \(n \neq 0\), and \(x^n = y^n\), then \(x = y\).
    \item We have \(\abs*{x^n} = \abs*{x}^n\).
\end{enumerate}
\end{proposition}

\begin{proof}{(a)}
We first prove that \(x^n x^m = x^{n + m}\).
By Lemma \ref{4.1.5}, exactly one of the following three statements is true:
\begin{enumerate}[label=(\Roman*)]
    \item \(n = 0\).
    Then \(x^0 x^m = 1x^m = x^m\) by Definition \ref{4.3.9} and Proposition \ref{4.2.4}.
    And \(x^{0 + m} = x^m\) by Proposition \ref{4.2.4}.
    So \(x^0 x^m = x^{0 + m}\).
    \item \(n\) is a positive integer.
    Again by Lemma \ref{4.1.5}, exactly one of the following three statements is true:
    \begin{enumerate}[label=(\roman*)]
        \item \(m = 0\).
        Then \(x^n x^0 = x^0 x^n\) and \(x^{n + 0} = x^{0 + n}\) by Proposition \ref{4.2.4} and Proposition \ref{4.1.6}, which is the same case as \(n = 0\).
        \item \(m\) is a positive integer.
        Then by Proposition \ref{4.3.10}, \(x^n x^m = x^{n + m}\).
        \item \(m\) is a negative integer.
        Again by Lemma \ref{4.1.5}, exactly one of the following three statements is true:
        \begin{enumerate}[label=(\arabic*)]
            \item \(n + m = 0\).
            Then \(n = -m\) and \(x^n x^m = x^{-m} x^m = 1\) by Proposition \ref{4.2.4}.
            And \(x^{n + m} = x^{(-m) + m} = x^0 = 1\) by Proposition \ref{4.2.4} and Definition \ref{4.3.9}.
            So \(x^{-m} x^m = x^{(-m) + m}\).
            \item \(n + m\) is a positive integer.
            By Proposition \ref{4.3.10}, \(x \neq 0 \implies x^{-m} \neq 0\).
            Then
            \begin{align*}
            & (x^n x^m) x^{-m} \\
            &= x^n (x^m x^{-m}) & \text{(by Proposition \ref{4.2.4})} \\
            &= x^n 1 & \text{(by Proposition \ref{4.2.4})} \\
            &= x^n & \text{(by Proposition \ref{4.2.4})} \\
            &= x^{n + 0} & \text{(by Proposition \ref{4.1.6})} \\
            &= x^{n + (m + (-m))} & \text{(by Proposition \ref{4.1.6})} \\
            &= x^{(n + m) + (-m)} & \text{(by Proposition \ref{4.1.6})} \\
            &= x^{n + m} x^{-m}. & \text{(by Proposition \ref{4.3.10})}
            \end{align*}
            So
            \begin{align*}
            & (x^n x^m) x^{-m} = x^{n + m} x^{-m} \\
            \implies & ((x^n x^m) x^{-m}) x^m = (x^{n + m} x^{-m}) x^m & \text{(by Lemma \ref{4.2.3})} \\
            \implies & (x^n x^m)(x^{-m} x^m) = x^{n + m} (x^{-m} x^m) & \text{(by Proposition \ref{4.2.4})} \\
            \implies & (x^n x^m)1 = x^{n + m} 1 & \text{(by Proposition \ref{4.2.4})} \\
            \implies & x^n x^m = x^{n + m}. & \text{(by Proposition \ref{4.2.4})} \\
            \end{align*}
            \item \(n + m\) is a negative integer.
            Then \(-(n + m)\) is a positive integer.
            By Proposition \ref{4.3.10}, \(x \neq 0 \implies x^n \neq 0\), \(x^{-m} \neq 0\) and \(x^{-(n + m)} \neq 0\).
            So
            \begin{align*}
            & x^n (x^{-n} x^{-m}) \\
            &= (x^n x^{-n})x^{-m} & \text{(by Proposition \ref{4.2.4})} \\
            &= 1x^{-m} & \text{(by Proposition \ref{4.2.4})} \\
            &= x^{-m} & \text{(by Proposition \ref{4.2.4})} \\
            &= x^{0 + (-m)} & \text{(by Proposition \ref{4.1.6})} \\
            &= x^{(n + (-n)) + (-m)} & \text{(by Proposition \ref{4.1.6})} \\
            &= x^{n + ((-n) + (-m))} & \text{(by Proposition \ref{4.1.6})} \\
            &= x^{n + ((-1)n + (-1)m)} & \text{(by Exercise \ref{ex 4.1.3})} \\
            &= x^{n + (-1)(n + m)} & \text{(by Proposition \ref{4.1.6})} \\
            &= x^{n + (-(n + m))} & \text{(by Exercise \ref{ex 4.1.3})} \\
            &= x^n x^{-(n + m)}. & \text{(by Proposition \ref{4.3.10})} \\
            \end{align*}
            And
            \begin{align*}
            & x^n (x^{-n} x^{-m}) = x^n x^{-(n + m)} \\
            \implies & x^{-n} (x^n (x^{-n} x^{-m})) = x^{-n} (x^n x^{-(n + m)}) & \text{(by Lemma \ref{4.2.3})} \\
            \implies & (x^{-n} x^n)(x^{-n} x^{-m}) = (x^{-n} x^n)x^{-(n + m)} & \text{(by Proposition \ref{4.2.4})} \\
            \implies & 1(x^{-n} x^{-m}) = 1x^{-(n + m)} & \text{(by Proposition \ref{4.2.4})} \\
            \implies & x^{-n} x^{-m} = x^{-(n + m)} & \text{(by Proposition \ref{4.2.4})} \\
            \implies & x^n (x^{-n} x^{-m}) = x^n x^{-(n + m)} & \text{(by Lemma \ref{4.2.3})} \\
            \implies & (x^n x^{-n}) x^{-m} = x^n x^{-(n + m)} & \text{(by Proposition \ref{4.2.4})} \\
            \implies & 1x^{-m} = x^n x^{-(n + m)} & \text{(by Proposition \ref{4.2.4})} \\
            \implies & x^{-m} = x^n x^{-(n + m)} & \text{(by Proposition \ref{4.2.4})} \\
            \implies & x^m x^{-m} = x^m (x^n x^{-(n + m)}) & \text{(by Lemma \ref{4.2.3})} \\
            \implies & 1 = x^m (x^n x^{-(n + m)}) & \text{(by Proposition \ref{4.2.4})} \\
            \implies & 1 = (x^m x^n) x^{-(n + m)} & \text{(by Proposition \ref{4.2.4})} \\
            \implies & 1x^{n + m} = ((x^m x^n) x^{-(n + m)}) x^{n + m} & \text{(by Lemma \ref{4.2.3})} \\
            \implies & 1x^{n + m} = (x^m x^n)(x^{-(n + m)} x^{n + m}) & \text{(by Proposition \ref{4.2.4})} \\
            \implies & 1x^{n + m} = (x^m x^n)1 & \text{(by Proposition \ref{4.2.4})} \\
            \implies & x^{n + m} = x^m x^n & \text{(by Proposition \ref{4.2.4})} \\
            \implies & x^{n + m} = x^n x^m. & \text{(by Proposition \ref{4.2.4})}
            \end{align*}
        \end{enumerate}
    \end{enumerate}
    \item \(n\) is a negative integer.
    Again by Lemma \ref{4.1.5}, exactly one of the following three statements is true:
    \begin{enumerate}[label=(\roman*)]
        \item \(m = 0\).
        Then \(x^n x^0 = x^0 x^n\) and \(x^{n + 0} = x^{0 + n}\) by Proposition \ref{4.2.4} and Proposition \ref{4.1.6}, which is the same case as \(n = 0\).
        \item \(m\) is a positive integer.
        Then \(x^n x^m = x^m x^n\) and \(x^{n + m} = x^{m + n}\) by Proposition \ref{4.2.4} and Proposition \ref{4.1.6}, which is the same case as \(n\) is positive and \(m\) is negative.
        \item \(m\) is a negative integer.
        Then \(-n, -m, -(n + m)\) are positive integers.
        By Proposition \ref{4.3.10}, \(x \neq 0 \implies x^{-n} \neq 0\), \(x^{-m} \neq 0\) and \(x^{-(n + m)} \neq 0\).
        So
        \begin{align*}
        & x^{-n} x^{-m} = x^{-(n + m)} & \text{(by Proposition \ref{4.3.10})} \\
        \implies & x^n (x^{-n} x^{-m}) = x^n x^{-(n + m)} & \text{(by Lemma \ref{4.2.3})} \\
        \implies & (x^n x^{-n}) x^{-m} = x^n x^{-(n + m)} & \text{(by Proposition \ref{4.2.4})} \\
        \implies & 1x^{-m} = x^n x^{-(n + m)} & \text{(by Proposition \ref{4.2.4})} \\
        \implies & x^{-m} = x^n x^{-(n + m)} & \text{(by Proposition \ref{4.2.4})} \\
        \implies & x^m x^{-m} = x^m (x^n x^{-(n + m)}) & \text{(by Lemma \ref{4.2.3})} \\
        \implies & 1 = x^m (x^n x^{-(n + m)}) & \text{(by Proposition \ref{4.2.4})} \\
        \implies & 1 = (x^m x^n) x^{-(n + m)} & \text{(by Proposition \ref{4.2.4})} \\
        \implies & 1x^{n + m} = ((x^m x^n) x^{-(n + m)})x^{n + m} & \text{(by Lemma \ref{4.2.3})} \\
        \implies & 1x^{n + m} = (x^m x^n)(x^{-(n + m)} x^{n + m}) & \text{(by Proposition \ref{4.2.4})} \\
        \implies & 1x^{n + m} = (x^m x^n)1 & \text{(by Proposition \ref{4.2.4})} \\
        \implies & x^{n + m} = x^m x^n & \text{(by Proposition \ref{4.2.4})} \\
        \implies & x^{n + m} = x^n x^m. & \text{(by Proposition \ref{4.2.4})}
        \end{align*}
    \end{enumerate}
\end{enumerate}
From all cases above, we can conclude that \(x^n x^m = x^{n + m}\).

Next we prove that \((x^n)^m = x^{nm}\).
By Lemma \ref{4.1.5}, exactly one of the following three statements is true:
\begin{enumerate}[label=(\Roman*)]
    \item \(n = 0\).
    Then by Definition \ref{4.3.9}, \((x^0)^m = 1^m\) and \(x^{0m} = x^0 = 1\).
    Again By Lemma \ref{4.1.5}, exactly one of the following three statements is true:
    \begin{enumerate}[label=(\roman*)]
        \item \(m = 0\).
        Then \(1^0 = 1\) by Definition \ref{4.3.9}, so \((x^0)^0 = x^{0 \times 0}\).
        \item \(m\) is a positive integer.
        We claim that \(1^m = 1\) by using induction on \(m\).
        For \(m = 0\), \(1^0 = 1\) by Definition \ref{4.3.9}, so the base case holds.
        Suppose inductively that for some \(m\), \(1^m = 1\).
        Then for \(m++\), \(1^{m++} = 1^m \times 1 = 1 \times 1 = 1\) by Definition \ref{4.3.9} and induction hypothesis, and this close the induction.
        So \((x^0)^m = x^{0m}\).
        \item \(m\) is a negative integer.
        Then \(-m\) is a positive integer, and \(1^m = 1 / 1^{-m}\) by Definition \ref{4.3.11}.
        From previous prove, we show that \(1^{-m} = 1\).
        So \((x^0)^m = x^{0m}\).
    \end{enumerate}
    \item \(n\) is a positive integer.
    Again by Lemma \ref{4.1.5}, exactly one of the following three statements is true:
    \begin{enumerate}[label=(\roman*)]
        \item \(m = 0\).
        Then by Definition \ref{4.3.9}, \((x^n)^0 = 1\) and \(x^{n0} = x^0 = 1\).
        So \((x^n)^0 = x^{n0}\).
        \item \(m\) is a positive integer.
        Then by Proposition \ref{4.3.10}, \((x^n)^m = x^{nm}\).
        \item \(m\) is a negative integer.
        Then \(-m\) is a positive integer.
        So
        \begin{align*}
        & (x^n)^{-m} = x^{n(-m)} & \text{(by Proposition \ref{4.3.10})} \\
        \implies & (x^n)^{-m} = x^{n((-1)m)} & \text{(by Additional Corollary \ref{ac 4.2.3})} \\
        \implies & (x^n)^{-m} = x^{(n(-1))m} & \text{(by Proposition \ref{4.1.6})} \\
        \implies & (x^n)^{-m} = x^{((-1)n)m} & \text{(by Proposition \ref{4.1.6})} \\
        \implies & (x^n)^{-m} = x^{(-1)(nm)} & \text{(by Proposition \ref{4.1.6})} \\
        \implies & (x^n)^{-m} = x^{-(nm)} & \text{(by Additional Corollary \ref{ac 4.2.3})} \\
        \implies & 1 / (x^n)^m = 1 / x^{nm} & \text{(by Definition \ref{4.3.11})} \\
        \implies & 1x^{nm} = 1(x^n)^m  & \text{(by Definition \ref{4.2.1})} \\
        \implies & x^{nm} = (x^n)^m.  & \text{(by Proposition \ref{4.2.4})}
        \end{align*}
    \end{enumerate}
    \item \(n\) is a negative integer.
    Then \(-n\) is a positive integer.
    Again by Lemma \ref{4.1.5}, exactly one of the following three statements is true:
    \begin{enumerate}[label=(\roman*)]
        \item \(m = 0\).
        Then by Definition \ref{4.3.9}, \((x^n)^0 = 1\) and \(x^{n0} = x^0 = 1\).
        So \((x^n)^0 = x^{n0}\).
        \item \(m\) is a positive integer.
        So
        \begin{align*}
        & (x^{-n})^m = x^{(-n)m} & \text{(by Proposition \ref{4.3.10})} \\
        \implies & (x^{-n})^m = x^{((-1)n)m} & \text{(by Additional Corollary \ref{ac 4.2.3})} \\
        \implies & (x^{-n})^m = x^{(-1)(nm)} & \text{(by Proposition \ref{4.1.6})} \\
        \implies & (x^{-n})^m = x^{-(nm)} & \text{(by Additional Corollary \ref{ac 4.2.3})} \\
        \implies & (1 / x^n)^m = 1 / x^{nm}. & \text{(by Definition \ref{4.3.11})}
        \end{align*}
        We claim that \((1 / x^n)^m = 1 / (x^n)^m\) by using induction on \(m\).
        For \(m = 0\), \((1 / x^n)^0 = 1\) and \(1 / (x^n)^0 = 1 / 1 = 1\) by Definition \ref{4.3.9}.
        So \((1 / x^n)^0 = 1 / (x^n)^0\), and the base case holds.
        Suppose inductively that for some \(m\), \((1 / x^n)^m = 1 / (x^n)^m\).
        Then for \(m++\), \((1 / x^n)^{m++} = (1 / x^n)^m \times (1 / x^n) = 1 / (x^n)^m \times (1 / x^n) = (1 \times 1) / ((x^n)^m \times x^n) = 1 / (x^n)^{m++}\) by Definition \ref{4.3.9}, induction hypothesis and Definition \ref{4.2.2}.
        This close the induction.
        So
        \begin{align*}
        & 1 / (x^n)^m = 1 / x^{nm} \\
        \implies & 1x^{nm} = 1(x^n)^m & \text{(by Definition \ref{4.2.1})} \\
        \implies & x^{nm} = (x^n)^m. & \text{(by Proposition \ref{4.2.4})}
        \end{align*}
        \item \(m\) is a negative integer.
        Then \(-m\) is a positive integer.
        So
        \begin{align*}
        & (x^n)^m \\
        &= 1 / (x^n)^{-m} & \text{(by Definition \ref{4.3.11})} \\
        &= 1 / x^{n(-m)} & \text{(from case above)} \\
        &= 1 / x^{n((-1)m)} & \text{(by Additional Corollary \ref{ac 4.2.3})} \\
        &= 1 / x^{(n(-1))m} & \text{(by Proposition \ref{4.2.4})} \\
        &= 1 / x^{((-1)n)m} & \text{(by Proposition \ref{4.2.4})} \\
        &= 1 / x^{(-1)(nm)} & \text{(by Proposition \ref{4.2.4})} \\
        &= 1 / x^{-(nm)} & \text{(by Additional Corollary \ref{ac 4.2.3})} \\
        &= x^{nm}. & \text{(by Definition \ref{4.3.11})}
        \end{align*}
    \end{enumerate}
\end{enumerate}
From all cases above, we can conclude that \((x^n)^m = x^{nm}\).

Finally we prove that \((xy)^n = x^n y^n\).
By Lemma \ref{4.1.5}, exactly one of the following three statements is true:
\begin{enumerate}[label=(\roman*)]
    \item \(n = 0\).
    Then by Proposition \ref{4.3.10}, \((xy)^0 = x^0 y^0\).
    \item \(n\) is a positive integer.
    Then by Proposition \ref{4.3.10}, \((xy)^n = x^n y^n\).
    \item \(n\) is a negative integer.
    Then \(-n\) is a positive integer.
    So
    \begin{align*}
    & (xy)^{-n} = x^{-n} y^{-n} & \text{(by Proposition \ref{4.3.10})} \\
    \implies & 1 / (xy)^n = (1 / x^n)(1 / y^n) & \text{(by Definition \ref{4.3.11})} \\
    \implies & 1 / (xy)^n = (1 \times 1) / (x^n y^n) & \text{(by Definition \ref{4.2.2})} \\
    \implies & 1 / (xy)^n = 1 / (x^n y^n) \\
    \implies & 1(x^n y^n) = 1(xy)^n & \text{(by Definition \ref{4.2.1})} \\
    \implies & (x^n y^n) = (xy)^n. & \text{(by Proposition \ref{4.2.4})}
    \end{align*}
\end{enumerate}
From all cases above, we can conclude that \((xy)^n = x^n y^n\).
\end{proof}

\begin{proof}{(b)}
By Definition \ref{4.2.8}, \(x \geq y > 0 \implies x \geq y \geq 0\).
If \(n\) is a positive integer, then by Proposition \ref{4.3.10}, \(x \geq y \geq 0 \implies x^n \geq y^n \geq 0\).
By the given conditions and Proposition \ref{4.3.10}, \(y \neq 0 \implies y^n \neq 0\).
So \(x \geq y > 0 \implies x^n \geq y^n > 0\) when \(n\) is a positive integer.

If \(n\) is a negative integer, then let \(n = -a\), where \(a\) is a positive integer.
By the given conditions we have two cases:
\begin{enumerate}[label=(\roman*)]
    \item \(x > y\).
    By Definition \ref{4.2.6}, \((x > 0 \implies 1 / x > 0) \land (y > 0 \implies 1 / y > 0)\).
    By Additional Corollary \ref{ac 4.2.5} and Definition \ref{4.2.2}, \((1 / x) \times (1 / y) = 1 / xy > 0\).
    So \(x > y \implies x \times (1 / xy) > y \times (1 / xy) \implies 1 / y > 1 / x\) by Proposition \ref{4.2.9}.
    From previous prove, we get \(1 / y \geq 1 / x > 0 \implies (1 / y)^a \geq (1 / x)^a > 0\).
    But by Definition \ref{4.3.11}, Proposition \ref{4.3.12}(a) and Additional Corollary \ref{ac 4.2.3}, \((1 / y)^a = (y^{-1})^a = y^{(-1)a} = y^{-a} = y^n\) and \((1 / x)^a = (x^{-1})^a = x^{(-1)a} = x^{-a} = x^n\).
    So \(x \geq y > 0 \implies y^n \geq x^n > 0\) when \(n\) is a negative integer.
    \item \(x = y\).
    Then \(x^a = y^a > 0\) by Proposition \ref{4.3.10}.
    By Definition \ref{4.2.6}, \((x^a > 0 \implies 1 / x^a > 0) \land (y^a > 0 \implies 1 / y^a > 0)\).
    But by Definition \ref{4.3.11}, \((1 / x^a = x^{-a} = x^n) \land (1 / y^a = y^{-a} = y^n)\).
    And by Definition \ref{4.2.8}, \(x^n = y^n \implies y^n \geq x^n\).
    So \(x \geq y > 0 \implies y^n \geq x^n > 0\) when \(n\) is a negative integer.
\end{enumerate}
From all cases above, we conclude that \(x \geq y > 0 \implies y^n \geq x^n > 0\) when \(n\) is a negative integer.
\end{proof}

\begin{proof}{(c)}
Suppose for sake of contradiction that \(x \neq y\)
Then by Proposition \ref{4.2.9}, exactly one of the following two statements is true:
\begin{enumerate}[label=(\Roman*)]
    \item \(x > y\).
    Then by Lemma \ref{4.1.5} and the given conditions, exactly one of the following two statements is true:
    \begin{enumerate}[label=(\roman*)]
        \item \(n\) is a positive integer.
        But by Proposition \ref{4.3.10}, \(x^n > y^n\), a contradiction.
        \item \(n\) is a negative integer.
        Let \(n = -a\), where \(a\) is a negative integer.
        By Definition \ref{4.2.6}, \((x > 0 \implies 1 / x > 0) \land (y > 0 \implies 1 / y > 0)\).
        By Additional Corollary \ref{ac 4.2.5} and Definition \ref{4.2.2}, \((1 / x) \times (1 / y) = 1 / xy > 0\).
        So \(x > y \implies x \times (1 / xy) > y \times (1 / xy) \implies 1 / y > 1 / x\) by Proposition \ref{4.2.9}.
        By Proposition \ref{4.3.10}, \((1 / y)^a > (1 / x)^a\).
        But by Definition \ref{4.3.11}, Proposition \ref{4.3.12}(a) and Additional Corollary \ref{ac 4.2.3}, \((1 / y)^a = (y^{-1})^a = y^{(-1)a} = y^{-a} = y^n\) and \((1 / x)^a = (x^{-1})^a = x^{(-1)a} = x^{-a} = x^n\).
        So \(x > y \implies y^n > x^n\), a contradiction.
    \end{enumerate}
    \item \(x < y\).
    By Proposition \ref{4.2.9}, \(x < y \implies y > x\), which just the same as \(x > y\).
\end{enumerate}
From all cases above we get a contradiction, so \(x = y\).
\end{proof}

\begin{proof}{(d)}
By Lemma \ref{4.1.5}, exactly one of the following three statements is true:
\begin{enumerate}[label=(\Roman*)]
    \item \(n = 0\).
    Then by Proposition \ref{4.3.10}, \(\abs*{x^0} = \abs*{x}^0\).
    \item \(n\) is a positive integer.
    Then by Proposition \ref{4.3.10}, \(\abs*{x^n} = \abs*{x}^n\).
    \item \(n\) is a negative integer.
    Then by Definition \ref{4.3.11}, \(\abs*{x^n} = \abs*{1 / x^{-n}}\) and \(\abs*{x}^n = 1 / \abs*{x}^{-n}\).
    By Lemma \ref{4.2.7}, exactly one of the following three statements is true:
    \begin{enumerate}[label=(\roman*)]
        \item \(x^{-n} = 0\).
        This case does not exist because \(x \neq 0 \implies x^{-n} \neq 0\).
        \item \(x^{-n}\) is a positive rational number.
        Then by Definition \ref{4.2.6} and Definition \ref{4.3.1}, \(1 / x^{-n}\) is a positive rational number and \(\abs*{1 / x^{-n}} = 1 / x^{-n} = 1 / \abs*{x^{-n}}\).
        And by Proposition \ref{4.3.10}, \(1 / \abs*{x}^{-n} = 1 / \abs*{x^{-n}}\).
        So \(\abs*{x^n} = \abs*{x}^n\).
        \item \(x^{-n}\) is a negative rational number.
        By Additional Corollary \ref{ac 4.2.3}, \(1 / x^{-n}\) is a negative rational number.
        By Definition \ref{4.3.1}, \(\abs*{1 / x^{-n}} = -(1 / x^{-n})\).
        Again by Additional Corollary \ref{ac 4.2.3}, \(-(1 / x^{-n}) = 1 / -(x^{-n})\).
        Again by Definition \ref{4.3.1}, \(1 / -(x^{-n}) = 1 / \abs*{x^{-n}}\).
        And by Proposition \ref{4.3.10}, \(1 / \abs*{x}^{-n} = 1 / \abs*{x^{-n}}\).
        So \(\abs*{x^n} = \abs*{x}^n\).
    \end{enumerate}
\end{enumerate}
From all cases above, we conclude that \(\abs*{x^n} = \abs*{x}^n\).
\end{proof}

\exercisesection

\begin{exercise}\label{ex 4.3.1}
Prove Proposition \ref{4.3.3}.
\end{exercise}

\begin{proof}
See Proposition \ref{4.3.3}.
\end{proof}

\begin{exercise}\label{ex 4.3.2}
Prove the remaining claims in Proposition \ref{4.3.7}.
\end{exercise}

\begin{proof}
See Proposition \ref{4.3.7}.
\end{proof}

\begin{exercise}\label{ex 4.3.3}
Prove Proposition \ref{4.3.10}.
\end{exercise}

\begin{proof}
See Proposition \ref{4.3.10}.
\end{proof}

\begin{exercise}
Prove Proposition \ref{4.3.12}.
\end{exercise}

\begin{proof}
See Proposition \ref{4.3.12}.
\end{proof}

\begin{exercise}\label{ex 4.3.5}
Prove that \(2^N \geq N\) for all positive integers \(N\).
\end{exercise}

\begin{proof}
We use induction on \(N\) and begin with \(N = 1\).
For \(N = 1\), \(2^1 = 2^0 \times 2 = 1 \times 2 = 2 \geq 1\) by Definition \ref{4.3.9}, so the base case holds.
Suppose inductively that for some \(N\), \(2^N \geq N\).
Then for \(N++\),
\begin{align*}
& (2N = N + N) \land (N \text{ is a positive integer}) \\
\implies & N < 2N & \text{(by Definition \ref{2.2.11})} \\
\implies & N++ \leq 2N. & \text{(by Proposition \ref{2.2.12})} \\
& 2^N \geq N & \text{(by induction hypothesis)} \\
\implies & N \leq 2^N & \text{(by Lemma \ref{4.2.3})} \\
\implies & 2N \leq 2 \times 2^N & \text{(by Lemma \ref{4.2.3})} \\
\implies & 2N \leq 2^N \times 2 & \text{(by Proposition \ref{4.2.4})} \\
\implies & 2N \leq 2^{N++} & \text{(by Definition \ref{4.3.9})} \\
\implies & N++ \leq 2^{N++} & \text{(by Proposition \ref{4.2.9})} \\
\implies & 2^{N++} \geq N++. & \text{(by Proposition \ref{4.2.9})}
\end{align*}
This close the induction.
\end{proof}