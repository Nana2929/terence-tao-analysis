\section{Absolute value and exponentiation}

\begin{definition}[Absolute value]\label{4.3.1}
If \(x\) is a rational number, the \emph{absolute value} \(|x|\) of \(x\) is defined as follows.
If \(x\) is positive, then \(|x| \coloneqq x\).
If \(x\) is negative, then \(|x| \coloneqq -x\).
If \(x\) is zero, then \(|x| \coloneqq 0\).
\end{definition}

\begin{definition}[Distance]\label{4.3.2}
Let \(x\) and \(y\) be rational numbers.
The quantity \(|x - y|\) is called the \emph{distance between \(x\) and \(y\)} and is sometimes denoted \(d(x, y)\), thus \(d(x, y) \coloneqq |x - y|\).
\end{definition}

\begin{proposition}[Basic properties of absolute value and distance]\label{4.3.3}
Let \(x\), \(y\), \(z\) be rational numbers.
\begin{enumerate}
    \item (Non-degeneracy of absolute value)
    We have \(|x| \geq 0\).
    Also, \(|x| = 0\) if and only if \(x\) is \(0\).
    \item (Triangle inequality for absolute value)
    We have \(|x + y| \leq |x| + |y|\).
    \item We have the inequalities \(-y \leq x \leq y\) if and only if \(y \geq |x|\).
    In particular, we have \(-|x| \leq x \leq |x|\).
    \item (Multiplicativity of absolute value)
    We have \(|xy| = |x| |y|\).
    In particular, \(|-x| = |x|\).
    \item (Non-degeneracy of distance)
    We have \(d(x, y) \geq 0\).
    Also, \(d(x, y) = 0\) if and only if \(x = y\).
    \item (Symmetry of distance)
    \(d(x, y) = d(y, x)\).
    \item (Triangle inequality for distance)
    \(d(x, z) \leq d(x, y) + d(y, z)\).
\end{enumerate}
\end{proposition}

\begin{proof}{(a)}
By Lemma \ref{4.2.7}, exactly one of the three statements is true:
\begin{enumerate}[label=(\roman*)]
    \item \(x = 0\).
    Then by Definition \ref{4.3.1}, \(|x| = 0\).
    \item \(x\) is a positive rational number.
    Then by Definition \ref{4.3.1}, \(|x| = x\), which is a positive rational number.
    \item \(x\) is a negative rational number.
    Then by Definition \ref{4.3.1}, \(|x| = -x\).
    By Definition \ref{4.2.6}, \(x = (-a) / b\), where \(a, b \in \mathds{Z}^+\).
    So \(-x = (-1)x = (-1)((-a) / b) = (-1)((-1)(a / b)) = ((-1)(-1))(a / b) = a / b\) by Additional Corollary \ref{ac 4.2.3} and Proposition \ref{4.2.4}.
    By Definition \ref{4.3.1}, \(-x = a / b\) is a positive rational number.
\end{enumerate}
So \(|x|\) is either \(0\) or positive rational number, which by Definition \ref{4.2.8}, \(|x| \geq 0\).

Now we proof that \(|x| = 0 \iff x = 0\).
By Definition \ref{4.3.1}, \(x = 0 \implies |x| = 0\), so we only need to show that \(|x| = 0 \implies x = 0\).
By Lemma \ref{4.2.7}, exactly one of the following is true:
\(x = 0\), \(x\) is positive rational number or \(x\) is negative rational number.
If \(x\) is positive rational number, then \(|x| = x \neq 0\).
If \(x\) is negative rational number, then \(|x| = -x \neq 0\).
So \(x\) can only be \(0\), which means \(|x| = 0 \implies x = 0\).
\end{proof}

\begin{proof}{(b)}
By Lemma \ref{4.2.7}, \(x\) and \(y\) can both have three different cases.
\begin{enumerate}[label=(\Roman*)]
    \item For \(x = 0\) and
    \begin{enumerate}[label=(\roman*)]
        \item \(y = 0\).
        By Proposition \ref{4.2.4} and Definition \ref{4.3.1}, \(|0 + 0| = |0| = 0\), and \(|0| + |0| = 0 + 0 = 0\).
        Thus \(|x + y| = 0 = |x| + |y|\).
        \item \(y\) is positive.
        By Proposition \ref{4.2.4} and Definition \ref{4.3.1}, \(|0 + y| = |y| = y\), and \(|0| + |y| = 0 + y = y\).
        Thus \(|x + y| = y = |x| + |y|\).
        \item \(y\) is negative.
        By Proposition \ref{4.2.4} and Definition \ref{4.3.1}, \(|0 + y| = |y| = -y\), and \(|0| + |y| = 0 + (-y) = -y\).
        Thus \(|x + y| = -y = |x| + |y|\).
    \end{enumerate}
    \item For \(x\) is positive and
    \begin{enumerate}[label=(\roman*)]
        \item \(y = 0\).
        Which is just equivalent to the case \(x = 0\) and \(y\) is positive.
        \item \(y\) is positive.
        By Additional Corollary \ref{ac 4.2.4} and Definition \ref{4.3.1}, \(|x + y| = x + y\), and \(|x| + |y| = x + y\).
        Thus \(|x + y| = x + y = |x| + |y|\).
        \item \(y\) is negative.
        Let \(y = -a\), where \(a\) is a positive rational number.
        By Proposition \ref{4.2.9}, exactly one of the following three statements is true:
        \begin{enumerate}[label=(\arabic*)]
            \item \(x = a\).
            By Proposition \ref{4.2.4} and Definition \ref{4.3.1}, \((|x| + |-a|) - |x - a| = (|x| + |-a|) - |0| = (x + a) - 0 = x + a\).
            By Additional Corollary \ref{ac 4.2.4}, \(x + a\) is a positive rational number.
            Thus by Definition \ref{4.2.8}, \(|x + y| = 0 < x + a = |x| + |y|\).
            \item \(x > a\).
            By Definition \ref{4.2.8}, \(x - a\) is a positive rational number, so by Definition \ref{4.3.1}, \(|x - a| = x - a\).
            By Definition \ref{4.3.1}, \(|x| + |-a| = x + a\).
            By Proposition \ref{4.2.4} and Additional Corollary \ref{ac 4.2.3}, \((|x| + |y|) - |x + y| = (x + a) - (x - a) = 2a\).
            By Additional Corollary \ref{ac 4.2.5}, \(2a\) is a positive rational number.
            Thus by Definition \ref{4.2.8}, \(|x + y| = x - a < x + a = |x| + |y|\).
            \item \(x < a\).
            By Definition \ref{4.2.8}, \(x - a\) is a negative rational number, so by Definition \ref{4.3.1}, \(|x - a| = -(x - a) = a - x\).
            By Definition \ref{4.3.1}, \(|x| + |-a| = x + a\).
            By Proposition \ref{4.2.4} and Additional Corollary \ref{ac 4.2.3}, \((|x| + |y|) - |x + y| = (x + a) - (a - x) = 2x\).
            By Additional Corollary \ref{ac 4.2.5}, \(2x\) is a positive rational number.
            Thus by Definition \ref{4.2.8}, \(|x + y| = a - x < x + a = |x| + |y|\).
        \end{enumerate}
    \end{enumerate}
    \item For \(x\) is negative and
    \begin{enumerate}[label=(\roman*)]
        \item \(y = 0\).
        Which is just equivalent to the case \(x = 0\) and \(y\) is negative.
        \item \(y\) is positive.
        Which is just equivalent to the case \(x\) is positive and \(y\) is negative.
        \item \(y\) is negative.
        Let \(x = -a\) and \(y = -b\), where \(a, b\) are positive rational number.
        By Proposition \ref{4.2.4} and Definition \ref{4.3.1}, \(|(-a) + (-b)| = |-(a + b)| = -(-(a + b)) = a + b\).
        By Definition \ref{4.3.1}, \(|(-a)| + |(-b)| = (-(-a)) + (-(-b)) = a + b\).
        Thus \(|x + y| = a + b = |x| + |y|\).
    \end{enumerate}
\end{enumerate}
For all cases above, we get either \(|x + y| = |x| + |y|\) or \(|x + y| < |x| + |y|\).
So by Definition \ref{4.2.8}, \(|x + y| \leq |x| + |y|\).
\end{proof}

\exercisesection

\begin{exercise}\label{ex 4.3.1}
Prove Proposition \ref{4.3.3}.
\end{exercise}

\begin{proof}
See Proposition \ref{4.3.3}.
\end{proof}