\section{Absolute value and exponentiation}\label{sec 4.3}

\begin{definition}[Absolute value]\label{4.3.1}
    If \(x\) is a rational number, the \emph{absolute value} \(\abs*{x}\) of \(x\) is defined as follows.
    If \(x\) is positive, then \(\abs*{x} \coloneqq x\).
    If \(x\) is negative, then \(\abs*{x} \coloneqq -x\).
    If \(x\) is zero, then \(\abs*{x} \coloneqq 0\).
\end{definition}

\begin{definition}[Distance]\label{4.3.2}
    Let \(x\) and \(y\) be rational numbers.
    The quantity \(\abs*{x - y}\) is called the \emph{distance between \(x\) and \(y\)} and is sometimes denoted \(d(x, y)\), thus \(d(x, y) \coloneqq \abs*{x - y}\).
\end{definition}

\begin{proposition}[Basic properties of absolute value and distance]\label{4.3.3}
    Let \(x\), \(y\), \(z\) be rational numbers.
    \begin{enumerate}
        \item (Non-degeneracy of absolute value)
              We have \(\abs*{x} \geq 0\).
              Also, \(\abs*{x} = 0\) if and only if \(x\) is \(0\).
        \item (Triangle inequality for absolute value)
              We have \(\abs*{x + y} \leq \abs*{x} + \abs*{y}\).
        \item We have the inequalities \(-y \leq x \leq y\) if and only if \(y \geq \abs*{x}\).
              In particular, we have \(-\abs*{x} \leq x \leq \abs*{x}\).
        \item (Multiplicativity of absolute value)
              We have \(\abs*{xy} = \abs*{x} \abs*{y}\).
              In particular, \(\abs*{-x} = \abs*{x}\).
        \item (Non-degeneracy of distance)
              We have \(d(x, y) \geq 0\).
              Also, \(d(x, y) = 0\) if and only if \(x = y\).
        \item (Symmetry of distance)
              \(d(x, y) = d(y, x)\).
        \item (Triangle inequality for distance)
              \(d(x, z) \leq d(x, y) + d(y, z)\).
    \end{enumerate}
\end{proposition}

\begin{proof}{(a)}
    By Lemma \ref{4.2.7} we know that exactly one of the three statements is true:
    \begin{enumerate}[label=(\Roman*)]
        \item \(x = 0\).
              Then by Definition \ref{4.3.1} we have \(\abs*{x} = 0\).
        \item \(x\) is a positive rational number.
              Then by Definition \ref{4.3.1} we have \(\abs*{x} = x\), which is a positive rational number.
              By Additional Corollary \ref{ac 4.2.7} we have \(\abs*{x} = x > 0\).
        \item \(x\) is a negative rational number.
              Then by Definition \ref{4.3.1} we have \(\abs*{x} = -x\).
              By Additional Corollary \ref{ac 4.2.3} and \ref{ac 4.2.5} we know that \(-x = (-1)x\) is a positive rational number.
              By Additional Corollary \ref{ac 4.2.7} we have \(\abs*{x} = -x > 0\).
    \end{enumerate}
    From all cases above we conclude that \(\abs*{x} \geq 0\) and \(\abs*{x} = 0 \iff x = 0\).
\end{proof}

\begin{proof}{(b)}
    By Lemma \ref{4.2.7} exactly one of the following three statements is true:
    \begin{enumerate}[label=(\Roman*)]
        \item \(x = 0\).
              Then we have
              \begin{align*}
                  \abs*{0 + y} & = \abs*{y}             & \text{(by Proposition \ref{4.2.4})}    \\
                               & = 0 + \abs*{y}         & \text{(by Proposition \ref{4.2.4})}    \\
                               & = \abs*{0} + \abs*{y}. & \text{(by Proposition \ref{4.3.3}(a))}
              \end{align*}
        \item \(x\) is positive.
              By Lemma \ref{4.2.7} again exactly one of the following three statements is true:
              \begin{enumerate}[label=(\roman*)]
                  \item \(y = 0\).
                        By Proposition \ref{4.2.4} we know that \(x + y = y + x\), thus this is the same case as \(x = 0\).
                  \item \(y\) is positive.
                        Then we have
                        \begin{align*}
                                     & x + y \text{ is positive}                   & \text{(by Additional Corollary \ref{ac 4.2.4})} \\
                            \implies & \abs*{x + y} = x + y = \abs*{x} + \abs*{y}. & \text{(by Definition \ref{4.3.1})}
                        \end{align*}
                  \item \(y\) is negative.
                        Then by Proposition \ref{4.3.3}(a) we know that \(\abs*{x} > 0\) and \(\abs*{y} > 0\).
                        By Proposition \ref{4.2.9}(a) exactly one of the following three statements is true:
                        \begin{enumerate}[label=(\arabic*)]
                            \item \(x + y = 0\).
                                  Then we have
                                  \begin{align*}
                                      \abs*{x + y} & = 0                                                             \\
                                                   & < \abs*{x}             & \text{(by Lemma \ref{4.3.3}(a))}       \\
                                                   & < \abs*{x} + \abs*{y}. & \text{(by Proposition \ref{4.2.9}(c))}
                                  \end{align*}
                            \item \(x + y > 0\).
                                  Then we have
                                  \begin{align*}
                                               & y < 0                                                                                   \\
                                      \implies & x + y < x + 0                                  & \text{(by Proposition \ref{4.2.9}(d))} \\
                                      \implies & x + y < x                                      & \text{(by Proposition \ref{4.2.4})}    \\
                                      \implies & 0 < x + y < x                                  & \text{(by Proposition \ref{4.2.9}(c))} \\
                                      \implies & \abs*{x + y} = x + y < x = \abs*{x}            & \text{(by Definition \ref{4.3.1})}     \\
                                      \implies & \abs*{x + y} < \abs*{x} < \abs*{x} + \abs*{y}. & \text{(by Proposition \ref{4.2.9}(c))}
                                  \end{align*}
                            \item \(x + y < 0\).
                                  Then we have
                                  \begin{align*}
                                               & -x < -0 = 0                                    & \text{(by Exercise \ref{ex 4.2.6})}             \\
                                      \implies & (-x) + (-y) < 0 + (-y)                         & \text{(by Proposition \ref{4.2.9}(d))}          \\
                                      \implies & (-x) + (-y) < -y                               & \text{(by Proposition \ref{4.2.4})}             \\
                                      \implies & (-1)x + (-1)y < -y                             & \text{(by Additional Corollary \ref{ac 4.2.3})} \\
                                      \implies & (-1)(x + y) < -y                               & \text{(by Proposition \ref{4.2.4})}             \\
                                      \implies & -(x + y) < -y                                  & \text{(by Additional Corollary \ref{ac 4.2.3})} \\
                                      \implies & 0 = -0 < -(x + y) < -y                         & \text{(by Exercise \ref{ex 4.2.6})}             \\
                                      \implies & \abs*{x + y} = -(x + y) < -y = \abs*{y}        & \text{(by Definition \ref{4.3.1})}              \\
                                      \implies & \abs*{x + y} < \abs*{y} < \abs*{x} + \abs*{y}. & \text{(by Proposition \ref{4.2.9}(c))}
                                  \end{align*}
                        \end{enumerate}
              \end{enumerate}
        \item \(x\) is negative.
              \begin{enumerate}[label=(\roman*)]
                  \item \(y = 0\).
                        By Proposition \ref{4.2.4} we know that \(x + y = y + x\), thus this is the same case as \(x = 0\).
                  \item \(y\) is positive.
                        By Proposition \ref{4.2.4} we know that \(x + y = y + x\), thus this is the same case as \(x\) is positive and \(y\) is negative.
                  \item \(y\) is negative.
                        Then we have
                        \begin{align*}
                                     & x + y < 0                                         & \text{(by Additional Corollary \ref{ac 4.2.4})} \\
                            \implies & -(x + y) > -0 = 0                                 & \text{(by Exercise \ref{ex 4.2.6})}             \\
                            \implies & (-1)(x + y) > 0                                   & \text{(by Additional Corollary \ref{ac 4.2.3})} \\
                            \implies & (-1)x + (-1)y > 0                                 & \text{(by Proposition \ref{4.2.4})}             \\
                            \implies & (-x) + (-y) > 0                                   & \text{(by Additional Corollary \ref{ac 4.2.3})} \\
                            \implies & \abs*{x + y} = (-x) + (-y) = \abs*{x} + \abs*{y}. & \text{(by Definition \ref{4.3.1})}
                        \end{align*}
              \end{enumerate}
    \end{enumerate}
    For all cases above we conclude that \(\abs*{x + y} \leq \abs*{x} + \abs*{y}\).
\end{proof}

\begin{proof}{(c)}
    We have
    \begin{align*}
             & -y \leq x \leq y                                                   \\
        \iff & (x \leq y) \land (-x \leq y) & \text{(by Exercise \ref{ex 4.2.6})} \\
        \iff & \abs*{x} \leq y.             & \text{(by Definition \ref{4.3.1})}
    \end{align*}
    In particular, we have
    \[
        \abs*{x} \leq \abs*{x} \iff -\abs*{x} \leq x \leq \abs*{x}.
    \]
\end{proof}

\begin{proof}{(d)}
    By Lemma \ref{4.2.7} we know that exactly one of the following three statements is true:
    \begin{enumerate}[label=(\Roman*)]
        \item \(x = 0\).
              Then we have
              \begin{align*}
                  \abs*{0y} & = \abs*{0}           & \text{(by Definition \ref{4.2.2})}     \\
                            & = 0                                                           \\
                            & = 0 \abs*{y}         & \text{(by Definition \ref{4.2.2})}     \\
                            & = \abs*{0} \abs*{y}. & \text{(by Proposition \ref{4.3.3}(a))}
              \end{align*}
        \item \(x\) is positive.
              By Lemma \ref{4.2.7} again we know that exactly one of the following three statements is true:
              \begin{enumerate}[label=(\roman*)]
                  \item \(y = 0\).
                        By Proposition \ref{4.2.4} we know that \(xy = yx\), thus this is the same case as \(x = 0\).
                  \item \(y\) is positive.
                        By Additional Corollary \ref{ac 4.2.5} we know that \(xy\) is positive.
                        Thus
                        \begin{align*}
                            \abs*{xy} & = xy                 & \text{(by Definition \ref{4.3.1})}     \\
                                      & = \abs*{x} \abs*{y}. & \text{(by Proposition \ref{4.3.3}(a))}
                        \end{align*}
                  \item \(y\) is negative.
                        By Additional Corollary \ref{ac 4.2.6} we know that \(xy\) is a negative rational number.
                        Thus
                        \begin{align*}
                            \abs*{xy} & = -xy                & \text{(by Definition \ref{4.3.1})}              \\
                                      & = (-1)xy             & \text{(by Additional Corollary \ref{ac 4.2.3})} \\
                                      & = x(-1)y             & \text{(by Proposition \ref{4.2.4})}             \\
                                      & = x(-y)              & \text{(by Additional Corollary \ref{ac 4.2.3})} \\
                                      & = \abs*{x} \abs*{y}. & \text{(by Definition \ref{4.3.1})}              \\
                        \end{align*}
              \end{enumerate}
        \item \(x\) is negative.
              By Lemma \ref{4.2.7}, exactly one of the following three statements is true:
              \begin{enumerate}[label=(\roman*)]
                  \item \(y = 0\).
                        By Proposition \ref{4.2.4} we know that \(xy = yx\), thus this is the same case as \(x = 0\).
                  \item \(y\) is positive.
                        By Proposition \ref{4.2.4} we know that \(xy = yx\), thus this is the same case as \(x\) is positive and \(y\) is negative.
                  \item \(y\) is negative.
                        By Additional Corollary \ref{ac 4.2.5}, \(xy\) is a positive.
                        Thus
                        \begin{align*}
                            \abs*{xy} & = xy                 & \text{(by Definition \ref{4.3.1})}              \\
                                      & = (-1)(-1)xy         & \text{(by Definition \ref{4.2.2})}              \\
                                      & = (-1)x(-1)y         & \text{(by Proposition \ref{4.2.4})}             \\
                                      & = (-x)(-y)           & \text{(by Additional Corollary \ref{ac 4.2.3})} \\
                                      & = \abs*{x} \abs*{y}. & \text{(by Definition \ref{4.3.1})}
                        \end{align*}
              \end{enumerate}
    \end{enumerate}
    From all cases above we conclude that \(\abs*{xy} = \abs*{x} \abs*{y}\).
    In particular, we have
    \begin{align*}
        \abs*{-x} & = \abs*{(-1)x}       & \text{(by Additional Corollary \ref{ac 4.2.3})} \\
                  & = \abs*{-1} \abs*{x}                                                   \\
                  & = -(-1) \abs*{x}     & \text{(by Definition \ref{4.3.1})}              \\
                  & = 1 \abs*{x}         & \text{(by Definition \ref{4.2.2})}              \\
                  & = \abs*{x}.          & \text{(by Proposition \ref{4.2.4})}
    \end{align*}
\end{proof}

\begin{proof}{(e)}
    Since \(x - y \in \mathbf{Q}\), by Proposition \ref{4.3.3}(a) we have \(d(x, y) = \abs*{x - y} \geq 0\) and
    \begin{align*}
        d(x, y) = 0
        \iff & \abs*{x - y} = 0 & \text{(by Definition \ref{4.3.2})}     \\
        \iff & x - y = 0        & \text{(by Proposition \ref{4.3.3}(a))} \\
        \iff & x = y.           & \text{(by Proposition \ref{4.2.4})}
    \end{align*}
\end{proof}

\begin{proof}{(f)}
    We have
    \begin{align*}
        d(x, y) & = \abs*{x - y}    & \text{(by Definition \ref{4.3.2})}     \\
                & = \abs*{-(x - y)} & \text{(by Proposition \ref{4.3.3}(d))} \\
                & = \abs*{y - x}    & \text{(by Proposition \ref{4.2.4})}    \\
                & = d(y, x).        & \text{(by Definition \ref{4.3.2})}
    \end{align*}
\end{proof}

\begin{proof}{(g)}
    We have
    \begin{align*}
        d(x, z) & = \abs*{x - z}                   & \text{(by Definition \ref{4.3.2})}     \\
                & = \abs*{x - y + y - z}           & \text{(by Proposition \ref{4.2.4})}    \\
                & \leq \abs*{x - y} + \abs*{y - z} & \text{(by Proposition \ref{4.3.3}(b))} \\
                & = d(x, y) + d(y, z).             & \text{(by Definition \ref{4.3.2})}     \\
    \end{align*}
\end{proof}

\begin{additional corollary}\label{ac 4.3.1}
Let \(x, y\) be rational numbers.
Then \(\abs*{x} - \abs*{y} \leq \abs*{x + y}\).
\end{additional corollary}

\begin{proof}
    \begin{align*}
                 & \abs*{x + y + (-y)} \leq \abs*{x + y} + \abs*{-y}                 & \text{(by Proposition \ref{4.3.3}(b))} \\
        \implies & \abs*{x} \leq \abs*{x + y} + \abs*{-y}                            & \text{(by Proposition \ref{4.2.4})}    \\
        \implies & \abs*{x} \leq \abs*{x + y} + \abs*{y}                             & \text{(by Proposition \ref{4.3.3}(d))} \\
        \implies & \abs*{x} + (-\abs*{y}) \leq \abs*{x + y} + \abs*{y} + (-\abs*{y}) & \text{(by Proposition \ref{4.2.9}(d))} \\
        \implies & \abs*{x} + (-\abs*{y}) \leq \abs*{x + y}                          & \text{(by Proposition \ref{4.2.4})}    \\
        \implies & \abs*{x} - \abs*{y} \leq \abs*{x + y}.
    \end{align*}
\end{proof}

\begin{definition}[\(\varepsilon\)-closeness]\label{4.3.4}
    Let \(\varepsilon > 0\) be a rational number, and let \(x\), \(y\) be rational numbers.
    We say that \(y\) is \emph{\(\varepsilon\)-close} to \(x\) iff we have \(d(y, x) \leq \varepsilon\).
\end{definition}

\begin{remark}\label{4.3.5}
    This definition is not standard in mathematics textbooks;
    we will use it as ``scaffolding'' to construct the more important notions of limits (and of Cauchy sequences) later on, and once we have those more advanced notions we will discard the notion of \(\varepsilon\)-close.
\end{remark}

\begin{note}
    We do not bother defining a notion of \(\varepsilon\)-close when \(\varepsilon\) is zero or negative, because if \(\varepsilon\) is zero then \(x\) and \(y\) are only \(\varepsilon\)-close when they are equal, and when \(\varepsilon\) is negative then \(x\) and \(y\) are never \(\varepsilon\)-close.
\end{note}

\begin{note}
    In any event it is a long-standing tradition in analysis that the Greek letters \(\varepsilon\), \(\delta\) should only denote small positive numbers.
\end{note}

\setcounter{theorem}{6}
\begin{proposition}\label{4.3.7}
    Let \(x, y, z, w\) be rational numbers.
    (extended to cover the \(0\)-close case)
    \begin{enumerate}
        \item If \(x = y\), then \(x\) is \(\varepsilon\)-close to \(y\) for every \(\varepsilon > 0\).
              Conversely, if \(x\) is \(\varepsilon\)-close to \(y\) for every \(\varepsilon > 0\), then we have \(x = y\).
        \item Let \(\varepsilon > 0\).
              If \(x\) is \(\varepsilon\)-close to \(y\), then \(y\) is \(\varepsilon\)-close to \(x\).
        \item Let \(\varepsilon, \delta > 0\).
              If \(x\) is \(\varepsilon\)-close to \(y\), and \(y\) is \(\delta\)-close to \(z\), then \(x\) and \(z\) are \((\varepsilon + \delta)\)-close.
        \item Let \(\varepsilon, \delta > 0\).
              If \(x\) and \(y\) are \(\varepsilon\)-close, and \(z\) and \(w\) are \(\delta\)-close, then \(x + z\) and \(y + w\) are \((\varepsilon + \delta)\)-close, and \(x - z\) and \(y - w\) are also \((\varepsilon + \delta)\)-close.
        \item Let \(\varepsilon > 0\).
              If \(x\) and \(y\) are \(\varepsilon\)-close, they are also \(\varepsilon'\)-close for every \(\varepsilon' > \varepsilon\).
        \item Let \(\varepsilon > 0\).
              If \(y\) and \(z\) are both \(\varepsilon\)-close to \(x\), and \(w\) is between \(y\) and \(z\) (i.e., \(y \leq w \leq z\) or \(z \leq w \leq y\)), then \(w\) is also \(\varepsilon\)-close to \(x\).
        \item Let \(\varepsilon > 0\).
              If \(x\) and \(y\) are \(\varepsilon\)-close, and \(z\) is non-zero, then \(xz\) and \(yz\) are \(\varepsilon\abs*{z}\)-close.
        \item Let \(\varepsilon, \delta > 0\).
              If \(x\) and \(y\) are \(\varepsilon\)-close, and \(z\) and \(w\) are \(\delta\)-close, then \(xz\) and \(yw\) are \((\varepsilon\abs*{z} + \delta\abs*{x} + \varepsilon\delta)\)-close.
    \end{enumerate}
\end{proposition}

\begin{proof}{(a)}
    We first show that if \(x = y\), then \(x\) is \(\varepsilon\)-close to \(y\) for every \(\varepsilon \in \mathbf{Q}^+\).
    \begin{align*}
                 & x = y                                                                                                                              \\
        \implies & x - y = 0                                                                        & \text{(by Proposition \ref{4.2.4})}             \\
        \implies & \abs*{x - y} = 0                                                                 & \text{(by Proposition \ref{4.3.3}(a))}          \\
        \implies & \forall\ \varepsilon \in \mathbf{Q}^+, \abs*{x - y} \leq \varepsilon             & \text{(by Additional Corollary \ref{ac 4.2.7})} \\
        \implies & \forall\ \varepsilon \in \mathbf{Q}^+, x \text{ is \(\varepsilon\)-close to } y. & \text{(by Definition \ref{4.3.4})}
    \end{align*}

    Now we show that if \(x\) is \(\varepsilon\)-close to \(y\) for every \(\varepsilon \in \mathbf{Q}^+\), then \(x = y\).
    Suppose for sake of contradiction that \(x \neq y\).
    Then by Proposition \ref{4.3.3}(e) we have \(d(x, y) > 0\).
    But then we have \(d(x, y) < d(x, y)\), a contradiction.
    Thus we must have \(x = y\).
\end{proof}

\begin{proof}{(b)}
    We have
    \begin{align*}
             & x \text{ is \(\varepsilon\)-close to } y                                           \\
        \iff & d(x, y) \leq \varepsilon                  & \text{(by Definition \ref{4.3.4})}     \\
        \iff & d(y, x) \leq \varepsilon                  & \text{(by Proposition \ref{4.3.3}(f))} \\
        \iff & y \text{ is \(\varepsilon\)-close to } x. & \text{(by Definition \ref{4.3.4})}
    \end{align*}
\end{proof}

\begin{proof}{(c)}
    We have
    \begin{align*}
                 & (x \text{ is \(\varepsilon\)-close to } y) \land (y \text{ is \(\delta\)-close to } z)                                             \\
        \implies & \big(d(x, y) \leq \varepsilon\big) \land \big(d(y, z) \leq \delta\big)                 & \text{(by Definition \ref{4.3.4})}        \\
        \implies & d(x, y) + d(y, z) \leq \varepsilon + d(y, z) \leq \varepsilon + \delta                 & \text{(by Proposition \ref{4.2.9}(c)(d))} \\
        \implies & d(x, z) \leq d(x, y) + d(y, z) \leq \varepsilon + \delta                               & \text{(by Proposition \ref{4.3.3}(g))}    \\
        \implies & x \text{ is \((\varepsilon + \delta)\)-close to } z.                                   & \text{(by Definition \ref{4.3.4})}
    \end{align*}
\end{proof}

\begin{proof}{(d)}
    \begin{align*}
                 & (x \text{ is \(\varepsilon\)-close to } y) \land (z \text{ is \(\delta\)-close to } w)                                             \\
        \implies & \big(d(x, y) \leq \varepsilon\big) \land \big(d(z, w) \leq \delta\big)                 & \text{(by Definition \ref{4.3.4})}        \\
        \implies & d(x, y) + d(z, w) \leq \varepsilon + d(z, w) \leq \varepsilon + \delta                 & \text{(by Proposition \ref{4.2.9}(c)(d))} \\
        \implies & \abs*{x - y} + \abs*{z - w} \leq \varepsilon + \delta                                  & \text{(by Definition \ref{4.3.2})}        \\
        \implies & \abs*{x - y + z - w} \leq \abs*{x - y} + \abs*{z - w} \leq \varepsilon + \delta        & \text{(by Proposition \ref{4.3.3}(b))}    \\
        \implies & \abs*{x + z - (y + w)} \leq \varepsilon + \delta                                       & \text{(by Proposition \ref{4.2.4})}       \\
        \implies & (x + z) \text{ is \((\varepsilon + \delta)\)-close to } (y + w)                        & \text{(by Definition \ref{4.3.4})}        \\
        \implies & \abs*{x - y} + \abs*{-(z - w)} \leq \varepsilon + \delta                               & \text{(by Proposition \ref{4.3.3})(d)}    \\
        \implies & \abs*{x - y} + \abs*{w - z} \leq \varepsilon + \delta                                  & \text{(by Proposition \ref{4.2.4})}       \\
        \implies & \abs*{x - y + w - z} \leq \abs*{x - y} + \abs*{w - z} \leq \varepsilon + \delta        & \text{(by Proposition \ref{4.3.3}(b))}    \\
        \implies & \abs*{x - z - (y - w)} \leq \varepsilon + \delta                                       & \text{(by Proposition \ref{4.2.4})}       \\
        \implies & (x - z) \text{ is \((\varepsilon + \delta)\)-close to } (y - w).                       & \text{(by Definition \ref{4.3.4})}
    \end{align*}
\end{proof}

\begin{proof}{(e)}
    \begin{align*}
                 & (x \text{ is \(\varepsilon\)-close to } y) \land (\varepsilon' > \varepsilon)                                          \\
        \implies & \big(d(x, y) \leq \varepsilon\big) \land (\varepsilon' > \varepsilon)         & \text{(by Definition \ref{4.3.4})}     \\
        \implies & d(x, y) < \varepsilon'                                                        & \text{(by Proposition \ref{4.2.9}(c))} \\
        \implies & x \text{ is \(\varepsilon'\)-close to } y.                                    & \text{(by Definition \ref{4.3.4})}
    \end{align*}
\end{proof}

\begin{proof}{(f)}
    We have
    \begin{align*}
                 & (y \text{ is } \varepsilon\text{-close to } x) \land (z \text{ is } \varepsilon\text{-close to } x)                                          \\
        \implies & \big(d(y, x) \leq \varepsilon\big) \land \big(d(z, x) \leq \varepsilon\big)                         & \text{(by Definition \ref{4.3.4})}     \\
        \implies & (\abs*{y - x} \leq \varepsilon) \land (\abs*{z - x} \leq \varepsilon)                               & \text{(by Definition \ref{4.3.2})}     \\
        \implies & (-\varepsilon \leq y - x \leq \varepsilon) \land (-\varepsilon \leq z - x \leq \varepsilon).        & \text{(by Proposition \ref{4.3.3}(c))}
    \end{align*}
    Now we split into two cases:
    \begin{enumerate}[label=(\Roman*)]
        \item If \(y \leq w \leq z\), then we have
              \begin{align*}
                           & y \leq w \leq z                                                                                         \\
                  \implies & y - x \leq w - x \leq z - x                                    & \text{(by Proposition \ref{4.2.9}(d))} \\
                  \implies & -\varepsilon \leq y - x \leq w - x \leq z - x \leq \varepsilon & \text{(by Proposition \ref{4.2.9}(c))} \\
                  \implies & \abs*{w - x} \leq \varepsilon                                  & \text{(by Proposition \ref{4.3.3}(c))} \\
                  \implies & d(w, x) \leq \varepsilon                                       & \text{(by Definition \ref{4.3.2})}     \\
                  \implies & w \text{ is } \varepsilon\text{-close to } x.                  & \text{(by Definition \ref{4.3.4})}
              \end{align*}
        \item If \(z \leq w \leq y\), then we have
              \begin{align*}
                           & z \leq w \leq y                                                                                         \\
                  \implies & z - x \leq w - x \leq y - x                                    & \text{(by Proposition \ref{4.2.9}(d))} \\
                  \implies & -\varepsilon \leq z - x \leq w - x \leq y - x \leq \varepsilon & \text{(by Proposition \ref{4.2.9}(c))} \\
                  \implies & \abs*{w - x} \leq \varepsilon                                  & \text{(by Proposition \ref{4.3.3}(c))} \\
                  \implies & d(w, x) \leq \varepsilon                                       & \text{(by Definition \ref{4.3.2})}     \\
                  \implies & w \text{ is } \varepsilon\text{-close to } x.                  & \text{(by Definition \ref{4.3.4})}
              \end{align*}
    \end{enumerate}
    From all cases above we conclude that \(w\) is \(\varepsilon\)-close to \(x\).
\end{proof}

\begin{proof}{(g)}
    \begin{align*}
                 & (x \text{ is } \varepsilon\text{-close to } y) \land (z \neq 0)                                          \\
        \implies & \big(d(x, y) \leq \varepsilon\big) \land (z \neq 0)             & \text{(by Definition \ref{4.3.4})}     \\
        \implies & (\abs*{x - y} \leq \varepsilon) \land (z \neq 0)                & \text{(by Definition \ref{4.3.2})}     \\
        \implies & (\abs*{x - y} \leq \varepsilon) \land (\abs*{z} > 0)            & \text{(by Proposition \ref{4.3.3}(a))} \\
        \implies & \abs*{x - y} \abs*{z} \leq \varepsilon \abs*{z}                 & \text{(by Proposition \ref{4.2.9}(e))} \\
        \implies & \abs*{(x - y)z} \leq \varepsilon \abs*{z}                       & \text{(by Proposition \ref{4.3.3}(d))} \\
        \implies & \abs*{xz - yz} \leq \varepsilon \abs*{z}                        & \text{(by Proposition \ref{4.2.4})}    \\
        \implies & d(xz, yz) \leq \varepsilon \abs*{z}                             & \text{(by Definition \ref{4.3.2})}     \\
        \implies & xz \text{ is } (\varepsilon \abs*{z})\text{-close to } yz.      & \text{(by Definition \ref{4.3.4})}
    \end{align*}
\end{proof}

\begin{proof}{(h)}
    Let \(\varepsilon, \delta > 0\), and suppose that \(x\) and \(y\) are \(\varepsilon\)-close.
    If we write \(a \coloneqq y - x\), then we have \(y = x + a\) and that \(\abs*{a} \leq \varepsilon\).
    Similarly, if \(z\) and \(w\) are \(\delta\)-close, and we define \(b \coloneqq w - z\), then \(w = z + b\) and \(\abs*{b} \leq \delta\).

    Since \(y = x + a\) and \(w = z + b\), we have
    \[
        yw = (x + a)(z + b) = xz + az + xb + ab.
    \]
    Thus
    \[
        \abs*{yw - xz} = \abs*{az + bx + ab} \leq \abs*{az} + \abs*{bx} + \abs*{ab} = \abs*{a}\abs*{z} + \abs*{b}\abs*{x} + \abs*{a}\abs*{b}.
    \]
    Since \(\abs*{a} \leq \varepsilon\) and \(\abs*{b} \leq \delta\), we thus have
    \[
        \abs*{yw - xz} \leq \varepsilon\abs*{z} + \delta\abs*{x} + \varepsilon\delta
    \]
    and thus that \(yw\) and \(xz\) are \((\varepsilon\abs*{z} + \delta\abs*{x} + \varepsilon\delta)\)-close.
\end{proof}

\begin{remark}\label{4.3.8}
    One should compare statements (a)-(c) of Proposition \ref{4.3.7} with the reflexive, symmetric, and transitive axioms of equality.
    It is often useful to think of the notion of ``\(\varepsilon\)-close'' as an approximate substitute for that of equality in analysis.
\end{remark}

\begin{definition}[Exponentiation to a natural number]\label{4.3.9}
    Let \(x\) be a rational number.
    To raise \(x\) to the power \(0\), we define \(x^0 \coloneqq 1\);
    in particular we define \(0^0 \coloneqq 1\).
    Now suppose inductively that \(x^n\) has been defined for some natural number \(n\), then we define \(x^{n+1} \coloneqq x^n \times x\).
\end{definition}

\begin{proposition}[Properties of exponentiation, I]\label{4.3.10}
    Let \(x\), \(y\) be rational numbers, and let \(n\), \(m\) be natural numbers.
    \begin{enumerate}
        \item We have \(x^n x^m = x^{n + m}\), \((x^n)^m = x^{nm}\), and \((xy)^n = x^n y^n\).
        \item Suppose \(n > 0\).
              Then we have \(x^n = 0\) if and only if \(x = 0\).
        \item If \(x \geq y \geq 0\), then \(x^n \geq y^n \geq 0\).
              If \(x > y \geq 0\) and \(n > 0\), then \(x^n > y^n \geq 0\).
        \item We have \(\abs*{x^n} = \abs*{x}^n\).
    \end{enumerate}
\end{proposition}

\begin{proof}{(a)}
    We first show that \(x^n x^m = x^{n + m}\).
    We use induction on \(n\).
    For \(n = 0\), we have
    \begin{align*}
        x^0 x^m & = 1 x^m     & \text{(by Definition \ref{4.3.9})}  \\
                & = x^m       & \text{(by Proposition \ref{4.2.4})} \\
                & = x^{0 + m} & \text{(by Definition \ref{2.2.1})}
    \end{align*}
    and the base case holds.
    Suppose inductively that for some \(n \geq 0\) we have \(x^n x^m = x^{n + m}\).
    Then for \(n + 1\), we have
    \begin{align*}
        x^{n + 1} x^m & = (x^n x) x^m     & \text{(by Definition \ref{4.3.9})}  \\
                      & = x^n (x x^m)     & \text{(by Proposition \ref{4.2.4})} \\
                      & = x^n (x^m x)     & \text{(by Proposition \ref{4.2.4})} \\
                      & = (x^n x^m) x     & \text{(by Proposition \ref{4.2.4})} \\
                      & = x^{n + m} x     & \text{(by induction hypothesis)}    \\
                      & = x^{(n + m) + 1} & \text{(by Definition \ref{4.3.9})}  \\
                      & = x^{n + (m + 1)} & \text{(by Proposition \ref{2.2.5})} \\
                      & = x^{n + (1 + m)} & \text{(by Proposition \ref{2.2.4})} \\
                      & = x^{(n + 1) + m} & \text{(by Proposition \ref{2.2.5})}
    \end{align*}
    and this close the induction.

    Next we show that \((x^n)^m = x^{nm}\).
    We use induction on \(m\).
    For \(m = 0\), we have
    \begin{align*}
        (x^n)^0 & = 1      & \text{(by Definition \ref{4.3.9})}              \\
                & = x^0    & \text{(by Definition \ref{4.3.9})}              \\
                & = x^{n0} & \text{(by Additional Corollary \ref{ac 2.3.2})}
    \end{align*}
    and the base case holds.
    Suppose inductively that for some \(m \geq 0\) we have \((x^n)^m = x^{nm}\).
    Then for \(m + 1\), we have
    \begin{align*}
        (x^n)^{m + 1} & = (x^n)^m (x^n) & \text{(by Definition \ref{4.3.9})}  \\
                      & = x^{nm} x^n    & \text{(by induction hypothesis)}    \\
                      & = x^{nm + n}                                          \\
                      & = x^{n(m + 1)}  & \text{(by Proposition \ref{2.3.4})}
    \end{align*}
    and this close the induction.

    Finally we show that \((xy)^n = x^n y^n\).
    We use induction on \(n\).
    For \(n = 0\), we have
    \begin{align*}
        (xy)^0 & = 1       & \text{(by Definition \ref{4.3.9})}  \\
               & = y^0     & \text{(by Definition \ref{4.3.9})}  \\
               & = 1y^0    & \text{(by Proposition \ref{4.2.4})} \\
               & = x^0 y^0 & \text{(by Definition \ref{4.3.9})}
    \end{align*}
    and the base case holds.
    Suppose inductively that for some \(n \geq 0\) we have \((xy)^n = x^n y^n\).
    Then for \(n + 1\), we have
    \begin{align*}
        (xy)^{n + 1} & = (xy)^n (xy)         & \text{(by Definition \ref{4.3.9})}  \\
                     & = (x^n y^n) (xy)      & \text{(by induction hypothesis)}    \\
                     & = x^n (y^n x) y       & \text{(by Proposition \ref{4.2.4})} \\
                     & = x^n (x y^n) y       & \text{(by Proposition \ref{4.2.4})} \\
                     & = (x^n x)(y^n y)      & \text{(by Proposition \ref{4.2.4})} \\
                     & = x^{n + 1} y^{n + 1} & \text{(by Definition \ref{4.3.9})}
    \end{align*}
    and this close the induction.
\end{proof}

\begin{proof}{(b)}
    We use induction on \(n\) and we start with \(n = 1\).
    For \(n = 1\), we have
    \begin{align*}
             & x^1 = 0                                         \\
        \iff & x^0 x = 0 & \text{(by Definition \ref{4.3.9})}  \\
        \iff & 1x = 0    & \text{(by Definition \ref{4.3.9})}  \\
        \iff & x = 0     & \text{(by Proposition \ref{4.2.4})}
    \end{align*}
    and the base case holds.
    Suppose inductively that for some \(n \geq 1\) we have \(x^n = 0 \iff x = 0\).
    Then for \(n + 1\), we have
    \begin{align*}
             & x^{n + 1} = 0                                                                               \\
        \iff & x^n x = 0              & \text{(by Definition \ref{4.3.9})}                                 \\
        \iff & (x^n = 0) \lor (x = 0) & \text{(by Additional Corollary \ref{ac 4.2.5} and \ref{ac 4.2.6})} \\
        \iff & x = 0                  & \text{(by induction hypothesis)}
    \end{align*}
    and this close the induction.
\end{proof}

\begin{proof}{(c)}
    We first show that if \(x \geq y \geq 0\), then \(x^n \geq y^n \geq 0\).
    We use induction on \(n\).
    For \(n = 0\), we have
    \begin{align*}
                 & x \geq y \geq 0                                                  \\
        \implies & x^0 = 1 \geq y^0 = 1 \geq 0 & \text{(by Definition \ref{4.3.9})}
    \end{align*}
    and the base case holds.
    Suppose inductively that for some \(n \geq 0\) we have \(x^n \geq y^n \geq 0\).
    Then for \(n + 1\), we have
    \begin{align*}
                 & (x \geq y \geq 0) \land (x^n \geq y^n \geq 0)                  & \text{(by induction hypothesis)}       \\
        \implies & (x^n x \geq y^n x \geq 0x) \land (y^n x \geq y^n y \geq y^n 0) & \text{(by Proposition \ref{4.2.9}(e))} \\
        \implies & x^n x \geq y^n x \geq y^n y \geq y^n 0                         & \text{(by Proposition \ref{4.2.9}(c))} \\
        \implies & x^n x \geq y^n y \geq 0                                        & \text{(by Definition \ref{4.2.2})}     \\
        \implies & x^{n + 1} \geq y^{n + 1} \geq 0                                & \text{(by Definition \ref{4.3.9})}
    \end{align*}
    and this close the induction.

    Now we show that if \(x > y \geq 0\) and \(n > 0\), then \(x^n > y^n \geq 0\).
    If \(y = 0\), then by Proposition \ref{4.3.10}(b) we know that \(y^n = 0\).
    By Proposition \ref{4.2.9}(e) \(x > 0 \implies x^n\), thus we have \(x^n > 0 \geq 0\).
    So suppose that \(y > 0\).
    We use induction on \(n\) and start with \(n = 1\).
    For \(n = 1\), we have
    \begin{align*}
                 & (x > y > 0) \land (x^1 = x^0 x = 1x) \land (y^1 = y^0 y = 1y) & \text{(by Definition \ref{4.3.9})}  \\
                 & (x > y > 0) \land (x^1 = x) \land (y^1 = y)                   & \text{(by Proposition \ref{4.2.4})} \\
        \implies & x^1 > y^1 > 0                                                 & \text{(by Definition \ref{4.3.9})}
    \end{align*}
    and the base case holds.
    Suppose inductively that for some \(n \geq 1\) we have \(x^n > y^n > 0\).
    Then for \(n + 1\), we have
    \begin{align*}
                 & (x > y > 0) \land (x^n > y^n > 0)                  & \text{(by induction hypothesis)}       \\
        \implies & (x^n x > y^n x > 0x) \land (y^n x > y^n y > y^n 0) & \text{(by Proposition \ref{4.2.9}(e))} \\
        \implies & x^n x > y^n x > y^n y > y^n 0                      & \text{(by Proposition \ref{4.2.9}(c))} \\
        \implies & x^n x > y^n y > 0                                  & \text{(by Definition \ref{4.2.2})}     \\
        \implies & x^{n + 1} > y^{n + 1} > 0                          & \text{(by Definition \ref{4.3.9})}
    \end{align*}
    and this close the induction.
    Combine with the result above we have
    \[
        x > y \geq 0 \implies x^n > y^n \geq 0.
    \]
\end{proof}

\begin{proof}{(d)}
    We use induction on \(n\).
    For \(n = 0\), we have
    \begin{align*}
        \abs*{x^0} & = \abs*{1}   & \text{(by Definition \ref{4.3.9})} \\
                   & = 1          & \text{(by Definition \ref{4.3.1})} \\
                   & = \abs*{x}^0 & \text{(by Definition \ref{4.3.9})}
    \end{align*}
    and the base case holds.
    Suppose inductively that for some \(n \geq 0\) we have \(\abs*{x^n} = \abs*{x}^n\).
    Then for \(n + 1\), we have
    \begin{align*}
        \abs*{x^{n + 1}} & = \abs*{x^n x}        & \text{(by Definition \ref{4.3.9})}     \\
                         & = \abs*{x^n} \abs*{x} & \text{(by Proposition \ref{4.3.3}(d))} \\
                         & = \abs*{x}^n \abs*{x} & \text{(by induction hypothesis)}       \\
                         & = \abs*{x}^{n + 1}    & \text{(by Definition \ref{4.3.9})}
    \end{align*}
    and this close the induction.
\end{proof}

\begin{definition}[Exponentiation to a negative number]\label{4.3.11}
    Let \(x\) be a non-zero rational number.
    Then for any negative integer \(-n\), we define \(x^{-n} \coloneqq 1 / x^n\).
\end{definition}

\begin{proposition}[Properties of exponentiation, II]\label{4.3.12}
    Let \(x\), \(y\) be nonzero rational numbers, and let \(n\), \(m\) be integers.
    \begin{enumerate}
        \item We have \(x^n x^m = x^{n + m}\), \((x^n)^m = x^{nm}\), and \((xy)^n = x^n y^n\).
        \item If \(x \geq y > 0\), then \(x^n \geq y^n > 0\) if \(n\) is positive, and \(0 < x^n \leq y^n\) if \(n\) is negative.
        \item If \(x, y > 0\), \(n \neq 0\), and \(x^n = y^n\), then \(x = y\).
        \item We have \(\abs*{x^n} = \abs*{x}^n\).
    \end{enumerate}
\end{proposition}

\begin{proof}{(a)}
    We first show that \(x^n x^m = x^{n + m}\).
    By Lemma \ref{4.1.5}, exactly one of the following three statements is true:
    \begin{enumerate}[label=(\Roman*)]
        \item \(n = 0\).
              Then \(x^0 x^m = 1x^m = x^m\) by Definition \ref{4.3.9} and Proposition \ref{4.2.4}.
              And \(x^{0 + m} = x^m\) by Proposition \ref{4.2.4}.
              So \(x^0 x^m = x^{0 + m}\).
        \item \(n\) is a positive integer.
              Again by Lemma \ref{4.1.5}, exactly one of the following three statements is true:
              \begin{enumerate}[label=(\roman*)]
                  \item \(m = 0\).
                        Then \(x^n x^0 = x^0 x^n\) and \(x^{n + 0} = x^{0 + n}\) by Proposition \ref{4.2.4} and Proposition \ref{4.1.6}, which is the same case as \(n = 0\).
                  \item \(m\) is a positive integer.
                        Then by Proposition \ref{4.3.10}, \(x^n x^m = x^{n + m}\).
                  \item \(m\) is a negative integer.
                        Again by Lemma \ref{4.1.5}, exactly one of the following three statements is true:
                        \begin{enumerate}[label=(\arabic*)]
                            \item \(n + m = 0\).
                                  Then \(n = -m\) and \(x^n x^m = x^{-m} x^m = 1\) by Proposition \ref{4.2.4}.
                                  And \(x^{n + m} = x^{(-m) + m} = x^0 = 1\) by Proposition \ref{4.2.4} and Definition \ref{4.3.9}.
                                  So \(x^{-m} x^m = x^{(-m) + m}\).
                            \item \(n + m\) is a positive integer.
                                  By Proposition \ref{4.3.10}, \(x \neq 0 \implies x^{-m} \neq 0\).
                                  Then
                                  \begin{align*}
                                       & (x^n x^m) x^{-m}                                            \\
                                       & = x^n (x^m x^{-m})   & \text{(by Proposition \ref{4.2.4})}  \\
                                       & = x^n 1              & \text{(by Proposition \ref{4.2.4})}  \\
                                       & = x^n                & \text{(by Proposition \ref{4.2.4})}  \\
                                       & = x^{n + 0}          & \text{(by Proposition \ref{4.1.6})}  \\
                                       & = x^{n + (m + (-m))} & \text{(by Proposition \ref{4.1.6})}  \\
                                       & = x^{(n + m) + (-m)} & \text{(by Proposition \ref{4.1.6})}  \\
                                       & = x^{n + m} x^{-m}.  & \text{(by Proposition \ref{4.3.10})}
                                  \end{align*}
                                  So
                                  \begin{align*}
                                               & (x^n x^m) x^{-m} = x^{n + m} x^{-m}                                                   \\
                                      \implies & ((x^n x^m) x^{-m}) x^m = (x^{n + m} x^{-m}) x^m & \text{(by Lemma \ref{4.2.3})}       \\
                                      \implies & (x^n x^m)(x^{-m} x^m) = x^{n + m} (x^{-m} x^m)  & \text{(by Proposition \ref{4.2.4})} \\
                                      \implies & (x^n x^m)1 = x^{n + m} 1                        & \text{(by Proposition \ref{4.2.4})} \\
                                      \implies & x^n x^m = x^{n + m}.                            & \text{(by Proposition \ref{4.2.4})} \\
                                  \end{align*}
                            \item \(n + m\) is a negative integer.
                                  Then \(-(n + m)\) is a positive integer.
                                  By Proposition \ref{4.3.10}, \(x \neq 0 \implies x^n \neq 0\), \(x^{-m} \neq 0\) and \(x^{-(n + m)} \neq 0\).
                                  So
                                  \begin{align*}
                                       & x^n (x^{-n} x^{-m})                                              \\
                                       & = (x^n x^{-n})x^{-m}      & \text{(by Proposition \ref{4.2.4})}  \\
                                       & = 1x^{-m}                 & \text{(by Proposition \ref{4.2.4})}  \\
                                       & = x^{-m}                  & \text{(by Proposition \ref{4.2.4})}  \\
                                       & = x^{0 + (-m)}            & \text{(by Proposition \ref{4.1.6})}  \\
                                       & = x^{(n + (-n)) + (-m)}   & \text{(by Proposition \ref{4.1.6})}  \\
                                       & = x^{n + ((-n) + (-m))}   & \text{(by Proposition \ref{4.1.6})}  \\
                                       & = x^{n + ((-1)n + (-1)m)} & \text{(by Exercise \ref{ex 4.1.3})}  \\
                                       & = x^{n + (-1)(n + m)}     & \text{(by Proposition \ref{4.1.6})}  \\
                                       & = x^{n + (-(n + m))}      & \text{(by Exercise \ref{ex 4.1.3})}  \\
                                       & = x^n x^{-(n + m)}.       & \text{(by Proposition \ref{4.3.10})} \\
                                  \end{align*}
                                  And
                                  \begin{align*}
                                               & x^n (x^{-n} x^{-m}) = x^n x^{-(n + m)}                                                         \\
                                      \implies & x^{-n} (x^n (x^{-n} x^{-m})) = x^{-n} (x^n x^{-(n + m)}) & \text{(by Lemma \ref{4.2.3})}       \\
                                      \implies & (x^{-n} x^n)(x^{-n} x^{-m}) = (x^{-n} x^n)x^{-(n + m)}   & \text{(by Proposition \ref{4.2.4})} \\
                                      \implies & 1(x^{-n} x^{-m}) = 1x^{-(n + m)}                         & \text{(by Proposition \ref{4.2.4})} \\
                                      \implies & x^{-n} x^{-m} = x^{-(n + m)}                             & \text{(by Proposition \ref{4.2.4})} \\
                                      \implies & x^n (x^{-n} x^{-m}) = x^n x^{-(n + m)}                   & \text{(by Lemma \ref{4.2.3})}       \\
                                      \implies & (x^n x^{-n}) x^{-m} = x^n x^{-(n + m)}                   & \text{(by Proposition \ref{4.2.4})} \\
                                      \implies & 1x^{-m} = x^n x^{-(n + m)}                               & \text{(by Proposition \ref{4.2.4})} \\
                                      \implies & x^{-m} = x^n x^{-(n + m)}                                & \text{(by Proposition \ref{4.2.4})} \\
                                      \implies & x^m x^{-m} = x^m (x^n x^{-(n + m)})                      & \text{(by Lemma \ref{4.2.3})}       \\
                                      \implies & 1 = x^m (x^n x^{-(n + m)})                               & \text{(by Proposition \ref{4.2.4})} \\
                                      \implies & 1 = (x^m x^n) x^{-(n + m)}                               & \text{(by Proposition \ref{4.2.4})} \\
                                      \implies & 1x^{n + m} = ((x^m x^n) x^{-(n + m)}) x^{n + m}          & \text{(by Lemma \ref{4.2.3})}       \\
                                      \implies & 1x^{n + m} = (x^m x^n)(x^{-(n + m)} x^{n + m})           & \text{(by Proposition \ref{4.2.4})} \\
                                      \implies & 1x^{n + m} = (x^m x^n)1                                  & \text{(by Proposition \ref{4.2.4})} \\
                                      \implies & x^{n + m} = x^m x^n                                      & \text{(by Proposition \ref{4.2.4})} \\
                                      \implies & x^{n + m} = x^n x^m.                                     & \text{(by Proposition \ref{4.2.4})}
                                  \end{align*}
                        \end{enumerate}
              \end{enumerate}
        \item \(n\) is a negative integer.
              Again by Lemma \ref{4.1.5}, exactly one of the following three statements is true:
              \begin{enumerate}[label=(\roman*)]
                  \item \(m = 0\).
                        Then \(x^n x^0 = x^0 x^n\) and \(x^{n + 0} = x^{0 + n}\) by Proposition \ref{4.2.4} and Proposition \ref{4.1.6}, which is the same case as \(n = 0\).
                  \item \(m\) is a positive integer.
                        Then \(x^n x^m = x^m x^n\) and \(x^{n + m} = x^{m + n}\) by Proposition \ref{4.2.4} and Proposition \ref{4.1.6}, which is the same case as \(n\) is positive and \(m\) is negative.
                  \item \(m\) is a negative integer.
                        Then \(-n, -m, -(n + m)\) are positive integers.
                        By Proposition \ref{4.3.10}, \(x \neq 0 \implies x^{-n} \neq 0\), \(x^{-m} \neq 0\) and \(x^{-(n + m)} \neq 0\).
                        So
                        \begin{align*}
                                     & x^{-n} x^{-m} = x^{-(n + m)}                   & \text{(by Proposition \ref{4.3.10})} \\
                            \implies & x^n (x^{-n} x^{-m}) = x^n x^{-(n + m)}         & \text{(by Lemma \ref{4.2.3})}        \\
                            \implies & (x^n x^{-n}) x^{-m} = x^n x^{-(n + m)}         & \text{(by Proposition \ref{4.2.4})}  \\
                            \implies & 1x^{-m} = x^n x^{-(n + m)}                     & \text{(by Proposition \ref{4.2.4})}  \\
                            \implies & x^{-m} = x^n x^{-(n + m)}                      & \text{(by Proposition \ref{4.2.4})}  \\
                            \implies & x^m x^{-m} = x^m (x^n x^{-(n + m)})            & \text{(by Lemma \ref{4.2.3})}        \\
                            \implies & 1 = x^m (x^n x^{-(n + m)})                     & \text{(by Proposition \ref{4.2.4})}  \\
                            \implies & 1 = (x^m x^n) x^{-(n + m)}                     & \text{(by Proposition \ref{4.2.4})}  \\
                            \implies & 1x^{n + m} = ((x^m x^n) x^{-(n + m)})x^{n + m} & \text{(by Lemma \ref{4.2.3})}        \\
                            \implies & 1x^{n + m} = (x^m x^n)(x^{-(n + m)} x^{n + m}) & \text{(by Proposition \ref{4.2.4})}  \\
                            \implies & 1x^{n + m} = (x^m x^n)1                        & \text{(by Proposition \ref{4.2.4})}  \\
                            \implies & x^{n + m} = x^m x^n                            & \text{(by Proposition \ref{4.2.4})}  \\
                            \implies & x^{n + m} = x^n x^m.                           & \text{(by Proposition \ref{4.2.4})}
                        \end{align*}
              \end{enumerate}
    \end{enumerate}
    From all cases above, we can conclude that \(x^n x^m = x^{n + m}\).

    Next we prove that \((x^n)^m = x^{nm}\).
    By Lemma \ref{4.1.5}, exactly one of the following three statements is true:
    \begin{enumerate}[label=(\Roman*)]
        \item \(n = 0\).
              Then by Definition \ref{4.3.9}, \((x^0)^m = 1^m\) and \(x^{0m} = x^0 = 1\).
              Again By Lemma \ref{4.1.5}, exactly one of the following three statements is true:
              \begin{enumerate}[label=(\roman*)]
                  \item \(m = 0\).
                        Then \(1^0 = 1\) by Definition \ref{4.3.9}, so \((x^0)^0 = x^{0 \times 0}\).
                  \item \(m\) is a positive integer.
                        We claim that \(1^m = 1\) by using induction on \(m\).
                        For \(m = 0\), \(1^0 = 1\) by Definition \ref{4.3.9}, so the base case holds.
                        Suppose inductively that for some \(m\), \(1^m = 1\).
                        Then for \(m++\), \(1^{m++} = 1^m \times 1 = 1 \times 1 = 1\) by Definition \ref{4.3.9} and induction hypothesis, and this close the induction.
                        So \((x^0)^m = x^{0m}\).
                  \item \(m\) is a negative integer.
                        Then \(-m\) is a positive integer, and \(1^m = 1 / 1^{-m}\) by Definition \ref{4.3.11}.
                        From previous prove, we show that \(1^{-m} = 1\).
                        So \((x^0)^m = x^{0m}\).
              \end{enumerate}
        \item \(n\) is a positive integer.
              Again by Lemma \ref{4.1.5}, exactly one of the following three statements is true:
              \begin{enumerate}[label=(\roman*)]
                  \item \(m = 0\).
                        Then by Definition \ref{4.3.9}, \((x^n)^0 = 1\) and \(x^{n0} = x^0 = 1\).
                        So \((x^n)^0 = x^{n0}\).
                  \item \(m\) is a positive integer.
                        Then by Proposition \ref{4.3.10}, \((x^n)^m = x^{nm}\).
                  \item \(m\) is a negative integer.
                        Then \(-m\) is a positive integer.
                        So
                        \begin{align*}
                                     & (x^n)^{-m} = x^{n(-m)}    & \text{(by Proposition \ref{4.3.10})}            \\
                            \implies & (x^n)^{-m} = x^{n((-1)m)} & \text{(by Additional Corollary \ref{ac 4.2.3})} \\
                            \implies & (x^n)^{-m} = x^{(n(-1))m} & \text{(by Proposition \ref{4.1.6})}             \\
                            \implies & (x^n)^{-m} = x^{((-1)n)m} & \text{(by Proposition \ref{4.1.6})}             \\
                            \implies & (x^n)^{-m} = x^{(-1)(nm)} & \text{(by Proposition \ref{4.1.6})}             \\
                            \implies & (x^n)^{-m} = x^{-(nm)}    & \text{(by Additional Corollary \ref{ac 4.2.3})} \\
                            \implies & 1 / (x^n)^m = 1 / x^{nm}  & \text{(by Definition \ref{4.3.11})}             \\
                            \implies & 1x^{nm} = 1(x^n)^m        & \text{(by Definition \ref{4.2.1})}              \\
                            \implies & x^{nm} = (x^n)^m.         & \text{(by Proposition \ref{4.2.4})}
                        \end{align*}
              \end{enumerate}
        \item \(n\) is a negative integer.
              Then \(-n\) is a positive integer.
              Again by Lemma \ref{4.1.5}, exactly one of the following three statements is true:
              \begin{enumerate}[label=(\roman*)]
                  \item \(m = 0\).
                        Then by Definition \ref{4.3.9}, \((x^n)^0 = 1\) and \(x^{n0} = x^0 = 1\).
                        So \((x^n)^0 = x^{n0}\).
                  \item \(m\) is a positive integer.
                        So
                        \begin{align*}
                                     & (x^{-n})^m = x^{(-n)m}    & \text{(by Proposition \ref{4.3.10})}            \\
                            \implies & (x^{-n})^m = x^{((-1)n)m} & \text{(by Additional Corollary \ref{ac 4.2.3})} \\
                            \implies & (x^{-n})^m = x^{(-1)(nm)} & \text{(by Proposition \ref{4.1.6})}             \\
                            \implies & (x^{-n})^m = x^{-(nm)}    & \text{(by Additional Corollary \ref{ac 4.2.3})} \\
                            \implies & (1 / x^n)^m = 1 / x^{nm}. & \text{(by Definition \ref{4.3.11})}
                        \end{align*}
                        We claim that \((1 / x^n)^m = 1 / (x^n)^m\) by using induction on \(m\).
                        For \(m = 0\), \((1 / x^n)^0 = 1\) and \(1 / (x^n)^0 = 1 / 1 = 1\) by Definition \ref{4.3.9}.
                        So \((1 / x^n)^0 = 1 / (x^n)^0\), and the base case holds.
                        Suppose inductively that for some \(m\), \((1 / x^n)^m = 1 / (x^n)^m\).
                        Then for \(m++\), \((1 / x^n)^{m++} = (1 / x^n)^m \times (1 / x^n) = 1 / (x^n)^m \times (1 / x^n) = (1 \times 1) / ((x^n)^m \times x^n) = 1 / (x^n)^{m++}\) by Definition \ref{4.3.9}, induction hypothesis and Definition \ref{4.2.2}.
                        This close the induction.
                        So
                        \begin{align*}
                                     & 1 / (x^n)^m = 1 / x^{nm}                                       \\
                            \implies & 1x^{nm} = 1(x^n)^m       & \text{(by Definition \ref{4.2.1})}  \\
                            \implies & x^{nm} = (x^n)^m.        & \text{(by Proposition \ref{4.2.4})}
                        \end{align*}
                  \item \(m\) is a negative integer.
                        Then \(-m\) is a positive integer.
                        So
                        \begin{align*}
                             & (x^n)^m                                                              \\
                             & = 1 / (x^n)^{-m}   & \text{(by Definition \ref{4.3.11})}             \\
                             & = 1 / x^{n(-m)}    & \text{(from case above)}                        \\
                             & = 1 / x^{n((-1)m)} & \text{(by Additional Corollary \ref{ac 4.2.3})} \\
                             & = 1 / x^{(n(-1))m} & \text{(by Proposition \ref{4.2.4})}             \\
                             & = 1 / x^{((-1)n)m} & \text{(by Proposition \ref{4.2.4})}             \\
                             & = 1 / x^{(-1)(nm)} & \text{(by Proposition \ref{4.2.4})}             \\
                             & = 1 / x^{-(nm)}    & \text{(by Additional Corollary \ref{ac 4.2.3})} \\
                             & = x^{nm}.          & \text{(by Definition \ref{4.3.11})}
                        \end{align*}
              \end{enumerate}
    \end{enumerate}
    From all cases above, we can conclude that \((x^n)^m = x^{nm}\).

    Finally we prove that \((xy)^n = x^n y^n\).
    By Lemma \ref{4.1.5}, exactly one of the following three statements is true:
    \begin{enumerate}[label=(\roman*)]
        \item \(n = 0\).
              Then by Proposition \ref{4.3.10}, \((xy)^0 = x^0 y^0\).
        \item \(n\) is a positive integer.
              Then by Proposition \ref{4.3.10}, \((xy)^n = x^n y^n\).
        \item \(n\) is a negative integer.
              Then \(-n\) is a positive integer.
              So
              \begin{align*}
                           & (xy)^{-n} = x^{-n} y^{-n}             & \text{(by Proposition \ref{4.3.10})} \\
                  \implies & 1 / (xy)^n = (1 / x^n)(1 / y^n)       & \text{(by Definition \ref{4.3.11})}  \\
                  \implies & 1 / (xy)^n = (1 \times 1) / (x^n y^n) & \text{(by Definition \ref{4.2.2})}   \\
                  \implies & 1 / (xy)^n = 1 / (x^n y^n)                                                   \\
                  \implies & 1(x^n y^n) = 1(xy)^n                  & \text{(by Definition \ref{4.2.1})}   \\
                  \implies & (x^n y^n) = (xy)^n.                   & \text{(by Proposition \ref{4.2.4})}
              \end{align*}
    \end{enumerate}
    From all cases above, we can conclude that \((xy)^n = x^n y^n\).
\end{proof}

\begin{proof}{(b)}
    By Definition \ref{4.2.8}, \(x \geq y > 0 \implies x \geq y \geq 0\).
    If \(n\) is a positive integer, then by Proposition \ref{4.3.10}, \(x \geq y \geq 0 \implies x^n \geq y^n \geq 0\).
    By the given conditions and Proposition \ref{4.3.10}, \(y \neq 0 \implies y^n \neq 0\).
    So \(x \geq y > 0 \implies x^n \geq y^n > 0\) when \(n\) is a positive integer.

    If \(n\) is a negative integer, then let \(n = -a\), where \(a\) is a positive integer.
    By the given conditions we have two cases:
    \begin{enumerate}[label=(\roman*)]
        \item \(x > y\).
              By Definition \ref{4.2.6}, \((x > 0 \implies 1 / x > 0) \land (y > 0 \implies 1 / y > 0)\).
              By Additional Corollary \ref{ac 4.2.5} and Definition \ref{4.2.2}, \((1 / x) \times (1 / y) = 1 / xy > 0\).
              So \(x > y \implies x \times (1 / xy) > y \times (1 / xy) \implies 1 / y > 1 / x\) by Proposition \ref{4.2.9}.
              From previous prove, we get \(1 / y \geq 1 / x > 0 \implies (1 / y)^a \geq (1 / x)^a > 0\).
              But by Definition \ref{4.3.11}, Proposition \ref{4.3.12}(a) and Additional Corollary \ref{ac 4.2.3}, \((1 / y)^a = (y^{-1})^a = y^{(-1)a} = y^{-a} = y^n\) and \((1 / x)^a = (x^{-1})^a = x^{(-1)a} = x^{-a} = x^n\).
              So \(x \geq y > 0 \implies y^n \geq x^n > 0\) when \(n\) is a negative integer.
        \item \(x = y\).
              Then \(x^a = y^a > 0\) by Proposition \ref{4.3.10}.
              By Definition \ref{4.2.6}, \((x^a > 0 \implies 1 / x^a > 0) \land (y^a > 0 \implies 1 / y^a > 0)\).
              But by Definition \ref{4.3.11}, \((1 / x^a = x^{-a} = x^n) \land (1 / y^a = y^{-a} = y^n)\).
              And by Definition \ref{4.2.8}, \(x^n = y^n \implies y^n \geq x^n\).
              So \(x \geq y > 0 \implies y^n \geq x^n > 0\) when \(n\) is a negative integer.
    \end{enumerate}
    From all cases above, we conclude that \(x \geq y > 0 \implies y^n \geq x^n > 0\) when \(n\) is a negative integer.
\end{proof}

\begin{proof}{(c)}
    Suppose for sake of contradiction that \(x \neq y\)
    Then by Proposition \ref{4.2.9}, exactly one of the following two statements is true:
    \begin{enumerate}[label=(\Roman*)]
        \item \(x > y\).
              Then by Lemma \ref{4.1.5} and the given conditions, exactly one of the following two statements is true:
              \begin{enumerate}[label=(\roman*)]
                  \item \(n\) is a positive integer.
                        But by Proposition \ref{4.3.10}, \(x^n > y^n\), a contradiction.
                  \item \(n\) is a negative integer.
                        Let \(n = -a\), where \(a\) is a negative integer.
                        By Definition \ref{4.2.6}, \((x > 0 \implies 1 / x > 0) \land (y > 0 \implies 1 / y > 0)\).
                        By Additional Corollary \ref{ac 4.2.5} and Definition \ref{4.2.2}, \((1 / x) \times (1 / y) = 1 / xy > 0\).
                        So \(x > y \implies x \times (1 / xy) > y \times (1 / xy) \implies 1 / y > 1 / x\) by Proposition \ref{4.2.9}.
                        By Proposition \ref{4.3.10}, \((1 / y)^a > (1 / x)^a\).
                        But by Definition \ref{4.3.11}, Proposition \ref{4.3.12}(a) and Additional Corollary \ref{ac 4.2.3}, \((1 / y)^a = (y^{-1})^a = y^{(-1)a} = y^{-a} = y^n\) and \((1 / x)^a = (x^{-1})^a = x^{(-1)a} = x^{-a} = x^n\).
                        So \(x > y \implies y^n > x^n\), a contradiction.
              \end{enumerate}
        \item \(x < y\).
              By Proposition \ref{4.2.9}, \(x < y \implies y > x\), which just the same as \(x > y\).
    \end{enumerate}
    From all cases above we get a contradiction, so \(x = y\).
\end{proof}

\begin{proof}{(d)}
    By Lemma \ref{4.1.5}, exactly one of the following three statements is true:
    \begin{enumerate}[label=(\Roman*)]
        \item \(n = 0\).
              Then by Proposition \ref{4.3.10}, \(\abs*{x^0} = \abs*{x}^0\).
        \item \(n\) is a positive integer.
              Then by Proposition \ref{4.3.10}, \(\abs*{x^n} = \abs*{x}^n\).
        \item \(n\) is a negative integer.
              Then by Definition \ref{4.3.11}, \(\abs*{x^n} = \abs*{1 / x^{-n}}\) and \(\abs*{x}^n = 1 / \abs*{x}^{-n}\).
              By Lemma \ref{4.2.7}, exactly one of the following three statements is true:
              \begin{enumerate}[label=(\roman*)]
                  \item \(x^{-n} = 0\).
                        This case does not exist because \(x \neq 0 \implies x^{-n} \neq 0\).
                  \item \(x^{-n}\) is a positive rational number.
                        Then by Definition \ref{4.2.6} and Definition \ref{4.3.1}, \(1 / x^{-n}\) is a positive rational number and \(\abs*{1 / x^{-n}} = 1 / x^{-n} = 1 / \abs*{x^{-n}}\).
                        And by Proposition \ref{4.3.10}, \(1 / \abs*{x}^{-n} = 1 / \abs*{x^{-n}}\).
                        So \(\abs*{x^n} = \abs*{x}^n\).
                  \item \(x^{-n}\) is a negative rational number.
                        By Additional Corollary \ref{ac 4.2.3}, \(1 / x^{-n}\) is a negative rational number.
                        By Definition \ref{4.3.1}, \(\abs*{1 / x^{-n}} = -(1 / x^{-n})\).
                        Again by Additional Corollary \ref{ac 4.2.3}, \(-(1 / x^{-n}) = 1 / -(x^{-n})\).
                        Again by Definition \ref{4.3.1}, \(1 / -(x^{-n}) = 1 / \abs*{x^{-n}}\).
                        And by Proposition \ref{4.3.10}, \(1 / \abs*{x}^{-n} = 1 / \abs*{x^{-n}}\).
                        So \(\abs*{x^n} = \abs*{x}^n\).
              \end{enumerate}
    \end{enumerate}
    From all cases above, we conclude that \(\abs*{x^n} = \abs*{x}^n\).
\end{proof}

\exercisesection

\begin{exercise}\label{ex 4.3.1}
    Prove Proposition \ref{4.3.3}.
\end{exercise}

\begin{proof}
    See Proposition \ref{4.3.3}.
\end{proof}

\begin{exercise}\label{ex 4.3.2}
    Prove the remaining claims in Proposition \ref{4.3.7}.
\end{exercise}

\begin{proof}
    See Proposition \ref{4.3.7}.
\end{proof}

\begin{exercise}\label{ex 4.3.3}
    Prove Proposition \ref{4.3.10}.
\end{exercise}

\begin{proof}
    See Proposition \ref{4.3.10}.
\end{proof}

\begin{exercise}
    Prove Proposition \ref{4.3.12}.
\end{exercise}

\begin{proof}
    See Proposition \ref{4.3.12}.
\end{proof}

\begin{exercise}\label{ex 4.3.5}
    Prove that \(2^N \geq N\) for all positive integers \(N\).
\end{exercise}

\begin{proof}
    We use induction on \(N\) and begin with \(N = 1\).
    For \(N = 1\), \(2^1 = 2^0 \times 2 = 1 \times 2 = 2 \geq 1\) by Definition \ref{4.3.9}, so the base case holds.
    Suppose inductively that for some \(N\), \(2^N \geq N\).
    Then for \(N++\),
    \begin{align*}
                 & (2N = N + N) \land (N \text{ is a positive integer})                                        \\
        \implies & N < 2N                                               & \text{(by Definition \ref{2.2.11})}  \\
        \implies & N++ \leq 2N.                                         & \text{(by Proposition \ref{2.2.12})} \\
                 & 2^N \geq N                                           & \text{(by induction hypothesis)}     \\
        \implies & N \leq 2^N                                           & \text{(by Lemma \ref{4.2.3})}        \\
        \implies & 2N \leq 2 \times 2^N                                 & \text{(by Lemma \ref{4.2.3})}        \\
        \implies & 2N \leq 2^N \times 2                                 & \text{(by Proposition \ref{4.2.4})}  \\
        \implies & 2N \leq 2^{N++}                                      & \text{(by Definition \ref{4.3.9})}   \\
        \implies & N++ \leq 2^{N++}                                     & \text{(by Proposition \ref{4.2.9})}  \\
        \implies & 2^{N++} \geq N++.                                    & \text{(by Proposition \ref{4.2.9})}
    \end{align*}
    This close the induction.
\end{proof}