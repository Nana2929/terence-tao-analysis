\section{Gaps in the rational numbers}\label{sec 4.4}

\begin{proposition}[Interspersing of integers by rationals]\label{4.4.1}
    Let \(x\) be a rational number.
    Then there exists an integer \(n\) such that \(n \leq x < n + 1\).
    In fact, this integer is unique (i.e., for each \(x\) there is only one \(n\) for which \(n \leq x < n + 1\)).
    In particular, there exists a natural number \(N\) such that \(N > x\)
    (i.e., there is no such thing as a rational number which is larger than all the natural numbers).
\end{proposition}

\begin{proof}
    We first prove that \(\forall\ x \in \mathbf{Q}\), \(\exists\ n \in \mathbf{Z}\) such that \(n \leq x < n + 1\).
    By Lemma \ref{4.2.7}, exactly one of the following three statements is true:
    \begin{enumerate}
        \item \(x = 0\).
              Then by Definition \ref{4.2.8}, \(0 \leq 0\).
              By Definition \ref{2.2.11} and Definition \ref{2.2.1}, \(0 + 1 = 1 \implies 0 < 1\).
              So \(0 \leq 0 < 1\).
        \item \(x\) is a positive rational number.
              Then by Definition \ref{4.2.6}, \(x = a / b\), where \(a, b \in \mathbf{Z}^+\).
              By Proposition \ref{2.3.9}, \(a = nb + r\), where \(n, r \in \mathbf{N}\) and \(0 \leq r < b\).
              So
              \begin{align*}
                  x & = a / b                                                      \\
                    & = (nb + r) / b                                               \\
                    & = (nb + r)b^{-1}       & \text{(by Definition \ref{4.3.11})} \\
                    & = (nb)b^{-1} + rb^{-1} & \text{(by Proposition \ref{4.2.4})} \\
                    & = n(bb^{-1}) + rb^{-1} & \text{(by Proposition \ref{4.2.4})} \\
                    & = n1 + rb^{-1}         & \text{(by Proposition \ref{4.2.4})} \\
                    & = n + rb^{-1}.         & \text{(by Proposition \ref{4.2.4})}
              \end{align*}
              By Proposition \ref{4.2.9}, \(0 \leq r < b \implies 0b^{-1} \leq rb^{-1} < bb^{-1}\).
              Since \(0b^{-1} = 0\) and \(bb^{-1} = 1\) by Proposition \ref{4.2.4}, \(0 \leq rb^{-1} < 1\).
              Again by Proposition \ref{4.2.9}, \(0 \leq rb^{-1} < 1 \implies n + 0 \leq n + rb^{-1} < n + 1\).
              Again By Proposition \ref{4.2.4}, \(n + 0 = n\).
              So \(n \leq x < n + 1\).
        \item \(x\) is a negative rational number.
              Then by Definition \ref{4.2.6}, \(x = (-a) / b\), where \(a, b \in \mathbf{Z}^+\).
              By Proposition \ref{2.3.9}, \(a = mb + r\), where \(m, r \in \mathbf{N}\) and \(0 \leq r < b\).
              So
              \begin{align*}
                  x & = (-a) / b                                                                     \\
                    & = -(mb + r) / b                                                                \\
                    & = ((-1)(mb + r)) / b         & \text{(by Additional Corollary \ref{ac 4.2.3})} \\
                    & = ((-1)(mb + r))b^{-1}       & \text{(by Definition \ref{4.3.11})}             \\
                    & = (-1)((mb + r)b^{-1})       & \text{(by Proposition \ref{4.2.4})}             \\
                    & = (-1)((mb)b^{-1} + rb^{-1}) & \text{(by Proposition \ref{4.2.4})}             \\
                    & = (-1)(m(bb^{-1}) + rb^{-1}) & \text{(by Proposition \ref{4.2.4})}             \\
                    & = (-1)(m1 + rb^{-1})         & \text{(by Proposition \ref{4.2.4})}             \\
                    & = (-1)(m + rb^{-1})          & \text{(by Proposition \ref{4.2.4})}             \\
                    & = (-1)m + (-1)(rb^{-1})      & \text{(by Proposition \ref{4.2.4})}             \\
                    & = (-m) + (-(rb^{-1})).       & \text{(by Additional Corollary \ref{ac 4.2.3})} \\
              \end{align*}
              By Proposition \ref{4.2.9}, \(0 \leq r < b \implies 0b^{-1} \leq rb^{-1} < bb^{-1}\).
              Since \(0b^{-1} = 0\) and \(bb^{-1} = 1\) by Proposition \ref{4.2.4}, \(0 \leq rb^{-1} < 1\).
              By Exercise \ref{ex 4.2.6}, \(0 \leq rb^{-1} < 1 \implies (-1)1 < (-1)(rb^{-1}) \leq (-1)0\).
              By Additional Corollary \ref{ac 4.2.3}, \((-1)(rb^{-1}) = -(rb^{-1})\).
              Since \((-1)1 = (-1)\) and \((-1)0 = 0\), \(-1 < -(rb^{-1}) \leq 0\).
              Again by Proposition \ref{4.2.9}, \(-1 < -(rb^{-1}) \leq 0 \implies (-m) + (-1) < (-m) + (-(rb^{-1})) \leq (-m) + 0\).
              Again By Proposition \ref{4.2.4}, \((-m) + 0 = m\).
              So \((-m) + (-1) < x \leq -m\).
              Now we futher split into two cases:
              \begin{enumerate}[label=(\roman*)]
                  \item \(x = -m\).
                        Then by Definition \ref{4.2.8}, \(x = -m \implies -m \leq x\).
                        And by Proposition \ref{4.2.4} and Definition \ref{4.2.8}, \(((-m) + 1) - (-m) = 1 > 0 \implies -m < (-m) + 1\).
                        Let \(n = -m\).
                        So \(-m \leq x < (-m) + 1 \implies n \leq x < n + 1\).
                  \item \(x < -m\).
                        Then by Definition \ref{4.2.8}, \((-m) + (-1) < x \implies (-m) + (-1) \leq x\).
                        Let \(n = (-m) + (-1)\), then by Proposition \ref{4.1.6}, \(n + 1 = ((-m) + (-1)) + 1 = (-m) + ((-1) + 1) = (-m) + 0 = -m\).
                        So \((-m) + (-1) \leq x < -m \implies n \leq x < n + 1\).
              \end{enumerate}
    \end{enumerate}
    From all cases above, we conclude that \(\exists\ n \in \mathbf{Z}\), \(n \leq x < n + 1\).

    Now we prove that \(\forall\ x \in \mathbf{Q}\), \(\exists!\ n\) for which \(n \leq x < n + 1\).
    Suppose that \(\exists\ n_1, n_2 \in \mathbf{N}\) such that \(n_1 \leq x < n_1 + 1\) and \(n_2 \leq x < n_2 + 1\).
    We want to show that \(n_1 = n_2\).
    Then we get \(n_1 < n_2 + 1\) and \(n_2 < n_1 + 1\).
    So
    \begin{align*}
                 & (n_1 < n_2 + 1) \land (n_2 < n_1 + 1)                                                      \\
        \implies & (n_1 + 1 \leq n_2 + 1) \land (n_2 + 1 \leq n_1 + 1) & \text{(by Proposition \ref{2.2.12})} \\
        \implies & (n_1 \leq n_2) \land (n_2 \leq n_1)                 & \text{(by Proposition \ref{2.2.12})} \\
        \implies & n_1 = n_2.                                          & \text{(by Proposition \ref{2.2.12})}
    \end{align*}

    Finally, we prove that \(\forall\ x \in \mathbf{Q}\), \(\exists\ N \in \mathbf{N}\) such that \(N > x\).
    Because \(\forall\ x \in \mathbf{Q}\), \(\exists!\ n \in \mathbf{N}\), \(n \leq x < n + 1\).
    By setting \(N = n + 1\), we get \(x < N\).
    By Proposition \ref{4.2.9}, \(x < N \implies N > x\).
\end{proof}

\begin{remark}\label{4.4.2}
    The integer \(n\) for which \(n \leq x < n + 1\) is sometimes referred to as the \emph{integer part} of \(x\) and is sometimes denoted \(n = \floor*{x}\).
\end{remark}

\begin{proposition}[Interspersing of rationals by rationals]\label{4.4.3}
    If \(x\) and \(y\) are two rationals such that \(x < y\), then there exists a third rational \(z\) such that \(x < z < y\).
\end{proposition}

\begin{proof}
    We set \(z \coloneqq (x + y) / 2\).
    Since \(x < y\), and \(1 / 2 = 1 // 2\) is positive, we have from Proposition \ref{4.2.9} that \(x / 2 < y / 2\).
    If we add \(y / 2\) to both sides using Proposition \ref{4.2.9} we obtain \(x / 2 + y / 2 < y / 2 + y / 2\), i.e., \(z < y\).
    If we instead add \(x / 2\) to both sides we obtain \(x / 2 + x / 2 < y / 2 + x / 2\), i.e., \(x < z\).
    Thus \(x < z < y\) as desired.
\end{proof}

\begin{note}
    Despite the rationals having this denseness property, they are still incomplete;
    there are still an infinite number of ``gaps'' or ``holes'' between the rationals, although this denseness property does ensure that these holes are in some sense infinitely small.
\end{note}

\begin{additional corollary}\label{ac 4.4.1}
Let \(n, m\) be two natural numbers.
Define \(n\) to be even if \(n = 2m\), and odd if \(n = 2m + 1\).
Then every natural number is either even or odd, but not both.
\end{additional corollary}

\begin{proof}
    We use induction on \(n\).
    For \(n = 0\), \(0 = 0 \times 2 = 2 \times 0\) by Definition \ref{2.3.1} and Lemma \ref{2.3.2}.
    And by Axiom \ref{2.3}, \(0 \neq 2m + 1\).
    So the base case holds.
    Suppose inductively that for some \(n\), \(\exists\ m \in \mathbf{N}\) such that either \(n = 2m\) or \(n = 2m + 1\) is true, but not both.
    Then for \(n++\), by induction hypothesis, we can split into two cases.
    If \(n = 2m\), then \(n++ = (2m)++ = 2m + 1\), which means \(n++\) is odd.
    If \(n = 2m + 1\), then \(n++ = (2m + 1)++ = (2m + 1) + 1 = 2m + (1 + 1) = 2m + 2 = 2(m + 1)\) by Proposition \ref{2.2.5} and Proposition \ref{2.3.4}, which means \(n++\) is even.
    Because by induction hypothesis, either even or odd, but not both.
    So \(n++\) is also either even or odd, but not both, and this close the induction.
\end{proof}

\begin{additional corollary}\label{ac 4.4.2}
Let \(n\) be a natural number.
If \(n\) is even, then \(n^2\) is also even.
If \(n\) is odd, then \(n^2\) is also odd.
\end{additional corollary}

\begin{proof}
    We first prove that \(n\) is even implies \(n^2\) is even.
    By Additional Corollary \ref{ac 4.4.1}, \(\exists\ m \in \mathbf{N}\) such that \(n = 2m\).
    So
    \begin{align*}
        n^2 & = (2m)^2                                         \\
            & = (2m)(2m) & \text{(by Definition \ref{2.3.11})} \\
            & = 2(m(2m)) & \text{(by Proposition \ref{2.3.5})} \\
            & = 2((2m)m) & \text{(by Lemma \ref{2.3.2})}       \\
            & = 2(2(mm)) & \text{(by Proposition \ref{2.3.5})} \\
            & = 2(2m^2). & \text{(by Definition \ref{2.3.11})}
    \end{align*}
    By Additional Corollary \ref{ac 4.4.1}, \(n^2\) is also even.

    Now we prove that \(n\) is odd implies \(n^2\) is odd.
    By Additional Corollary \ref{ac 4.4.1}, \(\exists\ m \in \mathbf{N}\) such that \(n = 2m + 1\).
    So
    \begin{align*}
        n^2 & = (2m + 1)^2                                                                  \\
            & = (2m + 1)(2m + 1)                      & \text{(by Definition \ref{2.3.11})} \\
            & = (2m)(2m) + (2m)1 + 1(2m) + 1 \times 1 & \text{(by Proposition \ref{2.3.4})} \\
            & = 2(2m^2) + (2m)1 + 1(2m) + 1 \times 1  & \text{(by previous prove)}          \\
            & = 2(2m^2) + (2m) + (2m) + 1                                                   \\
            & = 2(2m^2) + 2(m + m) + 1                & \text{(by Proposition \ref{2.3.4})} \\
            & = 2(2m^2) + 2(2m) + 1                                                         \\
            & = 2((2m^2) + (2m)) + 1.                 & \text{(by Proposition \ref{2.3.4})} \\
    \end{align*}
    By Additional Corollary \ref{ac 4.4.1}, \(n^2\) is also odd.
\end{proof}

\begin{proposition}\label{4.4.4}
    There does not exist any rational number \(x\) for which \(x^2 = 2\).
\end{proposition}

\begin{proof}
    Suppose for sake of contradiction that we had a rational number \(x\) for which \(x^2 = 2\).
    Clearly \(x\) is not zero.
    We may assume that \(x\) is positive, for if \(x\) were negative then we could just replace \(x\) by \(-x\)
    (since \(x^2 = (-x)^2\)).
    Thus \(x = p / q\) for some positive integers \(p, q\), so \((p / q)^2 = 2\), which we can rearrange as \(p^2 = 2q^2\).
    Define a natural number \(p\) to be even if \(p = 2k\) for some natural number \(k\), and odd if \(p = 2k + 1\) for some natural number \(k\).
    By Additional Corollary \ref{ac 4.4.1}, every natural number is either even or odd, but not both.
    By Additional Corollary \ref{ac 4.4.2}, if \(p\) is odd, then \(p^2\) is also odd, which contradicts \(p^2 = 2q^2\).
    Thus \(p\) is even, i.e., \(p = 2k\) for some natural number \(k\).
    Since \(p\) is positive, \(k\) must also be positive.
    Inserting \(p = 2k\) into \(p^2 = 2q^2\) we obtain \(4k^2 = 2q^2\), so that \(q^2 = 2k^2\).

    To summarize, we started with a pair \((p, q)\) of positive integers such that \(p^2 = 2q^2\), and ended up with a pair \((q, k)\) of positive integers such that \(q^2 = 2k^2\).
    Since \(p^2 = 2q^2\), so \(p^2 = q^2 + q^2 \implies p^2 > q^2\) by Proposition \ref{2.2.12}.
    If \(p < q\), then by Proposition \ref{2.3.6}, \(p^2 < pq\) and \(pq < q^2\).
    So by Proposition \ref{2.2.12}, \(p^2 < q^2\), a contradiction.
    Thus we have \(q < p\).
    If we rewrite \(p' \coloneqq q\) and \(q' \coloneqq k\), we thus can pass from one solution \((p, q)\) to the equation \(p^2 = 2q^2\) to a new solution \((p', q')\) to the same equation which has a smaller value of \(p\).
    But then we can repeat this procedure again and again, obtaining a sequence \((p'', q'')\), \((p''', q''')\), etc. of solutions to \(p^2 = 2q^2\), each one with a smaller value of \(p\) than the previous, and each one consisting of positive integers.
    But this contradicts the principle of infinite descent (see Exercise \ref{ex 4.4.2}).
    This contradiction shows that we could not have had a rational \(x\) for which \(x^2 = 2\).
\end{proof}

\begin{proposition}\label{4.4.5}
    For every rational number \(\varepsilon > 0\), there exists a non-negative rational number \(x\) such that \(x^2 < 2 < (x + \varepsilon)^2\).
\end{proposition}

\begin{proof}
    Let \(\varepsilon > 0\) be rational.
    Suppose for sake of contradiction that there is no non-negative rational number \(x\) for which \(x^2 < 2 < (x + \varepsilon)^2\).
    This means that whenever \(x\) is non-negative and \(x^2 < 2\), we must also have \((x + \varepsilon)^2 < 2\)
    (note that \((x + \varepsilon)^2\) cannot equal \(2\), by Proposition \ref{4.4.4}).
    Since \(0^2 < 2\), we thus have \(\varepsilon^2 < 2\), which then implies \((2\varepsilon)^2 < 2\), and indeed a simple induction shows that \((n\varepsilon)^2 < 2\) for every natural number \(n\).
    (Note that \(n\varepsilon\) is non-negative for every natural number \(n\) by Additional Corollary \ref{ac 4.2.5})
    But, by Proposition \ref{4.4.1} we can find an integer \(n\) such that \(n > 2 / \varepsilon\), which implies that \(n\varepsilon > 2\), which implies that \((n\varepsilon)^2 > 4 > 2\), contradicting the claim that \((n\varepsilon)^2 < 2\) for all natural numbers \(n\).
    This contradiction gives the proof.
\end{proof}

\begin{note}
    Proposition \ref{4.4.5} indicates that, while the set \(\mathbf{Q}\) of rationals does not actually have \(\sqrt{2}\) as a member, we can get as close as we wish to \(\sqrt{2}\).
    For instance, the sequence of rationals
    \[
        1.4, 1.41, 1.414, 1.4142, 1.41421, \dots
    \]
    seem to get closer and closer to \(\sqrt{2}\), as their squares indicate:
    \[
        1.96, 1.9881, 1.99396, 1.99996164, 1.9999899241, \dots
    \]
    Thus it seems that we can create a square root of \(2\) by taking a ``limit'' of a sequence of rationals.
    This is how we shall construct the real numbers in the next chapter.
\end{note}

\begin{note}
    There is another way to construct the real numbers, using something called ``Dedekind cuts'', which we will not pursue here.
    One can also proceed using infinite decimal expansions, but there are some sticky issues when doing so, e.g., one has to make \(0.999\dots\) equal to \(1.000\dots\), and this approach, despite being the most familiar, is actually more complicated than other approaches.
\end{note}

\exercisesection

\begin{exercise}\label{ex 4.4.1}
    Prove Proposition \ref{4.4.1}.
\end{exercise}

\begin{proof}
    See Proposition \ref{4.4.1}.
\end{proof}

\begin{exercise}\label{ex 4.4.2}
    A definition: a sequence \(a_0, a_1, a_2, \dots\) of numbers (natural numbers, integers, rationals, or reals) is said to be in \emph{infinite descent} if we have \(a_n > a_{n + 1}\) for all natural numbers \(n\)
    (i.e., \(a_0 > a_1 > a_2 > \dots\)).
    \begin{enumerate}
        \item Prove the \emph{principle of infinite descent}:
              that it is not possible to have a sequence of \emph{natural numbers} which is in infinite descent.
        \item Does the principle of infinite descent work if the sequence \(a_1, a_2, a_3, \dots\) is allowed to take integer values instead of natural number values?
              What about if it is allowed to take positive rational values instead of natural numbers?
              Explain.
    \end{enumerate}
\end{exercise}

\begin{proof}{(a)}
    Suppose for sake of contradiction that there exists a sequence of natural numbers \(a_0, a_1, a_2, \dots\) is in infinite descent.
    Since all the \(a_n\) are natural numbers, \(a_n \geq 0\) for all \(n \in \mathbf{N}\).
    Now we use induction on \(k\) to show in fact that \(a_n \geq k\) for all \(k \in \mathbf{N}\) and all \(n \in \mathbf{N}\).
    For \(k = 0\), because \(a_n \geq 0\) for all \(n \in \mathbf{N}\), so the base case holds.
    Suppose inductively that for some \(k\), \(a_n \geq k\) for all \(n \in \mathbf{N}\).
    Then for \(k++\), we want to show that \(a_n \geq k++\) for all \(n \in \mathbf{N}\).
    By induction hypothesis, \(\forall\ m \in \mathbf{N}\) such that \(a_m \geq k\), which also implies \(a_{m++} \geq k\).
    And because the sequence is in infinite descent, \(a_m > a_{m++}\).
    So \(a_m > a_{m++} \geq k \implies a_m > k\) by Proposition \ref{2.2.12}.
    Again by Proposition \ref{2.2.12}, \(a_m > k \implies a_m \geq k++\).
    This close the induction.

    Now we show that such sequence does not exist.
    Because \(\forall\ k \in \mathbf{N}\), \(a_n \geq k \ \forall\ n \in \mathbf{N}\).
    We set \(k = a_0\).
    Then \(a_n \geq a_0\), which contradicts to the sequence which is in infinite descent.
    So such sequence does not exist.
\end{proof}

\begin{proof}{(b)}
    By setting \(a_n = -n \ \forall\ n \in \mathbf{N}\), we can always have \(a_n > a_{n + 1}\).
    So the principle of infinite descent does not work on integers.

    Similarly, by setting \(a_n = 1 / n \ \forall\ n \in \mathbf{N}\), we can always have \(a_n > a_{n + 1}\).
    So the principle of infinite descent does not work on rationals.
\end{proof}

\begin{exercise}\label{ex 4.4.3}
    Fill in the gaps marked (why?) in the proof of Proposition \ref{4.4.4}.
\end{exercise}

\begin{proof}
    See Proposition \ref{4.4.4}.
\end{proof}