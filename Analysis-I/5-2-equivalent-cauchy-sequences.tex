\section{Equivalent Cauchy sequences}

\begin{definition}[\(\varepsilon\)-close sequences]\label{5.2.1}
Let \((a_n)_{n = 0}^{\infty}\) and \((b_n)_{n = 0}^{\infty}\) be two sequences, and let \(\varepsilon > 0\).
We say that the sequence \((a_n)_{n = 0}^{\infty}\) is \emph{\(\varepsilon\)-close} to \((b_n)_{n = 0}^{\infty}\) iff \(a_n\) is \(\varepsilon\)-close to \(b_n\) for each \(n \in \mathds{N}\).
In other words, the sequence \(a_0, a_1, a_2, \dots\) is \(\varepsilon\)-close to the sequence \(b_0, b_1, b_2, \dots\) iff \(\abs*{a_n - b_n} \leq \varepsilon\) for all \(n = 0, 1, 2, \dots\).
\end{definition}

\setcounter{theorem}{2}
\begin{definition}[\(Eventually \varepsilon\)-close sequences]\label{5.2.3}
Let \((a_n)_{n = 0}^{\infty}\) and \((b_n)_{n = 0}^{\infty}\) be two sequences, and let \(\varepsilon > 0\).
We say that the sequence \((a_n)_{n = 0}^{\infty}\) is \emph{eventually \(\varepsilon\)-close} to \((b_n)_{n = 0}^{\infty}\) iff there exists an \(N \geq 0\) such that the sequences \((a_n)_{n = N}^{\infty}\) and \((b_n)_{n = N}^{\infty}\) are \(\varepsilon\)-close.
In other words, \(a_0, a_1, a_2, \dots\) is eventually \(\varepsilon\)-close to \(b_0, b_1, b_2, \dots\) iff there exists an \(N \geq 0\) such that \(\abs*{a_n - b_n} \leq \varepsilon\) for all \(n \geq N\).
\end{definition}

\begin{remark}\label{5.2.4}
Again, the notations for \(\varepsilon\)-close sequences and eventually \(\varepsilon\)-close sequences are not standard in the literature, and we will not use them outside of this section.
\end{remark}

\setcounter{theorem}{5}
\begin{definition}[Equivalent sequences]\label{5.2.6}
Two sequences \((a_n)_{n = 0}^{\infty}\) and \((b_n)_{n = 0}^{\infty}\) are \emph{equivalent} iff for each rational \(\varepsilon > 0\), the sequences \((a_n)_{n = 0}^{\infty}\) and \((b_n)_{n = 0}^{\infty}\) are eventually \(\varepsilon\)-close.
In other words, \(a_0, a_1, a_2, \dots\) and \(b_0, b_1, b_2, \dots\) are equivalent iff for every rational \(\varepsilon > 0\), there exists an \(N \geq 0\) such that \(\abs*{a_n - b_n} \leq \varepsilon\) for all \(n \geq N\).
\end{definition}

\begin{remark}\label{5.2.7}
As with Definition \ref{5.1.8}, the quantity \(\varepsilon > 0\) is currently restricted to be a positive rational, rather than a positive real.
However, we shall eventually see that it makes no difference whether \(\varepsilon\) ranges over the positive rationals or positive reals.
\end{remark}

\begin{proposition}\label{5.2.8}
Let \((a_n)_{n = 1}^{\infty}\) and \((b_n)_{n = 1}^{\infty}\) be the sequences \(a_n = 1 + 10^{-n}\) and \(b_n = 1 - 10^{-n}\).
Then the sequences \(a_n, b_n\) are equivalent.
\end{proposition}

\begin{proof}
We need to prove that for every \(\varepsilon > 0\), the two sequences \((a_n)_{n = 1}^{\infty}\) and \((b_n)_{n = 1}^{\infty}\) are eventually \(\varepsilon\)-close to each other.
So we fix an \(\varepsilon > 0\).
We need to find an \(N > 0\) such that \((a_n)_{n = 1}^{\infty}\) and \((b_n)_{n = 1}^{\infty}\) are \(\varepsilon\)-close;
in other words, we need to find an \(N > 0\) such that
\[
    \abs*{a_n - b_n} \leq \varepsilon \text{ for all } n \geq N.
\]
However, we have
\[
    \abs*{a_n - b_n} = \abs*{(1 + 10^{-n}) - (1 - 10^{-n})} = 2 \times 10^{-n}.
\]
Since \(10^{-n}\) is a decreasing function of \(n\) (i.e., \(10^{-m} < 10^{-n}\) whenever \(m > n\);
this is easily proven by induction), and \(n \geq N\), we have \(2 \times 10^{-n} \leq 2 \times 10^{-N}\).
Thus we have
\[
    \abs*{a_n - b_n} \leq 2 \times 10^{-N} \text{ for all } n \geq N.
\]
Thus in order to obtain \(\abs*{a_n - b_n} \leq \varepsilon\) for all \(n \geq N\), it will be sufficient to choose \(N\) so that \(2 \times 10^{-N} \leq \varepsilon\).
This is easy to do using logarithms, but we have not yet developed logarithms yet, so we will use a cruder method.
First, we observe \(10^N\) is always greater than \(N\) for any \(N \geq 1\) (see Exercise \ref{ex 4.3.5}).
Thus \(10^{-N} \leq 1 / N\), and so \(2 \times 10^{-N} \leq 2 / N\).
Thus to get \(2 \times 10^{-N} \leq \varepsilon\), it will suffice to choose \(N\) so that \(2 / N \leq \varepsilon\), or equivalently that \(N \geq 2 / \varepsilon\).
But by Proposition \ref{4.4.1} we can always choose such an \(N\), and the claim follows.
\end{proof}

\begin{remark}\label{5.2.9}
Proposition \ref{5.2.8}, in decimal notation, asserts that
\[
    1.0000 \dots = 0.9999 \dots.
\]
\end{remark}

\exercisesection

\begin{exercise}\label{ex 5.2.1}
Show that if \((a_n)_{n = 1}^{\infty}\) and \((b_n)_{n = 1}^{\infty}\) are equivalent sequences of rationals, then \((a_n)_{n = 1}^{\infty}\) is a Cauchy sequence if and only if \((b_n)_{n = 1}^{\infty}\) is a Cauchy sequence.
\end{exercise}

\begin{proof}
Let \((a_n)_{n = 1}^{\infty}\) be a Cauchy sequence.
By Definition \ref{5.1.8}, \(\forall\ \varepsilon > 0\) and \(\varepsilon \in \mathds{Q}\), \(\exists\ N_1 \geq 1\) and \(N_1 \in \mathds{N}\) such that
\[
    \abs*{a_j - a_k} \leq \varepsilon \ \forall\ j, k \geq N_1
\]
where \(j, k \in \mathds{N}\).
Since \((a_n)_{n = 1}^{\infty}\) and \((b_n)_{n = 1}^{\infty}\) are equivalent sequences, by Definition \ref{5.2.6}, \(\forall\ \varepsilon > 0\) and \(\varepsilon \in \mathds{Q}\), \(\exists\ N_2 \geq 1\) and \(N_2 \in \mathds{N}\) such that
\[
    \abs*{a_m - b_m} \leq \varepsilon \ \forall\ m \geq N_2
\]
where \(m \in \mathds{N}\).
Let \(N = N_1 + N_2\).
Since \(N > N_1\) and \(N > N_2\) by Proposition \ref{2.2.11}, we have
\[
    \abs*{a_j - a_k} \leq \varepsilon \ \forall\ j, k \geq N
\]
and
\[
    \abs*{a_m - b_m} \leq \varepsilon \ \forall\ m \geq N.
\]
Since \(\varepsilon > 0\), \(\varepsilon / 3 > 0\) by Additional Corollary \ref{ac 4.2.5}, then we have
\[
    \abs*{a_j - a_k} \leq \varepsilon / 3 \ \forall\ j, k \geq N
\]
and
\[
    \abs*{a_m - b_m} \leq \varepsilon / 3 \ \forall\ m \geq N.
\]
So \(\forall\ \varepsilon > 0\) and \(\forall\ j, k \geq N\),
\begin{align*}
\abs*{b_j - b_k} &= \abs*{b_j + (-b_k)} \\
&= \abs*{(b_j + (-b_k)) + 0} & \text{(by Proposition \ref{4.2.4})} \\
&= \abs*{(b_j + (-b_k)) + ((-a_k) + a_k)} & \text{(by Proposition \ref{4.2.4})} \\
&= \abs*{b_j + ((-b_k) + ((-a_k) + a_k))} & \text{(by Proposition \ref{4.2.4})} \\
&= \abs*{b_j + (((-a_k) + a_k) + (-b_k))} & \text{(by Proposition \ref{4.2.4})} \\
&= \abs*{b_j + ((-a_k) + (a_k + (-b_k)))} & \text{(by Proposition \ref{4.2.4})} \\
&= \abs*{(b_j + (-a_k)) + (a_k + (-b_k))} & \text{(by Proposition \ref{4.2.4})} \\
&= \abs*{0 + ((b_j + (-a_k)) + (a_k + (-b_k)))} & \text{(by Proposition \ref{4.2.4})} \\
&= \abs*{(a_j + (-a_j)) + ((b_j + (-a_k)) + (a_k + (-b_k)))} & \text{(by Proposition \ref{4.2.4})} \\
&= \abs*{((a_j + (-a_j)) + (b_j + (-a_k))) + (a_k + (-b_k))} & \text{(by Proposition \ref{4.2.4})} \\
&= \abs*{(a_j + (((-a_j) + b_j) + (-a_k))) + (a_k + (-b_k))} & \text{(by Proposition \ref{4.2.4})} \\
&= \abs*{(a_j + ((-a_k) + ((-a_j) + b_j))) + (a_k + (-b_k))} & \text{(by Proposition \ref{4.2.4})} \\
&= \abs*{((a_j + (-a_k)) + ((-a_j) + b_j)) + (a_k + (-b_k))} & \text{(by Proposition \ref{4.2.4})} \\
&\leq \abs*{(a_j + (-a_k)) + ((-a_j) + b_j)} + \abs*{a_k + (-b_k)} & \text{(by Proposition \ref{4.3.3})} \\
&\leq (\abs*{a_j + (-a_k)} + \abs*{(-a_j) + b_j}) + \abs*{a_k + (-b_k)} & \text{(by Proposition \ref{4.3.3})} \\
&= (\abs*{a_j + (-a_k)} + \abs*{a_j + (-b_j)}) + \abs*{a_k + (-b_k)} & \text{(by Proposition \ref{4.3.3})} \\
&= (\abs*{a_j - a_k} + \abs*{a_j - b_j}) + \abs*{a_k - b_k} \\
&\leq (\varepsilon / 3 + \varepsilon / 3) + \varepsilon / 3 & \text{(by Proposition \ref{4.3.3})} \\
&= \varepsilon.
\end{align*}
By Definition \ref{5.1.8}, \((b_n)_{n = 1}^{\infty}\) is also a Cauchy sequence.
Similar proof can show that by the given conditions, \((b_n)_{n = 1}^{\infty}\) is a Cauchy sequence also implies \((a_n)_{n = 1}^{\infty}\) is a Cauchy sequence.
Thus we finished the proof.
\end{proof}

\begin{exercise}\label{ex 5.2.2}
Let \(\varepsilon > 0\).
Show that if \((a_n)_{n = 1}^{\infty}\) and \((b_n)_{n = 1}^{\infty}\) are eventually \(\varepsilon\)-close, then \((a_n)_{n = 1}^{\infty}\) is bounded if and only if \((b_n)_{n = 1}^{\infty}\) is bounded.
\end{exercise}

\begin{proof}
By Definition \ref{5.2.3}, \(\exists\ N \geq 1\) and \(N \in \mathds{N}\),
\[
    \abs*{a_n - b_n} \leq \varepsilon \ \forall\ n \geq N.
\]
If \((a_n)_{n = 1}^{\infty}\) is bounded, then by Definition \ref{5.1.12}, \(\exists\ M \geq 0\) and \(M \in \mathds{Q}\),
\[
    \abs*{a_n} \leq M \ \forall\ n \geq 1.
\]
So \(\forall\ n \geq 1\),
\begin{align*}
\abs*{b_n} &= \abs*{b_n + 0} & \text{(by Proposition \ref{4.2.4})} \\
&= \abs*{b_n + ((-a_n) + a_n)} & \text{(by Proposition \ref{4.2.4})} \\
&= \abs*{(b_n + (-a_n)) + a_n} & \text{(by Proposition \ref{4.2.4})} \\
&\leq \abs*{b_n + (-a_n)} + \abs*{a_n} & \text{(by Proposition \ref{4.3.3})} \\
&= \abs*{(-a_n) + b_n} + \abs*{a_n} & \text{(by Proposition \ref{4.2.4})} \\
&= \abs*{a_n + (-b_n)} + \abs*{a_n} & \text{(by Proposition \ref{4.3.3})} \\
&= \abs*{a_n - b_n} + \abs*{a_n} \\
&\leq \varepsilon + \abs*{a_n} & \text{(by Proposition \ref{4.2.9})} \\
&\leq \varepsilon + M & \text{(by Proposition \ref{4.2.9})} \\
\end{align*}
Because \(\varepsilon > 0\) and \(M > 0\), so \(\varepsilon + M > 0\) by Additional Corollary \ref{ac 4.2.4}.
So by Definition \ref{5.1.12}, \((b_n)_{n = 1}^{\infty}\) is bounded by \(\varepsilon + M\).
Similar proof can show that by the given conditions, \((b_n)_{n = 1}^{\infty}\) is bounded also implies \((a_n)_{n = 1}^{\infty}\) is bounded.
Thus we finished the proof.
\end{proof}