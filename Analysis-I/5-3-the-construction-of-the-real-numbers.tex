\section{The construction of the real numbers}

\begin{definition}[Real numbers]\label{5.3.1}
A \emph{real number} is defined to be an object of the form \(\text{LIM}_{n \to \infty} a_n\), where \((a_n)_{n = 1}^{\infty}\) is a Cauchy sequence of rational numbers.
Two real numbers \(\text{LIM}_{n \to \infty} a_n\) an and \(\text{LIM}_{n \to \infty} b_n\) are said to be equal iff \((a_n)_{n = 1}^{\infty}\) and \((b_n)_{n = 1}^{\infty}\) are equivalent Cauchy sequences.
The set of all real numbers is denoted \(\mathds{R}\).
\end{definition}

\begin{note}
We will refer to \(\text{LIM}_{n \to \infty} a_n\) as the \emph{formal limit} of the sequence \((a_n)_{n = 1}^{\infty}\).
Later on we will define a genuine notion of limit, and show that the formal limit of a Cauchy sequence is the same as the limit of that sequence;
after that, we will not need formal limits ever again.
\end{note}

\setcounter{theorem}{2}
\begin{proposition}[Formal limits are well-defined]\label{5.3.3}
Let \(x = \text{LIM}_{n \to \infty} a_n\), \(y = \text{LIM}_{n \to \infty} b_n\), and \(z = \text{LIM}_{n \to \infty} c_n\) be real numbers.
Then, with the above definition of equality for real numbers, we have \(x = x\).
Also, if \(x = y\), then \(y = x\).
Finally, if \(x = y\) and \(y = z\), then \(x = z\).
\end{proposition}

\begin{proof}
We first prove the reflexivity.
Because \(x = \text{LIM}_{n \to \infty} a_n\), by Definition \ref{5.3.1}, \((a_n)_{n = 1}^{\infty}\) is a Cauchy sequence of rational numbers.
By Definition \ref{5.1.8}, \(\forall\ \varepsilon > 0\) and \(\varepsilon \in \mathds{Q}\), \(\exists\ N \geq 1\) and \(N \in \mathds{N}\) such that
\[
    |a_j - a_k| \leq \varepsilon \ \forall\ j, k \geq N,
\]
where \(j, k \in \mathds{N}\).
In particular, we have
\[
    |a_j - a_j| \leq \varepsilon \ \forall\ j \geq N.
\]
By Definition \ref{5.2.6}, \((a_n)_{n = 1}^{\infty}\) and \((a_n)_{n = 1}^{\infty}\) are equivalent sequences.
Since \((a_n)_{n = 1}^{\infty}\) is a Cauchy sequence, by Definition \ref{5.3.1}, \(x = \text{LIM}_{n \to \infty} a_n = \text{LIM}_{n \to \infty} a_n = x\).

Next we prove the symmetry.
By Definition \ref{5.3.1}, \(x = y\) implies \((a_n)_{n = 1}^{\infty}\) and \((b_n)_{n = 1}^{\infty}\) are equivalent Cauchy sequences.
Then by Definition \ref{5.2.6}, \(\forall\ \varepsilon > 0\) and \(\varepsilon \in \mathds{Q}\), \(\exists\ N \geq 1\) and \(N \in \mathds{N}\) such that \(a_n\) is \(\varepsilon\)-close to \(b_n\) for all \(n \geq N\).
But by Proposition \ref{4.3.7}, \(a_n\) is \(\varepsilon\)-close to \(b_n\) implies that \(b_n\) is \(\varepsilon\)-close to \(a_n\).
So we have \(b_n\) is \(\varepsilon\)-close to \(a_n\), \(\forall\ \varepsilon > 0\) and \(\forall\ n \geq N\), which means \((b_n)_{n = 1}^{\infty}\) and \((a_n)_{n = 1}^{\infty}\) are equivalent sequences by Definition \ref{5.2.6}.
Since \((a_n)_{n = 1}^{\infty}\) and \((b_n)_{n = 1}^{\infty}\) are Cauchy sequences, by Definition \ref{5.3.1}, \(y = \text{LIM}_{n \to \infty} b_n = \text{LIM}_{n \to \infty} a_n = x\).

Finally we prove the transitivity.
By Definition \ref{5.3.1}, \(x = y\) implies that \((a_n)_{n = 1}^{\infty}\) and \((b_n)_{n = 1}^{\infty}\) are equivalent Cauchy sequences.
Then by Definition \ref{5.2.6}, \(\forall\ \varepsilon > 0\) and \(\varepsilon \in \mathds{Q}\), \(\exists\ N_1 \geq 1\) and \(N_1 \in \mathds{N}\) such that \(a_n\) is \(\varepsilon\)-close to \(b_n\) for all \(n \geq N_1\).
Since \(\varepsilon > 0\), by Additional Corollary \ref{ac 4.2.5}, \(\varepsilon / 2 > 0\), so \(a_n\) is also \((\varepsilon / 2)\)-close to \(b_n\).
Similarly, \(y = z\) implies that \(\forall\ \varepsilon > 0\) and \(\varepsilon \in \mathds{Q}\), \(\exists\ N_2 \geq 1\) and \(N_2 \in \mathds{N}\) such that \(b_n\) is \((\varepsilon / 2)\)-close to \(c_n\) for all \(n \geq N_2\).
Let \(N = N_1 + N_2\).
Then by Additional Corollary \ref{ac 2.2.1}, \(N \in \mathds{N}\).
And by Definition \ref{2.2.11}, \(N \geq N_1\) and \(N \geq N_2\).
So we have \(a_n\) is \((\varepsilon / 2)\)-close to \(b_n\) and \(b_n\) is \((\varepsilon / 2)\)-close to \(c_n\), \(\forall\ \varepsilon > 0\) and \(\forall\ n \geq N\).
Then by Proposition \ref{4.3.7}, \(a_n\) is \(\varepsilon\)-close to \(c_n\), \(\forall\ \varepsilon > 0\) and \(\forall\ n \geq N\).
Thus by Definition \ref{5.2.6}, \((a_n)_{n = 1}^{\infty}\) and \((c_n)_{n = 1}^{\infty}\) are equivalent sequences.
Since \((a_n)_{n = 1}^{\infty}\) and \((c_n)_{n = 1}^{\infty}\) are Cauchy sequences, by Definition \ref{5.3.1}, \(x = \text{LIM}_{n \to \infty} a_n = \text{LIM}_{n \to \infty} c_n = z\).
\end{proof}

\begin{note}
Because of Proposition \ref{5.3.3}, we know that our definition of equality between two real numbers is legitimate.
Of course, when we define other operations on the reals, we have to check that they obey the axiom of substitution:
two real number inputs which are equal should give equal outputs when applied to any operation on the real numbers.
\end{note}

\begin{definition}[Addition of reals]\label{5.3.4}
Let \(x = \text{LIM}_{n \to \infty} a_n\) and \(y = \text{LIM}_{n \to \infty} b_n\) be real numbers.
Then we define the sum \(x + y\) to be \(x + y \coloneqq \text{LIM}_{n \to \infty} (a_n + b_n)\).
\end{definition}

\setcounter{theorem}{5}
\begin{lemma}[Sum of Cauchy sequences is Cauchy]\label{5.3.6}
Let \(x = \text{LIM}_{n \to \infty} a_n\) and \(y = \text{LIM}_{n \to \infty} b_n\) be real numbers.
Then \(x + y\) is also a real number
(i.e., \((a_n + b_n)_{n = 1}^{\infty}\) is a Cauchy sequence of rationals).
\end{lemma}

\begin{proof}
We need to show that for every \(\varepsilon > 0\), the sequence \((a_n + b_n)_{n = 1}^{\infty}\) is eventually \(\varepsilon\)-steady.
Now from hypothesis we know that \((a_n)_{n = 1}^{\infty}\) is eventually \(\varepsilon\)-steady, and \((b_n)_{n = 1}^{\infty}\) is eventually \(\varepsilon\)-steady, but it turns out that this is not quite enough
(this can be used to imply that \((a_n + b_n)_{n = 1}^{\infty}\) is eventually \(2\varepsilon\)-steady, but that's not what we want).
So we need to do a little trick, which is to play with the value of \(\varepsilon\).

We know that \((a_n)_{n = 1}^{\infty}\) is eventually \(\delta\)-steady for every value of \(\delta\).
This implies not only that \((a_n)_{n = 1}^{\infty}\) is eventually \(\varepsilon\)-steady, but it is also eventually \(\varepsilon / 2\)-steady.
Similarly, the sequence \((b_n)_{n = 1}^{\infty}\) is also eventually \(\varepsilon / 2\)-steady.
This will turn out to be enough to conclude that \((a_n + b_n)_{n = 1}^{\infty}\) is eventually \(\varepsilon\)-steady.

Since \((a_n)_{n = 1}^{\infty}\) is eventually \(\varepsilon / 2\)-steady, we know that there exists an \(N \geq 1\) such that \((a_n)_{n = N}^{\infty}\) is \(\varepsilon / 2\)-steady, i.e., \(a_n\) and \(a_m\) are \(\varepsilon / 2\)-close for every \(n, m \geq N\).
Similarly there exists an \(M \geq 1\) such that \((b_n)_{n = M}^{\infty}\) is \(\varepsilon / 2\)-steady, i.e., \(b_n\) and \(b_m\) are \(\varepsilon / 2\)-close for every \(n, m \geq M\).

Let \(\max(N, M)\) be the larger of \(N\) and \(M\)
(we know from Proposition \ref{2.2.13} that one has to be greater than or equal to the other).
If \(n, m \geq \max(N, M)\), then we know that \(a_n\) and \(a_m\) are \(\varepsilon / 2\)-close, and \(b_n\) and \(b_m\) are \(\varepsilon / 2\)-close, and so by Proposition \ref{4.3.7} we see that \(a_n + b_n\) and \(a_m + b_m\) are \(\varepsilon\)-close for every \(n, m \geq \max(N, M)\).
This implies that the sequence \((a_n + b_n)_{n = 1}^{\infty}\) is eventually \(\varepsilon\)-steady, as desired.
\end{proof}

\begin{lemma}[Sums of equivalent Cauchy sequences are equivalent]\label{5.3.7}
Let \(x = \text{LIM}_{n \to \infty} a_n\), \(y = \text{LIM}_{n \to \infty} b_n\), and \(x' = \text{LIM}_{n \to \infty} a'_n\) be real numbers.
Suppose that \(x = x'\).
Then we have \(x + y = x' + y\).
\end{lemma}

\begin{proof}
Since \(x\) and \(x'\) are equal, we know that the Cauchy sequences \((a_n)_{n = 1}^{\infty}\) and \((a'_n)_{n = 1}^{\infty}\) are equivalent, so in other words they are eventually \(\varepsilon\)-close for each \(\varepsilon > 0\).
We need to show that the sequences \((a_n + b_n)_{n = 1}^{\infty}\) and \((a'_n + b_n)_{n = 1}^{\infty}\) are eventually \(\varepsilon\)-close for each \(\varepsilon > 0\).
But we already know that there is an \(N \geq 1\) such that \((a_n)_{n = N}^{\infty}\) and \((a'_n)_{n = N}^{\infty}\) are \(\varepsilon\)-close, i.e., that \(a_n\) and \(a'_n\) are \(\varepsilon\)-close for each \(n \geq N\).
Since \(b_n\) is of course \(0\)-close to \(b_n\), we thus see from Proposition \ref{4.3.7} (extended in the obvious manner to the \(\delta = 0\) case) that \(a_n + b_n\) and \(a'_n + b_n\) are \(\varepsilon\)-close for each \(n \geq N\).
This implies that \((a_n + b_n)_{n = 1}^{\infty}\) and \((a'_n + b_n)_{n = 1}^{\infty}\) are eventually \(\varepsilon\)-close for each \(\varepsilon > 0\), and we are done.
\end{proof}

\begin{remark}\label{5.3.8}
Lemma \ref{5.3.7} verifies the axiom of substitution for the ``x'' variable in \(x + y\), but one can similarly prove the axiom of substitution for the ``y'' variable.
(A quick way is to observe from the definition of \(x + y\) that we certainly have \(x + y = y + x\), since \(a_n + b_n = b_n + a_n\).)
\end{remark}

\begin{definition}[Multiplication of reals]\label{5.3.9}
Let \(x = \text{LIM}_{n \to \infty} a_n\) and \(y = \text{LIM}_{n \to \infty} b_n\) be real numbers.
Then we define the product \(xy\) to be \(xy \coloneqq \text{LIM}_{n \to \infty} a_n b_n\).
\end{definition}

\begin{proposition}[Multiplication is well defined]\label{5.3.10}
Let \(x = \text{LIM}_{n \to \infty} a_n\), \(y = \text{LIM}_{n \to \infty} b_n\), and \(x' = \text{LIM}_{n \to \infty} a'_n\) be real numbers.
Then \(xy\) is also a real number.
Furthermore, if \(x = x'\), then \(xy = x'y\).
\end{proposition}


\exercisesection

\begin{exercise}\label{ex 5.3.1}
Prove Proposition \ref{5.3.3}.
\end{exercise}

\begin{proof}
See Proposition \ref{5.3.3}.
\end{proof}

\begin{exercise}\label{ex 5.3.2}
Prove Proposition \ref{5.3.10}.
\end{exercise}

\begin{proof}
See Proposition \ref{5.3.10}.
\end{proof}