\section{Ordering the reals}\label{sec 5.4}

\begin{definition}\label{5.4.1}
    Let \((a_n)_{n = 1}^{\infty}\) be a sequence of rationals.
    We say that this sequence is \emph{positively bounded away from zero} iff we have a positive rational \(c > 0\) such that \(a_n \geq c\) for all \(n \geq 1\) (in particular, the sequence is entirely positive).
    The sequence is \emph{negatively bounded away from zero} iff we have a negative rational \(-c < 0\) such that \(a_n \leq -c\) for all \(n \geq 1\) (in particular, the sequence is entirely negative).
\end{definition}

\begin{note}
    It is clear that any sequence which is positively or negatively bounded away from zero, is bounded away from zero.
    Also, a sequence cannot be both positively bounded away from zero and negatively bounded away from zero at the same time.
\end{note}

\setcounter{theorem}{2}
\begin{definition}\label{5.4.3}
    A real number \(x\) is said to be \emph{positive} iff it can be written as \(x = \text{LIM}_{n \to \infty} a_n\) for some Cauchy sequence \((a_n)_{n = 1}^{\infty}\) which is positively bounded away from zero.
    \(x\) is said to be \emph{negative} iff it can be written as \(x = \text{LIM}_{n \to \infty} a_n\) for some sequence \((a_n)_{n = 1}^{\infty}\) which is negatively bounded away from zero.
\end{definition}

\begin{proposition}[Basic properties of positive reals]\label{5.4.4}
    For every real number \(x\), exactly one of the following three statements is true:
    \begin{enumerate*}
        \item \(x\) is zero;
        \item \(x\) is positive;
        \item \(x\) is negative.
    \end{enumerate*}
    A real number \(x\) is negative if and only if \(-x\) is positive.
    If \(x\) and \(y\) are positive, then so are \(x + y\) and \(xy\).
\end{proposition}

\begin{proof}
    We first show that at least one of the three statements is true.
    Let \(x\) be the formal limit of some rational sequence \((a_n)_{n = 1}^{\infty}\), let \(\varepsilon, c \in \mathbf{Q}^+\) and let \(N, j, k \in \mathbf{N}\).
    Consider the following two cases:
    \begin{itemize}
        \item If \((a_n)_{n = 1}^{\infty}\) is eventually \(\varepsilon\)-close to \(0\) for all \(\varepsilon > 0\), then by Definition \ref{5.2.6} we have \(x = 0\).
        \item If \((a_n)_{n = 1}^{\infty}\) is not eventually \(\varepsilon\)-close to \(0\) for all \(\varepsilon > 0\), then by Definition \ref{5.2.6} we have \(x \neq 0\).
    \end{itemize}
    By Lemma \ref{5.3.14}, \(x \neq 0\) implies \(\exists\ c > 0\) such that \(\abs*{a_n} \geq c > 0\) for every \(n \geq 1\).
    By Proposition \ref{4.3.3}(a) we have \(a_n \neq 0\).
    By Proposition \ref{4.2.9}(a) we now split into two cases:
    \begin{itemize}
        \item If \(a_n > 0\), then by Definition \ref{4.3.1} we have \(a_n \geq c > 0\).
        \item If \(a_n < 0\), then by Definition \ref{4.3.1} we have \(-a_n \geq c > 0\), and by Exercise \ref{ex 4.2.6} we have \(a_n \leq -c < 0\).
    \end{itemize}
    Since \((a_n)_{n = 1}^{\infty}\) is a Cauchy sequence, by Definition \ref{5.1.8} we have
    \[
        \forall\ \varepsilon > 0, \exists\ N \geq 1 : \forall\ j, k \geq N, \abs*{a_j - a_k} \leq \varepsilon.
    \]
    In particular,
    \[
        \exists\ N \geq 1 : \forall\ j, k \geq N, \abs*{a_j - a_k} \leq c.
    \]
    So
    \begin{align*}
                 & \abs*{a_j - a_N} \leq c                                               \\
        \implies & -c \leq a_j - a_N \leq c        & \text{(by Proposition \ref{4.3.3})} \\
        \implies & -c + a_N \leq a_j \leq c + a_N. & \text{(by Proposition \ref{4.2.9})} \\
    \end{align*}
    By Proposition \ref{4.2.9}(a) again we now split into two cases:
    \begin{itemize}
        \item If \(a_N > 0\), then we have
              \begin{align*}
                           & (-c + a_N \leq a_j \leq c + a_N) \land (0 < c \leq a_N)                                             \\
                  \implies & 0 \leq a_j \leq c + a_N                                 & \text{(by Proposition \ref{4.2.9}(c)(d))} \\
                  \implies & c < a_j \leq c + a_N.                                   & (x \neq 0)
              \end{align*}
              Since this is true for all \(j \geq N\), by Lemma \ref{5.3.14} and Definition \ref{5.4.1} we know that \((a_n)_{n = 1}^\infty\) is positively bounded away from zero.
              Thus by Definition \ref{5.4.3} \(x\) is positive.
        \item If \(a_N < 0\), then we have
              \begin{align*}
                           & (-c + a_N \leq a_j \leq c + a_N) \land (a_N \leq -c < 0)                                             \\
                  \implies & -c + a_N \leq a_j \leq 0                                 & \text{(by Proposition \ref{4.2.9}(c)(d))} \\
                  \implies & -c + a_N \leq a_j < -c.                                  & (x \neq 0)
              \end{align*}
              Since this is true for all \(j \geq N\), by Lemma \ref{5.3.14} and Definition \ref{5.4.1} we know that \((a_n)_{n = 1}^\infty\) is negatively bounded away from zero.
              Thus by Definition \ref{5.4.3} \(x\) is negative.
    \end{itemize}
    From all cases above we conclude that at least one of the three statements is true.

    Next we show that at most one of the three statements is true.
    Let \(x\) be the formal limit of some rational sequence \((a_n)_{n = 1}^{\infty}\) and let \(\varepsilon, c \in \mathbf{Q}^+\).
    \begin{itemize}
        \item If \(x = 0\) and \(x\) is positive, then we have \((a_n)\) eventually \(\varepsilon\)-close to \(0\) for all \(\varepsilon > 0\) and \(a_n \geq c\) for all \(n \geq 1\).
              But then we have \(\abs*{a_n - 0} \leq c / 2\) and \(a_n \geq c > 0\), a contradiction.
        \item If \(x = 0\) and \(x\) is negative, then we have \((a_n)\) eventually \(\varepsilon\)-close to \(0\) for all \(\varepsilon > 0\) and \(a_n \leq -c\) for all \(n \geq 1\).
              But then we have \(\abs*{a_n - 0} \leq c / 2\) and \(a_n \leq -c < 0\), a contradiction.
        \item If \(x\) is positive and \(x\) is negative, then we have \(a_n \geq c\) and \(a_n \leq -c\) for all \(n \geq 1\).
              But then we have \(a_n < 0\) and \(0 < a_n\), a contradiction.
    \end{itemize}
    From all cases above we conclude that at most one of the three statements is true.

    Next we show that \(x\) is negative iff \(-x\) is positive.
    Let \(x\) be the formal limit of some sequence \((a_n)_{n = 1}^{\infty}\).
    Then we have
    \begin{align*}
             & x \text{ is negative}                                                                                          \\
        \iff & (x = \text{LIM}_{n \to \infty} a_n)                                                                            \\
             & \land (\exists\ c \in \mathbf{Q}^+ : \forall\ n \geq 1, a_n \leq -c < 0) & \text{(by Definition \ref{5.4.3})}  \\
        \iff & (-x = \text{LIM}_{n \to \infty} -a_n)                                    & \text{(by Definition \ref{5.3.9})}  \\
             & \land (\exists\ c \in \mathbf{Q}^+ : \forall\ n \geq 1, a_n \leq -c < 0)                                       \\
        \iff & (-x = \text{LIM}_{n \to \infty} -a_n)                                                                          \\
             & \land (\exists\ c \in \mathbf{Q}^+ : \forall\ n \geq 1, -a_n \geq c > 0) & \text{(by Exercise \ref{ex 4.2.6})} \\
        \iff & -x \text{ is positive}.                                                  & \text{(by Definition \ref{5.4.3})}
    \end{align*}

    Next we show that \(x, y\) are positive implies \(x + y\) is also positive.
    Let \(x\) be the formal limit of some sequence \((a_n)_{n = 1}^{\infty}\) and let \(y\) be the formal limit of some sequence \((b_n)_{n = 1}^{\infty}\).
    Then we have
    \begin{align*}
                 & (x \text{ is positive}) \land (y \text{ is positive})                                                                                 \\
        \implies & (x = \text{LIM}_{n \to \infty} a_n) \land (y = \text{LIM}_{n \to \infty} b_n)                                                         \\
                 & \land (\exists\ c_1 \in \mathbf{Q}^+ : \forall\ n \geq 1, 0 < c_1 < a_n)                                                              \\
                 & \land (\exists\ c_2 \in \mathbf{Q}^+ : \forall\ n \geq 1, 0 < c_2 < b_n)                  & \text{(by Definition \ref{5.4.3})}        \\
        \implies & (x + y = \text{LIM}_{n \to \infty} a_n + b_n)                                             & \text{(by Definition \ref{5.3.4})}        \\
                 & \land (\exists\ c_1 \in \mathbf{Q}^+ : \forall\ n \geq 1, 0 < c_1 < a_n)                                                              \\
                 & \land (\exists\ c_2 \in \mathbf{Q}^+ : \forall\ n \geq 1, 0 < c_2 < b_n)                                                              \\
        \implies & (x + y = \text{LIM}_{n \to \infty} a_n + b_n)                                                                                         \\
                 & \land (\exists\ c_1, c_2 \in \mathbf{Q}^+ : \forall\ n \geq 1, 0 < c_1 + c_2 < a_n + b_n) & \text{(by Proposition \ref{4.2.9}(c)(d))} \\
        \implies & x + y \text{ is positive}.                                                                & \text{(by Definition \ref{5.4.3})}
    \end{align*}

    Finally we show that \(x, y\) are positive implies \(xy\) is also positive.
    Let \(x\) be the formal limit of some sequence \((a_n)_{n = 1}^{\infty}\) and let \(y\) be the formal limit of some sequence \((b_n)_{n = 1}^{\infty}\).
    Then we have
    \begin{align*}
                 & (x \text{ is positive}) \land (y \text{ is positive})                                                                             \\
        \implies & (x = \text{LIM}_{n \to \infty} a_n) \land (y = \text{LIM}_{n \to \infty} b_n)                                                     \\
                 & \land (\exists\ c_1 \in \mathbf{Q}^+ : \forall\ n \geq 1, 0 < c_1 < a_n)                                                          \\
                 & \land (\exists\ c_2 \in \mathbf{Q}^+ : \forall\ n \geq 1, 0 < c_2 < b_n)              & \text{(by Definition \ref{5.4.3})}        \\
        \implies & (xy = \text{LIM}_{n \to \infty} a_n b_n)                                              & \text{(by Definition \ref{5.3.9})}        \\
                 & \land (\exists\ c_1 \in \mathbf{Q}^+ : \forall\ n \geq 1, 0 < c_1 < a_n)                                                          \\
                 & \land (\exists\ c_2 \in \mathbf{Q}^+ : \forall\ n \geq 1, 0 < c_2 < b_n)                                                          \\
        \implies & (xy = \text{LIM}_{n \to \infty} a_n b_n)                                                                                          \\
                 & \land (\exists\ c_1, c_2 \in \mathbf{Q}^+ : \forall\ n \geq 1, 0 < c_1 c_2 < a_n b_n) & \text{(by Proposition \ref{4.2.9}(d)(e))} \\
        \implies & xy \text{ is positive}.                                                               & \text{(by Definition \ref{5.4.3})}
    \end{align*}
\end{proof}

\begin{note}
    If \(q\) is a positive rational number, then the Cauchy sequence \(q, q, q, \dots\) is positively bounded away from zero, and hence \(\text{LIM}_{n \to \infty} q = q\) is a positive real number.
    Thus the notion of positivity for rationals is consistent with that for reals.
    Similarly, the notion of negativity for rationals is consistent with that for reals.
\end{note}

\begin{definition}[Absolute value]\label{5.4.5}
    Let \(x\) be a real number.
    We define the \emph{absolute value} \(\abs*{x}\) of \(x\) to equal \(x\) if \(x\) is positive, \(-x\) when \(x\) is negative, and \(0\) when \(x\) is zero.
\end{definition}

\begin{definition}[Ordering of the real numbers]\label{5.4.6}
    Let \(x\) and \(y\) be real numbers.
    We say that \(x\) is \emph{greater than} \(y\), and write \(x > y\), iff \(x - y\) is a positive real number, and \(x < y\) iff \(x - y\) is a negative real number.
    We define \(x \geq y\) iff \(x > y\) or \(x = y\), and similarly define \(x \leq y\).
\end{definition}

\begin{note}
    Comparing this with the definition of order on the rationals from Definition \ref{4.2.8} we see that order on the reals is consistent with order on the rationals, i.e., if two rational numbers \(q, q'\) are such that \(q\) is less than \(q'\) in the rational number system, then \(q\) is still less than \(q'\) in the real number system, and similarly for ``greater than''.
    In the same way we see that the definition of absolute value given here is consistent with that in Definition \ref{4.3.1}.
\end{note}

\begin{proposition}\label{5.4.7}
    All the claims in Proposition \ref{4.2.9} which held for rationals, continue to hold for real numbers.
\end{proposition}

\begin{proof}{(a)}
    We first show that at least one of the three statements is true.
    By Proposition \ref{5.4.4}, exactly one of the following three statements is true:
    \begin{itemize}
        \item \(x - y = 0\).
              Then by Definition \ref{5.3.11} we have \(x = y\).
        \item \(x - y\) is positive.
              Then by Definition \ref{5.4.6} we have \(x > y\).
        \item \(x - y\) is negative.
              Then by Definition \ref{5.4.6} we have \(x < y\).
    \end{itemize}
    So at least one of the three statements is true.

    Now we show that at most one of the three statements is true.
    \begin{itemize}
        \item If \(x = y\) and \(x > y\) are true, then by Definition \ref{5.3.11} we have \(x - y = 0\) and by Definition \ref{5.4.6} we have \(x - y\) is positive.
              But this contradict to Proposition \ref{5.4.4}.
        \item If \(x = y\) and \(x < y\) are true, then by Definition \ref{5.3.11} we have \(x - y = 0\) and by Definition \ref{5.4.6} we have \(x - y\) is negative.
              But this contradict to Proposition \ref{5.4.4}.
        \item If \(x > y\) and \(x < y\) are true, then by Definition \ref{5.4.6}, \(x - y\) is both positive and negative.
              But this contradict to Proposition \ref{5.4.4}.
    \end{itemize}
    From all cases above we conclude that at most one of the three statements is true.
\end{proof}

\begin{proof}{(b)}
    We have
    \begin{align*}
             & x < y                                                               \\
        \iff & x - y \text{ is negative}    & \text{(by Definition \ref{5.4.6})}   \\
        \iff & -(x - y) \text{ is positive} & \text{(by Proposition \ref{5.4.4})}  \\
        \iff & y - x \text{ is positive}    & \text{(by Proposition \ref{5.3.11})} \\
        \iff & y > x.                       & \text{(by Definition \ref{5.4.6})}
    \end{align*}
\end{proof}

\begin{proof}{(c)}
    We have
    \begin{align*}
                 & (x < y) \land (y < z)                                                                                      \\
        \implies & (x - y \text{ is negative}) \land (y - z \text{ is negative})       & \text{(by Definition \ref{5.4.6})}   \\
        \implies & (-(x - y) \text{ is positive}) \land (-(y - z) \text{ is positive}) & \text{(by Proposition \ref{5.4.4})}  \\
        \implies & (y - x \text{ is positive}) \land (z - y \text{ is positive})       & \text{(by Proposition \ref{5.3.11})} \\
        \implies & y - x + z - y \text{ is positive}                                   & \text{(by Proposition \ref{5.4.4})}  \\
        \implies & - x + z \text{ is positive}                                         & \text{(by Proposition \ref{5.3.11})} \\
        \implies & -(x - z) \text{ is positive}                                        & \text{(by Proposition \ref{5.3.11})} \\
        \implies & x - z \text{ is negative}                                           & \text{(by Proposition \ref{5.4.4})}  \\
        \implies & x < z.                                                              & \text{(by Definition \ref{5.4.6})}
    \end{align*}
\end{proof}

\begin{proof}{(d)}
    We have
    \begin{align*}
                 & x < y                                                                      \\
        \implies & x - y \text{ is negative}           & \text{(by Definition \ref{5.4.6})}   \\
        \implies & x + z - z - y \text{ is negative}   & \text{(by Proposition \ref{5.3.11})} \\
        \implies & x + z - (y + z) \text{ is negative} & \text{(by Proposition \ref{5.3.11})} \\
        \implies & x + z < y + z.                      & \text{(by Definition \ref{5.4.6})}
    \end{align*}
\end{proof}

\begin{proof}{(e)}
    We have
    \begin{align*}
                 & x < y                                                                 \\
        \implies & y > x                        & \text{(by Proposition \ref{5.4.7}(b))} \\
        \implies & y - x \text{ is positive}    & \text{(by Definition \ref{5.4.6})}     \\
        \implies & (y - x)z \text{ is positive} & \text{(by Proposition \ref{5.4.4})}    \\
        \implies & yz - xz \text{ is positive}  & \text{(by Proposition \ref{5.3.11})}   \\
        \implies & yz > xz                      & \text{(by Definition \ref{5.4.6})}     \\
        \implies & xz < yz.                     & \text{(by Proposition \ref{5.4.7}(b))}
    \end{align*}
\end{proof}

\begin{proposition}\label{5.4.8}
    Let \(x\) be a positive real number.
    Then \(x^{-1}\) is also positive.
    Also, if \(y\) is another positive number and \(x > y\), then \(x^{-1} < y^{-1}\).
\end{proposition}

\begin{proof}
    Let \(x\) be positive.
    Since \(xx^{-1} = 1\), the real number \(x^{-1}\) cannot be zero (since \(x0 = 0 \neq 1\)).
    Also, from Proposition \ref{5.4.4} it is easy to see that a positive number times a negative number is negative;
    this shows that \(x^{-1}\) cannot be negative, since this would imply that \(xx^{-1} = 1\) is negative, a contradiction.
    Thus, by Proposition \ref{5.4.4}, the only possibility left is that \(x^{-1}\) is positive.

    Now let \(y\) be positive as well, so \(x^{-1}\) and \(y^{-1}\) are also positive.
    If \(x^{-1} \geq y^{-1}\), then by Proposition \ref{5.4.7} we have \(xx^{-1} > yx^{-1} \geq yy^{-1}\), thus \(1 > 1\), which is a contradiction.
    Thus we must have \(x^{-1} < y^{-1}\).
\end{proof}

\begin{proposition}[The non-negative reals are closed]\label{5.4.9}
    Let \(a_1, a_2, a_3, \dots\) be a Cauchy sequence of non-negative rational numbers.
    Then \(\text{LIM}_{n \to \infty} a_n\) is a non-negative real number.
\end{proposition}

\begin{proof}
    We argue by contradiction, and suppose that the real number \(x \coloneqq \text{LIM}_{n \to \infty} a_n\) is a negative number.
    Then by definition of negative real number, we have \(x = \text{LIM}_{n \to \infty} b_n\) for some sequence \(b_n\) which is negatively bounded away from zero, i.e., there is a negative rational \(-c < 0\) such that \(b_n \leq -c\) for all \(n \geq 1\).
    On the other hand, we have \(a_n \geq 0\) for all \(n \geq 1\), by hypothesis.
    Thus the numbers \(a_n\) and \(b_n\) are never \(c / 2\)-close, since \(c / 2 < c\).
    Thus the sequences \((a_n)_{n = 1}^{\infty}\) and \((b_n)_{n = 1}^{\infty}\) are not eventually \(c / 2\)-close.
    Since \(c / 2 > 0\), this implies that \((a_n)_{n = 1}^{\infty}\) and \((b_n)_{n = 1}^{\infty}\) are not equivalent.
    But this contradicts the fact that both these sequences have \(x\) as their formal limit.
\end{proof}

\begin{note}
    Eventually, we will see a better explanation of Proposition \ref{5.4.9}:
    the set of non-negative reals is \emph{closed}, whereas the set of positive reals is \emph{open}.
\end{note}

\begin{corollary}\label{5.4.10}
    Let \((a_n)_{n = 1}^{\infty}\) and \((b_n)_{n = 1}^{\infty}\) be Cauchy sequences of rationals such that \(a_n \geq b_n\) for all \(n \geq 1\).
    Then \(\text{LIM}_{n \to \infty} a_n \geq \text{LIM}_{n \to \infty} b_n\).
\end{corollary}

\begin{proof}
    Apply Proposition \ref{5.4.9} to the sequence \(a_n - b_n\).
\end{proof}

\begin{remark}\label{5.4.11}
    Note that Corollary \ref{5.4.10} does not work if the \(\geq\) signs are replaced by \(>\):
    for instance if \(a_n \coloneqq 1 + 1 / n\) and \(b_n \coloneqq 1 - 1 / n\), then \(a_n\) is always strictly greater than \(b_n\), but the formal limit of \(a_n\) is not greater than the formal limit of \(b_n\), instead they are equal.
\end{remark}

\begin{note}
    We now define distance \(d(x, y) \coloneqq \abs*{x - y}\) just as we did for the rationals.
    In fact, Propositions \ref{4.3.3} and \ref{4.3.7} hold not only for the rationals, but for the reals;
    the proof is identical, since the real numbers obey all the laws of algebra and order that the rationals do.
\end{note}

\begin{proposition}[Bounding of reals by rationals]\label{5.4.12}
    Let \(x\) be a positive real number.
    Then there exists a positive rational number \(q\) such that \(q \leq x\), and there exists a positive integer \(N\) such that \(x \leq N\).
\end{proposition}

\begin{proof}
    Since \(x\) is a positive real, it is the formal limit of some Cauchy sequence \((a_n)_{n = 1}^{\infty}\) which is positively bounded away from zero.
    Also, by Lemma \ref{5.1.15}, this sequence is bounded.
    Thus we have rationals \(q > 0\) and \(r\) such that \(q \leq a_n \leq r\) for all \(n \geq 1\).
    But by Proposition \ref{4.4.1} we know that there is some integer \(N\) such that \(r \leq N\);
    since \(q\) is positive and \(q \leq r \leq N\), we see that \(N\) is positive.
    Thus \(q \leq a_n \leq N\) for all \(n \geq 1\).
    Applying Corollary \ref{5.4.10} we obtain that \(q \leq x \leq N\), as desired.
\end{proof}

\begin{corollary}[Archimedean property]\label{5.4.13}
    Let \(x\) and \(\varepsilon\) be any positive real numbers.
    Then there exists a positive integer \(M\) such that \(M\varepsilon > x\).
\end{corollary}

\begin{proof}
    The number \(x / \varepsilon\) is positive, and hence by Proposition \ref{5.4.12} there exists a positive integer \(N\) such that \(x / \varepsilon \leq N\).
    If we set \(M \coloneqq N + 1\), then \(x / \varepsilon < M\).
    Now multiply by \(\varepsilon\).
\end{proof}

\begin{note}
    This property (Corollary \ref{5.4.13}) is quite important;
    it says that no matter how large \(x\) is and how small \(\varepsilon\) is, if one keeps adding \(\varepsilon\) to itself, one will eventually overtake \(x\).
\end{note}

\begin{proposition}\label{5.4.14}
    Given any two real numbers \(x < y\), we can find a rational number \(q\) such that \(x < q < y\).
\end{proposition}

\begin{proof}
    We have
    \begin{align*}
                 & x < y                                                                             \\
        \implies & y > x                                       & \text{(by Proposition \ref{5.4.7})} \\
        \implies & y - x \text{ is positive}                   & \text{(by Definition \ref{5.4.6})}  \\
        \implies & \exists\ N \in \mathbf{Z}^+ : y - x > 1 / N & \text{(by Exercise \ref{ex 5.4.4})} \\
        \implies & y > x + 1 / N.                              & \text{(by Proposition \ref{5.4.7})} \\
    \end{align*}
    Since \(x\) is a real number, by Proposition \ref{5.3.10} we know that \(Nx\) is also a real number.
    By Exercise \ref{ex 5.4.3}, \(\exists\ M \in \mathbf{Z}\) such that \(M \leq Nx < M + 1\).
    So we have
    \begin{align*}
                 & M \leq Nx < M + 1                                                                                        \\
        \implies & \frac{M}{N} \leq x < \frac{M + 1}{N}                               & \text{(by Proposition \ref{5.4.7})} \\
        \implies & (\frac{M}{N} \leq x) \land (x < \frac{M + 1}{N})                   & \text{(by Proposition \ref{5.4.7})} \\
        \implies & (\frac{M + 1}{N} \leq x + \frac{1}{N}) \land (x < \frac{M + 1}{N}) & \text{(by Proposition \ref{5.4.7})} \\
        \implies & x < \frac{M + 1}{N} \leq x + \frac{1}{N}                           & \text{(by Proposition \ref{5.4.7})} \\
        \implies & x < \frac{M + 1}{N} \leq x + \frac{1}{N} < y.                      & \text{(by Proposition \ref{5.4.7})}
    \end{align*}
    Since \((M + 1) / N \in \mathbf{Q}\), we conclude that \(\exists\ q = (M + 1) / N \in \mathbf{Q}\) such that \(x < q < y\) for arbitrary real numbers \(x, y\).
\end{proof}

\begin{remark}\label{5.4.15}
    Up until now, we have not addressed the fact that real numbers can be expressed using the decimal system.
    For instance, the formal limit of
    \[
        1.4, 1.41, 1.414, 1.4142, 1.41421, \dots
    \]
    is more conventionally represented as the decimal \(1.41421\dots\).
    There are some subtleties in the decimal system, for instance \(0.9999\dots\) and \(1.000\dots\) are in fact the same real number.
\end{remark}

\exercisesection

\begin{exercise}\label{ex 5.4.1}
    Prove Proposition \ref{5.4.4}.
\end{exercise}

\begin{proof}
    See Proposition \ref{5.4.4}.
\end{proof}

\begin{exercise}\label{ex 5.4.2}
    Prove the remaining claims in Proposition \ref{5.4.7}.
\end{exercise}

\begin{proof}
    See Proposition \ref{5.4.7}.
\end{proof}

\begin{exercise}\label{ex 5.4.3}
    Show that for every real number \(x\) there is exactly one integer \(N\) such that \(N \leq x < N + 1\).
    (This integer \(N\) is called the \emph{integer part} of \(x\), and is sometimes denoted \(N = \floor*{x}\).)
\end{exercise}

\begin{proof}
    We first prove the existence of the integer \(N\).
    By Proposition \ref{5.4.4}, exactly one of the following three statements is true:
    \begin{enumerate}
        \item \(x = 0\).
              Then we choose \(N = 0\) so that \(0 \leq 0 < 1\).
        \item \(x\) is positive.
              Then by Corollary \ref{5.4.13}, \(\exists\ U \in \mathbf{Z}\) and \(U > 0\) such that \(U = U1 > x\).
              And by Proposition \ref{5.4.12}, \(\exists\ q \in \mathbf{Q}\) and \(q > 0\) such that \(q \leq x\).
              Since \(q \in \mathbf{Q}\), by proposition \ref{4.4.1}, \(\exists!\ L \in \mathbf{Z}\) such that \(L \leq q < L + 1\).
              So we have \(L \leq x < U\) by Proposition \ref{5.4.7}.
              Let \(S_1 = \{n \in \mathbf{Z} : L \leq n < U\}\), then \(S_1\) is a finite set since \(\#(S_1) = U - L\).
              Now let \(S_2 = \{n \in S_1 : L \leq n \leq x\}\), then \(S_2\) is also a finite set by Proposition \ref{3.6.14}.
              Since \(L \in \mathbf{Z}\) and \(L \leq x\), \(S_2 \neq \emptyset\).
              Then we can choose an integer \(N\) from \(S_2\) where \(N\) is the largest integer.
              We must have \(x < N + 1\), otherwised \(x \geq N + 1 \implies N > N + 1\), a contradiction.
              So we have \(N \leq x < N + 1\), and we are done in this case.
        \item \(x\) is negative.
              By Proposition \ref{5.4.4}, \(-x\) is positive.
              Then by Proposition \ref{5.4.12}, \(\exists\ L \in \mathbf{Z}\) and \(L > 0\) such that \(-x \leq L\).
              And By Proposition \ref{5.4.7}, we have \(-L \leq x\).
              Again by Proposition \ref{5.4.12}, \(\exists\ q \in \mathbf{Q}\) and \(q > 0\) such that \(q \leq -x\).
              Since \(q \in \mathbf{Q}\), by proposition \ref{4.4.1}, \(\exists!\ U \in \mathbf{Z}\) such that \(U \leq q < U + 1\).
              We also have \(U - 1 < q\) by Proposition \ref{4.2.9}, combine with \(q \leq -x\) we have \(U - 1 < -x\), or equivalently \(x < -(U - 1)\) by Proposition \ref{5.4.7}.
              So we have \(-L \leq x < -(U - 1)\) by Proposition \ref{5.4.7}.
              Let \(S_1 = \{n \in \mathbf{Z} : -L \leq n < -(U + 1)\}\), then \(S_1\) is a finite set since \(\#(S_1) = L - (U + 1)\).
              Now let \(S_2 = \{n \in S_1 : -L \leq n \leq x\}\), then \(S_2\) is also a finite set by Proposition \ref{3.6.14}.
              Since \(-L \in \mathbf{Z}\) and \(-L \leq x\), \(S_2 \neq \emptyset\).
              Then we can choose an integer \(N\) from \(S_2\) where \(N\) is the largest integer.
              We must have \(x < N + 1\), otherwised \(x \geq N + 1 \implies N > N + 1\), a contradiction.
              So we have \(N \leq x < N + 1\), and we are done in this case.
    \end{enumerate}
    From all cases above, we can conclude that \(\exists\ N \in \mathbf{Z} : N \leq x < N + 1\).

    Now we prove the uniqueness of the integer \(N\).
    Suppose that \(\exists\ N_1, N_2 \in \mathbf{Z}\) such that \(N_1 \leq x < N_1 + 1\) and \(N_2 \leq x < N_2 + 1\).
    Then we get \(N_1 < N_2 + 1\) and \(N_2 < N_1 + 1\).
    So
    \begin{align*}
                 & (N_1 < N_2 + 1) \land (N_2 < N_1 + 1)                                                     \\
        \implies & (N_1 + 1 \leq N_2 + 1) \land (N_2 + 1 \leq N_1 + 1) & \text{(by Definition \ref{4.1.10})} \\
        \implies & (N_1 \leq N_2) \land (N_2 \leq N_1)                 & \text{(by Definition \ref{4.1.10})} \\
        \implies & N_1 = N_2.                                          & \text{(by Lemma \ref{4.1.11})}
    \end{align*}
    Thus we conclude that \(\exists!\ N \in \mathbf{Z} : N \leq x < N + 1\).
\end{proof}

\begin{exercise}\label{ex 5.4.4}
    Show that for any positive real number \(x > 0\) there exists a positive integer \(N\) such that \(x > 1 / N > 0\).
\end{exercise}

\begin{proof}
    Since \(x\) is positive, \(x^{-1}\) is also positive by Proposition \ref{5.4.8}.
    By Corollary \ref{5.4.13}, \(\exists\ N \in \mathbf{Z}\) and \(N > 0\) such that \(N = N1 > 1 / x\).
    Again by Proposition \ref{5.4.8} we have \(0 < 1 / N < x\).
    This conclude the proof.
\end{proof}

\begin{exercise}\label{ex 5.4.5}
    Prove Proposition \ref{5.4.14}.
\end{exercise}

\begin{proof}
    See Proposition \ref{5.4.14}.
\end{proof}

\begin{exercise}\label{ex 5.4.6}
    Let \(x, y\) be real numbers and let \(\varepsilon > 0\) be a positive real.
    Show that \(\abs*{x - y} < \varepsilon\) if and only if \(y - \varepsilon < x < y + \varepsilon\), and that \(\abs*{x - y} \leq \varepsilon\) if and only if \(y - \varepsilon \leq x \leq y + \varepsilon\).
\end{exercise}

\begin{proof}
    We first prove that \(\abs*{x - y} < \varepsilon \iff y - \varepsilon < x < y + \varepsilon\).
    \begin{align*}
             & \abs*{x - y} < \varepsilon                                                                  \\
        \iff & (x - y < \varepsilon) \land (-(x - y) < \varepsilon) & \text{(by Definition \ref{5.4.5})}   \\
        \iff & (x - y < \varepsilon) \land (y - x < \varepsilon)    & \text{(by Proposition \ref{5.3.11})} \\
        \iff & (x < y + \varepsilon) \land (y - \varepsilon < x)    & \text{(by Proposition \ref{5.4.7})}  \\
        \iff & y - \varepsilon < x < y + \varepsilon.               & \text{(by Proposition \ref{5.4.7})}
    \end{align*}

    Now we prove that \(\abs*{x - y} \leq \varepsilon \iff y - \varepsilon \leq x \leq y + \varepsilon\).
    \begin{align*}
             & \abs*{x - y} \leq \varepsilon                                                                     \\
        \iff & (x - y \leq \varepsilon) \land (-(x - y) \leq \varepsilon) & \text{(by Definition \ref{5.4.5})}   \\
        \iff & (x - y \leq \varepsilon) \land (y - x \leq \varepsilon)    & \text{(by Proposition \ref{5.3.11})} \\
        \iff & (x \leq y + \varepsilon) \land (y - \varepsilon \leq x)    & \text{(by Proposition \ref{5.4.7})}  \\
        \iff & y - \varepsilon \leq x \leq y + \varepsilon.               & \text{(by Proposition \ref{5.4.7})}
    \end{align*}
\end{proof}

\begin{exercise}\label{ex 5.4.7}
    Let \(x\) and \(y\) be real numbers.
    Show that \(x \leq y + \varepsilon\) for all real numbers \(\varepsilon > 0\) if and only if \(x \leq y\).
    Show that \(\abs*{x - y} \leq \varepsilon\) for all real numbers \(\varepsilon > 0\) if and only if \(x = y\).
\end{exercise}

\begin{proof}
    We first prove that \(x \leq y + \varepsilon\) for all real numbers \(\varepsilon > 0\) if and only if \(x \leq y\).
    \begin{align*}
             & (x \leq y + \varepsilon) \land (\varepsilon > 0)                                       \\
        \iff & (x - y \leq \varepsilon) \land (\varepsilon > 0) & \text{(by Proposition \ref{5.4.7})} \\
        \iff & x - y \leq 0                                     & \text{(by Proposition \ref{5.4.7})} \\
        \iff & x \leq y.                                        & \text{(by Definition \ref{5.4.6})}
    \end{align*}

    Now we prove that \(\abs*{x - y} \leq \varepsilon\) for all real numbers \(\varepsilon > 0\) if and only if \(x = y\).
    \begin{align*}
             & (\abs*{x - y} \leq \varepsilon) \land (\varepsilon > 0)                                                                   \\
        \iff & (x - y \leq \varepsilon) \land (-(x - y) \leq \varepsilon) \land (\varepsilon > 0) & \text{(by Definition \ref{5.4.5})}   \\
        \iff & (x - y \leq \varepsilon) \land (y - x \leq \varepsilon) \land (\varepsilon > 0)    & \text{(by Proposition \ref{5.3.11})} \\
        \iff & (x - y \leq 0) \land (y - x \leq 0)                                                & \text{(by Proposition \ref{5.4.7})}  \\
        \iff & (x \leq y) \land (y \leq x)                                                        & \text{(by Proposition \ref{5.4.7})}  \\
        \iff & x = y.                                                                             & \text{(by Proposition \ref{5.3.11})}
    \end{align*}
\end{proof}

\begin{exercise}\label{ex 5.4.8}
    Let \((a_n)_{n = 1}^{\infty}\) be a Cauchy sequence of rationals, and let \(x\) be a real number.
    Show that if \(a_n \leq x\) for all \(n \geq 1\), then \(\text{LIM}_{n \to \infty} a_n \leq x\).
    Similarly, show that if \(a_n \geq x\) for all \(n \geq 1\), then \(\text{LIM}_{n \to \infty} a_n \geq x\).
\end{exercise}

\begin{proof}
    We first prove that \(a_n \leq x\) for all \(n \geq 1\) implies \(\text{LIM}_{n \to \infty} a_n \leq x\).
    Let \(a = \text{LIM}_{n \to \infty} a_n\).
    Suppose for sake of contradiction that \(a > x\).
    Then by Proposition \ref{5.4.14}, \(\exists\ q \in \mathbf{Q}\) such that \(a > q > x\).
    So \(\forall\ n \geq 1\), \(a_n \leq x < q\).
    But by Corollary \ref{5.4.10}, \(\forall\ n \geq 1\), \(a_n < q\) implies \(a = \text{LIM}_{n \to \infty} a_n < \text{LIM}_{n \to \infty} q = q\), which contradict to \(a > q\).
    Thus we must have \(a \leq x\).

    Now we prove that \(a_n \geq x\) for all \(n \geq 1\) implies \(\text{LIM}_{n \to \infty} a_n \geq x\).
    Let \(a = \text{LIM}_{n \to \infty} a_n\).
    Suppose for sake of contradiction that \(a < x\).
    Then by Proposition \ref{5.4.14}, \(\exists\ q \in \mathbf{Q}\) such that \(a < q < x\).
    So \(\forall\ n \geq 1\), \(a_n \geq x > q\).
    But by Corollary \ref{5.4.10}, \(\forall\ n \geq 1\), \(a_n > q\) implies \(a = \text{LIM}_{n \to \infty} a_n > \text{LIM}_{n \to \infty} q = q\), which contradict to \(a < q\).
    Thus we must have \(a \geq x\).
\end{proof}