\chapter{The real numbers}

\begin{note}
We defined the natural numbers using the five Peano axioms, and postulated that such a number system existed;
this is plausible, since the natural numbers correspond to the very intuitive and fundamental notion of \emph{sequential counting}.
\end{note}

\begin{note}
The symbols \(\mathds{N}\), \(\mathds{Q}\), and \(\mathds{R}\) stand for ``natural'', ``quotient'', and ``real'' respectively.
\(\mathds{Z}\) stands for ``Zahlen'', the German word for numbers.
There is also the \emph{complex numbers} \(\mathds{C}\), which obviously stands for ``complex''.
\end{note}

\begin{note}
\emph{Formal} means ``having the form of'';
at the beginning of our construction the expression \(a \text{-----} b\) did not actually \emph{mean} the difference \(a - b\), since the symbol \text{-----} was meaningless.
It only had the \emph{form} of a difference.
Later on we defined subtraction and verified that the formal difference was equal to the actual difference, so this eventually became a non-issue, and our symbol for formal differencing was discarded.
Somewhat confusingly, this use of the term ``formal'' is unrelated to the notions of a formal argument and an informal argument.
\end{note}

\begin{note}
There is a fundamental area of mathematics where the rational number system does not suffice - that of \emph{geometry}
(the study of lengths, areas, etc.).
For instance, a right-angled triangle with both sides equal to \(1\) gives a hypotenuse of \(\sqrt{2}\), which is an \emph{irrational} number, i.e., not a rational number;
see Proposition \ref{4.4.4}.
Things get even worse when one starts to deal with the sub-field of geometry known as \emph{trigonometry}, when one sees numbers such as \(\pi\) or \(\cos(1)\), which turn out to be in some sense ``even more'' irrational than \(\sqrt{2}\).
(These numbers are known as \emph{transcendental numbers}, but to discuss this further would be far beyond the scope of this text.)
Thus, in order to have a number system which can adequately describe geometry
- or even something as simple as measuring lengths on a line
- one needs to replace the rational number system with the real number system.
\end{note}

\begin{note}
In the constructions of integers and rationals, the task was to introduce one more \emph{algebraic} operation to the number system
- e.g., one can get integers from naturals by introducing subtraction, and get the rationals from the integers by introducing division.
But to get the reals from the rationals is to pass from a ``discrete'' system to a ``continuous'' one, and requires the introduction of a somewhat different notion
- that of a \emph{limit}.
\end{note}

\begin{note}
The limit is a concept which on one level is quite intuitive, but to pin down rigorously turns out to be quite difficult.
(Even such great mathematicians as Euler and Newton had difficulty with this concept.
It was only in the nineteenth century that mathematicians such as Cauchy and Dedekind figured out how to deal with limits rigorously.)
\end{note}

\begin{note}
The procedure we give here of obtaining the real numbers as the limit of sequences of rational numbers may seem rather complicated.
However, it is in fact an instance of a very general and useful procedure, that of \emph{completing} one metric space to form another.
\end{note}

\section{Cauchy sequences}\label{sec 5.1}

\begin{definition}[Sequences]\label{5.1.1}
    Let \(m\) be an integer.
    A \emph{sequence \((a_n)_{n = m}^{\infty}\) of rational numbers} is any function from the set \(\{n \in \mathbf{Z} : n \geq m\}\) to \(\mathbf{Q}\), i.e., a mapping which assigns to each integer \(n\) greater than or equal to \(m\), a rational number \(a_n\).
    More informally, a sequence \((a_n)_{n = m}^{\infty}\) of rational numbers is a collection of rationals \(a_m, a_{m + 1}, a_{m + 2}, \dots\).
\end{definition}

\setcounter{theorem}{2}
\begin{definition}[\(\varepsilon\)-steadiness]\label{5.1.3}
    Let \(\varepsilon > 0\).
    A sequence \((a_n)_{n = 0}^{\infty}\) is said to be \emph{\(\varepsilon\)-steady} iff each pair \(a_j\), \(a_k\) of sequence elements is \(\varepsilon\)-close for every natural number \(j, k\).
    In other words, the sequence \(a_0, a_1, a_2, \dots\) is \(\varepsilon\)-steady iff \(d(a_j, a_k) \leq \varepsilon\) for all \(j, k\).
\end{definition}

\begin{remark}\label{5.1.4}
    Definition \ref{5.1.3} is not standard in the literature;
    we will not need it outside of this section;
    similarly for the concept of ``eventual \(\varepsilon\)-steadiness'' below.
    We have defined \(\varepsilon\)-steadiness for sequences whose index starts at \(0\), but clearly we can make a similar notion for sequences whose indices start from any other number:
    a sequence \(a_N, a_{N + 1}, \dots\) is \(\varepsilon\)-steady if one has \(d(a_j, a_k) \leq \varepsilon\) for all \(j, k \geq N\).
\end{remark}

\begin{note}
    The notion of \(\varepsilon\)-steadiness of a sequence is simple, but does not really capture the \emph{limiting} behavior of a sequence, because it is too sensitive to the initial members of the sequence.
    So we need a more robust notion of steadiness that does not care about the initial members of a sequence.
\end{note}

\setcounter{theorem}{5}
\begin{definition}[Eventual \(\varepsilon\)-steadiness]\label{5.1.6}
    Let \(\varepsilon > 0\).
    A sequence \((a_n)_{n = 0}^{\infty}\) is said to be \emph{eventually \(\varepsilon\)-steady} iff the sequence \(a_N, a_{N + 1}, a_{N + 2}, \dots\) is \(\varepsilon\)-steady for some natural number \(N \geq 0\).
    In other words, the sequence \(a_0, a_1, a_2, \dots\) is eventually \(\varepsilon\)-steady iff there exists an \(N \geq 0\) such that \(\abs*{a_j - a_k} \leq \varepsilon\) for all \(j, k \geq N\).
\end{definition}

\setcounter{theorem}{7}
\begin{definition}[Cauchy sequences]\label{5.1.8}
    A sequence \((a_n)_{n = 0}^{\infty}\) of rational numbers is said to be a \emph{Cauchy sequence} iff for every rational \(\varepsilon > 0\), the sequence \((a_n)_{n = 0}^{\infty}\) is eventually \(\varepsilon\)-steady.
    In other words, the sequence \(a_0, a_1, a_2, \dots\) is a Cauchy sequence iff for every \(\varepsilon > 0\), there exists an \(N \geq 0\) such that \(\abs*{a_j - a_k} \leq \varepsilon\) for all \(j, k \geq N\).
\end{definition}

\begin{remark}\label{5.1.9}
    At present, the parameter \(\varepsilon\) is restricted to be a positive rational;
    we cannot take \(\varepsilon\) to be an arbitrary positive real number, because the real numbers have not yet been constructed.
    However, once we do construct the real numbers, we shall see that Definition \ref{5.1.8} will not change if we require \(\varepsilon\) to be real instead of rational.
    In other words, we will eventually prove that a sequence is eventually \(\varepsilon\)-steady for every rational \(\varepsilon > 0\) if and only if it is eventually \(\varepsilon\)-steady for every real \(\varepsilon > 0\).
    This rather subtle distinction between a rational \(\varepsilon\) and a real \(\varepsilon\) turns out not to be very important in the long run, and the reader is advised not to pay too much attention as to what type of number \(\varepsilon\) should be.
\end{remark}

\setcounter{theorem}{10}
\begin{proposition}\label{5.1.11}
    The sequence \(a_1, a_2, a_3, \dots\) defined by \(a_n \coloneqq 1 / n\) (i.e., the sequence \(1, 1 / 2, 1 / 3, \dots\)) is a Cauchy sequence.
\end{proposition}

\begin{proof}
    We have to show that for every \(\varepsilon > 0\), the sequence \(a_1, a_2, \dots\) is eventually \(\varepsilon\)-steady.
    So let \(\varepsilon > 0\) be arbitrary.
    We now have to find a number \(N \geq 1\) such that the sequence \(a_N, a_{N + 1}, \dots\) is \(\varepsilon\)-steady.
    Let us see what this means.
    This means that \(d(a_j, a_k) \leq \varepsilon\) for every \(j, k \geq N\), i.e.
    \[
        \abs*{1 / j - 1 / k} \leq \varepsilon \text{ for every } j, k \geq N.
    \]
    Now since \(j, k \geq N\), we know that \(0 < 1 / j, 1 / k \leq 1 / N\), so that
    \begin{align*}
                 &
        \begin{cases}
            0 \leq \frac{1}{j} \leq \frac{1}{N} \\
            0 \leq \frac{1}{k} \leq \frac{1}{N} \\
        \end{cases}
        \\
        \implies &
        \begin{cases}
            0 \leq \frac{1}{j} \leq \frac{1}{N}   \\
            \frac{-1}{N} \leq \frac{-1}{k} \leq 0 \\
        \end{cases}
                 & \text{(by Exercise \ref{ex 4.2.6})}                                                                \\
        \implies & \frac{-1}{N} \leq \frac{1}{j} - \frac{1}{k} \leq \frac{1}{N} & \text{(by Proposition \ref{4.2.9})} \\
        \implies & \abs*{\frac{1}{j} - \frac{1}{k}} \leq \frac{1}{N}.           & \text{(by Proposition \ref{4.3.3})}
    \end{align*}
    So in order to force \(\abs*{1 / j - 1 / k}\) to be less than or equal to \(\varepsilon\), it would be sufficient for \(1 / N\) to be less than \(\varepsilon\).
    So all we need to do is choose an \(N\) such that \(1 / N\) is less than \(\varepsilon\), or in other words that \(N\) is greater than \(1 / \varepsilon\).
    But this can be done thanks to Proposition \ref{4.4.1}.
\end{proof}

\begin{note}
    As you can see, verifying from first principles (i.e., without using any of the machinery of limits, etc.) that a sequence is a Cauchy sequence requires some effort, even for a sequence as simple as \(1 / n\).
    The part about selecting an \(N\) can be particularly difficult for beginners
    - one has to think in reverse, working out what conditions on \(N\) would suffice to force the sequence \(a_N, a_{N + 1}, a_{N + 2}, \dots\) to be \(\varepsilon\)-steady, and then finding an \(N\) which obeys those conditions.
    Later we will develop some limit laws which allow us to determine when a sequence is Cauchy more easily.
\end{note}

\begin{definition}[Bounded sequences]\label{5.1.12}
    Let \(M \geq 0\) be rational.
    A finite sequence \(a_1, a_2, \dots, a_n\) is \emph{bounded by \(M\)} iff \(\abs*{a_i} \leq M\) for all \(1 \leq i \leq n\).
    An infinite sequence \((a_n)_{n = 1}^{\infty}\) is \emph{bounded by \(M\)} iff \(\abs*{a_i} \leq M\) for all \(i \geq 1\).
    A sequence is said to be \emph{bounded} iff it is bounded by \(M\) for some rational \(M \geq 0\).
\end{definition}

\setcounter{theorem}{13}
\begin{lemma}[Finite sequences are bounded]\label{5.1.14}
    Every finite sequence \(a_1, a_2, \dots, a_n\) is bounded.
\end{lemma}

\begin{proof}
    We prove this by induction on \(n\).
    When \(n = 1\) the sequence \(a_1\) is clearly bounded, for if we choose \(M \coloneqq \abs*{a_1}\) then clearly we have \(\abs*{a_i} \leq M\) for all \(1 \leq i \leq n\).
    Now suppose that we have already proved the lemma for some \(n \geq 1\);
    we now prove it for \(n + 1\), i.e., we prove every sequence \(a_1, a_2, \dots, a_{n + 1}\) is bounded.
    By the induction hypothesis we know that \(a_1, a_2, \dots, a_n\) is bounded by some \(M \geq 0\);
    in particular, it must be bounded by \(M + \abs*{a_{n + 1}}\).
    On the other hand, \(a_{n + 1}\) is also bounded by \(M + \abs*{a_{n + 1}}\).
    Thus \(a_1, a_2, \dots, a_n, a_{n++}\) is bounded by \(M + \abs*{a_{n + 1}}\), and is hence bounded.
    This closes the induction.
\end{proof}

\begin{note}
    While this argument shows that every finite sequence is bounded, no matter how long the finite sequence is, it does not say anything about whether an infinite sequence is bounded or not;
    infinity is not a natural number.
\end{note}

\begin{lemma}[Cauchy sequences are bounded]\label{5.1.15}
    Every Cauchy sequence \((a_n)_{n = 1}^{\infty}\) is bounded.
\end{lemma}

\begin{proof}
    Let \(n \in \mathbf{N}\) and \((a_n)_{n = 1}^{\infty}\) be a rational Cauchy sequence.
    Since \((a_n)_{n = 1}^{\infty}\) is a Cauchy sequence, by Definition \ref{5.1.8} \(\forall\ \varepsilon \in \mathbf{Q}^+\), we know that \((a_n)_{n = 1}^{\infty}\) is eventually \(\varepsilon\)-steady.
    In particular, \((a_n)_{n = 1}^{\infty}\) is eventually \(1\)-steady.
    By Definition \ref{5.1.6}, \(\exists\ N \in \mathbf{Z}^+\) such that \((a_n)_{n = N}^{\infty}\) is \(1\)-steady.
    Then we can split \((a_n)_{n = 1}^{\infty}\) into two sequences:
    \begin{itemize}
        \item A finite sequence \((a_n)_{n = 1}^{N - 1}\).
              Then by Lemma \ref{5.1.14} \((a_n)_{n = 1}^{N - 1}\) is bounded by some \(M \in \mathbf{Q} \setminus \mathbf{Q}^-\).
        \item An infinite sequence \((a_n)_{n = N}^{\infty}\).
              Since \((a_n)_{n = N}^\infty\) is \(1\)-steady, we have
              \begin{align*}
                           & \forall\ j \in \mathbf{Z}^+ \land j \geq N, \abs*{a_j - a_N} \leq 1           & \text{(by Definition \ref{5.1.3})}     \\
                  \implies & \abs*{a_j - a_N} + \abs*{a_N} \leq 1 + \abs*{a_N}                             & \text{(by Proposition \ref{4.2.9}(d))} \\
                  \implies & \abs*{a_j - a_N + a_N} \leq \abs*{a_j - a_N} + \abs*{a_N} \leq 1 + \abs*{a_N} & \text{(by Proposition \ref{4.3.3}(b))} \\
                  \implies & \abs*{a_j} \leq 1 + \abs*{a_N}.
              \end{align*}
              Thus by Definition \ref{5.1.12} \((a_n)_{n = N}^\infty\) is bounded by \(1 + \abs*{a_N}\).
    \end{itemize}
    Now let \(M' = M + 1 + \abs*{a_N}\).
    Then we have
    \begin{align*}
                 & \begin{cases}
            \abs*{a_n} \leq M              & \text{if } 1 \leq n \leq N - 1 \\
            \abs*{a_n} \leq 1 + \abs*{a_N} & \text{if } n \geq N
        \end{cases}                                                        \\
        \implies & \begin{cases}
            \abs*{a_n} \leq M \leq M'              & \text{if } 1 \leq n \leq N - 1 \\
            \abs*{a_n} \leq 1 + \abs*{a_N} \leq M' & \text{if } n \geq N
        \end{cases}               & \text{(by Proposition \ref{4.2.9}(c))} \\
        \implies & \forall\ n \geq 1, \abs*{a_n} \leq M'                                             \\
        \implies & (a_n)_{n = 1}^\infty \text{ is bounded}. & \text{(by Definition \ref{5.1.12})}
    \end{align*}
\end{proof}

\exercisesection

\begin{exercise}\label{ex 5.1.1}
    Prove Lemma \ref{5.1.15}.
\end{exercise}

\begin{proof}
    See Lemma \ref{5.1.15}.
\end{proof}
\section{Equivalent Cauchy sequences}

\begin{definition}[\(\varepsilon\)-close sequences]\label{5.2.1}
Let \((a_n)_{n = 0}^{\infty}\) and \((b_n)_{n = 0}^{\infty}\) be two sequences, and let \(\varepsilon > 0\).
We say that the sequence \((a_n)_{n = 0}^{\infty}\) is \emph{\(\varepsilon\)-close} to \((b_n)_{n = 0}^{\infty}\) iff \(a_n\) is \(\varepsilon\)-close to \(b_n\) for each \(n \in \mathds{N}\).
In other words, the sequence \(a_0, a_1, a_2, \dots\) is \(\varepsilon\)-close to the sequence \(b_0, b_1, b_2, \dots\) iff \(|a_n - b_n| \leq \varepsilon\) for all \(n = 0, 1, 2, \dots\).
\end{definition}

\setcounter{theorem}{2}
\begin{definition}[\(Eventually \varepsilon\)-close sequences]\label{5.2.3}
Let \((a_n)_{n = 0}^{\infty}\) and \((b_n)_{n = 0}^{\infty}\) be two sequences, and let \(\varepsilon > 0\).
We say that the sequence \((a_n)_{n = 0}^{\infty}\) is \emph{eventually \(\varepsilon\)-close} to \((b_n)_{n = 0}^{\infty}\) iff there exists an \(N \geq 0\) such that the sequences \((a_n)_{n = N}^{\infty}\) and \((b_n)_{n = N}^{\infty}\) are \(\varepsilon\)-close.
In other words, \(a_0, a_1, a_2, \dots\) is eventually \(\varepsilon\)-close to \(b_0, b_1, b_2, \dots\) iff there exists an \(N \geq 0\) such that \(|a_n - b_n| \leq \varepsilon\) for all \(n \geq N\).
\end{definition}

\begin{remark}\label{5.2.4}
Again, the notations for \(\varepsilon\)-close sequences and eventually \(\varepsilon\)-close sequences are not standard in the literature, and we will not use them outside of this section.
\end{remark}

\setcounter{theorem}{5}
\begin{definition}[Equivalent sequences]\label{5.2.6}
Two sequences \((a_n)_{n = 0}^{\infty}\) and \((b_n)_{n = 0}^{\infty}\) are \emph{equivalent} iff for each rational \(\varepsilon > 0\), the sequences \((a_n)_{n = 0}^{\infty}\) and \((b_n)_{n = 0}^{\infty}\) are eventually \(\varepsilon\)-close.
In other words, \(a_0, a_1, a_2, \dots\) and \(b_0, b_1, b_2, \dots\) are equivalent iff for every rational \(\varepsilon > 0\), there exists an \(N \geq 0\) such that \(|a_n - b_n| \leq \varepsilon\) for all \(n \geq N\).
\end{definition}

\begin{remark}\label{5.2.7}
As with Definition \ref{5.1.8}, the quantity \(\varepsilon > 0\) is currently restricted to be a positive rational, rather than a positive real.
However, we shall eventually see that it makes no difference whether \(\varepsilon\) ranges over the positive rationals or positive reals.
\end{remark}

\begin{proposition}\label{5.2.8}
Let \((a_n)_{n = 1}^{\infty}\) and \((b_n)_{n = 1}^{\infty}\) be the sequences \(a_n = 1 + 10^{-n}\) and \(b_n = 1 - 10^{-n}\).
Then the sequences \(a_n, b_n\) are equivalent.
\end{proposition}

\begin{proof}
We need to prove that for every \(\varepsilon > 0\), the two sequences \((a_n)_{n = 1}^{\infty}\) and \((b_n)_{n = 1}^{\infty}\) are eventually \(\varepsilon\)-close to each other.
So we fix an \(\varepsilon > 0\).
We need to find an \(N > 0\) such that \((a_n)_{n = 1}^{\infty}\) and \((b_n)_{n = 1}^{\infty}\) are \(\varepsilon\)-close;
in other words, we need to find an \(N > 0\) such that
\[
    |a_n - b_n| \leq \varepsilon \text{ for all } n \geq N.
\]
However, we have
\[
    |a_n - b_n| = |(1 + 10^{-n}) - (1 - 10^{-n})| = 2 \times 10^{-n}.
\]
Since \(10^{-n}\) is a decreasing function of \(n\) (i.e., \(10^{-m} < 10^{-n}\) whenever \(m > n\);
this is easily proven by induction), and \(n \geq N\), we have \(2 \times 10^{-n} \leq 2 \times 10^{-N}\).
Thus we have
\[
    |a_n - b_n| \leq 2 \times 10^{-N} \text{ for all } n \geq N.
\]
Thus in order to obtain \(|a_n - b_n| \leq \varepsilon\) for all \(n \geq N\), it will be sufficient to choose \(N\) so that \(2 \times 10^{-N} \leq \varepsilon\).
This is easy to do using logarithms, but we have not yet developed logarithms yet, so we will use a cruder method.
First, we observe \(10^N\) is always greater than \(N\) for any \(N \geq 1\) (see Exercise \ref{ex 4.3.5}).
Thus \(10^{-N} \leq 1 / N\), and so \(2 \times 10^{-N} \leq 2 / N\).
Thus to get \(2 \times 10^{-N} \leq \varepsilon\), it will suffice to choose \(N\) so that \(2 / N \leq \varepsilon\), or equivalently that \(N \geq 2 / \varepsilon\).
But by Proposition \ref{4.4.1} we can always choose such an \(N\), and the claim follows.
\end{proof}

\begin{remark}\label{5.2.9}
Proposition \ref{5.2.8}, in decimal notation, asserts that
\[
    1.0000 \dots = 0.9999 \dots.
\]
\end{remark}

\exercisesection

\begin{exercise}\label{ex 5.2.1}
Show that if \((a_n)_{n = 1}^{\infty}\) and \((b_n)_{n = 1}^{\infty}\) are equivalent sequences of rationals, then \((a_n)_{n = 1}^{\infty}\) is a Cauchy sequence if and only if \((b_n)_{n = 1}^{\infty}\) is a Cauchy sequence.
\end{exercise}

\begin{proof}
Let \((a_n)_{n = 1}^{\infty}\) be a Cauchy sequence.
By Definition \ref{5.1.8}, \(\forall\ \varepsilon > 0\) and \(\varepsilon \in \mathds{Q}\), \(\exists\ N_1 \geq 1\) and \(N_1 \in \mathds{N}\) such that
\[
    |a_j - a_k| \leq \varepsilon \ \forall\ j, k \geq N_1
\]
where \(j, k \in \mathds{N}\).
Since \((a_n)_{n = 1}^{\infty}\) and \((b_n)_{n = 1}^{\infty}\) are equivalent sequences, by Definition \ref{5.2.6}, \(\forall\ \varepsilon > 0\) and \(\varepsilon \in \mathds{Q}\), \(\exists\ N_2 \geq 1\) and \(N_2 \in \mathds{N}\) such that
\[
    |a_m - b_m| \leq \varepsilon \ \forall\ m \geq N_2
\]
where \(m \in \mathds{N}\).
Let \(N = N_1 + N_2\).
Since \(N > N_1\) and \(N > N_2\) by Proposition \ref{2.2.11}, we have
\[
    |a_j - a_k| \leq \varepsilon \ \forall\ j, k \geq N
\]
and
\[
    |a_m - b_m| \leq \varepsilon \ \forall\ m \geq N.
\]
Since \(\varepsilon > 0\), \(\varepsilon / 3 > 0\) by Additional Corollary \ref{ac 4.2.5}, then we have
\[
    |a_j - a_k| \leq \varepsilon / 3 \ \forall\ j, k \geq N
\]
and
\[
    |a_m - b_m| \leq \varepsilon / 3 \ \forall\ m \geq N.
\]
So \(\forall\ \varepsilon > 0\) and \(\forall\ j, k \geq N\),
\begin{align*}
|b_j - b_k| &= |b_j + (-b_k)| \\
&= |(b_j + (-b_k)) + 0| & \text{(by Proposition \ref{4.2.4})} \\
&= |(b_j + (-b_k)) + ((-a_k) + a_k)| & \text{(by Proposition \ref{4.2.4})} \\
&= |b_j + ((-b_k) + ((-a_k) + a_k))| & \text{(by Proposition \ref{4.2.4})} \\
&= |b_j + (((-a_k) + a_k) + (-b_k))| & \text{(by Proposition \ref{4.2.4})} \\
&= |b_j + ((-a_k) + (a_k + (-b_k)))| & \text{(by Proposition \ref{4.2.4})} \\
&= |(b_j + (-a_k)) + (a_k + (-b_k))| & \text{(by Proposition \ref{4.2.4})} \\
&= |0 + ((b_j + (-a_k)) + (a_k + (-b_k)))| & \text{(by Proposition \ref{4.2.4})} \\
&= |(a_j + (-a_j)) + ((b_j + (-a_k)) + (a_k + (-b_k)))| & \text{(by Proposition \ref{4.2.4})} \\
&= |((a_j + (-a_j)) + (b_j + (-a_k))) + (a_k + (-b_k))| & \text{(by Proposition \ref{4.2.4})} \\
&= |(a_j + (((-a_j) + b_j) + (-a_k))) + (a_k + (-b_k))| & \text{(by Proposition \ref{4.2.4})} \\
&= |(a_j + ((-a_k) + ((-a_j) + b_j))) + (a_k + (-b_k))| & \text{(by Proposition \ref{4.2.4})} \\
&= |((a_j + (-a_k)) + ((-a_j) + b_j)) + (a_k + (-b_k))| & \text{(by Proposition \ref{4.2.4})} \\
&\leq |(a_j + (-a_k)) + ((-a_j) + b_j)| + |a_k + (-b_k)| & \text{(by Proposition \ref{4.3.3})} \\
&\leq (|a_j + (-a_k)| + |(-a_j) + b_j|) + |a_k + (-b_k)| & \text{(by Proposition \ref{4.3.3})} \\
&= (|a_j + (-a_k)| + |a_j + (-b_j)|) + |a_k + (-b_k)| & \text{(by Proposition \ref{4.3.3})} \\
&= (|a_j - a_k| + |a_j - b_j|) + |a_k - b_k| \\
&\leq (\varepsilon / 3 + \varepsilon / 3) + \varepsilon / 3 & \text{(by Proposition \ref{4.3.3})} \\
&= \varepsilon.
\end{align*}
By Definition \ref{5.1.8}, \((b_n)_{n = 1}^{\infty}\) is also a Cauchy sequence.
Similar proof can show that \((b_n)_{n = 1}^{\infty}\) is a Cauchy sequence implies \((a_n)_{n = 1}^{\infty}\) is also a Cauchy sequence.
Thus we finished the proof.
\end{proof}

\begin{exercise}\label{ex 5.2.2}
Let \(\varepsilon > 0\).
Show that if \((a_n)_{n = 1}^{\infty}\) and \((b_n)_{n = 1}^{\infty}\) are eventually \(\varepsilon\)-close, then \((a_n)_{n = 1}^{\infty}\) is bounded if and only if \((b_n)_{n = 1}^{\infty}\) is bounded.
\end{exercise}