\section{Convergence and limit laws}\label{sec 6.1}

\begin{definition}[Distance between two real numbers]\label{6.1.1}
Given two real numbers \(x\) and \(y\), we define their distance \(d(x, y)\) to be \(d(x, y) \coloneqq \abs*{x - y}\).
\end{definition}

\begin{note}
Clearly Definition \ref{6.1.1} is consistent with Definition \ref{4.3.2}.
Further, Proposition \ref{4.3.3} works just as well for real numbers as it does for rationals, because the real numbers obey all the rules of algebra that the rationals do.
\end{note}

\begin{definition}[\(\varepsilon\)-close real numbers]\label{6.1.2}
Let \(\varepsilon > 0\) be a real number.
We say that two real numbers \(x, y\) are \emph{\(\varepsilon\)-close} iff we have \(d(y, x) \leq \varepsilon\).
\end{definition}

\begin{note}
Again, it is clear that Definition \ref{6.1.2} is consistent with Definition \ref{4.3.4}.
\end{note}

\begin{note}
Now let \((a_n)_{n = m}^\infty\) be a sequence of real numbers;
i.e., we assign a real number \(a_n\) for every integer \(n \geq m\).
The starting index \(m\) is some integer;
usually this will be \(1\), but in some cases we will start from some index other than \(1\).
(The choice of label used to index this sequence is unimportant; we could use for instance \((a_k)_{k = m}^{\infty}\) and this would represent exactly the same sequence as \((a_n)_{n = m}^{\infty}\).
We can define the notion of a Cauchy sequence in the same manner as before.
\end{note}

\begin{definition}[Cauchy sequences of reals]\label{6.1.3}
Let \(\varepsilon > 0\) be a real number.
A sequence \((a_n)_{n = N}^\infty\) of real numbers starting at some integer index \(N\) is said to be \emph{\(\varepsilon\)-steady} iff \(a_j\) and \(a_k\) are \(\varepsilon\)-close for every \(j, k \geq N\).
A sequence \((a_n)_{n = m}^\infty\) starting at some integer index \(m\) is said to be \emph{eventually \(\varepsilon\)-steady} iff there exists an \(N \geq m\) such that \((a_n)_{n = N}^\infty\) is \(\varepsilon\)-steady.
We say that \((a_n)_{n = m}^\infty\) is a \emph{Cauchy sequence} iff it is eventually \(\varepsilon\)-steady for every \(\varepsilon > 0\).
\end{definition}

\begin{note}
To put it another way, a sequence \((a_n)_{n = m}^\infty\) of real numbers is a Cauchy sequence if, for every real \(\varepsilon > 0\), there exists an \(N \geq m\) such that \(\abs*{a_n - a_n'} \leq \varepsilon\) for all \(n, n' \geq N\).
These definitions are consistent with the corresponding definitions for rational numbers (Definitions \ref{5.1.3}, \ref{5.1.6}, \ref{5.1.8}), although verifying consistency for Cauchy sequences takes a little bit of care.
\end{note}

\begin{proposition}\label{6.1.4}
Let \((a_n)_{n = m}^\infty\) be a sequence of rational numbers starting at some integer index \(m\).
Then \((a_n)_{n = m}^\infty\) is a Cauchy sequence in the sense of Definition \ref{5.1.8} if and only if it is a Cauchy sequence in the sense of Definition \ref{6.1.3}.
\end{proposition}

\begin{proof}
Suppose first that \((a_n)_{n = m}^\infty\) is a Cauchy sequence in the sense of Definition \ref{6.1.3};
then it is eventually \(\varepsilon\)-steady for every \emph{real} \(\varepsilon > 0\).
In particular, it is eventually \(\varepsilon\)-steady for every \emph{rational} \(\varepsilon > 0\), which makes it a Cauchy sequence in the sense of Definition \ref{5.1.8}.

Now suppose that \((a_n)_{n = m}^\infty\) is a Cauchy sequence in the sense of Definition \ref{5.1.8};
then it is eventually \(\varepsilon\)-steady for every \emph{rational} \(\varepsilon > 0\).
If \(\varepsilon > 0\) is a real number, then there exists a \emph{rational} \(\varepsilon' > 0\) which is smaller than \(\varepsilon\), by Proposition \ref{5.4.12}.
Since \(\varepsilon'\) is rational, we know that \((a_n)_{n = m}^\infty\) is eventually \(\varepsilon'\)-steady;
since \(\varepsilon' < \varepsilon\), this implies that \((a_n)_{n = m}^\infty\) is eventually \(\varepsilon\)-steady.
Since \(\varepsilon\) is an arbitrary positive real number, we thus see that \((a_n)_{n = m}^\infty\) is a Cauchy sequence in the sense of Definition \ref{6.1.3}.
\end{proof}

\begin{note}
Because of Proposition \ref{6.1.4}, we will no longer care about the distinction between Definition \ref{5.1.8} and Definition \ref{6.1.3}, and view the concept of a Cauchy sequence as a single unified concept.
\end{note}

\begin{definition}[Convergence of sequences]\label{6.1.5}
Let \(\varepsilon > 0\) be a real number, and let \(L\) be a real number.
A sequence \((a_n)_{n = N}^\infty\) of real numbers is said to be \emph{\(\varepsilon\)-close to \(L\)} iff \(a_n\) is \(\varepsilon\)-close to \(L\) for every \(n \geq N\), i.e., we have \(\abs*{a_n - L} \leq \varepsilon\) for every \(n \geq N\).
We say that a sequence \((a_n)_{n = m}^\infty\) is \emph{eventually \(\varepsilon\)-close to \(L\)} iff there exists an \(N \geq m\) such that \((a_n)_{n = N}^\infty\) is \(\varepsilon\)-close to \(L\).
We say that a sequence \((a_n)_{n = m}^\infty\) \emph{converges to \(L\)} iff it is eventually \(\varepsilon\)-close to \(L\) for every real \(\varepsilon > 0\).
\end{definition}

\setcounter{theorem}{6}
\begin{proposition}[Uniqueness of limits]\label{6.1.7}
Let \((a_n)_{n = m}^\infty\) be a real sequence starting at some integer index \(m\), and let \(L \neq L'\) be two distinct real numbers.
Then it is not possible for \((a_n)_{n = m}^\infty\) to converge to \(L\) while also converging to \(L'\).
\end{proposition}

\begin{proof}
Suppose for sake of contradiction that \((a_n)_{n = m}^\infty\) was converging to both \(L\) and \(L'\).
Let \(\varepsilon = \abs*{L - L'} / 3\);
note that \(\varepsilon\) is positive since \(L \neq L'\).
Since \((a_n)_{n = m}^\infty\) converges to \(L\), we know that \((a_n)_{n = m}^\infty\) is eventually \(\varepsilon\)-close to \(L\);
thus there is an \(N \geq m\) such that \(d(a_n, L) \leq \varepsilon\) for all \(n \geq N\).
Similarly, there is an \(M \geq m\) such that \(d(a_n, L') \leq \varepsilon\) for all \(n \geq M\).
In particular, if we set \(n \coloneqq \max(N, M)\), then we have \(d(a_n, L) \leq \varepsilon\) and \(d(a_n, L') \leq \varepsilon\), hence by the triangle inequality \(d(L, L') \leq 2\varepsilon = 2\abs*{L - L'} / 3\).
But then we have \(\abs*{L - L'} \leq 2\abs*{L - L'} / 3\), which contradicts the fact that \(\abs*{L - L'} > 0\).
Thus it is not possible to converge to both \(L\) and \(L'\).
\end{proof}

\begin{definition}[Limits of sequences]\label{6.1.8}
If a sequence \((a_n)_{n = m}^\infty\) converges to some real number \(L\), we say that \((a_n)_{n = m}^\infty\) is \emph{convergent} and that its \emph{limit} is \(L\);
we write
\[
    L = \lim_{n \to \infty} a_n
\]
to denote this fact.
If a sequence \((a_n)_{n = m}^\infty\) is not converging to any real number \(L\), we say that the sequence \((a_n)_{n = m}^\infty\) is \emph{divergent} and we leave \(\lim_{n \to \infty} a_n\) undefined.
\end{definition}

\begin{note}
Proposition \ref{6.1.7} ensures that a sequence can have at most one limit.
Thus, if the limit exists, it is a single real number, otherwise it is undefined.
\end{note}

\begin{remark}\label{6.1.9}
The notation \(\lim_{n \to \infty} a_n\) does not give any indication about the starting index \(m\) of the sequence, but the starting index is irrelevant.
Thus in the rest of this discussion we shall not be too careful as to where these sequences start, as we shall be mostly focused on their limits.
\end{remark}

\begin{note}
We sometimes use the phrase ``\(a_n \to x\) as \(n \to \infty\)'' as an alternate way of writing the statement ``\((a_n)_{n = m}^\infty\) converges to \(x\)''.
Bear in mind, though, that the individual statements \(a_n \to x\) and \(n \to \infty\) do not have any rigorous meaning;
this phrase is just a convention, though of course a very suggestive one.
\end{note}

\begin{remark}\label{6.1.10}
The exact choice of letter used to denote the index (in this case \(n\)) is irrelevant:
the phrase \(\lim_{n \to \infty} a_n\) has exactly the same meaning as \(\lim_{k \to \infty} a_k\), for instance.
Sometimes it will be convenient to change the label of the index to avoid conflicts of notation;
for instance, we might want to change \(n\) to \(k\) because \(n\) is simultaneously being used for some other purpose, and we want to reduce confusion.
\end{remark}

\begin{proposition}\label{6.1.11}
We have \(\lim_{n \to \infty} 1 / n = 0\).
\end{proposition}

\begin{proof}
We have to show that the sequence \((a_n)_{n = 1}^\infty\) converges to \(0\), where \(a_n \coloneqq 1 / n\).
In other words, for every \(\varepsilon > 0\), we need to show that the sequence \((a_n)_{n = 1}^\infty\) is eventually \(\varepsilon\)-close to \(0\).
So, let \(\varepsilon > 0\) be an arbitrary real number.
We have to find an \(N\) such that \(\abs*{a_n - 0} \leq \varepsilon\) for every \(n \geq N\).
But if \(n \geq N\), then
\[
    \abs*{a_n - 0} = \abs*{1 / n - 0} = 1 / n \leq 1 / N.
\]
Thus, if we pick \(N > 1 / \varepsilon\) (which we can do by the Archimedean principle), then \(1 / N < \varepsilon\), and so \((a_n)_{n = 1}^\infty\) is \(\varepsilon\)-close to \(0\).
Thus \((a_n)_{n = 1}^\infty\) is eventually \(\varepsilon\)-close to \(0\).
Since \(\varepsilon\) was arbitrary, \((a_n)_{n = 1}^\infty\) converges to \(0\).
\end{proof}

\begin{proposition}[Convergent sequences are Cauchy]\label{6.1.12}
Suppose that \((a_n)_{n = m}^\infty\) is a convergent sequence of real numbers.
Then \((a_n)_{n = m}^\infty\) is also a Cauchy sequence.
\end{proposition}

\begin{proof}
Let \((a_n)_{n = m}^\infty\) converges to \(L\).
Then by Definition \ref{6.1.5}, \(\forall\ \varepsilon \in \mathds{R}\) and \(\varepsilon > 0\), \(\exists\ N \in \mathds{N}\) and \(N \geq m\) such that \(\abs*{a_n - L} \leq \varepsilon \ \forall\ n \geq N\).
In particular, we have \(\abs*{a_n - L} \leq \varepsilon / 2\).
Let \(n' \in \mathds{N}\) and \(n' \geq N\).
So we have
\begin{align*}
\abs*{a_n - a_{n'}} &= \abs*{a_n - a_{n'} + L - L} \\
&= \abs*{(a_n - L) + (L - a_{n'})} \\
&\leq \abs*{a_n - L} + \abs*{L - a_{n'}} \\
&= \abs*{a_n - L} + \abs*{a_{n'} - L} \\
&\leq \varepsilon / 2 + \varepsilon / 2 \\
&= \varepsilon.
\end{align*}
Thus \((a_n)_{n = m}^\infty\) is a Cauchy sequence.
\end{proof}

\setcounter{theorem}{14}
\begin{proposition}[Formal limits are genuine limits]\label{6.1.15}
Suppose that \((a_n)_{n = 1}^\infty\) is a Cauchy sequence of rational numbers.
Then \((a_n)_{n = 1}^\infty\) converges to \(\text{LIM}_{n \to \infty} a_n\), i.e.
\[
    \text{LIM}_{n \to \infty} a_n = \lim_{n \to \infty} a_n.
\]
\end{proposition}

\begin{proof}
Let \((a_n)_{n = m}^\infty\) be a Cauchy sequence of rationals, and write \(L \coloneqq \text{LIM}_{n \to \infty} a_n\).
We have to show that \((a_n)_{n = m}^\infty\) converges to \(L\).
Suppose for sake of contradiction that sequence \(a_n\) is not eventually \(\varepsilon\)-close to \(L\).
Then \(\exists\ \varepsilon \in \mathds{Q}\) and \(\varepsilon > 0\) such that \(\forall\ n \in \mathds{N}\) and \(n \geq m\), \(\abs*{a_n - L} > \varepsilon\).
\begin{enumerate}
    \item If \(a_n - L > 0\), then \(\abs*{a_n - L} = a_n - L > \varepsilon\), so \(a_n > L + \varepsilon\).
    By Exercise \ref{ex 5.4.8}, we have \(\text{LIM}_{n \to \infty} a_n > L + \varepsilon\).
    But \(L > L + \varepsilon \implies \varepsilon < 0\), a contradiction.
    \item If \(a_n - L < 0\), then \(\abs*{a_n - L} = L - a_n > \varepsilon\), so \(a_n < L - \varepsilon\).
    By Exercise \ref{ex 5.4.8}, we have \(\text{LIM}_{n \to \infty} a_n < L - \varepsilon\).
    But \(L < L - \varepsilon \implies \varepsilon < 0\), a contradiction.
\end{enumerate}
Thus such \(\varepsilon\) does not exist.
Therefor we have \(a_n\) is eventually \(\varepsilon\)-close to \(L\), which means \(\lim_{n \to \infty} a_n = L\).
Thus we conclude that \(\text{LIM}_{n \to \infty} a_n = \lim_{n \to \infty} a_n\).
\end{proof}

\begin{definition}[Bounded sequences]\label{6.1.16}
A sequence \((a_n)_{n = m}^\infty\) of real numbers is \emph{bounded by} a real number \(M\) iff we have \(\abs*{a_n} \leq M\) for all \(n \geq m\).
We say that \((a_n)_{n = m}^\infty\) is bounded iff it is \emph{bounded} by \(M\) for some real number \(M > 0\).
\end{definition}

\begin{note}
Definition \ref{6.1.16} is consistent with Definition \ref{5.1.12}.
\end{note}

\begin{note}
Recall from Lemma \ref{5.1.15} that every Cauchy sequence of rational numbers is bounded.
An inspection of the proof of that Lemma shows that the same argument works for real numbers;
every Cauchy sequence of real numbers is bounded.
\end{note}

\begin{corollary}\label{6.1.17}
Every convergent sequence of real numbers is bounded.
\end{corollary}

\begin{proof}
From Proposition \ref{6.1.12} we have every convergent sequence of real numbers is a Cauchy sequence.
And by Lemma \ref{5.1.15}, every Cauchy sequence is bounded.
Thus every convergent sequence of real numbers is bounded.
\end{proof}

\setcounter{theorem}{18}
\begin{theorem}[Limit Laws]\label{6.1.19}
Let \((a_n)_{n = m}^\infty\) and \((b_n)_{n = m}^\infty\) be convergent sequences of real numbers, and let \(x, y\) be the real numbers \(x \coloneqq \lim_{n \to \infty} a_n\) and \(y \coloneqq \lim_{n \to \infty} b_n\).
\begin{enumerate}
    \item The sequence \((a_n + b_n)_{n = m}^\infty\) converges to \(x + y\);
    in other words,
    \[
        \lim_{n \to \infty} (a_n + b_n) = \lim_{n \to \infty} a_n + \lim_{n \to \infty} b_n.
    \]
    \item The sequence \((a_n b_n)_{n = m}^\infty\) converges to \(xy\);
    in other words,
    \[
        \lim_{n \to \infty} (a_n b_n) = (\lim_{n \to \infty} a_n)(\lim_{n \to \infty} b_n).
    \]
    \item For any real number \(c\), the sequence \((c a_n)_{n = m}^\infty\) converges to \(cx\);
    in other words,
    \[
        \lim_{n \to \infty} (c a_n) = c(\lim_{n \to \infty} a_n).
    \]
    \item The sequence \((a_n - b_n)_{n = m}^\infty\) converges to \(x - y\);
    in other words,
    \[
        \lim_{n \to \infty} (a_n - b_n) = \lim_{n \to \infty} a_n - \lim_{n \to \infty} b_n.
    \]
    \item Suppose that \(y \neq 0\), and that \(b_n \neq 0\) for all \(n \geq m\).
    Then the sequence \((b_n^{-1})_{n = m}^\infty\) converges to \(y^{-1}\);
    in other words,
    \[
        \lim_{n \to \infty} b_n^{-1} = (\lim_{n \to \infty} b_n)^{-1}.
    \]
    \item Suppose that \(y \neq 0\), and that \(b_n \neq 0\) for all \(n \geq m\).
    Then the sequence \((a_n / b_n)_{n = m}^\infty\) converges to \(x / y\);
    in other words,
    \[
        \lim_{n \to \infty} \frac{a_n}{b_n} = \frac{\lim_{n \to \infty} a_n}{\lim_{n \to \infty} b_n}.
    \]
    \item The sequence \((\max(a_n, b_n))_{n = m}^\infty\) converges to \(\max(x, y)\);
    in other words,
    \[
        \lim_{n \to \infty} \max(a_n, b_n) = \max(\lim_{n \to \infty} a_n, \lim_{n \to \infty} b_n).
    \]
    \item The sequence \((\min(a_n, b_n))_{n = m}^\infty\) converges to \(\min(x, y)\);
    in other words,
    \[
        \lim_{n \to \infty} \min(a_n, b_n) = \min(\lim_{n \to \infty} a_n, \lim_{n \to \infty} b_n).
    \]
\end{enumerate}
\end{theorem}

\begin{proof}{(a)}
By Definition \ref{6.1.8} we have \(\forall\ \varepsilon \in \mathds{R}\) and \(\varepsilon > 0\), \(\exists\ N_a \in \mathds{N}\) such that \(\abs*{a_n - x} \leq \varepsilon \ \forall\ n \geq N_a\).
Similarly \(\forall\ \varepsilon \in \mathds{R}\) and \(\varepsilon > 0\), \(\exists\ N_b \in \mathds{N}\) such that \(\abs*{b_n - y} \leq \varepsilon \ \forall\ n \geq N_b\).
Let \(N = \max(N_a, N_b)\).
So we have \(\abs*{a_n - x} \leq \varepsilon \ \forall\ n \geq N\) and \(\abs*{b_n - y} \leq \varepsilon \ \forall\ n \geq N\).
In particular, we have \(\abs*{a_n - x} \leq \varepsilon / 2\) and \(\abs*{b_n - y} \leq \varepsilon / 2\).
Then we have
\begin{align*}
\abs*{a_n + b_n - (x + y)} &= \abs*{(a_n - x) + (b_n - y)} \\
&\leq \abs*{a_n - x} + \abs*{b_n - y} \\
&\leq \varepsilon / 2 + \varepsilon / 2 \\
&= \varepsilon.
\end{align*}
Thus by Definition \ref{6.1.5}, \((a_n + b_n)_{n = m}^\infty\) converges to \(x + y\).
And by Definition \ref{6.1.8} we have \(\lim_{n \to \infty} (a_n + b_n) = x + y = \lim_{n \to \infty} a_n + \lim_{n \to \infty} b_n\).
\end{proof}

\begin{proof}{(b)}
By Definition \ref{6.1.8} we have \(\forall\ \varepsilon \in \mathds{R}\) and \(\varepsilon > 0\), \(\exists\ N_a \in \mathds{N}\) such that \(\abs*{a_n - x} \leq \varepsilon \ \forall\ n \geq N_a\).
Similarly \(\forall\ \varepsilon \in \mathds{R}\) and \(\varepsilon > 0\), \(\exists\ N_b \in \mathds{N}\) such that \(\abs*{b_n - y} \leq \varepsilon \ \forall\ n \geq N_b\).
Let \(N = \max(N_a, N_b)\).
So we have \(\abs*{a_n - x} \leq \varepsilon \ \forall\ n \geq N\) and \(\abs*{b_n - y} \leq \varepsilon \ \forall\ n \geq N\).
By Corollary \ref{6.1.17}, \(\exists\ A, B \in \mathds{R}\) and \(A, B > 0\) such that \(\abs*{a_n} \leq A\) and \(\abs*{b_n} \leq B\), \(\forall\ n \in \mathds{N}\).
Since \(A, B > 0\), we have \(\abs*{a_n - x} \leq \varepsilon / 2B\) and \(\abs*{b_n - y} \leq \varepsilon / 2A\).
Then we have
\begin{align*}
\abs*{a_n b_n - x y} &= \abs*{a_n b_n - x y + x b_n - x b_n} \\
&= \abs*{a_n b_n - x b_n + x b_n - x y} \\
&= \abs*{b_n(a_n - x) + x(b_n - y)} \\
&\leq \abs*{b_n(a_n - x)} + \abs*{x(b_n - y)} \\
&= \abs*{b_n}\abs*{a_n - x} + \abs*{x}\abs*{b_n - y} \\
&\leq B \times \frac{\varepsilon}{2B} + A \times \frac{\varepsilon}{2A} \\
&= \varepsilon.
\end{align*}
Thus by Definition \ref{6.1.5}, \((a_n b_n)_{n = m}^\infty\) converges to \(x y\).
And by Definition \ref{6.1.8} we have \(\lim_{n \to \infty} (a_n b_n) = x y = (\lim_{n \to \infty} a_n)(\lim_{n \to \infty} b_n)\).
\end{proof}

\begin{proof}{(c)}
Let \((b_n)_{n = m}^\infty\) be a sequence where \(b_n = c \ \forall\ n \geq m\).
Then we have \(\lim_{n \to \infty} c = c\).
So
\begin{align*}
\lim_{n \to \infty} (c a_n) &= \lim_{n \to \infty} (b_n a_n) \\
&= (\lim_{n \to \infty} b_n)(\lim_{n \to \infty} a_n) & \text{(by Theorem \ref{6.1.19}(b))} \\
&= c(\lim_{n \to \infty} a_n).
\end{align*}
\end{proof}

\begin{proof}{(d)}
\begin{align*}
\lim_{n \to \infty} (a_n - b_n) &= \lim_{n \to \infty} (a_n + (-1)(b_n)) & \text{(by Proposition \ref{5.3.11})} \\
&= \lim_{n \to \infty} a_n + \lim_{n \to \infty} ((-1)(b_n)) & \text{(by Theorem \ref{6.1.19}(a))} \\
&= \lim_{n \to \infty} a_n + (-1)(\lim_{n \to \infty} b_n) & \text{(by Theorem \ref{6.1.19}(c))} \\
&= \lim_{n \to \infty} a_n - \lim_{n \to \infty} b_n. & \text{(by Proposition \ref{5.3.11})}
\end{align*}
\end{proof}

\begin{proof}{(e)}
We first show that \(\forall\ n \in \mathds{N}\) if \(b_n \neq 0\) and \(\lim_{n \to \infty} b_n \neq 0\), then \((b_n)_{n = m}^\infty\) is bounded away from zero.
Since \(y = \lim_{n \to \infty} b_n \neq 0\), we must have some \(M \in \mathds{R}\) and \(M > 0\) such that \(\abs*{b_n - 0} > M \ \forall\ n \in \mathds{N}\).
Otherwise we will get \(y = \lim_{n \to \infty} b_n\) and \(0 = \lim_{n \to \infty} b_n\), contradicts to Proposition \ref{6.1.7}.
Since \(\abs*{b_n - 0} = \abs*{b_n} > M > 0 \ \forall\ n \in \mathds{N}\), we conclude that \((b_n)_{n = m}^\infty\) is bounded away from zero.

Now we show that \(\lim_{n \to \infty} b_n^{-1} = (\lim_{n \to \infty} b_n)^{-1}\).
By Definition \ref{6.1.8} we have \(\forall\ \varepsilon \in \mathds{R}\) and \(\varepsilon > 0\), \(\exists\ N \in \mathds{N}\) such that \(\abs*{b_n - y} \leq \varepsilon \ \forall\ n \geq N\).
In particular, we have \(\abs*{b_n - y} \leq \varepsilon M\abs*{y}\) (\(M\) is derived from above claim).
So
\begin{align*}
\abs*{b_n^{-1} - y^{-1}} &= \abs*{\frac{1}{b_n} - \frac{1}{y}} \\
&= \abs*{\frac{y - b_n}{b_n y}} \\
&= \abs*{y - b_n}\frac{1}{\abs*{b_n}\abs*{y}} \\
&< \abs*{y - b_n}\frac{1}{M\abs*{y}} & \text{(From above claim)} \\
&\leq \varepsilon M\abs*{y}\frac{1}{M\abs*{y}} \\
&= \varepsilon.
\end{align*}
Thus by Definition \ref{6.1.5}, \((b_n^{-1})_{n = m}^\infty\) converges to \(y^{-1}\).
And by Definition \ref{6.1.8} we have \(\lim_{n \to \infty} (b_n^{-1}) = y^{-1} = (\lim_{n \to \infty} b_n)^{-1}\).
\end{proof}

\begin{proof}{(f)}
\begin{align*}
\lim_{n \to \infty} \frac{a_n}{b_n} &= \lim_{n \to \infty} a_n b_n^{-1} \\
&= (\lim_{n \to \infty} a_n)(\lim_{n \to \infty} b_n^{-1}) & \text{(by Theorem \ref{6.1.19}(b))} \\
&= (\lim_{n \to \infty} a_n)(\lim_{n \to \infty} b_n)^{-1} & \text{(by Theorem \ref{6.1.19}(e))} \\
&= \frac{\lim_{n \to \infty} a_n}{\lim_{n \to \infty} b_n}.
\end{align*}
\end{proof}

\begin{proof}{(g)}
By Definition \ref{6.1.8} we have \(\forall\ \varepsilon \in \mathds{R}\) and \(\varepsilon > 0\), \(\exists\ N_a \in \mathds{N}\) such that \(\abs*{a_n - x} \leq \varepsilon \ \forall\ n \geq N_a\).
Similarly \(\forall\ \varepsilon \in \mathds{R}\) and \(\varepsilon > 0\), \(\exists\ N_b \in \mathds{N}\) such that \(\abs*{b_n - y} \leq \varepsilon \ \forall\ n \geq N_b\).
Let \(N = \max(N_a, N_b)\).
So we have \(\abs*{a_n - x} \leq \varepsilon \ \forall\ n \geq N\) and \(\abs*{b_n - y} \leq \varepsilon \ \forall\ n \geq N\).

If \(x = y\), then we have
\begin{align*}
& (\abs*{a_n - x} < \varepsilon) \land (\abs*{b_n - x} < \varepsilon) \\
\implies & \abs*{\max(a_n, b_n) - x} < \varepsilon \\
\implies & \abs*{\max(a_n, b_n) - \max(x, y)} < \varepsilon \\
\implies & \lim_{n \to \infty} \max(a_n, b_n) = \max(x, y) = \max(\lim_{n \to \infty} a_n, \lim_{n \to \infty} b_n).
\end{align*}

If \(x \neq y\), then we have either \(x < y\) or \(x > y\).
Assume without loss of generality that \(x < y\).
Since we have \(\abs*{a_n - x} \leq \varepsilon\) and \(\abs*{b_n - y} \leq \varepsilon\) for every positive real number \(\varepsilon\), we also have \(\abs*{a_n - x} \leq (y - x) / 2\) and \(\abs*{b_n - y} \leq (y - x) / 2\).
So \(\forall\ n \geq N\), we have
\begin{align*}
& (\abs*{a_n - x} \leq \frac{y - x}{2}) \land (\abs*{b_n - y} \leq \frac{y - x}{2}) \\
\implies & (-\frac{y - x}{2} \leq a_n - x \leq \frac{y - x}{2}) \land (-\frac{y - x}{2} \leq b_n - y \leq \frac{y - x}{2}) \\
\implies & (a_n - x \leq \frac{y - x}{2}) \land (-\frac{y - x}{2} \leq b_n - y) \\
\implies & (a_n \leq \frac{y - x}{2} + x) \land (y - \frac{y - x}{2} \leq b_n) \\
\implies & (a_n \leq \frac{x + y}{2}) \land (\frac{x + y}{2} \leq b_n) \\
\implies & a_n \leq \frac{x + y}{2} \leq b_n.
\end{align*}
This means \(\forall\ n \geq N, \max(a_n, b_n) = b_n\).
Thus \(\forall\ n \geq N\), we have
\begin{align*}
& \abs*{\max(a_n, b_n) - \max(x, y)} = \abs*{b_n - y} \leq \varepsilon \\
\implies & \lim_{n \to \infty} \max(a_n, b_n) = \max(x, y) = \max(\lim_{n \to \infty} a_n, \lim_{n \to \infty} b_n).
\end{align*}
\end{proof}

\begin{proof}{(h)}
By Definition \ref{6.1.8} we have \(\forall\ \varepsilon \in \mathds{R}\) and \(\varepsilon > 0\), \(\exists\ N_a \in \mathds{N}\) such that \(\abs*{a_n - x} \leq \varepsilon \ \forall\ n \geq N_a\).
Similarly \(\forall\ \varepsilon \in \mathds{R}\) and \(\varepsilon > 0\), \(\exists\ N_b \in \mathds{N}\) such that \(\abs*{b_n - y} \leq \varepsilon \ \forall\ n \geq N_b\).
Let \(N = \max(N_a, N_b)\).
So we have \(\abs*{a_n - x} \leq \varepsilon \ \forall\ n \geq N\) and \(\abs*{b_n - y} \leq \varepsilon \ \forall\ n \geq N\).

If \(x = y\), then we have
\begin{align*}
& (\abs*{a_n - x} < \varepsilon) \land (\abs*{b_n - x} < \varepsilon) \\
\implies & \abs*{\min(a_n, b_n) - x} < \varepsilon \\
\implies & \abs*{\min(a_n, b_n) - \min(x, y)} < \varepsilon \\
\implies & \lim_{n \to \infty} \min(a_n, b_n) = \min(x, y) = \min(\lim_{n \to \infty} a_n, \lim_{n \to \infty} b_n).
\end{align*}

If \(x \neq y\), then we have either \(x < y\) or \(x > y\).
Assume without loss of generality that \(x < y\).
Since we have \(\abs*{a_n - x} \leq \varepsilon\) and \(\abs*{b_n - y} \leq \varepsilon\) for every positive real number \(\varepsilon\), we also have \(\abs*{a_n - x} \leq (y - x) / 2\) and \(\abs*{b_n - y} \leq (y - x) / 2\).
So \(\forall\ n \geq N\), we have
\begin{align*}
& (\abs*{a_n - x} \leq \frac{y - x}{2}) \land (\abs*{b_n - y} \leq \frac{y - x}{2}) \\
\implies & (-\frac{y - x}{2} \leq a_n - x \leq \frac{y - x}{2}) \land (-\frac{y - x}{2} \leq b_n - y \leq \frac{y - x}{2}) \\
\implies & (a_n - x \leq \frac{y - x}{2}) \land (-\frac{y - x}{2} \leq b_n - y) \\
\implies & (a_n \leq \frac{y - x}{2} + x) \land (y - \frac{y - x}{2} \leq b_n) \\
\implies & (a_n \leq \frac{x + y}{2}) \land (\frac{x + y}{2} \leq b_n) \\
\implies & a_n \leq \frac{x + y}{2} \leq b_n.
\end{align*}
This means \(\forall\ n \geq N, \min(a_n, b_n) = a_n\).
Thus \(\forall\ n \geq N\), we have
\begin{align*}
& \abs*{\min(a_n, b_n) - \min(x, y)} = \abs*{a_n - x} \leq \varepsilon \\
\implies & \lim_{n \to \infty} \min(a_n, b_n) = \min(x, y) = \min(\lim_{n \to \infty} a_n, \lim_{n \to \infty} b_n).
\end{align*}
\end{proof}

\exercisesection

\begin{exercise}\label{ex 6.1.1}
Let \((a_n)_{n = m}^\infty\) be a sequence of real numbers, such that \(a_{n + 1} > a_n\) for each natural number \(n\).
Prove that whenever \(n\) and \(m\) are natural numbers such that \(m > n\), then we have \(a_m > a_n\).
(We refer to these sequences as \emph{increasing} sequences.)
\end{exercise}

\begin{proof}
Let \(E = \{z \in \mathds{N} : n \leq z \leq m\}\).
Then \(E\) is a finite set since \(m - n \in \mathds{N}\) and is non-empty set since \(n, m \in E\).
Let \((a_z)_{z = n}^m\) be a sequence by mapping \(z \in E\) to \(a_z\).
So \((a_z)_{z = n}^m\) is a finite sequence, and the elements in sequence \((a_z)_{z = n}^m\) are \(\{a_n, a_{n + 1}, \dots, a_{m - 1}, a_m\}\).
By the given conditions we have \(a_{z + 1} > a_z\) for each natural number \(z\).
Thus we have \(a_n < a_{n + 1} < \dots < a_{m - 1} < a_m\), and by Proposition \ref{5.4.7} we have \(a_n < a_m\).
\end{proof}

\begin{exercise}\label{ex 6.1.2}
Let \((a_n)_{n = m}^\infty\) be a sequence of real numbers, and let \(L\) be a real number.
Show that \((a_n)_{n = m}^\infty\) converges to \(L\) if and only if, given any real \(\varepsilon > 0\), one can find an \(N \geq m\) such that \(\abs*{a_n - L} \leq \varepsilon\) for all \(n \geq N\).
\end{exercise}

\begin{proof}
\begin{align*}
& (a_n)_{n = m}^\infty \text{ converges to } L \\
\iff & (\forall\ (\varepsilon \in \mathds{R}) \land (\varepsilon > 0)), (a_n)_{n = m}^\infty \text{ is eventually } \varepsilon\text{-close to } L & \text{(by Definition \ref{6.1.5})} \\
\iff & (\forall\ (\varepsilon \in \mathds{R}) \land (\varepsilon > 0)), (\exists\ (N \in \mathds{N}) \land (N \geq m)) : \\
& (a_n)_{n = N}^\infty \text{ is } \varepsilon\text{-close to } L & \text{(by Definition \ref{6.1.5})} \\
\iff & (\forall\ (\varepsilon \in \mathds{R}) \land (\varepsilon > 0)), (\exists\ (N \in \mathds{N}) \land (N \geq m)) : \\
& \abs*{a_n - L} \leq \varepsilon \ \forall\ n \geq N. & \text{(by Definition \ref{6.1.5})}
\end{align*}
\end{proof}

\begin{exercise}\label{ex 6.1.3}
Let \((a_n)_{n = m}^\infty\) be a sequence of real numbers, let \(c\) be a real number, and let \(m' \geq m\) be an integer.
Show that \((a_n)_{n = m}^\infty\) converges to \(c\) if and only if \((a_n)_{n = m'}^\infty\) converges to \(c\).
\end{exercise}

\begin{proof}
We first show that if \((a_n)_{n = m}^\infty\) converges to \(c\), then \((a_n)_{n = m'}^\infty\) converges to \(c\).
\begin{align*}
& (a_n)_{n = m}^\infty \text{ converges to } c \\
\implies & (\forall\ (\varepsilon \in \mathds{R}) \land (\varepsilon > 0)), (a_n)_{n = m}^\infty \text{ is eventually } \varepsilon\text{-close to } c & \text{(by Definition \ref{6.1.5})} \\
\implies & (\forall\ (\varepsilon \in \mathds{R}) \land (\varepsilon > 0)), (\exists\ (N \in \mathds{N}) \land (N \geq m)) : \\
& (a_n)_{n = N}^\infty \text{ is } \varepsilon\text{-close to } c. & \text{(by Definition \ref{6.1.5})}
\end{align*}
Now we can divide into two cases:
\begin{enumerate}
    \item \(N \geq m'\).
    This means \(\forall\ \varepsilon \in \mathds{R}\) and \(\varepsilon > 0\), \(\exists\ N \in \mathds{N}\) and \(N \geq m'\) such that \((a_n)_{n = N}^\infty\) is \(\varepsilon\)-close to \(c\).
    Thus \((a_n)_{n = m'}^\infty\) converges to \(c\).
    \item \(N < m'\).
    Then we can let \(N = m'\), so \(\forall\ \varepsilon \in \mathds{R}\) and \(\varepsilon > 0\), \(\exists\ N \in \mathds{N}\) and \(N \geq m'\) such that \((a_n)_{n = N}^\infty\) is \(\varepsilon\)-close to \(c\).
    Thus \((a_n)_{n = m'}^\infty\) converges to \(c\).
\end{enumerate}
In both cases we get \((a_n)_{n = m'}^\infty\) converges to \(c\).
Thus if \((a_n)_{n = m}^\infty\) converges to \(c\), then \((a_n)_{n = m'}^\infty\) converges to \(c\).

Now we shows that if \((a_n)_{n = m'}^\infty\) converges to \(c\), then \((a_n)_{n = m}^\infty\) converges to \(c\).
\begin{align*}
& (a_n)_{n = m'}^\infty \text{ converges to } c \\
\implies & (\forall\ (\varepsilon \in \mathds{R}) \land (\varepsilon > 0)), (a_n)_{n = m}^\infty \text{ is eventually } \varepsilon\text{-close to } c & \text{(by Definition \ref{6.1.5})} \\
\implies & (\forall\ (\varepsilon \in \mathds{R}) \land (\varepsilon > 0)), (\exists\ (N \in \mathds{N}) \land (N \geq m')) : \\
& (a_n)_{n = N}^\infty \text{ is } \varepsilon\text{-close to } c & \text{(by Definition \ref{6.1.5})} \\
\implies & (\forall\ (\varepsilon \in \mathds{R}) \land (\varepsilon > 0)), (\exists\ (N \in \mathds{N}) \land (N \geq m)) : \\
& (a_n)_{n = N}^\infty \text{ is } \varepsilon\text{-close to } c & \text{(by Definition \ref{6.1.5})} \\
\implies & (a_n)_{n = m}^\infty \text{ converges to } c.
\end{align*}
Thus if \((a_n)_{n = m'}^\infty\) converges to \(c\), then \((a_n)_{n = m}^\infty\) converges to \(c\).
We conclude that \((a_n)_{n = m}^\infty\) converges to \(c\) if and only if \((a_n)_{n = m'}^\infty\) converges to \(c\).
\end{proof}

\begin{exercise}\label{ex 6.1.4}
Let \((a_n)_{n = m}^\infty\) be a sequence of real numbers, let \(c\) be a real number, and let \(k \geq 0\) be a non-negative integer.
Show that \((a_n)_{n = m}^\infty\) converges to \(c\) if and only if \((a_{n + k})_{n = m}^\infty\) converges to \(c\).
\end{exercise}

\begin{proof}
Since \((a_{n + k})_{n = m}^\infty = (a_n)_{n = m + k}^\infty\) and \(m + k \geq m\), by Exercise \ref{ex 6.1.3} we have \((a_n)_{n = m}^\infty\) converges to \(c\) if and only if \((a_n)_{n = m + k}^\infty\) converges to \(c\).
Thus \((a_n)_{n = m}^\infty\) converges to \(c\) if and only if \((a_{n + k})_{n = m}^\infty\) converges to \(c\).
\end{proof}

\begin{exercise}\label{ex 6.1.5}
Prove Proposition \ref{6.1.12}.
\end{exercise}

\begin{proof}
See Proposition \ref{6.1.12}.
\end{proof}

\begin{exercise}\label{ex 6.1.6}
Prove Proposition \ref{6.1.15}.
\end{exercise}

\begin{proof}
See Proposition \ref{6.1.15}.
\end{proof}

\begin{exercise}\label{ex 6.1.7}
Show that Definition \ref{6.1.16} is consistent with Definition \ref{5.1.12}
(i.e., prove an analogue of Proposition \ref{6.1.4} for bounded sequences instead of Cauchy sequences).
\end{exercise}

\begin{proof}
Suppose first that \((a_n)_{n = m}^\infty\) is a bounded sequence in the sense of Definition \ref{6.1.16};
then \(\exists\ M \in \mathds{R}\) and \(M > 0\) such that \(\abs*{a_n} \leq M \ \forall\ n \geq m\).
By Proposition \ref{5.4.12}, \(\exists\ N \in \mathds{R}\) and \(N > 0\) such that \(M \leq N\).
Since \(N \in \mathds{N}\), we also have \(N \in \mathds{Q}\).
Thus \(\abs*{a_n} \leq N \ \forall\ n \geq m\), which makes it a bounded sequence in the sense of Definition \ref{5.1.12}.

Now suppose that \((a_n)_{n = m}^\infty\) is a bounded sequence in the sense of Definition \ref{5.1.12};
then \(\exists\ M \in \mathds{Q}\) and \(M > 0\) such that \(\abs*{a_n} \leq M \ \forall\ n \geq m\).
Since \(M\) is also a real number, we see that \((a_n)_{n = m}^\infty\) is a bounded sequence in the sense of Definition \ref{6.1.16}.
\end{proof}

\begin{exercise}\label{ex 6.1.8}
Proof Theorem \ref{6.1.19}.
\end{exercise}

\begin{proof}
See Theorem \ref{6.1.19}.
\end{proof}

\begin{exercise}\label{ex 6.1.9}
Explain why Theorem \ref{6.1.19}(f) fails when the limit of the denominator is \(0\).
\end{exercise}

\begin{proof}
Suppose for sake of contradiction that Theorem \ref{6.1.19}(f) works when denominator is \(0\).
Let \((a_n)_{n = 1}^\infty = (b_n)_{n = 1}^\infty = 1 / n\).
So we have
\[
    \lim_{n \to \infty} a_n / b_n = \lim_{n \to \infty} \frac{1 / n}{1 / n} = \lim_{n \to \infty} 1 = 1.
\]
But by Proposition \ref{6.1.11} we also have
\[
    \frac{\lim_{n \to \infty} a_n}{\lim_{n \to \infty} b_n} = \frac{0}{0}
\]
which is undefined.
Thus Theorem \ref{6.1.19}(f) fails when denominator is \(0\).
\end{proof}

\begin{exercise}\label{ex 6.1.10}
Show that the concept of equivalent Cauchy sequence, as defined in Definition \ref{5.2.6}, does not change if \(\varepsilon\) is required to be positive real instead of positive rational.
More precisely, if \((a_n)_{n = 0}^\infty\) and \((b_n)_{n = 0}^\infty\) are sequences of reals, show that \((a_n)_{n = 0}^\infty\) and \((b_n)_{n = 0}^\infty\) are eventually \(\varepsilon\)-close for every rational \(\varepsilon > 0\) if and only if they are eventually \(\varepsilon\)-close for every real \(\varepsilon > 0\).
\end{exercise}

\begin{proof}
Suppose first that \((a_n)_{n = 0}^\infty\) and \((b_n)_{n = 0}^\infty\) are eventually \(\varepsilon\)-close for every rational \(\varepsilon > 0\).
Then we have \(\abs*{a_n - b_n} \leq \varepsilon\).
If \(\varepsilon' \in \mathds{R}\) and \(\varepsilon' > 0\), then by Proposition \ref{5.4.12} \(\exists\ q \in \mathds{Q}\) such that \(0 < q \leq \varepsilon'\).
Since \(q \in \mathds{Q}\) and \(q > 0\), we have \(\abs*{a_n - b_n} \leq q \leq \varepsilon'\).
Because \(\varepsilon'\) can be arbitrary positive real number, we have \((a_n)_{n = 0}^\infty\) and \((b_n)_{n = 0}^\infty\) are eventually \(\varepsilon'\)-close for every real \(\varepsilon' > 0\).

Now suppose that \((a_n)_{n = 0}^\infty\) and \((b_n)_{n = 0}^\infty\) are eventually \(\varepsilon\)-close for every real \(\varepsilon > 0\).
This implies that \((a_n)_{n = 0}^\infty\) and \((b_n)_{n = 0}^\infty\) are eventually \(\varepsilon\)-close for every rational \(\varepsilon > 0\).
Thus we conclude that \((a_n)_{n = 0}^\infty\) and \((b_n)_{n = 0}^\infty\) are eventually \(\varepsilon\)-close for every rational \(\varepsilon > 0\) if and only if they are eventually \(\varepsilon\)-close for every real \(\varepsilon > 0\).
\end{proof}