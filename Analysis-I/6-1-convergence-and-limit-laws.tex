\section{Convergence and limit laws}

\begin{definition}[Distance between two real numbers]\label{6.1.1}
Given two real numbers \(x\) and \(y\), we define their distance \(d(x, y)\) to be \(d(x, y) \coloneqq \abs*{x - y}\).
\end{definition}

\begin{note}
Clearly Definition \ref{6.1.1} is consistent with Definition \ref{4.3.2}.
Further, Proposition \ref{4.3.3} works just as well for real numbers as it does for rationals, because the real numbers obey all the rules of algebra that the rationals do.
\end{note}

\begin{definition}[\(\varepsilon\)-close real numbers]\label{6.1.2}
Let \(\varepsilon > 0\) be a real number.
We say that two real numbers \(x, y\) are \emph{\(\varepsilon\)-close} iff we have \(d(y, x) \leq \varepsilon\).
\end{definition}

\begin{note}
Again, it is clear that Definition \ref{6.1.2} is consistent with Definition \ref{4.3.4}.
\end{note}

\begin{note}
Now let \((a_n)_{n = m}^\infty\) be a sequence of real numbers;
i.e., we assign a real number \(a_n\) for every integer \(n \geq m\).
The starting index \(m\) is some integer;
usually this will be \(1\), but in some cases we will start from some index other than \(1\).
(The choice of label used to index this sequence is unimportant; we could use for instance \((a_k)_{k = m}^{\infty}\) and this would represent exactly the same sequence as \((a_n)_{n = m}^{\infty}\).
We can define the notion of a Cauchy sequence in the same manner as before.
\end{note}

\begin{definition}[Cauchy sequences of reals]\label{6.1.3}
Let \(\varepsilon > 0\) be a real number.
A sequence \((a_n)_{n = N}^\infty\) of real numbers starting at some integer index \(N\) is said to be \emph{\(\varepsilon\)-steady} iff \(a_j\) and \(a_k\) are \(\varepsilon\)-close for every \(j, k \geq N\).
A sequence \((a_n)_{n = m}^\infty\) starting at some integer index \(m\) is said to be \emph{eventually \(\varepsilon\)-steady} iff there exists an \(N \geq m\) such that \((a_n)_{n = N}^\infty\) is \(\varepsilon\)-steady.
We say that \((a_n)_{n = m}^\infty\) is a \emph{Cauchy sequence} iff it is eventually \(\varepsilon\)-steady for every \(\varepsilon > 0\).
\end{definition}

\begin{note}
To put it another way, a sequence \((a_n)_{n = m}^\infty\) of real numbers is a Cauchy sequence if, for every real \(\varepsilon > 0\), there exists an \(N \geq m\) such that \(\abs*{a_n - a_n'} \leq \varepsilon\) for all \(n, n' \geq N\).
These definitions are consistent with the corresponding definitions for rational numbers (Definitions \ref{5.1.3}, \ref{5.1.6}, \ref{5.1.8}), although verifying consistency for Cauchy sequences takes a little bit of care.
\end{note}

\begin{proposition}\label{6.1.4}
Let \((a_n)_{n = m}^\infty\) be a sequence of rational numbers starting at some integer index \(m\).
Then \((a_n)_{n = m}^\infty\) is a Cauchy sequence in the sense of Definition \ref{5.1.8} if and only if it is a Cauchy sequence in the sense of Definition \ref{6.1.3}.
\end{proposition}

\begin{proof}
Suppose first that \((a_n)_{n = m}^\infty\) is a Cauchy sequence in the sense of Definition \ref{6.1.3};
then it is eventually \(\varepsilon\)-steady for every real \(\varepsilon > 0\).
In particular, it is eventually \(\varepsilon\)-steady for every rational \(\varepsilon > 0\), which makes it a Cauchy sequence in the sense of Definition \ref{5.1.8}.

Now suppose that \((a_n)_{n = m}^\infty\) is a Cauchy sequence in the sense of Definition \ref{5.1.8};
then it is eventually \(\varepsilon\)-steady for every rational \(\varepsilon > 0\).
If \(\varepsilon > 0\) is a real number, then there exists a rational \(\varepsilon' > 0\) which is smaller than \(\varepsilon\), by Proposition \ref{5.4.12}.
Since \(\varepsilon'\) is rational, we know that \((a_n)_{n = m}^\infty\) is eventually \(\varepsilon'\)-steady;
since \(\varepsilon' < \varepsilon\), this implies that \((a_n)_{n = m}^\infty\) is eventually \(\varepsilon\)-steady.
Since \(\varepsilon\) is an arbitrary positive real number, we thus see that \((a_n)_{n = m}^\infty\) is a Cauchy sequence in the sense of Definition \ref{6.1.3}.
\end{proof}

\begin{note}
Because of Proposition \ref{6.1.4}, we will no longer care about the distinction between Definition \ref{5.1.8} and Definition \ref{6.1.3}, and view the concept of a Cauchy sequence as a single unified concept.
\end{note}

\begin{definition}[Convergence of sequences]\label{6.1.5}
Let \(\varepsilon > 0\) be a real number, and let \(L\) be a real number.
A sequence \((a_n)_{n = N}^\infty\) of real numbers is said to be \emph{\(\varepsilon\)-close to \(L\)} iff \(a_n\) is \(\varepsilon\)-close to \(L\) for every \(n \geq N\), i.e., we have \(\abs*{a_n - L} \leq \varepsilon\) for every \(n \geq N\).
We say that a sequence \((a_n)_{n = m}^\infty\) is \emph{eventually \(\varepsilon\)-close to \(L\)} iff there exists an \(N \geq m\) such that \((a_n)_{n = N}^\infty\) is \(\varepsilon\)-close to \(L\).
We say that a sequence \((a_n)_{n = m}^\infty\) \emph{converges to \(L\)} iff it is eventually \(\varepsilon\)-close to \(L\) for every real \(\varepsilon > 0\).
\end{definition}

\setcounter{theorem}{6}
\begin{proposition}[Uniqueness of limits]\label{6.1.7}
Let \((a_n)_{n = m}^\infty\) be a real sequence starting at some integer index \(m\), and let \(L \neq L'\) be two distinct real numbers.
Then it is not possible for \((a_n)_{n = m}^\infty\) to converge to \(L\) while also converging to \(L'\).
\end{proposition}

\begin{proof}
Suppose for sake of contradiction that \((a_n)_{n = m}^\infty\) was converging to both \(L\) and \(L'\).
Let \(\varepsilon = \abs*{L - L'} / 3\);
note that \(\varepsilon\) is positive since \(L \neq L'\).
Since \((a_n)_{n = m}^\infty\) converges to \(L\), we know that \((a_n)_{n = m}^\infty\) is eventually \(\varepsilon\)-close to \(L\);
thus there is an \(N \geq m\) such that \(d(a_n, L) \leq \varepsilon\) for all \(n \geq N\).
Similarly, there is an \(M \geq m\) such that \(d(a_n, L') \leq \varepsilon\) for all \(n \geq M\).
In particular, if we set \(n \coloneqq \max(N, M)\), then we have \(d(a_n, L) \leq \varepsilon\) and \(d(a_n, L') \leq \varepsilon\), hence by the triangle inequality \(d(L, L') \leq 2\varepsilon = 2\abs*{L - L'} / 3\).
But then we have \(\abs*{L - L'} \leq 2\abs*{L - L'} / 3\), which contradicts the fact that \(\abs*{L - L'} > 0\).
Thus it is not possible to converge to both \(L\) and \(L'\).
\end{proof}

\begin{definition}[Limits of sequences]\label{6.1.8}
If a sequence \((a_n)_{n = m}^\infty\) converges to some real number \(L\), we say that \((a_n)_{n = m}^\infty\) is \emph{convergent} and that its \emph{limit} is \(L\);
we write
\[
    L = \lim_{n \to \infty} a_n
\]
to denote this fact.
If a sequence \((a_n)_{n = m}^\infty\) is not converging to any real number \(L\), we say that the sequence \((a_n)_{n = m}^\infty\) is \emph{divergent} and we leave \(\lim_{n \to \infty} a_n\) undefined.
\end{definition}

\begin{note}
Proposition \ref{6.1.7} ensures that a sequence can have at most one limit.
Thus, if the limit exists, it is a single real number, otherwise it is undefined.
\end{note}

\begin{remark}\label{6.1.9}
The notation \(\lim_{n \to \infty} a_n\) does not give any indication about the starting index \(m\) of the sequence, but the starting index is irrelevant (Exercise \ref{ex 6.1.3}).
Thus in the rest of this discussion we shall not be too careful as to where these sequences start, as we shall be mostly focused on their limits.
\end{remark}

\exercisesection

\begin{exercise}\label{ex 6.1.1}
Let \((a_n)_{n = m}^\infty\) be a sequence of real numbers, such that \(a_{n + 1} > a_n\) for each natural number \(n\).
Prove that whenever \(n\) and \(m\) are natural numbers such that \(m > n\), then we have \(a_m > a_n\).
(We refer to these sequences as \emph{increasing} sequences.)
\end{exercise}

\begin{proof}
Let \(E = \{z \in \mathds{N} : n \leq z \leq m\}\).
Then \(E\) is a finite set since \(m - n \in \mathds{N}\) and is non-empty set since \(n, m \in E\).
Let \((a_z)_{z = n}^m\) be a sequence by mapping \(z \in E\) to \(a_z\).
So \((a_z)_{z = n}^m\) is a finite sequence, and the elements in sequence \((a_z)_{z = n}^m\) are \(\{a_n, a_{n + 1}, \dots, a_{m - 1}, a_m\}\).
By the given conditions we have \(a_{z + 1} > a_z\) for each natural number \(z\).
Thus we have \(a_n < a_{n + 1} < \dots < a_{m - 1} < a_m\), and by Proposition \ref{5.4.7} we have \(a_n < a_m\).
\end{proof}

\begin{exercise}\label{ex 6.1.2}
Let \((a_n)_{n = m}^\infty\) be a sequence of real numbers, and let \(L\) be a real number.
Show that \((a_n)_{n = m}^\infty\) converges to \(L\) if and only if, given any real \(\varepsilon > 0\), one can find an \(N \geq m\) such that \(\abs*{a_n - L} \leq \varepsilon\) for all \(n \geq N\).
\end{exercise}

\begin{proof}
\begin{align*}
& (a_n)_{n = m}^\infty \text{ converges to } L \\
\iff & (\forall\ (\varepsilon \in \mathds{R}) \land (\varepsilon > 0)), (a_n)_{n = m}^\infty \text{ is eventually } \varepsilon\text{-close to } L & \text{(by Definition \ref{6.1.5})} \\
\iff & (\forall\ (\varepsilon \in \mathds{R}) \land (\varepsilon > 0)), (\exists\ (N \in \mathds{N}) \land (N \geq m)) : \\
& (a_n)_{n = N}^\infty \text{ is } \varepsilon\text{-close to } L & \text{(by Definition \ref{6.1.5})} \\
\iff & (\forall\ (\varepsilon \in \mathds{R}) \land (\varepsilon > 0)), (\exists\ (N \in \mathds{N}) \land (N \geq m)) : \\
& \abs*{a_n - L} \leq \varepsilon \ \forall\ n \geq N. & \text{(by Definition \ref{6.1.5})}
\end{align*}
\end{proof}

\begin{exercise}\label{ex 6.1.3}
Let \((a_n)_{n = m}^\infty\) be a sequence of real numbers, let \(c\) be a real number, and let \(m' \geq m\) be an integer.
Show that \((a_n)_{n = m}^\infty\) converges to \(c\) if and only if \((a_n)_{n = m'}^\infty\) converges to \(c\).
\end{exercise}

\begin{proof}
\begin{align*}
& (a_n)_{n = m}^\infty \text{ converges to } c \\
\implies & (\forall\ (\varepsilon \in \mathds{R}) \land (\varepsilon > 0)), (a_n)_{n = m}^\infty \text{ is eventually } \varepsilon\text{-close to } c & \text{(by Definition \ref{6.1.5})} \\
\implies & (\forall\ (\varepsilon \in \mathds{R}) \land (\varepsilon > 0)), (\exists\ (N \in \mathds{N}) \land (N \geq m)) : \\
& (a_n)_{n = N}^\infty \text{ is } \varepsilon\text{-close to } c & \text{(by Definition \ref{6.1.5})} \\
\implies & (\forall\ (\varepsilon \in \mathds{R}) \land (\varepsilon > 0)), (\exists\ (N \in \mathds{N}) \land (N \geq \max(m, m'))) : \\
& (a_n)_{n = N}^\infty \text{ is } \varepsilon\text{-close to } c. & \text{(by Definition \ref{6.1.5})}
\end{align*}
Now we can divide into two cases:
\begin{enumerate}
    \item If \(m \geq m'\), then we have \(N \geq m'\), so \(\exists\ N \in \mathds{N}\) and \(N \geq m'\) such that \((a_n)_{n = N}^\infty\) is \(\varepsilon\)-close to \(c\).
    Thus \((a_n)_{n = m'}^\infty\) converges to \(c\).
    \item If \(m < m'\), then we can let \(N \geq m'\), so \(\exists\ N \in \mathds{N}\) and \(N \geq m'\) such that \((a_n)_{n = N}^\infty\) is \(\varepsilon\)-close to \(c\).
    Thus \((a_n)_{n = m'}^\infty\) converges to \(c\).
\end{enumerate}
In both cases we get \((a_n)_{n = m'}^\infty\) converges to \(c\).
Thus if \((a_n)_{n = m}^\infty\) converges to \(c\), then \((a_n)_{n = m'}^\infty\) converges to \(c\).
Similar proof shows that if \((a_n)_{n = m'}^\infty\) converges to \(c\), then \((a_n)_{n = m}^\infty\) converges to \(c\).
Thus we conclude that \((a_n)_{n = m}^\infty\) converges to \(c\) if and only if \((a_n)_{n = m'}^\infty\) converges to \(c\).
\end{proof}