\section{The Extended real number system}\label{sec 6.2}

\begin{definition}[Extended real number system]\label{6.2.1}
The \emph{extended real number system \(\mathds{R}^*\)} is the real line \(\mathds{R}\) with two additional elements attached, called \(+\infty\) and \(-\infty\).
These elements are distinct from each other and also distinct from every real number.
An extended real number \(x\) is called \emph{finite} iff it is a real number, and \emph{infinite} iff it is equal to \(+\infty\) or \(-\infty\).
(This definition is not directly related to the notion of finite and infinite sets in Section \ref{sec 3.6}, though it is of course similar in spirit.)
\end{definition}

\begin{definition}[Negation of extended reals]\label{6.2.2}
The operation of negation \(x \to -x\) on \(\mathds{R}\), we now extend to \(\mathds{R}^*\) by defining \(-(+\infty) \coloneqq -\infty\) and \(-(-\infty) \coloneqq +\infty\).
\end{definition}

\begin{note}
Thus every extended real number \(x\) has a negation, and \(-(-x)\) is always equal to \(x\).
\end{note}

\begin{definition}[Ordering of extended reals]\label{6.2.3}
Let \(x\) and \(y\) be extended real numbers.
We say that \(x \leq y\), i.e., \(x\) is less than or equal to \(y\), iff one of the following three statements is true:
\begin{enumerate}
    \item \(x\) and \(y\) are real numbers, and \(x \leq y\) as real numbers.
    \item \(y = +\infty\).
    \item \(x = -\infty\).
\end{enumerate}
We say that \(x < y\) if we have \(x \leq y\) and \(x \neq y\).
We sometimes write \(x < y\) as \(y > x\), and \(x \leq y\) as \(y \geq x\).
\end{definition}

\setcounter{theorem}{4}
\begin{proposition}\label{6.2.5}
Let \(x, y, z\) be extended real numbers.
Then the following statements are true:
\begin{enumerate}
    \item (Reflexivity)
    We have \(x \leq x\).
    \item (Trichotomy)
    Exactly one of the statements \(x < y\), \(x = y\), or \(x > y\) is true.
    \item (Transitivity)
    If \(x \leq y\) and \(y \leq z\), then \(x \leq z\).
    \item (Negation reverses order) If \(x \leq y\), then \(-y \leq -x\).
\end{enumerate}
\end{proposition}

\begin{proof}{(a)}
By Proposition \ref{5.4.7}, we already have \(x \leq x\) when \(x \in \mathds{R}\).
So we only need to consider the cases \(x = +\infty\) or \(x = -\infty\).
By Definition \ref{6.2.3}, we have \(x \leq +\infty \ \forall\ x \in \mathds{R}^*\).
So we have \(+\infty \leq +\infty\).
Again by Definition \ref{6.2.3}, we have \(-\infty \leq x \ \forall\ x \in \mathds{R}^*\).
So we have \(-\infty \leq -\infty\).
Thus we conclude that \(x \leq x \ \forall\ x \in \mathds{R}^*\).
\end{proof}

\begin{proof}{(b)}
By Proposition \ref{5.4.7}, we already have exactly one of the statements \(x < y\), \(x = y\), or \(x > y\) is true when \(x, y \in \mathds{R}\).
So we only need to consider the cases \(x = +\infty\), \(x = -\infty\), \(y = +\infty\), or \(y = -\infty\).
\begin{enumerate}[label=(\Roman*)]
    \item If \(x = +\infty\), then by Definition \ref{6.2.3} we have \(x \geq y \ \forall\ y \in \mathds{R}^*\).
    \begin{enumerate}[label=(\roman*)]
        \item If \(y = +\infty\), then we have \(x = y\).
        \item If \(y \in \mathds{R}\), then by Definition \ref{6.2.1} \(x \neq y\).
        Thus by Definition \ref{6.2.3} we have \(x > y\).
        \item If \(y = -\infty\), then by Definition \ref{6.2.1} \(x \neq y\).
        Thus by Definition \ref{6.2.3} we have \(x > y\).
    \end{enumerate}
    \item If \(x = -\infty\), then by Definition \ref{6.2.3} we have \(x \leq y \ \forall\ y \in \mathds{R}^*\).
    \begin{enumerate}[label=(\roman*)]
        \item If \(y = +\infty\), then by Definition \ref{6.2.1} \(x \neq y\).
        Thus by Definition \ref{6.2.3} we have \(x < y\).
        \item If \(y \in \mathds{R}\), then by Definition \ref{6.2.1} \(x \neq y\).
        Thus by Definition \ref{6.2.3} we have \(x < y\).
        \item If \(y = -\infty\), then we have \(x = y\).
    \end{enumerate}
    \item If \(y = +\infty\), then by Definition \ref{6.2.3} we have \(x \leq y \ \forall\ x \in \mathds{R}^*\).
    \begin{enumerate}[label=(\roman*)]
        \item If \(x = +\infty\), then we have \(x = y\).
        \item If \(x \in \mathds{R}\), then by Definition \ref{6.2.1} \(x \neq y\).
        Thus by Definition \ref{6.2.3} we have \(x < y\).
        \item If \(x = -\infty\), then by Definition \ref{6.2.1} \(x \neq y\).
        Thus by Definition \ref{6.2.3} we have \(x < y\).
    \end{enumerate}
    \item If \(y = -\infty\), then by Definition \ref{6.2.3} we have \(x \geq y \ \forall\ x \in \mathds{R}^*\).
    \begin{enumerate}[label=(\roman*)]
        \item If \(x = +\infty\), then by Definition \ref{6.2.1} \(x \neq y\).
        Thus by Definition \ref{6.2.3} we have \(x > y\).
        \item If \(x \in \mathds{R}\), then by Definition \ref{6.2.1} \(x \neq y\).
        Thus by Definition \ref{6.2.3} we have \(x > y\).
        \item If \(x = -\infty\), then we have \(x = y\).
    \end{enumerate}
\end{enumerate}
From all cases above we show that exactly one of the statements \(x < y\), \(x = y\), or \(x > y\) is true.
Thus we finish the proof.
\end{proof}

\begin{proof}{(c)}
By Proposition \ref{5.4.7}, we already have \((x \leq y) \land (y \leq z) \implies x \leq z\) when \(x, y, z \in \mathds{R}\).
So we only need to consider the cases \(x = +\infty\), \(x = -\infty\), \(y = +\infty\), \(y = -\infty\), \(z = +\infty\), or \(z = -\infty\).
\begin{enumerate}[label=(\Roman*)]
    \item If \(x = +\infty\), then by Definition \ref{6.2.3} \((x = +\infty) \land (x \leq y) \implies y = +\infty\).
    Similarly \((y = +\infty) \land (y \leq z) \implies z = +\infty\).
    Thus we have \(x = +\infty = z\), and by Proposition \ref{6.2.5}(a) we have \(x \leq z\).
    \item If \(x = -\infty\), then by Definition \ref{6.2.3} \(x \leq z \ \forall\ z \in \mathds{R}^*\).
    \item If \(y = +\infty\), then by Definition \ref{6.2.3} \((y = +\infty) \land (y \leq z) \implies z = +\infty\).
    Again by Definition \ref{6.2.3}, we have \(x \leq +\infty = z \ \forall\ x \in \mathds{R}^*\).
    \item If \(y = -\infty\), then by Definition \ref{6.2.3} \((y = -\infty) \land (x \leq y) \implies x = -\infty\).
    Again by Definition \ref{6.2.3}, we have \(x = -\infty \leq z \ \forall\ z \in \mathds{R}^*\).
    \item If \(z = +\infty\), then by Definition \ref{6.2.3}, we have \(x \leq +\infty = z \ \forall\ x \in \mathds{R}^*\).
    \item If \(z = -\infty\), then by Definition \ref{6.2.3} \((z = -\infty) \land (y \leq z) \implies y = -\infty\).
    Similarly \((y = -\infty) \land (x \leq y) \implies x = -\infty\).
    Thus we have \(x = -\infty = z\), and by Proposition \ref{6.2.5}(a) we have \(x \leq z\).
\end{enumerate}
From all cases above we show that \((x \leq y) \land (y \leq z) \implies x \leq z\).
Thus we finish the proof.
\end{proof}

\begin{proof}{(d)}
By Proposition \ref{5.4.7}, we already have \(x \leq y \implies -y \leq -x\) when \(x, y \in \mathds{R}\).
So we only need to consider the cases \(x = +\infty\), \(x = -\infty\), \(y = +\infty\), or \(y = -\infty\).
\begin{enumerate}[label=(\Roman*)]
    \item If \(x = +\infty\), then by Definition \ref{6.2.3} \((x = +\infty) \land (x \leq y) \implies y = +\infty\).
    And by Definition \ref{6.2.2} we have \(-x = -\infty = -y\).
    Thus by Proposition \ref{6.2.5}(a) we have \(-y \leq -x\).
    \item If \(x = -\infty\), then by Definition \ref{6.2.2} \(-x = +\infty\).
    And by Definition \ref{6.2.3} we have \(-y \leq -x \ \forall\ -y \in \mathds{R}^*\).
    \item If \(y = +\infty\), then by Definition \ref{6.2.2} \(-y = -\infty\).
    And by Definition \ref{6.2.3} we have \(-y \leq -x \ \forall\ -x \in \mathds{R}^*\).
    \item If \(y = -\infty\), then by Definition \ref{6.2.3} \((y = -\infty) \land (x \leq y) \implies x = -\infty\).
    And by Definition \ref{6.2.2} we have \(-x = +\infty = -y\).
    Thus by Proposition \ref{6.2.5}(a) we have \(-y \leq -x\).
\end{enumerate}
From all cases above we show that \(x \leq y \implies -y \leq -x\).
Thus we finish the proof.
\end{proof}

\begin{note}
One could also introduce other operations on the extended real number system, such as addition, multiplication, etc.
However, this is somewhat dangerous as these operations will almost certainly fail to obey the familiar rules of algebra.
For instance, to define addition it seems reasonable (given one’s intuitive notion of infinity) to set \(+\infty + 5 = +\infty\) and \(+\infty + 3 = +\infty\), but then this implies that \(+\infty + 5 = +\infty + 3\), while \(5 \neq 3\).
So things like the cancellation law begin to break down once we try to operate involving infinity.
To avoid these issues we shall simply not define any arithmetic operations on the extended real number system other than negation and order.
\end{note}

\begin{definition}[Supremum of sets of extended reals]\label{6.2.6}
Let \(E\) be a subset of \(\mathds{R}^*\).
Then we define the \emph{supremum} \(\sup(E)\) or \emph{least upper bound} of \(E\) by the following rule.
\begin{enumerate}
    \item If \(E\) is contained in \(\mathds{R}\) (i.e., \(+\infty\) and \(-\infty\) are not elements of \(E\)), then we let \(\sup(E)\) be as defined in Definition \ref{5.5.10}.
    \item If \(E\) contains \(+\infty\), then we set \(\sup(E) \coloneqq +\infty\).
    \item If \(E\) does not contain \(+\infty\) but does contain \(-\infty\), then we set \(\sup(E) \coloneqq \sup(E \setminus \{-\infty\})\)
    (which is a subset of \(\mathds{R}\) and thus falls under case (a)).
\end{enumerate}
We also define the \emph{infimum} \(\inf(E)\) of \(E\) (also known as the \emph{greatest lower bound} of \(E\)) by the formula
\[
    \inf(E) \coloneqq -\sup(-E)
\]
where \(-E\) is the set \(-E \coloneqq \{-x : x \in E\}\).
\end{definition}

\setcounter{theorem}{9}
\begin{example}\label{6.2.10}
Let \(E\) be the empty set.
Then \(\sup(E) = -\infty\) and \(\inf(E) = +\infty\).
(Because \(-E\) is also empty, so \(\sup(-E) = -\infty\), thus \(\inf(E) = -\sup(-E) = -(-\infty) = +\infty\))
This is the only case in which the supremum can be less than the infimum.
We show that by let \(\sup(E)\) be three different value, namely \(+\infty\), \(-\infty\), or arbitrary real number.
\begin{enumerate}[label=(\Roman*)]
    \item If \(\sup(E) = +\infty\), then we can further divide into two cases:
    \begin{enumerate}[label=(\roman*)]
        \item If \(-\infty \in E\), then \(+\infty \in -E\).
        So \(\inf(E) = -\sup(-E) = -(+\infty) = -\infty\), and we have \(\inf(E) < \sup(E)\).
        \item If \(-\infty \not\in E\), then \(+\infty \not\in -E\).
        So \(\inf(E) \in \mathds{R}\), and we have \(\inf(E) < \sup(E)\).
    \end{enumerate}
    \item If \(\sup(E) = -\infty\), then \(\inf(E) = +\infty\).
    So we have \(\inf(E) > \sup(E)\).
    \item If \(\sup(E) \in \mathds{R}\), then we can further divide into two cases:
    \begin{enumerate}[label=(\roman*)]
        \item If \(-\infty \in E\), then \(+\infty \in -E\).
        So \(\inf(E) = -\sup(-E) = -(+\infty) = -\infty\), and we have \(\inf(E) < \sup(E)\).
        \item If \(-\infty \not\in E\), then \(+\infty \not\in -E\).
        So \(\inf(E) \in \mathds{R}\), and we have
        \begin{align*}
            & (x \in E) \land (x \leq \sup(E)) \\
            \implies & -x \geq -\sup(E) \\
            \implies & \sup(-E) \geq -x \geq -\sup(E) \\
            \implies & \inf(E) = -\sup(-E) \leq x \leq \sup(E).
        \end{align*}
    \end{enumerate}
\end{enumerate}
Thus only when \(\sup(E) = -\infty\) we have \(\inf(E) > \sup(E)\).
\end{example}

\begin{note}
One can intuitively think of the supremum of \(E\) as follows.
Imagine the real line with \(+\infty\) somehow on the far right, and \(-\infty\) on the far left.
Imagine a piston at \(+\infty\) moving leftward until it is stopped by the presence of a set \(E\);
the location where it stops is the supremum of \(E\).
Similarly if one imagines a piston at \(-\infty\) moving rightward until it is stopped by the presence of \(E\), the location where it stops is the infimum of \(E\).
In the case when \(E\) is the empty set, the pistons pass through each other, the supremum landing at \(-\infty\) and the infimum landing at \(+\infty\).
\end{note}

\begin{theorem}\label{6.2.11}
Let \(E\) be a subset of \(\mathds{R}^*\).
Then the following statements are true.
\begin{enumerate}
    \item For every \(x \in E\) we have \(x \leq \sup(E)\) and \(x \geq \inf(E)\).
    \item Suppose that \(M \in \mathds{R}^*\) is an upper bound for \(E\), i.e., \(x \leq M\) for all \(x \in E\).
    Then we have \(\sup(E) \leq M\).
    \item Suppose that \(M \in \mathds{R}^*\) is a lower bound for \(E\), i.e., \(x \geq M\) for all \(x \in E\).
    Then we have \(\inf(E) \geq M\).
\end{enumerate}
\end{theorem}

\begin{proof}{(a)}
We first show that \(\forall\ x \in E\), \(x \leq \sup(E)\).
Suppose first that \(E = \emptyset\).
Then the statements \(\forall\ x \in \emptyset\), \(x \leq \sup(E)\) is vacuously true.
Now suppose that \(E \neq \emptyset\).
We split into two different cases:
\begin{enumerate}[label=(\Roman*)]
    \item If \(+\infty \not\in E\), then we can further split into two cases:
    \begin{enumerate}[label=(\roman*)]
        \item If \(E \neq \{-\infty\}\), let \(E' = E \setminus \{-\infty\}\), so \(E' \neq \emptyset\).
        By Theorem \ref{5.5.9} we have \(x \leq \sup(E')\).
        And by Definition \ref{6.2.6} we have \(\sup(E) = \sup(E')\), so \(x \leq \sup(E)\).
        \item If \(E = \{-\infty\}\), then by Definition \ref{6.2.6} we have \(\sup(E) = \sup(\emptyset)\).
        And we already show that \(\forall\ x \in \emptyset\), \(x \leq \sup(E)\) is vacuously true.
    \end{enumerate}
    \item If \(+\infty \in E\), then \(\sup(E) = +\infty\), so by Definition \ref{6.2.3} \(x \leq \sup(E)\).
\end{enumerate}
Thus we conclude that \(\forall\ x \in E\), \(x \leq \sup(E)\).

Now we show that \(\forall\ x \in E\), \(x \geq \inf(E)\).
Suppose first that \(E \neq \emptyset\).
From above proof we have \(x \leq \sup(E)\).
So
\begin{align*}
& x \leq \sup(E) \\
\implies & -x \geq -\sup(E) \\
\implies & \sup(-E) \geq -x \geq -\sup(E) \\
\implies & \inf(E) = -\sup(-E) \leq x \leq \sup(E).
\end{align*}
Now suppose that \(E = \emptyset\).
So the statements \(\forall\ x \in \emptyset\), \(x \geq \inf(E)\) is vacuously true.
Thus we conclude that \(\forall\ x \in E\), \(x \geq \inf(E)\).
\end{proof}

\begin{proof}{(b)}
Suppose first that \(E = \emptyset\).
Then by Example \ref{6.2.10} and Definition \ref{6.2.3} we have \(\sup(E) = -\infty \leq M\).
Now suppose that \(E \neq \emptyset\).
We split into two different cases:
\begin{enumerate}[label=(\Roman*)]
    \item If \(+\infty \not\in E\), then we can further split into two cases:
    \begin{enumerate}[label=(\roman*)]
        \item If \(E \neq \{-\infty\}\), let \(E' = E \setminus \{-\infty\}\), so \(E' \neq \emptyset\).
        By Theorem \ref{5.5.9} and Definition \ref{5.5.5} we have \(x \leq \sup(E') \leq M\).
        And by Definition \ref{6.2.6} we have \(\sup(E) = \sup(E')\), so \(x \leq \sup(E) \leq M\).
        \item If \(E = \{-\infty\}\), then by Definition \ref{6.2.6} we have \(\sup(E) = \sup(\emptyset)\).
        And we already show that \(\sup(E) = -\infty \leq M\).
    \end{enumerate}
    \item If \(+\infty \in E\), then \(\sup(E) = +\infty\).
    By the given condition we have \(\sup(E) \in E \implies \sup(E) \leq M\).
\end{enumerate}
Thus we conclude that \(\sup(E) \leq M\).
\end{proof}

\begin{proof}{(c)}
Suppose first that \(E = \emptyset\).
Then by Example \ref{6.2.10} and Definition \ref{6.2.3} we have \(\inf(E) = +\infty \geq M\).
Now suppose that \(E \neq \emptyset\).
We split into two different cases:
\begin{enumerate}[label=(\Roman*)]
    \item If \(-\infty \not\in E\), then \(+\infty \not\in -E\).
    We can further split into two cases:
    \begin{enumerate}[label=(\roman*)]
        \item If \(-E \neq \{-\infty\}\), let \(-E' = -E \setminus \{-\infty\}\), so \(-E' \neq \emptyset\).
        By the given condition we have \(-M \geq -x \in -E'\), which means \(-M \geq \sup(-E') \geq -x\) by Theorem \ref{5.5.9}.
        And by Definition \ref{6.2.6} we have \(\sup(-E) = \sup(-E')\), so \(M \leq -\sup(-E) = \inf(E) \leq x\).
        \item If \(-E = \{-\infty\}\), then by Definition \ref{6.2.6} we have \(\sup(-E) = \sup(\emptyset)\).
        And we already show that \(\inf(E) = -\sup(-E) = -(-\infty) = +\infty \geq M\).
    \end{enumerate}
    \item If \(-\infty \in E\), then \(\sup(-E) = +\infty\).
    By the given condition we have \(\inf(E) = -\sup(-E) = -(+\infty) = -\infty \in E \implies \inf(E) \geq M\).
\end{enumerate}
Thus we conclude that \(\inf(E) \geq M\).
\end{proof}

\exercisesection

\begin{exercise}\label{ex 6.2.1}
Prove Proposition \ref{6.2.5}.
\end{exercise}

\begin{proof}
See Proposition \ref{6.2.5}.
\end{proof}

\begin{exercise}\label{ex 6.2.2}
Prove Theorem \ref{6.2.11}.
\end{exercise}

\begin{proof}
See Theorem \ref{6.2.11}.
\end{proof}