\section{Suprema and Infima of sequences}\label{sec 6.3}

\begin{definition}[Sup and inf of sequences]\label{6.3.1}
    Let \((a_n)_{n = m}^\infty\) be a sequence of real numbers.
    Then we define \(\sup(a_n)_{n = m}^\infty\) to be the supremum of the set \(\{a_n : n \geq m\}\), and \(\inf(a_n)_{n = m}^\infty\) to the infimum of the same set \(\{a_n : n \geq m\}\).
\end{definition}

\begin{remark}\label{6.3.2}
    The quantities \(\sup(a_n)_{n = m}^\infty\) and \(\inf(a_n)_{n = m}^\infty\) are sometimes written as \(\sup_{n \geq m} a_n\) and \(\inf_{n \geq m} a_n\) respectively.
\end{remark}

\setcounter{theorem}{3}
\begin{example}\label{6.3.4}
    Let \(a_n \coloneqq 1 / n\);
    thus \((a_n)_{n = 1}^\infty\) is the sequence \(1, 1 / 2, 1 / 3, \dots\).
    Then the set \(\{a_n : n \geq 1\}\) is the countable set \(\{1, 1 / 2, 1 / 3, 1 / 4, \dots\}\).
    Thus \(\sup(a_n)_{n = 1}^\infty = 1\) and \(\inf(a_n)_{n = 1}^\infty = 0\).
\end{example}

\begin{proof}
    We first show that \(\sup(a_n)_{n = 1}^\infty = 1\).
    By the given condition \(\forall\ n \in \mathbf{N}\) and \(n \geq 1\), we have \(a_n \leq 1\).
    If \(x \in \mathbf{R}\) and \(x < 1\), then \(x < a_1\), which means \(x\) is not an upper bound of \((a_n)_{n = 1}^\infty\).
    Thus \(\sup(a_n)_{n = 1}^\infty = 1\).

    Now we show that \(\inf(a_n)_{n = 1}^\infty = 0\).
    Because \(\forall\ n \in \mathbf{N}\) and \(n \geq 1\), we have \(-a_n = -1 / n \leq 0\).
    So \(0\) is an upper bound of \(\{-a_n : n \geq 1\}\), and \(\sup(\{-a_n : n \geq 1\}) \leq 0\).
    Then we have
    \begin{align*}
        \inf(a_n)_{n = 1}^\infty & = \inf(\{a_n : n \geq 1\})   & \text{(by Definition \ref{6.3.1})} \\
                                 & = -\sup(-\{a_n : n \geq 1\}) & \text{(by Definition \ref{6.2.6})} \\
                                 & = -\sup(\{-a_n : n \geq 1\}) & \text{(by Definition \ref{6.2.6})} \\
                                 & \geq 0.
    \end{align*}
    So \(0\) is a lower bound of \(\{a_n : n \geq 1\}\), i.e., \(0 \leq \inf(a_n)_{n = 1}^\infty\).
    Suppose for sake of contradiction that \(\exists\ x \in \mathbf{R}\) such that \(x > 0\) and \(x = \inf(a_n)_{n = 1}^\infty\).
    Then by Remark \ref{6.3.7} \(\forall\ n \in \mathbf{N}\) and \(n \geq 1\) we must have \(0 < x \leq a_n\).
    But by Proposition \ref{5.4.12}, \(\exists\ q \in \mathbf{Q}\) and \(q > 0\) such that \(q \leq x\).
    Let such \(q = a / b\), where \(a, b \in \mathbf{Z}\) and \(a, b > 0\).
    Since \(b \in \mathbf{Z}\) and \(b > 0\), we have both \(1 / b, 1 / (b + 1) \in \{a_n : n \geq 1\}\).
    Since \(1 / (b + 1) < 1 / b \leq a / b\), we have \(1 / (b + 1) < x\), which contradict to \(x \leq a_n\).
    Thus \(\nexists\ x \in \mathbf{R}\) such that \(x > 0\) and \(x = \inf(a_n)_{n = 1}^\infty\), therefore \(\inf(a_n)_{n = 1}^\infty = 0\).
\end{proof}

\begin{note}
    It is a little inaccurate to think of the supremum and infimum as the ``largest element of the sequence'' and ``smallest element of the sequence'' respectively.
\end{note}

\begin{note}
    It is possible for the supremum or infimum of a sequence to be \(+\infty\) or \(-\infty\).
    However, if a sequence \((a_n)_{n = m}^\infty\) is bounded, say bounded by \(M\), then all the elements \(a_n\) of the sequence lie between \(-M\) and \(M\), so that the set \(\{a_n : n \geq m\}\) has \(M\) as an upper bound and \(-M\) as a lower bound.
    Since this set is clearly non-empty, we can thus conclude that the supremum and infimum of a bounded sequence are real numbers (i.e., not \(+\infty\) and \(-\infty\)).
\end{note}

\setcounter{theorem}{5}
\begin{proposition}[Least upper bound property]\label{6.3.6}
    Let \((a_n)_{n = m}^\infty\) be a sequence of real numbers, and let \(x\) be the extended real number \(x \coloneqq \sup(a_n)_{n = m}^\infty\).
    Then we have \(a_n \leq x\) for all \(n \geq m\).
    Also, whenever \(M \in \mathbf{R}^*\) is an upper bound for \(a_n\) (i.e., \(a_n \leq M\) for all \(n \geq m\)), we have \(x \leq M\).
    Finally, for every extended real number \(y\) for which \(y < x\), there exists at least one \(n \geq m\) for which \(y < a_n \leq x\).
\end{proposition}

\begin{proof}
    We first show that \(\forall\ n \geq m\) we have \(a_n \leq x\).
    By Definition \ref{6.3.1} we have \(\sup(a_n)_{n = m}^\infty = \sup(\{a_n : n \geq m\})\).
    So by Theorem \ref{6.2.11}, \(\forall\ a_n \in \{a_n : n \geq m\}\) we have \(a_n \leq x\).

    Next we show that \(M \in \mathbf{R}^*\) is an upper bound of \((a_n)_{n = m}^\infty\) implies \(x \leq M\).
    By Definition \ref{6.3.1} we have \(\sup(a_n)_{n = m}^\infty = \sup(\{a_n : n \geq m\})\).
    So by Theorem \ref{6.2.11}, \(\forall\ a_n \in \{a_n : n \geq m\}\) we have \(a_n \leq x \leq M\).

    Finally we show that of \(y \in \mathbf{R}^*\) and \(y < x\), then \(\exists\ n \geq m\) such that \(y < a_n \leq x\).
    Suppose for sake of contradition that such \(n\) does not exist.
    Then \(\forall\ n \geq m\) we must have \(a_n \leq y < x\).
    Thus \(y\) is an upper bound of \((a_n)_{n = m}^\infty\).
    But from above proof we show that if \(y\) is an upper bound, then \(x \leq y\), a contradiction.
    Thus \(\exists\ n \geq m\) such that \(y < a_n \leq x\).
\end{proof}

\begin{remark}\label{6.3.7}
    Let \((a_n)_{n = m}^\infty\) be a sequence of real numbers, and let \(x\) be the extended real number \(x \coloneqq \inf(a_n)_{n = m}^\infty\).
    Then we have \(a_n \geq x\) for all \(n \geq m\).
    Also, whenever \(M \in \mathbf{R}^*\) is an lower bound for \(a_n\) (i.e., \(a_n \geq M\) for all \(n \geq m\)), we have \(x \geq M\).
    Finally, for every extended real number \(y\) for which \(y > x\), there exists at least one \(n \geq m\) for which \(y > a_n \geq x\).
    This is the corresponding Proposition for infima, but with all the references to order reversed, e.g., all upper bounds should now be lower bounds, etc.
    The proof is exactly the same.
\end{remark}

\begin{note}
    In the previous section we saw that all convergent sequences are bounded.
    It is natural to ask whether the converse is true:
    are all bounded sequences convergent?
    The answer is no;
    for instance, the sequence \(1, -1, 1, -1, \dots\) is bounded, but not Cauchy and hence not convergent.
    However, if we make the sequence both bounded and \emph{monotone} (i.e., increasing or decreasing), then it is true that it must converge.
\end{note}

\begin{proposition}[Monotone bounded sequences converge]\label{6.3.8}
    Let \((a_n)_{n = m}^\infty\) be a sequence of real numbers which has some finite upper bound \(M \in \mathbf{R}\), and which is also increasing (i.e., \(a_{n + 1} \geq a_n\) for all \(n \geq m\)).
    Then \((a_n)_{n = m}^\infty\) is convergent, and in fact
    \[
        \lim_{n \to \infty} a_n = \sup(a_n)_{n = m}^\infty \leq M.
    \]
\end{proposition}

\begin{proof}
    Since \((a_n)_{n = m}^\infty\) have an upper bound \(M\), the set \(E = \{a_n : n \geq m\}\) have an upper bound \(M\).
    By Theorem \ref{5.5.9}, the supremum \(\sup(E)\) must exist and \(\sup(E) \leq M\).
    Now we want to show that \(\lim_{n \to \infty} a_n = \sup(E)\).
    By Definition \ref{6.1.8} and \ref{6.1.5}, we need to show that \(\forall\ \varepsilon \in \mathbf{R}\) and \(\varepsilon > 0\), \(\exists\ N \in \mathbf{N}\) and \(N \geq m\) such that \(\abs*{a_n - \sup(E)} \leq \varepsilon \ \forall\ n \geq N\).
    Because \(\forall\ n \geq m\) we have \(a_n \leq \sup(E)\), we must also have \(\abs*{a_n - \sup(E)} = \sup(E) - a_n\).
    So
    \begin{align*}
                 & (\forall\ \varepsilon \in \mathbf{R}) \land (\varepsilon > 0)                                               \\
        \implies & -\varepsilon < 0                                                                                            \\
        \implies & \sup(E) - \varepsilon < \sup(E)                                                                             \\
        \implies & \exists\ N \geq m : \sup(E) - \varepsilon < a_N \leq \sup(E)          & \text{(by Proposition \ref{6.3.6})} \\
        \implies & \forall\ n \geq N : \sup(E) - \varepsilon < a_N \leq a_n \leq \sup(E) & \text{(by the given condition)}     \\
        \implies & \sup(E) - \varepsilon \leq a_n                                                                              \\
        \implies & \sup(E) - a_n \leq \varepsilon                                                                              \\
        \implies & \abs*{a_n - \sup(E)} \leq \varepsilon.
    \end{align*}
    Thus \(\lim_{n \to \infty} a_n = \sup(E) = \sup(a_n)_{n = m}^\infty\).
\end{proof}

\begin{additional corollary}\label{ac 6.3.1}
Let \((a_n)_{n = m}^\infty\) be a sequence of real numbers which has some finite lower bound \(M \in \mathbf{R}\), and which is also decreasing (i.e., \(a_{n + 1} \leq a_n\) for all \(n \geq m\)).
Then \((a_n)_{n = m}^\infty\) is convergent, and in fact
\[
    \lim_{n \to \infty} a_n = \inf(a_n)_{n = m}^\infty \geq M.
\]
\end{additional corollary}

\begin{proof}
    Since \((a_n)_{n = m}^\infty\) have an lower bound \(M\), the set \(E = \{a_n : n \geq m\}\) have an lower bound \(M\).
    By Theorem \ref{5.5.9}, the supremum of the set \(-E = \{-a_n : n \geq m\}\) must exist, so the infimum of the set \(E\) also exist since \(\inf(E) = -\sup(-E)\).
    By Theorem \ref{6.2.11}, we have \(\inf(E) \geq M\).
    Now we want to show that \(\lim_{n \to \infty} a_n = \inf(E)\).
    By Definition \ref{6.1.8} and \ref{6.1.5}, we need to show that \(\forall\ \varepsilon \in \mathbf{R}\) and \(\varepsilon > 0\), \(\exists\ N \in \mathbf{N}\) and \(N \geq m\) such that \(\abs*{a_n - \inf(E)} \leq \varepsilon \ \forall\ n \geq N\).
    Because \(\forall\ n \geq m\) we have \(a_n \geq \inf(E)\), we must also have \(\abs*{a_n - \inf(E)} = a_n - \inf(E)\).
    So
    \begin{align*}
                 & (\forall\ \varepsilon \in \mathbf{R}) \land (\varepsilon > 0)                                           \\
        \implies & \inf(E) < \inf(E) + \varepsilon                                                                         \\
        \implies & \exists\ N \geq m : \inf(E) \leq a_N < \inf(E) + \varepsilon          & \text{(by Remark \ref{6.3.7})}  \\
        \implies & \forall\ n \geq N : \inf(E) \leq a_n \leq a_N < \inf(E) + \varepsilon & \text{(by the given condition)} \\
        \implies & a_n \leq \inf(E) + \varepsilon                                                                          \\
        \implies & a_n - \inf(E) \leq \varepsilon                                                                          \\
        \implies & \abs*{a_n - \inf(E)} \leq \varepsilon.
    \end{align*}
    Thus \(\lim_{n \to \infty} a_n = \inf(E) = \inf(a_n)_{n = m}^\infty\).
\end{proof}

\begin{note}
    A sequence is said to be \emph{monotone} if it is either increasing or decreasing.
    From Proposition \ref{6.3.8} and Corollary \ref{6.1.17} we see that a monotone sequence converges if and only if it is bounded.
\end{note}

\begin{example}\label{6.3.9}
    The sequence \(3, 3.1, 3.14, 3.141, 3.1415, \dots\) is increasing, and is bounded above by \(4\).
    Hence by Proposition \ref{6.3.8} it must have a limit, which is a real number less than or equal to \(4\).
\end{example}

\begin{note}
    Proposition \ref{6.3.8} asserts that the limit of a monotone sequence exists, but does not directly say what that limit is.
    Nevertheless, with a little extra work one can often find the limit once one is given that the limit does exist.
\end{note}

\begin{proposition}\label{6.3.10}
    Let \(0 < x < 1\).
    Then we have \(\lim_{n \to \infty} x^n = 0\).
\end{proposition}

\begin{proof}
    Since \(0 < x < 1\), one can show that the sequence \((x^n)_{n = 1}^\infty\) is decreasing.
    (Because \(x^{n + 1} / x^n = x < 1\), so \(x^{n + 1} < x^n\))
    On the other hand, the sequence \((x^n)_{n = 1}^\infty\) has a lower bound of \(0\).
    Thus by Additional Corollary \ref{ac 6.3.1} the sequence \((x^n)_{n = 1}^\infty\) converges to some limit \(L\).
    Since \(x^{n + 1} = x \times x^n\), we thus see from the limit laws (Theorem \ref{6.1.19}) that \((x^{n + 1})_{n = 1}^\infty\) converges to \(xL\).
    But the sequence \((x^{n + 1})_{n = 1}^\infty\) is just the sequence \((x^n)_{n = 2}^\infty\) shifted by one, and so they must have the same limits by Exercise \ref{ex 6.1.3} and \ref{ex 6.1.4}.
    So \(xL = L\).
    Since \(x \neq 1\), we can solve for \(L\) to obtain \(L = 0\).
    Thus \((x^n)_{n = 1}^\infty\) converges to \(0\).
\end{proof}

\exercisesection

\begin{exercise}\label{ex 6.3.1}
    Verify the claim in Example \ref{6.3.4}.
\end{exercise}

\begin{proof}
    See Example \ref{6.3.4}.
\end{proof}

\begin{exercise}\label{ex 6.3.2}
    Prove Proposition \ref{6.3.6}.
\end{exercise}

\begin{proof}
    See Proposition \ref{6.3.6}.
\end{proof}

\begin{exercise}\label{ex 6.3.3}
    Prove Proposition \ref{6.3.8}.
\end{exercise}

\begin{proof}
    See Proposition \ref{6.3.8}.
\end{proof}

\begin{exercise}\label{ex 6.3.4}
    Explain why Proposition \ref{6.3.10} fails when \(x > 1\).
    In fact, show that the sequence \((x_n)_{n = 1}^\infty\) diverges when \(x > 1\).
\end{exercise}

\begin{proof}
    Since \(x = x^{n + 1} / x^n > 1\), we have \(x^{n + 1} > x^n\), which means \((x^n)_{n = 1}^\infty\) is increasing.
    Suppose for sake of contradiction that \((x^n)_{n = 1}^\infty\) has an upper bound of \(M\).
    Then by Proposition \ref{6.3.8} the sequence \((x^n)_{n = 1}^\infty\) converges to some limit \(L\).
    Since \((1 / x)^n x^n = 1\), we thus see from the limit laws (Theorem \ref{6.1.19}) that \(((1 / x)^n x^n)_{n = 1}^\infty\) converges to \(1\).
    But \(0 < (1 / x)^n < 1\), by Proposition \ref{6.3.10} we have \(((1 / x)^n)_{n = 1}^\infty\) converges to \(0\).
    So by Theorem \ref{6.1.19} we have \(0L = 1\), a contradiction.
    Thus \((x^n)_{n = 1}^\infty\) does not have an upper bound.
    This means \((x^n)_{n = 1}^\infty\) is diverge by Corollary \ref{6.1.17}.
\end{proof}