\section{Limsup, Liminf, and limit points}\label{sec 6.4}

\begin{definition}[Limit points]\label{6.4.1}
Let \((a_n)_{n = m}^\infty\) be a sequence of real numbers, let \(x\) be a real number, and let \(\varepsilon > 0\) be a real number.
We say that \(x\) is \emph{\(\varepsilon\)-adherent} to \((a_n)_{n = m}^\infty\) iff there exists an \(n \geq m\) such that \(a_n\) is \(\varepsilon\)-close to \(x\).
We say that \(x\) is \emph{continually \(\varepsilon\)-adherent} to \((a_n)_{n = m}^\infty\) iff it is \(\varepsilon\)-adherent to \((a_n)_{n = N}^\infty\) for every \(N \geq m\).
We say that \(x\) is a \emph{limit point} or \emph{adherent point} of \((a_n)_{n = m}^\infty\) iff it is continually \(\varepsilon\)-adherent to \((a_n)_{n = m}^\infty\) for every \(\varepsilon > 0\).
\end{definition}

\begin{remark}\label{6.4.2}
The verb ``to adhere'' means much the same as ``to stick to'';
hence the term ``adhesive''.
\end{remark}

\begin{note}
Unwrapping all the definitions, we see that \(x\) is a limit point of \((a_n)_{n = m}^\infty\) if, for every \(\varepsilon > 0\) and every \(N \geq m\), there exists an \(n \geq N\) such that \(\abs*{a_n - x} \leq \varepsilon\).
Note the difference between a sequence being \(\varepsilon\)-close to \(L\)
(which means that \emph{all} the elements of the sequence stay within a distance \(\varepsilon\) of \(L\))
and \(L\) being \(\varepsilon\)-adherent to the sequence
(which only needs a \emph{single} element of the sequence to stay within a distance \(\varepsilon\) of \(L\)).
Also, for \(L\) to be continually \(\varepsilon\)-adherent to \((a_n)_{n = m}^\infty\), it has to be \(\varepsilon\)-adherent to \((a_n)_{n = N}^\infty\) for \emph{all} \(N \geq m\), whereas for \((a_n)_{n = m}^\infty\) to be eventually \(\varepsilon\)-close to \(L\), we only need \((a_n)_{n = m}^\infty\) to be \(\varepsilon\)-close to \(L\) for \emph{some} \(N \geq m\).
Thus there are some subtle differences in quantifiers between limits and limit points.
\end{note}

\setcounter{theorem}{4}
\begin{proposition}[Limits are limit points]\label{6.4.5}
Let \((a_n)_{n = m}^\infty\) be a sequence which converges to a real number \(c\).
Then \(c\) is a limit point of \((a_n)_{n = m}^\infty\), and in fact it is the only limit point of \((a_n)_{n = m}^\infty\).
\end{proposition}

\begin{proof}
We first show that \(c\) is a limit point of \((a_n)_{n = m}^\infty\).
Since \(\lim_{n \to \infty} a_n = c\), we have \(\forall\ \varepsilon \in \mathds{R}\) and \(\varepsilon > 0\), \(\exists\ N_1 \in \mathds{N}\) and \(N_1 \geq m\) such that \(\abs*{a_n - c} \leq \varepsilon \ \forall\ n \geq N_1\).
This means \(\forall\ N_2 \in \mathds{N}\) and \(N_2 \geq m\), \(\exists\ n \geq \max(N_1, N_2)\) such that \(\abs*{a_n - c} \leq \varepsilon\).
So \(c\) is a limit point of \((a_n)_{n = m}^\infty\).

Now we show that \(c\) is the only limit point of \((a_n)_{n = m}^\infty\).
Suppose for sake of contradiction that \(c'\) is also a limit point of \((a_n)_{n = m}^\infty\) and \(c' \neq c\).
Since \(\lim_{n \to \infty} a_n = c\), we have \(\forall\ \varepsilon \in \mathds{R}\) and \(\varepsilon > 0\), \(\exists\ N \in \mathds{N}\) and \(N \geq m\) such that \(\abs*{a_n - c} \leq \varepsilon \ \forall\ n \geq N\).
In particular, \(\abs*{a_n - c} \leq \abs*{c - c'} / 2\).
So
\begin{align*}
\abs*{c - c'} &= \abs*{a_n - a_n + c - c'} \\
&= \abs*{c - a_n a_n - c'} \\
&\leq \abs*{c - a_n} + \abs*{a_n - c'} \\
&= \abs*{a_n - c} + \abs*{a_n - c'}.
\end{align*}
Which means \(\abs*{a_n - c'} \geq \abs*{c - c'} - \abs*{a_n - c}\).
Then \(\forall\ n \geq N\),
\begin{align*}
\abs*{a_n - c'} &\geq \abs*{c - c'} - \abs*{a_n - c} \\
&\geq \abs*{c - c'} - \abs*{c - c'} / 2 \\
&= \abs*{c - c'} / 2.
\end{align*}
Because \(\abs*{c - c'} / 2 > 0\), we can not find an \(n \geq N\) where \(\abs*{a_n - c'} < \abs*{c - c'} / 2\), which contradict to \(c'\) is limit point.
Thus \(\lim_{n \to \infty} a_n = c\) is the only limit point of \((a_n)_{n = m}^\infty\).
\end{proof}

\exercisesection

\begin{exercise}\label{ex 6.4.1}
Prove Proposition \ref{6.4.5}.
\end{exercise}

\begin{proof}
See Proposition \ref{6.4.5}.
\end{proof}