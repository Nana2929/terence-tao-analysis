\section{Limsup, Liminf, and limit points}\label{sec 6.4}

\begin{definition}[Limit points]\label{6.4.1}
Let \((a_n)_{n = m}^\infty\) be a sequence of real numbers, let \(x\) be a real number, and let \(\varepsilon > 0\) be a real number.
We say that \(x\) is \emph{\(\varepsilon\)-adherent} to \((a_n)_{n = m}^\infty\) iff there exists an \(n \geq m\) such that \(a_n\) is \(\varepsilon\)-close to \(x\).
We say that \(x\) is \emph{continually \(\varepsilon\)-adherent} to \((a_n)_{n = m}^\infty\) iff it is \(\varepsilon\)-adherent to \((a_n)_{n = N}^\infty\) for every \(N \geq m\).
We say that \(x\) is a \emph{limit point} or \emph{adherent point} of \((a_n)_{n = m}^\infty\) iff it is continually \(\varepsilon\)-adherent to \((a_n)_{n = m}^\infty\) for every \(\varepsilon > 0\).
\end{definition}

\begin{remark}\label{6.4.2}
The verb ``to adhere'' means much the same as ``to stick to'';
hence the term ``adhesive''.
\end{remark}

\begin{note}
Unwrapping all the definitions, we see that \(x\) is a limit point of \((a_n)_{n = m}^\infty\) if, for every \(\varepsilon > 0\) and every \(N \geq m\), there exists an \(n \geq N\) such that \(\abs*{a_n - x} \leq \varepsilon\).
Note the difference between a sequence being \(\varepsilon\)-close to \(L\)
(which means that \emph{all} the elements of the sequence stay within a distance \(\varepsilon\) of \(L\))
and \(L\) being \(\varepsilon\)-adherent to the sequence
(which only needs a \emph{single} element of the sequence to stay within a distance \(\varepsilon\) of \(L\)).
Also, for \(L\) to be continually \(\varepsilon\)-adherent to \((a_n)_{n = m}^\infty\), it has to be \(\varepsilon\)-adherent to \((a_n)_{n = N}^\infty\) for \emph{all} \(N \geq m\), whereas for \((a_n)_{n = m}^\infty\) to be eventually \(\varepsilon\)-close to \(L\), we only need \((a_n)_{n = m}^\infty\) to be \(\varepsilon\)-close to \(L\) for \emph{some} \(N \geq m\).
Thus there are some subtle differences in quantifiers between limits and limit points.
\end{note}

\setcounter{theorem}{4}
\begin{proposition}[Limits are limit points]\label{6.4.5}
Let \((a_n)_{n = m}^\infty\) be a sequence which converges to a real number \(c\).
Then \(c\) is a limit point of \((a_n)_{n = m}^\infty\), and in fact it is the only limit point of \((a_n)_{n = m}^\infty\).
\end{proposition}

\begin{proof}
We first show that \(c\) is a limit point of \((a_n)_{n = m}^\infty\).
Since \(\lim_{n \to \infty} a_n = c\), we have \(\forall\ \varepsilon \in \mathds{R}\) and \(\varepsilon > 0\), \(\exists\ N_1 \in \mathds{N}\) and \(N_1 \geq m\) such that \(\abs*{a_n - c} \leq \varepsilon \ \forall\ n \geq N_1\).
This means \(\forall\ N_2 \in \mathds{N}\) and \(N_2 \geq m\), \(\exists\ n \geq \max(N_1, N_2)\) such that \(\abs*{a_n - c} \leq \varepsilon\).
So \(c\) is a limit point of \((a_n)_{n = m}^\infty\).

Now we show that \(c\) is the only limit point of \((a_n)_{n = m}^\infty\).
Suppose for sake of contradiction that \(c'\) is also a limit point of \((a_n)_{n = m}^\infty\) and \(c' \neq c\).
Since \(\lim_{n \to \infty} a_n = c\), we have \(\forall\ \varepsilon \in \mathds{R}\) and \(\varepsilon > 0\), \(\exists\ N \in \mathds{N}\) and \(N \geq m\) such that \(\abs*{a_n - c} \leq \varepsilon \ \forall\ n \geq N\).
In particular, \(\abs*{a_n - c} \leq \abs*{c - c'} / 2\).
So
\begin{align*}
\abs*{c - c'} &= \abs*{a_n - a_n + c - c'} \\
&= \abs*{c - a_n + a_n - c'} \\
&\leq \abs*{c - a_n} + \abs*{a_n - c'} \\
&= \abs*{a_n - c} + \abs*{a_n - c'}.
\end{align*}
Which means \(\abs*{a_n - c'} \geq \abs*{c - c'} - \abs*{a_n - c}\).
Then \(\forall\ n \geq N\),
\begin{align*}
\abs*{a_n - c'} &\geq \abs*{c - c'} - \abs*{a_n - c} \\
&\geq \abs*{c - c'} - \abs*{c - c'} / 2 \\
&= \abs*{c - c'} / 2.
\end{align*}
Because \(\abs*{c - c'} / 2 > 0\), we can not find an \(n \geq N\) where \(\abs*{a_n - c'} < \abs*{c - c'} / 2\), which contradict to \(c'\) is limit point.
Thus \(\lim_{n \to \infty} a_n = c\) is the only limit point of \((a_n)_{n = m}^\infty\).
\end{proof}

\begin{definition}[Limit superior and limit inferior]\label{6.4.6}
Suppose that \((a_n)_{n = m}^\infty\) is a sequence.
We define a new sequence \((a_N^+)_{N = m}^\infty\) by the formula
\[
    a_N^+ \coloneqq \sup(a_n)_{n = N}^\infty.
\]
More informally, \(a_N^+\) is the supremum of all the elements in the sequence from \(a_N\) onwards.
We then define the \emph{limit superior} of the sequence \((a_n)_{n = m}^\infty\), denoted \(\lim\sup_{n \to \infty} a_n\), by the formula
\[
    \lim\sup_{n \to \infty} a_n \coloneqq \inf(a_N^+)_{N = m}^\infty.
\]
Similarly, we can define
\[
    a_N^- \coloneqq \inf(a_n)_{n = N}^\infty
\]
and define the \emph{limit inferior} of the sequence \((a_n)_{n = m}^\infty\), denoted \(\lim\inf_{n \to \infty} a_n\), by the formula
\[
    \lim\inf_{n \to \infty} a_n \coloneqq \sup(a_N^-)_{N = m}^\infty.
\]
\end{definition}

\setcounter{theorem}{10}
\begin{remark}\label{6.4.11}
Some authors use the notation \(\overline{\lim}_{n \to \infty} a_n\) instead of \(\lim\sup_{n \to \infty} a_n\), and \(\underline{\lim}_{n \to \infty} a_n\) instead of \(\lim\inf_{n \to \infty} a_n\).
Note that the starting index \(m\) of the sequence is irrelevant.
\end{remark}

\begin{note}
Returning to the piston analogy, imagine a piston at \(+\infty\) moving leftward until it is stopped by the presence of the sequence \(a_1, a_2, \dots\).
The place it will stop is the supremum of \(a_1, a_2, a_3, \dots\), which in our new notation is \(a_1^+\).
Now let us remove the first element \(a_1\) from the sequence;
this may cause our piston to slip leftward, to a new point \(a_2^+\)
(though in many cases the piston will not move and \(a_2^+\) will just be the same as \(a_1^+\)).
Then we remove the second element \(a_2\), causing the piston to slip a little more.
If we keep doing this the piston will keep slipping, but there will be some point where it cannot go any further, and this is the limit superior of the sequence.
A similar analogy can describe the limit inferior of the sequence.
\end{note}

\begin{proposition}\label{6.4.12}
Let \((a_n)_{n = m}^\infty\) be a sequence of real numbers, let \(L^+\) be the limit superior of this sequence, and let \(L^-\) be the limit inferior of this sequence
(thus both \(L^+\) and \(L^-\) are extended real numbers).
\begin{enumerate}
    \item For every \(x > L^+\), there exists an \(N \geq m\) such that \(a_n < x\) for all \(n \geq N\).
    (In other words, for every \(x > L^+\), the elements of the sequence \((a_n)_{n = m}^\infty\) are eventually less than \(x\).)
    Similarly, for every \(y < L^-\) there exists an \(N \geq m\) such that \(a_n > y\) for all \(n \geq N\).
    \item For every \(x < L^+\), and every \(N \geq m\), there exists an \(n \geq N\) such that \(a_n > x\).
    (In other words, for every \(x < L^+\), the elements of the sequence \((a_n)_{n = m}^\infty\) exceed \(x\) infinitely often.)
    Similarly, for every \(y > L^-\) and every \(N \geq m\), there exists an \(n \geq N\) such that \(a_n < y\).
    \item We have \(\inf(a_n)_{n = m}^\infty \leq L^- \leq L^+ \leq \sup(a_n)_{n = m}^\infty\).
    \item If \(c\) is any limit point of \((a_n)_{n = m}^\infty\), then we have \(L^- \leq c \leq L^+\).
    \item If \(L^+\) is finite, then it is a limit point of \((a_n)_{n = m}^\infty\).
    Similarly, if \(L^-\) is finite, then it is a limit point of \((a_n)_{n = m}^\infty\).
    \item Let \(c\) be a real number.
    If \((a_n)_{n = m}^\infty\) converges to \(c\), then we must have \(L^+ = L^- = c\).
    Conversely, if \(L^+ = L^- = c\), then \((a_n)_{n = m}^\infty\) converges to \(c\).
\end{enumerate}
\end{proposition}

\begin{proof}{(a)}
Suppose first that \(x > L^+\).
Then by definition of \(L^+\), we have \(x > \inf(a_N^+)_{N = m}^\infty\).
By Remark \ref{6.3.7}, there must then exist an integer \(N \geq m\) such that \(x > a_N^+\).
By definition of \(a_N^+\), this means that \(x > \sup(a_n)_{n = N}^\infty\).
Thus by Proposition \ref{6.3.6}, we have \(x > a_n\) for all \(n \geq N\), as desired.

Now suppose that \(y < L^-\).
Then by definition of \(L^-\), we have \(y < \sup(a_N^-)_{N = m}^\infty\).
By Proposition \ref{6.3.6}, there must then exist an integer \(N \geq m\) such that \(y < a_N^-\).
By definition of \(a_N^-\), this means that \(y < \inf(a_n)_{n = N}^\infty\).
Thus by Remark \ref{6.3.7}, we have \(y < a_n\) for all \(n \geq N\), as desired.
\end{proof}

\begin{proof}{(b)}
Suppose that \(x < L^+\).
Then we have \(x < \inf(a_N^+)_{N = m}^\infty\).
If we fix any \(N \geq m\), then by Remark \ref{6.3.7}, we thus have \(x < a_N^+\).
By definition of \(a_N^+\), this means that \(x < \sup(a_n)_{n = N}^\infty\).
By Proposition \ref{6.3.6}, there must thus exist \(n \geq N\) such that \(a_n > x\), as desired.

Now suppose that \(y > L^-\).
Then we have \(y > \sup(a_N^-)_{N = m}^\infty\).
If we fix any \(N \geq m\), then by Proposition \ref{6.3.6}, we thus have \(y > a_N^-\).
By definition of \(a_N^-\), this means that \(y > \inf(a_n)_{n = N}^\infty\).
By Remark \ref{6.3.7}, there must thus exist \(n \geq N\) such that \(a_n < y\), as desired.
\end{proof}

\begin{proof}{(c)}
We first show that \(\inf(a_n)_{n = m}^\infty \leq L^-\).
\begin{align*}
L^- &= \lim\inf_{n \to \infty} a_n & \text{(by Definition \ref{6.4.6})} \\
&= \sup(a_N^-)_{N = m}^\infty & \text{(by Definition \ref{6.4.6})} \\
&\geq a_m^- & \text{(by Definition \ref{6.3.1})} \\
&= \inf(a_n)_{n = m}^\infty. & \text{(by Definition \ref{6.4.6})}
\end{align*}

Next we show that \(L^+ \leq \sup(a_n)_{n = m}^\infty\).
\begin{align*}
L^+ &= \lim\sup_{n \to \infty} a_n & \text{(by Definition \ref{6.4.6})} \\
&= \inf(a_N^+)_{N = m}^\infty & \text{(by Definition \ref{6.4.6})} \\
&\leq a_m^+ & \text{(by Definition \ref{6.3.1})} \\
&= \sup(a_n)_{n = m}^\infty. & \text{(by Definition \ref{6.4.6})}
\end{align*}

Finally we show that \(L^- \leq L^+\).
Suppose for sake of contradiction that \(L^- > L^+\).
\begin{align*}
& L^+ < L^- \\
\implies & \inf(a_N^+)_{N = m}^\infty < \sup(a_N^-)_{N = m}^\infty & \text{(by Definition \ref{6.4.6})} \\
\implies & \exists\ N_1 \geq m : a_{N_1}^+ < \sup(a_N^-)_{N = m}^\infty & \text{(by Remark \ref{6.3.7})} \\
\implies & \exists\ N_2 \geq m : a_{N_1}^+ < a_{N_2}^- & \text{(by Proposition \ref{6.3.6})} \\
\implies & \sup(a_n)_{n = N_1}^\infty < \inf(a_n)_{n = N_2}^\infty. & \text{(by Definition \ref{6.4.6})}
\end{align*}
Let \(N = \max(N_1, N_2)\).
Then \(\forall\ n \geq N\), we have
\[
    a_n \leq \sup(a_n)_{n = N}^\infty \leq \sup(a_n)_{n = N_1}^\infty < \inf(a_n)_{n = N_2}^\infty \leq \inf(a_n)_{n = N}^\infty \leq a_n,
\]
a contradiction.
Thus \(L^- \leq L^+\).
And we conclude that \(\inf(a_n)_{n = m}^\infty \leq L^- \leq L^+ \leq \sup(a_n)_{n = m}^\infty\).
\end{proof}

\begin{proof}{(d)}
Suppose that \(c\) is a limit point of \((a_n)_{n = m}^\infty\).
We first show that \(L^- \leq c\).
Suppose for sake of contradiction that \(L^- > c\).
By Proposition \ref{6.4.12}(a), \(\exists\ N \in \mathds{N}\) and \(N \geq m\) such that \(a_n > c \ \forall\ n \geq N\).
Since \(c\) is a limit point, \(\forall\ \varepsilon \in \mathds{R}\) and \(\varepsilon > 0\), there must \(\exists\ n \geq N\) such that \(\abs*{a_n - c} \leq \varepsilon\).
But we also have \(a_n > c\), so when \(\varepsilon = (a_n - c) / 2\) we must also have \(\abs*{a_n - c} = a_n - c \leq \varepsilon = (a_n - c) / 2\), a contradiction.
Thus \(L^- \leq c\).

Now we show that \(L^+ \geq c\).
Suppose for sake of contradiction that \(L^+ < c\).
By Proposition \ref{6.4.12}(a), \(\exists\ N \in \mathds{N}\) and \(N \geq m\) such that \(a_n < c \ \forall\ n \geq N\).
Since \(c\) is a limit point, \(\forall\ \varepsilon \in \mathds{R}\) and \(\varepsilon > 0\), there must \(\exists\ n \geq N\) such that \(\abs*{a_n - c} \leq \varepsilon\).
But we also have \(a_n < c\), so when \(\varepsilon = (c - a_n) / 2\) we must also have \(\abs*{a_n - c} = c - a_n \leq \varepsilon = (c - a_n) / 2\), a contradiction.
Thus \(L^+ \geq c\).
And we conclude that \(L^- \leq c \leq L^+\).
\end{proof}

\begin{proof}{(e)}
We first show that if \(L^+\) is finite, then \(L^+\) is a limit point of \((a_n)_{n = m}^\infty\).
By Definition \ref{6.4.1}, we need to show that \(\forall\ \varepsilon \in \mathds{R}\) and \(\varepsilon > 0\), \(\forall\ N \in \mathds{N}\) and \(N \geq m\), \(\exists\ n \geq N\) such that \(\abs*{a_n - L^+} \leq \varepsilon\).
Since \(L^+ < L^+ + \varepsilon\), by Proposition \ref{6.4.12}(a) \(\exists\ N' \in \mathds{N}\) and \(N' \geq m\) such that \(a_n < L^+ + \varepsilon \ \forall\ n \geq N'\).
But this also means \(\forall\ N \geq m\), \(\exists\ n \geq N\) such that \(a_n < L^+ + \varepsilon\) as long as we always choose \(n \geq \max(N', N)\).
Similarly, since \(L^+ - \varepsilon < L^+\), by Proposition \ref{6.4.12}(b), \(\forall\ N \in \mathds{N}\) and \(N \geq m\), \(\exists\ n \geq N\) such that \(a_n > L^+ - \varepsilon\).
Combine the two statements above we have \(\forall\ N \geq m\), \(\exists\ n \geq N\) such that \(L^+ - \varepsilon < a_n < L^+ + \varepsilon\).
This means \(-\varepsilon < a_n - L^+ < \varepsilon\), so \(\abs*{a_n - L^+} < \varepsilon\).
Thus \(L^+\) is a limit point.

Now we show that if \(L^-\) is finite, then \(L^-\) is a limit point of \((a_n)_{n = m}^\infty\).
By Definition \ref{6.4.1}, we need to show that \(\forall\ \varepsilon \in \mathds{R}\) and \(\varepsilon > 0\), \(\forall\ N \in \mathds{N}\) and \(N \geq m\), \(\exists\ n \geq N\) such that \(\abs*{a_n - L^-} \leq \varepsilon\).
Since \(L^- < L^- + \varepsilon\), by Proposition \ref{6.4.12}(b), \(\forall\ N \in \mathds{N}\) and \(N \geq m\), \(\exists\ n \geq N\) such that \(a_n < L^- + \varepsilon\).
Similarly, since \(L^- - \varepsilon < L^-\), by Proposition \ref{6.4.12}(a) \(\exists\ N' \in \mathds{N}\) and \(N' \geq m\) such that \(a_n > L^+ - \varepsilon \ \forall\ n \geq N'\).
But this also means \(\forall\ N \geq m\), \(\exists\ n \geq N\) such that \(a_n > L^- - \varepsilon\) as long as we always choose \(n \geq \max(N', N)\).
Combine the two statements above we have \(\forall\ N \geq m\), \(\exists\ n \geq N\) such that \(L^- - \varepsilon < a_n < L^- + \varepsilon\).
This means \(-\varepsilon < a_n - L^- < \varepsilon\), so \(\abs*{a_n - L^-} < \varepsilon\).
Thus \(L^-\) is also a limit point.
\end{proof}

\begin{proof}{(f)}
We first show that if \(\lim_{n \to \infty} a_n = c\), then \(L^+ = L^- = c\).
By Proposition \ref{6.4.5}, \(c\) is the only limit point of \((a_n)_{n = m}^\infty\).
And by Proposition \ref{6.4.12}(d), we have \(L^- \leq c \leq L^+\).
Suppose for sake of contradiction that \(c \neq L^+\).
Then we must have \(c < L^+\).
By Proposition \ref{6.4.12}(b), \(\forall\ N_1 \in \mathds{N}\) and \(N_1 \geq m\), \(\exists\ n_1 \geq N\) such that \(a_{n_1} > c\).
Since \(\lim_{n \to \infty} a_n = c\), by Definition \ref{6.1.8} we must have \(\forall\ \varepsilon \in \mathds{R}\) and \(\varepsilon > 0\), \(\exists\ N_2 \geq m\) such that \(\abs*{a_{n_2} - c} \leq \varepsilon \ \forall\ n_2 \geq N_2\).
Let \(N = \max(N_1, N_2)\).
Then \(\forall\ N \geq m\), \(\exists\ n \geq N\) such that \(a_n > c\) and \(\abs*{a_n - c} \leq \varepsilon\).
But this means when \(\varepsilon = (a_n - c) / 2\) we must also have \(\abs*{a_n - c} = a_n - c \leq (a_n - c) / 2\), a contradiction.
Thus \(c = L^+\).
Similarly suppose for sake of contradiction that \(c \neq L^-\).
Then we must have \(c > L^-\).
By Proposition \ref{6.4.12}(b), \(\forall\ N_1 \in \mathds{N}\) and \(N_1 \geq m\), \(\exists\ n_1 \geq N\) such that \(a_{n_1} < c\).
Since \(\lim_{n \to \infty} a_n = c\), by Definition \ref{6.1.8} we must have \(\forall\ \varepsilon \in \mathds{R}\) and \(\varepsilon > 0\), \(\exists\ N_2 \geq m\) such that \(\abs*{a_{n_2} - c} \leq \varepsilon \ \forall\ n_2 \geq N_2\).
Let \(N = \max(N_1, N_2)\).
Then \(\forall\ N \geq m\), \(\exists\ n \geq N\) such that \(a_n < c\) and \(\abs*{a_n - c} \leq \varepsilon\).
But this means when \(\varepsilon = (c - a_n) / 2\) we must also have \(\abs*{a_n - c} = c - a_n \leq (c - a_n) / 2\), a contradiction.
Thus \(c = L^-\).
And we conclude that if \(\lim_{n \to \infty} a_n = c\), then \(L^+ = L^- = c\).

Now we show that if \(L^+ = L^- = c\), then \(\lim_{n \to \infty} a_n = c\).
Let \(\varepsilon \in \mathds{R}\) and \(\varepsilon > 0\).
Then we have \(c - \varepsilon < c < c + \varepsilon\).
Since \(c = L^+\), by Proposition \ref{6.4.11}(a), \(\exists\ N_1 \in \mathds{N}\) and \(N_1 \geq m\) such that \(a_{n_1} < c + \varepsilon \ \forall\ n_1 \in \mathds{N}\) and \(n_1 \geq N_1\).
Similarly since \(c = L^-\), by Proposition \ref{6.4.11}(a), \(\exists\ N_2 \in \mathds{N}\) and \(N_2 \geq m\) such that \(c - \varepsilon < a_{n_2} \ \forall\ n_2 \in \mathds{N}\) and \(n_2 \geq N_2\).
Let \(N = \max(N_1, N_2)\).
Then \(\exists\ N \geq m\) such that \(c - \varepsilon < a_n < c + \varepsilon \ \forall\ n \geq N\).
But this means \(-\varepsilon < a_n - c < \varepsilon\), therefore \(\abs*{a_n - c} < \varepsilon\).
Thus \(\lim_{n \to \infty} a_n = c\).
And we conclude that if \(L^+ = L^- = c\), then \(\lim_{n \to \infty} a_n = c\).
\end{proof}

\begin{note}
Parts (c) and (d) of Proposition \ref{6.4.12} say, in particular, that \(L^+\) is the largest limit point of \((a_n)_{n = m}^\infty\), and \(L^-\) is the smallest limit point
(providing that \(L^+\) and \(L^-\) are finite).
Proposition \ref{6.4.12} (f) then says that if \(L^+\) and \(L^-\) coincide (so there is only one limit point), then the sequence in fact converges.
This gives a way to test if a sequence converges: compute its limit superior and limit inferior, and see if they are equal.
\end{note}

\begin{lemma}[Comparison principle]\label{6.4.13}
Suppose that \((a_n)_{n = m}^\infty\) and \((b_n)_{n = m}^\infty\) are two sequences of real numbers such that \(a_n \leq b_n\) for all \(n \geq m\).
Then we have the inequalities
\begin{align*}
\sup(a_n)_{n = m}^\infty &\leq \sup(b_n)_{n = m}^\infty \\
\inf(a_n)_{n = m}^\infty &\leq \inf(b_n)_{n = m}^\infty \\
\lim\sup_{n \to \infty} a_n &\leq \lim\sup_{n \to \infty} b_n \\
\lim\inf_{n \to \infty} a_n &\leq \lim\inf_{n \to \infty} b_n
\end{align*}
\end{lemma}

\begin{proof}
We first show that \(\sup(a_n)_{n = m}^\infty \leq \sup(b_n)_{n = m}^\infty\).
Suppose for sake of contradiction that \(\sup(a_n)_{n = m}^\infty > \sup(b_n)_{n = m}^\infty\).
Then by Proposition \ref{6.3.6}, \(\exists\ n' \geq m\) such that \(a_{n'} > \sup(b_n)_{n = m}^\infty\).
Also by Definition \ref{6.3.1}, \(\forall\ n \geq m\) we have \(\sup(b_n)_{n = m}^\infty \geq b_n\).
But this means \(a_{n'} > b_n\), in particular, \(a_{n'} > b_{n'}\), a contradiction.
Thus we must have \(\sup(a_n)_{n = m}^\infty \leq \sup(b_n)_{n = m}^\infty\).

Next we show that \(\inf(a_n)_{n = m}^\infty \leq \inf(b_n)_{n = m}^\infty\).
Suppose for sake of contradiction that \(\inf(a_n)_{n = m}^\infty > \inf(b_n)_{n = m}^\infty\).
Then by Remark \ref{6.3.7}, \(\exists\ n' \geq m\) such that \(b_{n'} < \inf(a_n)_{n = m}^\infty\).
Also by Definition \ref{6.3.1}, \(\forall\ n \geq m\) we have \(\inf(a_n)_{n = m}^\infty \leq a_n\).
But this means \(b_{n'} < a_n\), in particular, \(b_{n'} < a_{n'}\), a contradiction.
Thus we must have \(\inf(a_n)_{n = m}^\infty \leq \inf(b_n)_{n = m}^\infty\).

Next we show that \(\lim\sup_{n \to \infty} a_n \leq \lim\sup_{n \to \infty} b_n\).
Suppose for sake of contradiction that \(\lim\sup_{n \to \infty} a_n > \lim\sup_{n \to \infty} b_n\).
By Definition \ref{6.4.6}, we have \(\inf(a_n^+)_{n = m}^\infty > \inf(b_n^+)_{n = m}^\infty\).
By Remark \ref{6.3.7}, \(\exists\ n' \geq m\) such that \(b_{n'}^+ < \inf(a_n^+)_{n = m}^\infty\).
By Definition \ref{6.3.1}, \(\forall\ n \geq m\) we have \(\inf(a_n^+)_{n = m}^\infty \leq a_n^+\).
But this means \(b_{n'}^+ < a_n^+\), in particular, \(b_{n'}^+ < a_{n'}^+\).
Again by Definition \ref{6.4.6}, we have \(\sup(b_n)_{n = n'}^\infty < \sup(a_n)_{n = n'}^\infty\).
But this contradict to the proof above that \(\sup(b_n)_{n = n'}^\infty \geq \sup(a_n)_{n = n'}^\infty\).
Thus we must have \(\lim\sup_{n \to \infty} a_n \leq \lim\sup_{n \to \infty} b_n\).

Finally we show that \(\lim\inf_{n \to \infty} a_n \leq \lim\inf_{n \to \infty} b_n\).
Suppose for sake of contradiction that \(\lim\inf_{n \to \infty} a_n > \lim\inf_{n \to \infty} b_n\).
By Definition \ref{6.4.6}, we have \(\sup(a_n^-)_{n = m}^\infty > \sup(b_n^-)_{n = m}^\infty\).
By Proposition \ref{6.3.6}, \(\exists\ n' \geq m\) such that \(a_{n'}^- > \sup(b_n^-)_{n = m}^\infty\).
By Definition \ref{6.3.1}, \(\forall\ n \geq m\) we have \(\sup(b_n^-)_{n = m}^\infty \geq b_n^-\).
But this means \(a_{n'}^- > b_n^-\), in particular, \(a_{n'}^- > b_{n'}^-\).
Again by Definition \ref{6.4.6}, we have \(\inf(a_n)_{n = n'}^\infty > \inf(b_n)_{n = n'}^\infty\).
But this contradict to the proof above that \(\inf(a_n)_{n = n'}^\infty \leq \inf(b_n)_{n = n'}^\infty\).
Thus we must have \(\lim\inf_{n \to \infty} a_n \leq \lim\inf_{n \to \infty} b_n\).
\end{proof}

\begin{corollary}[Squeeze test]\label{6.4.14}
Let \((a_n)_{n = m}^\infty\), \((b_n)_{n = m}^\infty\), and \((c_n)_{n = m}^\infty\) be sequences of real numbers such that
\[
    a_n \leq b_n \leq c_n
\]
for all \(n \geq m\).
Suppose also that \((a_n)_{n = m}^\infty\) and \((c_n)_{n = m}^\infty\) both converge to the same limit \(L\).
Then \((b_n)_{n = m}^\infty\) is also convergent to \(L\).
\end{corollary}

\begin{proof}
Since \(\lim_{n \to \infty} a_n = L\), by Proposition \ref{6.4.5} \(L\) is the only limit point.
And by Proposition \ref{6.4.12}, \(\lim\sup_{n \to \infty} a_n = \lim\inf_{n \to \infty} a_n = L\).
Similarly, \(\lim\sup_{n \to \infty} c_n = \lim\inf_{n \to \infty} c_n = L\).
So
\begin{align*}
& a_n \leq b_n \leq c_n \\
\implies & \lim\sup_{n \to \infty} a_n \leq \lim\sup_{n \to \infty} b_n \leq \lim\sup_{n \to \infty} c_n & \text{(by Lemma \ref{6.4.13})} \\
\implies & L \leq \lim\sup_{n \to \infty} b_n \leq L & \text{(by Lemma \ref{6.4.13})} \\
\implies & \lim\sup_{n \to \infty} b_n = L.
\end{align*}
And
\begin{align*}
& a_n \leq b_n \leq c_n \\
\implies & \lim\inf_{n \to \infty} a_n \leq \lim\inf_{n \to \infty} b_n \leq \lim\inf_{n \to \infty} c_n & \text{(by Lemma \ref{6.4.13})} \\
\implies & L \leq \lim\inf_{n \to \infty} b_n \leq L & \text{(by Lemma \ref{6.4.13})} \\
\implies & \lim\inf_{n \to \infty} b_n = L.
\end{align*}
Since \(\lim\sup_{n \to \infty} b_n = \lim\inf_{n \to \infty} b_n = L\), by Proposition \ref{6.4.12} \(\lim_{n \to \infty} b_n = L\).
\end{proof}

\setcounter{theorem}{15}
\begin{remark}\label{6.4.16}
The squeeze test, combined with the limit laws and the principle that monotone bounded sequences always have limits, allows to compute a large number of limits.
\end{remark}

\begin{corollary}[Zero test for sequences]\label{6.4.17}
Let \((a_n)_{n = M}^\infty\) be a sequence of real numbers.
Then the limit \(\lim_{n \to \infty} a_n\) exists and is equal to zero if and only if the limit \(\lim_{n \to \infty} \abs*{a_n}\) exists and is equal to zero.
\end{corollary}

\begin{proof}
We first show that the limit \(\lim_{n \to \infty} a_n\) exists and is equal to zero implies the limit \(\lim_{n \to \infty} \abs*{a_n}\) exists and is equal to zero.
\(\forall\ \varepsilon \in \mathds{R}\) and \(\varepsilon > 0\),
\begin{align*}
& \lim_{n \to \infty} a_n = 0 \\
\implies & \abs*{a_n - 0} \leq \varepsilon \\
\implies & \abs*{a_n} \leq \varepsilon \\
\implies & \abs*{\abs*{a_n} - 0} \leq \varepsilon \\
\implies & \lim_{n \to \infty} \abs*{a_n} = 0.
\end{align*}

Now we show that the limit \(\lim_{n \to \infty} \abs*{a_n}\) exists and is equal to zero implies the limit \(\lim_{n \to \infty} a_n\) exists and is equal to zero.
Since \(-\abs*{a_n} \leq a_n \leq \abs*{a_n}\), we have
\begin{align*}
& \lim_{n \to \infty} \abs*{a_n} = 0 \\
\implies & \lim_{n \to \infty} -\abs*{a_n} = 0 \\
\implies & \lim_{n \to \infty} a_n = 0. & \text{(by Corollary \ref{6.4.14})}
\end{align*}
And we conclude that the limit \(\lim_{n \to \infty} a_n\) exists and is equal to zero if and only if the limit \(\lim_{n \to \infty} \abs*{a_n}\) exists and is equal to zero.
\end{proof}

\begin{theorem}[Completeness of the reals]\label{6.4.18}
A sequence \((a_n)_{n = 1}^\infty\) of real numbers is a Cauchy sequence if and only if it is convergent.
\end{theorem}

\begin{proof}
Proposition \ref{6.1.12} already tells us that every convergent sequence is Cauchy, so it suffices to show that every Cauchy sequence is convergent.

Let \((a_n)_{n = 1}^\infty\) be a Cauchy sequence.
We know from Lemma \ref{5.1.15} (or more precisely, from the extension of this lemma to the real numbers, which is proven in exactly the same fashion) that the sequence \((a_n)_{n = 1}^\infty\) is bounded;
by Lemma \ref{6.4.13} (or Proposition \ref{6.4.12}(c)) this implies that \(L^- \coloneqq \lim\inf_{n \to \infty} a_n\) and \(L^+ \coloneqq \lim\sup_{n \to \infty} a_n\) of the sequence are both finite.
To show that the sequence converges, it will suffice by Proposition \ref{6.4.12}(f) to show that \(L^- = L^+\).

Now let \(\varepsilon > 0\) be any real number.
Since \((a_n)_{n = 1}^\infty\) is a Cauchy sequence, it must be eventually \(\varepsilon\)-steady, so in particular there exists an \(N \geq 1\) such that the sequence \((a_n)_{n = N}^\infty\) is \(\varepsilon\)-steady.
In particular, we have \(a_N - \varepsilon \leq a_n \leq a_N + \varepsilon\) for all \(n \geq N\).
By Proposition \ref{6.3.6} (or Lemma \ref{6.4.13}) this implies that
\[
    a_N - \varepsilon \leq \inf(a_n)_{n = N}^\infty \leq \sup(a_n)_{n = N}^\infty \leq a_N + \varepsilon
\]
and hence by the definition of \(L^-\) and \(L^+\) (and Proposition \ref{6.3.6} again)
\[
    a_N - \varepsilon \leq L^- \leq L^+ \leq a_N + \varepsilon.
\]
Thus we have
\[
    0 \leq L^+ - L^- \leq 2\varepsilon.
\]
But this is true for all \(\varepsilon > 0\), and \(L^+\) and \(L^-\) do not depend on \(\varepsilon\);
so we must therefore have \(L^+ = L^-\).
(If \(L^+ > L^-\) then we could set \(\varepsilon \coloneqq (L^+ - L^-) / 3\) and obtain a contradiction.)
By Proposition \ref{6.4.12}(f) we thus see that the sequence converges.
\end{proof}

\begin{remark}\label{6.4.19}
While Theorem \ref{6.4.18} is very similar in spirit to Proposition \ref{6.1.15}, it is a bit more general, since Proposition \ref{6.1.15} refers to Cauchy sequences of rationals instead of real numbers.
\end{remark}

\begin{remark}\label{6.4.20}
In the language of metric spaces, Theorem \ref{6.4.18} asserts that the real numbers are a \emph{complete} metric space
- that they do not contain ``holes'' the same way the rationals do.
(Certainly the rationals have lots of Cauchy sequences which do not converge to other rationals;
take for instance the sequence \(1, 1.4, 1.41, 1.414, 1.4142, \dots\) which converges to the irrational \(\sqrt{2}\).)
This property is closely related to the least upper bound property (Theorem \ref{5.5.9}), and is one of the principal characteristics which make the real numbers superior to the rational numbers for the purposes of doing analysis
(taking limits, taking derivatives and integrals, finding zeroes of functions, that kind of thing).
\end{remark}

\exercisesection

\begin{exercise}\label{ex 6.4.1}
Prove Proposition \ref{6.4.5}.
\end{exercise}

\begin{proof}
See Proposition \ref{6.4.5}.
\end{proof}

\begin{exercise}\label{ex 6.4.2}
Let \((a_n)_{n = m}^\infty\) be a sequence of real numbers, let \(c\) be a real number, let \(m' \geq m\) be an integer, and let \(k \geq 0\) be a non-negative integer.
Show that
\begin{enumerate}
    \item \(c\) is a limit point of \((a_n)_{n = m}^\infty\) if and only if \(c\) is a limit point of \((a_n)_{n = m'}^\infty\).
    \item \(c\) is the limit superior of \((a_n)_{n = m}^\infty\) if and only if \(c\) is the limit superior of \((a_n)_{n = m'}^\infty\).
    \item \(c\) is the limit inferior of \((a_n)_{n = m}^\infty\) if and only if \(c\) is the limit inferior of \((a_n)_{n = m'}^\infty\).
    \item \(c\) is a limit point of \((a_n)_{n = m}^\infty\) if and only if \(c\) is a limit point of \((a_{n + k})_{n = m}^\infty\).
    \item \(c\) is the limit superior of \((a_n)_{n = m}^\infty\) if and only if \(c\) is the limit superior of \((a_{n + k})_{n = m}^\infty\).
    \item \(c\) is the limit inferior of \((a_n)_{n = m}^\infty\) if and only if \(c\) is the limit inferior of \((a_{n + k})_{n = m}^\infty\).
\end{enumerate}
\end{exercise}

\begin{proof}{(a)}
We first show that \(c\) is a limit point of \((a_n)_{n = m}^\infty\) implies \(c\) is a limit point of \((a_n)_{n = m'}^\infty\).
Since \(c\) is a limit point of \((a_n)_{n = m}^\infty\), \(\forall\ \varepsilon \in \mathds{R}\) and \(\varepsilon > 0\), we have \(\forall\ N \in \mathds{N}\) and \(N \geq m\), \(\exists\ n \geq N\) such that \(\abs*{a_n - c} \leq \varepsilon\).
Since \(m' \geq m\), we must also have \(\forall\ N \geq m'\), \(\exists\ n \geq N\) such that \(\abs*{a_n - c} \leq \varepsilon\).
This means \(c\) is a limit point of \((a_n)_{n = m'}^\infty\).
Thus \(c\) is a limit point of \((a_n)_{n = m}^\infty\) implies \(c\) is a limit point of \((a_n)_{n = m'}^\infty\).

Now we show that \(c\) is a limit point of \((a_n)_{n = m'}^\infty\) implies \(c\) is a limit point of \((a_n)_{n = m}^\infty\).
Since \(c\) is a limit point of \((a_n)_{n = m'}^\infty\), \(\forall\ \varepsilon \in \mathds{R}\) and \(\varepsilon > 0\), we have \(\forall\ N \in \mathds{N}\) and \(N \geq m'\), \(\exists\ n \geq N\) such that \(\abs*{a_n - c} \leq \varepsilon\).
Since \(m' \geq m\), we only need to show that \(\forall\ N' \in \mathds{N}\) and \(m \leq N' < m'\), \(\exists\ n \geq N'\) such that \(\abs*{a_n - c} \leq \varepsilon\).
Since \(N \geq m' > N'\), we can always find an \(n \geq N > N'\) such that \(\abs*{a_n - c} \leq \varepsilon\).
This means \(c\) is a limit point of \((a_n)_{n = m}^\infty\).
Thus \(c\) is a limit point of \((a_n)_{n = m'}^\infty\) implies \(c\) is a limit point of \((a_n)_{n = m}^\infty\).
And combine with the proof above we conclude that \(c\) is a limit point of \((a_n)_{n = m}^\infty\) if and only if \(c\) is a limit point of \((a_n)_{n = m'}^\infty\).
\end{proof}

\begin{proof}{(b)}
By Definition \ref{6.4.6}, \(\lim\sup_{n \to \infty} a_n = \inf(a_N^+)_{N = m}^\infty\) and \(a_N^+ = \sup(a_n)_{n = N}^\infty\).
So
\begin{align*}
& m \leq m' \\
\implies & \{a_n : n \geq m'\} \subseteq \{a_n : n \geq m\} \\
\implies & \{a_n : n \geq m'\} \subseteq \big(\{a_n : n \geq m'\} \cup \{a_n : m \leq n < m'\}\big) \\
\implies & \sup(a_n)_{n = m}^\infty \geq \sup(a_n)_{n = m'}^\infty \\
\implies & a_m^+ \geq a_{m'}^+.
\end{align*}
Thus
\begin{align*}
& (m \leq m') \land (a_m^+ \geq a_{m'}^+) \\
\implies & \big(\{a_N^+ : n \geq m'\} \subseteq \{a_N^+ : n \geq m\}\big) \land (a_m^+ \geq a_{m'}^+)\\
\implies & \bigg(\{a_N^+ : n \geq m'\} \subseteq \big(\{a_N^+ : n \geq m'\} \cup \{a_N^+ : m \leq n < m'\}\big)\bigg) \\
& \land (a_m^+ \geq a_{m'}^+) \\
\implies & \inf(a_N^+)_{N = m}^\infty = \inf(a_N^+)_{N = m'}^\infty \\
\implies & c = \lim\sup_{n \to \infty} a_n = \inf(a_N^+)_{N = m}^\infty = \inf(a_N^+)_{N = m'}^\infty.
\end{align*}
\end{proof}

\begin{proof}{(c)}
By Definition \ref{6.4.6}, \(\lim\inf_{n \to \infty} a_n = \sup(a_N^+)_{N = m}^\infty\) and \(a_N^+ = \inf(a_n)_{n = N}^\infty\).
So
\begin{align*}
& m \leq m' \\
\implies & \{a_n : n \geq m'\} \subseteq \{a_n : n \geq m\} \\
\implies & \{a_n : n \geq m'\} \subseteq \big(\{a_n : n \geq m'\} \cup \{a_n : m \leq n < m'\}\big) \\
\implies & \inf(a_n)_{n = m}^\infty \leq \inf(a_n)_{n = m'}^\infty \\
\implies & a_m^+ \leq a_{m'}^+.
\end{align*}
Thus
\begin{align*}
& (m \leq m') \land (a_m^+ \leq a_{m'}^+) \\
\implies & \big(\{a_N^+ : n \geq m'\} \subseteq \{a_N^+ : n \geq m\}\big) \land (a_m^+ \leq a_{m'}^+)\\
\implies & \bigg(\{a_N^+ : n \geq m'\} \subseteq \big(\{a_N^+ : n \geq m'\} \cup \{a_N^+ : m \leq n < m'\}\big)\bigg) \\
& \land (a_m^+ \leq a_{m'}^+) \\
\implies & \sup(a_N^+)_{N = m}^\infty = \sup(a_N^+)_{N = m'}^\infty \\
\implies & c = \lim\inf_{n \to \infty} a_n = \sup(a_N^+)_{N = m}^\infty = \sup(a_N^+)_{N = m'}^\infty.
\end{align*}
\end{proof}

\begin{proof}{(d)}
Since \((a_{n + k})_{n = m}^\infty = (a_n)_{n = m + k}^\infty\), by Exercise \ref{ex 6.4.2}(a) we conclude that \(c\) is a limit point of \((a_n)_{n = m}^\infty\) if and only if \(c\) is a limit point of \((a_{n + k})_{n = m}^\infty\).
\end{proof}

\begin{proof}{(e)}
Since \((a_{n + k})_{n = m}^\infty = (a_n)_{n = m + k}^\infty\), by Exercise \ref{ex 6.4.2}(b) we conclude that \(c\) is the limit superior of \((a_n)_{n = m}^\infty\) if and only if \(c\) is the limit superior of \((a_{n + k})_{n = m}^\infty\).
\end{proof}

\begin{proof}{(f)}
Since \((a_{n + k})_{n = m}^\infty = (a_n)_{n = m + k}^\infty\), by Exercise \ref{ex 6.4.2}(c) we conclude that \(c\) is the limit inferior of \((a_n)_{n = m}^\infty\) if and only if \(c\) is the limit inferior of \((a_{n + k})_{n = m}^\infty\).
\end{proof}

\begin{exercise}\label{ex 6.4.3}
Prove Proposition \ref{6.4.12}.
\end{exercise}

\begin{proof}
See Proposition \ref{6.4.12}.
\end{proof}

\begin{exercise}\label{ex 6.4.4}
Prove Lemma \ref{6.4.13}.
\end{exercise}

\begin{proof}
See Lemma \ref{6.4.13}.
\end{proof}

\begin{exercise}\label{ex 6.4.5}
Use Lemma \ref{6.4.13} to prove Corollary \ref{6.4.14}.
\end{exercise}

\begin{proof}
See Corollary \ref{6.4.14}.
\end{proof}

\begin{exercise}\label{ex 6.4.6}
Give an example of two bounded sequences \((a_n)_{n = 1}^\infty\) and \((b_n)_{n = 1}^\infty\) such that \(a_n < b_n\) for all \(n \geq 1\), but that \(\sup(a_n)_{n = 1}^\infty \not< \sup(b_n)_{n = 1}^\infty\).
Explain why this does not contradict Lemma \ref{6.4.13}.
\end{exercise}

\begin{proof}
Let \(a_n = 1 / n\) and \(b_n = 2 / n\).
Then \(\lim_{n \to \infty} a_n = 0\) and \(\lim_{n \to \infty} b_n = 2 \times \lim_{n \to \infty} a_n = 0\).
Because non-negative sequence can converge to non-negative real number.
\end{proof}

\begin{exercise}\label{ex 6.4.7}
Prove Corollary \ref{6.4.17}.
Is the corollary still true if we replace zero in the statement of this Corollary by some other number?
\end{exercise}

\begin{proof}
See Corollary \ref{6.4.17}.

Now we show that Corollary \ref{6.4.17} is not true if we replace zero by some other number.
Let \(a_n = (-1)^n\).
So \(\lim_{n \to \infty} \abs*{a_n} = 1\), but \(\lim_{n \to \infty} a_n\) does not exist.
Thus Corollary \ref{6.4.17} is not true if we replace zero by some other number.
\end{proof}

\begin{exercise}\label{ex 6.4.8}
Let us say that a sequence \((a_n)_{n = M}^\infty\) of real numbers has \(+\infty\) as a limit point iff it has no finite upper bound, and that it has \(-\infty\) as a limit point iff it has no finite lower bound.
With this definition, show that \(\lim\sup_{n \to \infty} a_n\) is a limit point of \((a_n)_{n = M}^\infty\), and furthermore that it is larger than all the other limit points of \((a_n)_{n = M}^\infty\);
in other words, the limit superior is the largest limit point of a sequence.
Similarly, show that the limit inferior is the smallest limit point of a sequence.
\end{exercise}

\begin{proof}
We first show that the limit superior is the largest limit point of a sequence.
\begin{enumerate}
    \item If \((a_n)_{n = M}^\infty\) has a finite upper bound, then by Theorem \ref{6.2.11} \(\sup(a_n)_{n = M}^\infty\) exists.
    By Proposition \ref{6.4.12}, we have \(\lim\sup_{n \to \infty} a_n \leq \sup(a_n)_{n = M}^\infty\), thus \(\lim\sup_{n \to \infty} a_n\) is also finite.
    Again by Proposition \ref{6.4.12}, we have \(\lim\sup_{n \to \infty} a_n\) is a limit point, and is also the largest limit point.
    \item If \((a_n)_{n = M}^\infty\) has no finite upper bound, then by the given condition \((a_n)_{n = M}^\infty\) has \(+\infty\) as a limit point.
    Then by Proposition \ref{6.4.12}, we must have \(+\infty \leq \lim\sup_{n \to \infty} a_n\).
    But this means \(\lim\sup_{n \to \infty} a_n = +\infty\), so \(\lim\sup_{n \to \infty} a_n\) is the largest limit point.
\end{enumerate}
From all cases above we can conclude that \(\lim\sup_{n \to \infty} a_n\) is the largest limit point.

Now we show that the limit inferior is the smallest limit point of a sequence.
\begin{enumerate}
    \item If \((a_n)_{n = M}^\infty\) has a finite lower bound, then by Theorem \ref{6.2.11} \(\inf(a_n)_{n = M}^\infty\) exists.
    By Proposition \ref{6.4.12}, we have \(\lim\inf_{n \to \infty} a_n \geq \inf(a_n)_{n = M}^\infty\), thus \(\lim\inf_{n \to \infty} a_n\) is also finite.
    Again by Proposition \ref{6.4.12}, we have \(\lim\inf_{n \to \infty} a_n\) is a limit point, and is also the smallest limit point.
    \item If \((a_n)_{n = M}^\infty\) has no finite lower bound, then by the given condition \((a_n)_{n = M}^\infty\) has \(-\infty\) as a limit point.
    Then by Proposition \ref{6.4.12}, we must have \(-\infty \geq \lim\inf_{n \to \infty} a_n\).
    But this means \(\lim\inf_{n \to \infty} a_n = -\infty\), so \(\lim\inf_{n \to \infty} a_n\) is the smallest limit point.
\end{enumerate}
From all cases above we can conclude that \(\lim\inf_{n \to \infty} a_n\) is the smallest limit point.
\end{proof}