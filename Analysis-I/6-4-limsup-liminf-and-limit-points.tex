\section{Limsup, Liminf, and limit points}\label{sec 6.4}

\begin{definition}[Limit points]\label{6.4.1}
Let \((a_n)_{n = m}^\infty\) be a sequence of real numbers, let \(x\) be a real number, and let \(\varepsilon > 0\) be a real number.
We say that \(x\) is \emph{\(\varepsilon\)-adherent} to \((a_n)_{n = m}^\infty\) iff there exists an \(n \geq m\) such that \(a_n\) is \(\varepsilon\)-close to \(x\).
We say that \(x\) is \emph{continually \(\varepsilon\)-adherent} to \((a_n)_{n = m}^\infty\) iff it is \(\varepsilon\)-adherent to \((a_n)_{n = N}^\infty\) for every \(N \geq m\).
We say that \(x\) is a \emph{limit point} or \emph{adherent point} of \((a_n)_{n = m}^\infty\) iff it is continually \(\varepsilon\)-adherent to \((a_n)_{n = m}^\infty\) for every \(\varepsilon > 0\).
\end{definition}

\begin{remark}\label{6.4.2}
The verb ``to adhere'' means much the same as ``to stick to'';
hence the term ``adhesive''.
\end{remark}

\begin{note}
Unwrapping all the definitions, we see that \(x\) is a limit point of \((a_n)_{n = m}^\infty\) if, for every \(\varepsilon > 0\) and every \(N \geq m\), there exists an \(n \geq N\) such that \(\abs*{a_n - x} \leq \varepsilon\).
Note the difference between a sequence being \(\varepsilon\)-close to \(L\)
(which means that \emph{all} the elements of the sequence stay within a distance \(\varepsilon\) of \(L\))
and \(L\) being \(\varepsilon\)-adherent to the sequence
(which only needs a \emph{single} element of the sequence to stay within a distance \(\varepsilon\) of \(L\)).
Also, for \(L\) to be continually \(\varepsilon\)-adherent to \((a_n)_{n = m}^\infty\), it has to be \(\varepsilon\)-adherent to \((a_n)_{n = N}^\infty\) for \emph{all} \(N \geq m\), whereas for \((a_n)_{n = m}^\infty\) to be eventually \(\varepsilon\)-close to \(L\), we only need \((a_n)_{n = m}^\infty\) to be \(\varepsilon\)-close to \(L\) for \emph{some} \(N \geq m\).
Thus there are some subtle differences in quantifiers between limits and limit points.
\end{note}

\setcounter{theorem}{4}
\begin{proposition}[Limits are limit points]\label{6.4.5}
Let \((a_n)_{n = m}^\infty\) be a sequence which converges to a real number \(c\).
Then \(c\) is a limit point of \((a_n)_{n = m}^\infty\), and in fact it is the only limit point of \((a_n)_{n = m}^\infty\).
\end{proposition}

\begin{proof}
We first show that \(c\) is a limit point of \((a_n)_{n = m}^\infty\).
Since \(\lim_{n \to \infty} a_n = c\), we have \(\forall\ \varepsilon \in \mathds{R}\) and \(\varepsilon > 0\), \(\exists\ N_1 \in \mathds{N}\) and \(N_1 \geq m\) such that \(\abs*{a_n - c} \leq \varepsilon \ \forall\ n \geq N_1\).
This means \(\forall\ N_2 \in \mathds{N}\) and \(N_2 \geq m\), \(\exists\ n \geq \max(N_1, N_2)\) such that \(\abs*{a_n - c} \leq \varepsilon\).
So \(c\) is a limit point of \((a_n)_{n = m}^\infty\).

Now we show that \(c\) is the only limit point of \((a_n)_{n = m}^\infty\).
Suppose for sake of contradiction that \(c'\) is also a limit point of \((a_n)_{n = m}^\infty\) and \(c' \neq c\).
Since \(\lim_{n \to \infty} a_n = c\), we have \(\forall\ \varepsilon \in \mathds{R}\) and \(\varepsilon > 0\), \(\exists\ N \in \mathds{N}\) and \(N \geq m\) such that \(\abs*{a_n - c} \leq \varepsilon \ \forall\ n \geq N\).
In particular, \(\abs*{a_n - c} \leq \abs*{c - c'} / 2\).
So
\begin{align*}
\abs*{c - c'} &= \abs*{a_n - a_n + c - c'} \\
&= \abs*{c - a_n a_n - c'} \\
&\leq \abs*{c - a_n} + \abs*{a_n - c'} \\
&= \abs*{a_n - c} + \abs*{a_n - c'}.
\end{align*}
Which means \(\abs*{a_n - c'} \geq \abs*{c - c'} - \abs*{a_n - c}\).
Then \(\forall\ n \geq N\),
\begin{align*}
\abs*{a_n - c'} &\geq \abs*{c - c'} - \abs*{a_n - c} \\
&\geq \abs*{c - c'} - \abs*{c - c'} / 2 \\
&= \abs*{c - c'} / 2.
\end{align*}
Because \(\abs*{c - c'} / 2 > 0\), we can not find an \(n \geq N\) where \(\abs*{a_n - c'} < \abs*{c - c'} / 2\), which contradict to \(c'\) is limit point.
Thus \(\lim_{n \to \infty} a_n = c\) is the only limit point of \((a_n)_{n = m}^\infty\).
\end{proof}

\begin{definition}[Limit superior and limit inferior]\label{6.4.6}
Suppose that \((a_n)_{n = m}^\infty\) is a sequence.
We define a new sequence \((a_N^+)_{N = m}^\infty\) by the formula
\[
    a_N^+ \coloneqq \sup(a_n)_{n = N}^\infty.
\]
More informally, \(a_N^+\) is the supremum of all the elements in the sequence from \(a_N\) onwards.
We then define the \emph{limit superior} of the sequence \((a_n)_{n = m}^\infty\), denoted \(\lim\sup_{n \to \infty} a_n\), by the formula
\[
    \lim\sup_{n \to \infty} a_n \coloneqq \inf(a_N^+)_{N = m}^\infty.
\]
Similarly, we can define
\[
    a_N^- \coloneqq \inf(a_n)_{n = N}^\infty
\]
and define the \emph{limit inferior} of the sequence \((a_n)_{n = m}^\infty\), denoted \(\lim\inf_{n \to \infty} a_n\), by the formula
\[
    \lim\inf_{n \to \infty} a_n \coloneqq \sup(a_N^-)_{N = m}^\infty.
\]
\end{definition}

\setcounter{theorem}{10}
\begin{remark}\label{6.4.11}
Some authors use the notation \(\lim_{n \to \infty} a_n\) instead of \(\lim\sup_{n \to \infty} a_n\), and \(\lim_{n \to \infty} a_n\) instead of \(\lim\inf_{n \to \infty} a_n\).
Note that the starting index \(m\) of the sequence is irrelevant.
\end{remark}

\begin{note}
Returning to the piston analogy, imagine a piston at \(+\infty\) moving leftward until it is stopped by the presence of the sequence \(a_1, a_2, \dots\).
The place it will stop is the supremum of \(a_1, a_2, a_3, \dots\), which in our new notation is \(a_1^+\).
Now let us remove the first element \(a_1\) from the sequence;
this may cause our piston to slip leftward, to a new point \(a_2^+\)
(though in many cases the piston will not move and \(a_2^+\) will just be the same as \(a_1^+\)).
Then we remove the second element \(a_2\), causing the piston to slip a little more.
If we keep doing this the piston will keep slipping, but there will be some point where it cannot go any further, and this is the limit superior of the sequence.
A similar analogy can describe the limit inferior of the sequence.
\end{note}

\begin{proposition}\label{6.4.12}
Let \((a_n)_{n = m}^\infty\) be a sequence of real numbers, let \(L^+\) be the limit superior of this sequence, and let \(L^-\) be the limit inferior of this sequence
(thus both \(L^+\) and \(L^-\) are extended real numbers).
\begin{enumerate}
    \item For every \(x > L^+\), there exists an \(N \geq m\) such that \(a_n < x\) for all \(n \geq N\).
    (In other words, for every \(x > L^+\), the elements of the sequence \((a_n)_{n = m}^\infty\) are eventually less than \(x\).)
    Similarly, for every \(y < L^-\) there exists an \(N \geq m\) such that \(a_n > y\) for all \(n \geq N\).
    \item For every \(x < L^+\), and every \(N \geq m\), there exists an \(n \geq N\) such that \(a_n > x\).
    (In other words, for every \(x < L^+\), the elements of the sequence \((a_n)_{n = m}^\infty\) exceed \(x\) infinitely often.)
    Similarly, for every \(y > L^-\) and every \(N \geq m\), there exists an \(n \geq N\) such that \(a_n < y\).
    \item We have \(\inf(a_n)_{n = m}^\infty \leq L^- \leq L^+ \leq \sup(a_n)_{n = m}^\infty\).
    \item If \(c\) is any limit point of \((a_n)_{n = m}^\infty\), then we have \(L^- \leq c \leq L^+\).
    \item If \(L^+\) is finite, then it is a limit point of \((a_n)_{n = m}^\infty\).
    Similarly, if \(L^-\) is finite, then it is a limit point of \((a_n)_{n = m}^\infty\).
    \item Let \(c\) be a real number.
    If \((a_n)_{n = m}^\infty\) converges to \(c\), then we must have \(L^+ = L^- = c\).
    Conversely, if \(L^+ = L^- = c\), then \((a_n)_{n = m}^\infty\) converges to \(c\).
\end{enumerate}
\end{proposition}

\exercisesection

\begin{exercise}\label{ex 6.4.1}
Prove Proposition \ref{6.4.5}.
\end{exercise}

\begin{proof}
See Proposition \ref{6.4.5}.
\end{proof}