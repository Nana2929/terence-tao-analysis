\section{Some standard limits}\label{sec 6.5}

\begin{additional corollary}\label{ac 6.5.1}
We have
\[
    \lim_{n \to \infty} c = c
\]
for any constant \(c\).
\end{additional corollary}

\begin{proof}
Let \((a_n)_{n = 1}^\infty\) be a constant sequence where \(a_n = c\) for all \(n \geq 1\), and let \(N \in \mathds{N}\).
Then \(\forall\ \varepsilon \in \mathds{R}\) and \(\varepsilon > 0\), \(\exists\ N \geq 1\) such that
\[
    \abs*{a_n - c} = \abs*{c - c} = 0 \leq \varepsilon \ \forall \ n \geq N.
\]
So by Definition \ref{6.1.8} we have \(\lim_{n \to \infty} a_n = \lim_{n \to \infty} c = c\).
\end{proof}

\begin{corollary}\label{6.5.1}
We have \(\lim_{n \to \infty} 1 / n^{1 / k} = 0\) for every integer \(k \geq 1\).
\end{corollary}

\begin{proof}
From Lemma \ref{5.6.6} we know that \(1 / n^{1 / k}\) is a decreasing function of \(n\), while being bounded below by \(0\).
By Additional Corollary \ref{ac 6.3.1} (for decreasing sequences instead of increasing sequences) we thus know that this sequence converges to some limit \(L \geq 0\):
\[
    L = \lim_{n \to \infty} 1 / n^{1 / k}.
\]
Raising this to the \(k^{th}\) power and using the limit laws (or more precisely, Theorem \ref{6.1.19}(b) and induction), we obtain
\[
    L^k = \lim_{n \to \infty} 1 / n.
\]
By Proposition \ref{6.1.11} we thus have \(L^k = 0\);
but this means that \(L\) cannot be positive (else \(L^k\) would be positive), so \(L = 0\), and we are done.
\end{proof}