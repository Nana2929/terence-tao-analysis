\section{Subsequences}\label{sec 6.6}

\begin{definition}[Subsequences]\label{6.6.1}
    Let \((a_n)_{n = 0}^\infty\) and \((b_n)_{n = 0}^\infty\) be sequences of real numbers.
    We say that \((b_n)_{n = 0}^\infty\) is a \emph{subsequence} of \((a_n)_{n = 0}^\infty\) iff there exists a function \(f : \mathbf{N} \to \mathbf{N}\) which is strictly increasing (i.e., \(f(n + 1) > f(n)\) for all \(n \in \mathbf{N}\)) such that
    \[
        b_n = a_{f(n)} \text{ for all } n \in \mathbf{N}.
    \]
\end{definition}

\setcounter{theorem}{3}
\begin{lemma}\label{6.6.4}
    Let \((a_n)_{n = 0}^\infty\) and \((b_n)_{n = 0}^\infty\) be sequences of real numbers.
    Then \((a_n)_{n = 0}^\infty\) is a subsequence of \((a_n)_{n = 0}^\infty\).
    Furthermore, if \((b_n)_{n = 0}^\infty\) is a subsequence of \((a_n)_{n = 0}^\infty\), and \((c_n)_{n = 0}^\infty\) is a subsequence of \((b_n)_{n = 0}^\infty\), then \((c_n)_{n = 0}^\infty\) is a subsequence of \((a_n)_{n = 0}^\infty\).
\end{lemma}

\begin{proof}
    We first show that \((a_n)_{n = 0}^\infty\) is a subsequence of \((a_n)_{n = 0}^\infty\).
    Let \(f : \mathbf{N} \to \mathbf{N}\) be a function where \(f(n) = n\).
    Since \(f\) is strictly increasing, \((a_n)_{n = 0}^\infty\) is a subsequence of \((a_n)_{n = 0}^\infty\).

    Now we show that if \((b_n)_{n = 0}^\infty\) is a subsequence of \((a_n)_{n = 0}^\infty\), and \((c_n)_{n = 0}^\infty\) is a subsequence of \((b_n)_{n = 0}^\infty\), then \((c_n)_{n = 0}^\infty\) is a subsequence of \((a_n)_{n = 0}^\infty\).
    Since \((b_n)_{n = 0}^\infty\) is a subsequence of \((a_n)_{n = 0}^\infty\), \(\exists\ f : \mathbf{N} \to \mathbf{N}\) such that \(f\) is strictly increasing and \(b_n = a_{f(n)}\).
    Since \((c_n)_{n = 0}^\infty\) is a subsequence of \((b_n)_{n = 0}^\infty\), \(\exists\ g : \mathbf{N} \to \mathbf{N}\) such that \(f\) is strictly increasing and \(c_n = b_{g(n)}\).
    Let \(h = g \circ f\).
    Since \(f\) is strictly increasing, \(n_1 < n_2 \implies f(n_1) < f(n_2)\).
    Since \(g\) is strictly increasing, \(f(n_1) < f(n_2) \implies g(f(n_1)) < g(f(n_2))\).
    Thus \(h\) is also strictly increasing, and \((c_n)_{n = 0}^\infty\) is a subsequence of \((a_n)_{n = 0}^\infty\) where \(c_n = a_{g(f(n))}\).
\end{proof}

\begin{proposition}[Subsequences related to limits]\label{6.6.5}
    Let \((a_n)_{n = 0}^\infty\) be a sequence of real numbers, and let \(L\) be a real number.
    Then the following two statements are logically equivalent (each one implies the other):
    \begin{enumerate}
        \item The sequece \((a_n)_{n = 0}^\infty\) converges to \(L\).
        \item Every subsequence of \((a_n)_{n = 0}^\infty\) converges to \(L\).
    \end{enumerate}
\end{proposition}

\begin{proof}
    We first show that \((a_n)_{n = 0}^\infty\) converges to \(L\) implies every subsequence of \((a_n)_{n = 0}^\infty\) converges to \(L\).
    Let \((b_n)_{n = 0}^\infty\) be a subsequence of \((a_n)_{n = 0}^\infty\), and let \(f : \mathbf{N} \to \mathbf{N}\) be a function where \(b_n = a_{f(n)}\).
    Since \((a_n)_{n = 0}^\infty\) converges to \(L\), \(\forall\ \varepsilon \in \mathbf{R}\) and \(\varepsilon > 0\), \(\exists\ N \in \mathbf{N}\) and \(N \geq 0\) such that \(\abs*{a_n - L} \leq \varepsilon \ \forall\ n \geq N\).
    This means \(\forall\ f(n) \geq N\), we also have \(\abs*{a_{f(n)} - L} = \abs*{b_n - L} \leq \varepsilon\).
    Thus \((b_n)_{n = 0}^\infty\) also converges to \(L\).
    Since \((b_n)_{n = 0}^\infty\) is arbitrary, we thus conclude that \((a_n)_{n = 0}^\infty\) converges to \(L\) implies every subsequence of \((a_n)_{n = 0}^\infty\) converges to \(L\).

    Now we show that every subsequence of \((a_n)_{n = 0}^\infty\) converges to \(L\) implies \((a_n)_{n = 0}^\infty\) converges to \(L\).
    By Lemma \ref{6.6.4}, \((a_n)_{n = 0}^\infty\) is a subsequence of \((a_n)_{n = 0}^\infty\), so \((a_n)_{n = 0}^\infty\) converges to \(L\).
    Thus we conclude that every subsequence of \((a_n)_{n = 0}^\infty\) converges to \(L\) implies \((a_n)_{n = 0}^\infty\) converges to \(L\).
\end{proof}

\begin{proposition}[Subsequences related to limit points]\label{6.6.6}
    Let \((a_n)_{n = 0}^\infty\) be a sequence of real numbers, and let \(L\) be a real number.
    Then the following two statements are logically equivalent.
    \begin{enumerate}
        \item \(L\) is a limit point of \((a_n)_{n = 0}^\infty\).
        \item There exists a subsequence of \((a_n)_{n = 0}^\infty\) which converges to \(L\).
    \end{enumerate}
\end{proposition}

\begin{proof}
    We first show that \(L\) is a limit point of \((a_n)_{n = 0}^\infty\) implies there exists a subsequence of \((a_n)_{n = 0}^\infty\) which converges to \(L\).
    Let \(f : \mathbf{N} \to \mathbf{N}\) be a function where \(f(0) = 0\) and \(f(n_j) = \min\{n > n_{j - 1} : \abs*{a_n - L} \leq 1 / j\}\).
    Since \(L\) is a limit point of \((a_n)_{n = 0}^\infty\), \(\forall\ \varepsilon \in \mathbf{R}\) and \(\varepsilon \geq 0\), \(\forall\ n_j > 0\), \(\exists\ n \geq n_j\) such that \(\abs*{a_n - L} \leq \varepsilon\).
    In particular, \(\abs*{a_n - L} \leq 1 / j\).
    Thus such \(f\) exists and \((a_{f(n)})_{n = 1}^\infty\) is a subsequence of \((a_n)_{n = 0}^\infty\).
    Since \(\forall\ n \geq 1\) we have \(\abs*{a_{f(n)} - L} \leq \varepsilon\), the sequece \((a_{f(n)})_{n = 1}^\infty\) converges to \(L\), and we are done.

    Now we show that a subsequence of \((a_n)_{n = 0}^\infty\) converges to \(L\) implies \(L\) is a limit point of \((a_n)_{n = 0}^\infty\).
    Let \((b_n)_{n = 0}^\infty\) be a subsequence of \((a_n)_{n = 0}^\infty\) and \(\lim_{n \to \infty} b_n = L\).
    Let \(f : \mathbf{N} \to \mathbf{N}\) be a function where \(b_n = a_{f(n)}\).
    Since \(\lim_{n \to \infty} b_n = L\), \(\forall\ \varepsilon \in \mathbf{R}\) and \(\varepsilon > 0\), \(\exists\ n \geq 0\) such that \(\abs*{b_n - L} \leq \varepsilon\).
    This means \(\forall\ N \in \mathbf{N}\) and \(N \geq 0\), \(\exists\ n \geq N\) such that \(\abs*{b_n - L} = \abs*{a_{f(n)} - L} \leq \varepsilon\).
    Thus \(L\) is a limit point of \((a_n)_{n = 0}^\infty\), and we are done.
\end{proof}

\begin{remark}\label{6.6.7}
    Proposition \ref{6.6.5} and \ref{6.6.6} give a sharp contrast between the notion of a limit, and that of a limit point.
    When a sequence has a limit \(L\), then all subsequences also converge to \(L\).
    But when a sequence has \(L\) as a limit point, then only some subsequences converge to \(L\).
\end{remark}

\begin{note}
    We can now prove an important theorem in real analysis, due to Bernard Bolzano (1781--1848) and Karl Weierstrass (1815--1897):
    every bounded sequence has a convergent subsequence.
\end{note}

\begin{theorem}[Bolzano-Weierstrass theorem]\label{6.6.8}
    Let \((a_n)_{n = 0}^\infty\) be a bounded sequence
    (i.e., there exists a real number \(M > 0\) such that \(\abs*{a_n} \leq M\) for all \(n \in \mathbf{N}\)).
    Then there is at least one subsequence of \((a_n)_{n = 0}^\infty\) which converges.
\end{theorem}

\begin{proof}
    Let \(L\) be the limit superior of the sequence \((a_n)_{n = 0}^\infty\).
    Since we have \(-M \leq a_n \leq M\) for all natural numbers \(n\), it follows from the comparison principle (Lemma \ref{6.4.13}) that \(-M \leq L \leq M\).
    In particular, \(L\) is a real number (not \(+\infty\) or \(-\infty\)).
    By Proposition \ref{6.4.12}(e), \(L\) is thus a limit point of \((a_n)_{n = 0}^\infty\).
    Thus by Proposition \ref{6.6.6}, there exists a subsequence of \((a_n)_{n = 0}^\infty\) which converges
    (in fact, it converges to \(L\)).
\end{proof}

\begin{note}
    we could as well have used the limit inferior instead of the limit superior in the argument of Theorem \ref{6.6.8}.
\end{note}

\begin{remark}\label{6.6.9}
    The Bolzano-Weierstrass theorem says that if a sequence is bounded, then eventually it has no choice but to converge in some places;
    it has ``no room'' to spread out and stop itself from acquiring limit points.
    It is not true for unbounded sequences;
    In the language of topology, this means that the interval \(\{x \in \mathbf{R} : -M \leq x \leq M\}\) is \emph{compact}, whereas an unbounded set such as the real line \(\mathbf{R}\) is not compact.
\end{remark}

\exercisesection

\begin{exercise}\label{ex 6.6.1}
    Prove Lemma \ref{6.6.4}.
\end{exercise}

\begin{proof}
    See Lemma \ref{6.6.4}.
\end{proof}

\begin{exercise}\label{ex 6.6.2}
    Can you find two sequences \((a_n)_{n = 0}^\infty\) and \((b_n)_{n = 0}^\infty\) which are not the same sequence, but such that each is a subsequence of the other?
\end{exercise}

\begin{proof}
    Let \((a_n)_{n = 0}^\infty = \{0, 1, 0, 1, \dots\}\) and \((b_n)_{n = 0}^\infty = \{1, 0, 1, 0, \dots\}\).
\end{proof}

\begin{exercise}\label{ex 6.6.3}
    Let \((a_n)_{n = 0}^\infty\) be a sequence which is not bounded.
    Show that there exists a subsequence \((b_n)_{n = 0}^\infty\) of \((a_n)_{n = 0}^\infty\) such that \(\lim_{n \to \infty} 1 / b_n\) exists and is equal to zero.
\end{exercise}

\begin{proof}
    Let \(j \in \mathbf{N}\), and let \(f(n) : \mathbf{N} \to \mathbf{N}\) be a function where \(f(0) = \min\{n \in \mathbf{N} : \abs*{a_n} \geq 0\}\) and \(f(n_j) = \min\{n \in \mathbf{N} : (\abs*{a_n} \geq j) \land (n > n_{j - 1})\}\).
    Such \(f\) exists because \((a_n)_{n = 0}^\infty\) is not bounded.
    Let \((b_n)_{n = 1}^\infty\) be a subsequence of \((a_n)_{n = 0}^\infty\) where \(b_n = a_{f(n)}\).
    Then \(\abs*{1 / b_n} \leq 1 / n\).
    Since \(\lim_{n \to \infty} 0 = 0\) and \(\lim_{n \to \infty} 1 / n = 0\), by Squeeze test (Corollary \ref{6.4.14}), \(0 \leq \abs*{1 / b_n} \leq 1 / j\) implies \(\lim_{n \to \infty} \abs*{1 / b_n} = 0\).
    Since \(\lim_{n \to \infty} \abs*{1 / b_n} = 0\), by zero test (Corollary \ref{6.4.17}) we have \(\lim_{n \to \infty} 1 / b_n = 0\).
    Thus there exists a subsequence \((b_n)_{n = 0}^\infty\) of \((a_n)_{n = 0}^\infty\) such that \(\lim_{n \to \infty} 1 / b_n\) exists and is equal to zero.
\end{proof}

\begin{exercise}\label{ex 6.6.4}
    Prove Proposition \ref{6.6.5}.
\end{exercise}

\begin{proof}
    See Proposition \ref{6.6.5}.
\end{proof}

\begin{exercise}\label{ex 6.6.5}
    Prove Proposition \ref{6.6.6}.
\end{exercise}

\begin{proof}
    See Proposition \ref{6.6.6}.
\end{proof}