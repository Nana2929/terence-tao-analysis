\section{Real exponentiation, part II}\label{sec 6.7}

\begin{lemma}[Continuity of exponentiation]\label{6.7.1}
    Let \(x > 0\), and let \(\alpha\) be a real number.
    Let \((q_n)_{n = 1}^\infty\) be any sequence of rational numbers converging to \(\alpha\).
    Then \((x^{q_n})_{n = 1}^\infty\) is also a convergent sequence.
    Furthermore, if \((q_n')_{n = 1}^\infty\) is any other sequence of rational numbers converging to \(\alpha\), then \((x^{q_n'})_{n = 1}^\infty\) has the same limit as \((x^{q_n})_{n = 1}^\infty\):
    \[
        \lim_{n \to \infty} x^{q_n} = \lim_{n \to \infty} x^{q_n'}.
    \]
\end{lemma}

\begin{proof}
    There are three cases: \(x < 1\), \(x = 1\), and \(x > 1\).
    The case \(x = 1\) is rather easy (because then \(x^q = 1\) for all rational \(q\)).

    We first show that if \(x > 1\) then \((x^{q_n})_{n = 1}^\infty\) converges.
    By Proposition \ref{6.4.18} it is enough to show that \((x^{q_n})_{n = 1}^\infty\) is a Cauchy sequence.

    To do this, we need to estimate the distance between \(x^{q_n}\) and \(x^{q_m}\);
    let us say for the time being that \(q_n \geq q_m\), so that \(x^{q_n} \geq x^{q_m}\) (since \(x > 1\)).
    We have
    \[
        d(x^{q_n}, x^{q_m}) = x^{q_n} - x^{q_m} = x^{q_m} (x^{q_n - q_m} - 1).
    \]
    Since \((q_n)_{n = 1}^\infty\) is a convergent sequence, it has some upper bound \(M\);
    since \(x > 1\), we have \(x^{q_m} \leq x^M\).
    Thus
    \[
        d(x^{q_n}, x^{q_m}) = \abs*{x^{q_n} - x^{q_m}} \leq x^M (x^{q_n - q_m} - 1).
    \]
    Now let \(\varepsilon > 0\).
    We know by Lemma \ref{6.5.3} that the sequence \((x^{1 / k})_{k = 1}^\infty\) is eventually \(\varepsilon x^{-M}\)-close to \(1\).
    Thus there exists some \(K \geq 1\) such that
    \[
        \abs*{x^{1 / K} - 1} \leq \varepsilon x^{-M}.
    \]
    Now since \((q_n)_{n = 1}^\infty\) is convergent, it is a Cauchy sequence, and so there is an \(N \geq 1\) such that \(q_n\) and \(q_m\) are \(1 / K\)-close for all \(n, m \geq N\).
    Thus we have
    \[
        d(x^{q_n}, x^{q_m}) \leq x^M (x^{q_n - q_m} - 1) \leq x^M (x^{1 / K} - 1) \leq x^M \varepsilon x^{-M} = \varepsilon.
    \]
    for every \(n, m \geq N\) such that \(q_n \geq q_m\).
    By symmetry we also have this bound when \(n, m \geq N\) and \(q_n \leq q_m\).
    Thus the sequence \((x^{q_n})_{n = 1}^\infty\) is \(\varepsilon\)-steady.
    Thus the sequence \((x^{q_n})_{n = 1}^\infty\) is eventually \(\varepsilon\)-steady for every \(\varepsilon > 0\), and is thus a Cauchy sequence as desired.
    This proves the convergence of \((x^{q_n})_{n = 1}^\infty\) when \(x > 1\).

    Next we show that if \(x < 1\) then \((x^{q_n})_{n = 1}^\infty\) also converges.
    By Proposition \ref{6.4.18} it is enough to show that \((x^{q_n})_{n = 1}^\infty\) is a Cauchy sequence.

    To do this, we need to estimate the distance between \(x^{q_n}\) and \(x^{q_m}\);
    let us say for the time being that \(q_n \leq q_m\), so that \(x^{q_n} \geq x^{q_m}\) (since \(x < 1\)).
    We have
    \[
        d(x^{q_n}, x^{q_m}) = x^{q_n} - x^{q_m} = x^{q_m} (x^{q_n - q_m} - 1).
    \]
    Since \((q_n)_{n = 1}^\infty\) is a convergent sequence, it has some lower bound \(M\);
    since \(x < 1\), we have \(x^{q_m} \leq x^M\).
    Thus
    \[
        d(x^{q_n}, x^{q_m}) = \abs*{x^{q_n} - x^{q_m}} \leq x^M (x^{q_n - q_m} - 1).
    \]
    Now let \(\varepsilon > 0\).
    We know by Lemma \ref{6.5.3} that the sequence \((x^{1 / k})_{k = 1}^\infty\) is eventually \(\varepsilon x^{-M}\)-close to \(1\).
    Thus there exists some \(K \geq 1\) such that
    \[
        \abs*{x^{1 / K} - 1} \leq \varepsilon x^{-M}.
    \]
    Now since \((q_n)_{n = 1}^\infty\) is convergent, it is a Cauchy sequence, and so there is an \(N \geq 1\) such that \(q_n\) and \(q_m\) are \(1 / K\)-close for all \(n, m \geq N\).
    Thus we have
    \[
        d(x^{q_n}, x^{q_m}) \leq x^M (x^{q_n - q_m} - 1) \leq x^M (x^{1 / K} - 1) \leq x^M \varepsilon x^{-M} = \varepsilon.
    \]
    for every \(n, m \geq N\) such that \(q_n \leq q_m\).
    By symmetry we also have this bound when \(n, m \geq N\) and \(q_n \geq q_m\).
    Thus the sequence \((x^{q_n})_{n = 1}^\infty\) is \(\varepsilon\)-steady.
    Thus the sequence \((x^{q_n})_{n = 1}^\infty\) is eventually \(\varepsilon\)-steady for every \(\varepsilon > 0\), and is thus a Cauchy sequence as desired.
    This proves the convergence of \((x^{q_n})_{n = 1}^\infty\) when \(x < 1\).

    Now we prove the second claim.
    It will suffice to show that
    \[
        \lim_{n \to \infty} x^{q_n - q_n'} = 1,
    \]
    since the claim would then follow from limit laws
    (since \(x^{q_n} = x^{q_n - q_n'} x^{q_n'}\)).

    Write \(r_n \coloneqq q_n - q_n'\);
    by limit laws we know that \((r_n)_{n = 1}^\infty\) converges to \(0\).
    We have to show that for every \(\varepsilon > 0\), the sequence \((x^{r_n})_{n = 1}^\infty\) is eventually \(\varepsilon\)-close to \(1\).
    But from Lemma \ref{6.5.3} we know that the sequence \((x^{1 / k})_{k = 1}^\infty\) is eventually \(\varepsilon\)-close to \(1\).
    Since \(\lim_{k \to \infty} x^{-1 / k}\) is also equal to \(1\) by Lemma \ref{6.5.3}, we know that \((x^{-1 / k})_{k = 1}^\infty\) is also eventually \(\varepsilon\)-close to \(1\).
    Thus we can find a \(K\) such that \(x^{1 / K}\) and \(x^{-1 / K}\) are both \(\varepsilon\)-close to \(1\).
    But since \((r_n)_{n = 1}^\infty\) is convergent to \(0\), it is eventually \(1 / K\)-close to \(0\), so that eventually \(-1 / K \leq r_n \leq 1 / K\), and thus when \(x > 1\) we have \(x^{-1 / K} \leq x^{r_n} \leq x^{1 / K}\), when \(x < 1\) we have \(x^{1 / K} \leq x^{r_n} \leq x^{-1 / K}\).
    In particular \(x^{r_n}\) is also eventually \(\varepsilon\)-close to \(1\) (see Proposition \ref{4.3.7}(f)), as desired.
\end{proof}

\begin{definition}[Exponentiation to a real exponent]\label{6.7.2}
    Let \(x > 0\) be real, and let \(\alpha\) be a real number.
    We define the quantity \(x^\alpha\) by the formula \(x^\alpha = \lim_{n \to \infty} x^{q_n}\), where \((q_n)_{n = 1}^\infty\) is any sequence of rational numbers converging to \(\alpha\).
\end{definition}

\begin{note}
    Let us check that Definition \ref{6.7.2} is well-defined.
    First of all, given any real number \(\alpha\) we always have at least one sequence \((q_n)_{n = 1}^\infty\) of rational numbers converging to \(\alpha\), by the definition of real numbers (and Proposition \ref{6.1.15}).
    Secondly, given any such sequence \((q_n)_{n = 1}^\infty\), the limit \(\lim_{n \to \infty} x^{q_n}\) exists by Lemma \ref{6.7.1}.
    Finally, even though there can be multiple choices for the sequence \((q_n)_{n = 1}^\infty\), they all give the same limit by Lemma \ref{6.7.1}.
    Thus this definition is well-defined.
\end{note}

\begin{note}
    If \(\alpha\) is not just real but rational, i.e., \(\alpha = q\) for some rational \(q\), then Definition \ref{6.7.2} could in principle be inconsistent with our earlier definition of exponentiation in Section \ref{sec 6.7}.
    But in this case \(\alpha\) is clearly the limit of the sequence \((q)_{n = 1}^\infty\), so by definition \(x^\alpha = \lim_{n \to \infty} x^q = x^q\).
    Thus the new definition of exponentiation is consistent with the old one.
\end{note}

\begin{proposition}\label{6.7.3}
    All the results of Lemma \ref{5.6.9}, which held for rational numbers \(q\) and \(r\), continue to hold for real numbers \(q\) and \(r\).
\end{proposition}

\begin{proof}{(a)}
    Let \(r\) be a real number.
    Then we can write \(r = \lim_{n \to \infty} r_n\) for some sequences \((r_n)_{n = 1}^\infty\) of rationals, by the definition of real numbers (and Proposition \ref{6.1.15}).
    Since \((r_n)_{n = 1}^\infty\) is a Cauchy sequence, it is bounded by some \(M \in \mathbf{Q}\), i.e, \(-M \leq r_n \leq M \ \forall\ n \geq 1\).
    By Lemma \ref{5.6.9}, both \(x^M\) and \(x^{-M}\) are positive real numbers.
    If \(0 < x < 1\), then \(x^M \leq x^{r_n} \leq x^{-M}\).
    If \(x > 1\), then \(x^{-M} \leq x^{r_n} \leq x^M\).
    By Theorem \ref{6.1.19}, we have
    \begin{align*}
         & \lim_{n \to \infty} \min(x^{-M}, x^M, x^{r_n})                                           \\
         & = \min(\lim_{n \to \infty} x^{-M}, \lim_{n \to \infty} x^M, \lim_{n \to \infty} x^{r_n}) \\
         & = \min(x^{-M}, x^M, x^r)                                                                 \\
         & = x^{-M}.
    \end{align*}
    Since \(x^{-M}\) is positive real number, \(x^r\) must also be a positive real number, as desired.
\end{proof}

\begin{proof}{(b)}
    Let \(q\) and \(r\) be real numbers.
    Then we can write \(q = \lim_{n \to \infty} q_n\) and \(r = \lim_{n \to \infty} r_n\) for some sequences \((q_n)_{n = 1}^\infty\) and \((r_n)_{n = 1}^\infty\) of rationals, by the definition of real numbers (and Proposition \ref{6.1.15}).
    Then by the limit laws, \(q + r\) is the limit of \((q_n + r_n)_{n = 1}^\infty\).
    By definition of real exponentiation, we have
    \[
        x^{q + r} = \lim_{n \to \infty} x^{q_n + r_n} ; x^q = \lim_{n \to \infty} x^{q_n} ;  x^r = \lim_{n \to \infty} x^{r_n}.
    \]
    But by Lemma \ref{5.6.9}(b) (applied to \emph{rational} exponents) we have \(x^{q_n + r_n} = x^{q_n} x^{r_n}\).
    Thus by limit laws we have \(x^{q + r} = x^q x^r\), as desired.
\end{proof}

\begin{proof}{(c)}
    Let \(r\) be a real number.
    Then we can write \(r = \lim_{n \to \infty} r_n\) for some sequences \((r_n)_{n = 1}^\infty\) of rationals, by the definition of real numbers (and Proposition \ref{6.1.15}).
    Then by the limit laws, \(-r\) is the limit of \((-r_n)_{n = 1}^\infty\).
    By definition of real exponentiation, we have
    \[
        x^{-r} = \lim_{n \to \infty} x^{-r_n}
    \]
    But by Lemma \ref{5.6.9}(c) (applied to \emph{rational} exponents) we have \(x^{-r_n} = 1 / x^{r_n}\).
    Thus by limit laws we have \(x^{-r} = 1 / x^r\), as desired.
\end{proof}

\begin{proof}{(d)}
    Let \(x, y, r\) be positive real numbers.
    Then we can write \(r = \lim_{n \to \infty} r_n\) for some sequences \((r_n)_{n = 1}^\infty\) of rationals, by the definition of real numbers (and Proposition \ref{6.1.15}).
    By definition of real exponentiation, we have
    \[
        \max(x^r, y^r) = \lim_{n \to \infty} \max(x^{r_n}, y^{r_n}) ; x^r = \lim_{n \to \infty} x^{r_n} ; y^r = \lim_{n \to \infty} y^{r_n}.
    \]
    But by Lemma \ref{5.6.9}(d) (applied to \emph{rational} exponents) we have \(x > y \iff x^{r_n} > y^{r_n}\).
    Thus by limit laws we have \(x > y \iff x^r > y^r\), as desired.
\end{proof}

\begin{proof}{(e)}
    Let \(x, q, r\) be positive real numbers.
    Then we can write \(q = \lim_{n \to \infty} q_n\) and \(r = \lim_{n \to \infty} r_n\) for some sequences \((q_n)_{n = 1}^\infty\) and \((r_n)_{n = 1}^\infty\) of rationals, by the definition of real numbers (and Proposition \ref{6.1.15}).
    By definition of real exponentiation, we have
    \[
        \max(x^q, x^r) = \lim_{n \to \infty} \max(x^{q_n}, x^{r_n}); x^q = \lim_{n \to \infty} x^{q_n} ; x^r = \lim_{n \to \infty} x^{r_n}.
    \]
    But by Lemma \ref{5.6.9}(e) (applied to \emph{rational} exponents) we have if \(x > 1\) then \(x^{q_n} > x^{r_n} \iff q > r\) and if \(x < 1\) then \(x^{q_n} > x^{r_n} \iff q < r\).
    Thus by limit laws we have if \(x > 1\) then \(x^q > x^r \iff q > r\) and if \(x < 1\) then \(x^q > x^r \iff q < r\), as desired.
\end{proof}

\exercisesection

\begin{exercise}\label{ex 6.7.1}
    Prove the remaining components of Proposition \ref{6.7.3}.
\end{exercise}

\begin{proof}
    See Proposition \ref{6.7.3}.
\end{proof}