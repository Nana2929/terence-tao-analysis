\section{Finite series}\label{sec 7.1}

\begin{definition}[Finite series]\label{7.1.1}
    Let \(m, n\) be integers, and let \((a_i)_{i = m}^n\) be a finite sequence of real numbers, assigning a real number \(a_i\) to each integer \(i\) between \(m\) and \(n\) inclusive (i.e., \(m \leq i \leq n\)).
    Then we define the finite sum (or finite series) \(\sum_{i = m}^n a_i\) by the recursive formula
    \begin{align*}
         & \sum_{i = m}^n a_i \coloneqq 0 \text{ whenever } n < m ;                                                      \\
         & \sum_{i = m}^{n + 1} a_i \coloneqq \Bigg(\sum_{i = m}^n a_i\Bigg) + a_{n + 1} \text{ whenever } n \geq m - 1.
    \end{align*}
\end{definition}

\begin{note}
    we sometimes express \(\sum_{i = m}^n a_i\) less formally as
    \[
        \sum_{i = m}^n a_i = a_m + a_{m + 1} + \dots + a_n.
    \]
\end{note}

\begin{remark}\label{7.1.2}
    The difference between ``sum'' and ``series'' is a subtle linguistic one.
    Strictly speaking, a series is an \emph{expression} of the form \(\sum_{i = m}^n a_i\);
    this series is mathematically (but not semantically) equal to a real number, which is then the \emph{sum} of that series.
    For instance, \(1 + 2 + 3 + 4 + 5\) is a series, whose sum is \(15\);
    if one were to be very picky about semantics, one would not consider \(15\) a series and one would not consider \(1 + 2 + 3 + 4 + 5\) a sum, despite the two expressions having the same value.
    However, we will not be very careful about this distinction as it is purely linguistic and has no bearing on the mathematics;
    the expressions \(1 + 2 + 3 + 4 + 5\) and \(15\) are the same number, and thus \emph{mathematically} interchangeable, in the sense of the axiom of substitution, even if they are not semantically interchangeable.
\end{remark}

\begin{remark}\label{7.1.3}
    Note that the variable \(i\) (sometimes called the \emph{index of summation}) is a \emph{bound variable} (sometimes called a \emph{dummy variable});
    the expression \(\sum_{i = m}^n a_i\) does not actually depend on any quantity named \(i\).
    In particular, one can replace the index of summation \(i\) with any other symbol, and obtain the same sum:
    \[
        \sum_{i = m}^n a_i = \sum_{j = m}^n a_j.
    \]
\end{remark}

\begin{lemma}\label{7.1.4}
    \mbox{}
    \begin{enumerate}
        \item Let \(m \leq n < p\) be integers, and let \(a_i\) be a real number assigned to each integer \(m \leq i \leq p\).
              Then we have
              \[
                  \sum_{i = m}^n a_i + \sum_{i = n + 1}^p a_i = \sum_{i = m}^p a_i.
              \]
        \item Let \(m \leq n\) be integers, \(k\) be another integer, and let \(a_i\) be a real number assigned to each integer \(m \leq i \leq n\).
              Then we have
              \[
                  \sum_{i = m}^n a_i = \sum_{j = m + k}^{n + k} a_{j - k}.
              \]
        \item Let \(m \leq n\) be integers, and let \(a_i, b_i\) be real numbers assigned to each integer \(m \leq i \leq n\).
              Then we have
              \[
                  \sum_{i = m}^n (a_i + b_i) = \Bigg(\sum_{i = m}^n a_i\Bigg) + \Bigg(\sum_{i = m}^n b_i\Bigg).
              \]
        \item Let \(m \leq n\) be integers, and let \(a_i\) be a real number assigned to each integer \(m \leq i \leq n\), and let \(c\) be another real number.
              Then we have
              \[
                  \sum_{i = m}^n (ca_i) = c\Bigg(\sum_{i = m}^n a_i\Bigg).
              \]
        \item (Triangle inequality for finite series)
              Let \(m \leq n\) be integers, and let \(a_i\) be a real number assigned to each integer \(m \leq i \leq n\).
              Then we have
              \[
                  \abs*{\sum_{i = m}^n a_i} \leq \sum_{i = m}^n \abs*{a_i}.
              \]
        \item (Comparison test for finite series) Let \(m \leq n\) be integers, and let \(a_i\), \(b_i\) be real numbers assigned to each integer \(m \leq i \leq n\).
              Suppose that \(a_i \leq b_i\) for all \(m \leq i \leq n\).
              Then we have
              \[
                  \sum_{i = m}^n a_i \leq \sum_{i = m}^n b_i
              \]
    \end{enumerate}
\end{lemma}

\begin{proof}{(a)}
    Let \(k = p - m\).
    By hypothesis we know that \(k > 0\).
    Now we use induction on \(k\) to show that Lemma \ref{7.1.4}(a) is true and we start with \(k = 1\).
    For \(k = 1\), we have \(p = m + 1\) and by Definition \ref{7.1.1} we have
    \[
        \sum_{i = m}^n a_i + \sum_{i = n + 1}^p a_i = \sum_{i = m}^m a_i + \sum_{i = m + 1}^p a_i = a_m + a_{m + 1} = \sum_{i = m}^p a_i.
    \]
    Thus the base case holds.
    Suppose inductively that for some \(k \geq 1\) Lemma \ref{7.1.4}(a) is true.
    Then for \(k + 1\), we have \(p - 1 = k + m\) and
    \begin{align*}
        \sum_{i = m}^n a_i + \sum_{i = n + 1}^p a_i & = \Bigg(\sum_{i = m}^n a_i\Bigg) + \Bigg(\sum_{i = n + 1}^{p - 1} a_i\Bigg) + a_p & \text{(by Definition \ref{7.1.1})} \\
                                                    & = \Bigg(\sum_{i = m}^{p - 1} a_i\Bigg) + a_p                                      & \text{(by induction hypothesis)}   \\
                                                    & = \sum_{i = m}^p a_i.                                                             & \text{(by Definition \ref{7.1.1})}
    \end{align*}
    This close the induction.
\end{proof}

\begin{proof}{(b)}
    Let \(p = n - m\).
    By hypothesis we know that \(p \geq 0\).
    Now we use induction on \(p\) to show that Lemma \ref{7.1.4}(b) is true.
    For \(p = 0\), we have \(n = m\) and
    \begin{align*}
        \sum_{j = m + k}^{m + k} a_{j - k} & = \Bigg(\sum_{j = m + k}^{m + k - 1} a_{j - k}\Bigg) + a_{m + k - k} & \text{(by Definition \ref{7.1.1})} \\
                                           & = 0 + a_{m + k - k}                                                  & \text{(by Definition \ref{7.1.1})} \\
                                           & = 0 + a_m                                                                                                 \\
                                           & = \Bigg(\sum_{i = m}^{m - 1} a_i\Bigg) + a_m                         & \text{(by Definition \ref{7.1.1})} \\
                                           & = \sum_{i = m}^m a_i.                                                & \text{(by Definition \ref{7.1.1})}
    \end{align*}
    So the base case holds.
    Suppose inductively that for some \(p \geq 0\) Lemma \ref{7.1.4}(b) is true.
    Then for \(p + 1\), we have \(p = n - m - 1\) and
    \begin{align*}
        \sum_{j = m + k}^{n + k} a_{j - k} & = \Bigg(\sum_{j = m + k}^{n + k - 1} a_{j - k}\Bigg) + a_{n + k - k} & \text{(by Definition \ref{7.1.1})} \\
                                           & = \Bigg(\sum_{j = m + k}^{n + k - 1} a_{j - k}\Bigg) + a_n                                                \\
                                           & = \Bigg(\sum_{i = m}^{n - 1} a_i\Bigg) + a_n                         & \text{(by induction hypothesis)}   \\
                                           & = \sum_{i = m}^n a_i.                                                & \text{(by Definition \ref{7.1.1})}
    \end{align*}
    This close the induction.
\end{proof}

\begin{proof}{(c)}
    Let \(p = n - m\).
    By hypothesis we know that \(p \geq 0\).
    Now we use induction on \(p\) to show that Lemma \ref{7.1.4}(c) is true.
    For \(p = 0\), we have \(n = m\) and
    \begin{align*}
        \sum_{i = m}^m (a_i + b_i) & = \Bigg(\sum_{i = m}^{m - 1} (a_i + b_i)\Bigg) + a_m + b_m                                & \text{(by Definition \ref{7.1.1})} \\
                                   & = 0 + a_m + b_m                                                                           & \text{(by Definition \ref{7.1.1})} \\
                                   & = \Bigg(\sum_{i = m}^{m - 1} a_i\Bigg) + \Bigg(\sum_{i = m}^{m - 1} b_i\Bigg) + a_m + b_m & \text{(by Definition \ref{7.1.1})} \\
                                   & = \Bigg(\sum_{i = m}^m a_i\Bigg) + \Bigg(\sum_{i = m}^m b_i\Bigg).                        & \text{(by Definition \ref{7.1.1})}
    \end{align*}
    So the base case holds.
    Suppose inductively that for some \(p \geq 0\) Lemma \ref{7.1.4}(c) is true.
    Then for \(p + 1\), we have \(p = n - m - 1\) and
    \begin{align*}
        \sum_{i = m}^n (a_i + b_i) & = \Bigg(\sum_{i = m}^{n - 1} (a_i + b_i)\Bigg) + a_n + b_n                                & \text{(by Definition \ref{7.1.1})} \\
                                   & = \Bigg(\sum_{i = m}^{n - 1} a_i\Bigg) + \Bigg(\sum_{i = m}^{n - 1} b_i\Bigg) + a_n + b_n & \text{(by induction hypothesis)}   \\
                                   & = \Bigg(\sum_{i = m}^n a_i\Bigg) + \Bigg(\sum_{i = m}^n b_i\Bigg).                        & \text{(by Definition \ref{7.1.1})}
    \end{align*}
    This close the induction.
\end{proof}

\begin{proof}{(d)}
    Let \(p = n - m\).
    By hypothesis we know that \(p \geq 0\).
    Now we use induction on \(p\) to show that Lemma \ref{7.1.4}(d) is true.
    For \(p = 0\), we have \(n = m\) and
    \begin{align*}
        \sum_{i = m}^m ca_i & = \Bigg(\sum_{i = m}^{m - 1} ca_i\Bigg) + ca_m            & \text{(by Definition \ref{7.1.1})} \\
                            & = 0 + ca_m                                                & \text{(by Definition \ref{7.1.1})} \\
                            & = c \times 0 + ca_m                                                                            \\
                            & = c\Bigg(\sum_{i = m}^{m - 1} a_i\Bigg) + ca_m            & \text{(by Definition \ref{7.1.1})} \\
                            & = c\Bigg(\Bigg(\sum_{i = m}^{m - 1} a_i\Bigg) + a_m\Bigg)                                      \\
                            & = c\Bigg(\sum_{i = m}^m a_i\Bigg).                        & \text{(by Definition \ref{7.1.1})}
    \end{align*}
    So the base case holds.
    Suppose inductively that for some \(p \geq 0\) Lemma \ref{7.1.4}(d) is true.
    Then for \(p + 1\), we have \(p = n - m - 1\) and
    \begin{align*}
        \sum_{i = m}^n ca_i & = \Bigg(\sum_{i = m}^{n - 1} ca_i\Bigg) + ca_n            & \text{(by Definition \ref{7.1.1})} \\
                            & = c\Bigg(\sum_{i = m}^{n - 1} a_i\Bigg) + ca_n            & \text{(by induction hypothesis)}   \\
                            & = c\Bigg(\Bigg(\sum_{i = m}^{n - 1} a_i\Bigg) + a_n\Bigg)                                      \\
                            & = c\Bigg(\sum_{i = m}^n a_i\Bigg).                        & \text{(by Definition \ref{7.1.1})}
    \end{align*}
    This close the induction.
\end{proof}

\begin{proof}{(e)}
    Let \(p = n - m\).
    By hypothesis we know that \(p \geq 0\).
    Now we use induction on \(p\) to show that Lemma \ref{7.1.4}(e) is true.
    For \(p = 0\), we have \(n = m\) and
    \begin{align*}
        \abs*{\sum_{i = m}^m a_i} & = \abs*{\Bigg(\sum_{i = m}^{m - 1} a_i\Bigg) + a_m}        & \text{(by Definition \ref{7.1.1})} \\
                                  & = \abs*{0 + a_m}                                           & \text{(by Definition \ref{7.1.1})} \\
                                  & = 0 + \abs*{a_m}                                                                                \\
                                  & = \Bigg(\sum_{i = m}^{m - 1} \abs*{a_i}\Bigg) + \abs*{a_m} & \text{(by Definition \ref{7.1.1})} \\
                                  & = \sum_{i = m}^m \abs*{a_i}.                               & \text{(by Definition \ref{7.1.1})}
    \end{align*}
    So the base case holds.
    Suppose inductively that for some \(p \geq 0\) Lemma \ref{7.1.4}(e) is true.
    Then for \(p + 1\), we have \(p = n - m - 1\) and
    \begin{align*}
        \abs*{\sum_{i = m}^n a_i} & = \abs*{\Bigg(\sum_{i = m}^{n - 1} a_i\Bigg) + a_n} & \text{(by Definition \ref{7.1.1})} \\
                                  & \leq \abs*{\sum_{i = m}^{n - 1} a_i} + \abs*{a_n}                                        \\
                                  & \leq \sum_{i = m}^{n - 1} \abs*{a_i} + \abs*{a_n}   & \text{(by induction hypothesis)}   \\
                                  & = \sum_{i = m}^n \abs*{a_i}.                        & \text{(by Definition \ref{7.1.1})}
    \end{align*}
    This close the induction.
\end{proof}

\begin{proof}{(f)}
    Let \(p = n - m\).
    By hypothesis we know that \(p \geq 0\).
    Now we use induction on \(p\) to show that Lemma \ref{7.1.4}(f) is true.
    For \(p = 0\), we have \(n = m\) and
    \begin{align*}
        \sum_{i = m}^m a_i & = \Bigg(\sum_{i = m}^{m - 1} a_i\Bigg) + a_m & \text{(by Definition \ref{7.1.1})} \\
                           & = 0 + a_m                                    & \text{(by Definition \ref{7.1.1})} \\
                           & \leq 0 + b_m                                 & \text{(by hypothesis)}             \\
                           & = \Bigg(\sum_{i = m}^{m - 1} b_i\Bigg) + b_m & \text{(by Definition \ref{7.1.1})} \\
                           & = \sum_{i = m}^m b_i.                        & \text{(by Definition \ref{7.1.1})} \\
    \end{align*}
    So the base case holds.
    Suppose inductively that for some \(p \geq 0\) Lemma \ref{7.1.4}(f) is true.
    Then for \(p + 1\), we have \(p = n - m - 1\) and
    \begin{align*}
        \sum_{i = m}^n a_i & = \Bigg(\sum_{i = m}^{n - 1} a_i\Bigg) + a_n    & \text{(by Definition \ref{7.1.1})} \\
                           & \leq \Bigg(\sum_{i = m}^{n - 1} b_i\Bigg) + a_n & \text{(by induction hypothesis)}   \\
                           & \leq \Bigg(\sum_{i = m}^{n - 1} b_i\Bigg) + b_n & \text{(by hypothesis)}             \\
                           & = \sum_{i = m}^n b_i.                           & \text{(by Definition \ref{7.1.1})} \\
    \end{align*}
    This close the induction.
\end{proof}

\begin{remark}\label{7.1.5}
    In the future we may omit some of the parentheses in series expressions, for instance we may write \(\sum_{i = m}^n (a_i + b_i)\) simply as \(\sum_{i = m}^n a_i + b_i\).
    This is reasonably safe from being mis-interpreted, because the alternative interpretation \((\sum_{i = m}^n a_i) + b_i\) does not make any sense
    (the index \(i\) in \(b_i\) is meaningless outside of the summation, since \(i\) is only a dummy variable).
\end{remark}

\begin{definition}[Summations over finite sets]\label{7.1.6}
    Let \(X\) be a finite set with \(n\) elements (where \(n \in \mathbf{N}\)), and let \(f : X \to \mathbf{R}\) be a function from \(X\) to the real numbers
    (i.e., \(f\) assigns a real number \(f(x)\) to each element \(x\) of \(X\)).
    Then we can define the finite sum \(\sum_{x \in X} f(x)\) as follows.
    We first select any bijection \(g\) from \(\{i \in \mathbf{N} : 1 \leq i \leq n\}\) to \(X\);
    such a bijection exists since \(X\) is assumed to have \(n\) elements.
    We then define
    \[
        \sum_{x \in X} f(x) \coloneqq \sum_{i = 1}^n f(g(i)).
    \]
    In some cases we would like to define the sum \(\sum_{x \in X} f(x)\) when \(f : Y \to \mathbf{R}\) is defined on a larger set \(Y\) than \(X\).
    In such cases we use exactly the same definition as is given above.
\end{definition}

\setcounter{theorem}{7}
\begin{proposition}[Finite summations are well-defined]\label{7.1.8}
    Let \(X\) be a finite set with \(n\) elements (where \(n \in \mathbf{N}\)), let \(f : X \to \mathbf{R}\) be a function, and let \(g : \{i \in \mathbf{N} : 1 \leq i \leq n\} \to X\) and \(h : \{i \in \mathbf{N} : 1 \leq i \leq n\} \to X\) be bijections.
    Then we have
    \[
        \sum_{i = 1}^n f(g(i)) = \sum_{i = 1}^n f(h(i)).
    \]
\end{proposition}

\begin{proof}
    We use induction on \(n\);
    more precisely, we let \(P(n)\) be the assertion that ``For any set \(X\) of \(n\) elements, any function \(f : X \to \mathbf{R}\), and any two bijections \(g, h\) from \(\{i \in \mathbf{N} : 1 \leq i \leq n\}\) to \(X\), we have \(\sum_{i = 1}^n f(g(i)) = \sum_{i = 1}^n f(h(i))\)''.
    (More informally, \(P(n)\) is the assertion that Proposition \ref{7.1.8} is true for that value of \(n\).)
    We want to prove that \(P(n)\) is true for all natural numbers \(n\).

    We first check the base case \(P(0)\).
    In this case \(\sum_{i = 1}^0 f(g(i))\) and \(\sum_{i = 1}^0 f(h(i))\) both equal to \(0\), by definition of finite series (Definition \ref{7.1.1}), so we are done.

    Now suppose inductively that \(P(n)\) is true;
    we now prove that \(P(n + 1)\) is true.
    Thus, let \(X\) be a set with \(n + 1\) elements, let \(f : X \to \mathbf{R}\) be a function, and let \(g\) and \(h\) be bijections from \(\{i \in N : 1 \leq i \leq n + 1\}\) to \(X\).
    We have to prove that
    \[
        \sum_{i = 1}^{n + 1} f(g(i)) = \sum_{i = 1}^{n + 1} f(h(i)). \tag{7.1}\label{eq 7.1}
    \]
    Let \(x \coloneqq g(n + 1)\);
    thus \(x\) is an element of \(X\).
    By definition of finite series (Definition \ref{7.1.1}), we can expand the left-hand side of \eqref{eq 7.1} as
    \[
        \sum_{i = 1}^{n + 1} f(g(i)) = \Bigg(\sum_{i = 1}^n f(g(i))\Bigg) + f(x).
    \]
    Now let us look at the right-hand side of \eqref{eq 7.1}.
    Ideally we would like to have \(h(n + 1)\) also equal to \(x\)
    - this would allow us to use the inductive hypothesis \(P(n)\) much more easily
    - but we cannot assume this.
    However, since \(h\) is a bijection, we do know that there is \emph{some} index \(j\), with \(1 \leq j \leq n + 1\), for which \(h(j) = x\).
    We now use Lemma \ref{7.1.4} and the definition of finite series (Definition \ref{7.1.1}) to write
    \begin{align*}
        \sum_{i = 1}^{n + 1} f(h(i)) & = \Bigg(\sum_{i = 1}^j f(h(i))\Bigg) + \Bigg(\sum_{i = j + 1}^{n + 1} f(h(i))\Bigg)                 \\
                                     & = \Bigg(\sum_{i = 1}^{j - 1} f(h(i))\Bigg) + f(h(j)) + \Bigg(\sum_{i = j + 1}^{n + 1} f(h(i))\Bigg) \\
                                     & = \Bigg(\sum_{i = 1}^{j - 1} f(h(i))\Bigg) + f(x) + \Bigg(\sum_{i = j}^n f(h(i + 1))\Bigg).
    \end{align*}
    We now define the function \(\tilde{h} : \{i \in \mathbf{N} : 1 \leq i \leq n\} \to X - \{x\}\) by setting \(\tilde{h}(i) \coloneqq h(i)\) when \(i < j\) and \(\tilde{h}(i) \coloneqq h(i + 1)\) when \(i \geq j\).
    We can thus write the right-hand side of \eqref{eq 7.1} as
    \[
        = \Bigg(\sum_{i = 1}^{j - 1} f(\tilde{h}(i))\Bigg) + f(x) + \Bigg(\sum_{i = j}^n f(\tilde{h}(i))\Bigg) = \Bigg(\sum_{i = 1}^n f(\tilde{h}(i))\Bigg) + f(x)
    \]
    where we have used Lemma \ref{7.1.4} once again.
    Thus to finish the proof of \eqref{eq 7.1} we have to show that
    \[
        \sum_{i = 1}^n f(g(i)) = \sum_{i = 1}^n f(\tilde{h}(i)). \tag{7.2}\label{eq 7.2}
    \]
    But the function \(g\) (when restricted to \(\{i \in \mathbf{N} : 1 \leq i \leq n\}\)) is a bijection from \(\{i \in \mathbf{N} : 1 \leq i \leq n\} \to X - \{x\}\).
    The function \(\tilde{h}\) is also a bijection from \(\{i \in \mathbf{N} : 1 \leq i \leq n\} \to X - \{x\}\) (cf. Lemma \ref{3.6.9}).
    Since \(X - \{x\}\) has \(n\) elements (by Lemma \ref{3.6.9}), the claim \eqref{eq 7.2} then follows directly from the induction hypothesis \(P(n)\).
\end{proof}

\begin{remark}\label{7.1.9}
    The issue is somewhat more complicated when summing over infinite sets;
    See Section \ref{sec 8.2}.
\end{remark}

\begin{remark}\label{7.1.10}
    Suppose that \(X\) is a set, that \(P(x)\) is a property pertaining to an element \(x\) of \(X\), and \(f : \{y \in X : P(y) \text{ is true}\} \to \mathbf{R}\) is a function.
    Then we will often abbreviate
    \[
        \sum_{x \in \{y \in X : P(y) \text{ is true}\}} f(x)
    \]
    as \(\sum_{x \in X : P(x) \text{ is true}} f(x)\) or even as \(\sum_{P(x) \text{ is true}} f(x)\) when there is no change of confusion.
\end{remark}

\begin{proposition}[Basic properties of summation over finite sets]\label{7.1.11}
    \mbox{}
    \begin{enumerate}
        \item If \(X\) is empty, and \(f : X \to \mathbf{R}\) is a function (i.e., \(f\) is the empty function), we have
              \[
                  \sum_{x \in X} f(x) = 0.
              \]
        \item If \(X\) consists of a single element, \(X = \{x_0\}\), and \(f : X \to \mathbf{R}\) is a function, we have
              \[
                  \sum_{x \in X} f(x) = f(x_0).
              \]
        \item (Substitution, part I) If \(X\) is a finite set, \(f : X \to \mathbf{R}\) is a function, and \(g : Y \to X\) is a bijection, then
              \[
                  \sum_{x \in X} f(x) = \sum_{y \in Y} f(g(y)).
              \]
        \item (Substitution, part II) Let \(n \leq m\) be integers, and let \(X\) be the set \(X \coloneqq \{i \in \mathbf{Z} : n \leq i \leq m\}\).
              If \(a_i\) is a real number assigned to each integer \(i \in X\), then we have
              \[
                  \sum_{i = n}^m a_i = \sum_{i \in X} a_i.
              \]
        \item Let \(X, Y\) be disjoint finite sets (so \(X \cap Y = \emptyset\)), and \(f : X \cup Y \to \mathbf{R}\) is a function.
              Then we have
              \[
                  \sum_{z \in X \cup Y} f(z) = \Bigg(\sum_{x \in X} f(x)\Bigg) + \Bigg(\sum_{y \in Y} f(y)\Bigg).
              \]
        \item (Linearity, part I) Let \(X\) be a finite set, and let \(f : X \to \mathbf{R}\) and \(g : X \to \mathbf{R}\) be functions.
              Then
              \[
                  \sum_{x \in X} (f(x) + g(x)) = \sum_{x \in X} f(x) + \sum_{x \in X} g(x).
              \]
        \item (Linearity, part II) Let \(X\) be a finite set, let \(f : X \to \mathbf{R}\) be a function, and let \(c\) be a real number.
              Then
              \[
                  \sum_{x \in X} cf(x) = c\sum_{x \in X} f(x).
              \]
        \item (Monotonicity) Let \(X\) be a finite set, and let \(f : X \to \mathbf{R}\) and \(g : X \to \mathbf{R}\) be functions such that \(f(x) \leq g(x)\) for all \(x \in \mathbf{X}\).
              Then we have
              \[
                  \sum_{x \in X} f(x) \leq \sum_{x \in X} g(x).
              \]
        \item (Triangle inequality) Let \(X\) be a finite set, and let \(f : X \to \mathbf{R}\) be a function, then
              \[
                  \abs{\sum_{x \in X} f(x)} \leq \sum_{x \in X} \abs{f(x)}.
              \]
    \end{enumerate}
\end{proposition}

\begin{proof}{(a)}
    Let \(g : \{i \in \mathbf{N} : 1 \leq i \leq 0\} \to \emptyset\) be a function.
    Then \(g\) is a bijection and
    \begin{align*}
        \sum_{x \in X} f(x) & = \sum_{i = 1}^0 f(g(i)) & \text{(by Definition \ref{7.1.6})} \\
                            & = 0.                     & \text{(by Definition \ref{7.1.1})}
    \end{align*}
\end{proof}

\begin{proof}{(b)}
    Let \(g : \{1\} \to \{x_0\}\) be a function.
    Then \(g\) is a bijection and
    \begin{align*}
        \sum_{x \in X} f(x) & = \sum_{i = 1}^1 f(g(i))                       & \text{(by Definition \ref{7.1.6})} \\
                            & = \bigg(\sum_{i = 1}^0 f(g(i))\bigg) + f(g(1)) & \text{(by Definition \ref{7.1.1})} \\
                            & = 0 + f(g(1))                                  & \text{(by Definition \ref{7.1.1})} \\
                            & = f(x_0).
    \end{align*}
\end{proof}

\begin{proof}{(c)}
    Let \(h : \{i \in \mathbf{N} : 1 \leq i \leq \#(Y)\} \to Y\) be a bijection.
    Since \(X\) is finite and \(g\) is a bijection between \(X\) and \(Y\), we know that \(Y\) is finite and thus such \(h\) is well-defined.
    Then we know that \(g \circ h : \{i \in \mathbf{N} : 1 \leq i \leq \#(Y)\} \to X\) is also a bijection and
    \begin{align*}
        \sum_{x \in X} f(x) & = \sum_{i = 1}^{\#(Y)} f((g \circ h)(i)) & \text{(by Definition \ref{7.1.6})} \\
                            & = \sum_{i = 1}^{\#(Y)} f(g(h(i)))                                             \\
                            & = \sum_{i = 1}^{\#(Y)} (f \circ g)(h(i))                                      \\
                            & = \sum_{y \in Y} (f \circ g)(y)          & \text{(by Definition \ref{7.1.6})} \\
                            & = \sum_{y \in Y} f(g(y)).
    \end{align*}
\end{proof}

\begin{proof}{(d)}
    Let \(f : X \to \{a_i \in \mathbf{R} : n \leq i \leq m\}\) be a function where \(f = i \mapsto a_i\).
    Let \(g : \{i \in \mathbf{N} : 1 \leq i \leq m - n + 1\} \to X\) be a function where \(g = i \mapsto i + n - 1\).
    Then \(g\) is a bijection and
    \begin{align*}
        \sum_{i \in X} a_i & = \sum_{i \in X} f(i)                                                                                 \\
                           & = \sum_{i = 1}^{m - n + 1} f(g(i))                               & \text{(by Definition \ref{7.1.6})} \\
                           & = \sum_{i = 1}^{m - n + 1} f(i + n - 1)                                                               \\
                           & = \sum_{i = 1}^{m - n + 1} a_{i + n - 1}                                                              \\
                           & = \sum_{i = 1 + n - 1}^{m - n + 1 + n - 1} a_{i + n - 1 - n + 1} & \text{(by Lemma \ref{7.1.4}(b))}   \\
                           & = \sum_{i = n}^m a_i.
    \end{align*}
\end{proof}

\begin{proof}{(e)}
    Let \(g : \{i \in \mathbf{N} : 1 \leq i \leq \#(X)\} \to X\) and \(h : \{i \in \mathbf{N} : 1 \leq i \leq \#(Y)\} \to Y\) be bijections.
    Since \(X, Y\) are finite, we know that \(g, h\) are well-defined and \(X \cup Y\) is finite.
    Let \(k : \{i \in \mathbf{N} : 1 \leq i \leq \#(X \cup Y)\} \to X \cup Y\) be a bijection where
    \[
        k(i) = \begin{cases}
            g(i)         & \text{if } 1 \leq i \leq \#(X)                  \\
            h(i - \#(X)) & \text{if } \#(X) + 1 \leq i \leq \#(X) + \#(Y).
        \end{cases}
    \]
    Since \(X \cup Y\) is finite, we know that \(k\) is well-defined and \(\#(X \cup Y) = \#(X) + \#(Y)\).
    Then we have
    \begin{align*}
        \sum_{z \in X \cup Y} f(z) & = \sum_{i = 1}^{\#(X \cup Y)} f(k(i))                                                & \text{(by Definition \ref{7.1.6})} \\
                                   & = \sum_{i = 1}^{\#(X)} f(k(i)) + \sum_{i = \#(X) + 1}^{\#(X \cup Y)} f(k(i))         & \text{(by Lemma \ref{7.1.4}(a))}   \\
                                   & = \sum_{i = 1}^{\#(X)} f(g(i)) + \sum_{i = \#(X) + 1}^{\#(X \cup Y)} f(h(i - \#(X)))                                      \\
                                   & = \sum_{i = 1}^{\#(X)} f(g(i)) + \sum_{i = 1}^{\#(Y)} f(h(i))                        & \text{(by Lemma \ref{7.1.4}(b))}   \\
                                   & = \sum_{x \in X} f(x) + \sum_{y \in Y} f(y).                                         & \text{(by Definition \ref{7.1.6})}
    \end{align*}
\end{proof}

\begin{proof}{(f)}
    Let \(h : \{i \in \mathbf{N} : 1 \leq i \leq \#(X)\} \to X\) be a bijection.
    Since \(X\) is finite, we know that \(h\) is well-defined and
    \begin{align*}
        \sum_{x \in X} (f(x) + g(x)) & = \sum_{x \in X} (f + g)(x)                                                                        \\
                                     & = \sum_{i = 1}^{\#(X)} (f + g)(h(i))                          & \text{(by Definition \ref{7.1.6})} \\
                                     & = \sum_{i = 1}^{\#(X)} (f(h(i)) + g(h(i)))                                                         \\
                                     & = \sum_{i = 1}^{\#(X)} f(h(i)) + \sum_{i = 1}^{\#(X)} g(h(i)) & \text{(by Lemma \ref{7.1.4}(c))}   \\
                                     & = \sum_{x \in X} f(x) + \sum_{x \in X} g(x).                  & \text{(by Definition \ref{7.1.6})}
    \end{align*}
\end{proof}

\begin{proof}{(g)}
    Let \(g : \{i \in \mathbf{N} : 1 \leq i \leq \#(X)\} \to X\) be a bijection.
    Since \(X\) is finite, we know that \(g\) is well-defined and
    \begin{align*}
        \sum_{x \in X} cf(x) & = \sum_{x \in X} (cf)(x)                                               \\
                             & = \sum_{i = 1}^{\#(X)} (cf)(g(i)) & \text{(by Definition \ref{7.1.6})} \\
                             & = \sum_{i = 1}^{\#(X)} cf(g(i))                                        \\
                             & = c\sum_{i = 1}^{\#(X)} f(g(i))   & \text{(by Lemma \ref{7.1.4}(d))}   \\
                             & = c\sum_{x \in X} f(x).           & \text{(by Definition \ref{7.1.6})}
    \end{align*}
\end{proof}

\begin{proof}{(h)}
    Let \(h : \{i \in \mathbf{N} : 1 \leq i \leq \#(X)\} \to X\) be a bijection.
    Since \(X\) is finite, we know that \(h\) is well-defined and
    \begin{align*}
        \sum_{x \in X} f(x) & = \sum_{i = 1}^{\#(X)} f(h(i))    & \text{(by Definition \ref{7.1.6})} \\
                            & \leq \sum_{i = 1}^{\#(X)} g(h(i)) & \text{(by Lemma \ref{7.1.4}(f))}   \\
                            & = \sum_{x \in X} g(x).            & \text{(by Definition \ref{7.1.6})}
    \end{align*}
\end{proof}

\begin{proof}{(i)}
    Let \(g : \{i \in \mathbf{N} : 1 \leq i \leq \#(X)\} \to X\) be a bijection.
    Since \(X\) is finite, we know that \(g\) is well-defined and
    \begin{align*}
        \abs*{\sum_{x \in X} f(x)} & = \abs*{\sum_{i = 1}^{\#(X)} f(g(i))}    & \text{(by Definition \ref{7.1.6})} \\
                                   & \leq \sum_{i = 1}^{\#(X)} \abs*{f(g(i))} & \text{(by Lemma \ref{7.1.4}(e))}   \\
                                   & = \sum_{x \in X} \abs*{f(x)}.            & \text{(by Definition \ref{7.1.6})}
    \end{align*}
\end{proof}

\begin{remark}\label{7.1.12}
    The substitution rule in Proposition \ref{7.1.11}(c) can be thought of as making the substitution \(x \coloneqq g(y)\) (hence the name).
    Note that the assumption that \(g\) is a bijection is essential.
    From Proposition \ref{7.1.11}(c) and (d) we see that
    \[
        \sum_{i = n}^m a_i = \sum_{i = n}^m a_{f(i)}
    \]
    for any bijection \(f\) from the set \(\{i \in \mathbf{Z} : n \leq i \leq m\}\) to itself.
    Informally, this means that we can rearrange the elements of a finite sequence at will and still obtain the same value.
\end{remark}

\begin{lemma}\label{7.1.13}
    Let \(X, Y\) be finite sets, and let \(f : X \times Y \to \mathbf{R}\) be a function.
    Then
    \[
        \sum_{x \in X} \bigg(\sum_{y \in Y} f(x, y)\bigg) = \sum_{(x, y) \in X \times Y} f(x, y).
    \]
\end{lemma}

\begin{proof}
    Let \(n\) be the number of elements in \(X\).
    We will use induction on \(n\) (cf. Proposition \ref{7.1.8});
    i.e., we let \(P(n)\) be the assertion that Lemma \ref{7.1.13} is true for any set \(X\) with \(n\) elements, and any finite set \(Y\) and any function \(f : X \times Y \to \mathbf{R}\).
    We wish to prove \(P(n)\) for all natural numbers \(n\).

    The base case \(P(0)\) is easy, following from Proposition \ref{7.1.11}(a).
    Now suppose that \(P(n)\) is true;
    we now show that \(P(n + 1)\) is true.
    Let \(X\) be a set with \(n + 1\) elements.
    In particular, by Lemma \ref{3.6.9}, we can write \(X = X' \cup \{x_0\}\), where \(x_0\) is an element of \(X\) and \(X' \coloneqq X - \{x_0\}\) has \(n\) elements.
    Then by Proposition \ref{7.1.11}(e) we have
    \[
        \sum_{x \in X} \bigg(\sum_{y \in Y} f(x, y)\bigg) = \sum_{x \in X'} \bigg(\sum_{y \in Y} f(x, y)\bigg) + \bigg(\sum_{y \in Y} f(x_0, y)\bigg);
    \]
    by the induction hypothesis this is equal to
    \[
        \sum_{(x, y) \in X' \times Y} f(x, y) + \bigg(\sum_{y \in Y} f(x_0, y)\bigg).
    \]
    By Proposition \ref{7.1.11}(c) this is equal to
    \[
        \sum_{(x, y) \in X' \times Y} f(x, y) + \bigg(\sum_{(x, y) \in \{x_0\} \times Y} f(x, y)\bigg).
    \]
    By Proposition \ref{7.1.11}(e) this is equal to
    \[
        \sum_{(x, y) \in X \times Y} f(x, y)
    \]
    as desired.
\end{proof}

\begin{corollary}[Fubini’s theorem for finite series]\label{7.1.14}
    Let \(X, Y\) be finite sets, and let \(f : X \times Y \to \mathbf{R}\) be a function.
    Then
    \begin{align*}
        \sum_{x \in X} \bigg(\sum_{y \in Y} f(x, y)\bigg) & = \sum_{(x, y) \in X \times Y} f(x, y)               \\
                                                          & = \sum_{(y, x) \in Y \times X} f(x, y)               \\
                                                          & = \sum_{y \in Y} \bigg(\sum_{x \in X} f(x, y)\bigg).
    \end{align*}
\end{corollary}

\begin{proof}
    In light of Lemma \ref{7.1.13}, it suffices to show that
    \[
        \sum_{(x, y) \in X \times Y} f(x, y) = \sum_{(y, x) \in Y \times X} f(x, y).
    \]
    But this follows from Proposition \ref{7.1.11}(c) by applying the bijection \(h : X \times Y \to Y \times X\) defined by \(h(x, y) \coloneqq (y, x)\).
\end{proof}

\begin{remark}\label{7.1.15}
    We anticipate something interesting to happen when we move from finite sums to infinite sums.
\end{remark}

\begin{additional corollary}[Products over finite sets]\label{ac 7.1.1}
Let \(m, n\) be integers, and let \((a_i)_{i = m}^n\) be a finite sequence of real numbers, assigning a real number \(a_i\) to each integer \(i\) between \(m\) and \(n\) inclusive (i.e., \(m \leq i \leq n\)).
Then we define the finite product \(\prod_{i = m}^n a_i\) by the recursive formula
\begin{align*}
     & \prod_{i = m}^n a_i \coloneqq 1 \text{ whenever } n < m;                                                             \\
     & \prod_{i = m}^{n + 1} a_i \coloneqq \Bigg(\prod_{i = m}^n a_i\Bigg) \times a_{n + 1} \text{ whenever } n \geq m - 1.
\end{align*}
\end{additional corollary}

\begin{additional corollary}\label{ac 7.1.2}
\mbox{}
\begin{enumerate}
    \item Let \(m \leq n < p\) be integers, and let \(a_i\) be a real number assigned to each integer \(m \leq i \leq p\).
          Then we have
          \[
              \prod_{i = m}^n a_i \times \prod_{i = n + 1}^p a_i = \prod_{i = m}^p a_i.
          \]
    \item Let \(m \leq n\) be integers, \(k\) be another integer, and let \(a_i\) be a real number assigned to each integer \(m \leq i \leq n\).
          Then we have
          \[
              \prod_{i = m}^n a_i = \prod_{j = m + k}^{n + k} a_{j - k}.
          \]
    \item Let \(m \leq n\) be integers, and let \(a_i, b_i\) be real numbers assigned to each integer \(m \leq i \leq n\).
          Then we have
          \[
              \prod_{i = m}^n (a_i \times b_i) = \Bigg(\prod_{i = m}^n a_i\Bigg) \times \Bigg(\prod_{i = m}^n b_i\Bigg).
          \]
    \item Let \(m \leq n\) be integers, and let \(a_i\) be a real number assigned to each integer \(m \leq i \leq n\), and let \(c\) be another real number.
          Then we have
          \[
              \prod_{i = m}^n (ca_i) = c\Bigg(\prod_{i = m}^n a_i\Bigg).
          \]
    \item Let \(m \leq n\) be integers, and let \(a_i\) be a real number assigned to each integer \(m \leq i \leq n\).
          Then we have
          \[
              \abs*{\prod_{i = m}^n a_i} = \prod_{i = m}^n \abs*{a_i}.
          \]
\end{enumerate}
\end{additional corollary}

\begin{proof}{(a)}
    We use induction on \(p\) and start with \(p = m + 1\).
    For \(p = m + 1\), we have
    \[
        \prod_{i = m}^n a_i \times \prod_{i = n + 1}^p a_i = a_m \times a_{m + 1} = \prod_{i = m}^p a_i
    \]
    by Additional Corollary \ref{ac 7.1.1}.
    Thus the base case holds.
    Suppose inductively that for some \(p > m\) the statement holds.
    Then for \(p + 1\), we have
    \begin{align*}
         & \prod_{i = m}^n a_i \times \prod_{i = n + 1}^{p + 1} a_i                                                                                        \\
         & = \Bigg(\prod_{i = m}^n a_i\Bigg) \times \Bigg(\prod_{i = n + 1}^p a_i\Bigg) \times a_{p + 1} & \text{(by Additional Corollary \ref{ac 7.1.1})} \\
         & = \Bigg(\prod_{i = m}^p a_i\Bigg) \times a_{p + 1}                                            & \text{(by induction hypothesis)}                \\
         & = \prod_{i = m}^{p + 1} a_i.                                                                  & \text{(by Additional Corollary \ref{ac 7.1.1})}
    \end{align*}
    This close the induction.
\end{proof}

\begin{proof}{(b)}
    We use induction on \(n\) and start with \(n = m\) to show the statement is true.
    For \(n = m\), we have
    \begin{align*}
        \prod_{j = m + k}^{m + k} a_{j - k} & = \Bigg(\prod_{j = m + k}^{m + k - 1} a_{j - k}\Bigg) \times a_{m + k - k} & \text{(by Additional Corollary \ref{ac 7.1.1})} \\
                                            & = \Bigg(\prod_{j = m + k}^{m + k - 1} a_{j - k}\Bigg) \times a_m                                                             \\
                                            & = 1 \times a_m                                                             & \text{(by Additional Corollary \ref{ac 7.1.1})} \\
                                            & = \Bigg(\prod_{i = m}^{m - 1} a_i\Bigg) \times a_m                         & \text{(by Additional Corollary \ref{ac 7.1.1})} \\
                                            & = \prod_{i = m}^m a_i.                                                     & \text{(by Additional Corollary \ref{ac 7.1.1})}
    \end{align*}
    So the base case holds.
    Suppose inductively that for some \(n \geq m\) the statement holds.
    Then for \(n + 1\), we have
    \begin{align*}
        \prod_{j = m + k}^{n + 1 + k} a_{j - k} & = \Bigg(\prod_{j = m + k}^{n + 1 + k - 1} a_{j - k}\Bigg) \times a_{n + 1 + k - k} & \text{(by Additional Corollary \ref{ac 7.1.1})} \\
                                                & = \Bigg(\prod_{j = m + k}^{n + k} a_{j - k}\Bigg) \times a_{n + 1}                                                                   \\
                                                & = \Bigg(\prod_{i = m}^n a_i\Bigg) \times a_{n + 1}                                 & \text{(by induction hypothesis)}                \\
                                                & = \prod_{i = m}^{n + 1} a_i.                                                       & \text{(by Additional Corollary \ref{ac 7.1.1})}
    \end{align*}
    This close the induction.
\end{proof}

\begin{proof}{(c)}
    We use induction on \(n\) and start with \(n = m\) to show the statement is true.
    For \(n = m\), we have
    \begin{align*}
        \prod_{i = m}^m (a_i \times b_i) & = \Bigg(\prod_{i = m}^{m - 1} (a_i \times b_i)\Bigg) \times a_m \times b_m                                 & \text{(by Additional Corollary \ref{ac 7.1.1})} \\
                                         & = 1 \times a_m \times b_m                                                                                  & \text{(by Additional Corollary \ref{ac 7.1.1})} \\
                                         & = \Bigg(\prod_{i = m}^{m - 1} a_i\Bigg) \times \Bigg(\prod_{i = m}^{m - 1} b_i\Bigg) \times a_m \times b_m & \text{(by Additional Corollary \ref{ac 7.1.1})} \\
                                         & = \Bigg(\prod_{i = m}^m a_i\Bigg) \times \Bigg(\prod_{i = m}^m b_i\Bigg).                                  & \text{(by Additional Corollary \ref{ac 7.1.1})}
    \end{align*}
    So the base case holds.
    Suppose inductively that for some \(n \geq m\) the statement holds.
    Then for \(n + 1\), we have
    \begin{align*}
        \prod_{i = m}^{n + 1} (a_i \times b_i) & = \Bigg(\prod_{i = m}^{n + 1 - 1} (a_i \times b_i)\Bigg) \times a_{n + 1} \times b_{n + 1}                 & \text{(by Additional Corollary \ref{ac 7.1.1})} \\
                                               & = \Bigg(\prod_{i = m}^n (a_i \times b_i)\Bigg) \times a_{n + 1} \times b_{n + 1}                           & \text{(by Additional Corollary \ref{ac 7.1.1})} \\
                                               & = \Bigg(\prod_{i = m}^n a_i\Bigg) \times \Bigg(\prod_{i = m}^n b_i\Bigg) \times a_{n + 1} \times b_{n + 1} & \text{(by induction hypothesis)}                \\
                                               & = \Bigg(\prod_{i = m}^{n + 1} a_i\Bigg) \times \Bigg(\prod_{i = m}^{n + 1} b_i\Bigg).                      & \text{(by Additional Corollary \ref{ac 7.1.1})}
    \end{align*}
    This close the induction.
\end{proof}

\begin{proof}{(d)}
    We use induction on \(n\) and start with \(n = m\) to show the statement is true.
    For \(n = m\), we have
    \begin{align*}
        \prod_{i = m}^m ca_i & = \Bigg(\prod_{i = m}^{m - 1} ca_i\Bigg) \times ca_m & \text{(by Additional Corollary \ref{ac 7.1.1})} \\
                             & = 1 \times ca_m                                      & \text{(by Additional Corollary \ref{ac 7.1.1})} \\
                             & = c \times a_m                                                                                         \\
                             & = c\Bigg(\prod_{i = m}^m a_i\Bigg).                  & \text{(by Additional Corollary \ref{ac 7.1.1})}
    \end{align*}
    So the base case holds.
    Suppose inductively that for some \(n \geq m\) the statement holds.
    Then for \(n + 1\), we have
    \begin{align*}
        \prod_{i = m}^{n + 1} ca_i & = \Bigg(\prod_{i = m}^{n + 1 - 1} ca_i\Bigg) \times ca_{n + 1}  & \text{(by Additional Corollary \ref{ac 7.1.1})} \\
                                   & = \Bigg(\prod_{i = m}^n ca_i\Bigg) \times ca_{n + 1}                                                              \\
                                   & = c\Bigg(\prod_{i = m}^n a_i\Bigg) \times ca_{n + 1}            & \text{(by induction hypothesis)}                \\
                                   & = c\Bigg(\Bigg(\prod_{i = m}^n a_i\Bigg) \times a_{n + 1}\Bigg)                                                   \\
                                   & = c\Bigg(\prod_{i = m}^{n + 1} a_i\Bigg).                       & \text{(by Additional Corollary \ref{ac 7.1.1})}
    \end{align*}
    This close the induction.
\end{proof}

\begin{proof}{(e)}
    We use induction on \(n\) and start with \(n = m\) to show the statement is true.
    For \(n = m\), we have
    \begin{align*}
        \abs*{\prod_{i = m}^m a_i} & = \abs*{\Bigg(\prod_{i = m}^{m - 1} a_i\Bigg) \times a_m}        & \text{(by Additional Corollary \ref{ac 7.1.1})} \\
                                   & = \abs*{1a_m}                                                    & \text{(by Additional Corollary \ref{ac 7.1.1})} \\
                                   & = \abs*{1}\abs*{a_m}                                                                                               \\
                                   & = \Bigg(\prod_{i = m}^{m - 1} \abs*{a_i}\Bigg) \times \abs*{a_m} & \text{(by Additional Corollary \ref{ac 7.1.1})} \\
                                   & = \prod_{i = m}^m \abs*{a_i}.                                    & \text{(by Additional Corollary \ref{ac 7.1.1})}
    \end{align*}
    So the base case holds.
    Suppose inductively that for some \(n \geq m\) the statement holds.
    Then for \(n + 1\), we have
    \begin{align*}
        \abs*{\prod_{i = m}^{n + 1} a_i} & = \abs*{\Bigg(\prod_{i = m}^{n + 1 - 1} a_i\Bigg) \times a_{n + 1}} & \text{(by Additional Corollary \ref{ac 7.1.1})} \\
                                         & = \abs*{\Bigg(\prod_{i = m}^n a_i\Bigg) \times a_{n + 1}}                                                             \\
                                         & = \abs*{\prod_{i = m}^n a_i} \times \abs*{a_{n + 1}}                                                                  \\
                                         & = \prod_{i = m}^n \abs*{a_i} \times \abs*{a_{n + 1}}                & \text{(by induction hypothesis)}                \\
                                         & = \prod_{i = m}^{n + 1} \abs*{a_i}.                                 & \text{(by Additional Corollary \ref{ac 7.1.1})}
    \end{align*}
    This close the induction.
\end{proof}

\begin{additional corollary}\label{ac 7.1.3}
Let \(X\) be a finite set with \(n\) elements (where \(n \in \mathbf{N}\)), and let \(f : X \to \mathbf{R}\) be a function from \(X\) to the real numbers
(i.e., \(f\) assigns a real number \(f(x)\) to each element \(x\) of \(X\)).
Then we can define the finite sum \(\prod_{x \in X} f(x)\) as follows.
We first select any bijection \(g\) from \(\{i \in \mathbf{N} : 1 \leq i \leq n\}\) to \(X\);
such a bijection exists since \(X\) is assumed to have \(n\) elements.
We then define
\[
    \prod_{x \in X} f(x) \coloneqq \prod_{i = 1}^n f(g(i))
\]
\end{additional corollary}

\begin{additional corollary}[Finite products are well-defined]\label{ac 7.1.4}
Let \(X\) be a finite set with \(n\) elements (where \(n \in \mathbf{N}\)), let \(f : X \to \mathbf{R}\) be a function, and let \(g : \{i \in \mathbf{N} : 1 \leq i \leq n\} \to X\) and \(h : \{i \in \mathbf{N} : 1 \leq i \leq n\} \to X\) be bijections.
Then we have
\[
    \prod_{i = 1}^n f(g(i)) = \prod_{i = 1}^n f(h(i)).
\]
\end{additional corollary}

\begin{proof}
    We use induction on \(n\);
    more precisely, we let \(P(n)\) be the assertion that ``For any set \(X\) of \(n\) elements, any function \(f : X \to \mathbf{R}\), and any two bijections \(g, h\) from \(\{i \in \mathbf{N} : 1 \leq i \leq n\}\) to \(X\), we have \(\prod_{i = 1}^n f(g(i)) = \prod_{i = 1}^n f(h(i))\)''.
    (More informally, \(P(n)\) is the assertion that Proposition \ref{7.1.8} is true for that value of \(n\).)
    We want to prove that \(P(n)\) is true for all natural numbers \(n\).

    We first check the base case \(P(0)\).
    In this case \(\prod_{i = 1}^0 f(g(i))\) and \(\prod_{i = 1}^0 f(h(i))\) both equal to \(1\), by Additional Corollary \ref{ac 7.1.1}, so we are done.

    Now suppose inductively that \(P(n)\) is true;
    we now prove that \(P(n + 1)\) is true.
    Thus, let \(X\) be a set with \(n + 1\) elements, let \(f : X \to \mathbf{R}\) be a function, and let \(g\) and \(h\) be bijections from \(\{i \in N : 1 \leq i \leq n + 1\}\) to \(X\).
    We have to prove that
    \[
        \prod_{i = 1}^{n + 1} f(g(i)) = \prod_{i = 1}^{n + 1} f(h(i)). \tag{ac 7.1}\label{eq ac 7.1}
    \]
    Let \(x \coloneqq g(n + 1)\);
    thus \(x\) is an element of \(X\).
    By Additional Corollary \ref{ac 7.1.1}, we can expand the left-hand side of \eqref{eq ac 7.1} as
    \[
        \prod_{i = 1}^{n + 1} f(g(i)) = \Bigg(\prod_{i = 1}^n f(g(i))\Bigg) \times f(x).
    \]
    Now let us look at the right-hand side of \eqref{eq ac 7.1}.
    Ideally we would like to have \(h(n + 1)\) also equal to \(x\)
    - this would allow us to use the inductive hypothesis \(P(n)\) much more easily
    - but we cannot assume this.
    However, since \(h\) is a bijection, we do know that there is \emph{some} index \(j\), with \(1 \leq j \leq n + 1\), for which \(h(j) = x\).
    We now use Additional Corollary \ref{ac 7.1.1} and \ref{ac 7.1.2} to write
    \begin{align*}
        \prod_{i = 1}^{n + 1} f(h(i)) & = \Bigg(\prod_{i = 1}^j f(h(i))\Bigg) \times \Bigg(\prod_{i = j + 1}^{n + 1} f(h(i))\Bigg)                      \\
                                      & = \Bigg(\prod_{i = 1}^{j - 1} f(h(i))\Bigg) \times f(h(j)) \times \Bigg(\prod_{i = j + 1}^{n + 1} f(h(i))\Bigg) \\
                                      & = \Bigg(\prod_{i = 1}^{j - 1} f(h(i))\Bigg) \times f(x) \times \Bigg(\prod_{i = j}^n f(h(i + 1))\Bigg).
    \end{align*}
    We now define the function \(\tilde{h} : \{i \in \mathbf{N} : 1 \leq i \leq n\} \to X - \{x\}\) by setting \(\tilde{h}(i) \coloneqq h(i)\) when \(i < j\) and \(\tilde{h}(i) \coloneqq h(i + 1)\) when \(i \geq j\).
    We can thus write the right-hand side of \eqref{eq ac 7.1} as
    \[
        = \Bigg(\prod_{i = 1}^{j - 1} f(\tilde{h}(i))\Bigg) \times f(x) \times \Bigg(\prod_{i = j}^n f(\tilde{h}(i))\Bigg) = \Bigg(\prod_{i = 1}^n f(\tilde{h}(i))\Bigg) \times f(x)
    \]
    where we have used Additional Corollary \ref{ac 7.1.2} once again.
    Thus to finish the proof of \eqref{eq ac 7.1} we have to show that
    \[
        \prod_{i = 1}^n f(g(i)) = \prod_{i = 1}^n f(\tilde{h}(i)). \tag{ac 7.2}\label{eq ac 7.2}
    \]
    But the function \(g\) (when restricted to \(\{i \in \mathbf{N} : 1 \leq i \leq n\}\)) is a bijection from \(\{i \in \mathbf{N} : 1 \leq i \leq n\} \to X - \{x\}\).
    The function \(\tilde{h}\) is also a bijection from \(\{i \in \mathbf{N} : 1 \leq i \leq n\} \to X - \{x\}\) (cf. Lemma \ref{3.6.9}).
    Since \(X - \{x\}\) has \(n\) elements (by Lemma \ref{3.6.9}), the claim \eqref{eq ac 7.2} then follows directly from the induction hypothesis \(P(n)\).
\end{proof}

\begin{additional corollary}[Basic properties of product over finite sets]\label{ac 7.1.5}
\mbox{}
\begin{enumerate}
    \item If \(X\) is empty, and \(f : X \to \mathbf{R}\) is a function (i.e., \(f\) is the empty function), we have
          \[
              \prod_{x \in X} f(x) = 1.
          \]
    \item If \(X\) consists of a single element, \(X = \{x_0\}\), and \(f : X \to \mathbf{R}\) is a function, we have
          \[
              \prod_{x \in X} f(x) = f(x_0).
          \]
    \item (Substitution, part I) If \(X\) is a finite set, \(f : X \to \mathbf{R}\) is a function, and \(g : Y \to X\) is a bijection, then
          \[
              \prod_{x \in X} f(x) = \prod_{y \in Y} f(g(y)).
          \]
    \item (Substitution, part II) Let \(n \leq m\) be integers, and let \(X\) be the set \(X \coloneqq \{i \in \mathbf{Z} : n \leq i \leq m\}\).
          If \(a_i\) is a real number assigned to each integer \(i \in X\), then we have
          \[
              \prod_{i = n}^m a_i = \prod_{i \in X} a_i.
          \]
    \item Let \(X, Y\) be disjoint finite sets (so \(X \cap Y = \emptyset\)), and \(f : X \cup Y \to \mathbf{R}\) is a function.
          Then we have
          \[
              \prod_{z \in X \cup Y} f(z) = \Bigg(\prod_{x \in X} f(x)\Bigg) \times \Bigg(\prod_{y \in Y} f(y)\Bigg).
          \]
    \item Let \(X\) be a finite set, and let \(f : X \to \mathbf{R}\) and \(g : X \to \mathbf{R}\) be functions.
          Then
          \[
              \prod_{x \in X} (f(x) \times g(x)) = \prod_{x \in X} f(x) \times \prod_{x \in X} g(x).
          \]
    \item Let \(X\) be a finite set, let \(f : X \to \mathbf{R}\) be a function, and let \(c\) be a real number.
          Then
          \[
              \prod_{x \in X} cf(x) = c\prod_{x \in X} f(x).
          \]
    \item Let \(X\) be a finite set, and let \(f : X \to \mathbf{R}\) be a function, then
          \[
              \abs{\prod_{x \in X} f(x)} = \prod_{x \in X} \abs{f(x)}.
          \]
\end{enumerate}
\end{additional corollary}

\begin{proof}{(a)}
    Let \(g : \{i \in \mathbf{N} : 1 \leq i \leq 0\} \to \emptyset\) be a function.
    Then \(g\) is a bijection.
    So
    \begin{align*}
        \prod_{x \in X} f(x) & = \prod_{i = 1}^0 f(g(i)) & \text{(by Additional Corollary \ref{ac 7.1.3})} \\
                             & = 1.                      & \text{(by Additional Corollary \ref{ac 7.1.1})}
    \end{align*}
\end{proof}

\begin{proof}{(b)}
    Let \(g : \{1\} \to X\) be a function.
    Then \(g\) is a bijection.
    So
    \begin{align*}
        \prod_{x \in X} f(x) & = \prod_{i = 1}^1 f(g(i))                            & \text{(by Additional Corollary \ref{ac 7.1.3})} \\
                             & = \bigg(\prod_{i = 1}^0 f(g(i))\bigg) \times f(g(1)) & \text{(by Additional Corollary \ref{ac 7.1.1})} \\
                             & = 1 \times f(g(1))                                   & \text{(by Additional Corollary \ref{ac 7.1.1})} \\
                             & = f(x_0).
    \end{align*}
\end{proof}

\begin{proof}{(c)}
    Let \(h : \{i \in \mathbf{N} : 1 \leq i \leq \#(Y)\} \to Y\) be a bijection.
    Then \(g \circ h : \{i \in \mathbf{N} : 1 \leq i \leq \#(Y)\} \to X\) is also a bijection.
    So
    \begin{align*}
        \prod_{x \in X} f(x) & = \prod_{i = 1}^{\#(Y)} f((g \circ h)(i)) & \text{(by Additional Corollary \ref{ac 7.1.3})} \\
                             & = \prod_{i = 1}^{\#(Y)} f(g(h(i)))                                                          \\
                             & = \prod_{i = 1}^{\#(Y)} (f \circ g)(h(i))                                                   \\
                             & = \prod_{y \in Y} (f \circ g)(y)          & \text{(by Additional Corollary \ref{ac 7.1.3})} \\
                             & = \prod_{y \in Y} f(g(y)).
    \end{align*}
\end{proof}

\begin{proof}{(d)}
    Let \(f : X \to \{a_i \in \mathbf{R} : n \leq i \leq m\}\) be a function where \(f = i \mapsto a_i\).
    Let \(g : \{i \in \mathbf{N} : 1 \leq i \leq m - n + 1\} \to X\) be a function where \(g = i \mapsto i + n - 1\).
    Then \(g\) is a bijection.
    So
    \begin{align*}
        \prod_{i \in X} a_i & = \prod_{i \in X} f(i)                                                                                        \\
                            & = \prod_{i = 1}^{m - n + 1} f(g(i))                         & \text{(by Additional Corollary \ref{ac 7.1.3})} \\
                            & = \prod_{i = 1 + n - 1}^{m - n + 1 + n - 1} f(g(i - n + 1)) & \text{(by Additional Corollary \ref{ac 7.1.2})} \\
                            & = \prod_{i = n}^m f(g(i - n + 1))                                                                             \\
                            & = \prod_{i = n}^m f(i - n + 1 + n - 1)                                                                        \\
                            & = \prod_{i = n}^m f(i)                                                                                        \\
                            & = \prod_{i = n}^m a_i.
    \end{align*}
\end{proof}

\begin{proof}{(e)}
    Let \(g : \{i \in \mathbf{N} : 1 \leq i \leq \#(X)\} \to X\) and \(h : \{i \in \mathbf{N} : 1 \leq i \leq \#(Y)\} \to Y\) be bijections.
    Let \(k : \{i \in \mathbf{N} : 1 \leq i \leq \#(X \cup Y)\} \to X \cup Y\) be a function where \(k(i) = g(i)\) if \(1 \leq i \leq \#(X)\) and \(k(i) = h(i - \#(X))\) otherwise.
    Then \(k\) is a bijection.
    So
    \begin{align*}
        \prod_{z \in X \cup Y} f(z) & = \prod_{i = 1}^{\#(X \cup Y)} f(k(i))                                                      & \text{(by Additional Corollary \ref{ac 7.1.3})} \\
                                    & = \prod_{i = 1}^{\#(X)} f(k(i)) \times \prod_{i = \#(X) + 1}^{\#(X \cup Y)} f(k(i))         & \text{(by Additional Corollary \ref{ac 7.1.2})} \\
                                    & = \prod_{i = 1}^{\#(X)} f(g(i)) \times \prod_{i = \#(X) + 1}^{\#(X \cup Y)} f(h(i - \#(X)))                                                   \\
                                    & = \prod_{i = 1}^{\#(X)} f(g(i)) \times \prod_{i = 1}^{\#(Y)} f(h(i))                        & \text{(by Additional Corollary \ref{ac 7.1.2})} \\
                                    & = \prod_{x \in X} f(x) \times \prod_{y \in Y} f(y).                                         & \text{(by Additional Corollary \ref{ac 7.1.3})}
    \end{align*}
\end{proof}

\begin{proof}{(f)}
    Let \(h : \{i \in \mathbf{N} : 1 \leq i \leq \#(X)\} \to X\) be a bijection.
    So
    \begin{align*}
        \prod_{x \in X} (f(x) \times g(x)) & = \prod_{x \in X} (f \times g)(x)                                                                                      \\
                                           & = \prod_{i = 1}^{\#(X)} (f \times g)(h(i))                           & \text{(by Additional Corollary \ref{ac 7.1.3})} \\
                                           & = \prod_{i = 1}^{\#(X)} (f(h(i)) \times g(h(i)))                                                                       \\
                                           & = \prod_{i = 1}^{\#(X)} f(h(i)) \times \prod_{i = 1}^{\#(X)} g(h(i)) & \text{(by Additional Corollary \ref{ac 7.1.2})} \\
                                           & = \prod_{x \in X} f(x) \times \prod_{x \in X} g(x).                  & \text{(by Additional Corollary \ref{ac 7.1.3})}
    \end{align*}
\end{proof}

\begin{proof}{(g)}
    Let \(g : \{i \in \mathbf{N} : 1 \leq i \leq \#(X)\} \to X\) be a bijection.
    So
    \begin{align*}
        \prod_{x \in X} cf(x) & = \prod_{x \in X} (cf)(x)                                                            \\
                              & = \prod_{i = 1}^{\#(X)} (cf)(g(i)) & \text{(by Additional Corollary \ref{ac 7.1.3})} \\
                              & = \prod_{i = 1}^{\#(X)} cf(g(i))                                                     \\
                              & = c\prod_{i = 1}^{\#(X)} f(g(i))   & \text{(by Additional Corollary \ref{ac 7.1.2})} \\
                              & = c\prod_{x \in X} f(x).           & \text{(by Additional Corollary \ref{ac 7.1.3})}
    \end{align*}
\end{proof}

\begin{proof}{(h)}
    Let \(g : \{i \in \mathbf{N} : 1 \leq i \leq \#(X)\} \to X\) be a bijection.
    So
    \begin{align*}
        \abs*{\prod_{x \in X} f(x)} & = \abs*{\prod_{i = 1}^{\#(X)} f(g(i))} & \text{(by Additional Corollary \ref{ac 7.1.3})} \\
                                    & = \prod_{i = 1}^{\#(X)} \abs*{f(g(i))} & \text{(by Additional Corollary \ref{ac 7.1.2})} \\
                                    & = \prod_{x \in X} \abs*{f(x)}.         & \text{(by Additional Corollary \ref{ac 7.1.3})}
    \end{align*}
\end{proof}

\begin{additional corollary}\label{ac 7.1.6}
Let \(X, Y\) be finite sets, and let \(f : X \times Y \to \mathbf{R}\) be a function.
Then
\[
    \prod_{x \in X} \bigg(\prod_{y \in Y} f(x, y)\bigg) = \prod_{(x, y) \in X \times Y} f(x, y).
\]
\end{additional corollary}

\begin{proof}
    Let \(n\) be the number of elements in \(X\).
    We will use induction on \(n\) (cf. Additional Corollary \ref{ac 7.1.4});
    i.e., we let \(P(n)\) be the assertion that Additional Corollary \ref{ac 7.1.6} is true for any set \(X\) with \(n\) elements, and any finite set \(Y\) and any function \(f : X \times Y \to \mathbf{R}\).
    We wish to prove \(P(n)\) for all natural numbers \(n\).

    The base case \(P(0)\) is easy, following from Additional Corollary \ref{ac 7.1.5}(a).
    Now suppose that \(P(n)\) is true;
    we now show that \(P(n + 1)\) is true.
    Let \(X\) be a set with \(n + 1\) elements.
    In particular, by Lemma \ref{3.6.9}, we can write \(X = X' \cup \{x_0\}\), where \(x_0\) is an element of \(X\) and \(X' \coloneqq X - \{x_0\}\) has \(n\) elements.
    Then by Additional Corollary \ref{7.1.5}(e) we have
    \[
        \prod_{x \in X} \bigg(\prod_{y \in Y} f(x, y)\bigg) = \prod_{x \in X'} \bigg(\prod_{y \in Y} f(x, y)\bigg) \times \bigg(\prod_{y \in Y} f(x_0, y)\bigg);
    \]
    by the induction hypothesis this is equal to
    \[
        \prod_{(x, y) \in X' \times Y} f(x, y) \times \bigg(\prod_{y \in Y} f(x_0, y)\bigg).
    \]
    By Additional Corollary \ref{7.1.11}(c) this is equal to
    \[
        \prod_{(x, y) \in X' \times Y} f(x, y) \times \bigg(\prod_{(x, y) \in \{x_0\} \times Y} f(x, y)\bigg).
    \]
    By Additional Corollary \ref{7.1.11}(e) this is equal to
    \[
        \prod_{(x, y) \in X \times Y} f(x, y)
    \]
    as desired.
\end{proof}

\begin{additional corollary}\label{ac 7.1.7}
Let \(X, Y\) be finite sets, and let \(f : X \times Y \to \mathbf{R}\) be a function.
Then
\begin{align*}
    \prod_{x \in X} \bigg(\prod_{y \in Y} f(x, y)\bigg) & = \prod_{(x, y) \in X \times Y} f(x, y)                \\
                                                        & = \prod_{(y, x) \in Y \times X} f(x, y)                \\
                                                        & = \prod_{y \in Y} \bigg(\prod_{x \in X} f(x, y)\bigg).
\end{align*}
\end{additional corollary}

\begin{proof}
    In light of Additional Corollary \ref{ac 7.1.6}, it suffices to show that
    \[
        \prod_{(x, y) \in X \times Y} f(x, y) = \prod_{(y, x) \in Y \times X} f(x, y).
    \]
    But this follows from Additional Corollary \ref{ac 7.1.5}(c) by applying the bijection \(h : X \times Y \to Y \times X\) defined by \(h(x, y) \coloneqq (y, x)\).
\end{proof}

\exercisesection

\begin{exercise}\label{ex 7.1.1}
    Prove Lemma \ref{7.1.4}.
\end{exercise}

\begin{proof}
    See Lemma \ref{7.1.4}.
\end{proof}

\begin{exercise}\label{ex 7.1.2}
    Prove Proposition \ref{7.1.11}.
\end{exercise}

\begin{proof}
    See Proposition \ref{7.1.11}.
\end{proof}

\begin{exercise}\label{ex 7.1.3}
    Form a definition for the finite products \(\prod_{i = 1}^n a_i\) and \(\prod_{x \in X} f(x)\).
    Which of the above result for finite series have analoges for finite products?
\end{exercise}

\begin{proof}
    See Additional Corollary \ref{ac 7.1.1}, \ref{ac 7.1.2}, \ref{ac 7.1.3}, \ref{ac 7.1.4}, \ref{ac 7.1.5}, \ref{ac 7.1.6} and \ref{ac 7.1.7}.
\end{proof}

\begin{exercise}\label{ex 7.1.4}
    Define the \emph{factorial function} \(n!\) for natural numbers \(n\) by the recursive definition \(0! \coloneqq 1\) and \((n + 1)! \coloneqq n! \times (n + 1)\).
    If \(x\) and \(y\) are real numbers, prove the \emph{binomial formula}
    \[
        (x + y)^n = \sum_{j = 0}^n \frac{n!}{j!(n - j)!} x^j y^{n - j}
    \]
    for all natural numbers \(n\).
\end{exercise}

\begin{proof}
    We use induction on \(n\).
    For \(n = 0\), we have
    \begin{align*}
        (x + y)^0 & = 1                                                                                                                                \\
                  & = \frac{0!}{0!(0 - 0)!} x^0 y^{0 - 0}                                                         & \text{(by the given condition)}    \\
                  & = \sum_{j = 0}^{-1} \frac{0!}{j!(0 - j)!} x^j y^{0 - j} + \frac{0!}{0!(0 - 0)!} x^0 y^{0 - 0} & \text{(by Definition \ref{7.1.1})} \\
                  & = \sum_{j = 0}^0 \frac{0!}{j!(0 - j)!} x^j y^{0 - j}                                          & \text{(by Definition \ref{7.1.1})}
    \end{align*}
    So the base case holds.
    Suppose inductively that for some \(n >= 0\) the statement holds.
    Then for \(n + 1\), we have
    \begin{align*}
        (x + y)^{n + 1} & = (x + y)^n \times (x + y)                                                                                                      \\
                        & = \bigg(\sum_{j = 0}^n \frac{n!}{j!(n - j)!} x^j y^{n - j}\bigg) \times (x + y)            & \text{(by induction hypothesis)}   \\
                        & = \bigg(\sum_{j = 0}^n \frac{n!}{j!(n - j)!} x^{j + 1} y^{n - j}\bigg)                                                          \\
                        & \quad + \bigg(\sum_{j = 0}^n \frac{n!}{j!(n - j)!} x^j y^{n + 1 - j}\bigg)                                                      \\
                        & = \bigg(\sum_{j = 0}^{n - 1} \frac{n!}{j!(n - j)!} x^{j + 1} y^{n - j}\bigg)               & \text{(by Definition \ref{7.1.1})} \\
                        & \quad + \bigg(\frac{n!}{n!0!} x^{n + 1} y^0\bigg)                                                                               \\
                        & \quad + \bigg(\sum_{j = 1}^n \frac{n!}{j!(n - j)!} x^j y^{n + 1 - j}\bigg)                                                      \\
                        & \quad + \bigg(\frac{n!}{0!n!} x^0 y^{n + 1}\bigg)                                                                               \\
                        & = \bigg(\sum_{j = 0}^{n - 1} \frac{n!}{j!(n - j)!} x^{j + 1} y^{n - j}\bigg) + x^{n + 1}   & \text{(by the given condition)}    \\
                        & \quad + \bigg(\sum_{j = 1}^n \frac{n!}{j!(n - j)!} x^j y^{n + 1 - j}\bigg) + y^{n + 1}                                          \\
                        & = \bigg(\sum_{j = 1}^n \frac{n!}{(j - 1)!(n + 1 - j)!} x^j y^{n + 1 - j}\bigg) + x^{n + 1} & \text{(by Lemma \ref{7.1.4})}      \\
                        & \quad + \bigg(\sum_{j = 1}^n \frac{n!}{j!(n - j)!} x^j y^{n + 1 - j}\bigg) + y^{n + 1}                                          \\
                        & = x^{n + 1} + y^{n + 1}                                                                    & \text{(by Lemma \ref{7.1.4})}      \\
                        & \quad + \bigg(\sum_{j = 1}^n \frac{(n + 1)!}{j!(n + 1 - j)!} x^j y^{n + 1 - j}\bigg).
    \end{align*}
    And we also have
    \begin{align*}
         & \sum_{j = 0}^{n + 1} \frac{(n + 1)!}{j!(n + 1 - j)!} x^j y^{n + 1 - j}                                                                       \\
         & = \bigg(\sum_{j = n + 1}^{n + 1} \frac{(n + 1)!}{j!(n + 1 - j)!} x^j y^{n + 1 - j}\bigg)                & \text{(by Definition \ref{7.1.1})} \\
         & \quad + \bigg(\sum_{j = 0}^n \frac{(n + 1)!}{j!(n + 1 - j)!} x^j y^{n + 1 - j}\bigg)                                                         \\
         & = x^{n + 1} + \bigg(\sum_{j = 0}^n \frac{(n + 1)!}{j!(n + 1 - j)!} x^j y^{n + 1 - j}\bigg)              & \text{(by the given condition)}    \\
         & = x^{n + 1} + \bigg(\sum_{j = 0}^0 \frac{(n + 1)!}{j!(n + 1 - j)!} x^j y^{n + 1 - j}\bigg)              & \text{(by Definition \ref{7.1.1})} \\
         & \quad + \bigg(\sum_{j = 1}^n \frac{(n + 1)!}{j!(n + 1 - j)!} x^j y^{n + 1 - j}\bigg)                                                         \\
         & = x^{n + 1} + y^{n + 1} + \bigg(\sum_{j = 1}^n \frac{(n + 1)!}{j!(n + 1 - j)!} x^j y^{n + 1 - j}\bigg). & \text{(by the given condition)}
    \end{align*}
    Thus we have
    \[
        (x + y)^{n + 1} = \sum_{j = 0}^{n + 1} \frac{(n + 1)!}{j!(n + 1 - j)!} x^j y^{n + 1 - j}.
    \]
    And this close the induction.
\end{proof}

\begin{exercise}\label{ex 7.1.5}
    Let \(X\) be a finite set, let \(m\) be an integer, and for each \(x \in X\) let \((a_n(x))_{n = m}^\infty\) be a convergent sequence of real numbers.
    Show that the sequence \((\sum_{x \in X} a_n(x))_{n = m}^\infty\) is convergent, and
    \[
        \lim_{n \to \infty} \sum_{x \in X} a_n(x) = \sum_{x \in X} \lim_{n \to \infty} a_n(x).
    \]
    Thus we may always interchange finite sums with convergent limits.
    Things however get trickier with infinite sums.
\end{exercise}

\begin{proof}
    Let \(k = \#(X)\).
    We use induction on \(k\).
    For \(k = 0\), we have \(X = \emptyset\).
    So
    \begin{align*}
        \lim_{n \to \infty} \sum_{x \in X} a_n(x) & = \lim_{n \to \infty} 0                      & \text{(by Proposition \ref{7.1.11})} \\
                                                  & = 0                                                                                 \\
                                                  & = \sum_{x \in X} \lim_{n \to \infty} a_n(x). & \text{(by Proposition \ref{7.1.11})}
    \end{align*}
    Thus the base case holds.
    Suppose inductively that for some \(k \geq 0\) the statement is true.
    Then for \(k + 1\), we have to show that the statement is also true.
    Let \(x_0 \in X\) and \(X' = X \setminus \{x_0\}\).
    So \(\#(X') = \#(X) - 1 = n\), and we have
    \begin{align*}
        \lim_{n \to \infty} \sum_{x \in X} a_n(x) & = \lim_{n \to \infty} \sum_{x \in \{x_0\} \cup X'} a_n(x)                                                                                                     \\
                                                  & = \lim_{n \to \infty} \bigg(\sum_{x \in \{x_0\}} a_n(x) + \sum_{x \in X'} a_n(x)\bigg)                                 & \text{(by Proposition \ref{7.1.11})} \\
                                                  & = \bigg(\lim_{n \to \infty} \sum_{x \in \{x_0\}} a_n(x)\bigg) + \bigg(\lim_{n \to \infty} \sum_{x \in X'} a_n(x)\bigg) & \text{(by Theorem \ref{6.1.19})}     \\
                                                  & = \bigg(\lim_{n \to \infty} a_n(x_0)\bigg) + \bigg(\lim_{n \to \infty} \sum_{x \in X'} a_n(x)\bigg)                    & \text{(by Proposition \ref{7.1.11})} \\
                                                  & = \bigg(\sum_{x \in \{x_0\}} \lim_{n \to \infty} a_n(x)\bigg) + \bigg(\lim_{n \to \infty} \sum_{x \in X'} a_n(x)\bigg) & \text{(by Proposition \ref{7.1.11})} \\
                                                  & = \bigg(\sum_{x \in \{x_0\}} \lim_{n \to \infty} a_n(x)\bigg) + \bigg(\sum_{x \in X'} \lim_{n \to \infty} a_n(x)\bigg) & \text{(by induction hypothesis)}     \\
                                                  & = \bigg(\sum_{x \in \{x_0\} \cup X'} \lim_{n \to \infty} a_n(x)\bigg)                                                  & \text{(by Proposition \ref{7.1.11})} \\
                                                  & = \sum_{x \in X} \lim_{n \to \infty} a_n(x).
    \end{align*}
    This close the induction.
\end{proof}