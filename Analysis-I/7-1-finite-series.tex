\section{Finite series}\label{sec 7.1}

\begin{definition}[Finite series]\label{7.1.1}
Let \(m, n\) be integers, and let \((a_i)_{i = m}^n\) be a finite sequence of real numbers, assigning a real number \(a_i\) to each integer \(i\) between \(m\) and \(n\) inclusive (i.e., \(m \leq i \leq n\)).
Then we define the finite sum (or finite series) \(\sum_{i = m}^n a_i\) by the recursive formula
\begin{align*}
& \sum_{i = m}^n a_i \coloneqq 0 \text{ whenever } n < m ; \\
& \sum_{i = m}^{n + 1} a_i \coloneqq \Bigg(\sum_{i = m}^n a_i\Bigg) + a_{n + 1} \text{ whenever } n \geq m - 1.
\end{align*}
\end{definition}

\begin{note}
we sometimes express \(\sum_{i = m}^n a_i\) less formally as
\[
    \sum_{i = m}^n a_i = a_m + a_{m + 1} + \dots + a_n.
\]
\end{note}

\begin{remark}\label{7.1.2}
The difference between ``sum'' and ``series'' is a subtle linguistic one.
Strictly speaking, a series is an \emph{expression} of the form \(\sum_{i = m}^n a_i\);
this series is mathematically (but not semantically) equal to a real number, which is then the \emph{sum} of that series.
For instance, \(1 + 2 + 3 + 4 + 5\) is a series, whose sum is \(15\);
if one were to be very picky about semantics, one would not consider \(15\) a series and one would not consider \(1 + 2 + 3 + 4 + 5\) a sum, despite the two expressions having the same value.
However, we will not be very careful about this distinction as it is purely linguistic and has no bearing on the mathematics;
the expressions \(1 + 2 + 3 + 4 + 5\) and \(15\) are the same number, and thus \emph{mathematically} interchangeable, in the sense of the axiom of substitution, even if they are not semantically interchangeable.
\end{remark}

\begin{remark}\label{7.1.3}
Note that the variable \(i\) (sometimes called the \emph{index of summation}) is a \emph{bound variable} (sometimes called a \emph{dummy variable});
the expression \(\sum_{i = m}^n a_i\) does not actually depend on any quantity named \(i\).
In particular, one can replace the index of summation \(i\) with any other symbol, and obtain the same sum:
\[
    \sum_{i = m}^n a_i = \sum_{j = m}^n a_j.
\]
\end{remark}