\section{Infinite series}\label{sec 7.2}

\begin{definition}[Formal infinite series]\label{7.2.1}
A (formal) infinite series is any expression of the form
\[
    \sum_{n = m}^\infty a_n,
\]
where \(m\) is an integer, and \(a_n\) is a real number for any integer \(n \geq m\).
\end{definition}

\begin{note}
We sometimes write this series as
\[
    a_m + a_{m + 1} + a_{m + 2} + \dots.
\]
\end{note}

\begin{note}
At present, this series is only defined formally;
we have not set this sum equal to any real number;
the notation \(a_m + a_{m + 1} + a_{m + 2} + \dots\) is of course designed to look very suggestively like a sum, but is not actually a finite sum because of the ``\(\dots\)'' symbol.
To rigorously define what the series actually sums to, we need another definition.
\end{note}

\begin{definition}[Convergence of series]\label{7.2.2}
Let \(\sum_{n = m}^\infty a_n\) be a formal infinite series.
For any integer \(N \geq m\), we define the \emph{\(N^{\text{th}}\) partial sum} \(S_N\) of this series to be \(S_N \coloneqq \sum_{n = m}^N a_n\);
of course, \(S_N\) is a real number.
If the sequence \((S_N)_{N = m}^\infty\) converges to some limit \(L\) as \(N \to \infty\), then we say that the infinite series \(\sum_{n = m}^\infty a_n\) is \emph{convergent}, and \emph{converges to \(L\)};
we also write \(L = \sum_{n = m}^\infty a_n\), and say that \(L\) is the \emph{sum} of the infinite series \(\sum_{n = m}^\infty a_n\).
If the partial sums \(S_N\) diverge, then we say that the infinite series \(\sum_{n = m}^\infty a_n\) is \emph{divergent}, and we do not assign any real number value to that series.
\end{definition}

\begin{remark}\label{7.2.3}
Note that Proposition \ref{6.1.7} shows that if a series converges, then it has a unique sum, so it is safe to talk about \emph{the} sum \(L = \sum_{n = m}^\infty a_n\) of a convergent series.
\end{remark}

\setcounter{theorem}{4}
\begin{proposition}\label{7.2.5}
Let \(\sum_{n = m}^\infty a_n\) be a formal series of real numbers.
Then \(\sum_{n = m}^\infty a_n\) converges if and only if, for every real number \(\varepsilon > 0\), there exists an integer \(N \geq m\) such that
\[
    \abs*{\sum_{n = p + 1}^q a_n} \leq \varepsilon \text{ for all } p, q \geq N.
\]
\end{proposition}

\begin{proof}
We first show that if \(\sum_{n = m}^\infty a_n\) converges, then \(\forall\ \varepsilon \in \mathbf{R}\) and \(\varepsilon > 0\), \(\exists\ N \in \mathbf{N}\) and \(N \geq m\) such that \(\abs*{\sum_{n = p + 1}^q a_n} \leq \varepsilon \ \forall\ p, q \in \mathbf{N}\) and \(p, q \geq N\).
Let \(k \in \mathbf{N}\) and let \(S_k = \sum_{n = m}^k a_n\) be the \(k^{\text{th}}\) partial sum of \((a_n)_{n = m}^\infty\).
Since \((S_k)_{k = m}^\infty\) converges, by Theorem \ref{6.4.18} \((S_k)_{k = m}^\infty\) is a Cauchy sequence.
So we have \(\forall\ \varepsilon > 0\), \(\exists\ N \geq m\) such that \(\abs*{S_q - S_p} \leq \varepsilon \ \forall\ p, q \geq N\).
We now divide into two cases:
\begin{enumerate}
    \item If \(p > q\), then \(p + 1 > q\).
    By Definition \ref{7.1.1} \(\abs*{\sum_{n = p + 1}^q a_n} = \abs*{0} = 0 \leq \varepsilon\).
    \item If \(p \leq q\), then
    \begin{align*}
    & \abs*{S_q - S_p} \leq \varepsilon \\
    \implies & \abs*{\bigg(\sum_{n = m}^q a_n\bigg) - \bigg(\sum_{n = m}^p a_n\bigg)} \leq \varepsilon \\
    \implies & \abs*{\bigg(\sum_{n = m}^p a_n\bigg) + \bigg(\sum_{n = p + 1}^q a_n\bigg) - \bigg(\sum_{n = m}^p a_n\bigg)} \leq \varepsilon & \text{(by Lemma \ref{7.1.4})} \\
    \implies & \abs*{\sum_{n = p + 1}^q a_n} \leq \varepsilon.
    \end{align*}
\end{enumerate}
From all cases above we have \(\abs*{\sum_{n = p + 1}^q a_n} \leq \varepsilon \ \forall\ p, q \geq N\).
Thus we conclude that if \(\sum_{n = m}^\infty a_n\) converges, then \(\forall\ \varepsilon \in \mathbf{R}\) and \(\varepsilon > 0\), \(\exists\ N \in \mathbf{N}\) and \(N \geq m\) such that \(\abs*{\sum_{n = p + 1}^q a_n} \leq \varepsilon \ \forall\ p, q \in \mathbf{N}\) and \(p, q \geq N\).

Now we show that if \(\forall\ \varepsilon \in \mathbf{R}\) and \(\varepsilon > 0\), \(\exists\ N \in \mathbf{N}\) and \(N \geq m\) such that \(\abs*{\sum_{n = p + 1}^q a_n \leq \varepsilon} \ \forall\ p, q \in \mathbf{N}\) and \(p, q \geq N\), then \(\sum_{n = m}^\infty a_n\) converges.
Since
\[
    \abs*{\sum_{n = p + 1}^q a_n} \leq \varepsilon \ \forall\ p, q \geq N,
\]
we can choose some \(p < q\) such that
\begin{align*}
& \abs*{\sum_{n = p + 1}^q a_n} \leq \varepsilon \\
\implies & \abs*{\sum_{n = p + 1}^q a_n + \sum_{n = m}^p a_n - \sum_{n = m}^p a_n} \leq \varepsilon \\
\implies & \abs*{\sum_{n = m}^q a_n - \sum_{n = m}^p a_n} \leq \varepsilon & \text{(by Lemma \ref{7.1.4})} \\
\implies & \abs*{S_q - S_p} \leq \varepsilon.
\end{align*}
This means \((S_k)_{k = m}^\infty\) is a Cauchy Sequence, so by Theorem \ref{6.4.18} \((S_k)_{k = m}^\infty\) converges, and therefore \(\sum_{n = m}^\infty a_n\) converges.
Thus we conclude that \(\sum_{n = m}^\infty a_n\) converges if and only if \(\forall\ \varepsilon \in \mathbf{R}\) and \(\varepsilon > 0\), \(\exists\ N \in \mathbf{N}\) and \(N \geq m\) such that \(\abs*{\sum_{n = p + 1}^q a_n} \leq \varepsilon \ \forall\ p, q \in \mathbf{N}\) and \(p, q \geq N\).
\end{proof}
