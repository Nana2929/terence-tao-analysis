\section{Infinite series}\label{sec 7.2}

\begin{definition}[Formal infinite series]\label{7.2.1}
A (formal) infinite series is any expression of the form
\[
    \sum_{n = m}^\infty a_n,
\]
where \(m\) is an integer, and \(a_n\) is a real number for any integer \(n \geq m\).
\end{definition}

\begin{note}
We sometimes write this series as
\[
    a_m + a_{m + 1} + a_{m + 2} + \dots.
\]
\end{note}

\begin{note}
At present, this series is only defined formally;
we have not set this sum equal to any real number;
the notation \(a_m + a_{m + 1} + a_{m + 2} + \dots\) is of course designed to look very suggestively like a sum, but is not actually a finite sum because of the ``\(\dots\)'' symbol.
To rigorously define what the series actually sums to, we need another definition.
\end{note}

\begin{definition}[Convergence of series]\label{7.2.2}
Let \(\sum_{n = m}^\infty a_n\) be a formal infinite series.
For any integer \(N \geq m\), we define the \emph{\(N^{\text{th}}\) partial sum} \(S_N\) of this series to be \(S_N \coloneqq \sum_{n = m}^N a_n\);
of course, \(S_N\) is a real number.
If the sequence \((S_N)_{n = m}^\infty\) converges to some limit \(L\) as \(N \to \infty\), then we say that the infinite series \(\sum_{n = m}^\infty a_n\) is \emph{convergent}, and \emph{converges to \(L\)};
we also write \(L = \sum_{n = m}^\infty a_n\), and say that \(L\) is the \emph{sum} of the infinite series \(\sum_{n = m}^\infty a_n\).
If the partial sums \(S_N\) diverge, then we say that the infinite series \(\sum_{n = m}^\infty a_n\) is \emph{divergent}, and we do not assign any real number value to that series.
\end{definition}

\begin{remark}\label{7.2.3}
Note that Proposition \ref{6.1.7} shows that if a series converges, then it has a unique sum, so it is safe to talk about \emph{the} sum \(L = \sum_{n = m}^\infty a_n\) of a convergent series.
\end{remark}