\section{Infinite series}\label{sec 7.2}

\begin{definition}[Formal infinite series]\label{7.2.1}
    A (formal) infinite series is any expression of the form
    \[
        \sum_{n = m}^\infty a_n,
    \]
    where \(m\) is an integer, and \(a_n\) is a real number for any integer \(n \geq m\).
\end{definition}

\begin{note}
    We sometimes write this series as
    \[
        a_m + a_{m + 1} + a_{m + 2} + \dots.
    \]
\end{note}

\begin{note}
    At present, this series is only defined \emph{formally};
    we have not set this sum equal to any real number;
    the notation \(a_m + a_{m + 1} + a_{m + 2} + \dots\) is of course designed to look very suggestively like a sum, but is not actually a finite sum because of the ``\(\dots\)'' symbol.
    To rigorously define what the series actually sums to, we need another definition.
\end{note}

\begin{definition}[Convergence of series]\label{7.2.2}
    Let \(\sum_{n = m}^\infty a_n\) be a formal infinite series.
    For any integer \(N \geq m\), we define the \emph{\(N^{\text{th}}\) partial sum} \(S_N\) of this series to be \(S_N \coloneqq \sum_{n = m}^N a_n\);
    of course, \(S_N\) is a real number.
    If the sequence \((S_N)_{N = m}^\infty\) converges to some limit \(L\) as \(N \to \infty\), then we say that the infinite series \(\sum_{n = m}^\infty a_n\) is \emph{convergent}, and \emph{converges to \(L\)};
    we also write \(L = \sum_{n = m}^\infty a_n\), and say that \(L\) is the \emph{sum} of the infinite series \(\sum_{n = m}^\infty a_n\).
    If the partial sums \(S_N\) diverge, then we say that the infinite series \(\sum_{n = m}^\infty a_n\) is \emph{divergent}, and we do not assign any real number value to that series.
\end{definition}

\begin{remark}\label{7.2.3}
    Note that Proposition \ref{6.1.7} shows that if a series converges, then it has a unique sum, so it is safe to talk about \emph{the} sum \(L = \sum_{n = m}^\infty a_n\) of a convergent series.
\end{remark}

\setcounter{theorem}{4}
\begin{proposition}\label{7.2.5}
    Let \(\sum_{n = m}^\infty a_n\) be a formal series of real numbers.
    Then \(\sum_{n = m}^\infty a_n\) converges if and only if, for every real number \(\varepsilon > 0\), there exists an integer \(N \geq m\) such that
    \[
        \abs*{\sum_{n = p + 1}^q a_n} \leq \varepsilon \text{ for all } p, q \geq N.
    \]
\end{proposition}

\begin{proof}
    Let \(N, p, q \in \mathbf{N}\).
    We first show that if \(\sum_{n = m}^\infty a_n\) converges, then \(\forall\ \varepsilon \in \mathbf{R}^+\), \(\exists\ N \geq m\) such that \(\abs*{\sum_{n = p + 1}^q a_n} \leq \varepsilon\) for every \(p, q \geq N\).
    Let \(k \in \mathbf{N}\) and let \(S_k = \sum_{n = m}^k a_n\) be the \(k^{\text{th}}\) partial sum of \((a_n)_{n = m}^\infty\).
    Since \((S_k)_{k = m}^\infty\) converges, by Theorem \ref{6.4.18} \((S_k)_{k = m}^\infty\) is a Cauchy sequence.
    Then we have \(\forall\ \varepsilon \in \mathbf{R}^+\), \(\exists\ N \geq m\) such that \(\abs*{S_q - S_p} \leq \varepsilon\) for every \(p, q \geq N\).
    We now split into two cases:
    \begin{itemize}
        \item If \(p \geq q\), then \(p + 1 > q\).
              By Definition \ref{7.1.1} we have \(\abs*{\sum_{n = p + 1}^q a_n} = \abs*{0} = 0 \leq \varepsilon\).
        \item If \(p < q\), then
              \begin{align*}
                           & \abs*{S_q - S_p} \leq \varepsilon                                                                                                                               \\
                  \implies & \abs*{\bigg(\sum_{n = m}^q a_n\bigg) - \bigg(\sum_{n = m}^p a_n\bigg)} \leq \varepsilon                                                                         \\
                  \implies & \abs*{\bigg(\sum_{n = m}^p a_n\bigg) + \bigg(\sum_{n = p + 1}^q a_n\bigg) - \bigg(\sum_{n = m}^p a_n\bigg)} \leq \varepsilon & \text{(by Lemma \ref{7.1.4}(a))} \\
                  \implies & \abs*{\sum_{n = p + 1}^q a_n} \leq \varepsilon.
              \end{align*}
    \end{itemize}
    From all cases above we conclude that \(\abs*{\sum_{n = p + 1}^q a_n} \leq \varepsilon\) for every \(p, q \geq N\).

    Now we show that if \(\forall\ \varepsilon \in \mathbf{R}^+\), \(\exists\ N \geq m\) such that \(\abs*{\sum_{n = p + 1}^q a_n} \leq \varepsilon\) for every \(p, q \geq N\), then \(\sum_{n = m}^\infty a_n\) converges.
    Since \(\abs*{\sum_{n = p + 1}^q a_n} \leq \varepsilon\) for every \(p, q \geq N\), we can choose some \(p \leq q\) such that
    \begin{align*}
                 & \abs*{\sum_{n = p + 1}^q a_n} \leq \varepsilon                                                                           \\
        \implies & \abs*{\sum_{n = p + 1}^q a_n + \sum_{n = m}^p a_n - \sum_{n = m}^p a_n} \leq \varepsilon                                 \\
        \implies & \abs*{\sum_{n = m}^q a_n - \sum_{n = m}^p a_n} \leq \varepsilon                          & \text{(by Lemma \ref{7.1.4})} \\
        \implies & \abs*{S_q - S_p} \leq \varepsilon.
    \end{align*}
    This means \((S_k)_{k = m}^\infty\) is a Cauchy Sequence, so by Theorem \ref{6.4.18} \((S_k)_{k = m}^\infty\) converges, and by Definition \ref{7.2.2} \(\sum_{n = m}^\infty a_n\) converges.
    We conclude that \(\sum_{n = m}^\infty a_n\) converges iff \(\forall\ \varepsilon \in \mathbf{R}^+\), \(\exists\ N \geq m\) such that \(\abs*{\sum_{n = p + 1}^q a_n} \leq \varepsilon\) for every \(p, q \geq N\).
\end{proof}

\begin{corollary}[Zero test]\label{7.2.6}
    Let \(\sum_{n = m}^\infty a_n\) be a convergent series of real numbers.
    Then we must have \(\lim_{n \to \infty} a_n = 0\).
    To put this another way, if \(\lim_{n \to \infty} a_n\) is non-zero or divergent, then the series \(\sum_{n = m}^\infty a_n\) is divergent.
\end{corollary}

\begin{proof}
    Let \(N, p, q \in \mathbf{N}\).
    Then we have
    \begin{align*}
                 & \sum_{n = m}^\infty a_n \text{ converges}                                                                                                                                \\
        \implies & \forall\ \varepsilon \in \mathbf{R}^+, \exists\ N \geq m : \forall\ p, q \geq N, \abs*{\sum_{n = p + 1}^q a_n} \leq \varepsilon    & \text{(by Proposition \ref{7.2.5})} \\
        \implies & \forall\ \varepsilon \in \mathbf{R}^+, \exists\ N \geq m : \forall\ p \geq N, \abs*{\sum_{n = p + 1}^{p + 1} a_n} \leq \varepsilon                                       \\
        \implies & \forall\ \varepsilon \in \mathbf{R}^+, \exists\ N \geq m : \forall\ p \geq N, \abs*{a_{p + 1}} \leq \varepsilon                    & \text{(by Definition \ref{7.1.1})}  \\
        \implies & \forall\ \varepsilon \in \mathbf{R}^+, \exists\ N \geq m : \forall\ p \geq N, \abs*{a_{p + 1} - 0} \leq \varepsilon                                                      \\
        \implies & \lim_{n \to \infty} a_n = 0.                                                                                                       & \text{(by Definition \ref{6.1.8})}
    \end{align*}
\end{proof}

\begin{note}
    If a sequence \((a_n)_{n = m}^\infty\) \emph{does} converge to zero, then the series \(\sum_{n = m}^\infty a_n\) may or may not be convergent;
    it depends on the series.
\end{note}

\setcounter{theorem}{7}
\begin{definition}[Absolute convergence]\label{7.2.8}
    Let \(\sum_{n = m}^\infty a_n\) be a formal series of real numbers.
    We say that this series is \emph{absolutely convergent} iff the series \(\sum_{n = m}^\infty \abs*{a_n}\) is convergent.
\end{definition}

\begin{note}
    In order to distinguish convergence from absolute convergence, we sometimes refer to the former as \emph{conditional convergence}.
\end{note}

\begin{proposition}[Absolute convergence test]\label{7.2.9}
    Let \(\sum_{n = m}^\infty a_n\) be a formal series of real numbers.
    If this series is absolutely convergent, then it is also conditionally convergent.
    Furthermore, in this case we have the triangle inequality
    \[
        \abs*{\sum_{n = m}^\infty a_n} \leq \sum_{n = m}^\infty \abs*{a_n}.
    \]
\end{proposition}

\begin{proof}
    We first show that if \(\sum_{n = m}^\infty a_n\) is absolutely convergent, then it is also conditionally convergent.
    \begin{align*}
                 & \sum_{n = m}^\infty \abs*{a_n} \text{ converge}                                                                                               \\
        \implies & \forall\ \varepsilon \in \mathbf{R} \land \varepsilon > 0, \exists\ N \in \mathbf{N} \land N \geq m :                                         \\
                 & \abs*{\sum_{n = p + 1}^q \abs*{a_n}} \leq \varepsilon\ \forall\ p, q \in \mathbf{N} \land p, q \geq N   & \text{(by Proposition \ref{7.2.5})} \\
        \implies & \forall\ \varepsilon > 0, \ \exists\ N \geq m :                                                                                               \\
                 & \sum_{n = p + 1}^q \abs*{a_n} \leq \varepsilon\ \forall\ p, q \geq N                                                                          \\
        \implies & \forall\ \varepsilon > 0, \ \exists\ N \geq m :                                                                                               \\
                 & \abs*{\sum_{n = p + 1}^q a_n} \leq \sum_{n = p + 1}^q \abs*{a_n} \leq \varepsilon\ \forall\ p, q \geq N & \text{(by Lemma \ref{7.1.4})}       \\
        \implies & \sum_{n = m}^\infty a_n \text{ converge}.                                                               & \text{(by Proposition \ref{7.2.5})}
    \end{align*}

    Now we show that the triangle inequality is true.
    Let \(N \in \mathbf{N} \land N \geq m\).
    Let \((S_N)_{N = m}^\infty\) be a sequence where \(S_N = \abs*{\sum_{n = m}^N a_n}\).
    Let \((T_N)_{N = m}^\infty\) be a sequence where \(T_N = \sum_{n = m}^N \abs*{a_n}\).
    Since \(\lim_{N \to \infty} T_N\) exists, from proof above we know that \(\lim_{N \to \infty} S_N\) also exists.
    So
    \begin{align*}
                 & \abs*{\sum_{n = m}^N a_n} \leq \sum_{n = m}^N \abs*{a_n} \ \forall\ N \geq m & \text{(by Lemma \ref{7.1.4})}      \\
        \implies & S_N \leq T_N \ \forall\ N \geq m                                                                                  \\
        \implies & \lim_{N \to \infty} S_N \leq \lim_{N \to \infty} T_N                         & \text{(by Lemma \ref{6.4.13})}     \\
        \implies & \abs*{\sum_{n = m}^\infty a_n} \leq \sum_{n = m}^\infty \abs*{a_n}.          & \text{(by Definition \ref{7.2.2})}
    \end{align*}
\end{proof}

\begin{remark}\label{7.2.10}
    The converse to this proposition is not true;
    there exist series which are conditionally convergent but not absolutely convergent.
\end{remark}

\begin{remark}\label{7.2.11}
    We consider the class of conditionally convergent series to include the class of absolutely convergent series as a subclass.
    Thus when we say a statement such as ``\(\sum_{n = m}^\infty a_n\) is conditionally convergent'', this does not automatically mean that \(\sum_{n = m}^\infty a_n\) is not absolutely convergent.
    If we wish to say that a series is conditionally convergent but not absolutely convergent, then we will instead use a phrasing such as ``\(\sum_{n = m}^\infty a_n\) is \emph{only} conditionally convergent'', or ``\(\sum_{n = m}^\infty a_n\) converges conditionally, but not absolutely''.
\end{remark}

\begin{proposition}[Alternating series test]\label{7.2.12}
    Let \((a_n)_{n = m}^\infty\) be a sequence of real numbers which are non-negative and decreasing, thus \(a_n \geq 0\) and \(a_n \geq a_{n + 1}\) for every \(n \geq m\).
    Then the series \(\sum_{n = m}^\infty (-1)^n a_n\) is convergent if and only if the sequence \(a_n\) converges to \(0\) as \(n \to \infty\).
\end{proposition}

\begin{proof}
    From the zero test (Corollary \ref{7.2.6}), we know that if \(\sum_{n = m}^\infty (-1)^n a_n\) is a convergent series, then the sequence \(((-1)^n a_n)_{n = m}^\infty\) converges to \(0\), which implies that \(a_n\) also converges to \(0\), since \((-1)^n a_n\) and \(a_n\) have the same distance from \(0\).
    \begin{align*}
         & \forall\ \varepsilon \in \mathbf{R} \land \varepsilon > 0, \exists\ N \in \mathbf{N} \land N \geq m :        \\
         & \abs*{a_n - 0} = \abs*{a_n} = \abs*{(-1)^n a_n} = \abs*{(-1)^n a_n - 0} \leq \varepsilon \ \forall\ n \geq N
    \end{align*}

    Now suppose conversely that \(a_n\) converges to \(0\).
    For each \(N\), let \(S_N\) be the partial sum \(S_N \coloneqq \sum_{n = m}^N (-1)^n a_n\);
    our job is to show that \(S_N\) converges.
    Observe that
    \begin{align*}
        S_{N + 2} & = S_N + (-1)^{N + 1} a_{N + 1} + (-1)^{N + 2} a_{N + 2} \\
                  & = S_N + (-1)^{N + 1} (a_{N + 1} - a_{N + 2}).
    \end{align*}
    But by hypothesis, \((a_{N + 1} - a_{N + 2})\) is non-negative.
    Thus we have \(S_{N + 2} \geq S_N\) when \(N\) is odd and \(S_{N + 2} \leq S_N\) if \(N\) is even.

    Now suppose that \(N\) is even.
    From the above discussion and induction we see that \(S_{N + 2k} \leq S_N\) for all natural numbers \(k\).
    Also we have \(S_{N + 2k + 1} \geq S_{N + 1} = S_N - a_{N + 1}\).
    Finally, we have \(S_{N + 2k + 1} = S_{N + 2k} - a_{N + 2k + 1} \leq S_{N + 2k}\).
    Thus we have
    \[
        S_N - a_{N + 1} \leq S_{N + 2k + 1} \leq S_{N + 2k} \leq S_N
    \]
    for all \(k\).
    In particular, we have
    \[
        S_N - a_{N + 1} \leq S_n \leq S_N \ \forall\ n \geq N.
    \]
    In particular, the sequence \((S_n)_{n = m}^\infty\) is eventually \(a_{N + 1}\)-steady
    \[
        S_p - S_n \leq S_N - S_n \leq a_{N + 1} \ \forall\ p, n \geq N.
    \]
    But the sequence \((a_N)_{N = m}^\infty\) converges to \(0\) as \(N \to \infty\), thus this implies that \((S_n)_{n = m}^\infty\) is eventually \(\varepsilon\)-steady for every \(\varepsilon > 0\).
    Thus \((S_n)_{n = m}^\infty\) converges, and so the series \(\sum_{n = m}^\infty (-1)^n a_n\) is convergent.
\end{proof}

\begin{note}
    Lack of absolute convergence does not imply lack of conditional convergence, even though absolute convergence implies conditional convergence.
\end{note}

\setcounter{theorem}{13}
\begin{proposition}[Series law]\label{7.2.14}
    \mbox{}
    \begin{enumerate}
        \item If \(\sum_{n = m}^\infty a_n\) is a series of real numbers converging to \(x\), and \(\sum_{n = m}^\infty b_n\) is a series of real numbers converging to \(y\), then \(\sum_{n = m}^\infty (a_n + b_n)\) is also a convergent series, and converges to \(x + y\).
              In particular, we have
              \[
                  \sum_{n = m}^\infty (a_n + b_n) = \sum_{n = m}^\infty a_n + \sum_{n = m}^\infty b_n.
              \]
        \item If \(\sum_{n = m}^\infty a_n\) is a series of real numbers converging to \(x\), and \(c\) is a real number, then \(\sum_{n = m}^\infty (c a_n)\) is also a convergent series, and converges to \(cx\).
              In particular, we have
              \[
                  \sum_{n = m}^\infty (c a_n) = c \sum_{n = m}^\infty a_n.
              \]
        \item Let \(\sum_{n = m}^\infty a_n\) be a series of real numbers, and let \(k \geq 0\) be an integer.
              If one of the two series \(\sum_{n = m}^\infty a_n\) and \(\sum_{n = m + k}^\infty a_n\) are convergent, then the other one is also, and we have the identity
              \[
                  \sum_{n = m}^\infty a_n = \sum_{n = m}^{m + k - 1} a_n + \sum_{n = m + k}^\infty a_n.
              \]
        \item Let \(\sum_{n = m}^\infty a_n\) be a series of real numbers converging to \(x\), and let \(k\) be an integer.
              Then \(\sum_{n = m + k}^\infty a_{n - k}\) also converges to \(x\).
    \end{enumerate}
\end{proposition}

\begin{proof}{(a)}
    Let \(A_N = \sum_{n = m}^N a_n\) be the \(N^{\text{th}}\) partial sum of \(x\), and let \(B_N = \sum_{n = m}^N b_n\) be the \(N^{\text{th}}\) partial sum of \(y\).
    So
    \begin{align*}
        x + y & = \sum_{n = m}^\infty a_n + \sum_{n = m}^\infty b_n                                      \\
              & = \lim_{N \to \infty} A_N + \lim_{N \to \infty} B_N & \text{(by Definition \ref{7.2.2})} \\
              & = \lim_{N \to \infty} (A_N + B_N)                   & \text{(by Theorem \ref{6.1.19})}   \\
              & = \sum_{n = m}^\infty (a_n + b_n).                  & \text{(by Definition \ref{7.2.2})}
    \end{align*}
\end{proof}

\begin{proof}{(b)}
    Let \(S_N = \sum_{n = m}^N a_n\) be the \(N^{\text{th}}\) partial sum of \(x\).
    So
    \begin{align*}
        cx & = c \sum_{n = m}^\infty a_n                                         \\
           & = c \lim_{N \to \infty} S_N    & \text{(by Definition \ref{7.2.2})} \\
           & = \lim_{N \to \infty} (c S_N)  & \text{(by Theorem \ref{6.1.19})}   \\
           & = \sum_{n = m}^\infty (c a_n). & \text{(by Definition \ref{7.2.2})}
    \end{align*}
\end{proof}

\begin{proof}{(c)}
    We first show that \(\sum_{n = m}^\infty a_n\) converges if and only if \(\sum_{n = m + k}^\infty a_n\) converges.
    \begin{align*}
             & \sum_{n = m}^\infty a_n \text{ converges}                                                                                                       \\
        \iff & \forall\ \varepsilon \in \mathbf{R} \land \varepsilon > 0, \exists\ N \in \mathbf{N} \land N \geq m :                                           \\
             & \abs*{\sum_{n = p + 1}^q a_n} \leq \varepsilon \ \forall\ p, q \in \mathbf{N} \land p, q \geq N           & \text{(by Proposition \ref{7.2.5})} \\
        \iff & \forall\ \varepsilon \in \mathbf{R} \land \varepsilon > 0, \exists\ N \in \mathbf{N} \land N \geq m :                                           \\
             & \abs*{\sum_{n = p + k + 1}^q a_n} \leq \varepsilon \ \forall\ p, q \in \mathbf{N} \land p, q \geq N                                             \\
        \iff & \forall\ \varepsilon \in \mathbf{R} \land \varepsilon > 0, \exists\ N \in \mathbf{N} \land N \geq m + k :                                       \\
             & \abs*{\sum_{n = p + 1}^q a_n} \leq \varepsilon \ \forall\ p, q \in \mathbf{N} \land p, q \geq N                                                 \\
        \iff & \sum_{n = m + k}^\infty a_n \text{ converges}.                                                            & \text{(by Proposition \ref{7.2.5})}
    \end{align*}

    Now we show that \(\sum_{n = m}^\infty a_n = \sum_{n = m}^{n + k - 1} a_n + \sum_{n = m + k}^\infty a_n\).
    Let \(A_N = \sum_{n = m}^N a_n\) be the \(N^{\text{th}}\) partial sum of \(\sum_{n = m}^\infty a_n\), and let \(B_N = \sum_{n = m + k}^N a_n\) be the \(N^{\text{th}}\) partial sum of \(\sum_{n = m + k}^\infty a_n\).
    So
    \begin{align*}
        \sum_{n = m}^{m + k - 1} a_n + \sum_{n = m + k}^\infty a_n & = A_{m + k - 1} + \sum_{n = m + k}^\infty a_n                                                                      \\
                                                                   & = A_{m + k - 1} + \lim_{N \to \infty} B_N                                     & \text{(by Definition \ref{7.2.2})} \\
                                                                   & = \lim_{N \to \infty} A_{m + k - 1} + \lim_{N \to \infty} B_N                                                      \\
                                                                   & = \lim_{N \to \infty} (A_{m + k - 1} + B_N)                                   & \text{(by Theorem \ref{6.1.19})}   \\
                                                                   & = \lim_{N \to \infty} (\sum_{n = m}^{m + k - 1} a_n + \sum_{n = m + k}^N a_n)                                      \\
                                                                   & = \lim_{N \to \infty} \sum_{n = m}^N a_n = \lim_{N \to \infty} A_N            & \text{(by Lemma \ref{7.1.4})}      \\
                                                                   & = \sum_{n = m}^\infty a_n.                                                    & \text{(by Definition \ref{7.2.2})}
    \end{align*}
\end{proof}

\begin{proof}{(d)}
    Let \(A_N = \sum_{n = m}^N a_n\) be the \(N^{\text{th}}\) partial sum of \(x\), and let \(B_N = \sum_{n = m + k}^N a_{n - k}\) be the \(N^{\text{th}}\) partial sum of \(\sum_{n = m + k}^\infty a_{n - k}\).
    So
    \begin{align*}
        x & = \sum_{n = m}^\infty a_n = \lim_{N \to \infty} A_N = \lim_{N \to \infty} \sum_{n = m}^N a_n & \text{(by Definition \ref{7.2.2})} \\
          & = \lim_{N' \to \infty} \sum_{n = m}^{N' - k} a_n                                             & (N' = N + k)                       \\
          & = \lim_{N' \to \infty} \sum_{n = m + k}^{N' - k + k} a_{n - k}                               & \text{(by Lemma \ref{7.1.4})}      \\
          & = \lim_{N' \to \infty} \sum_{n = m + k}^{N'} a_{n - k} = \lim_{N' \to \infty} B_{N'}                                              \\
          & = \sum_{n = m + k}^\infty a_{n - k}.                                                         & \text{(by Definition \ref{7.2.2})}
    \end{align*}
\end{proof}

\begin{note}
    From Proposition \ref{7.2.14}(c) we see that the convergence of a series does not depend on the first few elements of the series
    (though of course those elements do influence which value the series converges to).
    Because of this, we will usually not pay much attention as to what the initial index \(m\) of the series is.
\end{note}

\begin{lemma}[Telescoping series]\label{7.2.15}
    Let \((a_n)_{n = 0}^\infty\) be a sequence of real numbers which converge to \(0\), i.e., \(\lim_{n \to \infty} a_n = 0\).
    Then the series \(\sum_{n = 0}^\infty (a_n - a_{n + 1})\) converges to \(a_0\).
    If \(\lim_{n \to \infty} a_n = L\), then the series \(\sum_{n = 0}^\infty (a_n - a_{n + 1})\) converges to \(a_0 + L\).
\end{lemma}

\begin{proof}
    Let \(S_N = \sum_{n = 0}^N (a_n - a_{n + 1})\) be the \(N^{\text{th}}\) partial sum of \(\sum_{n = m}^\infty (a_n - a_{n + 1})\).
    We first show that \(S_N = a_0 - a_{N + 1}\) by using induction on \(N\).
    For \(N = 0\), we have
    \[
        S_0 = \sum_{n = 0}^0 a_n - a_{n + 1} = a_0 - a_1
    \]
    by Definition \ref{7.1.1}, so the base case holds.
    Suppose inductively that for some \(N \geq 0\) the statement holds.
    Then for \(N + 1\), we have
    \begin{align*}
        S_{N + 1} & = \sum_{n = 0}^{N + 1} (a_n - a_{n + 1})                                                        \\
                  & = \sum_{n = 0}^N (a_n - a_{n + 1}) + a_{N + 1} - a_{N + 2} & \text{(by Definition \ref{7.1.1})} \\
                  & = a_0 - a_{N + 1} + a_{N + 1} - a_{N + 2}                  & \text{(by induction hypothesis)}   \\
                  & = a_0 - a_{N + 2}.
    \end{align*}
    This close the induction.

    Now we show that if \(\lim_{n \to \infty} a_n = 0\), then \(\sum_{n = 0}^\infty (a_n - a_{n + 1})\) converges to \(a_0\).
    \begin{align*}
        \sum_{n = 0}^\infty (a_n - a_{n + 1}) & = \lim_{N \to \infty} S_N       & \text{(by Definition \ref{7.2.2})} \\
                                              & = \lim_{N \to \infty} a_0 + a_N & \text{(from claim above)}          \\
                                              & = a_0 + \lim_{N \to \infty} a_N & \text{(by Theorem \ref{6.1.19})}   \\
                                              & = a_0 + 0                       & \text{(by the given condition)}    \\
                                              & = a_0.
    \end{align*}

    Finally we show that if \(\lim_{n \to \infty} a_n = L\), then \(\sum_{n = 0}^\infty (a_n - a_{n + 1})\) converges to \(a_0 + L\).
    \begin{align*}
        \sum_{n = 0}^\infty (a_n - a_{n + 1}) & = \lim_{N \to \infty} S_N       & \text{(by Definition \ref{7.2.2})} \\
                                              & = \lim_{N \to \infty} a_0 + a_N & \text{(from claim above)}          \\
                                              & = a_0 + \lim_{N \to \infty} a_N & \text{(by Theorem \ref{6.1.19})}   \\
                                              & = a_0 + L.                      & \text{(by the given condition)}
    \end{align*}
\end{proof}

\exercisesection

\begin{exercise}\label{ex 7.2.1}
    Is the series \(\sum_{n = 1}^\infty (-1)^n\) convergent or divergent?
    Justify your answer.
\end{exercise}

\begin{proof}
    Let \((a_n)_{n = 1}^\infty\) be a sequence where \(a_n = 1 \ \forall\ n \geq 1\).
    Since \(\lim_{n \to \infty} a_n = 1\), by Alternating series test (Proposition \ref{7.2.12}) \(\sum_{n = m}^\infty (-1)^n a_n\) diverges.
\end{proof}

\begin{exercise}\label{ex 7.2.2}
    Prove Proposition \ref{7.2.5}.
\end{exercise}

\begin{proof}
    See Proposition \ref{7.2.5}.
\end{proof}

\begin{exercise}\label{ex 7.2.3}
    Use Proposition \ref{7.2.5} to prove Corollary \ref{7.2.6}.
\end{exercise}

\begin{proof}
    See Corollary \ref{7.2.6}.
\end{proof}

\begin{exercise}\label{ex 7.2.4}
    Prove Proposition \ref{7.2.9}.
\end{exercise}

\begin{proof}
    See Proposition \ref{7.2.9}.
\end{proof}

\begin{exercise}\label{ex 7.2.5}
    Prove Proposition \ref{7.2.14}.
\end{exercise}

\begin{proof}
    See Proposition \ref{7.2.14}.
\end{proof}

\begin{exercise}\label{ex 7.2.6}
    Prove Lemma \ref{7.2.15}.
    How does the proposition change if we assume that an does not converge to zero, but instead converges to some other real number \(L\)?
\end{exercise}

\begin{proof}
    See Lemma \ref{7.2.15}.
\end{proof}