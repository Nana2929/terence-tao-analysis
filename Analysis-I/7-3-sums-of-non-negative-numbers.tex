\section{Sums of non-negative numbers}\label{sec 7.3}

\begin{note}
When all the terms in a series are non-negative, there is no distinction between conditional convergence and absolute convergence.
\end{note}

\begin{proposition}\label{7.3.1}
Let \(\sum_{n = m}^\infty a_n\) be a formal series of non-negative real numbers.
Then this series is convergent if and only if there is a real number \(M\) such that
\[
    \sum_{n = m}^N a_n \leq M \text{ for all integers } N \geq m.
\]
\end{proposition}

\begin{proof}
Suppose \(\sum_{n = m}^\infty a_n\) is a series of non-negative numbers.
Then the partial sums \(S_N \coloneqq \sum_{n = m}^N a_n\) are increasing, i.e., \(S_{N + 1} \geq S_N\) for all \(N \geq m\).
From Proposition \ref{6.3.8} and Corollary \ref{6.1.17}, we thus see that the sequence \((S_N)_{n = m}^\infty\) is convergent if and only if it has an upper bound \(M\).
\end{proof}

\begin{corollary}[Comparison test]\label{7.3.2}
Let \(\sum_{n = m}^\infty a_n\) and \(\sum_{n = m}^\infty b_n\) be two formal series of real numbers, and suppose that \(\abs*{a_n} \leq b_n\) for all \(n \geq m\).
Then if \(\sum_{n = m}^\infty b_n\) is convergent, then \(\sum_{n = m}^\infty a_n\) is absolutely convergent, and in fact
\[
    \abs*{\sum_{n = m}^\infty a_n} \leq \sum_{n = m}^\infty \abs*{a_n} \leq \sum_{n = m}^\infty b_n.
\]
\end{corollary}

\begin{proof}
\begin{align*}
& \sum_{n = m}^\infty b_n \text{ converges} \\
\implies & \exists\ M \in \mathbf{R} : \sum_{n = m}^N b_n \leq M \ \forall\ N \in \mathbf{N} \land N \geq m & \text{(by Proposition \ref{7.3.1})} \\
\implies & \exists\ M \in \mathbf{R} : \sum_{n = m}^N \abs*{a_n} \leq \sum_{n = m}^N b_n \leq M \ \forall\ N \geq m & \text{(by the given condition)} \\
\implies & \exists\ M \in \mathbf{R} : \abs*{\sum_{n = m}^N a_n} \leq \sum_{n = m}^N \abs*{a_n} \leq \sum_{n = m}^N b_n \leq M \ \forall\ N \geq m & \text{(by Lemma \ref{7.1.4})} \\
\implies & \abs*{\sum_{n = m}^\infty a_n} \text{ converges} \land \sum_{n = m}^\infty \abs*{a_n} \text{ converges} & \text{(by Proposition \ref{7.3.1})} \\
\implies & \abs*{\sum_{n = m}^\infty a_n} \leq \sum_{n = m}^\infty \abs*{a_n} \leq \sum_{n = m}^\infty b_n. & \text{(by Theorem \ref{6.1.19})}
\end{align*}
\end{proof}

\begin{note}
We can also run the comparison test in the contrapositive:
if we have \(\abs*{a_n} \leq b_n\) for all \(n \geq m\), and \(\sum_{n = m}^\infty a_n\) is not absolutely convergent, then \(\sum_{n = m}^\infty b_n\) is not conditionally convergent.
\end{note}

\begin{lemma}[Geometric series]\label{7.3.3}
Let \(x\) be a real number.
If \(\abs*{x} \geq 1\), then the series \(\sum_{n = 0}^\infty x^n\) is divergent.
If however \(\abs*{x} < 1\), then the series is absolutely convergent and
\[
    \sum_{n = 0}^\infty x^n = 1 / (1 - x).
\]
\end{lemma}

\begin{proof}
We first show that if \(\abs*{x} \geq 1\), then \(\sum_{n = 0}^\infty x^n\) is divergent.
We divide into two cases:
\begin{enumerate}
    \item If \(x = 1\), then \(\lim_{n \to \infty} x^n = 1\). By Zero test (Corollary \ref{7.2.6}) this means \(\sum_{n = 0}^\infty x^n\) is divergent.
    \item If \(x = -1\) or \(\abs*{x} > 1\), then by Lemma \ref{6.5.2} \(\lim_{n \to \infty} x^n\) diverges.
    By Zero test (Corollary \ref{7.2.6}) this means \(\sum_{n = 0}^\infty x^n\) is divergent.
\end{enumerate}
From all cases above we show that \(\sum_{n = 0}^\infty a_n\) is divergent.
Thus we conclude that if \(\abs*{x} \geq 1\), then \(\sum_{n = 0}^\infty x^n\) diverges.

Now we show that if \(\abs*{x} < 1\), then \(\sum_{n = 0}^\infty x^n\) is absolutely convergent and \(\sum_{n = 0}^\infty x^n = 1 / (1 - x)\).
We prove this by first show that \(\sum_{n = 0}^N x^n = (1 - x^{N + 1}) / (1 - x)\).
We use induction on \(N\).
For \(N = 0\),
\[
    \sum_{n = 0}^0 x^n = x^0 = 1 = (1 - x^1) / (1 - x).
\]
So the base case holds.
Suppose inductivly that for some \(N \geq 0\) the statement holds.
Then for \(N + 1\), we have
\begin{align*}
\sum_{n = 0}^{N + 1} x^n &= \sum_{n = 0}^N x^n + x^{N + 1} & \text{(by Definition \ref{7.1.1})} \\
&= \frac{1 - x^{N + 1}}{1 - x} + x^{N + 1} & \text{(by induction hypothesis)} \\
&= \frac{1 - x^{N + 1}}{1 - x} + \frac{(1 - x) x^{N + 1}}{1 - x} \\
&= \frac{1 - x^{N + 1}}{1 - x} + \frac{x^{N + 1} - x^{N + 2}}{1 - x} \\
&= \frac{1 - x^{N + 2}}{1 - x}.
\end{align*}
This close the induction.
Using similar argument we can show that
\[
    \sum_{n = 0}^N \abs*{x^n} = \frac{1 - \abs*{x^{N + 1}}}{1 - \abs*{x}}.
\]
Now we use the induction result to prove that \(\sum_{n = 0}^\infty x^n\) is absolutely convergent and \(\sum_{n = 0}^\infty x^n = 1 / (1 - x)\).
\begin{align*}
\sum_{n = 0}^\infty x^n &= \lim_{N \to \infty} \sum_{n = 0}^N x^n & \text{(by Definition \ref{7.2.2})} \\
&= \lim_{N \to \infty} \frac{1 - x^{N + 1}}{1 - x} & \text{(by the induction result)} \\
&= \frac{\lim_{N \to \infty} 1 - x^{N + 1}}{1 - x} & \text{(by Theorem \ref{6.1.19})} \\
&= \frac{(\lim_{N \to \infty} 1) - (\lim_{N \to \infty} x^{N + 1})}{1 - x} & \text{(by Theorem \ref{6.1.19})} \\
&= \frac{1 - (\lim_{N \to \infty} x^{N + 1})}{1 - x} \\
&= \frac{1 - 0}{1 - x} & \text{(by Lemma \ref{6.5.2})} \\
&= \frac{1}{1 - x}.
\end{align*}
Using similar argument we can show that \(\sum_{n = 0}^\infty \abs*{x^n} = 1 / (1 - \abs*{x})\).
Thus we conclude that if \(\abs*{x} < 1\), then \(\sum_{n = 0}^\infty x^n\) is absolutely convergent and \(\sum_{n = 0}^\infty x^n = 1 / (1 - x)\).
\end{proof}