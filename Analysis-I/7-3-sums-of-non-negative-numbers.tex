\section{Sums of non-negative numbers}\label{sec 7.3}

\begin{note}
When all the terms in a series are non-negative, there is no distinction between conditional convergence and absolute convergence.
\end{note}

\begin{proposition}\label{7.3.1}
Let \(\sum_{n = m}^\infty a_n\) be a formal series of non-negative real numbers.
Then this series is convergent if and only if there is a real number \(M\) such that
\[
    \sum_{n = m}^N a_n \leq M \text{ for all integers } N \geq m.
\]
\end{proposition}

\begin{proof}
Suppose \(\sum_{n = m}^\infty a_n\) is a series of non-negative numbers.
Then the partial sums \(S_N \coloneqq \sum_{n = m}^N a_n\) are are increasing, i.e., \(S_{N + 1} \geq S_N\) for all \(N \geq m\).
From Proposition \ref{6.3.8} and Corollary \ref{6.1.17}, we thus see that the sequence \((S_N)_{n = m}^\infty\) is convergent if and only if it has an upper bound \(M\).
\end{proof}
