\section{Summation on infinite sets}\label{sec 8.2}

\begin{definition}[Series on countable sets]\label{8.2.1}
    Let \(X\) be a countable set, and let \(f : X \to \mathbf{R}\) be a function.
    We say that the series \(\sum_{x \in X} f(x)\) is absolutely convergent iff for some bijection \(g : \mathbf{N} \to X\), the sum \(\sum_{n = 0}^\infty f(g(n))\) is absolutely convergent.
    We then define the sum of \(\sum_{x \in X} f(x)\) by the formula
    \[
        \sum_{x \in X} f(x) = \sum_{n = 0}^\infty f(g(n)).
    \]
\end{definition}

\begin{note}
    From Proposition \ref{7.4.3} (and Proposition \ref{3.6.4}), one can show that these definitions do not depend on the choice of \(g\), and so are well defined.
\end{note}

\begin{note}
    For finite sets \(X\) we adopt the convention that series \(\sum_{x \in X} f(x)\) are automatically considered to be absolutely convergent.
\end{note}

\begin{theorem}[Fubini's theorem for infinite sums]\label{8.2.2}
    Let \(f : \mathbf{N} \times \mathbf{N} \to \mathbf{R}\) be a function such that \(\sum_{(n, m) \in \mathbf{N} \times \mathbf{N}} f(n, m)\) is absolutely convergent.
    Then we have
    \begin{align*}
        \sum_{n = 0}^\infty \bigg(\sum_{m = 0}^\infty f(n, m)\bigg) & = \sum_{(n, m) \in \mathbf{N} \times \mathbf{N}} f(n, m)       \\
                                                                    & = \sum_{(m, n) \in \mathbf{N} \times \mathbf{N}} f(n, m)       \\
                                                                    & = \sum_{m = 0}^\infty \bigg(\sum_{n = 0}^\infty f(n, m)\bigg).
    \end{align*}
    In other words, we can switch the order of infinite sums \emph{provided that the entire sum is absolutely convergent}.
\end{theorem}

\begin{proof}
    The second equality follows easily from Proposition \ref{7.4.3} (and Proposition \ref{3.6.4}).

    Let us first consider the case when \(f(n, m)\) is always non-negative (we will deal with the general case later).
    Write
    \[
        L \coloneqq \sum_{(n, m) \in \mathbf{N} \times \mathbf{N}} f(n, m);
    \]
    our task is to show that the series \(\sum_{n = 0}^\infty (\sum_{m = 0}^\infty f(n, m))\) converges to \(L\).

    One can easily show that \(\sum_{(n, m) \in X} f(n, m) \leq L\) for all finite sets \(X \subseteq \mathbf{N} \times \mathbf{N}\).
    (Use a bijection \(g\) between \(\mathbf{N} \times \mathbf{N}\) and \(\mathbf{N}\), and then use the fact that \(g(X)\) is finite, hence bounded.)
    In particular, for every \(n \in \mathbf{N}\) and \(M \in \mathbf{N}\) we have \(\sum_{m = 0}^M f(n, m) \leq L\), which implies by Proposition \ref{6.3.8} that \(\sum_{m = 0}^\infty f(n, m)\) is convergent for each \(n\).
    Similarly, for any \(N \in \mathbf{N}\) and \(M \in \mathbf{N}\) we have (by Corollary \ref{7.1.14})
    \[
        \sum_{n = 0}^N \sum_{m = 0}^M f(n, m) = \sum_{(n, m) \in X} f(n, m) \leq L
    \]
    where \(X\) is the set \(\{(n,m) \in \mathbf{N} \times \mathbf{N} : n \leq N, m \leq M\}\) which is finite by Proposition \ref{3.6.14}.
    Taking limits of this as \(M \to \infty\) we have (by Exercise \ref{ex 7.1.5} and either Proposition \ref{6.3.8} or Lemma \ref{6.4.13})
    \[
        \sum_{n = 0}^N \sum_{m = 0}^\infty f(n, m) \leq L.
    \]
    By Proposition \ref{6.3.8}, this implies that \(\sum_{n = 0}^\infty \sum_{m = 0}^\infty f(n, m)\) converges, and
    \[
        \sum_{n = 0}^\infty \sum_{m = 0}^\infty f(n, m) \leq L.
    \]
    To finish the proof, it will suffice to show that
    \[
        \sum_{n = 0}^\infty \sum_{m = 0}^\infty f(n, m) \geq L - \varepsilon
    \]
    for every \(\varepsilon > 0\).
    \begin{align*}
                 & L \geq \sum_{n = 0}^\infty \sum_{m = 0}^\infty f(n, m) \geq L - \varepsilon               \\
        \implies & L + \varepsilon \geq \sum_{n = 0}^\infty \sum_{m = 0}^\infty f(n, m) \geq L - \varepsilon \\
        \implies & \varepsilon \geq \sum_{n = 0}^\infty \sum_{m = 0}^\infty f(n, m) - L \geq -\varepsilon    \\
        \implies & \abs*{\sum_{n = 0}^\infty \sum_{m = 0}^\infty f(n, m) - L} \leq \varepsilon               \\
    \end{align*}
    So, let \(\varepsilon > 0\).
    By definition of \(L\), we can then find a finite set \(X \subseteq \mathbf{N} \times \mathbf{N}\) such that \(\sum_{(n, m) \in X} f(n, m) \geq L - \varepsilon\).
    (Since \(\mathbf{N} \times \mathbf{N}\) is countable by Corollary \ref{8.1.13}, we can find a bijection \(g : \mathbf{N} \to \mathbf{N} \times \mathbf{N}\) such that \(\sum_{i = 0}^\infty f(g(i)) = L\), which means \(\forall\ \varepsilon > 0\), \(\exists\ H \in \mathbf{N} \land H \geq 0\) such that \(\abs*{\sum_{i = 0}^h f(g(i)) - L} \leq \varepsilon\) for all \(h \geq H\).
    Now we can choose \(X = \{g(i) : 0 \leq i \leq H\}\))
    This set, being finite, must be contained in some set of the form \(Y \coloneqq \{(n,m) \in \mathbf{N} \times \mathbf{N} : n \leq N; m \leq M \}\).
    Thus by Corollary \ref{7.1.14}
    \[
        \sum_{n = 0}^N \sum_{m = 0}^M f(n, m) = \sum_{(n, m) \in Y} f(n, m) \geq \sum_{(n, m) \in X} f(n, m) \geq L - \varepsilon
    \]
    and hence
    \[
        \sum_{n = 0}^\infty \sum_{m = 0}^\infty f(n, m) \geq \sum_{n = 0}^N \sum_{m = 0}^\infty f(n, m) \geq \sum_{n = 0}^N \sum_{m = 0}^M f(n, m) \geq L - \varepsilon
    \]
    as desired.

    This proves the claim when the \(f(n, m)\) are all non-negative.
    A similar argument works when the \(f(n, m)\) are all non-positive
    (in fact, one can simply apply the result just obtained to the function \(-f(n, m)\), and then use limit laws to remove the \(-\).
    For the general case, note that any function \(f(n, m)\) can be written as \(f_+(n, m) + f_-(n, m)\), where \(f_+(n, m)\) is the positive part of \(f(n, m)\)
    (i.e., it equals \(f(n, m)\) when \(f(n, m)\) is positive, and \(0\) otherwise),
    and \(f_-\) is the negative part of \(f(n, m)\)
    (it equals \(f(n, m)\) when \(f(n, m)\) is negative, and \(0\) otherwise).
    It is easy to show that if \(\sum_{(n, m) \in \mathbf{N} \times \mathbf{N}} f(n, m)\) is absolutely convergent, then so are \(\sum_{(n, m) \in \mathbf{N} \times \mathbf{N}} f_+(n, m)\) and \(\sum_{(n, m) \in \mathbf{N} \times \mathbf{N}} f_-(n, m)\).
    (We can construct a bijection \(g : \mathbf{N} \to \mathbf{N} \times \mathbf{N}\) and then since \(\forall\ n \in \mathbf{N}\) we have \(f_+(g(n)) \leq \abs*{f(g(n))}\) and \(\abs*{f_-(g(n))} \leq \abs*{f(g(n))}\), we know that \((f_+(g(n)))_{n = 0}^\infty\) and \((f_-(g(n)))_{n = 0}^\infty\) are absolutely convergent by comparison test (Proposition \ref{7.3.2}.))
    So now one applies the results just obtained to \(f_+\) and to \(f_-\) and adds them together using limit laws to obtain the result for a general \(f\).
\end{proof}

\begin{lemma}\label{8.2.3}
    Let \(X\) be a countable set, and let \(f : X \to \mathbf{R}\) be a function.
    Then the series \(\sum_{x \in X} f(x)\) is absolutely convergent if and only if
    \[
        \sup\Bigg\{\sum_{x \in A} \abs*{f(x)} : A \subseteq X, A \text{ finite}\Bigg\} < \infty.
    \]
\end{lemma}

\begin{proof}
    Let \(P(X, f)\) be the statement
    \[
        \sup\Bigg\{\sum_{x \in A} \abs*{f(x)} : A \subseteq X, A \text{ finite}\Bigg\} < \infty.
    \]
    We first show that if \(\sum_{x \in X} f(x)\) is absolutely convergent, then \(P(X, f)\) is true.
    Let \(L = \sum_{x \in X} f(x)\).
    Since \(\sum_{x \in X} f(x)\) is absolutely convergent, by Definition \ref{8.2.1} \(\exists\ g : \mathbf{N} \to X\) where \(g\) is a bijection such that
    \[
        L = \sum_{x \in X} \abs*{f(x)} = \sum_{n = 0}^\infty \abs*{f(g(n))}.
    \]
    Let \(A \subseteq X\) be a finite set.
    Since \(g\) is a bijection, we have
    \[
        \sum_{x \in A} \abs*{f(x)} = \sum_{n \in g^{-1}(A)} \abs*{f(g(n))}
    \]
    Since \(A\) is finite, by Exercise \ref{ex 3.6.3}, \(\exists\ M \in \mathbf{N}\) such that \(g^{-1}(A)\) is bounded by \(M\).
    So we have
    \[
        \sum_{x \in A} \abs*{f(x)} = \sum_{n \in g^{-1}(A)} \abs*{f(g(n))} \leq \sum_{n = 0}^M \abs*{f(g(n))} \leq L
    \]
    This is true \(\forall\ A \subseteq X\).
    Thus by Theorem \ref{5.5.9} \(P(X, f)\) is true.

    Now we show that if \(P(X, f)\) is true, then \(\sum_{x \in X} f(x)\) is absolutely convergent.
    Let \(L\) be the supremum described by \(P(X, f)\).
    Since \(X\) is countable, \(\exists\ g : \mathbf{N} \to X\) where \(g\) is a bijection.
    So we have
    \begin{align*}
                 & \forall\ n \in \mathbf{N} : \sum_{x \in g(\{i \in \mathbf{N} : 0 \leq i \leq n\})} \abs*{f(x)} \leq L & (P(X, f) \text{ is true})           \\
        \implies & \forall\ n \in \mathbf{N} : \sum_{i = 0}^n \abs*{f(g(i))} \leq L                                      & \text{(by Definition \ref{7.1.6})}  \\
        \implies & \sum_{i = 0}^\infty \abs*{f(g(i))} \text{ converges}                                                  & \text{(by Proposition \ref{7.3.1})} \\
        \implies & \sum_{x \in X} \abs*{f(x)} \text{ converges}.                                                         & \text{(by Definition \ref{8.2.1})}
    \end{align*}
\end{proof}

\begin{note}
    Inspired by Lemma \ref{8.2.3}, we may now define the concept of an absolutely convergent series even when the set \(X\) could be uncountable.
\end{note}

\begin{definition}\label{8.2.4}
    Let \(X\) be a set (which could be uncountable), and let \(f : X \to \mathbf{R}\) be a function.
    We say that the series \(\sum_{x \in X} f(x)\) is absolutely convergent iff
    \[
        \sup\Bigg\{\sum_{x \in A} \abs*{f(x)} : A \subseteq X, A \text{ finite}\Bigg\} < \infty.
    \]
\end{definition}

\begin{lemma}\label{8.2.5}
    Let \(X\) be a set (which could be uncountable), and let \(f : X \to \mathbf{R}\) be a function such that the series \(\sum_{x \in X} f(x)\) is absolutely convergent.
    Then the set \(\{x \in X : f(x) \neq 0\}\) is at most countable.
\end{lemma}

\begin{proof}
    Suppose that \(X\) is a set and \(f : X \to \mathbf{R}\) is a function such that \(\sum_{x \in X} f(x)\) is absolutely convergent.
    Since \(\sum_{x \in X} f(x)\) is absolutely convergent, by Definition \ref{8.2.4} we have
    \[
        M = \sup\Bigg\{\sum_{x \in A} \abs*{f(x)} : A \subseteq X, A \text{ finite}\Bigg\} < \infty.
    \]
    We first show that \(\forall\ n \in \mathbf{N} \setminus \{0\}\), the set \(S_n = \{x \in X : \abs*{f(x)} > 1 / n\}\) is finite and \(\#(S_n) \leq Mn\).

    Suppose for sake of contradiction that \(S_n\) is infinite.
    Then we can have a finite set \(S \subsetneq S_n\) where \(\#(S) = (M + 1)n\).
    Since \(S\) is finite, we have \(\sum_{x \in S} \abs*{f(x)} \leq M\).
    Since \(S \subsetneq S_n\), we have \(\forall\ x \in S : \abs*{x} > 1 / n\).
    But now we have
    \[
        M \geq \sum_{x \in S} \abs*{f(x)} > \frac{(M + 1)n}{n} = M + 1,
    \]
    a contradiction.
    Thus \(S\) must be finite.

    Now suppose for sake of contradiction \(\#(S_n) > Mn\).
    Again we have
    \[
        M \geq \sum_{x \in S} \abs*{f(x)} > \frac{\#(S_n)}{n} > \frac{Mn}{n} = M,
    \]
    a contradiction.
    Thus \(\#(S_n) \leq Mn\).

    Let \(x \in X\) where \(f(x) \neq 0\).
    If \(x\) does not exist, then we have \(\{x \in X : f(x) \neq 0\} = \emptyset\) which is at most countable.
    So suppose that such \(x\) exists.
    Since \(\abs*{f(x)} \in \mathbf{R}\), by Proposition \ref{5.4.12} we have
    \begin{align*}
                 & \exists\ N \in \mathbf{N} : \frac{1}{\abs*{f(x)}} < N                                                                       \\
        \implies & \abs*{f(x)} > \frac{1}{N}                                                                                                   \\
        \implies & x \in S_N                                                                           & \text{(by the definition of \(S_N\))} \\
        \implies & x \in \bigcup_{n \in \mathbf{N} \setminus \{0\}} S_n                                & \text{(by Axiom \ref{3.11})}          \\
        \implies & \{x \in X : f(x) \neq 0\} \subseteq \bigcup_{n \in \mathbf{N} \setminus \{0\}} S_n. & \text{(by Definition \ref{3.1.15})}
    \end{align*}
    Since \(\forall\ n \in \mathbf{N} \setminus \{0\}\), \(S_n\) is finite and thus at most countable, by Exercise \ref{8.1.9} we have \(\bigcup_{n \in \mathbf{N} \setminus \{0\}} S_n\) is at most countable.
    Since \(\{x \in X : f(x) \neq 0\} \subseteq \bigcup_{n \in \mathbf{N} \setminus \{0\}} S_n\), by Corollary \ref{8.1.7} we thus have \(\{x \in X : f(x) \neq 0\}\) is at most countable.
\end{proof}

\begin{note}
    Because Lemma \ref{8.2.5}, we can define the value of \(\sum_{x \in X} f(x)\) for any absolutely convergent series on an uncountable set \(X\) by the formula
    \[
        \sum_{x \in X} \coloneqq \sum_{x \in X : f(x) \neq 0} f(x),
    \]
    since we have replaced a sum on an uncountable set \(X\) by a sum on the countable set \(\{x \in X : f(x) \neq 0\}\).
    (If the former sum is absolutely convergent, then the latter one is also.)
    Definition \ref{8.2.4} is consistent with the definitions we already have for series on countable sets (Definition \ref{8.2.1}).
\end{note}

\begin{proposition}[Absolutely convergent series laws]\label{8.2.6}
    Let \(X\) be an arbitrary set (possibly uncountable), and let \(f : X \to \mathbf{R}\) and \(g : X \to \mathbf{R}\) be functions such that the series \(\sum_{x \in X} f(x)\) and \(\sum_{x \in X} g(x)\) are both absolutely convergent.
    \begin{enumerate}
        \item The series \(\sum_{x \in X} (f(x) + g(x))\) is absolutely convergent, and
              \[
                  \sum_{x \in X} (f(x) + g(x)) = \sum_{x \in X} f(x) + \sum_{x \in X} g(x).
              \]
        \item If \(c\) is a real number, then \(\sum_{x \in X} cf(x)\) is absolutely convergent, and
              \[
                  \sum_{x \in X} cf(x) = c \sum_{x \in X} f(x).
              \]
        \item If \(X = X_1 \cup X_2\) for some disjoint sets \(X_1\) and \(X_2\), then \(\sum_{x \in X_1} f(x)\) and \(\sum_{x \in X_2} f(x)\) are absolutely convergent, and
              \[
                  \sum_{x \in X_1 \cup X_2} f(x) = \sum_{x \in X_1} f(x) + \sum_{x \in X_2} f(x).
              \]
              Conversely, if \(h : X \to \mathbf{R}\) is such that \(\sum_{x \in X_1} h(x)\) and \(\sum_{x \in X_2} h(x)\) are absolutely convergent, then \(\sum_{x \in X_1 \cup X_2} h(x)\) is also absolute convergent, and
              \[
                  \sum_{x \in X_1 \cup X_2} h(x) = \sum_{x \in X_1} h(x) + \sum_{x \in X_2} h(x).
              \]
        \item If \(Y\) is another set, and \(\phi : Y \to X\) is a bijection, then \(\sum_{y \in Y} f(\phi(y))\) is absolutely convergent, and
              \[
                  \sum_{y \in Y} f(\phi(y)) = \sum_{x \in X} f(x).
              \]
    \end{enumerate}
\end{proposition}

\begin{proof}
    This shall be done once I learn Axiom of choice.
\end{proof}

\begin{lemma}\label{8.2.7}
    Let \(\sum_{n = 0}^\infty a_n\) be a series of real numbers which is conditionally convergent, but not absolutely convergent.
    Define the sets \(A_+ \coloneqq \{n \in \mathbf{N} : a_n \geq 0\}\) and \(A_- \coloneqq \{n \in \mathbf{N} : a_n < 0\}\), thus \(A_+ \cup A_- = \mathbf{N}\) and \(A_+ \cap A_- = \emptyset\).
    Then both of the series \(\sum_{n \in A_+} a_n\) and \(\sum_{n \in A_-} a_n\) are not absolutely convergent.
\end{lemma}

\begin{proof}
    This shall be done once I learn Axiom of choice.
\end{proof}

\begin{note}
    Theorem \ref{8.2.8} is done by Georg Riemann (1826 -- 1866), which asserts that a series which converges conditionally but not absolutely can be rearranged to converge to any value one pleases!
\end{note}

\begin{theorem}\label{8.2.8}
    Let \(\sum_{n = 0}^\infty a_n\) be a series which is conditionally convergent, but not absolutely convergent, and let \(L\) be any real number.
    Then there exists a bijection \(f : \mathbf{N} \to \mathbf{N}\) such that \(\sum_{m = 0}^\infty a_{f(m)}\) converges conditionally to \(L\).
\end{theorem}

\begin{proof}
    Let \(A_+\) and \(A_-\) be the sets in Lemma \ref{8.2.7};
    from Lemma \ref{8.2.7} we know that \(\sum_{n \in A_+} a_n\) and \(\sum_{n \in A_-} a_n\) both fail to be absolutely convergent.
    In particular \(A_+\) and \(A_-\) are infinite (why?).
    By Proposition \ref{8.1.5} we can then find increasing bijections \(f_+ : \mathbf{N} \to A_+\) and \(f_- : \mathbf{N} \to A_-\).
    Thus the sums \(\sum_{m = 0}^\infty a_{f_+(m)}\) and \(\sum_{m = 0}^\infty a_{f_-(m)}\) both fail to be absolutely convergent (why?).
    The plan shall be to select terms from the divergent series \(\sum_{m = 0}^\infty a_{f_+(m)}\) and \(\sum_{m = 0}^\infty a_{f_-(m)}\) in a well-chosen order in order to keep their difference converging towards \(L\).

    We define the sequence \(n_0, n_1, n_2, \dots\) of natural numbers recursively as follows.
    Suppose that \(j\) is a natural number, and that \(n_i\) has already been defined for all \(i < j\) (this is vacuously true if \(j = 0\)).
    We then define \(n_j\) by the following rule:
    \begin{enumerate}[label=(\Roman*)]
        \item If \(\sum_{0 \leq i < j} a_{n_i} < L\), then we set
              \[
                  n_j \coloneqq \min\{n \in A_+ : n \neq n_i \ \forall\ i < j\}.
              \]
        \item If instead \(\sum_{0 \leq i < j} a_{n_i} \geq L\), then we set
              \[
                  n_j \coloneqq \min\{n \in A_- : n \neq n_i \ \forall\ i < j\}.
              \]
    \end{enumerate}
    Note that this recursive definition is well-defined because \(A_+\) and \(A_-\) are infinite, and so the sets \(\{n \in A_+ : n \neq n_i \ \forall\ i < j\}\) and \(n_j \coloneqq \min\{n \in A_- : n \neq n_i \ \forall\ i < j\}\) are never empty.
    (Intuitively, we add a non-negative number to the series whenever the partial sum is too low, and add a negative number when the sum is too high.)
    One can then verify the following claims:
    \begin{enumerate}
        \item The map \(j \mapsto n_j\) is injective. (Why?)
        \item Case I occurs an infinite number of times, and Case II also occurs an infinite number of times. (Why? prove by contradiction.)
        \item The map \(j \mapsto n_j\) is surjective. (Why?)
        \item We have \(\lim_{j \to \infty} a_{n_j} = 0\). (Why? Note from Corollary \ref{7.2.6} that \(\lim_{n \to \infty} a_n = 0\).)
        \item We have \(\lim_{j \to \infty} \sum_{0 \leq i \leq j} a_{n_i} = L\). (Why?)
    \end{enumerate}
    The claim then follows by setting \(f(i) \coloneqq n_i\) for all \(i\).
\end{proof}

\exercisesection

\begin{exercise}\label{ex 8.2.1}
    Prove Lemma \ref{8.2.3}.
\end{exercise}

\begin{proof}
    See Lemma \ref{8.2.3}.
\end{proof}

\begin{exercise}\label{ex 8.2.2}
    Prove Lemma \ref{8.2.5}.
\end{exercise}

\begin{proof}
    See Lemma \ref{8.2.5}.
\end{proof}

\begin{exercise}\label{ex 8.2.3}
    Prove Proposition \ref{8.2.6}.
\end{exercise}

\begin{proof}
    See Proposition \ref{8.2.6}.
\end{proof}

\begin{exercise}\label{ex 8.2.4}
    Prove Lemma \ref{8.2.7}.
\end{exercise}

\begin{proof}
    See Lemma \ref{8.2.7}.
\end{proof}

\begin{exercise}\label{ex 8.2.5}
    Explain the gaps marked (why?) in the proof of Theorem \ref{8.2.8}.
\end{exercise}

\begin{proof}
    See Theorem \ref{8.2.8}.
\end{proof}

\begin{exercise}\label{ex 8.2.6}
    Let \(\sum_{n = 0}^\infty a_n\) be a series which is conditionally convergent, but not absolutely convergent.
    Show that there exists a bijection \(f : \mathbf{N} \to \mathbf{N}\) such that \(\sum_{m = 0}^\infty a_{f(m)}\) diverges to \(+\infty\), or more precisely that
    \[
        \lim\inf_{N \to \infty} \sum_{m = N}^\infty a_{f(m)} = \lim\sup_{N \to \infty} \sum_{m = N}^\infty a_{f(m)} = +\infty.
    \]
    (Of course, a similar statement holds with \(+\infty\) replaced by \(-\infty\).)
\end{exercise}

\begin{proof}
    This shall be done once I learn Axiom of choice.
\end{proof}