\section{Summation on infinite sets}\label{sec 8.2}

\begin{definition}[Series on countable sets]\label{8.2.1}
Let \(X\) be a countable set, and let \(f : X \to \mathbf{R}\) be a function.
We say that the series \(\sum_{x \in X} f(x)\) is absolutely convergent iff for some bijection \(g : \mathbf{N} \to X\), the sum \(\sum_{n = 0}^\infty f(g(n))\) is absolutely convergent.
We then define the sum of \(\sum_{x \in X} f(x)\) by the formula
\[
    \sum_{x \in X} f(x) = \sum_{n = 0}^\infty f(g(n)).
\]
\end{definition}

\begin{note}
From Proposition \ref{7.4.3} (and Proposition \ref{3.6.4}), one can show that these definitions do not depend on the choice of \(g\), and so are well defined.
\end{note}

\begin{note}
For finite sets \(X\) we adopt the convention that series \(\sum_{x \in X} f(x)\) are automatically considered to be absolutely convergent.
\end{note}

\begin{theorem}[Fubini's theorem for infinite sums]\label{8.2.2}
Let \(f : \mathbf{N} \times \mathbf{N} \to \mathbf{R}\) be a function such that \(\sum_{(n, m) \in \mathbf{N} \times \mathbf{N}} f(n, m)\) is absolutely convergent.
Then we have
\begin{align*}
\sum_{n = 0}^\infty \bigg(\sum_{m = 0}^\infty f(n, m)\bigg) &= \sum_{(n, m) \in \mathbf{N} \times \mathbf{N}} f(n, m) \\
&= \sum_{(m, n) \in \mathbf{N} \times \mathbf{N}} f(n, m) \\
&= \sum_{m = 0}^\infty \bigg(\sum_{n = 0}^\infty f(n, m)\bigg).
\end{align*}
In other words, we can switch the order of infinite sums \emph{provided that the entire sum is absolutely convergent}.
\end{theorem}

\begin{proof}
The second equality follows easily from Proposition \ref{7.4.3} (and Proposition \ref{3.6.4}).

Let us first consider the case when \(f(n, m)\) is always non-negative (we will deal with the general case later).
Write
\[
    L \coloneqq \sum_{(n, m) \in \mathbf{N} \times \mathbf{N}} f(n, m);
\]
our task is to show that the series \(\sum_{n = 0}^\infty (\sum_{m = 0}^\infty f(n, m))\) converges to \(L\).

One can easily show that \(\sum_{(n, m) \in X} f(n, m) \leq L\) for all finite sets \(X \subseteq \mathbf{N} \times \mathbf{N}\).
(Use a bijection \(g\) between \(\mathbf{N} \times \mathbf{N}\) and \(\mathbf{N} \times \mathbf{N}\), and then use the fact that \(g(X)\) is finite, hence bounded.)
In particular, for every \(n \in \mathbf{N}\) and \(M \in \mathbf{N}\) we have \(\sum_{m = 0}^M f(n, m) \leq L\), which implies by Proposition \ref{6.3.8} that \(\sum_{m = 0}^\infty f(n, m)\) is convergent for each \(m\).
Similarly, for any \(N \in \mathbf{N}\) and \(M \in \mathbf{N}\) we have (by Corollary \ref{7.1.14})
\[
    \sum_{n = 0}^N \sum_{m = 0}^M f(n, m) = \sum_{(n, m) \in X} f(n, m) \leq L
\]
where \(X\) is the set \(\{(n,m) \in \mathbf{N} \times \mathbf{N} : n \leq N, m \leq M\}\) which is finite by Proposition \ref{3.6.14}.
Taking limits of this as \(M \to \infty\) we have (by Exercise \ref{ex 7.1.5} and either Proposition \ref{6.3.8} or Lemma \ref{6.4.13})
\[
    \sum_{n = 0}^N \sum_{m = 0}^\infty f(n, m) \leq L.
\]
By Proposition \ref{6.3.8}, this implies that \(\sum_{n = 0}^\infty \sum_{m = 0}^\infty f(n, m)\) converges, and
\[
    \sum_{n = 0}^\infty \sum_{m = 0}^\infty f(n, m) \leq L.
\]
To finish the proof, it will suffice to show that
\[
    \sum_{n = 0}^\infty \sum_{m = 0}^\infty f(n, m) \geq L - \varepsilon
\]
for every \(\varepsilon > 0\).
\begin{align*}
& L \geq \sum_{n = 0}^\infty \sum_{m = 0}^\infty f(n, m) \geq L - \varepsilon \\
\implies & L + \varepsilon \geq \sum_{n = 0}^\infty \sum_{m = 0}^\infty f(n, m) \geq L - \varepsilon \\
\implies & \varepsilon \geq \sum_{n = 0}^\infty \sum_{m = 0}^\infty f(n, m) - L \geq -\varepsilon \\
\implies & \abs*{\sum_{n = 0}^\infty \sum_{m = 0}^\infty f(n, m) - L} \leq \varepsilon \\
\end{align*}
So, let \(\varepsilon > 0\).
By definition of \(L\), we can then find a finite set \(X \subseteq \mathbf{N} \times \mathbf{N}\) such that \(\sum_{(n, m) \in X} f(n, m) \geq L - \varepsilon\).
(Since \(\mathbf{N} \times \mathbf{N}\) is countable by Corollary \ref{8.1.13}, we can find a bijection \(g : \mathbf{N} \to \mathbf{N} \times \mathbf{N}\) such that \(\sum_{i = 0}^\infty f(g(i)) = L\), which means \(\forall\ \varepsilon > 0\), \(\exists\ H \in \mathbf{N} \land H \geq 0\) such that \(\abs*{\sum_{i = 0}^h f(g(i)) - L} \leq \varepsilon\) for all \(h \geq H\).
Now we can choose \(X = \{g(i) : 0 \leq i \leq H\}\))
This set, being finite, must be contained in some set of the form \(Y \coloneqq \{(n,m) \in \mathbf{N} \times \mathbf{N} : n \leq N; m \leq M \}\).
Thus by Corollary \ref{7.1.14}
\[
    \sum_{n = 0}^N \sum_{m = 0}^M f(n, m) = \sum_{(n, m) \in Y} f(n, m) \geq \sum_{(n, m) \in X} f(n, m) \geq L - \varepsilon
\]
and hence
\[
    \sum_{n = 0}^\infty \sum_{m = 0}^\infty f(n, m) \geq \sum_{n = 0}^N \sum_{m = 0}^\infty f(n, m) \geq \sum_{n = 0}^N \sum_{m = 0}^M f(n, m) \geq L - \varepsilon
\]
as desired.

This proves the claim when the \(f(n, m)\) are all non-negative.
A similar argument works when the \(f(n, m)\) are all non-positive
(in fact, one can simply apply the result just obtained to the function \(-f(n, m)\), and then use limit laws to remove the \(-\).
For the general case, note that any function \(f(n, m)\) can be written as \(f_+(n, m) + f_-(n, m)\), where \(f_+(n, m)\) is the positive part of \(f(n, m)\)
(i.e., it equals \(f(n, m)\) when \(f(n, m)\) is positive, and \(0\) otherwise),
and \(f_-\) is the negative part of \(f(n, m)\)
(it equals \(f(n, m)\) when \(f(n, m)\) is negative, and \(0\) otherwise).
It is easy to show that if \(\sum_{(n, m) \in \mathbf{N} \times \mathbf{N}} f(n, m)\) is absolutely convergent, then so are \(\sum_{(n, m) \in \mathbf{N} \times \mathbf{N}} f_+(n, m)\) and \(\sum_{(n, m) \in \mathbf{N} \times \mathbf{N}} f_-(n, m)\).
So now one applies the results just obtained to \(f_+\) and to \(f_-\) and adds them together using limit laws to obtain the result for a general \(f\).
\end{proof}