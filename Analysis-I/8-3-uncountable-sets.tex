\section{Uncountable sets}\label{sec 8.3}

\begin{note}
    It was great shock when Georg Cantor (1845 -- 1918) showed in 1873 that certain sets
    - including the real numbers \(\mathbf{R}\) are in fact uncountable -
    no matter how hard you try, you cannot arrange the real numbers \(\mathbf{R}\) as a sequence \(a_0, a_1, a_2, \dots\).
    (Of course, the real numbers \(\mathbf{R}\) can contain many infinite sequences, e.g., the sequence \(0, 1, 2, 3, 4, \dots\).
    However, what Cantor proved is that no such sequence can ever exhaust the real numbers;
    no matter what sequence of real numbers you choose, there will always be some real numbers that are not covered by that sequence.)
\end{note}

\begin{theorem}[Cantor's theorem]\label{8.3.1}
    Let \(X\) be an arbitrary set (finite or infinite).
    Then the sets \(X\) and \(2^X\) cannot have equal cardinality.
\end{theorem}

\begin{proof}
    Suppose for sake of contradiction that the sets \(X\) and \(2^X\) had equal cardinality.
    Then there exists a bijection \(f : X \to 2^X\) between \(X\) and the power set of \(X\).
    Now consider the set
    \[
        A \coloneqq \{x \in X : x \notin f(x)\}.
    \]
    Note that this set is well-defined since \(f(x)\) is an element of \(2^X\) and is hence a subset of \(X\).
    Clearly \(A\) is a subset of \(X\), hence is an element of \(2^X\).
    Since \(f\) is a bijection, there must therefore exist \(x \in X\) such that \(f(x) = A\).
    There are now two cases, depending on whether \(x \in A\) or \(x \notin A\).
    If \(x \in A\), then by definition of \(A\) we have \(x \notin f(x)\), hence \(x \notin A\), a contradiction.
    But if \(x \notin A\), then \(x \notin f(x)\), hence by definition of \(A\) we have \(x \in A\), a contradiction.
    Thus in either case we have a contradiction.
\end{proof}

\begin{remark}\label{8.3.2}
    The reader should compare the proof of Cantor's theorem with the statement of Russell's paradox (Section \ref{sec 3.2}).
    The point is that a bijection between \(X\) and \(2^X\) would come dangerously close to the concept of a set \(X\) ``containing itself''.
\end{remark}

\begin{corollary}\label{8.3.3}
    \(2^{\mathbf{N}}\) is uncountable.
\end{corollary}

\begin{proof}
    By Theorem \ref{8.3.1}, \(2^{\mathbf{N}}\) cannot have equal cardinality with \(\mathbf{N}\), hence is either uncountable or finite.
    However, \(2^{\mathbf{N}}\) contains as a subset the set of singletons \(\{\{n\} : n \in \mathbf{N}\}\), which is clearly bijective to \(\mathbf{N}\) and hence countably infinite.
    Thus \(2^{\mathbf{N}}\) cannot be finite (by Proposition \ref{3.6.14}), and is hence uncountable.
\end{proof}

\begin{corollary}\label{8.3.4}
    \(\mathbf{R}\) is uncountable.
\end{corollary}

\begin{proof}
    Let us define the map \(f : 2^{\mathbf{N}} \to \mathbf{R}\) by the formula
    \[
        f(A) \coloneqq \sum_{n \in A} 10^{-n}.
    \]
    Observe that since \(\sum_{n = 0}^\infty 10^{-n}\) is an absolutely convergent series (by Lemma \ref{7.3.3}), the series \(\sum_{n \in A} 10^{-n}\) is also absolutely convergent (by Proposition \ref{8.2.6}(c)).
    Thus the map \(f\)  is well defined.
    We now claim that \(f\) is injective.
    Suppose for sake of contradiction that there were two distinct sets \(A, B \in 2^{\mathbf{N}}\) such that \(f(A) = f(B)\).
    Since \(A \neq B\), the set \((A \setminus B) \cup (B \setminus A)\) is a non-empty subset of \(\mathbf{N}\).
    By the well-ordering principle (Proposition \ref{8.1.4}), we can then define the minimum of this set, say \(n_0 \coloneqq \min(A \setminus B) \cup (B \setminus A)\).
    Thus \(n_0\) either lies in \(A \setminus B\) or \(B \setminus A\).
    By symmetry we may assume it lies in \(A \setminus B\).
    Then \(n_0 \in A\), \(n_0 \notin B\), and for all \(n < n_0\) we either have \(n \in A, B\) or \(n \notin A, B\).
    Thus
    \begin{align*}
        0 & = f(A) - f(B)                                                                                 \\
          & = \sum_{n \in A} 10^{-n} - \sum_{n \in B} 10^{-n}                                             \\
          & = \Bigg(\sum_{n < n_0 : n \in A} 10^{-n} + 10^{-n_0} + \sum_{n > n_0 : n \in A} 10^{-n}\Bigg) \\
          & - \Bigg(\sum_{n < n_0 : n \in B} 10^{-n} + \sum_{n > n_0 : n \in B} 10^{-n}\Bigg)             \\
          & = 10^{-n_0} + \sum_{n > n_0 : n \in A} 10^{-n} - \sum_{n > n_0 : n \in B} 10^{-n}             \\
          & \geq 10^{-n_0} + 0 - \sum_{n > n_0} 10^{-n}                                                   \\
          & \geq 10^{-n_0} - \frac{1}{9} 10^{-n_0}                                                        \\
          & > 0,
    \end{align*}
    a contradiction, where we have used the geometric series lemma (Lemma \ref{7.3.3}) to sum
    \[
        \sum_{n > n_0} 10^{-n} = \sum_{m = 0}^\infty 10^{-(n_0 + 1 + m)} = 10^{-n_0 - 1} \sum_{m = 0}^\infty 10^{-m} = \frac{1}{9} 10^{-n_0}.
    \]
    Thus \(f\) is injective, which means that \(f(2^{\mathbf{N}})\) has the same cardinality as \(2^{\mathbf{N}}\) and is thus uncountable.
    Since \(f(2^{\mathbf{N}})\) is a subset of \(\mathbf{R}\), this forces \(\mathbf{R}\) to be uncountable also (otherwise this would contradict Corollary \ref{8.1.7}), and we are done.
\end{proof}

\setcounter{theorem}{5}
\begin{remark}\label{8.3.6}
    Corollary \ref{8.3.4} shows that the reals have strictly larger cardinality than the natural numbers (in the sense of Exercise \ref{ex 3.6.7}).
    One could ask whether there exist any sets which have strictly larger cardinality than the natural numbers, but strictly smaller cardinality than the reals.
    The \emph{Continuum Hypothesis} asserts that no such sets exist.
    Interestingly, it was shown in separate works of Kurt Gödel (1906 -- 1978) and Paul Cohen (1934 -- 2007) that this hypothesis is independent of the other axioms of set theory;
    it can neither be proved nor disproved in that set of axioms
    (unless those axioms are inconsistent, which is highly unlikely).
\end{remark}

\exercisesection

\begin{exercise}\label{ex 8.3.1}
    Let \(X\) be a finite set of cardinality \(n\).
    Show that \(2^X\) is a finite set of cardinality \(2^n\).
\end{exercise}

\begin{proof}
    We use induction on \(n\).
    For \(n = 0\), \(X = \emptyset\) and \(2^{\emptyset} = \{\emptyset\}\).
    Clearly \(\#(2^{\emptyset}) = 1\).
    So the base case holds.
    Suppose inductively that \(\#(2^X) = 2^n\) for some \(n\).
    We need to show that for \(n + 1\), \(\#(2^X) = 2^{n + 1}\).
    Since \(\#(X) = n + 1\), we have \(X \neq \emptyset\).
    Let \(x \in X\).
    Then we have
    \begin{align*}
        \#(2^X) & = \#(2^{(X \setminus \{x\}) \cup \{x\}}) \\
        & = \#(2^{X \setminus \{x\}}) \times \#(2^{\{x\}}) & \text{(by Exercise \ref{ex 3.6.6})} \\
        & = 2^n \times \#(2^{\{x\}}) & \text{(by induction hypothesis)} \\
        & = 2^n \times \#(\{\emptyset, \{x\}\}) \\
        & = 2^n \times 2 \\
        & = 2^{n + 1}.
    \end{align*}
    This close the induction.
\end{proof}

\begin{exercise}\label{ex 8.3.2}
    Let \(A, B, C\) be sets such that \(A \subseteq B \subseteq C\), and suppose that there is a injection \(f : C \to A\).
    Define the sets \(D_0, D_1, D_2, \dots\) recursively by setting \(D_0 \coloneqq B \setminus A\), and then \(D_{n + 1} \coloneqq f(D_n)\) for all natural numbers \(n\).
    Prove that the sets \(D_0, D_1, \dots\) are all disjoint from each other
    (i.e., \(D_n \cap D_m = \emptyset\) whenever \(n \neq m\)).
    Also show that if \(g : A \to B\) is the function defined by setting \(g(x) \coloneqq f^{-1}(x)\) when \(x \in \bigcup_{n = 1}^\infty D_n\), and \(g(x) \coloneqq x\) when \(x \notin \bigcup_{n = 1}^\infty D_n\), then \(g\) does indeed map \(A\) to \(B\) and is a bijection between the two.
    In particular, \(A\) and \(B\) have the same cardinality.
\end{exercise}

\begin{proof}
    We first show that \(\forall\ n, m \in \mathbf{N} : D_n \cap D_m = \emptyset\).
    Let \(P(n)\) be the statement ``\(\forall\ k \in \mathbf{N} \land k > 0 : D_n \cap D_{n + k} = \emptyset\)''.
    We use induction on \(n\) to show that \(P(n)\) is true.
    For \(n = 0\), we have \(D_0 = B \setminus A\) and \(D_k = f(D_{k - 1}) \subseteq A\).
    Thus we have \(D_0 \cap D_k = \emptyset\) and the base case holds.
    Suppose inductively that \(P(n)\) is true for some \(n\).
    Then for \(n + 1\), we need to show that \(P(n + 1)\) is also true.
    Suppose for sake of contradiction that \(D_{n + 1} \cap D_{n + k + 1} \neq \emptyset\).
    Then we have
    \begin{align*}
        & \exists\ x \in D_{n + 1} \cap D_{n + k + 1} \\
        \implies & x \in f(D_n) \cap f(D_{n + k}) \\
        \implies & \exists\ (z \in D_n \land z' \in D_{n + k}) : f(z) = f(z') \\
        \implies & z = z'. & \text{(\(f\) is injective)}
    \end{align*}
    But by induction hypothesis we know that \(D_n \cap D_{n + k} = \emptyset\), a contradiction.
    Thus \(D_{n + 1} \cap D_{n + k + 1} = \emptyset\).
    This close the induction.

    Now we use \(P(n)\) to show that \(\forall\ n, m \in \mathbf{N} : D_n \cap D_m = \emptyset\).
    Since \(m \neq n \implies m < n \lor m > n\), we can let
    \[
        \begin{cases}
            n = m + k & \text{if } m < n, \\
            m = n + k & \text{if } n < m,
        \end{cases}
    \]
    where \(k > 0\).
    Using \(P(n)\) we can thus have \(D_n \cap D_m = \emptyset\).

    Finally we show that if \(g : A \to B\) is a function where
    \[
        \forall\ x \in A : g(x) = \begin{cases}
            f^{-1}(x) & \text{if } x \in \bigcup_{n = 1}^\infty D_n, \\
            x & \text{otherwise},
        \end{cases}
    \]
    then \(g\) is bijective.
    Since \(f\) is injective, \(f\) is thus bijective from \(\bigcup_{n = 0}^\infty D_n\) to \(f(\bigcup_{n = 0}^\infty D_n)\).
    By definition we have \(f(\bigcup_{n = 0}^\infty D_n) = \bigcup_{n = 1}^\infty D_n\).
    Thus \(g\) is well-defined.
\end{proof}