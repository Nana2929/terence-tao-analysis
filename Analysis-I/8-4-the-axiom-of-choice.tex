\section{The axiom of choice}\label{sec 8.4}

\begin{note}
We now discuss the final axiom of the standard Zermelo-Fraenkel-Choice system of set theory, namely the \emph{axiom of choice}.
We have delayed introducing this axiom for a while now, to demonstrate that a large portion of the foundations of analysis can be constructed without appealing to this axiom.
However, in many further developments of the theory, it is very convenient (and in some cases even essential) to employ this powerful axiom.
On the other hand, the axiom of choice can lead to a number of unintuitive consequences (for instance the \emph{Banach-Tarski paradox}), and can lead to proofs that are philosophically somewhat unsatisfying.
Nevertheless, the axiom is almost universally accepted by mathematicians.
One reason for this confidence is a theorem due to the great logician Kurt Gödel, who showed that a result proven using the axiom of choice will never contradict a result proven without the axiom of choice
(unless all the other axioms of set theory are themselves inconsistent, which is highly unlikely).
More precisely, Gödel demonstrated that the axiom of choice is \emph{undecidable};
it can neither be proved nor disproved from the other axioms of set theory, so long as those axioms are themselves consistent.
(From a set of inconsistent axioms one can prove that every statement is both true and false.)
In practice, this means that any ``real-life'' application of analysis
(more precisely, any application involving only ``decidable'' questions)
which can be rigorously supported using the axiom of choice, can also be rigorously supported without the axiom of choice, though in many cases it would take a much more complicated and lengthier argument to do so if one were not allowed to use the axiom of choice.
Thus one can view the axiom of choice as a convenient and safe labour-saving device in analysis.
In other disciplines of mathematics, notably in set theory in which many of the questions are not decidable, the issue of whether to accept the axiom of choice is more open to debate, and involves some philosophical concerns as well as mathematical and logical ones.
\end{note}