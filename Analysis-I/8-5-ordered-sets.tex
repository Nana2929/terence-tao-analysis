\section{Ordered sets}\label{sec 8.5}

\begin{definition}[Partially ordered sets]\label{8.5.1}
    A \emph{partially ordered set} (or \emph{poset}) is a set \(X\), together with a relation \(\leq_X\) on \(X\)
    (thus for any two objects \(x, y \in X\), the statement \(x \leq_X y\) is either a true statement or a false statement).
    Furthermore, this relation is assumed to obey the following three properties:
    \begin{itemize}
        \item (Reflexivity) For any \(x \in X\), we have \(x \leq_X x\).
        \item (Anti-symmetry) If \(x, y \in X\) are such that \(x \leq_X y\) and \(y \leq_X x\), then \(x = y\).
        \item (Transitivity) If \(x, y, z \in X\) are such that \(x \leq_X y\) and \(y \leq_X z\), then \(x \leq_X z\).
    \end{itemize}
    We refer to \(\leq_X\) as the \emph{ordering relation}.
    In most situations it is understood what the set \(X\) is from context, and in those cases we shall simply write \(\leq\) instead of \(\leq_X\).
    We write \(x <_X y\) (or \(x < y\) for short) if \(x \leq_X y\) and \(x \neq y\).
\end{definition}

\begin{note}
    Strictly speaking, a partially ordered set is not a set \(X\), but rather a pair \((X, \leq_X)\).
    But in many cases the ordering \(\leq_X\) will be clear from context, and so we shall refer to \(X\) itself as the partially ordered set even though this is technically incorrect.
\end{note}

\begin{example}\label{8.5.2}
    The natural numbers \(\mathbf{N}\) together with the usual less-than-or-equal-to relation \(\leq\) (as defined in Definition \ref{2.2.11}) forms a partially ordered set, by Proposition \ref{2.2.12}.
    Similar arguments (using the appropriate definitions and propositions) show that the integers \(\mathbf{Z}\), the rationals \(\mathbf{Q}\), the reals \(\mathbf{R}\), and the extended reals \(\mathbf{R}^*\) are also partially ordered sets.
    Meanwhile, if \(X\) is any collection of sets, and one uses the relation of is-a-subset-of \(\subseteq\) (as defined in Definition \ref{3.1.15}) for the ordering relation \(\leq_X\), then \(X\) is also partially ordered (Proposition \ref{3.1.18}).
    Note that it is certainly possible to give these sets a different partial ordering than the standard one.
\end{example}

\begin{definition}[Totally ordered set]\label{8.5.3}
    Let \(X\) be a partially ordered set with some order relation \(\leq_X\).
    A subset \(Y\) of \(X\) is said to be \emph{totally ordered} if, given any two \(y, y' \in Y\), we either have \(y \leq_X y'\) or \(y' \leq_X y\) (or both).
    If \(X\) itself is totally ordered, we say that \(X\) is a \emph{totally ordered set} (or \emph{chain}) with order relation \(\leq_X\).
\end{definition}

\begin{example}\label{8.5.4}
    The natural numbers \(\mathbf{N}\), the integers \(\mathbf{Z}\), the rationals \(\mathbf{Q}\), reals \(\mathbf{R}\), and the extended reals \(\mathbf{R}^*\), all with the usual ordering relation \(\leq\), are totally ordered
    (by Proposition \ref{2.2.13}, Lemma \ref{4.1.11}, Proposition \ref{4.2.9}, Proposition \ref{5.4.7}, and Proposition \ref{6.2.5} respectively).
    Also, any subset of a totally ordered set is again totally ordered. On the other hand, a collection of sets with the \(\subseteq\) relation is usually not totally ordered.
\end{example}

\begin{definition}[Maximal and minimal elements]\label{8.5.5}
    Let \(X\) be a partially ordered set, and let \(Y\) be a subset of \(X\).
    We say that \(y\) is a \emph{minimal element} of \(Y\) if \(y \in Y\) and there is no element \(y' \in Y\) such that \(y' < y\).
    We say that \(y\) is a \emph{maximal element} of \(Y\) if \(y \in Y\) and there is no element \(y' \in Y\) such that \(y < y'\).
\end{definition}

\setcounter{theorem}{6}
\begin{example}\label{8.5.7}
    The natural numbers \(\mathbf{N}\) (ordered by \(\leq\)) has a minimal element, namely \(0\), but no maximal element.
    The set of integers \(\mathbf{Z}\) has no maximal and no minimal element.
\end{example}

\begin{definition}[Well-ordered sets]\label{8.5.8}
    Let \(X\) be a partially ordered set, and let \(Y\) be a totally ordered subset of \(X\).
    We say that \(Y\) is \emph{well-ordered} if every non-empty subset \(Z\) of \(Y\) has a minimal element \(\min(Z)\).
\end{definition}

\begin{example}\label{8.5.9}
    The natural numbers \(\mathbf{N}\) are well-ordered by Proposition \ref{8.1.4}.
    However, the integers \(\mathbf{Z}\), the rationals \(\mathbf{Q}\), and the real numbers \(\mathbf{R}\) are not (see Exercise \ref{ex 8.1.2}).
    Every subset of a well-ordered set is again well-ordered.
\end{example}

\begin{proposition}[Principle of strong induction]\label{8.5.10}
    Let \(X\) be a well-ordered set with an ordering relation \(\leq\), and let \(P(n)\) be a property pertaining to an element \(n \in X\)
    (i.e., for each \(n \in X\), \(P(n)\) is either a true statement or a false statement).
    Suppose that for every \(n \in X\), we have the following implication:
    if \(P(m)\) is true for all \(m \in X\) with \(m <_X n\), then \(P(n)\) is also true.
    Then \(P(n)\) is true for all \(n \in X\).
\end{proposition}

\begin{proof}
    Since \(X\) is well-ordered, by Definition \ref{8.5.8} we know that \(X \neq \emptyset\) and \(\min(X)\) exists.
    By Definition \ref{8.5.5} we know that the statement ``\(\forall\ m \in X\), \(m <_X \min(X) \implies P(m)\) is true'' is vacuously true since there is no \(m <_X n\).
    Thus by hypothesis we know that \(P\big(\min(X)\big)\) is vacuously true.

    Now let \(Y\) be the set
    \[
        Y = \{m \in X : P(m) \text{ is false}\}.
    \]
    Suppose for sake of contradiction that \(Y \neq \emptyset\).
    Since \(X\) is well-ordered and \(Y \subseteq X\), by Definition \ref{8.5.8} we know that \(\min(Y)\) exists.
    From previous claim we know that \(\min(Y) \neq \min(X)\), thus the set
    \[
        Y' = \{m \in X : m <_X \min(Y)\}
    \]
    is not empty.
    Also by Definition \ref{8.5.5} we know that \(\forall\ m \in Y'\), \(P(m)\) is true, otherwise contradict to the definition of \(\min(Y)\).
    But this means \(\forall\ m \in X\), \(m <_X \min(Y) \implies P\big(\min(Y)\big)\) is false, which contradict to the hypothesis.
    Thus we must have \(P(n)\) is true for all \(n \in X\).
\end{proof}

\begin{remark}\label{8.5.11}
    It may seem strange that there is no ``base'' case in strong induction, corresponding to the hypothesis \(P(0)\) in Axiom \ref{2.5}.
    However, such a base case is automatically included in the strong induction hypothesis.
    Indeed, if \(0\) is the minimal element of \(X\), then by specializing the hypothesis ``if \(P(m)\) is true for all \(m \in X\) with \(m <_X n\), then \(P(n)\) is also true'' to the \(n = 0\) case, we automatically obtain that \(P(0)\) is true.
    (Since there is no element \(m <_X n\), such statement ``if \(P(m)\) is true for all \(m \in X\) with \(m <_X n\)'' is false, thus the implication holds vacuously.)
\end{remark}

\begin{definition}[Upper bounds and strict upper bounds]\label{8.5.12}
    Let \(X\) be a partially ordered set with ordering relation \(\leq\), and let \(Y\) be a subset of \(X\).
    If \(x \in X\), we say that \(x\) is an \emph{upper bound} for \(Y\) iff \(y \leq x\) for all \(y \in Y\).
    If in addition \(x \notin Y\), we say that \(x\) is a \emph{strict upper bound} for \(Y\).
    Equivalently, \(x\) is a strict upper bound for \(Y\) iff \(y < x\) for all \(y \in Y\).
\end{definition}

\setcounter{theorem}{13}
\begin{lemma}\label{8.5.14}
    Let \(X\) be a partially ordered set with ordering relation \(\leq\), and let \(x_0\) be an element of \(X\).
    Then there is a well-ordered subset \(Y\) of \(X\) which has \(x_0\) as its minimal element, and which has no strict upper bound.
\end{lemma}

\begin{proof}
    The intuition behind this lemma is that one is trying to perform the following algorithm:
    we initalize \(Y \coloneqq \{x_0\}\).
    If \(Y\) has no strict upper bound, then we are done;
    otherwise, we choose a strict upper bound and add it to \(Y\).
    Then we look again to see if \(Y\) has a strict upper bound or not.
    If not, we are done;
    otherwise we choose another strict upper bound and add it to \(Y\).
    We continue this algorithm ``infinitely often'' until we exhaust all the strict upper bounds;
    the axiom of choice comes in because infinitely many choices are involved.
    This is however not a rigorous proof because it is quite difficult to precisely pin down what it means to perform an algorithm ``infinitely often''.
    Instead, what we will do is that we will isolate a collection of ``partially completed'' sets \(Y\), which we shall call \emph{good sets}, and then take the union of all these good sets to obtain a ``completed'' object \(Y_{\infty}\) which will indeed have no strict upper bound.

    We now begin the rigorous proof.
    Suppose for sake of contradiction that every well-ordered subset \(Y\) of \(X\) which has \(x_0\) as its minimal element has at least one strict upper bound.
    Using the axiom of choice (in the form of Proposition \ref{8.4.7}), we can thus assign a strict upper bound \(s(Y) \in X\) to each well-ordered subset \(Y\) of \(X\) which has \(x_0\) as its minimal element.

    Henceforth we fix a single such strict upper bound function \(s\).
    Let us define a special class of subsets \(Y\) of \(X\).
    We say that a subset \(Y\) of \(X\) is \emph{good} iff it is well-ordered, contains \(x_0\) as its minimal element, and obeys the property that
    \[
        x = s(\{y \in Y : y < x\}) \text{ for all } x \in Y \setminus \{x_0\}.
    \]
    Note that if \(x \in Y \setminus \{x_0\}\) then the set \(\{y \in Y : y < x\}\) is a subset of \(X\) which is well-ordered and contains \(x_0\) as its minimal element.
    Let \(\Omega \coloneqq \{Y \subseteq X : Y \text{ is good}\}\) be the collection of all good subsets of \(X\).
    This collection is not empty, since the subset \(\{x_0\}\) of \(X\) is clearly good
    (which is vacuously true).

    We make the following important observation:
    if \(Y\) and \(Y'\) are two good subsets of \(X\), then every element of \(Y' \setminus Y\) is a strict upper bound for \(Y\), and every element of \(Y \setminus Y'\) is a strict upper bound for \(Y'\).
    (Exercise \ref{ex 8.5.13}).
    In particular, given any two good sets \(Y\) and \(Y'\), at least one of \(Y' \setminus Y\) and \(Y \setminus Y'\) must be empty
    (since they are both strict upper bounds of each other).
    In other words, \(\Omega\) is totally ordered by set inclusion:
    given any two good sets \(Y\) and \(Y'\), either \(Y \subseteq Y'\) or \(Y' \subseteq Y\).

    Let \(Y_{\infty} = \bigcup \Omega\), i.e., \(Y_{\infty}\) is the set of all elements of \(X\) which belong to at least one good subset of \(X\).
    Clearly \(x_0 \in Y_{\infty}\).
    Also, since each good subset of \(X\) has \(x_0\) as its minimal element, the set \(Y_{\infty}\) also has \(x_0\) as its minimal element.

    Next, we show that \(Y_{\infty}\) is totally ordered.
    Let \(x, x'\) be two elements of \(Y_{\infty}\).
    By definition of \(Y_{\infty}\), we know that \(x\) lies in some good set \(Y\) and \(x'\) lies in some good set \(Y'\).
    But since \(\Omega\) is totally ordered, one of these good sets contains the other.
    Thus \(x, x'\) are contained in a single good set (either \(Y\) or \(Y'\));
    since good sets are totally ordered, we thus see that either \(x \leq x'\) or \(x' \leq x\) as desired.

    Next, we show that \(Y_{\infty}\) is well-ordered.
    Let \(A\) be any non-empty subset of \(Y_{\infty}\).
    Then we can pick an element \(a \in A\), which then lies in \(Y_{\infty}\).
    Therefore there is a good set \(Y\) such that \(a \in Y\).
    Then \(A \cap Y\) is a non-empty subset of \(Y\);
    since \(Y\) is well-ordered, the set \(A \cap Y\) thus has a minimal element, call it \(b\).
    Now recall that for any other good set \(Y'\), every element of \(Y' \setminus Y\) is a strict upper bound for \(Y\), and in particular is larger than \(b\).
    Since \(b\) is a minimal element of \(A \cap Y\), this implies that \(b\) is also a minimal element of \(A \cap Y'\) for any good set \(Y'\) with \(A \cap Y' \neq \emptyset\).
    This is true since
    \begin{align*}
                 & \forall\ y_1 \in Y, y_1 < \min(Y' \setminus Y)                                                    \\
        \implies & y_1 \in Y \cap Y'                                                                                 \\
        \implies & \forall\ y_2 \in A \cap Y, \big(y_2 < \min(Y' \setminus Y)\big) \land \big(y_2 \in Y \cap Y'\big) \\
        \implies & y_2 \in A \cap Y'                                                                                 \\
        \implies & b = \min(A \cap Y) = \min(A \cap Y').
    \end{align*}
    Since every element of \(A\) belongs to \(Y_{\infty}\) and hence belongs to at least one good set \(Y'\), we thus see that \(b\) is a minimal element of \(A\).
    Thus \(Y_{\infty}\) is well-ordered as claimed.

    Since \(Y_{\infty}\) is well-ordered with \(x_0\) as its minimal element, it has a strict upper bound \(s(Y_{\infty})\).
    But then \(Y_{\infty} \cup \{s(Y_{\infty})\}\) is well-ordered (by Exercise \ref{ex 8.5.11}) and has \(x_0\) as its minimal element.
    We now claim that \(Y_{\infty} \cup \{s(Y_{\infty})\}\) is good.
    By the preceding discussion, it suffices to show that \(x = s\big(\big\{y \in Y_{\infty} \cup \{s(Y_{\infty})\} : y < x\big\}\big)\) when \(x \in \big(Y_{\infty} \cup \{s(Y_{\infty})\}\big) \setminus \{x_0\}\).
    If \(x = s(Y_{\infty})\) this is clear since \(\big\{y \in Y_{\infty} \cup \{s(Y_{\infty})\} : y < x\big\} = Y_{\infty}\) in this case.
    If instead \(x \in Y_{\infty}\), then \(x \in Y\) for some good \(Y\).
    Then the set \(\big\{y \in Y_{\infty} \cup \{s(Y_{\infty})\}: y < x\big\}\) is equal to \(\{y \in Y : y < x\}\)
    (why? use the previous observation that every element of \(Y' \setminus Y\) is an upper bound for \(x\) for every good \(Y'\)), and the claim then follows since \(Y\) is good.
    By definition of \(Y_{\infty}\), we conclude that the good set \(Y_{\infty} \cup \{s(Y_{\infty})\}\) is contained in \(Y_{\infty}\).
    But this is a contradiction since \(s(Y_{\infty})\) is a strict upper bound for \(Y_{\infty}\).
    Thus we have constructed a set with no strict upper bound, as desired.
\end{proof}

\begin{lemma}[Zorn's lemma]\label{8.5.15}
    Let \(X\) be a non-empty partially ordered set, with the property that every totally ordered subset \(Y\) of \(X\) has an upper bound.
    Then \(X\) contains at least one maximal element.
\end{lemma}

\begin{proof}
    Suppose for sake of contradiction that \(X\) has no maximal element.
    We show that any subset \(Y \subseteq X\) which has an upper bound also has a strict upper bound.
    Let \(s\) be an upper bound of \(Y\).
    By Definition \ref{8.5.12}, we know that \(s \in X\) and \(\forall\ y \in Y \implies y \leq s\).
    Since \(X\) has no maximal element, we know \(\exists\ s' \in X\) such that \(s < s'\), otherwise \(s\) is an maximal element of \(X\).
    By Definition \ref{8.5.5}, we know that \(s' \notin Y\), and thus by Definition \ref{8.5.12} \(s'\) is a strict upper bound of \(Y\).

    Since \(X \neq \emptyset\), let \(x_0 \in X\).
    By Lemma \ref{8.5.14}, \(\exists\ Y \subseteq X\) such that \(Y\) is well-ordered, \(\min(Y) = x_0\) and \(Y\) has no strict upper bound.
    But by hypothesis we know that \(Y\) has an upper bound and thus has a strict upper bound, a contradiction.
    Thus \(X\) must has at least one maximal element.
\end{proof}

\begin{note}
    Zorn's lemma is also called the \emph{principle of transfinite induction}.
\end{note}

\exercisesection

\begin{exercise}\label{ex 8.5.1}
    Consider the empty set \(\emptyset\) with the empty order relation \(\leq_\emptyset\)
    (this relation is vacuous because the empty set has no elements).
    Is this set partially ordered? totally ordered? well-ordered? Explain.
\end{exercise}

\begin{proof}
    Since
    \[
        \forall\ x \in \emptyset, x \leq_{\emptyset} x
    \]
    is vacuously true, \(\leq_{\emptyset}\) is reflexive.
    Since
    \[
        \forall\ x, y \in \emptyset, (x \leq_{\emptyset} y) \land (y \leq_{\emptyset} x) \implies x = y
    \]
    is vacuously true, \(\leq_{\emptyset}\) is anti-symmetry.
    Since
    \[
        \forall\ x, y, z \in \emptyset, (x \leq_{\emptyset} y) \land (y \leq_{\emptyset} z) \implies x \leq_{\emptyset} z
    \]
    is vacuously true, \(\leq_{\emptyset}\) is transitive.
    Since \((\emptyset, \leq_{\emptyset})\) is reflexive, anti-symmetric and transitive, by Definition \ref{8.5.1} \((\emptyset, \leq_{\emptyset})\) is partially ordered.
    Since
    \[
        \forall\ x, y \in \emptyset, (x \leq_{\emptyset} y) \lor (y \leq_{\emptyset} x)
    \]
    is vacuously true, by Definition \ref{8.5.3} we know that \((\emptyset, \leq_{\emptyset})\) is totally ordered.
    Since
    \[
        \forall\ X \subseteq \emptyset, X \neq \emptyset \implies \exists\ \min(X) \in X
    \]
    is vacuously true, by Definition \ref{8.5.8} we know that \((\emptyset, \leq_{\emptyset})\) is well-ordered.
\end{proof}

\begin{exercise}\label{ex 8.5.2}
    Give examples of a set \(X\) and a relation \(\leq_X\) such that
    \begin{enumerate}
        \item The relation \(\leq_X\) is reflexive and anti-symmetric, but not transitive;
        \item The relation \(\leq_X\) is reflexive and transitive, but not anti-symmetric;
        \item The relation \(\leq_X\) is anti-symmetric and transitive, but not reflexive.
    \end{enumerate}
\end{exercise}

\begin{proof}
    \begin{enumerate}
        \item Let \(X = \{1, 2, 4\}\) be a set and let \(\leq_X\) be the relation
              \[
                  \forall\ a, b \in X, a \leq_X b \iff (a = b) \lor (2a = b).
              \]
              Since
              \[
                  \forall\ a \in X, a = a \implies a \leq_X a,
              \]
              we know that \((X, \leq_X)\) is reflexive.
              Since
              \[
                  (1 \leq_X 1) \land (1 \leq_X 2) \land (2 \leq_X 2) \land (2 \leq_X 4) \land (4 \leq_X 4),
              \]
              we know that when pairs of \(a, b \in X\) satisfing \(a \leq_X b \land b \leq_X a\) we must have \(a = b\), thus \((X, \leq_X)\) is anti-symmetric.
              Since we have \((1 \leq_X 2) \land (2 \leq_X 4)\) but not \(1 \leq_X 4\), we know that \((X, \leq_X)\) is not transitive.
        \item Let \(X = \mathbf{Z}\) be a set and let \(\leq_X\) be the relation
              \[
                  \forall\ a, b \in \mathbf{Z}, a \leq_X b \iff \abs*{a} \leq \abs*{b}.
              \]
              Since
              \[
                  \forall\ a \in \mathbf{Z}, \abs*{a} \leq \abs*{a} \implies a \leq_X a,
              \]
              we know that \((\mathbf{Z}, \leq_X)\) is reflexive.
              Since
              \[
                  (\abs*{1} \leq \abs*{-1}) \land (\abs*{-1} \leq \abs*{1}) \implies (1 \leq_X -1) \land (-1 \leq_X 1)
              \]
              but \(1 \neq -1\), we know that \((\mathbf{Z}, \leq_X)\) is not anti-symmetric.
              Since
              \begin{align*}
                           & \forall\ a, b, c \in \mathbf{Z}, (a \leq_X b) \land (b \leq_X c) \\
                  \implies & (\abs*{a} \leq \abs*{b}) \land (\abs*{b} \leq \abs*{c})          \\
                  \implies & \abs*{a} \leq \abs*{c}                                           \\
                  \implies & a \leq_X c,
              \end{align*}
              we know that \((\mathbf{Z}, \leq_X)\) is transitive.
        \item Let \(X = \{0, 1\}\) be a set and let \(\leq_X\) be the relation
              \[
                  \forall\ a, b \in X, a \leq_X b \iff a \leq a + 1.
              \]
              Since
              \[
                  \forall\ a \in X, a + 1 \not\leq a,
              \]
              we know that \((X, \leq_X)\) is not reflexive.
              Since
              \[
                  \forall\ a, b \in X, (a \leq_X b) \land (b \leq_X a) \implies a = b
              \]
              is vacuously true, we know that \((X, \leq_X)\) is anti-symmetric.
              Since
              \[
                  \forall\ a, b \in X, (a \leq_X b) \land (b \leq_X c) \implies a \leq_X b
              \]
              is vacuously true, we know that \((X, \leq_X)\) is transitive.
    \end{enumerate}
\end{proof}

\begin{exercise}\label{ex 8.5.3}
    Given two positive integers \(n, m \in \mathbf{N} \setminus \{0\}\), we say that \emph{\(n\) divides \(m\)}, and write \(n | m\), if there exists a positive integer \(a\) such that \(m = na\).
    Show that the set \(\mathbf{N} \setminus \{0\}\) with the ordering relation \(|\) is a partially ordered set but not a totally ordered one.
    Note that this is a different ordering relation from the usual \(\leq\) ordering of \(\mathbf{N} \setminus \{0\}\).
\end{exercise}

\begin{proof}
    Since
    \[
        \forall\ n \in \mathbf{N} \setminus \{0\}, n = 1n \implies n | n,
    \]
    we know that \((X, |)\) is reflexive.
    Since
    \begin{align*}
                 & \forall\ n, m \in \mathbf{N} \setminus \{0\}, (n | m) \land (m | n)   \\
        \implies & \exists\ a, b \in \mathbf{N} \setminus \{0\}, (m = na) \land (n = bm) \\
        \implies & n = abn                                                               \\
        \implies & ab = 1                                                                \\
        \implies & (a = 1) \land (b = 1)                                                 \\
        \implies & n = m,
    \end{align*}
    we know that \((X, |)\) is anti-symmetric.
    Since
    \begin{align*}
                 & \forall\ n, m, p \in \mathbf{N} \setminus \{0\}, (n | m) \land (m | p) \\
        \implies & \exists\ a, b \in \mathbf{N} \setminus \{0\}, (m = na) \land (p = bm)  \\
        \implies & (p = abn) \land (ab > 0)                                               \\
        \implies & n | p,
    \end{align*}
    we know that \((X, |)\) is transitive.
    Since \((X, |)\) is reflexive, anti-symmetric and transitive, by Definition \ref{8.5.1} \((X, |)\) is partially ordered.
    Since \((2 | 3) \lor (3 | 2)\) is false, by Definition \ref{8.5.3} \((X, |)\) is not totally ordered.
\end{proof}

\begin{exercise}\label{ex 8.5.4}
    Show that the set of positive reals \(\mathbf{R}^+ \coloneqq \{x \in \mathbf{R} : x > 0\}\) have no minimal element.
\end{exercise}

\begin{proof}
    Suppose for sake of contradiction that \(\exists\ x \in \mathbf{R}^+\) such that \(x = \min(\mathbf{R}^+)\).
    Since \(x > 0\), we know that \(x / 2 > 0\) and \(x / 2 \in \mathbf{R}^+\).
    But \(x = \min(\mathbf{R}^+)\) implies we have \(x < x / 2\), a contradiction.
    Thus \(\nexists\ x \in \mathbf{R}^+\) such that \(x = \min(\mathbf{R}^+)\).
\end{proof}

\begin{exercise}\label{ex 8.5.5}
    Let \(f : X \to Y\) be a function from one set \(X\) to another set \(Y\).
    Suppose that \(Y\) is partially ordered with some ordering relation \(\leq_Y\).
    Define a relation \(\leq_X\) on \(X\) by defining \(x \leq_X x'\) if and only if \(f(x) <_Y f(x')\) or \(x = x'\).
    Show that this relation \(\leq_X\) turns \(X\) into a partially ordered set.
    If we know in addition that the relation \(\leq_Y\) makes \(Y\) totally ordered, does this mean that the relation \(\leq_X\) makes \(X\) totally ordered also?
    If not, what additional assumption needs to be made on \(f\) in order to ensure that \(\leq_X\) makes \(X\) totally ordered?
\end{exercise}

\begin{proof}
    We first show that \((X, \leq_X)\) is partially ordered.
    Since
    \[
        \forall\ x \in X, x = x \implies x \leq_X x,
    \]
    we know that \((X, \leq_X)\) is reflexive.
    Since
    \begin{align*}
                 & \forall\ x, x' \in X, (x \leq_X x') \land (x' \leq_X x)                                                                 \\
        \implies & \Big(\big(f(x) <_Y f(x')\big) \lor \big(x = x'\big)\Big) \land \Big(\big(f(x') <_Y f(x)\big) \lor \big(x = x'\big)\Big) \\
        \implies & x = x',
    \end{align*}
    we know that \((X, \leq_X)\) is anti-symmetric.
    Since
    \begin{align*}
                 & \forall\ x_1, x_2, x_3 \in X, (x_1 \leq_X x_2) \land (x_2 \leq_X x_3)                                                               \\
        \implies & \Big(\big(f(x_1) <_Y f(x_2)\big) \lor \big(x_1 = x_2\big)\Big) \land \Big(\big(f(x_2) <_Y f(x_3)\big) \lor \big(x_2 = x_3\big)\Big) \\
        \implies & \bigg(\Big(\big(f(x_1) <_Y f(x_2)\big) \lor \big(x_1 = x_2\big)\Big) \land \big(f(x_2) <_Y f(x_3)\big)\bigg)                        \\
                 & \lor \bigg(\Big(\big(f(x_1) <_Y f(x_2)\big) \lor \big(x_1 = x_2\big)\Big) \land \big(x_2 = x_3\big)\bigg)                           \\
        \implies & \big(f(x_1) <_Y f(x_2) <_Y f(x_3)\big)                                                                                              \\
                 & \lor \Big(\big(x_1 = x_2\big) \land \big(f(x_1) <_Y f(x_3)\big)\Big)                                                                \\
                 & \lor \Big(\big(f(x_1) <_Y f(x_2)\big) \land \big(x_2 = x_3\big)\Big)                                                                \\
                 & \lor \big(x_1 = x_2 = x_3\big)                                                                                                      \\
        \implies & \big(f(x_1) <_Y f(x_3)\big)                                                                                                         \\
                 & \lor \big(f(x_1) <_Y f(x_3)\big)                                                                                                    \\
                 & \lor \big(f(x_1) <_Y f(x_3)\big)                                                                                                    \\
                 & \lor \big(x_1 = x_3\big)                                                                                                            \\
        \implies & x_1 \leq_X x_3,
    \end{align*}
    we know that \((X, \leq_X)\) is transitive.
    Since \((X, \leq_X)\) is reflexive, anti-symmetric and transitive, by Definition \ref{8.5.1} \((X, \leq_X)\) is partially ordered.

    Next we show that \((X, \leq_X)\) may not be totally ordered when \((Y, \leq_Y)\) is totally ordered.
    If \(\exists\ x, x' \in X : x \neq x' \implies f(x) = f(x')\), then we do not have the ordering relation \(x \leq_X x'\) and \(x' \leq_X x\).
    Thus by Definition \ref{8.5.3} \((X, \leq_X)\) is not totally ordered.

    Next we show that if \(f : X \to Y\) is injective and \((Y, \leq_Y)\) is totally ordered, then \((X, \leq_X)\) is totally ordered.
    Since \(f\) is injective, we know that \(\forall\ x, x' \in X, x \neq x' \implies f(x) \neq f(x')\).
    Since \(Y\) is totally ordered, we know that \(\big(f(x) \leq_Y f(x')\big) \lor \big(f(x') \leq_Y f(x)\big)\) is true.
    Since \(f(x) \neq f(x')\), we know that exactly one of \(\big(f(x) <_Y f(x')\big) \lor \big(f(x') <_Y f(x)\big)\) is true.
    In either cases we get exactly one of \((x \leq_X x') \lor (x' \leq_X x)\) is true.
    Thus by Definition \ref{8.5.3} \((X, \leq_X)\) is totally ordered.
\end{proof}

\begin{exercise}\label{ex 8.5.6}
    Let \(X\) be a partially ordered set.
    For any \(x\) in \(X\), define the \emph{order ideal} \((x) \subseteq X\) to be the set \((x) \coloneqq \{y \in X : y \leq_X x\}\).
    Let \((X) \coloneqq \{(x) : x \in X\}\) be the set of all order ideals, and let \(f : X \to (X)\) be the map \(f(x) \coloneqq (x)\) that sends every element of \(x\) to its order ideal.
    Show that \(f\) is a bijection, and that given any \(x, y \in X\), that \(x \leq_X y\) if and only if \(f(x) \subseteq f(y)\).
    This exercise shows that any partially ordered set can be \emph{represented} by a collection of sets whose ordering relation is given by set inclusion.
\end{exercise}

\begin{proof}
    We first show that \(f\) is bijective.
    We start by showing \(f\) is injective.
    Since \((X, \leq_X)\) is partially ordered, by Definition \ref{8.5.1} we know that
    \begin{align*}
                 & \forall\ x, x' \in X, f(x) = f(x')                                                           \\
        \implies & (x) = (x')                                                                                   \\
        \implies & \big(x \in (x')\big) \land \big(x' \in (x)\big) & \text{(\((X, \leq_X)\) is reflexive)}      \\
        \implies & (x \leq_X x') \land (x' \leq_X x')                                                           \\
        \implies & x = x'.                                         & \text{(\((X, \leq_X)\) is anti-symmetric)}
    \end{align*}
    Thus \(f\) is injective.
    Next we show that \(f\) is surjective.
    \begin{align*}
                 & \forall\ (x) \in (X), x \in X \\
        \implies & f(x) = (x).
    \end{align*}
    Thus \(f\) is surjective.
    Since \(f\) is both injective and surjective, we know that \(f\) is bijective.

    Finally we show that \(\forall\ x, y \in X, x \leq_X y \iff f(x) \subseteq f(y)\).
    \begin{align*}
             & \forall\ x, y \in X, x \leq_X y                                                             \\
        \iff & \forall\ z \in X, (z \leq_X x \implies z \leq_X y) & \text{(\((X, \leq_X)\) is transitive)} \\
        \iff & (x) \subseteq (y)                                                                           \\
        \iff & f(x) \subseteq f(y).
    \end{align*}
\end{proof}

\begin{exercise}\label{ex 8.5.7}
    Let \(X\) be a partially ordered set with ordering relation \(\leq_X\), and let \(Y\) be a totally ordered subset of \(X\).
    Show that \(Y\) can have at most one maximum and at most one minimum.
\end{exercise}

\begin{proof}
    Suppose for sake of contradiction that \(\exists\ y, y' \in Y\) such that \(y \neq y'\) and both \(y, y'\) are maximum of \(Y\).
    Then by Definition \ref{8.5.5} we have \((y \leq_X y') \land (y' \leq_X y)\), so \(y = y'\), a contradiction.
    Thus \(Y\) can have at most one maximum.
    Similar arguments show that \(Y\) can have at most one minimum.
\end{proof}

\begin{exercise}\label{ex 8.5.8}
    Show that every finite non-empty subset of a totally ordered set has a minimum and a maximum.
    Conclude in particular that every finite totally ordered set is well-ordered.
\end{exercise}

\begin{proof}
    Let \(X \neq \emptyset\) be a totally ordered set with ordering relation \(\leq_X\) and let \(Y \subseteq X\) be a finite set where \(Y \neq \emptyset\).
    Let \(n = \#(Y)\).
    We use induction on \(n\) to show that \(Y\) has a minimum and maximum.
    For \(n = 1\), let \(y \in Y\).
    Then \(\forall\ y' \in Y\), we have \(y \leq_X y'\) and \(y' \leq_X y\).
    Thus by Definition \ref{8.5.5} \(y\) is the minimum and the maximum of \(Y\), and the base case holds.
    Suppose inductively that for some \(n \geq 1\) we know that \(Y\) has a minimum and maximum.
    Then for \(n + 1\), let \(y \in Y\).
    By induction hypothesis we know that \(Y \setminus \{y\}\) has a minimum and maximum.
    Let \(y_l = \min(Y \setminus \{y\})\).
    Since \(y \notin Y \setminus \{y\}\), we know that \(y \neq y_l\).
    Since \(Y \subseteq X\), we know that \(Y\) is totally ordered, and thus we have \((y <_X y_l) \lor (y_l <_X y)\).
    Now we split into two cases:
    \begin{itemize}
        \item If \(y <_X y_l\), then \(y\) is the minimum of \(Y\) since
              \begin{align*}
                           & \forall\ y' \in Y                                                                                          \\
                  \implies & (y' = y) \lor (y' \in Y \setminus \{y\})                                                                   \\
                  \implies & (y' = y) \lor (y_l <_X y')               & \text{(by Definition \ref{8.5.5})}                              \\
                  \implies & y \leq_X y'.                             & \text{(by Definition \ref{8.5.1}, \(X, \leq_X\) is transitive)}
              \end{align*}
        \item If \(y_l <_X y\), then \(y_l\) is the minimum of \(Y\) since \(\forall\ y' \in Y, y_l \leq_X y'\).
    \end{itemize}
    From all cases above we conclude that \(Y\) has a minimum.
    Similar arguments show that \(Y\) also has a maximum.
    This close the induction.
\end{proof}

\begin{exercise}\label{ex 8.5.9}
    Let \(X\) be a totally ordered set such that every non-empty subset of \(X\) has both a minimum and a maximum.
    Show that \(X\) is finite.
\end{exercise}

\begin{proof}
    Let \(X\) be a totally ordered set with ordering relation \(\leq_X\) where every non-empty subset of \(X\) has both a minimum and a maximum.
    Suppose for sake of contradiction that \(X\) is infinite.
    Since \(X \subseteq X\), by hypothesis \(X\) has both a minimum and a maximum.
    Define \(x_n\) recursively as follow
    \[
        \forall\ n \in \mathbf{N}, x_n = \begin{cases}
            \min(X)                                           & \text{if } n = 0 \\
            \min(X \setminus \bigcup_{m = 0}^{n - 1} \{x_m\}) & \text{if } n > 0
        \end{cases}
    \]
    Then we have an strictly increasing sequence \((x_n)_{n = 0}^\infty\).
    Let \(X_n = \{x_n : n \in \mathbf{N}\}\) be the set of all elements in sequence \((x_n)_{n = 0}^\infty\).
    We know that \(X_n \subseteq X\) and \(X_n \neq \emptyset\).
    By hypothesis we know that \(\exists\ \max(X_n) \in X_n\).
    Let \(m \in \mathbf{N}\) be the index of the maximum in \(X_n\), i.e., \(x_m = \min(X_n)\).
    But \((x_n)_{n = 0}^\infty\) is an strictly increasing sequence, so we have \(x_m <_X x_{m + 1}\) and \(x_m\) is not a maximum, a contradiction.
    Thus \(X\) must be finite.
\end{proof}

\begin{exercise}\label{ex 8.5.10}
    Prove Proposition \ref{8.5.10}, without using the axiom of choice.
\end{exercise}

\begin{proof}
    See Proposition \ref{8.5.10}.
\end{proof}

\begin{exercise}\label{ex 8.5.11}
    Let \(X\) be a partially ordered set with ordering relation \(\leq_X\), and let \(Y\) and \(Y'\) be well-ordered subsets of \(X\).
    Show that \(Y \cup Y'\) is well-ordered if and only if it is totally ordered.
\end{exercise}

\begin{proof}
    By Definition \ref{8.5.8} we know that if \(Y \cup Y'\) is well-ordered, then \(Y \cup Y'\) is totally ordered.
    So we only need to show that if \(Y \cup Y'\) is totally ordered, then \(Y \cup Y'\) is well-ordered.
    Suppose that \(Y \cup Y'\) is totally ordered.
    Let \(Z \subseteq Y \cup Y' \land Z \neq \emptyset\), let \(Z_Y = Z \cap Y\) and let \(Z_{Y'} = Z \cap Y'\).
    If \(Z_Y = \emptyset \lor Z_{Y'} = \emptyset\), then \(Z \subseteq Y' \lor Z \subseteq Y\) and \(Z\) is well-ordered.
    So suppose that \(Z_Y \neq \emptyset \land Z_{Y'} \neq \emptyset\).
    Since \(Y, Y'\) are well-ordered, by Definition \ref{8.5.8} \(Z_Y, Z_{Y'}\) are also well-ordered.
    Let \(z_Y = \min(Z_Y)\) and \(z_{Y'} = \min(Z_{Y'})\).
    Since \(Y \cup Y'\) is totally order, we know that \((z_Y \leq_X z_{Y'}) \lor (z_{Y'} \leq_X z_Y)\).
    Now we split into two cases:
    \begin{itemize}
        \item If \(z_Y \leq_X z_{Y'}\), then
              \begin{align*}
                           & \forall\ z \in Z                                                                      \\
                  \implies & z \in Y \cup Y'                                                                       \\
                  \implies & z \in Y \lor z \in Y'                                                                 \\
                  \implies & (z_Y \leq_X z) \lor (z_{Y'} \leq_X z)            & \text{(by Definition \ref{8.5.5})} \\
                  \implies & (z_Y \leq_X z) \lor (z_Y \leq_X z_{Y'} \leq_X z) & \text{(by Definition \ref{8.5.1})} \\
                  \implies & z_Y \leq_X z.
              \end{align*}
              Thus by Definition \ref{8.5.5} \(z_Y = \min(Z)\).
        \item If \(z_{Y'} \leq_X z_Y\), then using similar arguments above we know that \(z_{Y'} = \min(Z)\).
    \end{itemize}
    From all cases above we conclude that \(\min(Z)\) exists.
    Since \(Z\) is arbitrary, by Definition \ref{8.5.8} \(Y \cup Y'\) is well-ordered.
\end{proof}

\begin{exercise}\label{ex 8.5.12}
    Let \(X\) and \(Y\) be partially ordered sets with ordering relations \(\leq_X\) and \(\leq_Y\) respectively.
    Define a relation \(\leq_{X \times Y}\) on the Cartesian product \(X \times Y\) by defining \((x, y) \leq_{X \times Y} (x', y')\) if \(x <_X x'\), or if \(x = x'\) and \(y \leq_Y y'\).
    (This is called the \emph{lexicographical ordering} on \(X \times Y\), and is similar to the alphabetical ordering of words;
    a word \(w\) appears earlier in a dictionary than another word \(w'\) if the first letter of \(w\) is earlier in the alphabet than the first letter of \(w'\), or if the first letters match and the second letter of \(w\) is earlier than the second letter of \(w'\), and so forth.)
    Show that \(\leq_{X \times Y}\) defines a partial ordering on \(X \times Y\).
    Furthermore, show that if \(X\) and \(Y\) are totally ordered, then so is \(X \times Y\), and if \(X\) and \(Y\) are well-ordered, then so is \(X \times Y\).
\end{exercise}

\begin{proof}
    We first show that \((X \times Y, \leq_{X \times Y})\) is partially ordered.
    If \(X = \emptyset \lor Y = \emptyset\), then by Exercise \ref{ex 3.5.8} \(X \times Y = \emptyset\) and by Exercise \ref{ex 8.5.1} we know that \(\emptyset\) is partially ordered.
    So suppose that \(X \neq \emptyset \land Y \neq \emptyset\).
    Since
    \[
        \forall\ (x, y) \in X \times Y, (x, y) = (x, y) \implies (x, y) \leq_{X \times Y} (x, y),
    \]
    we know that \((X \times Y, \leq_{X \times Y})\) is reflexive.
    Since
    \begin{align*}
                 & \forall\ (x, y), (x', y') \in X \times Y,                                                                                          \\
                 & \big((x, y) \leq_{X \times Y} (x', y')\big) \land \big((x', y') \leq_{X \times Y} (x, y)\big)                                      \\
        \implies & \Big((x <_X x') \lor \big((x = x') \land (y \leq_Y y')\big)\Big)                                                                   \\
                 & \land \Big((x' <_X x) \lor \big((x = x') \land (y' \leq_Y y)\big)\Big)                                                             \\
        \implies & \Big((x \leq_X x') \land \big((x <_X x') \lor (y \leq_Y y')\big)\Big)                                                              \\
                 & \land \Big((x' \leq_X x) \land \big((x' <_X x) \lor (y' \leq_Y y)\big)\Big)                                                        \\
        \implies & (x = x') \land (y \leq_Y y') \land (y' \leq_Y y)                                                                                   \\
        \implies & (x = x') \land (y = y')                                                                       & \text{(by Definition \ref{8.5.1})} \\
        \implies & (x, y) = (x', y'),                                                                            & \text{(by Definition \ref{3.5.1})}
    \end{align*}
    we know that \((X \times Y, \leq_{X \times Y})\) is anti-symmetric.
    Since
    \begin{align*}
                 & \forall\ (x_1, y_1), (x_2, y_2), (x_3, y_3) \in X \times Y :                                                                                   \\
                 & \big((x_1, y_1) \leq_{X \times Y} (x_2, y_2)\big) \land \big((x_2, y_2) \leq_{X \times Y} (x_3, y_3)\big)                                      \\
        \implies & \Big((x_1 <_X x_2) \lor \big((x_1 = x_2) \land (y_1 \leq_Y y_2)\big)\Big)                                                                      \\
                 & \land \Big((x_2 <_X x_3) \lor \big((x_2 = x_3) \land (y_2 \leq_Y y_3)\big)\Big)                                                                \\
        \implies & (x_1 <_X x_2 <_X x_3)                                                                                     & \text{(by Definition \ref{8.5.1})} \\
                 & \lor \big((x_1 = x_2 <_X x_3) \land (y_1 \leq_Y y_2)\big)                                                                                      \\
                 & \lor \big((x_1 <_X x_2 = x_3) \land (y_2 \leq_Y y_3)\big)                                                                                      \\
                 & \lor \big((x_1 = x_2 = x_3) \land (y_1 \leq_Y y_2 \leq_Y y_3)\big)                                        & \text{(by Definition \ref{8.5.1})} \\
        \implies & (x_1, y_1) \leq_{X \times Y} (x_3, y_3),
    \end{align*}
    we know that \((X \times Y, \leq_{X \times Y})\) is transitive.
    Since \((X \times Y, \leq_{X \times Y})\) is reflexive, anti-symmetric and transitive, by Definition \ref{8.5.1} \((X \times Y, \leq_{X \times Y})\) is partially ordered.

    Now we show that if \((X, \leq_X), (Y, \leq_Y)\) are totally ordered, then \((X \times Y, \leq_{X \times Y})\) is totally ordered.
    If \(X = \emptyset \lor Y = \emptyset\), then by Exercise \ref{ex 3.5.8} \(X \times Y = \emptyset\) and by Exercise \ref{ex 8.5.1} we know that \(\emptyset\) is totally ordered.
    So suppose that \(X \neq \emptyset \land Y \neq \emptyset\).
    From prove above we know that \((X \times Y, \leq_{X \times Y})\) is partially ordered.
    Since
    \begin{align*}
                 & \forall\ (x, y), (x', y') \in X \times Y, (x, y) \neq (x', y')                                      \\
        \implies & (x \neq x') \lor \big((x = x') \land (y \neq y')\big),         & \text{(by Definition \ref{3.5.1})}
    \end{align*}
    we split into two cases:
    \begin{itemize}
        \item If \(x \neq x'\), then \((x <_X x') \lor (x' <_X x)\) since \(X\) is totally ordered.
              Thus we have \(\big((x, y) \leq_{X \times Y} (x', y')\big) \lor \big((x, y) \leq_{X \times Y} (x', y')\big)\).
        \item If \((x = x') \land (y \neq y')\), then \((y <_Y y') \lor (y' <_Y y)\) since \(Y\) is totally ordered.
              Thus we have \(\big((x, y) \leq_{X \times Y} (x', y')\big) \lor \big((x, y) \leq_{X \times Y} (x', y')\big)\).
    \end{itemize}
    From all cases above we conclude that \(\big((x, y) \leq_{X \times Y} (x', y')\big) \lor \big((x, y) \leq_{X \times Y} (x', y')\big)\).
    Thus \((X \times Y, \leq_{X \times Y})\) is totally ordered.

    Now we show that if \((X, \leq_X), (Y, \leq_Y)\) are well-ordered, then \((X \times Y, \leq_{X \times Y})\) is well-ordered.
    If \(X = \emptyset \lor Y = \emptyset\), then by Exercise \ref{ex 3.5.8} \(X \times Y = \emptyset\) and by Exercise \ref{ex 8.5.1} we know that \(\emptyset\) is well-ordered.
    So suppose that \(X \neq \emptyset \land Y \neq \emptyset\).
    From prove above we know that \((X \times Y, \leq_{X \times Y})\) is totally ordered.
    Let \(Z \subseteq X \times Y\) and \(Z \neq \emptyset\), let \(Z_X = \{x \in X | \exists\ y \in Y : (x, y) \in Z\}\).
    Since \(X\) is well-ordered and \(Z_X \subseteq X\), by Definition \ref{8.5.8} we know that \(z_x = \min(Z_X)\) exists.
    Since \(z_x \in Z_X\), we know that the set \(Z_Y = \{y \in Y : (z_x, y) \in Z\}\) is not empty.
    Since \(Y\) is well-ordered and \(Z_Y \subseteq Y\), by Definition \ref{8.5.8} we know that \(z_y = \min(Z_Y)\) exists.
    Then we have
    \begin{align*}
                 & \forall\ (x, y) \in Z                                                                       \\
        \implies & (z_x <_X x) \lor (x = z_x)                                                                  \\
        \implies & \big((z_x, z_y) \leq_{X \times Y} (x, y)\big) \lor (x = z_x)                                \\
        \implies & \big((z_x, z_y) \leq_{X \times Y} (x, y)\big) \lor \big((x = z_x) \land (y \in Z_y)\big)    \\
        \implies & \big((z_x, z_y) \leq_{X \times Y} (x, y)\big) \lor \big((x = z_x) \land (z_y \leq_Y y)\big) \\
        \implies & (z_x, z_y) \leq_{X \times Y} (x, y).
    \end{align*}
    Thus \(\min(Z) = (z_x, z_y)\).
    Since \(Z\) is arbitrary, by Definition \ref{8.5.8} we know that \((X \times Y, \leq_{X \times Y})\) is well-ordered.
\end{proof}

\begin{exercise}\label{ex 8.5.13}
    Prove the claim in the proof of Lemma \ref{8.5.14}, namely that every element of \(Y' \setminus Y\) is an strict upper bound for \(Y\) and vice versa.
\end{exercise}

\begin{proof}
    Since \(Y, Y'\) are good, we know that \(x_0 \in Y \cap Y'\) and thus \(Y \cap Y' \neq \emptyset\).
    Let \(n \in Y \cap Y'\) and let \(P(n)\) be the statement define as follow:
    \[
        \{y \in Y : y \leq n\} = \{y \in Y' : y \leq n\} = \{y \in Y \cap Y' : y \leq n\}.
    \]
    Let \(Q(n)\) be the statement ``if \(P(m)\) is true for every \(m \in Y \cap Y'\) and \(m < n\), then \(P(n)\) is true.''
    Note that if we can show that \(Q(n)\) is true for every \(n \in Y \cap Y'\), then by principle of strong induction (Proposition \ref{8.5.10}) we know that \(P(n)\) is true for every \(n \in Y \cap Y'\).

    Since \(\min(Y) = \min(Y') = x_0\), we know that \(\min(Y \cap Y') = x_0\) and
    \[
        \{y \in Y : y \leq x_0\} = \{y \in Y' : y \leq x_0\} = \{y \in Y \cap Y' : y \leq x_0\} = \{x_0\}.
    \]
    Thus \(Q(x_0)\) is vacuously true (there is no element \(z \in Y \cap Y'\) which satisfy \(z < x_0\)).
    Suppose inductively that \(Q(m)\) is true for some \(m \geq x_0\).
    We need to show that \(Q(n)\) is also true for the next smallest item \(n \in Y \cap Y'\).
    By smallest we mean that
    \[
        n = \min\big(\{n \in Y \cap Y' : Q(m) \text{ is true and } x_0 \leq m < n \text { for every } m \in Y \cap Y'\}\big).
    \]
    Such \(n\) is well-defined since in the proof of Lemma \ref{8.5.14} we suppose for sake of contradiction that every well-ordered subset \(Y\) of \(X\) which has \(x_0\) as its minimal element has at least one strict upper bound.
    We know that the set \(\{y \in Y : y \leq n\}\) is not empty since it contains \(x_0\), so let \(x \in \{y \in Y : y \leq n\}\).
    Now we split into two cases:
    \begin{itemize}
        \item If \(x = n\), then by the definition of \(n\) we know that \(x \in \{y \in Y \cap Y' : y \leq n\}\).
        \item If \(x < n\), then by the definition of \(n\) we know that by induction hypothesis \(Q(x)\) is true.
              Then we have
              \begin{align*}
                           & P(x) \text{ is true}                                    \\
                  \implies & \{y \in Y : y \leq x\} = \{y \in Y \cap Y' : y \leq x\} \\
                  \implies & x \in \{y \in Y \cap Y' : y \leq x < n\}.
              \end{align*}
    \end{itemize}
    From all cases above we conclude that \(\{y \in Y : y \leq n\} \subseteq \{y \in Y \cap Y' : y \leq n\}\).
    We also have \(\{y \in Y \cap Y' : y \leq n\} \subseteq \{y \in Y : y \leq n\}\) since
    \begin{align*}
                 & \forall\ x \in \{y \in Y \cap Y' : y \leq n\} \\
        \implies & (x \in Y \cap Y') \land (x \leq n)            \\
        \implies & (x \in Y) \land (x \leq n)                    \\
        \implies & x \in \{y \in Y : y \leq n\}.
    \end{align*}
    Thus we have \(\{y \in Y : y \leq n\} = \{y \in Y \cap Y' : y \leq n\}\).
    Using similar arguments as above we have \(\{y \in Y' : y \leq n\} = \{y \in Y \cap Y' : y \leq n\}\).
    Thus we conclude that \(P(n)\) is also true, i.e.,
    \[
        \{y \in Y : y \leq n\} = \{y \in Y' : y \leq n\} = \{y \in Y \cap Y' : y \leq n\}.
    \]
    By induction hypothesis we know that \(Q(m)\) is true for every \(m \in Y \cap Y'\) and \(m < n\).
    Thus we conclude that \(Q(n)\) is also true, this close the induction.

    By principle of strong induction (Proposition \ref{8.5.10}) we know that \(P(n)\) is true for all \(n \in Y \cap Y'\).
    Thus we have
    \begin{align*}
                 & \forall\ x \in (Y \cap Y') \setminus \{x_0\}, P(x) \text{ is true}                                               \\
        \implies & \{y \in Y \cap Y' : y \leq x\} = \{y \in Y : y \leq x\}                                                          \\
        \implies & \{y \in Y \cap Y' : y \leq x\} \setminus \{x\} = \{y \in Y : y \leq x\} \setminus \{x\}                          \\
        \implies & \{y \in Y \cap Y' : y < x\} = \{y \in Y : y < x\}                                                                \\
        \implies & s(\{y \in Y \cap Y' : y < x\}) = s(\{y \in Y : y < x\}) = x                             & \text{(\(Y\) is good)}
    \end{align*}
    and \(Y \cap Y'\) is good.

    Now we show that if \(Y \setminus Y' \neq \emptyset\), then \(s(Y \cap Y') = \min(Y \setminus Y')\).
    Since \(Y \cap Y'\) is good, \(s(Y \cap Y')\) is well-defined.
    Since \(Y\) is well-ordered and \(Y \setminus Y' \subseteq Y\), \(Y \setminus Y'\) is also well-ordered and thus \(\min(Y \setminus Y')\) is well-defined.
    We know that \(\forall\ y_1 \in Y\), \(y_1 < \min(Y \setminus Y') \implies y_1 \notin Y \setminus Y'\), otherwise \(y_1 = \min(Y \setminus Y')\), a contradiction.
    Thus \(y_1 \in Y \cap Y'\) and \(\{y \in Y : y < \min(Y \setminus Y')\} \subseteq Y \cap Y\).
    Similarly, we know that \(\forall\ y_2 \in Y \cap Y'\), \(y_2 \in Y\) and \(y_2 < \min(Y \setminus Y')\), so \(y_2 \in \{y \in Y : y < \min(Y \setminus Y')\}\) and \(Y \cap Y' \subseteq \{y \in Y : y < \min(Y \setminus Y')\}\).
    We now conclude that \(\{y \in Y : y < \min(Y \setminus Y')\} = Y \cap Y'\) and
    \begin{align*}
                 & \min(Y \setminus Y') \in Y \setminus \{x_0\}                                                      \\
        \implies & \min(Y \setminus Y') = s\big(\{y \in Y : y < \min(Y \setminus Y')\}\big) & \text{(\(Y\) is good)} \\
        \implies & \min(Y \setminus Y') = s(Y \cap Y').
    \end{align*}
    Similar arguments show that if \(Y' \setminus Y \neq \emptyset\), then \(s(Y \cap Y') = \min(Y' \setminus Y)\).

    Finally we show that every element of \(Y' \setminus Y\) is an upper bound for \(Y\) and vice versa.
    Since \((Y \setminus Y') \cap (Y' \setminus Y) = \emptyset\), we know that at least one of \(Y \setminus Y'\) or \(Y' \setminus Y\) is empty.
    Otherwise we have \(s(Y \cap Y') = \min(Y \setminus Y') = \min(Y' \setminus Y)\), which means \(\big(s(Y \cap Y') \in Y \setminus Y'\big) \land \big(s(Y \cap Y') \in Y' \setminus Y\big)\), a contradiction.
    Now we split into two cases:
    \begin{itemize}
        \item If \(Y \setminus Y' = \emptyset\), then it is vacuously true that every element of \(Y \setminus Y'\) is an strict upper bound of \(Y'\).
        \item If \(Y \setminus Y' \neq \emptyset\), then \(Y' \setminus Y = \emptyset\) and \(Y' \subseteq Y\).
              Since \(Y \cap Y' = Y'\) and \(s(Y \cap Y') = s(Y') = \min(Y \setminus Y')\), by Definition \ref{8.5.5} we know that every element of \(Y \setminus Y'\) is an strict upper bound of \(Y'\).
    \end{itemize}
    From all cases above we conclude that every element of \(Y \setminus Y'\) is an strict upper bound of \(Y'\).
    Using similar arguments we can show that every element of \(Y' \setminus Y\) is an strict upper bound of \(Y\).
\end{proof}

\begin{exercise}\label{ex 8.5.14}
    Use Lemma \ref{8.5.14} to prove Lemma \ref{8.5.15}.
\end{exercise}

\begin{proof}
    See Lemma \ref{8.5.15}
\end{proof}

\begin{exercise}\label{ex 8.5.15}
    Let \(A\) and \(B\) be two non-empty sets such that \(A\) does not have lesser or equal cardinality to \(B\).
    Using Zorn's lemma, prove that \(B\) has lesser or equal cardinality to \(A\).
    This exercise (combined with Exercise \ref{ex 8.3.3}) shows that the cardinality of any two sets is comparable, as long as one assumes the axiom of choice.
\end{exercise}

\begin{proof}
    For every subset \(X \subseteq B\), let \(P(X)\) denote the property that there exists an injective map from \(X \to A\).
    Let \(S = \{X \subseteq B : P(X)\}\) be a set.
    Since \(B \neq \emptyset\), let \(b \in B\) and let \(f_{\{b\}} : \{b\} \to A\).
    Clearly \(f_{\{b\}}\) is injective, so we know that \(S \neq \emptyset\).
    Let \(\leq_S\) be a ordering relation of \(S\) by setting
    \begin{align*}
             & \forall\ X, Y \in S, X \leq_S Y                                           \\
        \iff & (X \subseteq Y)                                                           \\
             & \land \big(\exists\ f_X : X \to A \text{ where \(f_X\) is injective}\big) \\
             & \land \big(\exists\ f_Y : Y \to A \text{ where \(f_Y\) is injective}\big) \\
             & \land \big(\forall\ x \in X, f_X(x) = f_Y(x)\big).
    \end{align*}
    Since
    \[
        \forall\ X \in S, (X \subseteq X) \land \big(\forall\ x \in X, f_X(x) = f_X(x)\big) \implies X \leq_S X,
    \]
    we know that \((S, \leq_S)\) is reflexive.
    Since
    \begin{align*}
                 & \forall\ X, Y \in S, (X \leq_S Y) \land (Y \leq_S X)                                        \\
        \implies & (X \subseteq Y) \land (Y \subseteq X)                                                       \\
        \implies & X = Y,                                               & \text{(by Proposition \ref{3.1.18})}
    \end{align*}
    we know that \((S, \leq_S)\) is anti-symmetric.
    Since
    \begin{align*}
                 & \forall\ X, Y, Z \in S, (X \leq_S Y) \land (Y \leq_S Z)                                                                                   \\
        \implies & (X \subseteq Y) \land (Y \subseteq Z) \land \big(\forall\ x \in X, f_X(x) = f_Y(x)\big) \land \big(\forall\ y \in Y, f_Y(y) = f_Z(y)\big) \\
        \implies & (X \subseteq Z) \land \big(\forall\ x \in X, f_X(x) = f_Z(x)\big)                                                                         \\
        \implies & X \leq_S Z,
    \end{align*}
    we know that \((S, \leq_S)\) is transitive.
    Since \((S, \leq_S)\) is reflexive, anti-symmetric and transitive, by Definition \ref{8.5.1} we know that \((S, \leq_S)\) is partially ordered.

    Next we show that \(\forall\ S' \subseteq S\), if \(S'\) is totally ordered, then \(\bigcup S' \in S\).
    Clearly, we have \(\bigcup S' \subseteq B\).
    So we only need to show that there exists a function \(f : \bigcup S' \to A\) such that \(f\) is injective.
    For each \(X \in S'\), let \(F_X = \{f_X : X \to A \text{ is injective}\}\) be a set.
    Since \(P(X)\) is true, we know that \(F_X \neq \emptyset\).
    By axiom of choice (Axiom \ref{8.1}) we know that \(\prod_{X \in S'} F_X \neq \emptyset\).
    Let \((f_X)_{X \in S'} \in \prod_{X \in S'} F_X\).
    Fix such \((f_X)_{X \in S'}\).
    We define a function \(f : \bigcup S' \to A\) as follow:
    \[
        \forall\ x_0 \in \bigcup S', f(x_0) = f_Y(x_0)
    \]
    for some \(Y \in S'\), \(x_0 \in Y\) and \(f_Y = (f_X)_{X \in S'}(Y)\).
    To show that such \(f\) is well-defined, we need to show that \(Y\) is unique for each \(x_0 \in \bigcup S'\).
    Since \(S'\) is totally ordered, we know that \(\forall\ Z \in S'\), we have \(Z \leq_S Y\) or \(Y \leq_S Z\).
    Let \(f_Z = (f_X)_{X \in S'}(Z)\).
    We split into two cases:
    \begin{itemize}
        \item If \(Z \leq_S Y\), then by the definition of \(\leq_S\) we have \(Z \subseteq Y\) and \(\forall\ z \in Z\), \(f_Z(z) = f_Y(z)\).
              Now we further split into two cases:
              \begin{itemize}
                  \item If \(x_0 \in Z\), then \(f_Z(x_0) = f_Y(x_0)\).
                  \item If \(x_0 \notin Z\), then \(\forall\ z \in Z\), we have \(f_Z(z) = f_Y(z) \neq f_Y(x_0)\) since \(f_Y, f_Z\) are injective.
              \end{itemize}
        \item If \(Y \leq_S Z\), then by the definition of \(\leq_S\) we have \(Y \subseteq Z\) and \(\forall\ y \in Y\), \(f_Y(y) = f_Z(y)\).
              Thus \(x_0 \in Z\) and \(f_Y(x_0) = f_Z(x_0)\).
    \end{itemize}
    From all cases above we conclude that \(f\) is well-defined.
    Now we show that \(f\) is injective.
    Let \(y, z \in \bigcup S'\) and \(y \neq z\).
    Then \(\exists\ Y, Z \in S'\) such that \(y \in Y\) and \(z \in Z\).
    Again we have either \(Y \leq_S Z\) or \(Z \leq_S Y\).
    Without the loss of generality suppose that \(Y \leq_S Z\).
    Then we have
    \[
        f_Y(y) = f_Z(y) \neq f_Z(z)
    \]
    where \(f_Y = (f_X)_{X \in S'}(Y)\) and \(f_Z = (f_X)_{X \in S'}(Z)\).
    Thus \(f\) is injective and \(\bigcup S' \in S\).

    Now we show that \(\forall\ S' \subseteq S\), if \(S'\) is totally ordered with ordering relation \(\leq_S\), then there exists an upper bound of \(S'\).
    Let \(S'\) be an totally ordered subset of \(S\) with ordering relation \(\leq_S\).
    Since
    \begin{align*}
                 & \forall\ Y \in S'                                                                                              \\
        \implies & \Big(Y \subseteq \bigcup S' \in S\Big)                                             & \text{(from claim above)} \\
                 & \land \Big(\forall\ y \in Y, f_Y(x) = \big((f_X)_{X \in S'}(Y)\big)(x) = f(y)\Big)                             \\
        \implies & Y \leq_S \bigcup S',
    \end{align*}
    we know that \(\bigcup S'\) is an upper bound of \(S'\).
    Since \(S'\) is arbitrary, we conclude that every totally ordered subset of \(S\) with oredering relation \(\leq_S\) has an upper bound.

    By Zorn's lemma (Lemma \ref{8.5.15}) we know that there exists at least one maximal element of \(S\).
    Suppose for sake of contradiction that \(B \neq \max(S)\).
    Let \(X = \max(S)\).
    So \(B \setminus X \neq \emptyset\).
    Then we know that \(P(X)\) is true, i.e., \(\exists\ f_X : X \to A\) such that \(f_X\) is injective.
    We must have \(f_X(X) = A\), otherwise \(f_X(X) \subseteq A\), we know that \(\exists\ a \in A \setminus f_X(X)\), and we can let \(b \in B \setminus X\) map to \(a\), i.e., \(f_X(b) = a\).
    This cause \((X \subseteq X \cup \{b\}) \land \big(\forall\ x \in X, f_X(x) = f_{X \cup \{b\}}(x)\big)\), which means \(X \leq_S X \cup \{b\}\) and contradicts to \(X = \max(S)\).
    So we have \(f_X(X) = A\).
    But this also means \(f_X\) is a bijection from \(X\) to \(A\).
    So we can set \(g : A \to B\) as \(g(a) = f_X^{-1}(a)\), which is injective.
    By hypothesis we know that we cannot have a injection from \(A\) to \(B\) (this is the definition of \(A\) having lesser or equal cardinality than \(B\), see Exercise \ref{ex 3.6.7}).
    Thus we derived a contradiction.
    So \(B = \max(S) \in S\) and \(\exists\ f_B : B \to A\) such that \(f_B\) is injective.
    By Exercise \ref{ex 8.3.3} \(B\) has lesser or equal cardinality than \(A\).
\end{proof}

\begin{exercise}\label{ex 8.5.16}
    Let \(X\) be a set, and let \(P\) be the set of all partial orderings of \(X\).
    (For instance, if \(X \coloneqq \mathbf{N} \setminus \{0\}\), then both the usual partial ordering \(\leq\), and the partial ordering in Exercise \ref{ex 8.5.3}, are elements of \(P\).)
    We say that one partial ordering \(\leq \in P\) is \emph{coarser} than another partial ordering \(\leq' \in P\) if for any \(x, y \in X\), we have the implication \((x \leq y) \implies (x \leq' y)\).
    Thus for instance the partial ordering in Exercise \ref{ex 8.5.3} is coarser than the usual ordering \(\leq\).
    Let us write \(\leq \preceq \leq'\) if \(\leq\) is coarser than \(\leq'\).
    Show that \(\preceq\) turns \(P\) into a partially ordered set;
    thus the set of partial orderings on \(X\) is itself partially ordered.
    There is exactly one minimal element of \(P\);
    what is it?
    Show that the maximal elements of \(P\) are precisely the total orderings of \(P\).
    Using Zorn's lemma (Lemma \ref{8.5.15}), show that given any partial ordering \(\leq\) of \(X\) there exists a total ordering \(\leq'\) such that \(\leq\) is coarser than \(\leq'\).
\end{exercise}

\begin{proof}
    Since
    \begin{align*}
                 & \forall\ \leq \in P                                \\
        \implies & (\forall\ x, y \in X : x \leq y \implies x \leq y) \\
        \implies & \leq \preceq \leq,
    \end{align*}
    we know that \((P, \preceq)\) is reflexive.
    Since
    \begin{align*}
                 & \forall\ \leq, \leq' \in P : (\leq \preceq \leq') \land (\leq' \preceq \leq)                      \\
        \implies & \big(\forall\ x, y \in X : (x \leq y \implies x \leq y') \land (x \leq' y \implies x \leq y)\big) \\
        \implies & \big(\forall\ x, y \in X : (x \leq y \iff x \leq' y)\big)                                         \\
        \implies & \leq = \leq',
    \end{align*}
    we know that \((P, \preceq)\) is anti-symmetric.
    Since
    \begin{align*}
                 & \forall\ \leq_1, \leq_2, \leq_3 \in P : (\leq_1 \preceq \leq_2) \land (\leq_2 \preceq \leq_3)           \\
        \implies & \big(\forall\ x, y \in X : (x \leq_1 y \implies x \leq_2 y) \land (x \leq_2 y \implies x \leq_3 y)\big) \\
        \implies & \big(\forall\ x, y \in X : (x \leq_1 y \implies x \leq_3 y)\big)                                        \\
        \implies & \leq_1 \preceq \leq_3,
    \end{align*}
    we know that \((P, \preceq)\) is transitive.
    Since \((P, \preceq)\) is reflexive, anti-symmetric and transitive, by Definition \ref{8.5.1} we know that \((P, \preceq)\) is partially ordered.

    Next we show that \((P, \preceq)\) has exactly one minimal element.
    \begin{enumerate}
        \item If \(X = \emptyset\), then \(\{\leq_{\emptyset}\} = P\) since
              \begin{align*}
                           & \forall\ \leq \in P                                                                  \\
                  \implies & (\forall\ x, y \in X : x \leq_{\emptyset} y \iff x \leq y) & \text{(vacuously true)} \\
                  \implies & \leq_{\emptyset} = \leq.
              \end{align*}
              Thus \(\leq_{\emptyset}\) is the minimal element of \((P, \preceq)\).
        \item If \(X \neq \emptyset\), then we have \(=\) is the minimal element of \((P, \preceq)\) since
              \begin{align*}
                           & \forall\ \leq \in P                                                                   \\
                  \implies & (\forall\ x, y \in X : x = y \implies x \leq y) & \text{(\(\leq\) is anti-symmetric)} \\
                  \implies & = \preceq \leq.
              \end{align*}
    \end{enumerate}

    From prove above we know that \(P \neq \emptyset\).
    Now let \(C\) be a totally ordered subset of \(P\).
    In order to apply Zorn's lemma (Lemma \ref{8.5.15}), we have to show that there is an upper bound of \(C\).
    Let \(\leq_C\) be defined as follows:
    \[
        \forall\ x, y \in X, x \leq_C y \iff \exists\ \leq' \in C : x \leq' y.
    \]
    We have to show \(\leq_C\) is well-defined.
    Suppose \(\leq, \leq' \in C\) and there exists \(x, y \in X\) such that \(x \leq y\) and \(y \leq' x\) but \(y \neq x\).
    Since both are in \(C\), either \(\leq \preceq \leq' \lor \leq' \preceq \leq\).
    Without loss of generality suppose that \(\leq \preceq \leq'\), then \(x \leq y\) implies \(x \leq' y\) and since \(x \neq y\) so \(x <' y\) which contradicts our assumption that \(y <' x\).
    Hence \(\leq_C\) is well-defined.
    Also is not difficult to show that is a partial ordering of \(X\).
    Clearly \(\leq_C\) is an upper bound for \(C\) because whenever \(x \leq' y\) for \(\leq' \in C\) would imply \(x \leq_C y\), which means \(\leq' \preceq \leq_C\).
    Thus by Zorn's lemma (Lemma \ref{8.5.15}) \(X\) contains at least one maximal element which we call \(\leq_P\).
    We claim that \(\leq_P\) is a total ordering of \(X\).
    Suppose for the sake of contradiction that \(\leq_P\) is not a total ordering of \(X\).
    Thus there must be a pair \(x, y \in X\) for which neither \(x \not\leq_p y\) nor \(y \not\leq_p x\).
    Thus we can define \(\leq = \leq_p\) for all \(X \setminus \{x, y\}\) and we set \(x \leq y\).
    Thus is a partial ordering of \(X\), i.e., \(\leq \in P\) and also \(\leq_p \preceq \leq\), note that the implication \(x \leq_p y \implies x \leq y\) is vacuously true, which is a contradiction.
    Hence \(\leq_p\) is a total ordering of \(X\).
\end{proof}

\begin{exercise}\label{ex 8.5.17}
    Use Zorn's lemma (Lemma \ref{8.5.15}) to give another proof of the claim in Exercise \ref{ex 8.4.2}.
    Deduce that Zorn's lemma (Lemma \ref{8.5.15}) and the axiom of choice are in fact logically equivalent
    (i.e., they can be deduced from each other).
\end{exercise}

\begin{exercise}\label{ex 8.5.18}
    Using Zorn's lemma (Lemma \ref{8.5.15}), prove \emph{Hausdorff's maximality principle}:
    if \(X\) is a partially ordered set, then there exists a totally ordered subset \(Y\) of \(X\) which is maximal with respect to set inclusion
    (i.e. there is no other totally ordered subset \(Y'\) of \(X\) which contains \(Y\)).
    Conversely, show that if Hausdorff's maximality principle is true, then Zorn's lemma (Lemma \ref{8.5.15}) is true.
    Thus by Exercise \ref{ex 8.5.17}, these two statements are logically equivalent to the axiom of choice.
\end{exercise}

\begin{exercise}\label{ex 8.5.19}
    Let \(X\) be a set, and let \(\Omega\) be the space of all pairs \((Y, \leq)\), where \(Y\) is a subset of \(X\) and \(\leq\) is a well-ordering of \(Y\).
    If \((Y, \leq)\) and \((Y', \leq')\) are elements of \(\Omega\), we say that \((Y, \leq)\) is an \emph{initial segment} of \((Y', \leq')\) if there exists an \(x \in Y'\) such that \(Y \coloneqq \{y \in Y' : y <' x\}\) (so in particular \(Y \subsetneq Y'\)), and for any \(y, y' \in Y\), \(y \leq y'\) if and only if \(y \leq' y'\).
    Define a relation \(\preceq\) on \(\Omega\) by defining \((Y, \leq) \preceq (Y', \leq')\) if either \((Y, \leq) = (Y', \leq')\), or if \((Y, \leq)\) is an initial segment of \((Y', \leq')\).
    Show that \(\preceq\) is a partial ordering of \(\Omega\).
    There is exactly one minimal element of \(\Omega\);
    what is it?
    Show that the maximal elements of \(\Omega\) are precisely the well-orderings \((X, \leq)\) of \(X\).
    Using Zorn's lemma (Lemma \ref{8.5.15}), conclude the well ordering principle:
    every set \(X\) has at least one well-ordering.
    Conversely, use the well-ordering principle to prove the axiom of choice, Axiom \ref{8.1}.
    We thus see that the axiom of choice, Zorn's lemma (Lemma \ref{8.5.15}), and the well-ordering principle are all logically equivalent to each other.
\end{exercise}

\begin{exercise}\label{ex 8.5.20}
    Let \(X\) be a set, and let \(\Omega \subseteq 2^X\) be a collection of subsets of \(X\).
    Assume that \(\Omega\) does not contain the empty set \(\emptyset\).
    Using Zorn's lemma, show that there is a subcollection \(\Omega' \subseteq \Omega\) such that all the elements of \(\Omega'\) are disjoint from each other (i.e., \(A \cap B = \emptyset\) whenever \(A, B\) are distinct elements of \(\Omega'\)), but that all the elements of \(\Omega\) intersect at least one element of \(\Omega'\) (i.e., for all \(C \in \Omega\) there exists \(A \in \Omega'\) such that \(C \cap A \neq \emptyset\)).
    Conversely, if the above claim is true, show that it implies the claim in Exercise \ref{ex 8.4.2}, and thus this is yet another claim which is logically equivalent to the axiom of choice.
\end{exercise}