\section{Ordered sets}\label{sec 8.5}

\begin{definition}[Partially ordered sets]\label{8.5.1}
    A \emph{partially ordered set} (or \emph{poset}) is a set \(X\), together with a relation \(\leq_X\) on \(X\)
    (thus for any two objects \(x, y \in X\), the statement \(x \leq_X y\) is either a true statement or a false statement).
    Furthermore, this relation is assumed to obey the following three properties:
    \begin{itemize}
        \item (Reflexivity) For any \(x \in X\), we have \(x \leq_X x\).
        \item (Anti-symmetry) If \(x, y \in X\) are such that \(x \leq_X y\) and \(y \leq_X x\), then \(x = y\).
        \item (Transitivity) If \(x, y, z \in X\) are such that \(x \leq_X y\) and \(y \leq_X z\), then \(x \leq_X z\).
    \end{itemize}
    We refer to \(\leq_X\) as the \emph{ordering relation}.
    In most situations it is understood what the set \(X\) is from context, and in those cases we shall simply write \(\leq\) instead of \(\leq_X\).
    We write \(x <_X y\) (or \(x < y\) for short) if \(x \leq_X y\) and \(x \neq y\).
\end{definition}

\begin{note}
    Strictly speaking, a partially ordered set is not a set \(X\), but rather a pair \((X, \leq_X)\).
    But in many cases the ordering \(\leq_X\) will be clear from context, and so we shall refer to \(X\) itself as the partially ordered set even though this is technically incorrect.
\end{note}

\begin{example}\label{8.5.2}
    The natural numbers \(\mathbf{N}\) together with the usual less-than-or-equal-to relation \(\leq\) (as defined in Definition \ref{2.2.11}) forms a partially ordered set, by Proposition \ref{2.2.12}.
    Similar arguments (using the appropriate definitions and propositions) show that the integers \(\mathbf{Z}\), the rationals \(\mathbf{Q}\), the reals \(\mathbf{R}\), and the extended reals \(\mathbf{R}^*\) are also partially ordered sets.
    Meanwhile, if \(X\) is any collection of sets, and one uses the relation of is-a-subset-of \(\subseteq\) (as defined in Definition \ref{3.1.15}) for the ordering relation \(\leq_X\), then \(X\) is also partially ordered (Proposition \ref{3.1.18}).
    Note that it is certainly possible to give these sets a different partial ordering than the standard one.
\end{example}

\begin{definition}[Totally ordered set]\label{8.5.3}
    Let \(X\) be a partially ordered set with some order relation \(\leq_X\).
    A subset \(Y\) of \(X\) is said to be \emph{totally ordered} if, given any two \(y, y' \in Y\), we either have \(y \leq_X y'\) or \(y' \leq_X y\) (or both).
    If \(X\) itself is totally ordered, we say that \(X\) is a \emph{totally ordered set} (or \emph{chain}) with order relation \(\leq_X\).
\end{definition}

\begin{example}\label{8.5.4}
    The natural numbers \(\mathbf{N}\), the integers \(\mathbf{Z}\), the rationals \(\mathbf{Q}\), reals \(\mathbf{R}\), and the extended reals \(\mathbf{R}^*\), all with the usual ordering relation \(\leq\), are totally ordered
    (by Proposition \ref{2.2.13}, Lemma \ref{4.1.11}, Proposition \ref{4.2.9}, Proposition \ref{5.4.7}, and Proposition \ref{6.2.5} respectively).
    Also, any subset of a totally ordered set is again totally ordered. On the other hand, a collection of sets with the \(\subseteq\) relation is usually not totally ordered.
\end{example}