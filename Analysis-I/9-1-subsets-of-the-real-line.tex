\section{Subsets of the real line}\label{sec 9.1}

\begin{definition}[Intervals]\label{9.1.1}
    Let \(a, b \in \mathbf{R}^*\) be extended real numbers.
    We define the \emph{closed interval} \([a, b]\) by
    \[
        [a, b] \coloneqq \{x \in \mathbf{R}^* : a \leq x \leq b\},
    \]
    the \emph{half-open intervals} \([a, b)\) and \((a, b]\) by
    \[
        [a, b) \coloneqq \{x \in \mathbf{R}^* : a \leq x < b\}; (a, b] \coloneqq \{x \in \mathbf{R}^* : a < x \leq b\},
    \]
    and the \emph{open interval} \((a, b)\) by
    \[
        (a, b) \coloneqq \{x \in \mathbf{R}^* : a < x < b\}.
    \]
    We call \(a\) the \emph{left endpoint} of these intervals, and \(b\) the \emph{right endpoint}.
\end{definition}

\begin{remark}\label{9.1.2}
    Once again, we are overloading the parenthesis notation;
    for instance, we are now using \((2, 3)\) to denote both an open interval from \(2\) to \(3\), as well as an ordered pair in the Cartesian plane \(\mathbf{R}^2 \coloneqq \mathbf{R} \times \mathbf{R}\).
    This can cause some genuine ambiguity, but the reader should still be able to resolve which meaning of the parentheses is intended from context.
    In some texts, this issue is resolved by using reversed brackets instead of parenthesis, thus for instance \([a, b)\) would now be \([a, b[\), \((a, b]\) would be \(]a, b]\), and \((a, b)\) would be \(]a, b[\).
\end{remark}

\begin{note}
    We sometimes refer to an interval in which one endpoint is infinite (either \(+\infty\) or \(-\infty\)) as \emph{half-infinite} intervals, and intervals in which both endpoints are infinite as \emph{doubly-infinite} intervals;
    all other intervals are \emph{bounded intervals}.
    Thus the positive and negative real axes are half-infinite intervals, and \(\mathbf{R}\) and \(\mathbf{R}^*\) are infinite intervals.
\end{note}

\setcounter{theorem}{3}
\begin{example}\label{9.1.4}
    If \(a > b\) then all four of the intervals \([a, b], [a, b), (a, b]\), and \((a, b)\) are the empty set (by trichotomy, see Proposition \ref{5.4.7}(a)).
    If \(a = b\), then the three intervals \([a, b), (a, b]\), and \((a, b)\) are the empty set, while \([a, b]\) is just the singleton set \(\{a\}\).
    Because of this, we call these intervals \emph{degenerate};
    most (but not all) of our analysis will be restricted to non-degenerate intervals.
\end{example}

\begin{definition}[\(\varepsilon\)-adherent points]\label{9.1.5}
    Let \(X\) be a subset of \(\mathbf{R}\), let \(\varepsilon > 0\), and let \(x \in \mathbf{R}\).
    We say that \(x\) is \emph{\(\varepsilon\)-adherent to \(X\)} iff there exists a \(y \in X\) which is \(\varepsilon\)-close to \(x\)
    (i.e., \(\abs*{x - y} \leq \varepsilon\)).
\end{definition}

\begin{remark}\label{9.1.6}
    The terminology ``\(\varepsilon\)-adherent'' is not standard in the literature.
    However, we shall shortly use it to define the notion of an adherent point, which is standard.
\end{remark}

\setcounter{theorem}{7}
\begin{definition}[Adherent points]\label{9.1.8}
    Let \(X\) be a subset of \(\mathbf{R}\), and let \(x \in \mathbf{R}\).
    We say that \(x\) is an \emph{adherent point} of \(X\) iff it is \(\varepsilon\)-adherent to \(X\) for every \(\varepsilon > 0\).
\end{definition}

\setcounter{theorem}{9}
\begin{definition}[Closure]\label{9.1.10}
    Let \(X\) be a subset of \(\mathbf{R}\).
    The \emph{closure} of \(X\), sometimes denoted \(\overline{X}\) is defined to be the set of all the adherent points of \(X\).
\end{definition}

\begin{lemma}[Elementary properties of closures]\label{9.1.11}
    Let \(X\) and \(Y\) be arbitrary subsets of \(\mathbf{R}\).
    Then \(X \subseteq \overline{X}\), \(\overline{X \cup Y} = \overline{X} \cup \overline{Y}\), and \(\overline{X \cap Y} \subseteq \overline{X} \cap \overline{Y}\).
    If \(X \subseteq Y\), then \(\overline{X} \subseteq \overline{Y}\).
\end{lemma}

\begin{proof}
    Let \(\varepsilon \in \mathbf{R}^+\).
    Since
    \begin{align*}
                 & \forall\ x \in X                                                        \\
        \implies & \abs*{x - x} = 0 \leq \varepsilon                                       \\
        \implies & x \in \overline{X},               & \text{(by Definition \ref{9.1.10})}
    \end{align*}
    by Definition \ref{3.1.15} we have \(X \subseteq \overline{X}\).
    Since
    \begin{align*}
                 & \forall\ x \in \overline{X \cup Y}                                                            \\
        \implies & \exists\ y \in X \cup Y : \abs*{x - y} \leq \varepsilon & \text{(by Definition \ref{9.1.10})} \\
        \implies & y \in X \lor y \in Y                                    & \text{(by Axiom \ref{3.4})}         \\
        \implies & x \in \overline{X} \lor x \in \overline{Y}              & \text{(by Definition \ref{9.1.10})} \\
        \implies & x \in \overline{X} \cup \overline{Y}                    & \text{(by Axiom \ref{3.4})}
    \end{align*}
    and
    \begin{align*}
                 & \forall\ x \in \overline{X} \cup \overline{Y}                                                 \\
        \implies & x \in \overline{X} \lor x \in \overline{Y}              & \text{(by Axiom \ref{3.4})}         \\
        \implies & (\exists\ y \in X : \abs*{x - y} \leq \varepsilon)                                            \\
                 & \lor (\exists\ y \in Y : \abs*{x - y} \leq \varepsilon) & \text{(by Definition \ref{9.1.10})} \\
        \implies & \exists\ y \in X \cup Y : \abs*{x - y} \leq \varepsilon & \text{(by Axiom \ref{3.4})}         \\
        \implies & x \in \overline{X \cup Y},                              & \text{(by Definition \ref{9.1.10})}
    \end{align*}
    by Proposition \ref{3.1.18} we have \(\overline{X \cup Y} = \overline{X} \cup \overline{Y}\).
    Since
    \begin{align*}
                 & \forall\ x \in \overline{X \cap Y}                                                            \\
        \implies & \exists\ y \in X \cap Y : \abs*{x - y} \leq \varepsilon & \text{(by Definition \ref{9.1.10})} \\
        \implies & y \in X \land y \in Y                                   & \text{(by Definition \ref{3.1.23})} \\
        \implies & x \in \overline{X} \land x \in \overline{Y}             & \text{(by Definition \ref{9.1.10})} \\
        \implies & x \in \overline{X} \cap \overline{Y},                   & \text{(by Definition \ref{3.1.23})}
    \end{align*}
    by Definition \ref{3.1.15} we have \(\overline{X \cap Y} \subseteq \overline{X} \cap \overline{Y}\).
    Now suppose that \(X \subseteq Y\).
    Then we have
    \begin{align*}
                 & \forall\ x \in \overline{X}                                                            \\
        \implies & \exists\ y \in X : \abs*{x - y} \leq \varepsilon & \text{(by Definition \ref{9.1.10})} \\
        \implies & y \in Y                                          & (X \subseteq X)                     \\
        \implies & x \in \overline{Y}.                              & \text{(by Definition \ref{9.1.10})}
    \end{align*}
    Thus by Definition \ref{3.1.15} we have \(\overline{X} \subseteq \overline{Y}\).
    And we conclude that \(X \subseteq Y \implies \overline{X} \subseteq \overline{Y}\).
\end{proof}

\begin{lemma}[Closures of intervals]\label{9.1.12}
    Let \(a < b\) be real numbers, and let \(I\) be any one of the four intervals \((a, b)\), \((a, b]\), \([a, b)\), or \([a, b]\).
    Then the closure of \(I\) is \([a, b]\).
    Similarly, the closure of \((a, \infty)\) or \([a, \infty)\) is \([a, \infty)\), while the closure of \((-\infty, a)\) or \((-\infty, a]\) is \((-\infty, a]\).
    Finally, the closure of \((-\infty, \infty)\) is \((-\infty, \infty)\).
\end{lemma}

\begin{proof}
    First let us show that every element of \([a, b]\) is adherent to \((a, b)\).
    Let \(x \in [a, b]\).
    If \(x \in (a, b)\) then it is definitely adherent to \((a, b)\).
    This is true since \(\forall\ \varepsilon \in \mathbf{R}^+\) we have \(\abs*{x - x} \leq \varepsilon\).
    If \(x = b\) then \(x\) is also adherent to \((a, b)\).
    Otherwise \(\exists\ \varepsilon \in \mathbf{R}^+\) such that
    \[
        \forall\ y \in (a, b), \abs*{x - y} > \varepsilon.
    \]
    But this means
    \begin{align*}
                 & \abs*{x - y} > \varepsilon                                                               \\
        \implies & \abs*{b - y} > \varepsilon                        & (x = b)                              \\
        \implies & b - y > \varepsilon                               & (y < b)                              \\
        \implies & b - \varepsilon > y                                                                      \\
        \implies & b > b - \varepsilon > y > a                       & (\varepsilon > 0 \land y \in (a, b)) \\
        \implies & b - \varepsilon \in (a, b)                        & \text{(by Definition \ref{9.1.1})}   \\
        \implies & 0 = \abs*{b - (b - \varepsilon)} \leq \varepsilon                                        \\
        \implies & \abs*{x - (b - \varepsilon)} \leq \varepsilon,
    \end{align*}
    a contradiction.
    Thus \(x = b\) implies \(x\) is also adherent to \((a, b)\).
    Similarly when \(x = a\).
    Thus every point in \([a, b]\) is adherent to \((a, b)\).

    Now we show that every point \(x\) that is adherent to \((a, b)\) lies in \([a, b]\).
    Suppose for sake of contradiction that \(x\) does not lie in \([a, b]\), then either \(x > b\) or \(x < a\).
    If \(x > b\) then \(x\) is not \((x - b)\)-adherent to \((a, b)\), and is hence not an adherent point to \((a, b)\)
    (by setting \(\varepsilon = x - b\) we have \(\forall\ y \in (a, b)\), \(\abs*{x - y} = x - y > x - b = \varepsilon\)).
    Similarly, if \(x < a\), then \(x\) is not \((a - x)\)-adherent to \((a, b)\), and is hence not an adherent point to \((a, b)\).
    This contradiction shows that \(x\) is in fact in \([a, b]\) as claimed.

    Using similar arguments we can show that every point in \([a, b]\) is also adherent to \((a, b]\) and \([a, b)\), and every point \(x\) that is adherent to \((a, b]\) and \([a, b)\) lies in \([a, b]\).

    Now we show that the closure of \((a, \infty)\) or \([a, \infty)\) is \([a, \infty)\).
    From the proof above we know that the closure of \((a, \infty)\) and \([a, \infty)\) is \([a, \infty]\).
    But since \(\forall\ x \in (a, \infty)\) the statement \(\abs*{x - \infty} \leq \varepsilon\) is undefined, thus we can only have \([a, \infty)\) as the closure of \((a, \infty)\) and \([a, \infty)\).
                    Similar arguments show that the closure of \((-\infty, a)\) or \((-\infty, a]\) is \((-\infty, a]\), and the closure of \((-\infty, \infty)\) is \((-\infty, \infty)\).
\end{proof}

\begin{lemma}\label{9.1.13}
    The closure of \(\mathbf{N}\) is \(\mathbf{N}\).
    The closure of \(\mathbf{Z}\) is \(\mathbf{Z}\).
    The closure of \(\mathbf{Q}\) is \(\mathbf{R}\), and the closure of \(\mathbf{R}\) is \(\mathbf{R}\).
    The closure of the empty set \(\emptyset\) is \(\emptyset\).
\end{lemma}

\begin{proof}
    We first show that \(\overline{\mathbf{N}} = \mathbf{N}\).
    Let \(\overline{\mathbf{N}}\) be the closure of \(\mathbf{N}\).
    By Lemma \ref{9.1.11} we have \(\mathbf{N} \subseteq \overline{\mathbf{N}}\).
    Since
    \begin{align*}
                 & \forall\ n \in \mathbf{N}                                      \\
        \implies & 0 \leq n < \infty                                              \\
        \implies & n \in [0, \infty),        & \text{(by Definition \ref{9.1.1})}
    \end{align*}
    by Definition \ref{3.1.15} we have \(\mathbf{N} \subseteq [0, \infty)\), thus
    \begin{align*}
                 & \mathbf{N} \subseteq [0, \infty)                                                        \\
        \implies & \overline{\mathbf{N}} \subseteq \overline{[0, \infty)} & \text{(by Lemma \ref{9.1.11})} \\
        \implies & \overline{\mathbf{N}} \subseteq [0, \infty)            & \text{(by Lemma \ref{9.1.12})} \\
        \implies & \overline{\mathbf{N}} \subseteq \mathbf{R}.
    \end{align*}
    Now we show that \(\overline{\mathbf{N}} \subseteq \mathbf{N}\).
    Suppose for sake of contradiction that \(\overline{\mathbf{N}} \not\subseteq \mathbf{N}\), i.e., \(\exists\ x \in \overline{\mathbf{N}}\) such that \(x \notin \mathbf{N}\).
    Then we have \(x > 0\) and by Proposition \ref{5.4.12} \(\exists\ n \in \mathbf{N} : n < x < n + 1\).
    Let \(\varepsilon = \min(x - n, n + 1 - x) / 2\).
    By Definition \ref{9.1.10}, \(\exists\ m \in \mathbf{N}\) such that \(\abs*{x - m} \leq \varepsilon\).
    We now split into three cases:
    \begin{enumerate}
        \item If \(m = n\), then we have \(x - n \geq \min(x - n, n + 1 - x) > \varepsilon\), a contradiction.
        \item If \(m < n\), then we have \(x - m > x - n \geq \min(x - n, n + 1 - x) > \varepsilon\), a contradiction.
        \item If \(m > n\), then we have \(m \geq n + 1\) and \(m - x \geq n + 1 - x \geq \min(x - n, n + 1 - x) > \varepsilon\), a contradiction.
    \end{enumerate}
    From all cases above we derived contradictions.
    Thus such \(m\) does not exists and by Definition \ref{9.1.10} \(x \notin \overline{\mathbf{N}}\).
    So we have \(\overline{\mathbf{N}} \subseteq \mathbf{N}\).
    Since \(\mathbf{N} \subseteq \overline{\mathbf{N}} \land \overline{\mathbf{N}} \subseteq \mathbf{N}\), by Proposition \ref{3.1.18} we have \(\mathbf{N} = \overline{\mathbf{N}}\).

    Next we show that \(\overline{\mathbf{Z}} = \mathbf{Z}\).
    Let \(\overline{\mathbf{Z}}\) be the closure of \(\mathbf{Z}\).
    Let \(\mathbf{Z}^- = \{z \in \mathbf{Z} : z < 0\}\).
    Then we have
    \begin{align*}
        \overline{\mathbf{Z}} & = \overline{\mathbf{N} \cup \mathbf{Z}^-}                                             \\
                              & = \overline{\mathbf{N}} \cup \overline{\mathbf{Z}^-} & \text{(by Lemma \ref{9.1.11})} \\
                              & = \mathbf{N} \cup \overline{\mathbf{Z}^-}            & \text{(from proof above)}
    \end{align*}
    Thus to show that \(\mathbf{Z} = \overline{\mathbf{Z}}\) it is suffice to show that \(\mathbf{Z}^- = \overline{\mathbf{Z}^-}\).
    By Lemma \ref{9.1.11} we have \(\mathbf{Z}^- \subseteq \overline{\mathbf{Z}^-}\).
    We now to show that \(\overline{\mathbf{Z}^-} \subseteq \mathbf{Z}^-\).
    Suppose for sake of contradiction that \(\overline{\mathbf{Z}^-} \not\subseteq \mathbf{Z}^-\), i.e., \(\exists\ x \in \overline{\mathbf{Z}^-}\) such that \(x \notin \mathbf{Z}^-\).
    Then we have \(x < 0\) and by Proposition \ref{5.4.12} \(\exists\ n \in \mathbf{Z}^- : n < x < n + 1\).
    Let \(\varepsilon = \min(x - n, n + 1 - x) / 2\).
    By Definition \ref{9.1.10}, \(\exists\ m \in \mathbf{Z}^-\) such that \(\abs*{x - m} \leq \varepsilon\).
    We now split into three cases:
    \begin{enumerate}
        \item If \(m = n\), then we have \(x - n \geq \min(x - n, n + 1 - x) > \varepsilon\), a contradiction.
        \item If \(m < n\), then we have \(x - m > x - n \geq \min(x - n, n + 1 - x) > \varepsilon\), a contradiction.
        \item If \(m > n\), then we have \(m \geq n + 1\) and \(m - x \geq n + 1 - x \geq \min(x - n, n + 1 - x) > \varepsilon\), a contradiction.
    \end{enumerate}
    From all cases above we derived contradictions.
    Thus such \(m\) does not exists and by Definition \ref{9.1.10} \(x \notin \overline{\mathbf{Z}^-}\).
    So we have \(\overline{\mathbf{Z}^-} \subseteq \mathbf{Z}^-\).
    Since \(\mathbf{Z}^- \subseteq \overline{\mathbf{Z}^-} \land \overline{\mathbf{Z}^-} \subseteq \mathbf{Z}^-\), by Proposition \ref{3.1.18} we have \(\mathbf{Z}^- = \overline{\mathbf{Z}^-}\), and thus \(\mathbf{Z} = \overline{\mathbf{Z}}\).

    Next we show that \(\overline{\mathbf{Q}} = \mathbf{R}\).
    Let \(\overline{\mathbf{Q}}\) be the closure of \(\mathbf{Q}\).
    We have
    \begin{align*}
                 & \mathbf{Q} \subseteq \mathbf{R}                                                                   \\
        \implies & \mathbf{Q} \subseteq (-\infty, \infty)                       & \text{(by Definition \ref{9.1.1})} \\
        \implies & \overline{\mathbf{Q}} \subseteq \overline{(-\infty, \infty)} & \text{(by Lemma \ref{9.1.11})}     \\
        \implies & \overline{\mathbf{Q}} \subseteq (-\infty, \infty)            & \text{(by Lemma \ref{9.1.12})}     \\
        \implies & \overline{\mathbf{Q}} \subseteq \mathbf{R}.
    \end{align*}
    Since
    \begin{align*}
                 & \forall\ x \in \mathbf{R}                                                                                            \\
        \implies & \forall\ \varepsilon \in \mathbf{R}^+ : x - \varepsilon < x < x + \varepsilon                                        \\
        \implies & \exists\ q \in \mathbf{Q} : x - \varepsilon < q < x + \varepsilon             & \text{(by Proposition \ref{5.4.14})} \\
        \implies & \abs*{x - q} < \varepsilon                                                                                           \\
        \implies & x \in \overline{\mathbf{Q}},                                                  & \text{(by Definition \ref{9.1.10})}
    \end{align*}
    by Definition \ref{3.1.15} we have \(\mathbf{R} \subseteq \overline{\mathbf{Q}}\).
    Since \(\mathbf{R} \subseteq \overline{\mathbf{Q}} \land \overline{\mathbf{Q}} \subseteq \mathbf{R}\), by Proposition \ref{3.1.18} we have \(\mathbf{R} = \overline{\mathbf{Q}}\).

    Next we show that \(\overline{\mathbf{R}} = \mathbf{R}\).
    Since
    \begin{align*}
             & \forall\ x \in \mathbf{R}                                              \\
        \iff & -\infty < x < \infty                                                   \\
        \iff & x \in (\infty, \infty)            & \text{(by Definition \ref{9.1.1})} \\
        \iff & x \in \overline{(\infty, \infty)} & \text{(by Lemma \ref{9.1.12})}     \\
        \iff & x \in \overline{\mathbf{R}},
    \end{align*}
    we know that \(\overline{\mathbf{R}} = \mathbf{R}\).

    Finally we show that \(\overline{\emptyset} = \emptyset\).
    Suppose for sake of contradiction that \(\overline{\emptyset} \neq \emptyset\).
    Let \(x \in \overline{\emptyset}\)
    Then by Definition \ref{9.1.10} \(\forall\ \varepsilon \in \mathbf{R}^+\), \(\exists\ y \in \emptyset\) such that \(\abs*{x - y} \leq \varepsilon\), a contradiction.
    Thus \(\overline{\emptyset} = \emptyset\).
\end{proof}

\begin{lemma}\label{9.1.14}
    Let \(X\) be a subset of \(\mathbf{R}\), and let \(x \in \mathbf{R}\).
    Then \(x\) is an adherent point of \(X\) if and only if there exists a sequence \((a_n)_{n = 0}^\infty\), consisting entirely of elements in \(X\), which converges to \(x\).
\end{lemma}

\begin{proof}
    We first show that if \(x\) is an adherent point of \(X\), then there exists a sequence \((a_n)_{n = 0}^\infty\) such that \(\forall\ n \in \mathbf{N}\), \(a_n \in X\) and \(\lim_{n \to \infty} a_n = x\).
    For each \(n \in \mathbf{N}\) let \(A_n\) be a set where
    \[
        A_n = \{y \in X : \abs*{x - y} \leq \frac{1}{n}\}.
    \]
    We know by Definition \ref{9.1.10} that \(A_n \neq \emptyset\).
    By axiom of choice (Axiom \ref{8.1}) we know \(\prod_{n \in \mathbf{N}} A_n \neq \emptyset\).
    Let \(f \in \prod_{n \in \mathbf{N}} A_n\).
    We can define a sequence \((a_n)_{n = 0}^\infty\) by setting \(a_n = f(n)\).
    Then we have
    \begin{align*}
                 & \forall\ n \in \mathbf{N}                                                                                           \\
        \implies & 0 \leq \abs*{x - a_n} \leq \frac{1}{n}                                                                              \\
        \implies & \lim_{n \to \infty} \abs*{x - a_n} = 0                                         & \text{(by Corollary \ref{6.4.14})} \\
        \implies & \lim_{n \to \infty} \max(x - a_n, a_n - x) = 0                                                                      \\
        \implies & \max(\lim_{n \to \infty} x - a_n, \lim_{n \to \infty} a_n - x) = 0             & \text{(by Theorem \ref{6.1.19})}   \\
        \implies & \max\big(x - (\lim_{n \to \infty} a_n), (\lim_{n \to \infty} a_n) - x\big) = 0 & \text{(by Theorem \ref{6.1.19})}   \\
        \implies & x = \lim_{n \to \infty} a_n.
    \end{align*}

    Now we show that if there exists a sequence \((a_n)_{n = 0}^\infty\) such that \(\forall\ n \in \mathbf{N}\), \(a_n \in X\) and \(\lim_{n \to \infty} a_n = x\), then \(x\) is an adherent point of \(X\).
    Since \(\lim_{n \to \infty} a_n = x\), by Proposition \ref{6.4.5} \(x\) is the only limit point of \((a_n)_{n = m}^\infty\).
    So we have
    \begin{align*}
                 & \forall\ \varepsilon \in \mathbf{R}^+, \exists\ n \in \mathbf{N} : \abs*{x - a_n} \leq \varepsilon & \text{(by Definition \ref{6.4.1})}  \\
        \implies & \forall\ \varepsilon \in \mathbf{R}^+, \exists\ a_n \in X : \abs*{x - a_n} \leq \varepsilon                                              \\
        \implies & x \in \overline{X}.                                                                                & \text{(by Definition \ref{9.1.10})}
    \end{align*}
    And we conclude that \(x\) is an adherent point of \(X\) if and only if there exists a sequence \((a_n)_{n = 0}^\infty\) such that \(\forall\ n \in \mathbf{N}\), \(a_n \in X\) and \(\lim_{n \to \infty} a_n = x\).
\end{proof}

\begin{definition}\label{9.1.15}
    A subset \(E \subseteq \mathbf{R}\) is said to be \emph{closed} if \(\overline{E} = E\), or in other words that \(E\) contains all of its adherent points.
\end{definition}

\begin{example}\label{9.1.16}
    From Lemma \ref{9.1.12} we see that if \(a < b\) are real numbers, then \([a, b]\), \([a, +\infty)\), \((-\infty, a]\), and \((-\infty, +\infty)\) are closed, while \((a, b)\), \((a, b]\), \([a, b)\), \((a, +\infty)\), and \((-\infty, a)\) are not.
    From Lemma \ref{9.1.13} we see that \(\mathbf{N}\), \(\mathbf{Z}\), \(\mathbf{R}\), \(\emptyset\) are closed, while \(\mathbf{Q}\) is not.
\end{example}

\begin{corollary}\label{9.1.17}
    Let \(X\) be a subset of \(\mathbf{R}\).
    If \(X\) is closed, and \((a_n)_{n = 0}^\infty\) is a convergent sequence consisting of elements in \(X\), then \(\lim_{n \to \infty} a_n\) also lies in \(X\).
    Conversely, if it is true that every convergent sequence \((a_n)_{n = 0}^\infty\) of elements in \(X\) has its limit in \(X\) as well, then \(X\) is necessarily closed.
\end{corollary}

\begin{proof}
    We first show that if \(X\) is closed, and \((a_n)_{n = 0}^\infty\) is a convergent sequence consisting of elements in \(X\), then \(\lim_{n \to \infty} a_n\) also lies in \(X\).
    Let \(x = \lim_{n \to \infty} a_n\).
    Then we have
    \begin{align*}
                 & \forall\ \varepsilon \in \mathbf{R}^+, \exists\ n \in \mathbf{N} : \abs*{x - a_{n'}} \leq \varepsilon \ \forall\ n' \in \mathbf{N} \land n' \geq n                                       \\
        \implies & \forall\ \varepsilon \in \mathbf{R}^+, \exists\ a_n \in X : \abs*{x - a_n} \leq \varepsilon                                                                                              \\
        \implies & x \in \overline{X}                                                                                                                                 & \text{(by Definition \ref{9.1.10})} \\
        \implies & x \in X.                                                                                                                                           & \text{(by Definition \ref{9.1.15})}
    \end{align*}

    Now we show that if every convergent sequence \((a_n)_{n = 0}^\infty\) of elements in \(X\) has its limit in \(X\) as well, then \(X\) is closed.
    By Lemma \ref{9.1.11} we have \(X \subseteq \overline{X}\).
    Since
    \begin{align*}
                 & \forall\ x \in \overline{X}                                                                                                                   \\
        \implies & \exists\ (a_n)_{n = 0}^\infty : (\forall\ n \in \mathbf{N}, a_n \in X) \land (\lim_{n \to \infty} a_n = x) & \text{(by Lemma \ref{9.1.14})}   \\
        \implies & x \in X,                                                                                                   & \text{(by the given hypothesis)}
    \end{align*}
    by Definition \ref{3.1.15} we have \(\overline{X} \subseteq X\).
    Since \(X \subseteq \overline{X} \land \overline{X} \subseteq X\), by Proposition \ref{3.1.8} we have \(X = \overline{X}\), and thus by Definition \ref{9.1.15} \(X\) is closed.
\end{proof}

\begin{definition}[Limit points]\label{9.1.18}
    Let \(X\) be a subset of the real line.
    We say that \(x\) is a \emph{limit point} (or a \emph{cluster point}) of \(X\) iff it is an adherent point of \(X \setminus \{x\}\).
    We say that \(x\) is an \emph{isolated point} of \(X\) if \(x \in X\) and there exists some \(\varepsilon > 0\) such that \(\abs*{x - y} > \varepsilon\) for all \(y \in X \setminus \{x\}\).
\end{definition}

\setcounter{theorem}{19}
\begin{remark}\label{9.1.20}
    From Lemma \ref{9.1.14} we see that \(x\) is a limit point of \(X\) iff there exists a sequence \((a_n)_{n = 0}^\infty\), consisting entirely of elements in \(X\) that are distinct from \(x\), and such that \((a_n)_{n = 0}^\infty\) converges to \(x\).
    It turns out that the set of adherent points splits into the set of limit points and the set of isolated points.
\end{remark}

\begin{lemma}\label{9.1.21}
    Let \(I\) be an interval (possibly infinite), i.e., \(I\) is a set of the form \((a, b)\), \((a, b]\), \([a, b)\), \([a, b]\), \((a, +\infty)\), \([a, +\infty)\), \((-\infty, a)\), or \((-\infty, a]\), with \(a < b\) in the first four cases.
    Then every element of \(I\) is a limit point of \(I\).
\end{lemma}

\begin{proof}
    We show this for the case \(I = [a, b]\);
    the other cases are similar.
    Let \(x \in I\);
    we have to show that \(x\) is a limit point of \(I\).
    There are three cases: \(x = a\), \(a < x < b\), and \(x = b\).
    If \(x = a\), then consider the sequence \((x + \frac{1}{n})_{n = N}^\infty\).
    This sequence converges to \(x\), and will lie inside \(I - \{a\} = (a, b]\) if \(N\) is chosen large enough (by Proposition \ref{5.4.14}).
    Thus by Remark \ref{9.1.20} we see that \(x = a\) is a limit point of \([a, b]\).
    A similar argument works when \(a < x < b\).
    When \(x = b\) one has to use the sequence \((x - \frac{1}{n})_{n = N}^\infty\) instead.
    This sequence converges to \(x\), and will lie inside \(I - \{b\} = [a, b)\) if \(N\) is chosen large enough (by Proposition \ref{5.4.14}).
    Thus by Remark \ref{9.1.20} we see that \(x = b\) is a limit point of \([a, b]\).
\end{proof}

\begin{definition}[Bounded sets]\label{9.1.22}
    A subset \(X\) of the real line is said to be \emph{bounded} if we have \(X \subseteq [-M, M]\) for some real number \(M > 0\).
\end{definition}

\begin{example}\label{9.1.23}
    For any real numbers \(a, b\), the interval \([a, b]\) is bounded, because it is contained inside \([-M, M]\), where \(M \coloneqq \max(\abs*{a}, \abs*{b})\).
    However, the half-infinite interval \([0, +\infty)\) is unbounded.
    In fact, no half-infinite interval or doubly infinite interval can be bounded.
    The sets \(\mathbf{N}\), \(\mathbf{Z}\), \(\mathbf{Q}\), and \(\mathbf{R}\) are all unbounded.
\end{example}

\begin{theorem}[Heine-Borel theorem for the line]\label{9.1.24}
    Let \(X\) be a subset of \(\mathbf{R}\).
    Then the following two statements are equivalent:
    \begin{enumerate}
        \item \(X\) is closed and bounded.
        \item Given any sequence \((a_n)_{n = 0}^\infty\) of real numbers which takes values in \(X\) (i.e., \(a_n \in X\) for all \(n\)), there exists a subsequence \((a_{n_j})_{j = 0}^\infty\) of the original sequence, which converges to some number \(L\) in \(X\).
    \end{enumerate}
\end{theorem}

\begin{proof}
    We first show that statement (a) implies statement (b).
    Suppose that \(X\) is a set such that \(X\) is closed and bounded.
    Let \((a_n)_{n = 0}^\infty\) be a sequence where \(\forall\ n \in \mathbf{N}\), \(a_n \in X\).
    Since \(X\) is bounded, by Definition \ref{9.1.22} \(\exists\ M \in \mathbf{R}^+\) such that \(X \subseteq [-M, M]\), thus \((a_n)_{n = 0}^\infty\) is also bounded by \(M\), i.e., \(\forall\ n \in \mathbf{N}\), \(\abs*{a_n} \leq M\).
    By Bolzano-Weierstrass theorem (Theorem \ref{6.6.8}) we know that there exists a subsequence \((a_{n_j})_{j = 0}^\infty\) of \((a_n)_{n = 0}^\infty\) such that \((a_{n_j})_{j = 0}^\infty\) converges.
    Since \(X\) is closed, by Corollary \ref{9.1.17} we know that \(\lim_{j \to \infty} a_{n_j} \in X\).

    Now we show that statement (b) implies statement (a).
    Suppose for sake of contradiction that statement (b) does not implies statement (a).
    Then \(X\) is not closed or \(X\) is unbounded.
    Since given any sequence \((a_n)_{n = 0}^\infty\) we can always find a subsequence \((a_{n_j})_{j = 0}^\infty\) such that \(\lim_{j \to \infty} a_{n_j} \in X\), we know that if \((a_n)_{n = 0}^\infty\) converges then \(\lim_{n \to \infty} a_n \in X\).
    Thus we have every convergent sequence \((a_n)_{n = 0}^\infty\) have its limit in \(X\), and by Corollary \ref{9.1.17} we know that \(X\) is closed.
    Then we must have \(X\) is unbounded, i.e., \(\nexists\ M \in \mathbf{R}^+\) such that \(X \subseteq [-M, M]\).
    Now we define \(X_n = \{x \in X : \abs*{x} > n\}\) for every \(n \in \mathbf{N}\).
    We know that \(X_n \neq \emptyset\) since \(X_n\) is unbounded.
    By axiom of choice (Axiom \ref{8.1}) we know that \(\prod_{n \in \mathbf{N}} X_n \neq \emptyset\).
    Let \(f \in \prod_{n \in \mathbf{N}} X_n\).
    We can define a sequence \((a_n)_{n = 0}^\infty\) by setting \(a_n = f(n)\).
    By hypothesis we know that there exists a subsequence \((a_{n_j})_{j = 0}^\infty\) such that \(L = \lim_{j \to \infty} a_{n_j} \in X\).
    But this means
    \begin{align*}
                 & \forall\ n \in \mathbf{N} \land n \geq \abs*{L} + 1         & \text{(by Proposition \ref{5.4.12})} \\
        \implies & \abs*{a_n} > n \geq \abs*{L} + 1                            & (a_n \in X_n)                        \\
        \implies & \forall\ n_j > \abs*{L} + 1 : \abs*{a_{n_j}} > \abs*{L} + 1                                        \\
        \implies & \abs*{a_{n_j}} - \abs*{L} > 1                                                                      \\
        \implies & \abs*{a_{n_j} - L} \geq \abs*{a_{n_j}} - \abs*{L} > 1                                              \\
        \implies & \lim_{j \to \infty} a_{n_j} \neq L,
    \end{align*}
    a contradiction.
    Thus \(X\) is closed and bounded.
\end{proof}

\exercisesection

\begin{exercise}\label{ex 9.1.1}
    Let \(X\) be any subset of the real line, and let \(Y\) be a set such that \(X \subseteq Y \subseteq \overline{X}\).
    Show that \(\overline{Y} = \overline{X}\).
\end{exercise}

\begin{proof}
    By Lemma \ref{9.1.11} we have \(X \subseteq Y \implies \overline{X} \subseteq \overline{Y}\).
    Since
    \begin{align*}
                 & \forall\ y \in \overline{Y}                                                                                                                        \\
        \implies & \forall\ \varepsilon \in \mathbf{R}^+, \exists\ x \in Y : \abs*{y - x} \leq \frac{\varepsilon}{2}            & \text{(by Definition \ref{9.1.10})} \\
        \implies & \forall\ \varepsilon \in \mathbf{R}^+, \exists\ x \in \overline{X} : \abs*{y - x} \leq \frac{\varepsilon}{2} & (Y \subseteq \overline{X})          \\
        \implies & \forall\ \varepsilon \in \mathbf{R}^+, \exists\ z \in X : \abs*{x - z} \leq \frac{\varepsilon}{2}            & \text{(by Definition \ref{9.1.10})} \\
        \implies & \abs*{y - x} + \abs*{x - z} \leq \frac{\varepsilon}{2} + \frac{\varepsilon}{2}                                                                     \\
        \implies & \abs*{y - x + x - z} \leq \abs*{y - x} + \abs*{x - z} \leq \varepsilon                                                                             \\
        \implies & \abs*{y - z} \leq \varepsilon                                                                                                                      \\
        \implies & y \in \overline{X},                                                                                          & \text{(by Definition \ref{9.1.10})}
    \end{align*}
    by Definition \ref{3.1.15} we know that \(\overline{Y} \subseteq \overline{X}\).
    Since \(\overline{Y} \subseteq \overline{X} \land \overline{X} \subseteq \overline{Y}\), by Proposition \ref{3.1.18} we have \(\overline{Y} = \overline{X}\).
\end{proof}

\begin{exercise}\label{ex 9.1.2}
    Prove Lemma \ref{9.1.11}.
\end{exercise}

\begin{proof}
    See Lemma \ref{9.1.11}.
\end{proof}

\begin{exercise}\label{ex 9.1.3}
    Prove Lemma \ref{9.1.13}.
\end{exercise}

\begin{proof}
    See Lemma \ref{9.1.13}.
\end{proof}

\begin{exercise}\label{ex 9.1.4}
    Give an example of two subsets \(X, Y\) of the real line such that \(\overline{X \cap Y} \neq \overline{X} \cap \overline{Y}\).
\end{exercise}

\begin{proof}
    Let \(X = [0, 0.5)\) and \(Y = (0.5, 1]\).
    By Lemma \ref{9.1.12} we have \(\overline{X} = [0, 0.5]\) and \(\overline{Y} = [0.5, 1]\), so \(\overline{X} \cap \overline{Y} = \{0.5\}\).
    By Lemma \ref{9.1.13} we have \(\overline{X \cap Y} = \overline{\emptyset} = \emptyset\).
    Thus \(\overline{X \cap Y} \neq \overline{X} \cap \overline{Y}\).
\end{proof}

\begin{exercise}\label{ex 9.1.5}
    Prove Lemma \ref{9.1.14}.
\end{exercise}

\begin{proof}
    See Lemma \ref{9.1.14}.
\end{proof}

\begin{exercise}\label{ex 9.1.6}
    Let \(X\) be a subset of \(\mathbf{R}\).
    Show that \(X\) is closed (i.e., \(\overline{\overline{X}} = \overline{X}\)).
    Furthermore, show that if \(Y\) is any closed set that contains \(X\), then \(Y\) also contains \(\overline{X}\).
    Thus the closure \(\overline{X}\) of \(X\) is the smallest closed set which contains \(X\).
\end{exercise}

\begin{proof}
    We first show that \(X \subseteq \mathbf{R} \implies \overline{\overline{X}} = \overline{X}\).
    We have
    \begin{align*}
                 & X \subseteq \mathbf{R}                                                           \\
        \implies & \overline{X} \subseteq \overline{\mathbf{R}}    & \text{(by Lemma \ref{9.1.11})} \\
        \implies & \overline{X} \subseteq \mathbf{R}               & \text{(by Lemma \ref{9.1.13})} \\
        \implies & \overline{X} \subseteq \overline{\overline{X}}. & \text{(by Lemma \ref{9.1.11})}
    \end{align*}
    Since
    \begin{align*}
                 & \forall\ x \in \overline{\overline{X}}                                                                                                             \\
        \implies & \forall\ \varepsilon \in \mathbf{R}^+, \exists\ y \in \overline{X} : \abs*{x - y} \leq \frac{\varepsilon}{2} & \text{(by Definition \ref{9.1.10})} \\
        \implies & \forall\ \varepsilon \in \mathbf{R}^+, \exists\ z \in X : \abs*{y - z} \leq \frac{\varepsilon}{2}            & \text{(by Definition \ref{9.1.10})} \\
        \implies & \abs*{x - y} + \abs*{y - z} \leq \frac{\varepsilon}{2} + \frac{\varepsilon}{2}                                                                     \\
        \implies & \abs*{x - y + y - z} \leq \abs*{x - y} + \abs*{y - z} \leq \varepsilon                                                                             \\
        \implies & \abs*{x - z} \leq \varepsilon                                                                                                                      \\
        \implies & x \in \overline{X},                                                                                          & \text{(by Definition \ref{9.1.10})}
    \end{align*}
    by Definition \ref{3.1.15} we know that \(\overline{\overline{X}} \subseteq \overline{X}\).
    Since \(\overline{\overline{X}} \subseteq \overline{X} \land \overline{X} \subseteq \overline{\overline{X}}\), by Proposition \ref{3.1.18} we have \(\overline{\overline{X}} = \overline{X}\).
    So by Definition \ref{9.1.15} \(\overline{X}\) is closed.

    Now we show that if \(Y\) is any closed set that contains \(X\), then \(Y\) also contains \(\overline{X}\).
    \begin{align*}
                 & (X \subseteq Y) \land (Y = \overline{Y})                       & \text{(by Definition \ref{9.1.15})} \\
        \implies & (\overline{X} \subseteq \overline{Y}) \land (Y = \overline{Y}) & \text{(by Lemma \ref{9.1.11})}      \\
        \implies & \overline{X} \subseteq Y.
    \end{align*}
\end{proof}

\begin{exercise}\label{ex 9.1.7}
    Let \(n \geq 1\) be a positive integer, and let \(X_1, \dots, X_n\) be closed subsets of \(\mathbf{R}\).
    Show that \(X_1 \cup X_2 \cup \dots \cup X_n\) is also closed.
\end{exercise}

\begin{proof}
    Suppose that \(\forall\ m \in \mathbf{N}\) we have \(X_m\) is a closed subset of \(\mathbf{R}\).
    We use induction on \(n\) to show that \(X_1 \cup \dots \cup X_n\) is closed and we start with \(n = 1\).
    For \(n = 1\), by the given hypothesis we have \(X_1\) is closed.
    So the base case holds.
    Suppose inductively that for some \(n \geq 1\) we have \(X_1 \cup \dots \cup X_n\) is closed.
    Then for \(n + 1\), we have
    \begin{align*}
          & \overline{X_1 \cup \dots \cup X_n \cup X_{n + 1}}                                                        \\
        = & \overline{(X_1 \cup \dots \cup X_n) \cup X_{n + 1}}            & \text{(by Proposition \ref{3.1.28}(e))} \\
        = & \overline{(X_1 \cup \dots \cup X_n)} \cup \overline{X_{n + 1}} & \text{(by Lemma \ref{9.1.11})}          \\
        = & (X_1 \cup \dots \cup X_n) \cup \overline{X_{n + 1}}            & \text{(by induction hypothesis)}        \\
        = & (X_1 \cup \dots \cup X_n) \cup X_{n + 1}                       & \text{(by the given hypothesis)}        \\
        = & X_1 \cup \dots \cup X_n \cup X_{n + 1}.                        & \text{(by Proposition \ref{3.1.28}(e))}
    \end{align*}
    This close the induction.
    Thus \(\forall\ n \in \mathbf{N}\), if \(X_1, \dots, X_n\) are closed subset of \(\mathbf{R}\), then \(X_1 \cup \dots \cup X_n\) is also closed.
\end{proof}

\begin{exercise}\label{ex 9.1.8}
    Let \(I\) be a set (possibly infinite), and for each \(\alpha \in I\) let \(X_{\alpha}\) be a closed subset of \(\mathbf{R}\).
    Show that the intersection \(\bigcap_{\alpha \in I} X_{\alpha}\) is also closed.
\end{exercise}

\begin{exercise}\label{ex 9.1.9}
    Let \(X\) be a subset of the real line, and \(x\) be a real number.
    Show that every adherent point of \(X\) is either a limit point or an isolated point of \(X\), but cannot be both.
    Conversely, show that every limit point and every isolated point of \(X\) is an adherent point of \(X\).
\end{exercise}

\begin{exercise}\label{ex 9.1.10}
    If \(X\) is a non-empty subset of \(\mathbf{R}\), show that \(X\) is bounded if and only if \(\inf(X)\) and \(\sup(X)\) are finite.
\end{exercise}

\begin{exercise}\label{ex 9.1.11}
    Show that if \(X\) is a bounded subset of \(\mathbf{R}\), then the closure \(X\) is also bounded.
\end{exercise}

\begin{exercise}\label{ex 9.1.12}
    Show that the union of any finite collection of bounded subsets of \(\mathbf{R}\) is still a bounded set.
    Is this conclusion still true if one takes an infinite collection of bounded subsets of \(\mathbf{R}\)?
\end{exercise}

\begin{exercise}\label{ex 9.1.13}
    Prove Theorem \ref{9.1.24}.
\end{exercise}

\begin{proof}
    See Theorem \ref{9.1.24}.
\end{proof}

\begin{exercise}\label{ex 9.1.14}
    Show that any finite subset of \(\mathbf{R}\) is closed and bounded.
\end{exercise}

\begin{exercise}\label{ex 9.1.15}
    Let \(E\) be a non-empty bounded subset of \(\mathbf{R}\), and let \(S \coloneqq \sup(E)\) be the least upper bound of \(E\).
    (Note from the least upper bound principle, Theorem \ref{5.5.9}, that \(S\) is a real number.)
    Show that \(S\) is an adherent point of \(E\), and is also an adherent point of \(\mathbf{R} \setminus E\).
\end{exercise}