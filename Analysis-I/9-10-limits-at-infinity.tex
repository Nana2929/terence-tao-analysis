\section{Limits at infinity}\label{sec 9.10}

\begin{definition}[Infinite adherent points]\label{9.10.1}
    Let \(X\) be a subset of \(\mathbf{R}\).
    We say that \(+\infty\) is \emph{adherent} to \(X\) iff for every \(M \in \mathbf{R}\) there exists an \(x \in X\) such that \(x > M\);
    we say that \(-\infty\) is \emph{adherent} to \(X\) iff for every \(M \in \mathbf{R}\) there exists an \(x \in X\) such that \(x < M\).
\end{definition}

\begin{note}
    In other words, \(+\infty\) is adherent to \(X\) iff \(X\) has no upper bound, or equivalently iff \(\sup(X) = +\infty\).
    Similarly \(-\infty\) is adherent to \(X\) iff \(X\) has no lower bound, or iff \(\inf(X) = -\infty\).
    Thus a set is bounded if and only if \(+\infty\) and \(-\infty\) are not adherent points.
\end{note}

\begin{remark}\label{9.10.2}
    Definition \ref{9.10.1} may seem rather different from Definition \ref{9.1.8}, but can be unified using the topological structure of the extended real line \(\mathbf{R}^*\).
\end{remark}

\begin{definition}[Limits at infinity]\label{9.10.3}
    Let \(X\) be a subset of \(\mathbf{R}\) with \(+\infty\) as an adherent point, and let \(f : X \to \mathbf{R}\) be a function.
    We say that \emph{\(f(x)\) converges to \(L\)} as \(x \to +\infty\) in \(X\), and write \(\lim_{x \to +\infty ; x \in X} f(x) = L\), iff for every \(\varepsilon > 0\) there exists an \(M\) such that \(f\) is \(\varepsilon\)-close to \(L\) on \(X \cap (M, +\infty)\)
    (i.e., \(\abs*{f(x) - L} \leq \varepsilon\) for all \(x \in X\) such that \(x > M\)).
    Similarly we say that \emph{\(f(x)\) converges to \(L\)} as \(x \to -\infty\) iff for every \(\varepsilon > 0\) there exists an \(M\) such that \(f\) is \(\varepsilon\)-close to \(L\) on \(X \cap (-\infty, M)\).
\end{definition}

\begin{note}
    One can do many of the same things with these limits at infinity as we have been doing with limits at other points \(x_0\);
    for instance, it turns out that all of the limit laws continue to hold.
    However, as we will not be using these limits much in this text, we will not devote much attention to these matters.
    We will note though that this definition is consistent with the notion of a limit \(\lim_{n \to \infty} a_n\) of a sequence.
\end{note}

\exercisesection

\begin{exercise}\label{ex 9.10.1}
    Let \((a_n)_{n = 0}^\infty\) be a sequence of real numbers, then \(a_n\) can also be thought of as a function from \(\mathbf{N}\) to \(\mathbf{R}\), which takes each natural number \(n\) to a real number \(a_n\).
    Show that
    \[
        \lim_{n \to +\infty ; n \in \mathbf{N}} a_n = \lim_{n \to \infty} a_n
    \]
    where the left-hand limit is defined by Definition \ref{9.10.3} and the right-hand limit is defined by Definition \ref{6.1.8}.
    More precisely, show that if one of the above two limits exists then so does the other, and then they both have the same value.
    Thus the two notions of limit here are compatible.
\end{exercise}

