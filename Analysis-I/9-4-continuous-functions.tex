\section{Continuous functions}\label{sec 9.4}

\begin{definition}[Continuity]\label{9.4.1}
    Let \(X\) be a subset of \(\mathbf{R}\), and let \(f : X \to \mathbf{R}\) be a function.
    Let \(x_0\) be an element of \(X\).
    We say that \(f\) is \emph{continuous at \(x_0\)} iff we have
    \[
        \lim_{x \to x_0 ; x \in X} f(x) = f(x_0);
    \]
    in other words, the limit of \(f(x)\) as \(x\) converges to \(x_0\) in \(X\) exists and is equal to \(f(x_0)\).
    We say that \(f\) is \emph{continuous on \(X\)} (or simply \emph{continuous}) iff \(f\) is continuous at \(x_0\) for every \(x_0 \in X\).
    We say that \(f\) is \emph{discontinuous at \(x_0\)} iff it is not continuous at \(x_0\).
\end{definition}

\begin{note}
    Restricting the domain of a function can make a discontinuous function continuous again.
\end{note}

\setcounter{theorem}{6}
\begin{proposition}[Equivalent formulations of continuity]\label{9.4.7}
    Let \(X\) be a subset of \(\mathbf{R}\), let \(f : X \to \mathbf{R}\) be a function, and let \(x_0\) be an element of \(X\).
    Then the following four statements are logically equivalent:
    \begin{enumerate}
        \item \(f\) is continuous at \(x_0\).
        \item For every sequence \((a_n)_{n = 0}^\infty\) consisting of elements of \(X\) with \(\lim_{n \to \infty} a_n = x_0\), we have \(\lim_{n \to \infty} f(a_n) = f(x_0)\).
        \item For every \(\varepsilon > 0\), there exists a \(\delta > 0\) such that \(\abs*{f(x) - f(x_0)} < \varepsilon\) for all \(x \in X\) with \(\abs*{x - x_0} < \delta\).
        \item For every \(\varepsilon > 0\), there exists a \(\delta > 0\) such that \(\abs*{f(x) - f(x_0)} \leq \varepsilon\) for all \(x \in X\) with \(\abs*{x - x_0} \leq \delta\).
    \end{enumerate}
\end{proposition}

\begin{proof}
    We first show that the statement (a) and the statement (b) are equivalent.
    By Definition \ref{9.4.1}, \(f\) is continuous at \(x_0\) iff \(f\) converges to \(f(x_0)\) at \(x_0\) in \(X\).
    Thus by Proposition \ref{9.3.9} we know that the statement (a) and the statement (b) are equivalent.

    Next we show that the statement (a) and the statement (c) are equivalent.
    By Definition \ref{9.4.1}, \(f\) is continuous at \(x_0\) iff \(f\) converges to \(f(x_0)\) at \(x_0\) in \(X\).
    By Definition \ref{9.3.6} this is equivalent to the statement
    \[
        \forall\ \varepsilon' \in \mathbf{R}^+, \exists\ \delta \in \mathbf{R}^+ : \bigg(\forall\ x \in X, \abs*{x - x_0} < \delta \implies \abs*{f(x) - f(x_0)} \leq \varepsilon'\bigg).
    \]
    Let \(\varepsilon \in \mathbf{R}^+ \land \varepsilon > \varepsilon'\).
    Then we have
    \[
        \forall\ \varepsilon \in \mathbf{R}^+, \exists\ \delta \in \mathbf{R}^+ : \bigg(\forall\ x \in X, \abs*{x - x_0} < \delta \implies \abs*{f(x) - f(x_0)} \leq \varepsilon' < \varepsilon\bigg).
    \]
    Thus the statement (a) and the statement (c) are equivalent.

    Finally we show that the statement (a) and the statement (d) are equivalent.
    By Definition \ref{9.4.1}, \(f\) is continuous at \(x_0\) iff \(f\) converges to \(f(x_0)\) at \(x_0\) in \(X\).
    By Definition \ref{9.3.6} this is equivalent to the statement
    \[
        \forall\ \varepsilon \in \mathbf{R}^+, \exists\ \delta' \in \mathbf{R}^+ : \bigg(\forall\ x \in X, \abs*{x - x_0} < \delta' \implies \abs*{f(x) - f(x_0)} \leq \varepsilon\bigg).
    \]
    Let \(\delta \in \mathbf{R}^+ \land \abs*{x - x_0} \leq \delta < \delta'\).
    By Proposition \ref{5.4.14} we know such \(\delta\) exists.
    Then we have
    \[
        \forall\ \varepsilon \in \mathbf{R}^+, \exists\ \delta \in \mathbf{R}^+ : \bigg(\forall\ x \in X, \abs*{x - x_0} \leq \delta \implies \abs*{f(x) - f(x_0)} \leq \varepsilon\bigg).
    \]
    Thus the statement (a) and the statement (d) are equivalent.
\end{proof}

\begin{remark}\label{9.4.8}
    A particularly useful consequence of Proposition \ref{9.4.7} is the following:
    if \(f\) is continuous at \(x_0\), and \(a_n \to x_0\) as \(n \to \infty\), then \(f(a_n) \to f(x_0)\) as \(n \to \infty\)
    (provided that all the elements of the sequence \((a_n)_{n = 0}^\infty\) lie in the domain of \(f\), of course).
    Thus continuous functions are very useful in computing limits.
\end{remark}

\begin{proposition}[Arithmetic preserves continuity]\label{9.4.9}
    Let \(X\) be a subset of \(\mathbf{R}\), and let \(f : X \to \mathbf{R}\) and \(g : X \to \mathbf{R}\) be functions.
    Let \(x_0 \in X\).
    Then if \(f\) and \(g\) are both continuous at \(x_0\), then the functions \(f + g\), \(f - g\), \(\max(f, g)\), \(\min(f, g)\) and \(fg\) are also continuous at \(x_0\).
    If \(g\) is non-zero on \(X\), then \(f / g\) is also continuous at \(x_0\).
\end{proposition}

\begin{proof}
    By Proposition \ref{9.3.14}, we have
    \begin{align*}
        \lim_{x \to x_0 ; x \in X} f(x) + g(x)      & = \lim_{x \to x_0 ; x \in X} f(x) + \lim_{x \to x_0 ; x \in X} g(x)                                                            \\
                                                    & = f(x_0) + g(x_0);                                                                       & \text{(by Definition \ref{9.4.1})}  \\
        \lim_{x \to x_0 ; x \in X} f(x) - g(x)      & = \lim_{x \to x_0 ; x \in X} f(x) - \lim_{x \to x_0 ; x \in X} g(x)                                                            \\
                                                    & = f(x_0) - g(x_0);                                                                       & \text{(by Definition \ref{9.4.1})}  \\
        \lim_{x \to x_0 ; x \in X} \max(f(x), g(x)) & = \max(\lim_{x \to x_0 ; x \in X} f(x), \lim_{x \to x_0 ; x \in X} g(x))                                                       \\
                                                    & = \max(f(x_0), g(x_0));                                                                  & \text{(by Definition \ref{9.4.1})}  \\
        \lim_{x \to x_0 ; x \in X} \min(f(x), g(x)) & = \min(\lim_{x \to x_0 ; x \in X} f(x), \lim_{x \to x_0 ; x \in X} g(x))                                                       \\
                                                    & = \min(f(x_0), g(x_0));                                                                  & \text{(by Definition \ref{9.4.1})}  \\
        \lim_{x \to x_0 ; x \in X} f(x) g(x)        & = \bigg(\lim_{x \to x_0 ; x \in X} f(x)\bigg)\bigg(\lim_{x \to x_0 ; x \in X} g(x)\bigg)                                       \\
                                                    & = f(x_0) g(x_0);                                                                         & \text{(by Definition \ref{9.4.1})}  \\
        \lim_{x \to x_0 ; x \in X} f(x) / g(x)      & = \lim_{x \to x_0 ; x \in X} f(x) / \lim_{x \to x_0 ; x \in X} g(x)                      & \text{(\(g\) is non-zero on \(X\))} \\
                                                    & = f(x_0) / g(x_0).                                                                       & \text{(by Definition \ref{9.4.1})}  \\
    \end{align*}
\end{proof}

\begin{proposition}[Exponentiation is continuous, I]\label{9.4.10}
    Let \(a > 0\) be a positive real number.
    Then the function \(f : \mathbf{R} \to \mathbf{R}\) defined by \(f(x) \coloneqq a^x\) is continuous.
\end{proposition}

\begin{proof}
    If \(a = 1\), then by Exercise \ref{ex 9.4.2} we know that
    \[
        \forall\ x_0 \in X, \lim_{x \to x_0 ; x \in \mathbf{R}} 1^x = \lim_{x \to x_0 ; x \in \mathbf{R}} 1 = 1.
    \]
    So suppose that \(a \neq 1\).
    Let \(x_0 \in \mathbf{R}\).
    By Lemma \ref{9.1.13}, \(x_0\) is an adherent point.
    By Proposition \ref{6.7.3} we know that \(a^{x_0} > 0\).
    By Lemma \ref{6.5.3}, we know that \(\lim_{n \to \infty} a^{1 / n} = 1\).
    This means
    \[
        \forall\ \varepsilon \in \mathbf{R}^+, \exists\ n \geq 0 : \abs*{a^{1 / N} - 1} \leq \varepsilon \ \forall\ N \geq n.
    \]
    In particular, we have
    \[
        \forall\ \varepsilon \in \mathbf{R}^+, \exists\ n \geq 0 : \abs*{a^{1 / N} - 1} \leq \varepsilon / a^{x_0} \ \forall\ N \geq n.
    \]
    Now we fix \(\varepsilon\) and \(n\).
    Let \(x \in \mathbf{R} \land x - x_0 \in [0, 1 / n)\).
    By Exercise \ref{ex 5.4.3} \(\exists\ k \geq n\) such that \(1 / (k + 1) < x - x_0 < 1 / k\).
    Now we split into two cases:
    \begin{enumerate}
        \item If \(a > 1\), then by Proposition \ref{6.7.3} we have
              \begin{align*}
                           & 1 / (k + 1) < x - x_0 < 1 / k                                                                \\
                  \implies & a^{1 / (k + 1)} < a^{x - x_0} < a^{1 / k}                                                    \\
                  \implies & a^{1 / (k + 1)} - 1 < a^{x - x_0} - 1 < a^{1 / k} - 1                                        \\
                  \implies & \bigg(\abs*{a^{x - x_0} - 1} < \abs*{a^{1 / k} - 1} < \varepsilon / a^{x_0}\bigg)            \\
                           & \lor \bigg(\abs*{a^{x - x_0} - 1} < \abs*{a^{1 / (k + 1)} - 1} < \varepsilon / a^{x_0}\bigg) \\
                  \implies & \abs*{a^{x - x_0} - 1} \leq \varepsilon / a^{x_0}                                            \\
                  \implies & \abs*{a^x - a^{x_0}} = a^{x_0} \abs*{a^{x - x_0} - 1} \leq \varepsilon                       \\
              \end{align*}
        \item if \(a < 1\), then by Proposition \ref{6.7.3} we have
              \begin{align*}
                           & 1 / (k + 1) < x - x_0 < 1 / k                                                                \\
                  \implies & a^{1 / k} < a^{x - x_0} < a^{1 / (k + 1)}                                                    \\
                  \implies & a^{1 / k} - 1 < a^{x - x_0} - 1 < a^{1 / (k + 1)} - 1                                        \\
                  \implies & \bigg(\abs*{a^{x - x_0} - 1} < \abs*{a^{1 / k} - 1} < \varepsilon / a^{x_0}\bigg)            \\
                           & \lor \bigg(\abs*{a^{x - x_0} - 1} < \abs*{a^{1 / (k + 1)} - 1} < \varepsilon / a^{x_0}\bigg) \\
                  \implies & \abs*{a^{x - x_0} - 1} \leq \varepsilon / a^{x_0}                                            \\
                  \implies & \abs*{a^x - a^{x_0}} = a^{x_0} \abs*{a^{x - x_0} - 1} \leq \varepsilon                       \\
              \end{align*}
    \end{enumerate}
    From all cases above we have \(\abs*{a^x - a^{x_0}} \leq \varepsilon\).
    Now let \(\delta = 1 / n\).
    Then we have
    \[
        \forall\ \varepsilon \in \mathbf{R}^+, \exists\ \delta \in \mathbf{R}^+ : \bigg(\forall\ x \in \mathbf{R} \land \abs*{x - x_0} < \delta \implies \abs*{a^x - a^{x_0}} \leq \varepsilon\bigg).
    \]
    Thus by Definition \ref{9.3.6} we have \(\lim_{x \to x_0 ; x \in \mathbf{R}} a^x = a^{x_0}\).
    Since \(x_0\) is arbitrary, by Definition \ref{9.4.1} \(a^x\) is continuous on \(\mathbf{R}\).
\end{proof}

\begin{proposition}[Exponentiation is continuous, II]\label{9.4.11}
    Let \(p\) be a real number.
    Then the function \(f : (0, \infty) \to \mathbf{R}\) defined by \(f(x) \coloneqq x^p\) is continuous.
\end{proposition}

\begin{proof}
    By Proposition \ref{9.3.14} we know that
    \[
        \forall\ n \in \mathbf{N}, \lim_{x \to 1 ; x \in (0, \infty)} x^n = 1.
    \]
    Again by Proposition \ref{9.3.14} we know that
    \[
        \forall\ n \in \mathbf{N}, \lim_{x \to 1 ; x \in (0, \infty)} x^{-n} = \lim_{x \to 1 ; x \in (0, \infty)} 1 / x^n = 1.
    \]
    By Exercise \ref{ex 5.4.3}, \(\forall\ p \in \mathbf{R}\), \(\exists\ N \in \mathbf{Z}\) such that \(N \leq p < N + 1\).
    \(\forall\ x \in (0, \infty)\), we have either \(x^N \leq x^p < x^{N + 1}\) (when \(x \in (0, 1)\)) or \(x^{N + 1} < x^p \leq x^{N}\) (when \(x \in [1, \infty)\)).
    Thus by squeeze test (Exercise \ref{ex 9.3.5}) we have \(\lim_{x \to 1 ; x \in (0, \infty)} x^p = 1\).

    Let \(x_0 \in (0, \infty)\).
    By Lemma \ref{9.1.12} \(x_0\) is an adherent point.
    By Proposition \ref{6.7.3} \(x_0^p > 0\).
    Since \(\lim_{x \to 1 ; x \in (0, \infty)} x^p = 1\), by Definition \ref{9.3.6} we have
    \[
        \forall\ \varepsilon \in \mathbf{R}^+, \exists\ \delta \in \mathbf{R}^+ : \bigg(\forall\ x \in (0, \infty), \abs*{x - 1} < \delta \implies \abs*{x^p - 1} \leq \varepsilon\bigg).
    \]
    Let \(y = x \cdot x_0\).
    Then \(x = y / x_0\) and
    \[
        \forall\ \varepsilon \in \mathbf{R}^+, \exists\ \delta \in \mathbf{R}^+ : \bigg(\forall\ y \in (0, \infty), \abs*{y / x_0 - 1} < \delta \implies \abs*{(\frac{y}{x_0})^p - 1} \leq \varepsilon\bigg).
    \]
    Since \(\delta / x_0 < \delta\), we have
    \[
        \forall\ \varepsilon \in \mathbf{R}^+, \exists\ \delta \in \mathbf{R}^+ : \bigg(\forall\ y \in (0, \infty), \abs*{y / x_0 - 1} < \delta / x_0 \implies \abs*{(\frac{y}{x_0})^p - 1} \leq \varepsilon\bigg).
    \]
    This means
    \[
        \forall\ \varepsilon \in \mathbf{R}^+, \exists\ \delta \in \mathbf{R}^+ : \bigg(\forall\ y \in (0, \infty), \abs*{y - x_0} < \delta \implies \abs*{(\frac{y}{x_0})^p - 1} \leq \varepsilon\bigg).
    \]
    In particular, we have
    \[
        \forall\ \varepsilon \in \mathbf{R}^+, \exists\ \delta \in \mathbf{R}^+ : \bigg(\forall\ y \in (0, \infty), \abs*{y - x_0} < \delta \implies \abs*{(\frac{y}{x_0})^p - 1} \leq \frac{\varepsilon}{x_0^p}\bigg).
    \]
    This means
    \[
        \forall\ \varepsilon \in \mathbf{R}^+, \exists\ \delta \in \mathbf{R}^+ : \bigg(\forall\ y \in (0, \infty), \abs*{y - x_0} < \delta \implies \abs*{y^p - x_0^p} \leq \varepsilon\bigg).
    \]
    Thus by Definition \ref{9.3.6} we have \(\lim_{y \to x_0 ; y \in (0, \infty)} y^p = x_0^p\).
    Since \(x_0\) is arbitrary, by Definition \ref{9.4.1} \(x^p\) is continuous on \((0, \infty)\).
\end{proof}

\begin{proposition}[Absolute value is continuous]\label{9.4.12}
    The function \(f : \mathbf{R} \to \mathbf{R}\) defined by \(f(x) \coloneqq \abs*{x}\) is continuous.
\end{proposition}

\begin{proof}
    This follows since \(\abs*{x} = \max(x, -x)\) and the functions \(x, -x\) are already continuous.
\end{proof}

\begin{proposition}[Composition preserves continuity]\label{9.4.13}
    Let \(X\) and \(Y\) be subsets of \(\mathbf{R}\), and let \(f : X \to Y\) and \(g : Y \to \mathbf{R}\) be functions.
    Let \(x_0\) be a point in \(X\).
    If \(f\) is continuous at \(x_0\), and \(g\) is continuous at \(f(x_0)\), then the composition \(g \circ f : X \to \mathbf{R}\) is continuous at \(x_0\).
\end{proposition}

\begin{proof}
    Since \(\lim_{y \to f(x_0) ; y \in Y} g(x) = g(f(x_0))\), by Definition \ref{9.3.6} we have
    \[
        \forall\ \varepsilon \in \mathbf{R}^+, \exists\ \delta' \in \mathbf{R}^+ : \bigg(\forall\ y \in Y, \abs*{y - f(x_0)} < \delta' \implies \abs*{g(y) - g(f(x_0))} < \varepsilon\bigg).
    \]
    Since \(\lim_{x \to x_0 ; x \in X} f(x) = f(x_0)\), by Definition \ref{9.3.6} we have
    \[
        \forall\ \varepsilon' \in \mathbf{R}^+, \exists\ \delta \in \mathbf{R}^+ : \bigg(\forall\ x \in X, \abs*{x - x_0} < \delta \implies \abs*{f(x) - f(x_0)} < \varepsilon'\bigg).
    \]
    In particular, we have
    \[
        \exists\ \delta \in \mathbf{R}^+ : \bigg(\forall\ x \in X, \abs*{x - x_0} < \delta \implies \abs*{f(x) - f(x_0)} < \delta'\bigg).
    \]
    Since \(f(x) \in Y\) and \(\abs*{f(x) - f(x_0)} < \delta'\) implies \(\abs*{g(f(x)) - g(f(x_0))} < \varepsilon\), we have
    \[
        \forall\ \varepsilon \in \mathbf{R}^+, \exists\ \delta \in \mathbf{R}^+ : \bigg(\forall\ x \in X, \abs*{x - x_0} < \delta \implies \abs*{g(f(x)) - g(f(x_0))} < \varepsilon\bigg).
    \]
    Thus by Definition \ref{9.3.6} we have \(\lim_{x \to x_0 ; x \in X} g(f(x)) = g(f(x_0))\).
    By Definition \ref{9.4.1} \(g \circ f\) is continuous at \(x_0\).
\end{proof}

\exercisesection

\begin{exercise}\label{ex 9.4.1}
    Prove Proposition \ref{9.4.7}.
\end{exercise}

\begin{proof}
    See Proposition \ref{9.4.7}
\end{proof}

\begin{exercise}\label{ex 9.4.2}
    Let \(X\) be a subset of \(\mathbf{R}\), and let \(c \in \mathbf{R}\).
    Show that the constant function \(f : X \to \mathbf{R}\) defined by \(f(x) \coloneqq c\) is continuous, and show that the identity function \(g : X \to \mathbf{R}\) defined by \(g(x) \coloneqq x\) is also continuous.
\end{exercise}

\begin{proof}
    We first show that the constant functions \(f\) is continuous.
    Let \(x_0 \in X\).
    By Lemma \ref{9.1.11} we know that \(X \subseteq \overline{X}\), thus \(x_0\) is an adherent point of \(X\).
    Since
    \[
        \forall\ \varepsilon \in \mathbf{R}^+, \forall\ \delta \in \mathbf{R}^+, \abs*{x - x_0} < \delta \implies \abs*{f(x) - c} = \abs*{c - c} = 0 < \varepsilon,
    \]
    we know that \(\lim_{x \to x_0 ; x \in X} f(x) = c\).
    Since \(f(x_0) = c\) and \(x_0\) is arbitrary, by Definition \ref{9.4.1} we know that the constant functions \(f\) is continuous on \(X\).

    Now we show that the identity function \(g\) is continuous.
    Let \(x_0 \in X\).
    By Lemma \ref{9.1.11} we know that \(X \subseteq \overline{X}\), thus \(x_0\) is an adherent point of \(X\).
    Since
    \[
        \forall\ \varepsilon \in \mathbf{R}^+, \abs*{x - x_0} < \varepsilon \implies \abs*{g(x) - x_0} = \abs*{x - x_0} < \varepsilon,
    \]
    we know that \(\lim_{x \to x_0 ; x \in X} g(x) = x_0\).
    Since \(g(x_0) = x_0\) and \(x_0\) is arbitrary, by Definition \ref{9.4.1} we know that the constant functions \(g\) is continuous on \(X\).
\end{proof}

\begin{exercise}\label{ex 9.4.3}
    Prove Proposition \ref{9.4.10}.
\end{exercise}

\begin{proof}
    See Proposition \ref{9.4.10}.
\end{proof}

\begin{exercise}\label{ex 9.4.4}
    Prove Proposition \ref{9.4.11}.
\end{exercise}

\begin{proof}
    See Proposition \ref{9.4.11}.
\end{proof}

\begin{exercise}\label{ex 9.4.5}
    Prove Proposition \ref{9.4.13}.
\end{exercise}

\begin{proof}
    See Proposition \ref{9.4.13}.
\end{proof}

\begin{exercise}\label{ex 9.4.6}
    Let \(X\) be a subset of \(\mathbf{R}\), and let \(f : X \to \mathbf{R}\) be a continuous function.
    If \(Y\) is a subset of \(X\), show that the restriction \(f|_Y : Y \to \mathbf{R}\) of \(f\) to \(Y\) is also a continuous function.
\end{exercise}

\begin{proof}
    By Definition \ref{9.4.1}, \(\forall\ x_0 \in X\), we have \(\lim_{x \to x_0 ; x \in X} f(x) = f(x_0)\).
    Since \(Y \subseteq X\), we have \(\forall\ y \in Y \implies y \in X\).
    Thus \(\forall\ y_0 \in Y\) we have \(\lim_{y \to y_0 ; y \in Y} f(y) = f(y_0)\).
    By Definition \ref{9.4.1}, \(f|_Y\) is continuous on \(Y\).
\end{proof}

\begin{exercise}\label{ex 9.4.7}
    Let \(n \geq 0\) be an integer, and for each \(0 \leq i \leq n\) let \(c_i\) be a real number.
    Let \(P : \mathbf{R} \to \mathbf{R}\) be the function
    \[
        P(x) \coloneqq \sum_{i = 0}^n c_i x^i;
    \]
    Such a function is known as a \emph{polynomial of one variable};
    Show that \(P\) is continuous.
\end{exercise}