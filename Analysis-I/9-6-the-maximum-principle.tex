\section{The maximum principle}\label{sec 9.6}

\begin{definition}\label{9.6.1}
    Let \(X\) be a subset of \(\mathbf{R}\), and let \(f : X \to \mathbf{R}\) be a function.
    We say that \(f\) is \emph{bounded from above} if there exists a real number \(M\) such that \(f(x) \leq M\) for all \(x \in X\).
    We say that \(f\) is \emph{bounded from below} if there exists a real number \(M\) such that \(f(x) \geq -M\) for all \(x \in X\).
    We say that \(f\) is \emph{bounded} if there exists a real number \(M\) such that \(\abs*{f(x)} \leq M\) for all \(x \in X\).
\end{definition}

\begin{remark}\label{9.6.2}
    A function is bounded if and only if it is bounded both from above and below.
    Also, a function \(f : X \to \mathbf{R}\) is bounded if and only if its image \(f(X)\) is a bounded set in the sense of Definition \ref{9.1.22}.
\end{remark}