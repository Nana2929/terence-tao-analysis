\section{The maximum principle}\label{sec 9.6}

\begin{definition}\label{9.6.1}
    Let \(X\) be a subset of \(\mathbf{R}\), and let \(f : X \to \mathbf{R}\) be a function.
    We say that \(f\) is \emph{bounded from above} iff there exists a real number \(M\) such that \(f(x) \leq M\) for all \(x \in X\).
    We say that \(f\) is \emph{bounded from below} iff there exists a real number \(M\) such that \(f(x) \geq -M\) for all \(x \in X\).
    We say that \(f\) is \emph{bounded} iff there exists a real number \(M\) such that \(\abs*{f(x)} \leq M\) for all \(x \in X\).
\end{definition}

\begin{remark}\label{9.6.2}
    A function is bounded if and only if it is bounded both from above and below.
    Also, a function \(f : X \to \mathbf{R}\) is bounded if and only if its image \(f(X)\) is a bounded set in the sense of Definition \ref{9.1.22}.
\end{remark}

\begin{lemma}\label{9.6.3}
    Let \(a < b\) be real numbers, and let \(f : [a, b] \to \mathbf{R}\) be a function continuous on \([a, b]\).
    Then \(f\) is a bounded function.
\end{lemma}

\begin{proof}
    Suppose for sake of contradiction that \(f\) is not bounded.
    Thus for every real number \(M\) there exists an element \(x \in [a, b]\) such that \(\abs*{f(x)} \geq M\).

    In particular, for every natural number \(n\), the set \(\{x \in [a, b] : \abs*{f(x)} \geq n\}\) is non-empty.
    We can thus choose a sequence \((x_n)_{n = 0}^\infty\) in \([a, b]\) such that \(\abs*{f(x_n)} \geq n\) for all \(n\).
    This sequence lies in \([a, b]\), and so by Theorem \ref{9.1.24} there exists a subsequence \((x_{n_j})_{j = 0}^\infty\) which converges to some limit \(L \in [a, b]\), where \(n_0 < n_1 < n_2 < \dots\) is an increasing sequence of natural numbers.
    In particular, we see that \(n_j \geq j\) for all \(j \in \mathbf{N}\) (use induction).

    Since \(f\) is continuous on \([a, b]\), it is continuous at \(L\), and in particular we see that
    \[
        \lim_{j \to \infty} f(x_{n_j}) = f(L).
    \]
    Thus the sequence \((f(x_{n_j}))_{j = 0}^\infty\) is convergent, and hence it is bounded.
    On the other hand, we know from the construction that \(\abs*{f(x_{n_j})} \geq n_j \geq j\) for all \(j\), and hence the sequence \((f(x_{n_j}))_{j = 0}^\infty\) is not bounded, a contradiction.
\end{proof}

\begin{remark}\label{9.6.4}
    There are two things about the proof of Lemma \ref{9.6.3} that are worth noting.
    Firstly, it shows how useful the Heine-Borel theorem (Theorem \ref{9.1.24}) is.
    Secondly, it is an indirect proof;
    it doesn't say \emph{how} to find the bound for \(f\), but it shows that having \(f\) unbounded leads to a contradiction.
\end{remark}