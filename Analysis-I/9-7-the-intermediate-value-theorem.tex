\section{The intermediate value theorem}\label{sec 9.7}

\begin{theorem}[Intermediate value theorem]\label{9.7.1}
    Let \(a < b\), and let \(f : [a, b] \to \mathbf{R}\) be a continuous function on \([a, b]\).
    Let \(y\) be a real number between \(f(a)\) and \(f(b)\), i.e., either \(f(a) \leq y \leq f(b)\) or \(f(a) \geq y \geq f(b)\).
    Then there exists \(c \in [a, b]\) such that \(f(c) = y\).
\end{theorem}

\begin{proof}
    We have two cases: \(f(a) \leq y \leq f(b)\) or \(f(a) \geq y \geq f(b)\).
    We will assume the former, that \(f(a) \leq y \leq f(b)\);
    the latter is proven similarly.

    If \(y = f(a)\) or \(y = f(b)\) then the claim is easy, as one can simply set \(c = a\) or \(c = b\), so we will assume that \(f(a) < y < f(b)\).
    Let \(E\) denote the set
    \[
        E \coloneqq \{x \in [a, b] : f(x) < y\}.
    \]
    Clearly \(E\) is a subset of \([a, b]\), and is hence bounded.
    Also, since \(f(a) < y\), we see that \(a\) is an element of \(E\), so \(E\) is non-empty.
    By the least upper bound principle, the supremum
    \[
        c \coloneqq \sup(E)
    \]
    is thus finite.
    Since \(E\) is bounded by \(b\), we know that \(c \leq b\);
    since \(E\) contains \(a\), we know that \(c \geq a\).
    Thus we have \(c \in [a, b]\).
    To complete the proof we now show that \(f(c) = y\).
    The idea is to work from the left of \(c\) to show that \(f(c) \leq y\), and to work from the right of \(c\) to show that \(f(c) \geq y\).

    Let \(n \geq 1\) be an integer.
    The number \(c - \frac{1}{n}\) is less than \(c = \sup(E)\) and hence cannot be an upper bound for \(E\).
    Thus there exists a point, call it \(x_n\), which lies in \(E\) and which is greater than \(c - \frac{1}{n}\).
    Also \(x_n \leq c\) since \(c\) is an upper bound for \(E\).
    Thus
    \[
        c - \frac{1}{n} \leq x_n \leq c.
    \]
    By the squeeze test (Corollary \ref{6.4.14}) we thus have \(\lim_{n \to \infty} x_n = c\).
    Since \(f\) is continuous at \(c\), this implies that \(\lim_{n \to \infty} f(x_n) = f(c)\).
    But since \(x_n\) lies in \(E\) for every \(n\), we have \(f(x_n) < y\) for every \(n\).
    By the comparison principle (Lemma \ref{6.4.13}) we thus have \(f(c) \leq y\).
    Since \(f(b) > f(c)\), we conclude \(c \neq b\).

    Since \(c \neq b\) and \(c \in [a, b]\), we must have \(c < b\).
    In particular there is an \(N > 0\) such that \(c + \frac{1}{n} < b\) for all \(n > N\)
    (since \(c + \frac{1}{n}\) converges to \(c\) as \(n \to \infty\)).
    Since \(c\) is the supremum of \(E\) and \(c + \frac{1}{n} > c\), we thus have \(c + \frac{1}{n} \notin E\) for all \(n > N\).
    Since \(c + \frac{1}{n} \in [a, b]\), we thus have \(f(c + \frac{1}{n}) \geq y\) for all \(n \geq N\).
    But \(c + \frac{1}{n}\) converges to \(c\), and \(f\) is continuous at \(c\), thus \(f(c) \geq y\).
    But we already knew that \(f(c) \leq y\), thus \(f(c) = y\), as desired.
\end{proof}

\begin{note}
    The intermediate value theorem says that if \(f\) takes the values \(f(a)\) and \(f(b)\), then it must also take all the values in between.
    If \(f\) is not assumed to be continuous, then the intermediate value theorem no longer applies.
    If a function is discontinuous, it can ``jump'' past intermediate values;
    however continuous functions cannot do so.
\end{note}

\begin{remark}\label{9.7.2}
    A continuous function may take an intermediate value multiple times.
\end{remark}

\begin{remark}\label{9.7.3}
    The intermediate value theorem gives another way to show that one can take \(n^{\text{th}}\) roots of a number.
    For instance, to construct the square root of \(2\), consider the function \(f : [0, 2] \to \mathbf{R}\) defined by \(f(x) = x^2\).
    This function is continuous, with \(f(0) = 0\) and \(f(2) = 4\).
    Thus there exists a \(c \in [0, 2]\) such that \(f(c) = 2\), i.e., \(c^2 = 2\).
    (This argument does not show that there is just one square root of \(2\), but it does prove that there is at least one square root of \(2\).)
\end{remark}

\begin{corollary}[Images of continuous functions]\label{9.7.4}
    Let \(a < b\), and let \(f : [a, b] \to \mathbf{R}\) be a continuous function on \([a, b]\).
    Let \(M \coloneqq \sup_{x \in [a, b]} f(x)\) be the maximum value of \(f\), and let \(m \coloneqq \inf_{x \in [a, b]} f(x)\) be the minimum value.
    Let \(y\) be a real number between \(m\) and \(M\) (i.e., \(m \leq y \leq M\)).
    Then there exists a \(c \in [a, b]\) such that \(f(c) = y\).
    Furthermore, we have \(f([a, b]) = [m, M]\).
\end{corollary}

\begin{proof}
    By maximum principle (Proposition \ref{9.6.7} we have \(x_M, x_m \in [a, b]\) such that \(f(x_M) = M\) and \(f(x_m) = m\).
    We have either \(x_m \leq x_M\) or \(x_m \geq x_M\).
    Without the loss of generality suppose that \(x_m \leq x_M\).
    Then we know that \([x_m, x_M] \subseteq [a, b]\) and by Exercise \ref{ex 9.4.6} we know that \(f\) is continuous on \([x_m, x_M]\).
    Since \(m \leq y \leq M\), by Theorem \ref{9.7.1} \(\exists\ c \in [x_m, x_M]\) such that \(f(c) = y\).
    Since \([x_m, x_M] \subseteq [a, b]\), we have \(c \in [a, b]\).
\end{proof}

\exercisesection

\begin{exercise}\label{ex 9.7.1}
    Prove Corollary \ref{9.7.4}.
\end{exercise}

\begin{proof}
    See Corollary \ref{9.7.4}.
\end{proof}

\begin{exercise}\label{ex 9.7.2}
    Let \(f : [0, 1] \to [0, 1]\) be a continuous function.
    Show that there exists a real number \(x\) in \([0, 1]\) such that \(f(x) = x\).
    This point \(x\) is known as a \emph{fixed point} of \(f\), and this result is a basic example of a \emph{fixed point theorem}, which play an important role in certain types of analysis.
\end{exercise}