\section{Monotonic functions}\label{sec 9.8}

\begin{definition}[Monotonic functions]\label{9.8.1}
    Let \(X\) be a subset of \(\mathbf{R}\), and let \(f : X \to \mathbf{R}\) be a function.
    We say that \(f\) is \emph{monotone increasing} iff \(f(y) \geq f(x)\) whenever \(x, y \in X\) and \(y > x\).
    We say that \(f\) is \emph{strictly monotone increasing} iff \(f(y) > f(x)\) whenever \(x, y \in X\) and \(y > x\).
    Similarly, we say \(f\) is \emph{monotone decreasing} iff \(f(y) \leq f(x)\) whenever \(x, y \in X\) and \(y > x\), and \emph{strictly monotone decreasing} iff \(f(y) < f(x)\) whenever \(x, y \in X\) and \(y > x\).
    We say that \(f\) is \emph{monotone} if it is monotone increasing or monotone decreasing, and \emph{strictly monotone} if it is strictly monotone increasing or strictly monotone decreasing.
\end{definition}