\section{Uniform continuity}\label{sec 9.9}

\setcounter{theorem}{2}
\begin{definition}[Uniform continuity]\label{9.9.2}
    Let \(X\) be a subset of \(\mathbf{R}\), and let \(f : X \to \mathbf{R}\) be a function.
    We say that \(f\) is uniformly continuous if, for every \(\varepsilon > 0\), there exists a \(\delta > 0\) such that \(f(x)\) and \(f(x_0)\) are \(\varepsilon\)-close whenever \(x, x_0 \in X\) are two points in \(X\) which are \(\delta\)-close.
\end{definition}

