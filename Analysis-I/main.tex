% We use chapter structure.
\documentclass[12pt,oneside]{book}

%==============================================================================
% Preamble.
%==============================================================================

% Correctly showing characters outside ASCII.
\usepackage[T1]{fontenc}
% File is written and read with utf8 encoding.
\usepackage[utf8]{inputenc}
% Set paging layout.
\usepackage[margin=1.2in]{geometry}
% Including `amsfonts'.  Must be loaded before `mathtools'.
\usepackage{amssymb}
% Including `amsmath' and fixing bugs for `amsmath'.
\usepackage{mathtools}
% Must be loaded after `amsmath' and `mathtools'.
\usepackage{amsthm}
% Automatically adjust character spacing at margins.
\usepackage{microtype}
% Provide further utilities and fix bugs for `enumerate', `itemize' and
% `description'.
\usepackage{enumitem}
% Provide better quoting environment.
\usepackage{dirtytalk}
% Parsing list inside `\newcommand'.
\usepackage{listofitems}
% Nice looking if-then-else structure with comparison functionality.
\usepackage{ifthen}
% Automatically add hyperlinks to labels/refs.  Must be loaded after all
% packages above and before `cleveref'.  Recommend to use with `natbib' when
% you need bibtex.
\usepackage{hyperref}

\hypersetup{         % This macro come with `hyperref'.
	colorlinks=true, % Color hyperlinks.
	linkcolor=blue,  % Color local hyperlinks with blue.
	urlcolor=cyan,   % Color url links with cyan.
}

%------------------------------------------------------------------------------
% Define environments.
%------------------------------------------------------------------------------

% Text inside the body of theorem-like environments are set to Roman font.
% theorem-like environments share their counters, counters follow section and
% reset in every sections (except for axioms, axioms counters are reset in each
% chapter).  Exercises has their owned counter.  Notes do not use counter.
% See `amsthm' for details.
\theoremstyle{definition}
\newtheorem{axiom}{Axiom}[chapter]
\newtheorem{additional corollary}{Additional Corollary}[section]
\newtheorem{exercise}{Exercise}[section]
\newtheorem{theorem}{Theorem}[section]
\newtheorem{corollary}[theorem]{Corollary}
\newtheorem{definition}[theorem]{Definition}
\newtheorem{example}[theorem]{Example}
\newtheorem{lemma}[theorem]{Lemma}
\newtheorem{proposition}[theorem]{Proposition}
\newtheorem{remark}[theorem]{Remark}
\newtheorem*{note}{Note}

\theoremstyle{remark}
\newtheorem*{meta-proof}{Meta-proof}

% In `enumerate' enviroments, items' label are alphabets and surrounded by
% parentheses.  See `enumitem' for details.
\renewcommand{\labelenumi}{\textnormal{(}\alph{enumi}\textnormal{)}}

% Formatting exercises section.
\newcommand{\exercisesection}{
    \begin{center}
        --- Exercises ---
    \end{center}
}

%------------------------------------------------------------------------------
% Define operators and symbols.
%------------------------------------------------------------------------------

% Absolute value.
\DeclarePairedDelimiter\abs{\lvert}{\rvert}
% Ceiling.
\DeclarePairedDelimiter\ceil{\lceil}{\rceil}
% Floor.
\DeclarePairedDelimiter\floor{\lfloor}{\rfloor}

%==============================================================================
% Document.
%==============================================================================

\begin{document}

%------------------------------------------------------------------------------
% Front matters.
%------------------------------------------------------------------------------

\frontmatter

% Author informations.
\title{Analysis I}
\author{ProFatXuanAll}
\maketitle

% Table of contents.
\tableofcontents{}

%------------------------------------------------------------------------------
% Main matters.
%------------------------------------------------------------------------------

\mainmatter

% All chapters are in separated files.  We include them here.
This is a note of the textbook "Analysis I", 3rd edition by Terence Tao.
The note covers all axioms, theories, lemma, propositions and remark appear in the book.
Additionally, I also wrote proofs if there are none in the book.

\begin{note}\label{circularity}
\emph{circularity}: using an advanced fact to prove a more elementary fact, and then later using the elementary fact to prove the advanced fact.
When do a mathematics proofs, one should avoid \emph{circularity}.
\end{note}

\begin{note}
From a logical point of view, there is no difference between a lemma, proposition, theorem, or corollary - they are all claims waiting to be proved.
However, we use these terms to suggest different levels of importance and difficulty.
A lemma is an easily proved claim which is helpful for proving other propositions and theorems, but is usually not particularly interesting in its own right.
A proposition is a statement which is interesting in its own right, while a theorem is a more important statement than a proposition which says something definitive on the subject, and often takes more effort to prove than a proposition or lemma.
A corollary is a quick consequence of a proposition or theorem that was proven recently.
\end{note}
\chapter{Natural Number}

\section{The Peano axioms}

\begin{note}
We now present one standard way to define the natural numbers, in terms of the \emph{Peano axioms}, which were first laid out by Giuseppe Peano (1858–1932).
This is not the only way to define the natural numbers.
For instance, another approach is to talk about the cardinality of finite sets, for instance one could take a set of five elements and define \(5\) to be the number of elements in that set.
\end{note}

\begin{note}
In some texts the natural numbers start at \(1\) instead of \(0\), but this is a matter of notational convention more than anything else.
In this text we shall refer to the set \(\{1, 2, 3,...\}\) as the positive integers \(\mathds{Z}^+\) rather than the natural numbers.
Natural numbers are sometimes also known as \emph{whole numbers}.
\end{note}

\begin{note}
In mathematics we try not to define a variable more than once in any given setting, as it can often lead to confusion;
many of the statements which were true for the old value of the variable can now become false, and vice versa.
\end{note}

\begin{axiom}\label{2.1}
\(0\) is a natural number.
\end{axiom}

\begin{axiom}\label{2.2}
If \(n\) is a natural number, then \(n++\) is also a natural number.
\end{axiom}

\begin{axiom}\label{2.3}
\(0\) is not the successor of any natural number;
i.e., we have \(n++ \neq 0\) for every natural number \(n\).
\end{axiom}

\begin{axiom}\label{2.4}
Different natural numbers must have different successors;
i.e., if \(n\), \(m\) are natural numbers and \(n \neq m\), then \(n++ \neq m++\).
Equivalently, if \(n++ = m++\), then we must have \(n = m\).
\end{axiom}

\begin{axiom}[Principle of mathematical induction]\label{2.5}
Let \(P(n)\) be any property pertaining to a natural number \(n\).
Suppose that \(P(0)\) is true, and suppose that whenever \(P(n)\) is true, \(P(n++)\) is also true.
Then \(P(n)\) is true for every natural number \(n\).
\end{axiom}

\begin{note}
Axioms \ref{2.1}-\ref{2.5} are known as the Peano axioms for the natural numbers.
\end{note}

\section{Addition}

\begin{definition}[Addition of natural numbers]\label{2.2.1}
Let \(m\) be a natural numbers.
To add zero to \(m\), we define \(0+m \coloneqq m\).
Now suppose inductively that we have defined how to add \(n\) to \(m\).
Then we can add \(n++\) to \(m\) by defining \((n++) + m \coloneqq (n + m)++\).
\end{definition}

\begin{additional corollary}\label{ac 2.2.1}
The sum of two natural numbers is again a natural number.
\end{additional corollary}

\begin{proof}
We use induction.
Let \(m\) be a natural number.
\(0 + m = m\) is a natural number according to the given condition.
Now suppose inductively that \(n\) is a natural number such that \(n + m\) is a natural number.
We wish to show that \((n++) + m\) is also a natural number.
But by Definition \ref{2.2.1} and Axiom \ref{2.2},
\((n++) + m = (n + m)++\) is a natural number.
This close the induction.
\end{proof}

\begin{lemma}\label{2.2.2}
For any natural number \(n\), \(n + 0 = n\).
\end{lemma}

\begin{proof}
We use induction.
The base case \(0 + 0 = 0\) follows since we know that \(0 + m = m\) for every natural number \(m\), and \(0\) is a natural number.
Now suppose inductively that \(n + 0 = n\).
We wish to show that \((n++) + 0 = n++\).
But by Definition \ref{2.2.1}, \((n++) + 0\) is equal to \((n + 0)++\), which is equal to \(n++\) since \(n + 0 = n\).
This closes the induction.
\end{proof}

\begin{lemma}\label{2.2.3}
For any natural numbers \(n\) and \(m\), \(n + (m++) = (n + m)++\).
\end{lemma}

\begin{proof}
We induct on \(n\) (keeping \(m\) fixed).
We first consider the base case \(n = 0\).
In this case we have to prove \(0 + (m++) = (0 + m)++\).
But by Definition \ref{2.2.1}, \(0 + (m++) = m++\) and \(0 + m = m\), so both sides are equal to \(m++\) and are thus equal to each other.
Now we assume inductively that \(n + (m++) = (n + m)++\);
we now have to show that \((n++) + (m++) = ((n++) + m)++\).
The left-hand side is \((n + (m++))++\) by Definition \ref{2.2.1}, which is equal to \(((n+m)++)++\) by the inductive hypothesis.
Similarly, we have \((n++) + m = (n + m)++\) by the Definition \ref{2.2.1}, and so the right-hand side is also equal to \(((n + m)++)++\).
Thus both sides are equal to each other, and we have closed the induction.
\end{proof}

\begin{additional corollary}\label{ac 2.2.2}
\(n++ = n + 1\).
\end{additional corollary}

\begin{proof}
\begin{align*}
n++ &= (n++) + 0 & \text{(by Lemma \ref{2.2.1})} \\
&= (n + 0)++ & \text{(by Definition \ref{2.2.1})} \\
&= n + (0++) & \text{(by Lemma \ref{2.2.3})} \\
&= n + 1.
\end{align*}
\end{proof}

\begin{proposition}[Addition is commutative]\label{2.2.4}
For any natural numbers \(n\) and \(m\), \(n + m = m + n\).
\end{proposition}

\begin{proof}
We shall use induction on \(n\) (keeping \(m\) fixed).
First we do the base case \(n = 0\), i.e., we show \(0 + m = m + 0\).
By the Definition \ref{2.2.1}, \(0 + m = m\), while by Lemma \ref{2.2.1}, \(m + 0 = m\).
Thus the base case is done.
Now suppose inductively that \(n + m = m + n\), now we have to prove that \((n++) + m = m + (n++)\) to close the induction.
By the Definition \ref{2.2.1}, \((n++) + m = (n + m)++\).
By Lemma \ref{2.2.3}, \(m + (n++) = (m + n)++\), but this is equal to \((n + m)++\) by the inductive hypothesis \(n+m=m+n\).
Thus \((n++) + m = m + (n++)\) and we have closed the induction.
\end{proof}

\begin{proposition}[Addition is associative]\label{2.2.5}
For any natural numbers \(a\), \(b\), \(c\), we have \((a + b) + c = a + (b + c)\).
\end{proposition}

\begin{proof}
We shall use induction on \(c\) and keeping both \(a\) and \(b\) fixed.
First we do the base case \(c = 0\), i.e., we show \((a + b) + 0 = a + (b + 0)\).
By Lemma \ref{2.2.1}, \((a + b) + 0 = a + b\) and \(a + (b + 0) = a + b\).
Thus the base case is done.
Now suppose inductively that \((a + b) + c = a + (b + c)\), we have to prove that \((a + b) + (c++) = a + (b + (c++))\) to close the induction.
By Lemma \ref{2.2.3}, \((a + b) + (c++) = ((a + b) + c)++\).
Also By Lemma \ref{2.2.3}, \(a + (b + (c++)) = a + ((b + c)++) = (a + (b + c))++\), but this is equal to \(((a + b) + c)++\) by the inductive hypothesis \((a + b) + c = a + (b + c)\).
Thus \((a + b) + (c++) = a + (b + (c++))\) and we have closed the induction.
\end{proof}

\begin{note}
Because of this associativity we can write sums such as \(a + b + c\) without having to worry about which order the numbers are being added together.
\end{note}

\begin{proposition}[Cancellation law]\label{2.2.6}
Let \(a\), \(b\), \(c\) be natural numbers such that \(a + b = a + c\).
Then we have \(b = c\).
\end{proposition}

\begin{proof}
We prove this by induction on \(a\).
First consider the base case \(a = 0\).
Then we have \(0 + b = 0 + c\), which by Definition \ref{2.2.1} implies that \(b = c\) as desired.
Now suppose inductively that we have the cancellation law for \(a\) (so that \(a + b = a + c\) implies \(b = c\));
we now have to prove the cancellation law for \(a++\).
In other words, we assume that \((a++) + b = (a++) + c\) and need to show that \(b = c\).
By the Definition \ref{2.2.1}, \((a++) + b = (a + b)++\) and \((a++) + c = (a + c)++\) and so we have \((a + b)++ = (a + c)++\).
By Axiom \ref{2.4}, we have \(a + b = a + c\).
Since we already have the cancellation law for \(a\), we thus have \(b = c\) as desired.
This closes the induction.
\end{proof}

\begin{definition}[Positive natural numbers]\label{2.2.7}
A natural number \(n\) is said to be \emph{positive} iff it is not equal to \(0\).
\end{definition}

\begin{proposition}\label{2.2.8}
If \(a\) is positive and \(b\) is a natural number, then \(a + b\) is positive (and hence \(b + a\) is also, by Proposition \ref{2.2.4}).
\end{proposition}

\begin{proof}
We use induction on \(b\).
If \(b = 0\), then \(a + b = a + 0 = a\), which is positive, so this proves the base case.
Now suppose inductively that \(a + b\) is positive.
Then \(a + (b++) = (a + b)++\), which cannot be zero by Axiom \ref{2.3}, and is hence positive.
This closes the induction.
\end{proof}

\begin{corollary}\label{2.2.9}
If \(a\) and \(b\) are natural numbers such that \(a + b = 0\), then \(a = 0\) and \(b = 0\).
\end{corollary}

\begin{proof}
Suppose for sake of contradiction that \(a \neq 0\) or \(b \neq 0\).
If \(a \neq 0\) then \(a\) is positive, and hence \(a + b = 0\) is positive by Proposition \ref{2.2.8}, a contradiction.
Similarly if \(b \neq 0\) then \(b\) is positive, and again \(a + b = 0\) is positive by Proposition \ref{2.2.8}, a contradiction.
Thus \(a\) and \(b\) must both be zero.
\end{proof}

\begin{lemma}\label{2.2.10}
Let \(a\) be a positive number.
Then there exists exactly one natural number \(b\) such that \(b++ = a\).
\end{lemma}

\begin{proof}
We use induction.
If \(a = 1\) and \(b++ = a\), then \(b = 0\).
We want to show that \(0\) is unique, so assume that \(0'\) is a natural number and \(0'++ = 1\).
By Axiom \ref{2.4}, if \(0++ = 0'++\), then \(0 = 0'\), thus \(0\) is unique, and this proves the base case.
Suppose inductively that \(a\) is positive, there exist exactly one natural number \(b\) such that \(b++ = a\).
Then \(a++ = (b++)++\) by induction hypothesis, so there exist a natural number \(b++\) such that \((b++)++ = a++\).
Now we need to show that \(b++\) is unique.
Assume that there exist another natural number \(c\) such that \(a++ = c++\).
Then by Axiom \ref{2.4}, \(a = c\), and by induction hypothesis, \(c = b++\), so \(c++ = (b++)++\), which means \(b++\) is unique.
This close the induction.
\end{proof}

\begin{definition}[Ordering of the natural numbers]\label{2.2.11}
Let \(n\) and \(m\) be natural numbers.
We say that \(n\) is greater than or equal to \(m\), and write \(n \geq m\) or \(m \leq n\), iff we have \(n = m + a\) for some natural number \(a\).
We say that \(n\) is strictly greater than \(m\), and write \(n > m\) or \(m < n\), iff \(n \geq m\) and \(n \neq m\).
\end{definition}

\begin{note}
There is no largest natural number \(n\), because the next number \(n++\) is always larger.
\end{note}

\begin{proposition}[Basic properties of order for natural numbers]\label{2.2.12}
Let \(a\), \(b\), \(c\) be natural numbers.
Then
\begin{enumerate}
\item (Order is reflexive) \(a \geq a\).
\item (Order is transitive) If \(a \geq b\) and \(b \geq c\), then \(a \geq c\).
\item (Order is anti-symmetric) If \(a \geq b\) and \(b \geq a\), then \(a = b\).
\item (Addition preserves order) \(a \geq b\) iff \(a + c \geq b + c\).
\item \(a < b\) iff \(a++ \leq b\).
\item \(a < b\) iff \(b = a + d\) for some \emph{positive} number \(d\).
\end{enumerate}
\end{proposition}

\begin{proof}{(a)}
\(\because a = a + 0\), \(\therefore a \geq a\) by Definition \ref{2.2.11}.
\end{proof}

\begin{proof}{(b)}
Let \(a = b + d\) and \(b = c + e\), where \(d\), \(e\) are natural numbers.
Substitute \(b\) with equation, we can derive \(a = b + d = (c + e) + d = c + (e + d)\).
By Additional Corollary \ref{ac 2.2.1}, \(e + d\) is a natural number.
Therefore by Definition \ref{2.2.11}, \(a \geq c\).
\end{proof}

\begin{proof}{(c)}
By the given conditions \(a \geq b\) and \(b \geq a\), let \(a = b + c\) and \(b = a + d\), where \(c\) and \(d\) are natural numbers.
Substitute \(b\) with equation, we can derive that \(a = b + c = (a + d) + c = a + (d + c)\).
By Proposition \ref{2.2.6}, \(a = a + (d + c) \implies 0 = d + c\), and by Corollary \ref{2.2.9}, \(d = c = 0\).
Therefore \(a = b + 0 = b\) and \(b = a + 0 = a\).
\end{proof}

\begin{proof}{(d)}
First we prove the necessary condition of (d).
Let \(a = b + d\), where \(d\) is a natural number.
Then \(a = b + d \implies (a + c) = (b + c) + d\), which means \(a + c \geq b + c\) by Definition \ref{2.2.11}.

Now we prove the sufficient condition of (d).
Let \(a + c = b + c + d\), where \(d\) is a natural number.
Then by Proposition \ref{2.2.6}, \(a + c = b + c + d \implies a = b + d\), which means \(a \geq b\) by Definition \ref{2.2.11}.
\end{proof}

\begin{proof}{(e)}
First we prove the necessary condition of (e).
Let \(b = a + c\), where \(c\) is a natural number.
Because \(a \neq b\), so \(c \neq 0\).
By Lemma \ref{2.2.10}, there exists a natural number \(d\) such that \(d++ = c\).
So \(b = a + c = a + (d++) = (a + d)++ = (a++) + d\) by Lemma \ref{2.2.3} and Definition \ref{2.2.1}.
Thus \(a++ \leq b\) by Definition \ref{2.2.11}.

Now we prove the sufficient condition of (e).
Let \(b = (a++) + c\), where \(c\) is a natural number.
Then \(b = a + (c++)\) and \(c++\) must not be zero, otherwise contradict to Axiom \ref{2.3}.
So \(b \neq a\), thus \(a < b\).
\end{proof}

\begin{proof}{(f)}
First we prove the necessary condition of (f).
Let \(b = a + d\), where \(d\) is a natural number.
Because \(a \neq b\), so \(d \neq 0\), which means \(d\) is positive by Definition \ref{2.2.7}.

Now we prove the sufficient condition of (f).
Let \(b = a + d\), where \(d\) is a positive natural number.
Because \(d \neq 0\), \(b \neq a\), thus \(a < b\).
\end{proof}

\begin{proposition}[Trichotomy of order for natural numbers]\label{2.2.13}
Let \(a\) and \(b\) be natural numbers.
Then exactly one of the following statements is true: \(a < b\), \(a = b\), or \(a > b\).
\end{proposition}

\begin{proof}
First we show that we cannot have more than one of the statements \(a < b\), \(a = b\), \(a > b\) holding at the same time.
If \(a < b\) then \(a \neq b\) by Definition \ref{2.2.11}, and if \(a > b\) then \(a \neq b\) by Definition \ref{2.2.11}.
If \(a > b\) and \(a < b\) then by Proposition \ref{2.2.12} we have \(a = b\), a contradiction.
Thus no more than one of the statements is true.
Now we show that at least one of the statements is true.
We keep \(b\) fixed and induct on \(a\).
When \(a = 0\) we have \(0 \leq b\) for all \(b\) because \(b = b + 0\), so we have either \(0 = b\) or \(0 < b\), which proves the base case.
Now suppose we have proven the proposition for \(a\), and now we prove the proposition for \(a++\).
From the trichotomy for \(a\), there are three cases: \(a < b\), \(a = b\), and \(a > b\).
If \(a > b\), then \(a++ > b\) because \(a++ > a > b\) and Proposition \ref{2.2.12}.
If \(a = b\), then \(a++ > b\) because \(a++ = b++ > b\).
Now suppose that \(a < b\).
Then by Proposition \ref{2.2.12}, we have \(a++ \leq b\).
Thus either \(a++ = b\) or \(a++ < b\), and in either case we are done.
This closes the induction.
\end{proof}

\begin{proposition}[Strong principle of induction]\label{2.2.14}
Let \(m_0\) be a natural number, and let \(P(m)\) be a property pertaining to an arbitrary natural number \(m\).
Suppose that for each \(m \geq m_0\), we have the following implication: if \(P(m')\) is true for all natural numbers \(m_0 \leq m' < m\), then \(P(m)\) is also true.
(In particular, this means that \(P(m_0)\) is true, since in this case the hypothesis is vacuous.)
Then we can conclude that \(P(m)\) is true for all natural numbers \(m \geq m_0\).
\end{proposition}

\begin{proof}
Let \(n\) be a natural number and let \(Q(n)\) be the property that \(P(m)\) is true for all \(m_0 \leq m < n\) for \(n \geq m_0\).
Using induction on \(n\), for the base case \(n = 0\), we want to show that \(Q(0)\) is true.
However, we know that \(0 \leq m_0\) for all natural number \(m_0\).
Thus, either \(0 = m_0\) or \(0 < m_0\) and so we split into cases.
If \(n = 0 < m_0\), the statement \(P(m) \ \forall\ m_0 \leq m < n\) is vacuously true (since the hypothesis applies for \(n \geq m_0\)) and thus \(Q(0)\) is true in this case.
For the second case, if \(n = 0 = m_0\), then the statement \(P(m) \ \forall\ m_0 \leq m < n\) is also vacuously true since there is no natural number \(m'\) such that \(0 \leq m' < 0\). Hence, \(Q(0)\) is true for this case and that completes the base case of the induction.

Now suppose inductively that for some \(n \geq m_0\), \(Q(n)\) is true, i.e \(P(m) \ \forall\ m_0 \leq m < n\) is true.
We need to show that \(Q(n++)\) is true.
By the definition of \(P\) in the hypothesis, \(P(n)\) is also true (because \(Q(n)\) is true).
Since \(n < n++\), then \(P(m) \ \forall\ m_0 \leq m \leq n < n++\) is true so \(P(m) \ \forall\ m_0 \leq m < n++\) is true which in turn implies that \(Q(n++)\) is true.
Which closes the induction and hence we can conclude that \(Q(n) \ \forall\ n\) is true.
However, \(Q(n)\) true implies \(P(m) \ \forall\ m_0 \leq m < n\) is true for all \(n \geq m_0\) and by the definition of \(P\), \(P(n)\) is also true for all \(n \geq m_0\) which concludes the proof.
\end{proof}

\begin{remark}\label{2.2.15}
In applications we usually use Proposition \ref{2.2.14} with \(m_0 = 0\) or \(m_0 = 1\).
\end{remark}

\exercisesection

\begin{exercise}\label{ex 2.2.1}
Prove Proposition \ref{2.2.5}.
\end{exercise}

\begin{proof}
See Proposition \ref{2.2.5}.
\end{proof}

\begin{exercise}\label{ex 2.2.2}
Prove Lemma \ref{2.2.10}.
\end{exercise}

\begin{proof}
See Lemma \ref{2.2.10}.
\end{proof}

\begin{exercise}\label{ex 2.2.3}
Prove Proposition \ref{2.2.12}.
\end{exercise}

\begin{proof}
See Proposition \ref{2.2.12}.
\end{proof}

\begin{exercise}\label{ex 2.2.4}
Justify the three statements marked in the proof of Proposition \ref{2.2.13}.
\end{exercise}

\begin{proof}
See Proposition \ref{2.2.13}.
\end{proof}

\begin{exercise}\label{ex 2.2.5}
Prove Proposition \ref{2.2.14}.
\end{exercise}

\begin{proof}
See Proposition \ref{2.2.14}.
\end{proof}

\begin{exercise}[Principle of backwards induction]\label{ex 2.2.6}
Let \(n\) be a natural number, and let \(P(m)\) be a property pertaining to the natural numbers such that whenever \(P(m++)\) is true, then \(P(m)\) is true.
Suppose that \(P(n)\) is also true.
Prove that \(P(m)\) is true for all natural numbers \(m \leq n\);
\end{exercise}

\begin{proof}
We use induction.
The base case \(n = 0\) is trivially true because \(\forall\ m \leq 0\) means \(m = 0\), thus \(P(m) = P(n) = P(0)\) is true.
Suppose inductively that \(P(n)\) is true, and \(P(m)\) is true \(\forall\ m \leq n\).
Then when \(P(n++)\) is true, by the given condition \(P(n)\) is true.
And by induction hypothesis \(P(m)\) is true \(\forall\ m \leq n \leq n++\), so \(P(n++)\) is true implies \(P(m)\) is true \(\forall\ m \leq n++\).
This close the induction.
\end{proof}

\section{Multiplication}
\begin{note}
In the previous section we have proven all the basic facts that we know to be true about addition and order.
To save space and to avoid belaboring the obvious, we will now allow ourselves to use all the rules of algebra concerning addition and order that we are familiar with, without further comment.
\end{note}

\begin{definition}[Multiplication of natural numbers]\label{2.3.1}
Let \(m\) be a natural number.
To multiply zero to \(m\), we define \(0 \times m \coloneqq 0\).
Now suppose inductively that we have defined how to multiply \(n\) to \(m\).
Then we can multiply \(n++\) to \(m\) by defining \((n++) \times m \coloneqq (n \times m) + m\).
\end{definition}

\begin{additional corollary}\label{ac 2.3.1}
The product of two natural numbers is a natural number.
\end{additional corollary}

\begin{proof}
Let \(n\), \(m\) be two natural numbers.
We use induction on \(n\).
For \(n = 0\), \(n \times m = 0 \times m = 0\) is a natural number.
Now suppose inductively that \(n\) is a natural number such that \(n \times m\) is a natural number.
We wish to show that \((n++) \times m\) is also a natural number.
By Definition \ref{2.3.1}, \((n++) \times m = (n \times m) + m\).
By induction hypothesis and by Additional Corollary \ref{ac 2.2.1}, \((n \times m)\) is a natural number and sum of two natural number is again a natural number.
This close the induction.
\end{proof}

\begin{additional corollary}\label{ac 2.3.2}
Let \(n\) be a natural number.
Then \(n \times 0 = 0\).
\end{additional corollary}

\begin{proof}
We use induction.
For base case \(n = 0\), \(0 \times 0 = 0\) by Definition \ref{2.3.1}.
Suppose inductively that \(n \times 0 = 0\).
Then \((n++) \times 0 = (n \times 0) + 0 = 0 + 0 = 0\) by Definition \ref{2.3.1} and induction hypothesis.
This close the induction.
\end{proof}

\begin{additional corollary}\label{ac 2.3.3}
Let \(n\), \(m\) be natural numbers.
Then \(n \times (m++) = (n \times m) + n\).
\end{additional corollary}

\begin{proof}
We use induction on \(n\) (fixed \(m\).
For base case \(n = 0\), \(0 \times (m++) = 0\) by Definition \ref{2.3.1}.
Suppose inductively that \(n \times (m++) = (n \times m) + n\).
Then
    \begin{align*}
        (n++) \times (m++)
        &= (n \times (m++)) + (m++) & \text{(by Definition \ref{2.3.1})} \\
        &= (n \times m) + n + (m++) & \text{(by induction hypothesis)} \\
        &= (n \times m) + (n++) + m \\
        &= (n \times m) + m + (n++) \\
        &= ((n++) \times m) + (n++). & \text{(by Definition \ref{2.3.1})}
    \end{align*}
This close the induction.
\end{proof}

\begin{lemma}[Multiplication is commutative]\label{2.3.2}
Let \(n\), \(m\) be natural numbers.
Then \(n \times m = m \times n\)
\end{lemma}

\begin{proof}
We use induction on \(n\) (fixed \(m\)).
For base case \(n = 0\), we want to prove that \(0 \times m = m \times 0\).
By Definition \ref{2.3.1}, \(0 \times m = 0\), and by Lemma \ref{ac 2.3.2}, \(m \times 0 = 0\), thus base case is true.
Suppose inductively that \(n \times m = m \times n\) is true.
Then we want to show that \((n++) \times m = m \times (n++)\) is also true.
By Definition \ref{2.3.1}, \((n++) \times m = (n \times m) + m\).
By Additional Corollary \ref{ac 2.3.3} and induction hypothesis, \(m \times (n++) = (m \times n) + m = (n \times m) + m\).
Thus \((n++) \times m = m \times (n++)\), and this close the induction.
\end{proof}

\begin{note}
We will now abbreviate \(n \times m\) as \(nm\), and use the usual convention that multiplication takes precedence over addition, thus for instance \(ab + c\) means \((a \times b) + c\), not \(a \times (b + c)\).
\end{note}

\begin{lemma}[Positive natural numbers have no zero divisors]\label{2.3.3}
Let \(n\), \(m\) be natural numbers.
Then \(n \times m = 0\) if and only if at least one of \(n\), \(m\) is equal to zero.
In particular, if \(n\) and \(m\) are both positive, then \(nm\) is also positive.
\end{lemma}

\begin{proof}
We first prove the necessary condition of Lemma \ref{2.3.3}.
Suppose for sake of contradiction that \(n\) and \(m\) are positive, and \(n = a++\), \(m = b++\) for some natural number \(a\), \(b\)
Then \(n \times m = (a++) \times (b++) = (a \times (b++)) + (b++) = (a \times b) + a + (b++)\) by Definition \ref{2.3.1} and Lemma \ref{ac 2.3.3}.
But \(b++\) is positive, so \(n \times m\) is positive, a contradiction with \(n \times m = 0\).
Thus at least one of \(n\), \(m\) is \(0\).

Now we prove the sufficient condition of Lemma \ref{2.3.3}.
By Definition \ref{2.3.1}, \(n \times m = 0\) if \(n = 0\), and by Additional Corollary \ref{ac 2.3.2}, \(n \times m = 0\) if \(m = 0\).
\end{proof}

\begin{proposition}[Distributive law]\label{2.3.4}
For any natural numbers \(a\), \(b\), \(c\), we have \(a(b + c) = ab + ac\) and \((b + c)a = ba + ca\).
\end{proposition}

\begin{proof}
Since multiplication is commutative we only need to show the first identity \(a(b + c) = ab + ac\).
We keep \(a\) and \(b\) fixed, and use induction on \(c\).
Let’s prove the base case \(c = 0\), i.e., \(a(b + 0) = ab + a0\).
The left-hand side is \(ab\), while the right-hand side is \(ab + 0 = ab\), so we are done with the base case.
Now let us suppose inductively that \(a(b + c) = ab + ac\), and let us prove that \(a(b + (c++)) = ab + a(c++)\).
The left-hand side is \(a((b + c)++) = a(b + c) + a\), while the right-hand side is \(ab + ac + a = a(b + c) + a\) by the induction hypothesis, and so we can close the induction.
\end{proof}

\begin{proposition}[Multiplication is associative]\label{2.3.5}
For any natural numbers \(a\), \(b\), \(c\), we have \((a \times b) \times c = a \times (b \times c)\).
\end{proposition}

\begin{proof}
We keep \(a\) and \(b\) fixed, and use induction on \(c\).
The base case \(c = 0\), i.e., \((a \times b) \times 0 = a \times (b \times 0)\).
The left-hand side is \(0\), while the right-hand side is \(a \times 0 = 0\), so we are done with the base case.
Now let us suppose inductively that \((a \times b) \times c = a \times (b \times c)\), and let us prove that \((a \times b) \times (c++) = a \times (b \times (c++))\).
The left-hand side is \((a \times b) \times c + (a \times b)\), while the right-hand side is \(a \times (b \times c + b) = a \times (b \times c) + a \times b = (a \times b) \times c + a \times b\) by the Proposition \ref{2.3.4} and induction hypothesis, and so we can close the induction.
\end{proof}

\begin{proposition}[Multiplication preserves order]\label{2.3.6}
If \(a\), \(b\) are natural numbers such that \(a < b\), and \(c\) is positive, then \(ac < bc\).
\end{proposition}

\begin{proof}
Since \(a < b\), we have \(b = a + d\) for some positive \(d\).
Multiplying by \(c\) and using the distributive law we obtain \(bc = ac + dc\).
Since \(d\) is positive, and \(c\) is positive, \(dc\) is positive, and hence \(ac < bc\) as desired.
\end{proof}

\begin{corollary}[Cancellation law]\label{2.3.7}
Let \(a\), \(b\), \(c\) be natural numbers such that \(ac = bc\) and \(c\) is non-zero.
Then \(a = b\).
\end{corollary}

\begin{proof}
By the trichotomy of order, we have three cases: \(a < b\), \(a = b\), \(a > b\).
Suppose first that \(a < b\), then by Proposition \ref{2.3.6} we have \(ac < bc\), a contradiction.
We can obtain a similar contradiction when \(a > b\).
Thus the only possibility is that \(a = b\), as desired.
\end{proof}

\begin{remark}\label{2.3.8}
Just as Proposition \ref{2.2.6} will allow for a ``virtual subtraction'' which will eventually let us define genuine subtraction, Corollary \ref{2.3.7} provides a ``virtual division'' which will be needed to define genuine division later on.
\end{remark}

\begin{proposition}[Euclidean algorithm]\label{2.3.9}
Let \(n\) be a natural number, and let \(q\) be a positive number.
Then there exist natural numbers \(m\), \(r\) such that \(0 \leq r<q\) and \(n = mq + r\).
\end{proposition}

\begin{proof}
We use induction on \(n\) (fixed \(q\)).
For the base case \(n = 0\), let \(r = m = 0\), then \(0 = 0q + 0\), \(0 \leq 0 < q\), so we are done with base case.
Suppose inductively that \(n = mq + r\), \(0 \leq r < q\).
Then \(n++ = (mq + r)++ = mq + (r++)\).
By induction hypothesis, because \(r < q\), so \(r++ \leq q\) by Proposition \ref{2.2.12}.
By Proposition \ref{2.2.13}, we have two cases: \(r++ < q\), \(r++ = q\).
If \(r++ < q\), then \(n++ = mq + (r++)\) satisfied the condition.
If \(r++ = q\), then \(n++ = mq + q = (m++) \times q\), also satisfied the condition.
This close the induction.
\end{proof}

\begin{remark}\label{2.3.10}
In other words, we can divide a natural number \(n\) by a positive number \(q\) to obtain a quotient \(m\) (which is another natural number) and a remainder \(r\) (which is less than \(q\)).
This algorithm marks the beginning of \emph{number theory}, which is a beautiful and important subject but one which is beyond the scope of this text.
\end{remark}

\begin{definition}[Exponentiation for natural numbers]\label{2.3.11}
Let \(m\) be a natural number.
To raise \(m\) to the power \(0\), we define \(m^0 \coloneqq 1\); in particular, we define \(0^0 \coloneqq 1\).
Now suppose recursively that \(m^n\) has been defined for some natural number \(n\), then we define \(m^{n++} \coloneqq m^n \times m\).
\end{definition}

\exercisesection

\begin{exercise}\label{ex 2.3.1}
Prove Lemma \ref{2.3.2}.
\end{exercise}

\begin{proof}
See Lemma \ref{2.3.2}
\end{proof}

\begin{exercise}\label{ex 2.3.2}
Prove Lemma \ref{2.3.3}
\end{exercise}

\begin{proof}
See Lemma \ref{2.3.3}
\end{proof}

\begin{exercise}\label{ex 2.3.3}
Prove Proposition \ref{2.3.5}
\end{exercise}

\begin{proof}
See Proposition \ref{2.3.5}
\end{proof}

\begin{exercise}\label{ex 2.3.4}
Prove the identity \((a + b)^2 = a^2 + 2ab + b^2\) for all natural numbers \(a\), \(b\).
\end{exercise}

\begin{proof}
\begin{align*}
    (a + b)^2 &= (a + b)^1 \times (a + b) & \text{(By Definition \ref{2.3.11})} \\
    &= (a + b)^0 \times (a + b) \times (a + b) & \text{(By Definition \ref{2.3.11})} \\
    &= 1 \times (a + b) \times (a + b) & \text{(By Definition \ref{2.3.11})} \\
    &= (1 \times (a + b)) \times (a + b) & \text{(By Proposition \ref{2.3.5})} \\
    &= (1 \times a + 1 \times b)) \times (a + b) & \text{(By Proposition \ref{2.3.4})} \\
    &= ((0 \times a + a) + (0 \times b + b)) \times (a + b) & \text{(By Definition \ref{2.3.1})} \\
    &= (a + b) \times (a + b) & \text{(By Definition \ref{2.3.1})} \\
    &= a(a + b) + b(a + b) & \text{(By Proposition \ref{2.3.5})} \\
    &= aa + ab + ba + bb & \text{(By Proposition \ref{2.3.5})} \\
    &= a^2 + ab + ba + b^2 & \text{(By Definition \ref{2.3.11})} \\
    &= a^2 + ab + ab + b^2 & \text{(By Lemma \ref{2.3.2})} \\
    &= a^2 + (0 + ab) + ab + b^2 & \text{(By Definition \ref{2.2.1})} \\
    &= a^2 + 0ab + ab + ab + b^2 & \text{(By Definition \ref{2.3.1})} \\
    &= a^2 + 1ab + ab + b^2 & \text{(By Lemma \ref{2.3.2})} \\
    &= a^2 + 2ab + b^2 & \text{(By Definition \ref{2.3.1})} \\
\end{align*}
\end{proof}

\begin{exercise}\label{ex 2.3.5}
Prove Proposition \ref{2.3.9}
\end{exercise}

\begin{proof}
See Proposition \ref{2.3.9}
\end{proof}
\chapter{Set Theory}\label{ch 3}

\section{Fundamentals}\label{sec 3.1}

\begin{definition}\label{3.1.1}
    We define a \emph{set} \(A\) to be any unordered collection of objects.
    If \(x\) is an object, we say that \emph{\(x\) is an element of \(A\)} or \(x \in A\) if \(x\) lies in the collection;
    otherwise we say that \(x \notin A\).
\end{definition}

\begin{axiom}[Sets are objects]\label{3.1}
    If \(A\) is a set, then \(A\) is also an object.
    In particular, given two sets \(A\) and \(B\), it is meaningful to ask whether \(A\) is also an element of \(B\).
\end{axiom}

\setcounter{theorem}{2}
\begin{remark}\label{3.1.3}
    There is a special case of set theory, called ``pure set theory'', in which \emph{all} objects are sets;
    for instance the number \(0\) might be identified with the empty set \(\emptyset = \{\}\), the number \(1\) might be identified with \(\{0\} = \{\{\}\}\), the number \(2\) might be identified with \(\{0, 1\} = \{\{\}, \{\{\}\}\}\), and so forth.
    From a logical point of view, pure set theory is a simpler theory, since one only has to deal with sets and not with objects;
    however, from a conceptual point of view it is often easier to deal with impure set theories in which some objects are not considered to be sets.
    The two types of theories are more or less equivalent for the purposes of doing mathematics, and so we shall take an agnostic position as to whether all objects are sets or not.
\end{remark}

\begin{definition}[Equality of sets]\label{3.1.4}
    Two sets \(A\) and \(B\) are \emph{equal}, \(A = B\), iff every element of \(A\) is an element of \(B\) and vice versa.
    To put it another way, \(A = B\) if and only if every element \(x\) of \(A\) belongs also to \(B\), and every element \(y\) of \(B\) belongs also to \(A\).
\end{definition}

\begin{additional corollary}\label{ac 3.1.1}
The definition of equality in Definition \ref{3.1.4} is reflexive, symmetric and transitive.
\end{additional corollary}

\begin{proof}
    We first prove that Definition \ref{3.1.4} is reflexive.
    Let \(A\) be a set.
    Since
    \[
        \forall x : x \in A \implies x \in A,
    \]
    we have Definition \ref{3.1.4} is reflexive, i.e., \(A = A\).

    Next we prove that Definition \ref{3.1.4} is symmetric.
    Let \(A, B\) be sets and suppose \(A = B\).
    Since
    \[
        (\forall x : x \in A \iff x \in B) \iff (\forall x : x \in B \iff x \in A),
    \]
    we have Definition \ref{3.1.4} is symmetric, i.e., \(A = B \iff B = A\).

    Finally, we prove that Definition \ref{3.1.4} is transitive.
    Let \(A, B, C\) be sets and suppose \(A = B\) and \(B = C\).
    Since
    \[
        (\forall x : (x \in A \implies x \in B) \land (x \in B \implies x \in C)) \implies (\forall x : x \in A \implies x \in C),
    \]
    we have Definition \ref{3.1.4} is transitive, i.e., \(A = B \land B = C \implies A = C\).
\end{proof}

\begin{note}
    Observe that if \(x \in A\) and \(A = B\), then \(x \in B\), by Definition \ref{3.1.4}.
    Thus the ``is an element of'' relation \(\in\) obeys the axiom of substitution.
    Because of this, any new operation we define on sets will also obey the axiom of substitution, as long as we can define that operation purely in terms of the relation \(\in\).
\end{note}

\begin{note}
    Next, we turn to the issue of exactly which objects are sets and which objects are not.
    The situation is analogous to how we defined the natural numbers in the previous chapter;
    we started with a single natural number, \(0\), and started building more numbers out of \(0\) using the increment operation.
    We will try something similar here, starting with a single set, the \emph{empty set},
    and building more sets out of the empty set by various operations.
    We begin by postulating the existence of the empty set.
\end{note}

\begin{axiom}[Empty set]\label{3.2}
    There exists a set \(\emptyset\), known as the empty set, which contains no elements, i.e., for every object \(x\) we have \(x \notin \emptyset\).
\end{axiom}

\begin{note}
    The empty set is also denoted \(\{\}\).
\end{note}

\begin{additional corollary}\label{ac 3.1.2}
There can only be one empty set;
if there were two sets \(\emptyset\) and \(\emptyset'\) which were both empty, then they would be equal to each other.
\end{additional corollary}

\begin{proof}
    Suppose there exist two empty set \(\emptyset\) and \(\emptyset'\).
    Then we have
    \begin{align*}
             & (\forall x : (x \in \emptyset \implies x \in \emptyset') \land (x \in \emptyset' \implies x \in \emptyset)) & \text{(vacuously true)}            \\
        \iff & (\forall x : x \in \emptyset \iff x \in \emptyset')                                                                                              \\
        \iff & \emptyset = \emptyset'.                                                                                     & \text{(by Definition \ref{3.1.4})}
    \end{align*}
\end{proof}

\begin{note}
    If a set is not equal to the empty set, we call it \emph{non-empty}.
\end{note}

\setcounter{theorem}{5}
\begin{lemma}[Single choice]\label{3.1.6}
    Let \(A\) be a non-empty set.
    Then there exists an object \(x\) such that \(x \in A\).
\end{lemma}

\begin{proof}
    We prove by contradiction.
    Suppose there does not exist any object \(x\) such that \(x \in A\).
    Then for all objects \(x\), we have \(x \notin A\).
    Also, by Axiom \ref{3.2} we have \(x \notin \emptyset\).
    Thus \(x \in A \iff x \in \emptyset\) (both statements are equally false), and so \(A = \emptyset\) by Definition \ref{3.1.4}, a contradiction.
\end{proof}

\begin{remark}\label{3.1.7}
    The above Lemma asserts that given any non-empty set \(A\), we are allowed to ``choose'' an element \(x\) of \(A\) which demonstrates this non-emptyness.
    Later on (in Lemma \ref{3.5.12}) we will show that given any finite number of non-empty sets, say \(A_1, \dots, A_n\), it is possible to choose one element \(x_1, \dots, x_n\) from each set \(A_1, \dots, A_n\);
    this is known as ``finite choice''.
    However, in order to choose elements from an infinite number of sets, we need an additional axiom, the \emph{axiom of choice} (Axiom \ref{8.1}).
\end{remark}

\begin{remark}\label{3.1.8}
    Note that the empty set is \emph{not} the same thing as the natural number \(0\).
    One is a set;
    the other is a number.
    However, it is true that the \emph{cardinality} of the empty set is \(0\).
\end{remark}

\begin{axiom}[Singleton sets and pair sets]\label{3.3}
    If \(a\) is an object, then there exists a set \(\{a\}\) whose only element is \(a\), i.e., for every object \(y\), we have \(y \in \{a\}\) if and only if \(y = a\);
    we refer to \(\{a\}\) as the \emph{singleton set} whose element is \(a\).
    Furthermore, if \(a\) and \(b\) are objects, then there exists a set \(\{a, b\}\) whose only elements are \(a\) and \(b\);
    i.e., for every object \(y\), we have \(y \in \{a, b\}\) if and only if \(y = a\) or \(y = b\);
    we refer to this set as the \emph{pair set} formed by \(a\) and \(b\).
\end{axiom}

\begin{remark}\label{3.1.9}
    There is only one singleton set for each object \(a\).
    Similarly, given any two objects \(a\) and \(b\), there is only one pair set formed by \(a\) and \(b\).
    Thus the singleton set axiom is in fact redundant, being a consequence of the pair set axiom.
    Conversely, the pair set axiom will follow from the singleton set axiom and the pairwise union axiom (Axiom \ref{3.4}).
\end{remark}

\begin{proof}
    We first show the uniqueness of singleton set.
    Suppose there exists two sets \(A\) and \(A'\) which are singleton sets of object \(a\).
    Then we have
    \begin{align*}
             & (\forall x : x \in A \iff x = a) \land (\forall x : x \in A' \iff x = a) & \text{(by Axiom \ref{3.3})}        \\
        \iff & \forall x : x \in A \iff x \in A'                                                                             \\
        \iff & A = A'.                                                                  & \text{(by Definition \ref{3.1.4})}
    \end{align*}

    Next we show the uniqueness of pair set.
    Suppose there exists two sets \(X\) and \(X'\) which are pair sets of object \(a\) and \(b\).
    Then we have
    \begin{align*}
             & (\forall x : x \in X \iff (x = a) \lor (x = b))                                             \\
             & \land (\forall x : x \in X' \iff (x = a) \lor (x = b)) & \text{(by Axiom \ref{3.3})}        \\
        \iff & \forall x : x \in X \iff x \in X'                                                           \\
        \iff & X = X'.                                                & \text{(by Definition \ref{3.1.4})}
    \end{align*}
\end{proof}

\begin{example}\label{3.1.10}
    Since \(\emptyset\) is a set (and hence an object), so is the singleton set \(\{\emptyset\}\), i.e., the set whose only element is \(\emptyset\), is a set (and it is not the same set as \(\emptyset\), \(\{\emptyset\} \neq \emptyset\)).
    Similarly, the singleton set \(\{\{\emptyset\}\}\) and the pair set \(\{\emptyset, \{\emptyset\}\}\) are also sets.
    These three sets are not equal to each other.
\end{example}

\begin{axiom}[Pairwise union]\label{3.4}
    Given any two sets \(A\), \(B\), there exists a set \(A \cup B\), called the \emph{union} \(A \cup B\) of \(A\) and \(B\), whose elements consist of all the elements which belong to \(A\) or \(B\) or both.
    In other words, for any object \(x\),
    \[
        x \in A \cup B \iff (x \in A \lor x \in B).
    \]
\end{axiom}

\setcounter{theorem}{11}
\begin{remark}\label{3.1.12}
    If \(A\), \(B\), \(A'\) are sets, and \(A\) is equal to \(A'\), then \(A \cup B\) is equal to \(A' \cup B\).
    Similarly if \(B'\) is a set which is equal to \(B\), then \(A \cup B\) is equal to \(A \cup B'\).
    Thus the operation of union obeys the axiom of substitution, and is thus well-defined on sets.
\end{remark}

\begin{proof}
    Suppose \(A, A', B\) are sets such that \(A = A'\).
    Then we have
    \begin{align*}
             & \forall x : x \in A \cup B                                      \\
        \iff & x \in A \lor x \in B       & \text{(by Axiom \ref{3.4})}        \\
        \iff & x \in A' \lor x \in B      & \text{(by Definition \ref{3.1.4})} \\
        \iff & x \in A' \cup B.           & \text{(by Axiom \ref{3.4})}
    \end{align*}

    Similarly, suppose \(A, B, B'\) are sets such that \(B = B'\).
    Then we have
    \begin{align*}
             & \forall x : x \in A \cup B                                      \\
        \iff & x \in A \lor x \in B       & \text{(by Axiom \ref{3.4})}        \\
        \iff & x \in A \lor x \in B'      & \text{(by Definition \ref{3.1.4})} \\
        \iff & x \in A \cup B'.           & \text{(by Axiom \ref{3.4})}
    \end{align*}
\end{proof}

\begin{lemma}\label{3.1.13}
    If \(a\) and \(b\) are objects, then \(\{a, b\} = \{a\} \cup \{b\}\).
    If \(A\), \(B\), \(C\) are sets, then the union operation is commutative (i.e., \(A \cup B = B \cup A\)) and associative (i.e., \((A \cup B) \cup C = A \cup (B \cup C)\)).
    Also, we have \(A \cup A = A \cup \emptyset = \emptyset \cup A = A\).
\end{lemma}

\begin{proof}
    We first show that \(\{a, b\} = \{a\} \cup \{b\}\).
    By Axiom \ref{3.3}, the sets \(\{a\}, \{b\}, \{a, b\}\) exist.
    And by Axiom \ref{3.4}, the set \(\{a\} \cup \{b\}\) exists.
    Then we have
    \begin{align*}
             & (\forall x : x \in \{a, b\} \iff x = a \lor x = b)             & \text{(by Axiom \ref{3.3})}        \\
        \iff & (\forall x : x \in \{a, b\} \iff x \in \{a\} \lor x \in \{b\}) & \text{(by Axiom \ref{3.3})}        \\
        \iff & (\forall x : x \in \{a, b\} \iff x \in \{a\} \cup \{b\})       & \text{(by Axiom \ref{3.4})}        \\
        \iff & \{a, b\} = \{a\} \cup \{b\}.                                   & \text{(by Definition \ref{3.1.4})}
    \end{align*}

    Next we show the commutative identity of union sets.
    Suppose that \(A, B\) are sets.
    By Axiom \ref{3.4}, the sets \(A \cup B\) and \(B \cup A\) exists.
    Then we have
    \begin{align*}
             & (\forall x : x \in A \cup B \iff x \in A \lor x \in B) & \text{(by Axiom \ref{3.4})}        \\
        \iff & (\forall x : x \in A \cup B \iff x \in B \lor x \in A)                                      \\
        \iff & (\forall x : x \in A \cup B \iff x \in B \cup A)       & \text{(by Axiom \ref{3.4})}        \\
        \iff & A \cup B = B \cup A.                                   & \text{(by Definition \ref{3.1.4})}
    \end{align*}

    Next we show the associativity identity of union sets.
    By Definition \ref{3.1.4}, we need to show that every element \(x\) of \((A \cup B) \cup C\) is an element of \(A \cup (B \cup C)\), and vice versa.
    So suppose first that \(x\) is an element of \((A \cup B) \cup C\).
    By Axiom \ref{3.4}, this means that at least one of \(x \in A \cup B\) or \(x \in C\) is true.
    We now divide into two cases.
    If \(x \in C\), then by Axiom \ref{3.4} again \(x \in B \cup C\), and so by Axiom \ref{3.4} again we have \(x \in A \cup (B \cup C)\).
    Now suppose instead \(x \in A \cup B\), then by Axiom \ref{3.4} again \(x \in A\) or \(x \in B\).
    If \(x \in A\) then \(x \in A \cup (B \cup C)\) by Axiom \ref{3.4}, while if \(x \in B\) then by consecutive applications of Axiom \ref{3.4} we have \(x \in B \cup C\) and hence \(x \in A \cup (B \cup C)\).
    Thus in all cases we see that every element of \((A \cup B) \cup C\) lies in \(A \cup (B \cup C)\).
    A similar argument shows that every element of \(A \cup (B \cup C)\) lies in \((A \cup B) \cup C\), and so \((A \cup B) \cup C = A \cup (B \cup C) \) as desired.

    Finally we show that \(A \cup A = A \cup \emptyset = \emptyset \cup A = A\).
    Suppose that \(A\) is a set.
    Then we have
    \begin{align*}
             & (\forall x : x \in A \iff x \in A \lor x \in A)                                              \\
        \iff & (\forall x : x \in A \iff x \in A \cup A)               & \text{(by Axiom \ref{3.4})}        \\
        \iff & (A = A \cup A)                                          & \text{(by Definition \ref{3.1.4})} \\
        \iff & (\forall x : x \in A \iff x \in A \lor x \in \emptyset) & \text{(vacuously true)}            \\
        \iff & A = A \cup \emptyset                                    & \text{(by Definition \ref{3.1.4})} \\
        \iff & (\forall x : x \in A \iff x \in \emptyset \lor x \in A) & \text{(vacuously true)}            \\
        \iff & A = \emptyset \cup A.                                   & \text{(by Definition \ref{3.1.4})}
    \end{align*}
\end{proof}

\begin{note}
    Because of Lemma \ref{3.1.13}, we do not need to use parentheses to denote multiple unions, thus for instance we can write \(A \cup B \cup C\) instead of \((A \cup B) \cup C\) or \(A \cup (B \cup C)\).
    Similarly for unions of four sets, \(A \cup B \cup C \cup D\), etc.
\end{note}

\begin{remark}\label{3.1.14}
    While the operation of union has some similarities with addition, the two operations are \emph{not} identical.
\end{remark}

\begin{note}
    Axiom \ref{3.4} allows us to define triplet sets, quadruplet sets, and so forth: if \(a, b, c\) are three objects, we define \(\{a, b, c\} \coloneqq \{a\} \cup \{b\} \cup \{c\}\);
    if \(a, b, c, d\) are four objects, then we define \(\{a, b, c, d\} \coloneqq \{a\} \cup \{b\} \cup \{c\} \cup \{d\}\), and so forth.
    On the other hand, we are not yet in a position to define sets consisting of \(n\) objects for any given natural number \(n\);
    this would require iterating the above construction ``\(n\) times'', but the concept of \(n\)-fold iteration has not yet been rigorously defined.
    For similar reasons, we cannot yet define sets consisting of infinitely many objects, because that would require iterating the axiom of pairwise union (Axiom \ref{3.4}) infinitely often, and it is not clear at this stage that one can do this rigorously.
    Later on, we will introduce other axioms of set theory which allow one to construct arbitrarily large, and even infinite, sets.
\end{note}

\begin{definition}[Subsets]\label{3.1.15}
    Let \(A\), \(B\) be sets.
    We say that \(A\) is a \emph{subset} of \(B\), denoted \(A \subseteq B\), iff every element of \(A\) is also an element of \(B\), i.e.
    \[
        \text{For any object } x, x \in A \implies x \in B.
    \]
    We say that \(A\) is a \emph{proper subset} of \(B\), denoted \(A \subsetneq B\), if \(A \subseteq B\) and \(A \neq B\).
\end{definition}

\begin{remark}\label{3.1.16}
    Because these definitions involve only the notions of equality and the ``is an element of'' relation, both of which already obey the axiom of substitution, the notion of subset also automatically obeys the axiom of substitution.
    Thus for instance if \(A \subseteq B\) and \(A = A'\), then \(A' \subseteq B\).
\end{remark}

\begin{example}\label{3.1.17}
    Given any set \(A\), we always have \(A \subseteq A\) and \(\emptyset \subseteq A\).
\end{example}

\begin{proof}
    Suppose that \(A\) is a set.
    Then we have
    \begin{align*}
        \top \iff & (\forall x : x \in A \implies x \in A)                                       \\
        \iff      & A \subseteq A.                         & \text{(by Definition \ref{3.1.15})}
    \end{align*}
    And we also have
    \begin{align*}
             & (\forall x : x \in \emptyset \implies x \in A) & \text{(vacuously true)}             \\
        \iff & \emptyset \subseteq A.                         & \text{(by Definition \ref{3.1.15})}
    \end{align*}
\end{proof}

\begin{proposition}[Sets are partially ordered by set inclusion]\label{3.1.18}
    Let \(A\), \(B\), \(C\) be sets.
    If \(A \subseteq B\) and \(B \subseteq C\) then \(A \subseteq C\).
    \(A \subseteq B\) and \(B \subseteq A\) if and only if \(A = B\).
    Finally, if \(A \subsetneq B\) and \(B \subsetneq C\) then \(A \subsetneq C\).
\end{proposition}

\begin{proof}
    We first show that \(A \subseteq B \land B \subseteq C \implies A \subseteq C\).
    Suppose that \(A \subseteq B\) and \(B \subseteq C\).
    To prove that \(A \subseteq C\), we have to prove that every element of \(A\) is an element of \(C\).
    So, let us pick an arbitrary element \(x\) of \(A\).
    Then, since \(A \subseteq B\), \(x\) must then be an element of \(B\).
    But then since \(B \subseteq C\), \(x\) is an element of \(C\).
    Thus every element of \(A\) is indeed an element of \(C\), as claimed.

    Next we show that \(A \subseteq B \land B \subseteq A \iff A = B\).
    Suppose that \(A, B\) are sets and \(A \subseteq B \land B \subseteq A\).
    Then we have
    \begin{align*}
             & A \subseteq B \land B \subseteq A                                                                               \\
        \iff & (\forall x : (x \in A \implies x \in B) \land (x \in B \implies x \in A)) & \text{(by Definition \ref{3.1.15})} \\
        \iff & (\forall x : x \in A \iff x \in B)                                                                              \\
        \iff & A = B.                                                                    & \text{(by Definition \ref{3.1.4})}
    \end{align*}

    Finally we show that \(A \subsetneq B \land B \subsetneq C \implies A \subsetneq C\).
    Suppose that \(A, B, C\) are sets and \(A \subsetneq B \land B \subsetneq C\).
    Then we have
    \begin{align*}
                 & A \subsetneq B \land B \subsetneq C                                                                                                       \\
        \implies & (\forall x : x \in A \implies x \in B) \land (A \neq B)                                                                                   \\
                 & \land (\forall x : x \in B \implies x \in C) \land (B \neq C)                   & \text{(by Definition \ref{3.1.15})}                     \\
        \implies & (\forall x : x \in A \implies x \in B)                                                                                                    \\
                 & \land \lnot(\forall x : x \in A \iff x \in B)                                                                                             \\
                 & \land (\forall x : x \in B \implies x \in C)                                                                                              \\
                 & \land (B \neq C)                                                                                                                          \\
        \implies & (\forall x : x \in A \implies x \in B)                                                                                                    \\
                 & \land (\exists\ x : (x \in A \land x \notin B) \lor (x \in B \land x \notin A))                                                           \\
                 & \land (\forall x : x \in B \implies x \in C)                                                                                              \\
                 & \land (B \neq C)                                                                                                                          \\
        \implies & (\forall x : x \in A \implies x \in B)                                                                                                    \\
                 & \land (\exists\ x : x \in B \land x \notin A)                                   & \text{(since \(\forall x : x \in A \implies x \in B\))} \\
                 & \land (\forall x : x \in B \implies x \in C)                                                                                              \\
                 & \land (B \neq C)                                                                                                                          \\
        \implies & (\forall x : x \in A \implies x \in B)                                                                                                    \\
                 & \land (\exists\ x : x \in C \land x \notin A)                                   & \text{(since \(\forall x : x \in B \implies x \in C\))} \\
                 & \land (\forall x : x \in B \implies x \in C)                                                                                              \\
                 & \land (B \neq C)                                                                                                                          \\
        \implies & (\forall x : x \in A \implies x \in C)                                                                                                    \\
                 & \land (A \neq C) \land (B \neq C)                                                                                                         \\
        \implies & A \subseteq C \land A \neq C \land B \neq C                                     & \text{(by Definition \ref{3.1.15})}                     \\
        \implies & A \subsetneq C \land B \neq C                                                   & \text{(by Definition \ref{3.1.15})}                     \\
        \implies & A \subsetneq C.
    \end{align*}
\end{proof}

\setcounter{theorem}{19}
\begin{remark}\label{3.1.20}
    There is one important difference between the subset relation \(\subsetneq\) and the less than relation \(<\).
    Given any two distinct natural numbers \(n\), \(m\), we know that one of them is smaller than the other (Proposition \ref{2.2.13});
    however, given two distinct sets, it is not in general true that one of them is a subset of the other.
    we say that sets are only \emph{partially ordered}, whereas the natural numbers are \emph{totally ordered}.
\end{remark}

\begin{remark}\label{3.1.21}
    We should also caution that the subset relation \(\subseteq\) is not the same as the element relation \(\in\).
    It is important to distinguish sets from their elements, as they can have different properties.
    For instance, it is possible to have an infinite set consisting of finite numbers (the set \(\N\) of natural numbers is one such example), and it is also possible to have a finite set consisting of infinite objects
    (consider for instance the finite set \(\{\N, \Z, \Q, \R\}\), which has four elements, all of which are infinite).
\end{remark}

\begin{axiom}[Axiom of specification]\label{3.5}
    Let \(A\) be a set, and for each \(x \in A\), let \(P(x)\) be a property pertaining to \(x\) (i.e., \(P(x)\) is either a true statement or a false statement).
    Then there exists a set, called \(\{x \in A : P(x) \text{ is true}\}\) (or simply \(\{x \in A : P(x)\}\) for short), whose elements are precisely the elements \(x\) in \(A\) for which \(P(x)\) is true.
    In other words, for any object \(y\),
    \[
        y \in \{x \in A : P(x) \text{ is true}\} \iff (y \in A \text{ and } P(y) \text{ is true}).
    \]
\end{axiom}

\begin{note}
    Axiom \ref{3.5} is also known as the \emph{axiom of separation}.
    We sometimes write \(\{x \in A \mid P(x)\}\) instead of \(\{x \in A : P(x)\}\);
    this is useful when we are using the colon ``:'' to denote something else.
\end{note}

\setcounter{theorem}{22}
\begin{definition}[Intersections]\label{3.1.23}
    The intersection \(S_1 \cap S_2\) of two sets is defined to be the set
    \[
        S_1 \cap S_2 \coloneqq \{x \in S_1 : x \in S_2\}.
    \]
    In other words, \(S_1 \cap S_2\) consists of all the elements which belong to both \(S_1\) and \(S_2\).
    Thus, for all objects \(x\),
    \[
        x \in S_1 \cap S_2 \iff x \in S_1 \text{ and } x \in S_2.
    \]
\end{definition}

\begin{note}
    Two sets \(A\), \(B\) are said to be \emph{disjoint} if \(A \cap B = \emptyset\).
    This is not the same concept as being \emph{distinct}, \(A \neq B\).
    Meanwhile, the sets \(\emptyset\) and \(\emptyset\) are disjoint but not distinct.
\end{note}

\setcounter{theorem}{26}
\begin{definition}[Difference sets]\label{3.1.27}
    Given two sets \(A\) and \(B\), we define the set \(A - B\) or \(A \setminus B\) to be the set \(A\) with any elements of \(B\) removed:
    \[
        A \setminus B \coloneqq \{x \in A : x \notin B\}.
    \]
\end{definition}

\begin{proposition}[Sets form a boolean algebra]\label{3.1.28}
    Let \(A\), \(B\), \(C\) be sets, and let \(X\) be a set containing \(A\), \(B\), \(C\) as subsets.
    \begin{enumerate}
        \item (Minimal element) We have \(A \cup \emptyset = A\) and \(A \cap \emptyset = \emptyset\).
        \item (Maximal element) We have \(A \cup X = X\) and \(A \cap X = A\).
        \item (Identity) We have \(A \cap A = A\) and \(A \cup A = A\).
        \item (Commutativity) We have \(A \cup B = B \cup A\) and \(A \cap B = B \cap A\).
        \item (Associativity) We have \((A \cup B) \cup C = A \cup (B \cup C)\) and \((A \cap B) \cap C = A \cap (B \cap C)\).
        \item (Distributivity) We have \(A \cap (B \cup C) = (A \cap B) \cup (A \cap C)\) and \(A \cup (B \cap C) = (A \cup B) \cap (A \cup C)\).
        \item (Partition) We have \(A \cup (X \setminus A) = X\) and \(A \cap (X \setminus A) = \emptyset\).
        \item (De Morgan laws) We have \(X \setminus (A \cup B) = (X \setminus A) \cap (X \setminus B)\) and \(X \setminus (A \cap B) = (X \setminus A) \cup (X \setminus B)\).
    \end{enumerate}
\end{proposition}

\begin{proof}{(a)}
    Suppose that \(A\) is a set.
    By Lemma \ref{3.1.13} we have \(A \cup \emptyset = A\).
    We only need to show that \(A \cap \emptyset = \emptyset\).
    \begin{align*}
        \top \iff & (\bot \iff \bot)                                                 & \text{(vacuously true)}             \\
        \iff      & (\forall x : x \in \emptyset \iff x \in \emptyset)               & \text{(vacuously true)}             \\
        \iff      & (\forall x : x \in \emptyset \iff x \in A \land x \in \emptyset)                                       \\
        \iff      & (\forall x : x \in \emptyset \iff x \in A \cap \emptyset)        & \text{(by Definition \ref{3.1.23})} \\
        \iff      & \emptyset = A \cap \emptyset.                                    & \text{(by Definition \ref{3.1.4})}
    \end{align*}
\end{proof}

\begin{proof}{(b)}
    Suppose that \(A, X\) are sets and \(A \subseteq X\).
    Then we have
    \begin{align*}
        \top \iff                     & (\forall x : x \in X \implies x \in A \lor x \in X)                                       \\
        \iff                          & (\forall x : x \in X \implies x \in A \cup X).      & \text{(by Axiom \ref{3.4})}         \\
        A \subseteq X \iff            & (\forall x : x \in A \implies x \in X)              & \text{(by Definition \ref{3.1.15})} \\
        \iff                          & (\forall x : x \in A \lor x \in X \implies x \in X)                                       \\
        \iff                          & (\forall x : x \in A \cup X \implies x \in X).      & \text{(by Axiom \ref{3.4})}         \\
        \top \land A \subseteq X \iff & (\forall x : x \in X \iff x \in A \cup X)                                                 \\
        \iff                          & X = A \cup X.                                       & \text{(by Definition \ref{3.1.4})}
    \end{align*}
    And we also have
    \begin{align*}
        \top \iff                     & (\forall x : x \in A \land x \in X \implies x \in A)                                       \\
        \iff                          & (\forall x : x \in A \cap X \implies x \in A).       & \text{(by Definition \ref{3.1.23})} \\
        A \subseteq X \iff            & (\forall x : x \in A \implies x \in X)               & \text{(by Definition \ref{3.1.15})} \\
        \iff                          & (\forall x : x \in A \implies x \in A \land x \in X)                                       \\
        \iff                          & (\forall x : x \in A \implies x \in A \cap X).       & \text{(by Definition \ref{3.1.23})} \\
        \top \land A \subseteq X \iff & (\forall x : x \in A \cap X \iff x \in A)                                                  \\
        \iff                          & A \cap X = A.                                        & \text{(by Definition \ref{3.1.4})}
    \end{align*}
\end{proof}

\begin{proof}{(c)}
    Suppose that \(A\) is a set.
    By Lemma \ref{3.1.13} we have \(A \cup A = A\).
    We only need to show that \(A \cap A = A\).
    \begin{align*}
        \top \iff & (\forall x : x \in A \iff x \in A)                                                     \\
        \iff      & (\forall x : x \in A \land x \in A \iff x \in A)                                       \\
        \iff      & (\forall x : x \in A \cap A \iff x \in A)        & \text{(by Definition \ref{3.1.23})} \\
        \iff      & A \cap A = A.                                    & \text{(by Definition \ref{3.1.4})}
    \end{align*}
\end{proof}

\begin{proof}{(d)}
    Suppose that \(A, B\) are sets.
    Then we have
    \begin{align*}
             & (\forall x : x \in A \cup B \iff x \in A \lor x \in B) & \text{(by Axiom \ref{3.4})}        \\
        \iff & (\forall x : x \in A \cup B \iff x \in B \lor x \in A)                                      \\
        \iff & (\forall x : x \in A \cup B \iff x \in B \cup A)       & \text{(by Axiom \ref{3.4})}        \\
        \iff & A \cup B = B \cup A.                                   & \text{(by Definition \ref{3.1.4})}
    \end{align*}
    And we also have
    \begin{align*}
             & (\forall x : x \in A \cap B \iff x \in A \land x \in B) & \text{(by Definition \ref{3.1.23})} \\
        \iff & (\forall x : x \in A \cap B \iff x \in B \land x \in A)                                       \\
        \iff & (\forall x : x \in A \cap B \iff x \in B \cap A)        & \text{(by Definition \ref{3.1.23})} \\
        \iff & A \cap B = B \cap A.                                    & \text{(by Definition \ref{3.1.4})}
    \end{align*}
\end{proof}

\begin{proof}{(e)}
    Suppose that \(A, B, C\) are sets.
    Then we have
    \begin{align*}
             & (\forall x : x \in (A \cup B) \cup C \iff (x \in A \lor x \in B) \lor x \in C) & \text{(by Axiom \ref{3.4})}        \\
        \iff & (\forall x : x \in (A \cup B) \cup C \iff x \in A \lor (x \in B \lor x \in C))                                      \\
        \iff & (\forall x : x \in (A \cup B) \cup C \iff x \in A \cup (B \cup C))             & \text{(by Axiom \ref{3.4})}        \\
        \iff & (A \cup B) \cup C = A \cup (B \cup C).                                         & \text{(by Definition \ref{3.1.4})}
    \end{align*}
    And we also have
    \begin{align*}
             & (\forall x : x \in (A \cap B) \cap C \iff (x \in A \land x \in B) \land x \in C) & \text{(by Definition \ref{3.1.23})} \\
        \iff & (\forall x : x \in (A \cap B) \cap C \iff x \in A \land (x \in B \land x \in C))                                       \\
        \iff & (\forall x : x \in (A \cap B) \cap C \iff x \in A \cap (B \cap C))               & \text{(by Definition \ref{3.1.23})} \\
        \iff & (A \cap B) \cap C = A \cap (B \cap C).                                           & \text{(by Definition \ref{3.1.4})}
    \end{align*}
\end{proof}

\begin{proof}{(f)}
    Suppose that \(A, B, C\) are sets.
    Then we have
    \begin{align*}
             & \forall x : x \in A \cap (B \cup C)                                                        \\
        \iff & x \in A \land x \in B \cup C                         & \text{(by Definition \ref{3.1.23})} \\
        \iff & x \in A \land (x \in B \lor x \in C)                 & \text{(by Axiom \ref{3.4})}         \\
        \iff & (x \in A \land x \in B) \lor (x \in A \land x \in C)                                       \\
        \iff & (x \in A \cap B) \lor (x \in A \cap C)               & \text{(by Definition \ref{3.1.23})} \\
        \iff & x \in (A \cap B) \cup (A \cap C).                    & \text{(by Axiom \ref{3.4})}
    \end{align*}
    Thus by Definition \ref{3.1.4} we have \(A \cap (B \cup C) = (A \cap B) \cup (A \cap C)\).
    Similarly we have
    \begin{align*}
             & \forall x : x \in A \cup (B \cap C)                                                       \\
        \iff & x \in A \lor x \in B \cap C                         & \text{(by Axiom \ref{3.4})}         \\
        \iff & x \in A \lor (x \in B \land x \in C)                & \text{(by Definition \ref{3.1.23})} \\
        \iff & (x \in A \lor x \in B) \land (x \in A \lor x \in C)                                       \\
        \iff & (x \in A \cup B) \land (x \in A \cup C)             & \text{(by Axiom \ref{3.4})}         \\
        \iff & x \in (A \cup B) \cap (A \cup C).                   & \text{(by Definition \ref{3.1.23})}
    \end{align*}
    Thus by Definition \ref{3.1.4} we have \(A \cup (B \cap C) = (A \cup B) \cap (A \cup C)\).
\end{proof}

\begin{proof}{(g)}
    Suppose that \(A, X\) are sets and \(A \subseteq X\).
    Then we have
    \begin{align*}
             & \forall x : x \in A \cup (X \setminus A)                                                         \\
        \iff & x \in A \lor x \in (X \setminus A)                     & \text{(by Axiom \ref{3.4})}             \\
        \iff & x \in A \lor (x \in X \land x \notin A)                & \text{(by Definition \ref{3.1.27})}     \\
        \iff & (x \in A \lor x \in X) \land (x \in A \lor x \notin A)                                           \\
        \iff & (x \in A \lor x \in X) \land \top                                                                \\
        \iff & x \in A \lor x \in X                                                                             \\
        \iff & x \in A \cup X                                         & \text{(by Axiom \ref{3.4})}             \\
        \iff & x \in X.                                               & \text{(by Proposition \ref{3.1.28})(b)}
    \end{align*}
    Thus by Definition \ref{3.1.4} we have \(A \cup (X \setminus A) = X\).
    Similarly we have
    \begin{align*}
             & \forall x : x \in A \cap (X \setminus A)                                       \\
        \iff & x \in A \land x \in (X \setminus A)      & \text{(by Definition \ref{3.1.23})} \\
        \iff & x \in A \land (x \in X \land x \notin A) & \text{(by Definition \ref{3.1.27})} \\
        \iff & (x \in A \land x \notin A) \land x \in X                                       \\
        \iff & \bot \land x \in X                                                             \\
        \iff & \bot                                                                           \\
        \iff & x \in \emptyset.                         & \text{(vacuously true)}
    \end{align*}
    Thus by Definition \ref{3.1.4} we have \(A \cap (X \setminus A) = \emptyset\).
\end{proof}

\begin{proof}{(h)}
    Suppose that \(A, B, X\) are sets such that \(A \subseteq X\) and \(B \subseteq X\).
    Then we have
    \begin{align*}
             & \forall x : x \in X \setminus (A \cup B)                                                          \\
        \iff & x \in X \land x \notin (A \cup B)                           & \text{(by Definition \ref{3.1.27})} \\
        \iff & x \in X \land \lnot (x \in A \cup B)                                                              \\
        \iff & x \in X \land \lnot (x \in A \lor x \in B)                  & \text{(by Axiom \ref{3.4})}         \\
        \iff & x \in X \land (x \notin A \land x \notin B)                                                       \\
        \iff & (x \in X \land x \notin A) \land (x \in X \land x \notin B)                                       \\
        \iff & (x \in X \setminus A) \land (x \in X \setminus B)           & \text{(by Definition \ref{3.1.27})} \\
        \iff & x \in (X \setminus A) \cap (X \setminus B).                 & \text{(by Definition \ref{3.1.23})}
    \end{align*}
    Thus by Definition \ref{3.1.4} we have \(X \setminus (A \cup B) = (X \setminus A) \cap (X \setminus B)\).
    Similarly we have
    \begin{align*}
             & \forall x : x \in X \setminus (A \cap B)                                                         \\
        \iff & x \in X \land x \notin (A \cap B)                          & \text{(by Definition \ref{3.1.27})} \\
        \iff & x \in X \land \lnot (x \in A \cap B)                                                             \\
        \iff & x \in X \land \lnot (x \in A \land x \in B)                & \text{(by Definition \ref{3.1.23})} \\
        \iff & x \in X \land (x \notin A \lor x \notin B)                                                       \\
        \iff & (x \in X \land x \notin A) \lor (x \in X \land x \notin B)                                       \\
        \iff & (x \in X \setminus A) \lor (x \in X \setminus B)           & \text{(by Definition \ref{3.1.27})} \\
        \iff & x \in (X \setminus A) \cup (X \setminus B).                & \text{(by Axiom \ref{3.4})}
    \end{align*}
    Thus by Definition \ref{3.1.4} we have \(X \setminus (A \cap B) = (X \setminus A) \cup (X \setminus B)\).
\end{proof}

\begin{remark}\label{3.1.29}
    The de Morgan laws are named after the logician Augustus De Morgan (1806 -- 1871), who identified them as one of the basic laws of set theory.
\end{remark}

\begin{remark}\label{3.1.30}
    The reader may observe a certain symmetry in the above laws between \(\cup\) and \(\cap\), and between \(X\) and \(\emptyset\).
    This is an example of \emph{duality} - two distinct properties or objects being dual to each other.
    In this case, the duality is manifested by the complementation relation \(A \mapsto X \setminus A\);
    the de Morgan laws assert that this relation converts unions into intersections and vice versa.
    (It also interchanges \(X\) and the empty set.)
    Proposition \ref{3.1.28} are collectively known as the \emph{laws of Boolean algebra}, after the mathematician George Boole (1815 -- 1864), and are also applicable to a number of other objects other than sets;
    it plays a particularly important role in logic.
\end{remark}

\begin{axiom}[Replacement]\label{3.6}
    Let \(A\) be a set.
    For any object \(x \in A\), and any object \(y\), suppose we have a statement \(P(x, y)\) pertaining to \(x\) and \(y\), such that for each \(x \in A\) there is at most one \(y\) for which \(P(x, y)\) is true.
    Then there exists a set \(\{y : P(x, y) \text{ is true for some } x \in A\}\), such that for any object \(z\),
    \[
        z \in \{y: P(x, y) \text{ is true for some } x \in A\} \iff P(x, y) \text{ is true for some } x \in A.
    \]
\end{axiom}

\begin{note}
    The keyword here is ``suppose'';
    We have to assume that there exists a set \(E = \{x \in A : \exists!\ y \text{ such that } P(x, y) \text{ is true}\}\).
    \(E\) must exist first so we can apply Axiom \ref{3.6}.
    This means we assert the existence of a function \(f : E \to \{y : P(x, y) \text{ is true for some } x\}\).
\end{note}

\begin{note}
    We often abbreviate a set of the form
    \[
        \{y : y = f(x) \text{ for some } x \in A\}
    \]
    as \(\{f(x) : x \in A\}\) or \(\{f(x) \mid x \in A\}\).
    We can of course combine the axiom of replacement with the axiom of specification, thus for instance we can create sets such as \(\{f(x) : x \in A; P(x) \text{ is true}\}\) by starting with the set \(A\), using the axiom of specification to create the set \(\{x \in A : P(x) \text{ is true}\}\), and then applying the axiom of replacement to create \(\{f(x) : x \in A; P(x) \text{ is true}\}\).
\end{note}

\begin{axiom}[Infinity]\label{3.7}
    There exists a set \(\N\), whose elements are called natural numbers, as well as an object \(0\) in \(\N\), and an object \(n++\) assigned to every natural number \(n \in \N\), such that the Peano axioms (Axioms \ref{2.1} - \ref{2.5}) hold.
\end{axiom}

\exercisesection

\begin{exercise}\label{ex 3.1.1}
    Show that the definition of equality in Definition \ref{3.1.4} is reflexive, symmetric, and transitive.
\end{exercise}

\begin{proof}
    See Additional Corollary \ref{ac 3.1.1}.
\end{proof}

\begin{exercise}\label{ex 3.1.2}
    Using only Definition \ref{3.1.4}, Axiom \ref{3.1}, Axiom \ref{3.2}, and Axiom \ref{3.3}, prove that the sets \(\emptyset\), \(\{\emptyset\}\), \(\{\{\emptyset\}\}\), and \(\{\emptyset, \{\emptyset\}\}\) are all distinct
    (i.e., no two of them are equal to each other).
\end{exercise}

\begin{proof}
    We first show that \(\emptyset \neq \{\emptyset\}\), \(\emptyset \neq \{\{\emptyset\}\}\) and \(\emptyset \neq \{\emptyset, \{\emptyset\}\}\).
    \begin{align*}
                 & \emptyset \in \{\emptyset\} \land \emptyset \notin \emptyset                & \text{(by Axiom \ref{3.2})}        \\
        \implies & \{\emptyset\} \neq \emptyset.                                               & \text{(by Definition \ref{3.1.4})} \\
                 & \{\emptyset\} \in \{\{\emptyset\}\} \land \{\emptyset\} \notin \emptyset    & \text{(by Axiom \ref{3.2})}        \\
        \implies & \{\{\emptyset\}\} \neq \emptyset.                                           & \text{(by Definition \ref{3.1.4})} \\
                 & \emptyset \in \{\emptyset, \{\emptyset\}\} \land \emptyset \notin \emptyset & \text{(by Axiom \ref{3.2})}        \\
        \implies & \{\emptyset\} \neq \emptyset.                                               & \text{(by Definition \ref{3.1.4})}
    \end{align*}

    Next we show that \(\{\emptyset\} \neq \{\{\emptyset\}\}\) and \(\{\emptyset\} \neq \{\emptyset, \{\emptyset\}\}\).
    \begin{align*}
                 & \{\emptyset\} \in \{\{\emptyset\}\} \land \{\emptyset\} \notin \{\emptyset\}            & \text{(by Axiom \ref{3.3})}        \\
        \implies & \{\emptyset\} \neq \{\{\emptyset\}\}.                                                   & \text{(by Definition \ref{3.1.4})} \\
                 & \{\emptyset\} \in \{\emptyset, \{\emptyset\}\} \land \{\emptyset\} \notin \{\emptyset\} & \text{(by Axiom \ref{3.2})}        \\
        \implies & \{\emptyset\} \neq \{\{\emptyset\}\}.                                                   & \text{(by Definition \ref{3.1.4})}
    \end{align*}

    Finally we show that \(\{\{\emptyset\}\} \neq \{\emptyset, \{\emptyset\}\}\).
    \begin{align*}
                 & \emptyset \in \{\emptyset, \{\emptyset\}\} \land \emptyset \notin \{\{\emptyset\}\} & \text{(by Axiom \ref{3.3})}        \\
        \implies & \{\{\emptyset\}\} \neq \{\emptyset, \{\emptyset\}\}.                                & \text{(by Definition \ref{3.1.4})}
    \end{align*}
\end{proof}

\begin{exercise}\label{ex 3.1.3}
    Prove the remaining claims in Lemma \ref{3.1.13}.
\end{exercise}

\begin{proof}
    See Lemma \ref{3.1.13}.
\end{proof}

\begin{exercise}\label{ex 3.1.4}
    Prove the remaining claims in Proposition \ref{3.1.18}.
\end{exercise}

\begin{proof}
    See Proposition \ref{3.1.18}.
\end{proof}

\begin{exercise}\label{ex 3.1.5}
    Let \(A\), \(B\) be sets.
    Show that the three statements \(A \subseteq B\), \(A \cup B = B\), \(A \cap B = A\) are logically equivalent (any one of them implies the other two).
\end{exercise}

\begin{proof}
    We first show that \(A \subseteq B \iff A \cup B = B\).
    Suppose that \(A, B\) are sets.
    Then we have
    \begin{align*}
        A \subseteq B \implies & A \cup B = B.                                   & \text{(by Proposition \ref{3.1.28}(b))} \\
        A \cup B = B \implies  & (\forall x : x \in A \cup B \iff x \in B)       & \text{(by Definition \ref{3.1.4})}      \\
        \implies               & (\forall x : x \in A \lor x \in B \iff x \in B) & \text{(by Axiom \ref{3.4})}             \\
        \implies               & (\forall x : x \in A \implies x \in B)                                                    \\
        \implies               & A \subseteq B.                                  & \text{(by Definition \ref{3.1.15})}     \\
        A \cup B = B \iff      & A \subseteq B.
    \end{align*}

    Now we show that \(A \subseteq B \iff A \cap B = A\).
    Suppose that \(A, B\) are sets.
    Then we have
    \begin{align*}
        A \subseteq B \implies & A \cap B = A.                                    & \text{(by Proposition \ref{3.1.28}(b))} \\
        A \cap B = A \implies  & (\forall x : x \in A \cap B \iff x \in A)        & \text{(by Definition \ref{3.1.4})}      \\
        \implies               & (\forall x : x \in A \land x \in B \iff x \in A) & \text{(by Definition \ref{3.1.23})}     \\
        \implies               & (\forall x : x \in A \implies x \in B)                                                     \\
        \implies               & A \subseteq B.                                   & \text{(by Definition \ref{3.1.15})}     \\
        A \cap B = A \iff      & A \subseteq B.
    \end{align*}
\end{proof}

\begin{exercise}\label{ex 3.1.6}
    Prove Proposition \ref{3.1.28}.
\end{exercise}

\begin{proof}
    See Proposition \ref{3.1.28}.
\end{proof}

\begin{exercise}\label{ex 3.1.7}
    Let \(A\), \(B\), \(C\) be sets.
    Show that \(A \cap B \subseteq A\) and \(A \cap B \subseteq B\).
    Furthermore, show that \(C \subseteq A\) and \(C \subseteq B\) if and only if \(C \subseteq A \cap B\).
    In a similar spirit, show that \(A \subseteq A \cup B\) and \(B \subseteq A \cup B\), and furthermore that \(A \subseteq C\) and \(B \subseteq C\) if and only if \(A \cup B \subseteq C\).
\end{exercise}

\begin{proof}
    We first show that \(A \cap B \subseteq A\) and \(A \cap B \subseteq B\).
    Suppose that \(A, B\) are sets.
    Then we have
    \begin{align*}
                 & (\forall x : x \in A \cap B \iff x \in A \land x \in B) & \text{(by Definition \ref{3.1.23})} \\
        \implies & ((\forall x : x \in A \cap B \implies x \in A)                                                \\
                 & \land (\forall x : x \in A \cap B \implies x \in B))                                          \\
        \implies & (A \cap B \subseteq A) \land (A \cap B \subseteq B).    & \text{(by Definition \ref{3.1.15})}
    \end{align*}

    Next we show that \(C \subseteq A \land C \subseteq B \iff C \subseteq A \cap B\).
    Suppose that \(A, B, C\) are sets.
    Then we have
    \begin{align*}
             & (C \subseteq A \land C \subseteq B)                                                                                       \\
        \iff & (\forall x : x \in C \implies x \in A) \land (\forall x : x \in C \implies x \in B) & \text{(by Definition \ref{3.1.15})} \\
        \iff & (\forall x : x \in C \implies x \in A \land x \in B)                                                                      \\
        \iff & (\forall x : x \in C \implies x \in A \cap B)                                       & \text{(by Definition \ref{3.1.23})} \\
        \iff & (C \subseteq A \cap B).                                                             & \text{(by Definition \ref{3.1.15})}
    \end{align*}

    Next we show that \(A \subseteq A \cup B\) and \(B \subseteq A \cup B\).
    Suppose that \(A, B\) are sets.
    Then we have
    \begin{align*}
        \top \iff & (\forall x : x \in A \implies \forall x : x \in A \lor x \in B)                                       \\
        \iff      & (\forall x : x \in A \implies \forall x : x \in A \cup B)       & \text{(by Axiom \ref{3.4})}         \\
        \iff      & (A \subseteq A \cup B).                                         & \text{(by Definition \ref{3.1.15})} \\
        \top \iff & (\forall x : x \in B \implies \forall x : x \in A \lor x \in B)                                       \\
        \iff      & (\forall x : x \in B \implies \forall x : x \in A \cup B)       & \text{(by Axiom \ref{3.4})}         \\
        \iff      & (B \subseteq A \cup B).                                         & \text{(by Definition \ref{3.1.15})}
    \end{align*}

    Finally we show that \(A \subseteq C \land B \subseteq C \iff A \cup B \subseteq C\).
    Suppose that \(A, B, C\) are sets.
    Then we have
    \begin{align*}
             & (A \subseteq C \land B \subseteq C)                                                                                       \\
        \iff & (\forall x : x \in A \implies x \in C) \land (\forall x : x \in B \implies x \in C) & \text{(by Definition \ref{3.1.15})} \\
        \iff & (\forall x : x \in A \lor x \in B \implies x \in C)                                                                       \\
        \iff & (\forall x : x \in A \cup B \implies x \in C)                                       & \text{(by Axiom \ref{3.4})}         \\
        \iff & (A \cup B \subseteq C).                                                             & \text{(by Definition \ref{3.1.15})}
    \end{align*}
\end{proof}

\begin{exercise}\label{ex 3.1.8}
    Let \(A\), \(B\) be sets.
    Prove the \emph{absorption laws} \(A \cap (A \cup B) = A\) and \(A \cup (A \cap B) = A\).
\end{exercise}

\begin{proof}
    Suppose that \(A, B\) are sets.
    Then we have
    \begin{align*}
             & \forall x : x \in A \cap (A \cup B)                                        \\
        \iff & x \in A \land x \in A \cup B         & \text{(by Definition \ref{3.1.23})} \\
        \iff & x \in A \land (x \in A \lor x \in B) & \text{(by Axiom \ref{3.4})}         \\
        \iff & x \in A.
    \end{align*}
    Thus by Definition \ref{3.1.4} we have \(A \cap (A \cup B) = A\).
    Similarly we have
    \begin{align*}
             & \forall x : x \in A \cup (A \cap B)                                        \\
        \iff & x \in A \lor x \in A \cap B          & \text{(by Axiom \ref{3.4})}         \\
        \iff & x \in A \lor (x \in A \land x \in B) & \text{(by Definition \ref{3.1.23})} \\
        \iff & x \in A.
    \end{align*}
    Thus by Definition \ref{3.1.4} we have \(A \cup (A \cap B) = A\).
\end{proof}

\begin{exercise}\label{ex 3.1.9}
    Let \(A\), \(B\), \(X\) be sets such that \(A \cup B = X\) and \(A \cap B = \emptyset\).
    Show that \(A = X \setminus B\) and \(B = X \setminus A\).
\end{exercise}

\begin{proof}
    Suppose that \(A, B, X\) are sets such that \(A \cup B = X\) and \(A \cap B = \emptyset\).
    We first show that \(A \cap B = \emptyset \iff \forall x : (x \in A \implies x \notin B) \land (x \in B \implies x \notin A)\).
    \begin{align*}
             & (A \cap B = \emptyset)                                                                                                \\
        \iff & (\forall x : x \notin A \cap B)                                                 & \text{(by Axiom \ref{3.2})}         \\
        \iff & (\forall x : \lnot (x \in A \cap B))                                                                                  \\
        \iff & (\forall x : \lnot (x \in A \land x \in B))                                     & \text{(by Definition \ref{3.1.23})} \\
        \iff & (\forall x : x \notin A \lor x \notin B)                                                                              \\
        \iff & (\forall x : (x \in A \implies x \notin B) \land (x \in B \implies x \notin A))                                       \\
    \end{align*}

    Now we show that \(A = X \setminus B\).
    From the proof above we have
    \begin{align*}
                 & (A \cup B = X) \land (A \cap B = \emptyset)                                                               \\
        \implies & (A \subseteq X) \land (A \cap B = \emptyset)                        & \text{(by Exercise \ref{ex 3.1.7})} \\
        \implies & (\forall x : x \in A \implies x \in X) \land (A \cap B = \emptyset) & \text{(by Definition \ref{3.1.15})} \\
        \implies & (\forall x : x \in A \implies x \in X)                                                                    \\
                 & \land (\forall x : x \in A \implies x \notin B)                                                           \\
        \implies & (\forall x : x \in A \implies x \in X \land x \notin B)                                                   \\
        \implies & (\forall x : x \in A \implies x \in X \setminus B).                 & \text{(by Definition \ref{3.1.27})}
    \end{align*}
    And we also have
    \begin{align*}
                 & \forall x : x \in X \setminus B                                               \\
        \implies & x \in X \land x \notin B                & \text{(by Definition \ref{3.1.27})} \\
        \implies & x \in A \cup B \land x \notin B                                               \\
        \implies & (x \in A \lor x \in B) \land x \notin B & \text{(by Axiom \ref{3.4})}         \\
        \implies & x \in A \land x \notin B                                                      \\
        \implies & x \in A.
    \end{align*}
    Thus by Definition \ref{3.1.4} we have \(A = X \setminus B\).
    Similarly argument show that \(B = X \setminus A\), as desired.
\end{proof}

\begin{exercise}\label{ex 3.1.10}
    Let \(A\) and \(B\) be sets.
    Show that the three sets \(A \setminus B\), \(A \cap B\), and \(B \setminus A\) are disjoint, and that their union is \(A \cup B\).
\end{exercise}

\begin{proof}
    Suppose that \(A, B\) are sets.
    Then we have
    \begin{align*}
                 & \forall x : x \in A \setminus B                                       \\
        \implies & x \in A \land x \notin B        & \text{(by Definition \ref{3.1.27})} \\
        \implies & x \notin A \cap B.              & \text{(by Definition \ref{3.1.23})}
    \end{align*}
    And we also have
    \begin{align*}
                 & \forall x : x \in A \cap B                                       \\
        \implies & x \in A \land x \in B      & \text{(by Definition \ref{3.1.23})} \\
        \implies & x \notin A \setminus B.    & \text{(by Definition \ref{3.1.27})}
    \end{align*}
    Thus \(A \setminus B\) and \(A \cap B\) are disjoint.
    Similarly argument show that \(B \setminus A\) and \(A \cap B\) are disjoint.
    For \(A \setminus B\) and \(B \setminus A\), we have
    \begin{align*}
                 & \forall x : x \in A \setminus B                                       \\
        \implies & x \in A \land x \notin B        & \text{(by Definition \ref{3.1.27})} \\
        \implies & x \notin B \setminus A.         & \text{(by Definition \ref{3.1.27})}
    \end{align*}
    Similarly argument show that \(\forall x : x \in B \setminus A \implies x \notin A \setminus B\).
    Thus \(A \setminus B\) and \(B \setminus A\) are disjoint.

    Now we show that \((A \setminus B) \cup (A \cap B) \cup (B \setminus A) = A \cup B\).
    \begin{align*}
             & \forall x : x \in (A \setminus B) \cup (A \cap B) \cup (B \setminus A)                                                        \\
        \iff & x \in (A \setminus B) \lor x \in (A \cap B) \lor x \in (B \setminus A)                  & \text{(by Axiom \ref{3.4})}         \\
        \iff & (x \in A \land x \notin B) \lor (x \in A \cap B) \lor (x \in B \land x \notin A)        & \text{(by Definition \ref{3.1.27})} \\
        \iff & (x \in A \land x \notin B) \lor (x \in A \land x \in B) \lor (x \in B \land x \notin A) & \text{(by Definition \ref{3.1.23})} \\
        \iff & ((x \in A \lor (x \in A \land x \in B))                                                                                       \\
             & \land (x \notin B \lor (x \in A \land x \in B)))                                                                              \\
             & \lor (x \in B \land x \notin A)                                                                                               \\
        \iff & ((x \in A) \land (x \notin B \lor x \in A)) \lor (x \in B \land x \notin A)                                                   \\
        \iff & (x \in A) \lor (x \in B \land x \notin A)                                                                                     \\
        \iff & x \in A \lor x \in B                                                                                                          \\
        \iff & x \in A \cup B.                                                                         & \text{(by Axiom \ref{3.4})}
    \end{align*}
\end{proof}

\begin{exercise}\label{ex 3.1.11}
    Show that the axiom of replacement implies the axiom of specification.
\end{exercise}

\begin{proof}
    By Axiom \ref{3.6}, \(z \in \{y : P(x, y) \text{ is true for some } x \in A\} \iff P(x, z)\) is true for some \(x \in A\).
    Change all \(y\) and \(z\) into \(x\), and replace \(P(x, x)\) with \(P(x)\), we derive \(x \in \{x : P(x) \text{ is true for some } x \in A\} \iff P(x)\) is true for some \(x \in A\), which is the same as Axiom \ref{3.5}.
    Thus we conclude that Axiom \ref{3.6} implies Axiom \ref{3.5}.
\end{proof}
\section{Russell's paradox}\label{sec 3.2}

\begin{axiom}[Universal specification]\label{3.8}
(Dangerous!)
Suppose for every object \(x\) we have a property \(P(x)\) pertaining to \(x\) (so that for every \(x\), \(P(x)\) is either a true statement or a false statement).
Then there exists a set \(\{x : P(x) \text{ is true}\}\) such that for every object \(y\),
\[
    y \in \{x : P(x) \text{ is true}\} \iff P(y) \text{ is true}.
\]
\end{axiom}

\begin{note}
Compare to Axiom \ref{3.5}, any object \(x\) does not need to be in a set \(A\) to apply this axiom.
\end{note}

\begin{note}
Axiom \ref{3.8} is also known as the \emph{axiom of comprehension}.
Unfortunately, this axiom cannot be introduced into set theory, because it creates a logical contradiction known as \emph{Russell’s paradox}, discovered by the philosopher and logician Bertrand Russell (1872--1970) in 1901.
The paradox runs as follows.
Let \(P(x)\) be the statement
\[
    P(x) \iff \text{``\(x\) is a set, and \(x \notin x\)''};
\]
i.e., \(P(x)\) is true only when \(x\) is a set which does not contain itself.
Now use the axiom of universal specification to create the set
\[
    \Omega \coloneqq \{x : P(x) \text{ is true}\} = \{x : x \text{ is a set and } x \notin x\},
\]
i.e., the set of all sets which do not contain themselves.
Now ask the question: does \(\Omega\) contain itself, i.e. is \(\Omega \in \Omega\)?
If \(\Omega\) did contain itself, then by definition this means that \(P(\Omega)\) is true, i.e., \(\Omega\) is a set and \(\Omega \notin \Omega\).
On the other hand, if \(\Omega\) did not contain itself, then \(P(\Omega)\) would be true, and hence \(\Omega \in \Omega\).
Thus in either case we have both \(\Omega \in \Omega\) and \(\Omega \notin \Omega\), which is absurd.
\end{note}

\begin{note}
The problem with Axiom \ref{3.8} is that it creates sets which are far too ``large''.
Since sets are themselves objects (Axiom \ref{3.1}), this means that sets are allowed to contain themselves, which is a somewhat silly state of affairs.
One way to informally resolve this issue is to think of objects as being arranged in a hierarchy.
At the bottom of the hierarchy are the \emph{primitive objects} - the objects that are not sets.
Then on the next rung of the hierarchy there are sets whose elements consist only of primitive objects, let’s call these ``primitive sets'' for now.
Then there are sets whose elements consist only of primitive objects and primitive sets, and we can form sets out of these objects, and so forth.
The point is that at each stage of the hierarchy we only see sets whose elements consist of objects at lower stages of the hierarchy, and so at no stage do we ever construct a set which contains itself.
\end{note}

\begin{axiom}[Regularity]\label{3.9}
If \(A\) is a non-empty set, then there is at least one element \(x\) of \(A\) which is either not a set, or is disjoint from \(A\).
\end{axiom}

\begin{note}
The point of this axiom (which is also known as the \emph{axiom of foundation}) is that it is asserting that at least one of the elements of \(A\) is so low on the hierarchy of objects that it does not contain any of the other elements of \(A\).
\end{note}

\exercisesection

\begin{exercise}\label{ex 3.2.1}
Show that the universal specification axiom, Axiom 3.8, if assumed to be true, would imply Axioms \ref{3.2}, \ref{3.3}, \ref{3.4}, \ref{3.5}, and \ref{3.6}.
(If we assume that all natural numbers are objects, we also obtain Axiom \ref{3.7}.)
Thus, this axiom, if permitted, would simplify the foundations of set theory tremendously (and can be viewed as one basis for an intuitive model of set theory known as ``naive set theory'').
Unfortunately, as we have seen, Axiom \ref{3.8} is ``too good to be true''!
\end{exercise}

\begin{proof}
We first prove Axiom \ref{3.2}.
By Axiom \ref{3.8}, there exist a set \(\emptyset\) such that \(\{x: x \notin \emptyset\}\).

Next we prove Axiom \ref{3.3}.
For singleton sets, if \(a\) is an object, then by Axiom \ref{3.8} there exist a set \(\{x: x = a\}\).
For pair sets, if \(a\) and \(b\) are objects, then by Axiom \ref{3.8} there exist a set \(\{x: x = a \text{ or } x = b\}\).

Next we prove Axiom \ref{3.4}.
By Axiom \ref{3.8}, there exist sets \(A\), \(B\) and \(\{x : x \in A \text{ or } x \in B\}\).

Next we prove Axiom \ref{3.5}.
By Axiom \ref{3.8}, there exist sets \(A\) and \(\{x \in A : P(x) \text{ is true}\}\).

Next we prove Axiom \ref{3.6}.
By Axiom \ref{3.8}, there exist sets \(A\) and \(\{y : P(x, y), x \in A\}\).

Finally, we prove Axiom \ref{3.7}.
By Axiom \ref{3.8}, there exists a set \(\mathbf{N}\), as well as an object \(0\) in \(\mathbf{N}\), and an object \(n++\) assigned to every object \(n \in \mathbf{N}\), such that the Peano axioms holds.
\end{proof}

\begin{exercise}\label{ex 3.2.2}
Use the axiom of regularity (and the singleton set axiom) to show that if \(A\) is a set, then \(A \notin A\).
Furthermore, show that if \(A\) and \(B\) are two sets, then either \(A \notin B\) or \(B \notin A\) (or both).
\end{exercise}

\begin{proof}
Suppose for sake of contradiction that there exist a set \(A\) such that \(A \in A\) is true.
By Axiom \ref{3.3}, there exist a set \(\{A\}\) and \(A \in \{A\}\) is true.
Then \(A \in A \cap \{A\}\) is true, but by Axiom \ref{3.9}, the only element \(A\) in \(\{A\}\) must be disjoint from \(\{A\}\), which mean \(A \cap \{A\} = \emptyset\), a contradiction.
Thus there does not exist a \(A\) such that \(A \in A\) is true, i.e., \(\forall\ A\) is a set, \(A \notin A\).

Next we show that if \(A\) and \(B\) are two sets, then either \(A \notin B\) or \(B \notin A\) (or both).
If \(A \in B\), we want to show that \(B \notin A\).
So suppose for sake of contradiction that \(B \in A\).
Then \(A \in A \cup B\) and \(B \in A \cup B\), which means \(A \cup B \in A \cup B\), contradict to Axiom \ref{3.9}.
Thus \(B \notin A\).
Similar argument show that if \(B \in A\), then \(A \notin B\).
And if \(A \notin B\), then \(B \notin A\) can also be true.
So we conclude that either \(A \notin B\) or \(B \notin A\) (or both).
\end{proof}

\begin{exercise}\label{ex 3.2.3}
Show (assuming the other axioms of set theory) that the universal specification axiom, Axiom \ref{3.8}, is equivalent to an axiom postulating the existence of a ``universal set'' \(\Omega\) consisting of all objects (i.e., for all objects \(x\), we have \(x \in \Omega\)).
In other words, if Axiom \ref{3.8} is true, then a universal set exists, and conversely, if a universal set exists, then Axiom \ref{3.8} is true.
(This may explain why Axiom \ref{3.8} is called the axiom of universal specification.)
Note that if a universal set \(\Omega\) existed, then we would have \(\Omega \in \Omega\) by Axiom \ref{3.1}, contradicting Exercise \ref{ex 3.2.2}.
Thus the axiom of foundation specifically rules out the axiom of universal specification.
\end{exercise}

\begin{proof}
If Axiom \ref{3.8} is true, then there exist a set \(\Omega = \{x: x \text{ is a object}\}\), and \(\Omega \in \Omega\).
Thus Axiom \ref{3.8} implies a universal set exist.
If a universal set \(\Omega\) exist, then for any set \(A = \{x: P(x)\}\), \(A \in \Omega\) is true.
Thus a universal set exist implies Axiom \ref{3.8} is true.
Since we prove both necessary and sufficient condition, we conclude that Axiom \ref{3.8} is logically equivalent to a universal set exist.
\end{proof}
\section{Functions}\label{sec 3.3}

\begin{definition}[Functions]\label{3.3.1}
Let \(X\), \(Y\) be sets, and let \(P(x, y)\) be a property pertaining to an object \(x \in X\) and an object \(y \in Y\), such that for every \(x \in X\), there is exactly one \(y \in Y\) for which \(P(x, y)\) is true (this is sometimes known as the \emph{vertical line test}).
Then we define the \emph{function} \(f : X \to Y\) \emph{defined by} \(P\) \emph{on the domain} \(X\) \emph{and range} \(Y\) to be the object which, given any input \(x \in X\), assigns an output \(f(x) \in Y\), defined to be the unique object \(f(x)\) for which \(P(x, f(x))\) is true.
Thus, for any \(x \in X\) and \(y \in Y\),
\[
    y = f(x) \iff P(x, y) \text{ is true}.
\]
\end{definition}

\begin{note}
Functions are also referred to as \emph{maps} or \emph{transformations}, depending on the context.
They are also sometimes called \emph{morphisms}, although to be more precise, a morphism refers to a more general class of object, which may or may not correspond to actual functions, depending on the context.
\end{note}

\begin{note}
One common way to define a function is simply to specify its domain, its range, and how one generates the output \(f(x)\) from each input;
this is known as an \emph{explicit} definition of a function.
In other cases we only define a function \(f\) by specifying what property \(P(x, y)\) links the input \(x\) with the output \(f(x)\);
this is an \emph{implicit} definition of a function.
\end{note}

\begin{note}
In many cases we omit specifying the domain and range of a function for brevity.
However, too much of this abbreviation can be dangerous;
sometimes it is important to know what the domain and range of the function is.
\end{note}

\begin{note}
We observe that functions obey the axiom of substitution: if \(x = x'\), then \(f(x) = f(x')\).
In other words, equal inputs imply equal outputs.
On the other hand, unequal inputs do not necessarily ensure unequal outputs.
For example, \emph{constant function} simply assign each input with the same output.
\end{note}

\setcounter{theorem}{4}
\begin{remark}\label{3.3.5}
We are now using parentheses () to denote several different things in mathematics;
on one hand, we are using them to clarify the order of operations, but on the other hand we also use parentheses to enclose the argument of a function \(f(x)\) or of a property such as \(P(x)\).
However, the two usages of parentheses usually are unambiguous from context.
For instance, if \(a\) is a number, then \(a(b + c)\) denotes the expression \(a \times (b + c)\), whereas if \(f\) is a function, then \(f(b + c)\) denotes the output of \(f\) when the input is \(b + c\).
Sometimes the argument of a function is denoted by subscripting instead of parentheses;
for instance, a sequence of natural numbers \(a_0\), \(a_1\), \(a_2\), \(a_3\), \(\cdots\) is, strictly speaking, a function from \(\mathbf{N}\) to \(\mathbf{N}\), but is denoted by \(n \mapsto a_n\) rather than \(n \mapsto a(n)\).
\end{remark}

\begin{remark}\label{3.3.6}
Strictly speaking, functions are not necessarily sets, and sets are not necessarily functions;
it does not make sense to ask whether an object \(x\) is an element of a function \(f\), and it does not make sense to apply a set \(A\) to an input \(x\) to create an output \(A(x)\).
On the other hand, it is possible to start with a function \(f : X \to Y\) and construct its graph \(\{(x, f(x)) : x \in X\}\), which describes the function completely once the domain \(X\) and range \(Y\) are specified.
\end{remark}

\begin{definition}[Equality of functions]\label{3.3.7}
Two functions \(f : X \to Y\), \(g : X \to Y\) with the same domain and range are said to be equal, \(f = g\), if and only if \(f(x) = g(x)\) for all \(x \in X\).
(If \(f(x)\) and \(g(x)\) agree for some values of \(x\), but not others, then we do not consider \(f\) and \(g\) to be equal.)
Two functions \(f : X \to Y\) and \(g : X' \to Y'\) are considered to be unequal if they have different domains \(X \neq X'\) or different ranges \(Y \neq Y'\) (or both)
\end{definition}

\begin{note}
A rather boring example of a function is the \emph{empty function} \(f : \emptyset \to X\) from the empty set to an arbitrary set \(X\).
Since the empty set has no elements, we do not need to specify what \(f\) does to any input.
Nevertheless, just as the empty set is a set, the empty function is a function, albeit not a particularly interesting one.
Note that for each set \(X\), there is only one function from \(\emptyset\) to \(X\), since Definition \ref{3.3.7} asserts that all functions from \(\emptyset\) to \(X\) are equal.
\end{note}

\setcounter{theorem}{9}
\begin{definition}[Composition]\label{3.3.10}
Let \(f : X \to Y\) and \(g : Y \to Z\) be two functions, such that the range of \(f\) is the same set as the domain of \(g\).
We then define the composition \(g \circ f : X \to Z\) of the two functions \(g\) and \(f\) to be the function defined explicitly by the formula
\[
    (g \circ f)(x) \coloneqq g(f(x)).
\]
If the range of \(f\) does not match the domain of \(g\), we leave the composition \(g \circ f\) undefined.
\end{definition}

\begin{note}
Composition is not commutative: \(f \circ g\) and \(g \circ f\) are not necessarily the same function.
\end{note}

\setcounter{theorem}{11}
\begin{lemma}[Composition is associative]\label{3.3.12}
Let \(f : Z \to W\), \(g : Y \to Z\), and \(h : X \to Y\) be functions.
Then \(f \circ (g \circ h) = (f \circ g) \circ h\).
\end{lemma}

\begin{proof}
Since \(g \circ h\) is a function from \(X\) to \(Z\), \(f \circ (g \circ h)\) is a function from \(X\) to \(W\).
Similarly \(f \circ g\) is a function from \(Y\) to \(W\), and hence \((f \circ g) \circ h\) is a function from \(X\) to \(W\).
Thus \(f \circ (g \circ h)\) and \((f \circ g) \circ h\) have the same domain and range.
In order to check that they are equal, we see from Definition \ref{3.3.7} that we have to verify that \((f \circ (g \circ h))(x) = ((f \circ g) \circ h)(x)\) for all \(x \in X\).
But by Definition \ref{3.3.10}
\begin{align*}
(f \circ (g \circ h))(x)
&= f((g \circ h)(x)) \\
&= f(g(h(x))) \\
&= (f \circ g)(h(x)) \\
&= ((f \circ g) \circ h)(x)
\end{align*}
as desired.
\end{proof}

\setcounter{theorem}{13}
\begin{definition}[One-to-one function]\label{3.3.14}
A function \(f\) is \emph{one-to-one} (or \emph{injective}) if different elements map to different elements:
\[
    x \neq x' \implies f(x) \neq f(x').
\]

Equivalently, a function is one-to-one if
\[
    f(x) = f(x') \implies x = x'.
\]
\end{definition}

\setcounter{theorem}{15}
\begin{remark}\label{3.3.16}
If a function \(f : X \to Y\) is not one-to-one, then one can find distinct \(x\) and \(x'\) in the domain \(X\) such that \(f(x) = f(x')\), thus one can find two inputs which map to one output.
Because of this, we say that \(f\) is \emph{two-to-one} instead of \emph{one-to-one}.
\end{remark}

\begin{definition}[Onto functions]\label{3.3.17}
A function \(f\) is \emph{onto} (or \emph{surjective}) if \(f(X) = Y\), i.e., every element in \(Y\) comes from applying \(f\) to some element in \(X\):
\[
    \text{For every } y \in Y, \text{there exists } x \in X \text{ such that } f(x) = y.
\]
\end{definition}

\setcounter{theorem}{18}
\begin{remark}\label{3.3.19}
The concepts of injectivity and surjectivity are in many ways dual to each other.
\end{remark}

\begin{definition}[Bijective functions]\label{3.3.20}
Functions \(f : X \to Y\) which are both one-to-one and onto are also called \emph{bijective} or \emph{invertible}.
\end{definition}

\setcounter{theorem}{22}
\begin{remark}\label{3.3.23}
If a function \(x \mapsto f(x)\) is bijective, then we sometimes call \(f\) a \emph{perfect matching} or a \emph{one-to-one correspondence} (not to be confused with the notion of a one-to-one function), and denote the action of \(f\) using the notation \(x \leftrightarrow f(x)\) instead of \(x \mapsto f(x)\).
\end{remark}

\begin{note}
If \(f\) is bijective, then for every \(y \in Y\), there is exactly one \(x\) such that \(f(x) = y\) (there is at least one because of surjectivity, and at most one because of injectivity).
This value of \(x\) is denoted \(f^{-1}(y)\); thus \(f^{-1}\) is a function from \(Y\) to \(X\).
We call \(f^{-1}\) the \emph{inverse} of \(f\).
\end{note}

\exercisesection

\begin{exercise}\label{ex 3.3.1}
Show that the definition of equality in Definition \ref{3.3.7} is reflexive, symmetric, and transitive.
Also verify the substitution property: if \(f, \tilde{f} : X \to Y\) and \(g, \tilde{g} : Y \to Z\) are functions such that \(f = \tilde{f}\) and \(g = \tilde{g}\), then \(g \circ f = \tilde{g} \circ \tilde{f}\).
Of course, these statements are immediate from the axioms of equality in Appendix A.7 applied directly to the functions in question, but the point of the exercise is to show that they can also be established by instead applying the axioms of equality to elements of the domain and range of these functions, rather than to the functions itself.
\end{exercise}

\begin{proof}
We first show that Definition \ref{3.3.7} is reflexive.
Suppose that \(X, Y\) are sets and \(f : X \to Y\) is a function.
Then we have
\begin{align*}
& (X = X) \land (Y = Y) & \text{(by Additional Corollary \ref{ac 3.1.1})} \\
& \land (\forall\ x \in X : f(x) = f(x)) & \text{(by Definition \ref{3.3.1})} \\
\implies & (f = f). & \text{(by Definition \ref{3.3.7})}
\end{align*}
Thus Definition \ref{3.3.7} is reflexive.

Next we show that Definition \ref{3.3.7} is symmetric.
Suppose that \(X, Y\) are sets and \(f : X \to Y, g : X \to Y\) are functions such that \(f = g\).
Then we have
\begin{align*}
& f = g \\
\iff & (X = X) \land (Y = Y) \land (\forall\ x \in X : f(x) = g(x)) & \text{(by Definition \ref{3.3.7})} \\
\iff & (X = X) \land (Y = Y) \land (\forall\ x \in X : g(x) = f(x)) \\
\iff & g = f. & \text{(by Definition \ref{3.3.7})}
\end{align*}
Thus Definition \ref{3.3.7} is symmetric.

Next we show that Definition \ref{3.3.7} is transitive.
Suppose that \(X, Y\) are sets and \(f : X \to Y, g : X \to Y, h : X \to Y\) are functions such that \(f = g \land g = h\).
Then we have
\begin{align*}
& (f = g) \land (g = h) \\
\implies & (X = X) \land (Y = Y) \\
& \land (\forall\ x \in X : f(x) = g(x)) & \text{(by Definition \ref{3.3.7})} \\
& \land (\forall\ x \in X : g(x) = h(x)) & \text{(by Definition \ref{3.3.7})} \\
\implies & (X = X) \land (Y = Y) \land (\forall\ x \in X : f(x) = h(x)) \\
\implies & f = h. & \text{(by Definition \ref{3.3.7})}
\end{align*}
Thus Definition \ref{3.3.7} is transitive.

Now we show that Axiom of substitution holds for composition.
Suppose that \(X, Y, Z\) are sets and \(f : X \to Y, \tilde{f} : X \to Y, g : Y \to Z, \tilde{g} : Y \to Z\) are functions such that \(f = \tilde(f) \land g = \tilde{g}\).
By Definition \ref{3.3.10} we have \(g \circ f : X \to Z\) and \(\tilde{g} \circ \tilde{f} : X \to Z\).
Then we have
\begin{align*}
& \forall\ x \in X : (g \circ f)(x) = g(f(x)) & \text{(by Definition \ref{3.3.10})} \\
\iff & \forall\ x \in X : (g \circ f)(x) = g(\tilde{f}(x)) & \text{(by Definition \ref{3.3.7})} \\
\iff & \forall\ x \in X : (g \circ f)(x) = \tilde{g}(\tilde{f}(x)) & \text{(by Definition \ref{3.3.7})} \\
\iff & \forall\ x \in X : (g \circ f)(x) = (\tilde{g} \circ \tilde{f})(x) & \text{(by Definition \ref{3.3.10})} \\
\iff & g \circ f = \tilde{g} \circ \tilde{f}. & \text{(by Definition \ref{3.3.7})}
\end{align*}
\end{proof}

\begin{exercise}\label{ex 3.3.2}
Let \(f : X \to Y\) and \(g : Y \to Z\) be functions.
Show that if \(f\) and \(g\) are both injective, then so is \(g \circ f\);
similarly, show that if \(f\) and \(g\) are both surjective, then so is \(g \circ f\).
\end{exercise}

\begin{proof}
We first show that \(f, g\) are injective implies \(g \circ f\) is injective.
Suppose that \(X, Y, Z\) are sets and \(f : X \to Y, g : Y \to Z\) are functions such that \(f, g\) are injective.
Then we have
\begin{align*}
& \forall\ x, x' \in X : (g \circ f)(x) = (g \circ f)(x') \\
\implies & g(f(x)) = g(f(x')) & \text{(by Definition \ref{3.3.10})} \\
\implies & f(x) = f(x') & \text{(by Definition \ref{3.3.14})} \\
\implies & x = x'. & \text{(by Definition \ref{3.3.14})}
\end{align*}
Since \((g \circ f)(x) = (g \circ f)(x') \implies x = x'\), by Definition \ref{3.3.14} \(g \circ f\) is injective.

Now we show that \(f, g\) are surjective implies \(g \circ f\) is surjective.
Suppose that \(X, Y, Z\) are sets and \(f : X \to Y, g : Y \to Z\) are functions such that \(f, g\) are surjective.
Then we have
\begin{align*}
& (\forall\ z \in Z, \exists\ y \in Y : z = g(y)) & \text{(by Definition \ref{3.3.17})} \\
& \land (\forall\ y \in Y, \exists\ x \in X : y = f(x)) & \text{(by Definition \ref{3.3.17})} \\
\implies & \forall\ z \in Z, \exists\ x \in X : z = g(f(x)) \\
\implies & \forall\ z \in Z, \exists\ x \in X : z = (g \circ f)(x). & \text{(by Definition \ref{3.3.10})}
\end{align*}
Since \(\forall\ z \in Z, \exists\ x \in X : z = (g \circ f)(x)\), by Definition \ref{3.3.17} \(g \circ f\) is surjective.
\end{proof}

\begin{exercise}\label{ex 3.3.3}
When is the empty function into a given set \(X\) injective?
surjective?
bijective?
\end{exercise}

\begin{proof}
Suppose that \(X\) is a set and \(f : \emptyset \to X\) is the empty function.
\(f\) is always injective since \(\forall\ x, x' \in \emptyset\), \(f(x) = f(x') \implies x = x'\) (which is vacuously true).
\(f\) is never surjective, since \(\forall\ y \in X\), \(\nexists\ x \in \emptyset\) such that \(f(x) = y\).
And because \(f\) is never surjective, \(f\) is never bijective.
\end{proof}

\begin{exercise}\label{ex 3.3.4}
In this section we give some cancellation laws for composition.
Let \(f : X \to Y\), \(\tilde{f} : X \to Y\), \(g : Y \to Z\), and \(\tilde{g} : Y \to Z\) be functions.
Show that if \(g \circ f = g \circ \tilde{f}\) and g is injective, then \(f = \tilde{f}\).
Is the same statement true if \(g\) is not injective?
Show that if \(g \circ f = \tilde{g} \circ f\) and \(f\) is surjective, then \(g = \tilde{g}\).
Is the same statement true if \(f\) is not surjective?
\end{exercise}

\begin{proof}
We first show that \(g\) is injective and \(g \circ f = g \circ \tilde{f}\) implies \(f = \tilde{f}\).
Suppose that \(X, Y, Z\) are sets and \(f : X \to Y, \tilde{f} : X \to Y, g : Y \to Z\) are functions such that \(g\) is injective and \(g \circ f = g \circ \tilde{f}\).
Then we have
\begin{align*}
& g \circ f = g \circ \tilde{f} \\
\implies & \forall\ x \in X : (g \circ f)(x) = (g \circ \tilde{f})(x) & \text{(by Definition \ref{3.3.7})} \\
\implies & g(f(x)) = g(\tilde{f}(x)) & \text{(by Definition \ref{3.3.10})} \\
\implies & f(x) = \tilde{f}(x) & \text{(by Definition \ref{3.3.14})} \\
\implies & f = \tilde{f}. & \text{(by Definition \ref{3.3.7})}
\end{align*}
The statement is not true when \(g\) is not injective.

Now we show that \(f\) is surjective and \(g \circ f = \tilde{g} \circ f\) implies \(g = \tilde{g}\).
Suppose that \(X, Y, Z\) are sets and \(f : X \to Y, g : Y \to Z, \tilde{g} : Y \to Z\) are functions such that \(f\) is surjective and \(g \circ f = \tilde{g} \circ f\).
Then we have
\begin{align*}
& \forall\ y \in Y, \exists\ x \in X : y = f(x) & \text{(by Definition \ref{3.3.17})} \\
\implies & g(y) = g(f(x)) & \text{(by Definition \ref{3.3.1})} \\
\implies & g(y) = (g \circ f)(x) & \text{(by Definition \ref{3.3.10})} \\
\implies & g(y) = (\tilde{g} \circ f)(x) & \text{(by Definition \ref{3.3.7})} \\
\implies & g(y) = \tilde{g}(f(x)) = \tilde{g}(y) & \text{(by Definition \ref{3.3.10})} \\
\implies & g = \tilde{g}. & \text{(by Definition \ref{3.3.7})}
\end{align*}
The statement is not true when \(f\) is not surjective.
\end{proof}

\begin{exercise}\label{ex 3.3.5}
Let \(f : X \to Y\) and \(g : Y \to Z\) be functions.
Show that if \(g \circ f\) is injective, then \(f\) must be injective.
Is it true that \(g\) must also be injective?
Show that if \(g \circ f\) is surjective, then \(g\) must be surjective.
Is it true that \(f\) must also be surjective?
\end{exercise}

\begin{proof}
We first show that \(g \circ f\) is injective implies \(f\) is injective.
Suppose \(X, Y, Z\) are sets and \(f : X \to Y, g : Y \to Z\) are functions such that \(g \circ f\) is injective.
Then we have
\begin{align*}
& \forall\ x, x' \in X : f(x) = f(x') \\
\implies & g(f(x)) = g(f(x')) & \text{(by Definition \ref{3.3.1})} \\
\implies & (g \circ f)(x) = (g \circ f)(x') & \text{(by Definition \ref{3.3.10})} \\
\implies & x = x'. & \text{(by Definition \ref{3.3.14})}
\end{align*}
Since \(\forall\ x, x' \in X : f(x) = f(x') \implies x = x'\), by Definition \ref{3.3.14} \(f\) is injective.
And we don't need \(g\) to be injective.

Now we show that \(g \circ f\) is surjective implies \(g\) is surjective.
Suppose \(X, Y, Z\) are sets and \(f : X \to Y, g : Y \to Z\) are functions such that \(g \circ f\) is surjective.
Then we have
\begin{align*}
& \forall\ z \in Z, \exists\ x \in X : z = (g \circ f)(x) & \text{(by Definition \ref{3.3.17})} \\
\implies & \forall\ z \in Z, \exists\ x \in X : z = g(f(x)) & \text{(by Definition \ref{3.3.10})} \\
\implies & \forall\ z \in Z, \exists\ f(x) \in Y : z = g(f(x)) & \text{(by Definition \ref{3.3.1})} \\
\implies & g \text{ is surjective}. & \text{(by Definition \ref{3.3.17})}
\end{align*}
And we don't need \(f\) to be surjective.
\end{proof}

\begin{exercise}\label{ex 3.3.6}
Let \(f : X \to Y\) be a bijective function, and let \(f^{-1} : Y \to X\) be its inverse.
Verify the cancellation laws \(f^{-1}(f(x)) = x\) for all \(x \in X\) and \(f(f^{-1}(y)) = y\) for all \(y \in Y\).
Conclude that \(f^{-1}\) is also invertible, and has \(f\) as its inverse (thus \((f^{-1})^{-1} = f\)).
\end{exercise}

\begin{proof}
We first show that \(f\) is bijective implies \(f^{-1}(f(x)) = x \land f(f^{-1}(y)) = y\).
Suppose \(X, Y\) are sets and \(f : X \to Y\) is a function such that \(f\) is bijective.
Then we have \(f^{-1} : Y \to X\) as inverse of \(f\) and
\begin{align*}
& \forall\ x \in X, \forall\ y \in Y : f(x) = y \iff f^{-1}(y) = x \\
\implies & f^{-1}(f(x)) = x \land f(f^{-1}(y)) = y. & \text{(by Definition \ref{3.3.1})}
\end{align*}

Now we show that \(f\) is bijective implies \((f^{-1})^{-1} = f\).
Suppose \(X, Y\) are sets and \(f : X \to Y\) is a function such that \(f\) is bijective.
Then we have \(f^{-1} : Y \to X\) as inverse of \(f\) and \((f^{-1})^{-1} : X \to Y\) as inverse of \(f^{-1}\) such that
\begin{align*}
& \forall\ x \in X, \forall\ y \in Y : \\
& (f(x) = y \iff f^{-1}(y) = x \iff (f^{-1})^{-1}(x) = y) \\
\implies & f = (f^{-1})^{-1}. & \text{(by Definition \ref{3.3.7})}
\end{align*}
\end{proof}

\begin{exercise}\label{ex 3.3.7}
Let \(f : X \to Y\) and \(g : Y \to Z\) be functions.
Show that if \(f\) and \(g\) are bijective, then so is \(g \circ f\), and we have \((g \circ f)^{-1} = f^{-1} \circ g^{-1}\).
\end{exercise}

\begin{proof}
We first show that \(f, g\) are bijective implies \(g \circ f\) is bijective.
Suppose \(X, Y, Z\) are sets and \(f : X \to Y, g : Y \to Z\) are functions such that \(f, g\) are bijective.
Then we have
\begin{align*}
& f, g \text{ are bijective} \\
\implies & f, g \text{ are injective} \land f, g \text{ are surjective} & \text{(by Definition \ref{3.3.20})} \\
\implies & g \circ f \text{ are injective} \land g \circ f \text{ are surjective} & \text{(by Exercise \ref{ex 3.3.2})} \\
\implies & g \circ f \text{ is bijective}. & \text{(by Definition \ref{3.3.20})}
\end{align*}

Now we show that \(f, g\) are bijective implies \((g \circ f)^{-1} = f^{-1} \circ g^{-1}\).
Suppose \(X, Y, Z\) are sets and \(f : X \to Y, g : Y \to Z\) are functions such that \(f, g\) are bijective.
From proof above we know that \(g \circ f\) is also bijective.
Then we have
\begin{align*}
& \forall\ x \in X, \forall\ y \in Y, \forall\ z \in Z : \\
& (y = f(x) \iff x = f^{-1}(y) \\
& \land (z = g(y) \iff y = g^{-1}(z) \\
& \land (z = (g \circ f)(x) \iff x = (g \circ f)^{-1}(z)) \\
\implies & f^{-1}(g^{-1}(z)) = (g \circ f)^{-1}(z) & \text{(by Definition \ref{3.3.1})} \\
\implies & (f^{-1} \circ g^{-1})(z) = (g \circ f)^{-1}(z) & \text{(by Definition \ref{3.3.10})} \\
\implies & f^{-1} \circ g^{-1} = (g \circ f)^{-1}. & \text{(by Definition \ref{3.3.7})}
\end{align*}
\end{proof}

\begin{exercise}\label{ex 3.3.8}
If \(X\) is a subset of \(Y\), let \(\iota_{X \to Y} : X \to Y\) be the \emph{inclusive map from \(X\) to \(Y\)}, defined by mapping \(x \mapsto x\) for all \(x \in X\), i.e., \(\iota_{X \to Y}(x) \coloneqq x\) for all \(x \in X\).
The map \(\iota_{X \to X}\) is in particular called the \emph{identity map} on \(X\).
\begin{enumerate}
    \item Show that if \(X \subseteq Y \subseteq Z\) then \(\iota_{Y \to Z} \circ \iota_{X \to Y} = \iota_{X \to Z}\).
    \item Show that if \(f : A \to B\) is any function, then \(f = f \circ \iota_{A \to A} = \iota_{B \to B} \circ f\).
    \item Show that if \(f : A \to B\) is a bijective function, then \(f \circ f^{-1} = \iota_{B \to B}\) and \(f^{-1} \circ f = \iota_{A \to A}\).
    \item Show that if \(X\) and \(Y\) are disjoint sets, and \(f : X \to Z\) and \(g : Y \to Z\) are functions, then there is a unique function \(h : X \cup Y \to Z\) such that \(h \circ \iota_{X \to X \cup Y} = f\) and \(h \circ \iota_{Y \to X \cup Y} = g\).
\end{enumerate}
\end{exercise}

\begin{proof}{(a)}
Suppose that \(X, Y, Z\) are sets such that \(X \subseteq Y \subseteq Z\).
Let \(\iota_{X \to Y} : X \to Y, \iota_{Y \to Z} : Y \to Z, \iota_{X \to Z} : X \to Z\) be functions such that \(\forall\ x \in X : \iota_{X \to Y}(x) = \iota_{X \to Z}(x) = x\) and \(\forall\ y \in Y : \iota_{Y \to Z}(y) = y\).
Then we have
\begin{align*}
\forall\ x \in X : (\iota_{Y \to Z} \circ \iota_{X \to Y})(x) &= \iota_{Y \to Z}(\iota_{X \to Y}(x)) & \text{(by Definition \ref{3.3.10})} \\
&= \iota_{Y \to Z}(x) \\
&= x \\
&= \iota_{X \to Z}(x).
\end{align*}
Thus by Definition \ref{3.3.7} we have \(\iota_{Y \to Z} \circ \iota_{X \to Y} = \iota_{X \to Z}\).
\end{proof}

\begin{proof}{(b)}
Suppose that \(A, B\) are sets and \(f : A \to B, \iota_{A \to A} : A \to A, \iota_{B \to B} : B \to B\) are functions such that \(\forall\ a \in A : \iota_{A \to A}(a) = a\) and \(\forall\ b \in B : \iota_{B \to B}(b) = b\).
Then we have
\begin{align*}
\forall\ a \in A : f(a) &= f(\iota_{A \to A}(a)) \\
&= (f \circ \iota_{A \to A})(a) & \text{(by Definition \ref{3.3.10})} \\
&= \iota_{B \to B}(f(a)) \\
&= (\iota_{B \to B} \circ f)(a). & \text{(by Definition \ref{3.3.10})}
\end{align*}
Thus by Definition \ref{3.3.7} we have \(f = f \circ \iota_{A \to A} = \iota_{B \to B} \circ f\).
\end{proof}

\begin{proof}{(c)}
Suppose that \(A, B\) are sets and \(f : A \to B, \iota_{A \to A} : A \to A, \iota_{B \to B} : B \to B\) are functions such that \(f\) is bijective, \(\forall\ a \in A : \iota_{A \to A}(a) = a\) and \(\forall\ b \in B : \iota_{B \to B}(b) = b\).
Then we have
\begin{align*}
\forall\ a \in A : a &= f^{-1}(f(a)) & \text{(by Exercise \ref{ex 3.3.6})} \\
&= (f^{-1} \circ f)(a) & \text{(by Definition \ref{3.3.10})} \\
&= \iota_{A \to A}(a). \\
\forall\ b \in B : b &= f(f^{-1}(b)) & \text{(by Exercise \ref{ex 3.3.6})} \\
&= (f \circ f^{-1})(b) & \text{(by Definition \ref{3.3.10})} \\
&= \iota_{B \to B}(b).
\end{align*}
Thus by Definition \ref{3.3.7} we have \(f^{-1} \circ f = \iota_{A \to A}\) and \(f \circ f^{-1} = \iota_{B \to B}\).
\end{proof}

\begin{proof}{(d)}
Suppose that \(X, Y, Z\) are sets such that \(X \cap Y = \emptyset\).
Let \(f : X \to Z, g : Y \to Z\) be functions.
Let \(\iota_{X \to X \cup Y} : X \to X \cup Y, \iota_{Y \to X \cup Y} : Y \to X \cup Y\) be functions such that \(\forall\ x \in X : \iota_{X \to X \cup Y}(x) = x\) and \(\forall\ y \in Y : \iota_{Y \to X \cup Y}(y) = y\).
We now define a function \(h : X \cup Y \to Z\) as follow:
\[
h(i) = \begin{cases}
f(i) & \text{ if } i \in X \\
g(i) & \text{ if } i \in Y
\end{cases}
\]
This function is well-defined since by Axiom \ref{3.4} \(i \in X \cup Y \iff i \in X \lor i \in Y\), and \(X \cap Y = \emptyset \iff \lnot(i \in X \land i \in Y)\).
Now we have
\begin{align*}
\forall\ x \in X : h(x) &= h(\iota_{X \to X \cup Y}(x)) \\
&= (h \circ \iota_{X \to X \cup Y})(x) & \text{(by Definition \ref{3.3.10})} \\
&= f(x). \\
\forall\ y \in Y : h(y) &= h(\iota_{Y \to X \cup Y}(y)) \\
&= (h \circ \iota_{Y \to X \cup Y})(y) & \text{(by Definition \ref{3.3.10})} \\
&= g(y).
\end{align*}
Thus by Definition \ref{3.3.7} we have \(h \circ \iota_{X \to X \cup Y} = f\) and \(h \circ \iota_{Y \to X \cup Y} = g\).

Now suppose there exists another function \(h' : X \cup Y \to Z\) such that \(h' \circ \iota_{X \to X \cup Y} = f\) and \(h' \circ \iota_{Y \to X \cup Y} = g\).
Then we have
\begin{align*}
\forall\ x \in X : f(x) &= (h' \circ \iota_{X \to X \cup Y})(x) \\
&= h'(\iota_{X \to X \cup Y}(x)) & \text{(by Definition \ref{3.3.10})} \\
&= h'(x) \\
&= (h \circ \iota_{X \to X \cup Y})(x) \\
&= h(\iota_{X \to X \cup Y}(x)) & \text{(by Definition \ref{3.3.10})} \\
&= h(x). \\
\forall\ y \in Y : g(y) &= (h' \circ \iota_{Y \to X \cup Y})(y) \\
&= h'(\iota_{Y \to X \cup Y}(y)) & \text{(by Definition \ref{3.3.10})} \\
&= h'(y) \\
&= (h \circ \iota_{Y \to X \cup Y})(y) \\
&= h(\iota_{Y \to X \cup Y}(y)) & \text{(by Definition \ref{3.3.10})} \\
&= h(y).
\end{align*}
Thus by Definition \ref{3.3.7} we have \(h = h'\), so \(h\) is unique.
\end{proof}
\section{Images and inverse images}\label{sec 3.4}

\begin{definition}[Images of sets]\label{3.4.1}
  If \(f : X \to Y\) is a function from \(X\) to \(Y\), and \(S\) is a set in \(X\), we define \(f(S)\) to be the set
  \[
    f(S) \coloneqq \{f(x) : x \in S\};
  \]
  this set is a subset of \(Y\), and is sometimes called the \emph{image} of \(S\) under the map \(f\).
  We sometimes call \(f(S)\) the \emph{forward image} of \(S\) to distinguish it from the concept of the \emph{inverse image} \(f^{-1}(S)\) of \(S\).
\end{definition}

\setcounter{theorem}{3}
\begin{definition}[Inverse images]\label{3.4.4}
  If \(U\) is a subset of \(Y\), we define the set \(f^{-1}(U)\) to be the set
  \[
    f^{-1}(U) \coloneqq \{x \in X : f(x) \in U\}.
  \]
  In other words, \(f^{-1}(U)\) consists of all the elements of \(X\) which map into \(U\):
  \[
    f(x) \in U \iff x \in f^{-1}(U).
  \]
  We call \(f^{-1}(U)\) the \emph{inverse image} of \(U\).
\end{definition}

\setcounter{theorem}{6}
\begin{remark}\label{3.4.6}
  If \(f\) is a bijective function, then we have defined \(f^{-1}\) in two slightly different ways, but this is not an issue because both definitions are equivalent.
\end{remark}

\begin{axiom}[Power set axiom]\label{3.10}
  Let \(X\) and \(Y\) be sets.
  Then there exists a set, denoted \(Y^X\), which consists of all the functions from \(X\) to \(Y\), thus
  \[
    f \in Y^X \iff (f \text{ is a function with domain } X \text{ and range } Y).
  \]
\end{axiom}

\begin{note}
  The reason we use the notation \(Y^X\) to denote this set is that if \(Y\) has \(n\) elements and \(X\) has \(m\) elements, then one can show that \(Y^X\) has \(n^m\) elements.
\end{note}

\setcounter{theorem}{8}
\begin{lemma}\label{3.4.9}
  Let \(X\) be a set.
  Then the set
  \[
    \{Y : Y \text{ is a subset of } X\}
  \]
  is a set.
\end{lemma}

\begin{proof}
  Suppose that \(X\) is a set.
  By \cref{3.10}, there exists a set \(\{0, 1\}^X\) which consists of all the functions from \(X\) to \(\{0, 1\}\).
  \[
    f \in \{0, 1\}^X \iff (f \text{ is a function with domain } X \text{ and range } \{0, 1\}).
  \]
  By \cref{3.6}, we can replace each \(f \in \{0, 1\}^X\) with \(f^{-1}(\{1\})\), i.e., there exists a set
  \[
    S = \{f^{-1}(\{1\}) : f \in \{0, 1\}^X\}.
  \]
  By \cref{3.4.4} we have
  \begin{align*}
             & \forall Y \in S                                                              \\
    \implies & \exists\ f \in \{0, 1\}^X : Y = f^{-1}(\{1\}) = \{x \in X : f(x) \in \{1\}\} \\
    \implies & Y \subseteq X.
  \end{align*}
  Now \(\forall Y' \subseteq X\) we can define the following sets:
  \begin{align*}
    A_0 & = \{0 : x \in X \setminus Y'\}. & \text{(by \cref{3.1.27} and \cref{3.6})} \\
    A_1 & = \{1 : x \in Y'\}.             & \text{(by \cref{3.6})}
  \end{align*}
  So we have
  \begin{align*}
             & \forall Y' \subseteq X                                     \\
    \implies & \exists\ A_0, A_1                                          \\
    \implies & \exists\ f : X \to A_0 \cup A_1 & \text{(by \cref{3.6})}   \\
    \implies & f \in \{0, 1\}^X                & \text{(by \cref{3.10})}  \\
    \implies & f^{-1}(A_1) = Y'                & \text{(by \cref{3.4.4})} \\
    \implies & Y' \in S.
  \end{align*}
  Since \(\forall Y : Y \in S \iff Y \subseteq X\), we have show that \(S = \{Y : Y \subseteq X\}\) exists.
\end{proof}

\begin{remark}\label{3.4.10}
  The set \(\{Y : Y \text{ is a subset of } X\}\) is know as the \emph{power set} of \(X\) and is denoted \(2^X\).
\end{remark}

\begin{axiom}[Union]\label{3.11}
  Let \(A\) be a set, all of whose elements are themselves sets.
  Then there exists a set \(\bigcup A\) whose elements are precisely those objects which are elements of the elements of \(A\), thus for all objects \(x\)
  \[
    x \in \bigcup A \iff (x \in S \text{ for some } S \in A)
  \]
\end{axiom}

\begin{note}
  The axiom of union (\cref{3.11}), combined with the axiom of pair set (\cref{3.3}), implies the axiom of pairwise union (\cref{3.4}).
  Another important consequence of \cref{3.11} is that if one has some set \(I\), and for every element \(\alpha \in I\) we have some set \(A_{\alpha}\), then we can form the union set \(\bigcup_{\alpha \in I} A_{\alpha}\) by defining
  \[
    \bigcup_{\alpha \in I} A_{\alpha} \coloneqq \bigcup \{A_{\alpha} : \alpha \in I\},
  \]
  which is a set thanks to the axiom of replacement (\cref{3.6}) and the axiom of union (\cref{3.11}).
  More generally, we see that for any object \(y\),
  \[
    y \in \bigcup_{\alpha \in I} A_{\alpha} \iff (y \in A_{\alpha} \text{ for some } \alpha \in I).
  \]
  In situations like this, we often refer to \(I\) as an \emph{index set}, and the elements \(\alpha\) of this index set as \emph{labels};
  the sets \(A_{\alpha}\) are then called a \emph{family of sets}, and are \emph{indexed} by the labels \(\alpha \in I\).
  Note that if \(I\) was empty, then \(\bigcup_{\alpha \in I} A_{\alpha}\) would automatically also be empty.
\end{note}

\begin{note}
  We can similarly form intersections of families of sets, as long as the index set is non-empty.
  More specifically, given any non-empty set \(I\), and given an assignment of a set \(A_{\alpha}\) to each \(\alpha \in I\), we can define the intersection \(\bigcap_{\alpha \in I} A_{\alpha}\) by first choosing some element \(\beta\) of \(I\) (which we can do since \(I\) is non-empty), and setting
  \[
    \bigcap_{\alpha \in I} A_{\alpha} \coloneqq \{x \in A_{\beta} : x \in A_{\alpha} \text{ for all } \alpha \in I\},
  \]
  which is a set by the axiom of specification (\cref{3.5}).
  This definition may look like it depends on the choice of \(\beta\), but it does not.
  Observe that for any object \(y\),
  \[
    y \in \bigcap_{\alpha \in I} A_{\alpha} \iff (y \in A_{\alpha} \text{ for all } \alpha \in I).
  \]
\end{note}

\setcounter{theorem}{11}
\begin{remark}\label{3.4.12}
  The axioms of set theory that we have introduced (\crefrange{3.1}{3.11}, excluding the dangerous \cref{3.8}) are known as the \emph{Zermelo-Fraenkel axioms of set theory}, after Ernst Zermelo (1871 -- 1953) and Abraham Fraenkel (1891 -- 1965).
  There is one further axiom we will eventually need, the famous \emph{axiom of choice}, giving rise to the \emph{Zermelo-Fraenkel-Choice (ZFC) axioms of set theory}, but we will not need this axiom for some time.
\end{remark}

\exercisesection

\begin{exercise}\label{ex 3.4.1}
  Let \(f : X \to Y\) be a bijective function, and let \(f^{-1} : Y \to X\) be its inverse.
  Let \(V\) be any subset of \(Y\).
  Prove that the forward image of \(V\) under \(f^{-1}\) is the same set as the inverse image of \(V\) under \(f\);
  thus the fact that both sets are denoted by \(f^{-1}(V)\) will not lead to any inconsistency.
\end{exercise}

\begin{proof}
  Suppose that \(X, Y, V\) are sets and \(f : X \to Y\) is a function such that \(V \subseteq Y\) and \(f\) is bijective.
  Let \(f^{-1} : Y \to X\) be the inverse of \(f\).
  Let \(A\) be the set of the forward image of \(V\) under \(f^{-1}\).
  Let \(B\) be the set of the inverse image of \(V\) under \(f\).
  Then we have
  \begin{align*}
    \forall x \in A \iff & \exists\ v \in V : f^{-1}(v) = x & \text{(by \cref{3.4.1})}  \\
    \iff                 & f(f^{-1}(v)) = v = f(x)          & \text{(by \cref{3.3.20})} \\
    \iff                 & x \in B.                         & \text{(by \cref{3.4.4})}
  \end{align*}
  Thus by \cref{3.1.4} we have \(A = B\).
\end{proof}

\begin{exercise}\label{ex 3.4.2}
  Let \(f : X \to Y\) be a function from one set \(X\) to another set \(Y\), let \(S\) be a subset of \(X\), and let \(U\) be a subset of \(Y\).
  What, in general, can one say about \(f^{-1}(f(S))\) and \(S\)?
  What about \(f(f^{-1}(U))\) and \(U\)?
\end{exercise}

\begin{proof}
  We first show that \(S \subseteq f^{-1}(f(S))\).
  Suppose that \(X, Y, S\) are sets such that \(S \subseteq X\) and \(f : X \to Y\) is a function.
  Then we have
  \begin{align*}
    \forall x \in S \implies & f(x) \in f(S)       & \text{(by \cref{3.4.1})} \\
    \implies                 & x \in f^{-1}(f(S)). & \text{(by \cref{3.4.4})}
  \end{align*}
  Thus by \cref{3.1.15} we have \(S \subseteq f^{-1}(f(S))\).

  Now we show that \(f(f^{-1}(U)) \subseteq U\).
  Suppose that \(X, Y, U\) are sets such that \(U \subseteq Y\) and \(f : X \to Y\) is a function.
  Then we have
  \begin{align*}
    \forall y \in f(f^{-1}(U)) \implies & \exists\ x \in f^{-1}(U) : f(x) = y & \text{(by \cref{3.4.1})} \\
    \implies                            & y \in U.                            & \text{(by \cref{3.4.4})}
  \end{align*}
  Thus by \cref{3.1.15} we have \(f(f^{-1}(U)) \subseteq U\).
\end{proof}

\begin{exercise}\label{ex 3.4.3}
  Let \(A\), \(B\) be two subsets of a set \(X\), and let \(f : X \to Y\) be a function.
  Show that \(f(A \cap B) \subseteq f(A) \cap f(B)\), that \(f(A) \setminus f(B) \subseteq f(A \setminus B)\), \(f(A \cup B) = f(A) \cup f(B)\).
  For the first two statements, is it true that the \(\subseteq\) relation can be imporved to \(=\)?
\end{exercise}

\begin{proof}
  We first show that \(f(A \cap B) \subseteq f(A) \cap f(B)\).
  Suppose that \(A, B, X, Y\) are sets such that \(A \subseteq X\) and \(B \subseteq X\).
  Suppose that \(f : X \to Y\) is a function.
  Then we have
  \begin{align*}
             & \forall y \in f(A \cap B)                                                           \\
    \implies & y \in \{f(x) : x \in A \cap B\}                         & \text{(by \cref{3.4.1})}  \\
    \implies & y \in \{f(x) : x \in A \land x \in B\}                  & \text{(by \cref{3.1.23})} \\
    \implies & y \in \{f(x) : x \in A\} \land y \in \{f(x) : x \in B\}                             \\
    \implies & y \in f(A) \land y \in f(B)                             & \text{(by \cref{3.4.1})}  \\
    \implies & y \in f(A) \cap f(B).                                   & \text{(by \cref{3.1.23})}
  \end{align*}
  Thus by \cref{3.1.15} we have \(f(A \cap B) \subseteq f(A) \cap f(B)\).

  Next we show that \(f(A) \setminus f(B) \subseteq f(A \setminus B)\).
  Suppose that \(A, B, X, Y\) are sets such that \(A \subseteq X\) and \(B \subseteq X\).
  Suppose that \(f : X \to Y\) is a function.
  Then we have
  \begin{align*}
             & \forall y \in f(A) \setminus f(B)                                                      \\
    \implies & y \in f(A) \land y \notin f(B)                             & \text{(by \cref{3.1.27})} \\
    \implies & y \in \{f(x) : x \in A\} \land y \notin \{f(x) : x \in B\} & \text{(by \cref{3.4.1})}  \\
    \implies & y \in \{f(x) : x \in A \land x \notin B\}                                              \\
    \implies & y \in \{f(x) : x \in A \setminus B\}                       & \text{(by \cref{3.1.27})} \\
    \implies & y \in f(A \setminus B).                                    & \text{(by \cref{3.4.1})}
  \end{align*}
  Thus by \cref{3.1.15} we have \(f(A) \setminus f(B) \subseteq f(A \setminus B)\).

  Finally we show that \(f(A \cup B) = f(A) \cup f(B)\).
  Suppose that \(A, B, X, Y\) are sets such that \(A \subseteq X\) and \(B \subseteq X\).
  Suppose that \(f : X \to Y\) is a function.
  Then we have
  \begin{align*}
         & \forall y \in f(A \cup B)                                                         \\
    \iff & y \in \{f(x) : x \in A \cup B\}                        & \text{(by \cref{3.4.1})} \\
    \iff & y \in \{f(x) : x \in A \lor x \in B\}                  & \text{(by \cref{3.4})}   \\
    \iff & y \in \{f(x) : x \in A\} \lor y \in \{f(x) : x \in B\}                            \\
    \iff & y \in f(A) \lor y \in f(B)                             & \text{(by \cref{3.4.1})} \\
    \iff & y \in f(A) \cup f(B).                                  & \text{(by \cref{3.4})}
  \end{align*}
  Thus by \cref{3.1.4} we have \(f(A \cup B) = f(A) \cup f(B)\).
\end{proof}

\begin{exercise}\label{ex 3.4.4}
  Let \(f : X \to Y\) be a function from one set \(X\) to another set \(Y\), and let \(U\), \(V\) be subsets of \(Y\). Show that \(f^{-1}(U \cup V) = f^{-1}(U) \cup f^{-1}(V)\), that
  \(f^{-1}(U \cap V) = f^{-1}(U) \cap f^{-1}(V)\), and that \(f^{-1}(U \setminus V) = f^{-1}(U) \setminus f^{-1}(V)\).
\end{exercise}

\begin{proof}
  We first show that \(f^{-1}(U \cup V) = f^{-1}(U) \cup f^{-1}(V)\).
  Suppose that \(U, V, X, Y\) are sets such that \(U \subseteq Y\) and \(V \subseteq Y\).
  Suppose that \(f : X \to Y\) is a function.
  Then we have
  \begin{align*}
         & \forall x \in f^{-1}(U \cup V)                                  \\
    \iff & f(x) \in U \cup V                    & \text{(by \cref{3.4.4})} \\
    \iff & f(x) \in U \lor f(x) \in V           & \text{(by \cref{3.4})}   \\
    \iff & x \in f^{-1}(U) \lor x \in f^{-1}(V) & \text{(by \cref{3.4.4})} \\
    \iff & x \in f^{-1}(U) \cup f^{-1}(V).      & \text{(by \cref{3.4})}
  \end{align*}
  Thus by \cref{3.1.4} we have \(f^{-1}(U \cup V) = f^{-1}(U) \cup f^{-1}(V)\).

  Next we show that \(f^{-1}(U \cap V) = f^{-1}(U) \cap f^{-1}(V)\).
  Suppose that \(U, V, X, Y\) are sets such that \(U \subseteq Y\) and \(V \subseteq Y\).
  Suppose that \(f : X \to Y\) is a function.
  Then we have
  \begin{align*}
         & \forall x \in f^{-1}(U \cap V)                                    \\
    \iff & f(x) \in U \cap V                     & \text{(by \cref{3.4.4})}  \\
    \iff & f(x) \in U \land f(x) \in V           & \text{(by \cref{3.1.23})} \\
    \iff & x \in f^{-1}(U) \land x \in f^{-1}(V) & \text{(by \cref{3.4.4})}  \\
    \iff & x \in f^{-1}(U) \cap f^{-1}(V).       & \text{(by \cref{3.1.23})}
  \end{align*}
  Thus by \cref{3.1.4} we have \(f^{-1}(U \cap V) = f^{-1}(U) \cap f^{-1}(V)\).

  Finally we show that \(f^{-1}(U \setminus V) = f^{-1}(U) \setminus f^{-1}(V)\).
  Suppose that \(U, V, X, Y\) are sets such that \(U \subseteq Y\) and \(V \subseteq Y\).
  Suppose that \(f : X \to Y\) is a function.
  Then we have
  \begin{align*}
         & \forall x \in f^{-1}(U \setminus V)                                  \\
    \iff & f(x) \in U \setminus V                   & \text{(by \cref{3.4.4})}  \\
    \iff & f(x) \in U \land f(x) \notin V           & \text{(by \cref{3.1.23})} \\
    \iff & x \in f^{-1}(U) \land x \notin f^{-1}(V) & \text{(by \cref{3.4.4})}  \\
    \iff & x \in f^{-1}(U) \setminus f^{-1}(V).     & \text{(by \cref{3.1.23})}
  \end{align*}
  Thus by \cref{3.1.4} we have \(f^{-1}(U \setminus V) = f^{-1}(U) \setminus f^{-1}(V)\).
\end{proof}

\begin{exercise}\label{ex 3.4.5}
  Let \(f : X \to Y\) be a function from one set \(X\) to another set \(Y\).
  Show that \(f(f^{-1}(S)) = S\) for every \(S \subseteq Y\) if and only if \(f\) is surjective.
  Show that \(f^{-1}(f(S)) = S\) for every \(S \subseteq X\) if and only if \(f\) is injective.
\end{exercise}

\begin{proof}
  We first show that \(\forall S \subseteq Y : f(f^{-1}(S)) = S \iff f\) is surjective.
  Suppose that \(X, Y, S\) are sets such that \(S \subseteq Y\) and \(f : X \to Y\) is a function.
  Then we have
  \begin{align*}
         & f \text{ is surjective}                                                                                         \\
    \iff & (\forall S \subseteq Y : y \in S \implies \exists\ x \in X : f(x) = y)            & \text{(by \cref{3.3.17})}   \\
    \iff & (\forall S \subseteq Y : y \in S \implies \exists\ x \in f^{-1}(S) : f(x) = y)    & \text{(by \cref{3.4.4})}    \\
    \iff & (\forall S \subseteq Y : y \in S \implies y \in f(f^{-1}(S)))                     & \text{(by \cref{3.4.1})}    \\
    \iff & (\forall S \subseteq Y : S \subseteq f(f^{-1}(S)))                                & \text{(by \cref{3.1.15})}   \\
    \iff & (\forall S \subseteq Y : S \subseteq f(f^{-1}(S)) \land f(f^{-1}(S)) \subseteq S) & \text{(by \cref{ex 3.4.2})} \\
    \iff & (\forall S \subseteq Y : S = f(f^{-1}(S))).                                       & \text{(by \cref{3.1.18})}
  \end{align*}

  Now we show that \(\forall S \subseteq X : f^{-1}(f(S)) = S \iff f\) is injective.
  Suppose that \(X, Y, S\) are sets such that \(S \subseteq X\) and \(f : X \to Y\) is a function.
  If \(f\) is injective, then \(\forall S \subseteq X\) we have
  \begin{align*}
             & x \in f^{-1}(f(S))                                                             \\
    \implies & f(x) \in f(S)                                      & \text{(by \cref{3.4.4})}  \\
    \implies & \exists\ x' \in S : (f(x) = f(x') \implies x = x') & \text{(by \cref{3.3.14})} \\
    \implies & x \in S.
  \end{align*}
  Thus \(f^{-1}(f(S)) \subseteq S\).
  By \cref{ex 3.4.2} we have \(S \subseteq f^{-1}(f(S))\), thus by \cref{3.1.18} we have \(S = f^{-1}(f(S))\).
  On the other hand, if \(\forall S \subseteq X : f^{-1}(f(S)) = S\), then we have
  \begin{align*}
             & \forall x, x' \in S : f(x) = f(x')                                              \\
    \implies & x \in f^{-1}(f(\{x\})) = f^{-1}(f(\{x'\})) = \{x'\}                             \\
    \implies & x = x'                                                                          \\
    \implies & f \text{ is injective}.                             & \text{(by \cref{3.3.14})}
  \end{align*}
  Thus we conclude that \(\forall S \subseteq X : f^{-1}(f(S)) = S \iff f\) is injective.
\end{proof}

\begin{exercise}\label{ex 3.4.6}
  Prove \cref{3.4.9}.
\end{exercise}

\begin{proof}
  See \cref{3.4.9}.
\end{proof}

\begin{exercise}\label{ex 3.4.7}
  Let \(X\), \(Y\) be sets.
  Define a \emph{partial function} from \(X\) to \(Y\) to be any function \(f : X' \to Y'\) whose domain \(X'\) is a subset of \(X\), and whose range \(Y'\) is a subset of \(Y\).
  Show that the collection of all partial functions from \(X\) to \(Y\) is itself a set.
\end{exercise}

\begin{proof}
  Suppose that \(X, Y\) are sets.
  Then by \cref{3.4.9}, the set \(A = \{X' : X' \subseteq X\}\) exists, so does \(B = \{Y' : Y' \subseteq Y\}\).
  Now we have
  \begin{align*}
    C_1 & = \{Y'^{X'} : X' \in A \land Y' \in B\}             & \text{(by \cref{3.10})} \\
    C_2 & = \bigcup C_1 = \{f \in Y'^{X'} : Y'^{X'} \in C_1\} & \text{(by \cref{3.11})} \\
  \end{align*}
  If \(f : X' \to Y'\) is a partial function whose domain \(X' \subseteq X\) and whose range \(Y' \subseteq Y\), then we have \(Y'^{X'} \in C_1\), and thus \(f \in C_2\).
\end{proof}

\begin{exercise}\label{ex 3.4.8}
  Show that \cref{3.4} can be deduced from \cref{3.1}, \cref{3.3} and \cref{3.11}.
\end{exercise}

\begin{proof}
  By \cref{3.1}, \(A\) is a set and \(B\) is a set.
  And if \(x\) is a object, we can say \(x \in A\) or \(x \in B\).
  By \cref{3.3}, there exists a set \(\{A, B\}\) whose only elements are \(A\) and \(B\).
  By \cref{3.11}, \(x \in \bigcup \{A, B\} \iff x \in A \lor x \in B\).
  By defining \(A \cup B \coloneqq \bigcup \{A, B\}\), we show that \(x \in A \cup B \iff x \in A \lor x \in B\) is true.
\end{proof}

\begin{exercise}\label{ex 3.4.9}
  Show that if \(\beta\) and \(\beta'\) are two elements of a set \(I\), and to each \(\alpha \in I\) we assign a set \(A_{\alpha}\), then
  \[
    \{x \in A_{\beta} : x \in A_{\alpha} \ \forall \alpha \in I\} = \{x \in A_{\beta'} : x \in A_{\alpha} \ \forall \alpha \in I\},
  \]
  and so the definition of \(\bigcap_{\alpha \in I} A_{\alpha}\) does not depend on \(\beta\).
\end{exercise}

\begin{proof}
  Suppose that \(I\) is a set and \(\forall \alpha \in I : A_{\alpha}\) is a set.
  Let \(\beta, \beta' \in I\) and \(B, B'\) be sets
  \begin{align*}
    B  & = \{x \in A_{\beta} : x \in A_{\alpha} \ \forall \alpha \in I\}   \\
    B' & = \{x \in A_{\beta'} : x \in A_{\alpha} \ \forall \alpha \in I\}.
  \end{align*}
  We now show that \(B = B'\).
  \begin{align*}
         & \forall x \in B                                                                 \\
    \iff & x \in A_{\beta} \land x \in A_{\alpha} \ \forall \alpha \in I                   \\
    \iff & x \in A_{\alpha} \ \forall \alpha \in I                        & (\beta \in I)  \\
    \iff & x \in A_{\beta'} \land x \in A_{\alpha} \ \forall \alpha \in I & (\beta' \in I) \\
    \iff & x \in B'.
  \end{align*}
  Thus by \cref{3.1.4} we have \(B = B'\).
\end{proof}

\begin{exercise}\label{ex 3.4.10}
  Suppose that \(I\) and \(J\) are two sets, and for all \(\alpha \in I \cup J\) let \(A_{\alpha}\) be a set.
  Show that \((\bigcup_{\alpha \in I} A_{\alpha}) \cup (\bigcup_{\alpha \in J} A_{\alpha}) = \bigcup_{\alpha \in I \cup J} A_{\alpha}\).
  If \(I\) and \(J\) are non-empty, show that \((\bigcap_{\alpha \in I} A_{\alpha}) \cap (\bigcap_{\alpha \in J} A_{\alpha}) = \bigcap_{\alpha \in I \cup J} A_{\alpha}\).
\end{exercise}

\begin{proof}
  We first show that \((\bigcup_{\alpha \in I} A_{\alpha}) \cup (\bigcup_{\alpha \in J} A_{\alpha}) = \bigcup_{\alpha \in I \cup J} A_{\alpha}\).
  Suppose that \(I\) and \(J\) are two sets, and \(\forall \alpha \in I \cup J : A_{\alpha}\) be a set.
  Then we have
  \begin{align*}
         & \forall x \in (\bigcup_{\alpha \in I} A_{\alpha}) \cup (\bigcup_{\alpha \in J} A_{\alpha})                           \\
    \iff & x \in \bigcup_{\alpha \in I} A_{\alpha} \lor x \in \bigcup_{\alpha \in J} A_{\alpha}       & \text{(by \cref{3.4})}  \\
    \iff & (\exists\ \alpha \in I : x \in A_{\alpha}) \lor (\exists\ \alpha \in J : x \in A_{\alpha}) & \text{(by \cref{3.11})} \\
    \iff & \exists\ \alpha \in I \lor \alpha \in J : x \in A_{\alpha}                                                           \\
    \iff & \exists\ \alpha \in I \cup J : x \in A_{\alpha}                                            & \text{(by \cref{3.4})}  \\
    \iff & x \in \bigcup_{\alpha \in I \cup J} A_{\alpha}.                                            & \text{(by \cref{3.11})}
  \end{align*}
  Thus by \cref{3.1.4} we have \((\bigcup_{\alpha \in I} A_{\alpha}) \cup (\bigcup_{\alpha \in J} A_{\alpha}) = \bigcup_{\alpha \in I \cup J} A_{\alpha}\).

  Now we show that \(I \neq \emptyset \land J \neq \emptyset \implies (\bigcap_{\alpha \in I} A_{\alpha}) \cap (\bigcap_{\alpha \in J} A_{\alpha}) = \bigcap_{\alpha \in I \cup J} A_{\alpha}\).
  Suppose that \(I\) and \(J\) are two non-empty sets, and \(\forall \alpha \in I \cup J : A_{\alpha}\) be a set.
  Then we have
  \begin{align*}
         & \forall x \in (\bigcap_{\alpha \in I} A_{\alpha}) \cap (\bigcap_{\alpha \in J} A_{\alpha})                               \\
    \iff & x \in \bigcap_{\alpha \in I} A_{\alpha} \land x \in \bigcap_{\alpha \in J} A_{\alpha}      & \text{(by \cref{3.1.23})}   \\
    \iff & (\forall \alpha \in I : x \in A_{\alpha}) \land (\forall \alpha \in J : x \in A_{\alpha})  & \text{(by \cref{ex 3.4.9})} \\
    \iff & \forall \alpha \in I \lor \alpha \in J : x \in A_{\alpha}                                                                \\
    \iff & \forall \alpha \in I \cup J : x \in A_{\alpha}                                             & \text{(by \cref{3.4})}      \\
    \iff & x \in \bigcap_{\alpha \in I \cup J} A_{\alpha}.                                            & \text{(by \cref{ex 3.4.9})}
  \end{align*}
  Thus by \cref{3.1.4} we have \((\bigcap_{\alpha \in I} A_{\alpha}) \cap (\bigcap_{\alpha \in J} A_{\alpha}) = \bigcap_{\alpha \in I \cup J} A_{\alpha}\).
\end{proof}

\begin{exercise}\label{ex 3.4.11}
  Let \(X\) be a set, let \(I\) be a non-empty set, and for all \(\alpha \in I\) let \(A_{\alpha}\) be a subset of \(X\).
  Show that
  \[
    X \setminus \bigcup_{\alpha \in I} A_{\alpha} = \bigcap_{\alpha \in I} (X \setminus A_{\alpha})
  \]
  and
  \[
    X \setminus \bigcap_{\alpha \in I} A_{\alpha} = \bigcup_{\alpha \in I} (X \setminus A_{\alpha}).
  \]
  This should be compared with de Morgan's laws in \cref{3.1.28}
  (although one cannot derive the above identities directly from de Morgan's laws, as \(I\) could be infinite).
\end{exercise}

\begin{proof}
  We first show that \(X \setminus \bigcup_{\alpha \in I} A_{\alpha} = \bigcap_{\alpha \in I} (X \setminus A_{\alpha})\).
  Suppose that \(X, I\) are sets, \(I \neq \emptyset\), \(\forall \alpha \in I : A_{\alpha}\) is a set and \(A_{\alpha} \subseteq X\).
  Then we have
  \begin{align*}
         & \forall x \in X \setminus \bigcup_{\alpha \in I} A_{\alpha}                                 \\
    \iff & x \in X \land x \notin \bigcup_{\alpha \in I} A_{\alpha}      & \text{(by \cref{3.1.27})}   \\
    \iff & x \in X \land \lnot(\exists\ \alpha \in I : x \in A_{\alpha}) & \text{(by \cref{3.11})}     \\
    \iff & x \in X \land (\forall \alpha \in I : x \notin A_{\alpha})                                  \\
    \iff & \forall \alpha \in I : x \in X \land x \notin A_{\alpha}                                    \\
    \iff & \forall \alpha \in I : x \in X \setminus A_{\alpha}           & \text{(by \cref{3.1.27})}   \\
    \iff & x \in \bigcap_{\alpha \in I} (X \setminus A_{\alpha})         & \text{(by \cref{ex 3.4.9})} \\
  \end{align*}
  Thus by \cref{3.1.4} we have \(X \setminus \bigcup_{\alpha \in I} A_{\alpha} = \bigcap_{\alpha \in I} (X \setminus A_{\alpha})\).

  Now we show that \(X \setminus \bigcap_{\alpha \in I} A_{\alpha} = \bigcup_{\alpha \in I} (X \setminus A_{\alpha})\).
  Suppose that \(X, I\) are sets, \(I \neq \emptyset\), \(\forall \alpha \in I : A_{\alpha}\) is a set and \(A_{\alpha} \subseteq X\).
  Then we have
  \begin{align*}
         & \forall x \in X \setminus \bigcap_{\alpha \in I} A_{\alpha}                                \\
    \iff & x \in X \land x \notin \bigcap_{\alpha \in I} A_{\alpha}     & \text{(by \cref{3.1.27})}   \\
    \iff & x \in X \land \lnot(\forall \alpha \in I : x \in A_{\alpha}) & \text{(by \cref{ex 3.4.9})} \\
    \iff & x \in X \land (\exists\ \alpha \in I : x \notin A_{\alpha})                                \\
    \iff & \exists\ \alpha \in I : x \in X \land x \notin A_{\alpha}                                  \\
    \iff & \exists\ \alpha \in I : x \in X \setminus A_{\alpha}         & \text{(by \cref{3.1.27})}   \\
    \iff & x \in \bigcup_{\alpha \in I} (X \setminus A_{\alpha})        & \text{(by \cref{3.11})}     \\
  \end{align*}
  Thus by \cref{3.1.4} we have \(X \setminus \bigcap_{\alpha \in I} A_{\alpha} = \bigcup_{\alpha \in I} (X \setminus A_{\alpha})\).
\end{proof}
\section{Cartesian products}\label{sec 3.5}

\begin{definition}[Ordered pair]\label{3.5.1}
If \(x\) and \(y\) are any objects (possibly equal), we define the \emph{ordered pair} \((x, y)\) to be a new object, consisting of \(x\) as its first component and \(y\) as its second component.
Two ordered pairs \((x, y)\) and \((x', y')\) are considered equal if and only if both their components match, i.e.
\[
    (x, y) = (x', y') \iff (x = x' \text{ and } y = y').
\]
\end{definition}

\begin{remark}\label{3.5.2}
Strictly speaking, this definition is partly an axiom, because we have simply postulated that given any two objects \(x\) and \(y\), that an object of the form \((x, y)\) exists.
However, it is possible to define an ordered pair using the axioms of set theory in such a way that we do not need any further postulates.
\end{remark}

\begin{remark}\label{3.5.3}
We have now ``overloaded'' the parenthesis symbols \(()\) once again;
they now are not only used to denote grouping of operators and arguments of functions, but also to enclose ordered pairs.
This is usually not a problem in practice as one can still determine what usage the symbols \(()\) were intended for from context.
\end{remark}

\begin{definition}[Cartesian product]\label{3.5.4}
If \(X\) and \(Y\) are sets, then we define the \emph{Cartesian product} \(X \times Y\) to be the collection of ordered pairs, whose first component lies in \(X\) and second component lies in \(Y\), thus
\[
    X \times Y \coloneqq \{(x, y) : x \in X, y \in Y\}
\]
or equivalently,
\[
    a \in X \times Y \iff (a = (x, y) \text{ for some } x \in X \text{ and } y \in Y).
\]
\end{definition}

\begin{remark}\label{3.5.5}
We shall simply assume that our notion of ordered pair is such that whenever \(X\) and \(Y\) are sets, the Cartesian product \(X \times Y\) is also a set.
\end{remark}

\begin{note}
Let \(f : X \times Y \to Z\) be a function whose domain \(X \times Y\) is a Cartesian product of two other sets \(X\) and \(Y\).
Then \(f\) can either be thought of as a function of one variable, mapping the single input of an ordered pair \((x, y)\) in \(X \times Y\) to an output \(f(x, y)\) in \(Z\), or as a function of two variables, mapping an input \(x \in X\) and another input \(y \in Y\) to a single output \(f(x, y)\) in \(Z\).
While the two notions are technically different, we will not bother to distinguish the two, and think of \(f\) simultaneously as a function of one variable with domain \(X \times Y\) and as a function of two variables with domains \(X\) and \(Y\).
Thus for instance the addition operation \(+\) on the natural numbers can now be re-interpreted as a function \(+ : N \times N \to N\), defined by \((x, y) \mapsto x + y\).
\end{note}

\setcounter{theorem}{6}
\begin{definition}[Ordered \(n\)-tuple and \(n\)-fold Cartesian product]\label{3.5.7}
Let \(n\) be a natural number.
An \emph{ordered \(n\)-tuple} \((x_i)_{1 \leq i \leq n}\) (also denoted \((x_1, \cdots, x_n)\)) is a collection of objects \(x_i\), one for every natural number \(i\) between \(1\) and \(n\);
we refer to \(x_i\) as the \emph{\(i^{th}\) component} of the ordered \(n\)-tuple.
Two ordered \(n\)-tuples \((x_i)_{1 \leq i \leq n}\) and \((y_i)_{1 \leq i \leq n}\) are said to be equal iff \(x_i = y_i\) for all \(1 \leq i \leq n\).
If \((X_i)_{1 \leq i \leq n}\) is an ordered \(n\)-tuple of sets, we define their \emph{Cartesian product} \(\prod_{1 \leq i \leq n} X_i\) (also denoted \(\prod_{i=1}^n X_i\) or \(X_1 \times \cdots \times X_n\)) by
\[
    \prod_{1 \leq i \leq n} X_i \coloneqq \{(x_i)_{1 \leq i \leq n} : x_i \in X_i \text{ for all } 1 \leq i \leq n\}.
\]
\end{definition}

\begin{remark}\label{3.5.8}
One can show that \(\prod_{1 \leq i \leq n} X_i\) is indeed a set.
Indeed, from the power set axiom we can consider the set of all functions \(i \mapsto x_i\) from the domain \(\{1 \leq i \leq n\}\) to the range \(\bigcup_{1 \leq i \leq n} X_i\), and then we can restrict using the axiom of specification to restrict to those functions \(i \mapsto x_i\) for which \(x_i \in X_i\) for all \(1 \leq i \leq n\).
\end{remark}

\begin{note}
Strictly speaking, the sets \(X_1 \times X_2 \times X_3\), \((X_1 \times X_2) \times X_3\), and \(X_1 \times (X_2 \times X_3)\) are distinct.
However, they are clearly very related to each other (for instance, there are obvious bijections between any two of the three sets), and it is common in practice to neglect the minor distinctions between these sets and pretend that they are in fact equal.
Thus a function \(f : X_1 \times X_2 \times X_3 \to Y\) can be thought of as a function of one variable \((x_1, x_2, x_3) \in X_1 \times X_2 \times X_3\), or as a function of three variables \(x_1 \in X_1\), \(x_2 \in X_2\), \(x_3 \in X_3\), or as a function of two variables \(x_1 \in X_1\), \((x_2, x_3) \in X_2 \times X_3\), and so forth;
we will not bother to distinguish between these different perspectives.
\end{note}

\setcounter{theorem}{9}
\begin{remark}\label{3.5.10}
An ordered \(n\)-tuple \(x_1, \cdots, x_n\) of objects is also called an \emph{ordered sequence} of \(n\) elements, or a \emph{finite sequence} for short.
\end{remark}

\begin{note}
If \(x\) is an object, then \((x)\) is a \(1\)-tuple, which we shall identify with \(x\) itself (even though the two are, strictly speaking, not the same object).
Then if \(X_1\) is any set, then the Cartesian product \(\prod_{1 \leq i \leq 1} X_i\) is just \(X_1\).
Also, the \emph{empty Cartesian product} \(\prod_{1 \leq i \leq 0} X_i\) gives, not the empty set \(\{\}\), but rather the singleton set \(\{()\}\) whose only element is the \emph{\(0\)-tuple} \(()\), also known as the \emph{empty tuple}.
\end{note}

\begin{note}
If \(n\) is a natural number, we often write \(X^n\) as shorthand for the \(n\)-fold Cartesian product \(X^n \coloneqq \prod_{1 \leq i \leq n} X\).
Thus \(X^1\) is essentially the same set as \(X\) (if we ignore the distinction between an object \(x\) and the \(1\)-tuple \((x)\)), while \(X^2\) is the Cartesian product \(X \times X\).
The set \(X^0\) is a singleton set \(\{()\}\).
\end{note}

\setcounter{theorem}{11}
\begin{lemma}[Finite choice]\label{3.5.12}
Let \(n \geq 1\) be a natural number, and for each natural number \(1 \leq i \leq n\), let \(X_i\) be a non-empty set.
Then there exists an \(n\)-tuple \((x_i)_{1 \leq i \leq n}\) such that \(x_i \in X_i\) for all \(1 \leq i \leq n\).
In other words, if each \(X_i\) is non-empty, then the set \(\prod_{1 \leq i \leq n} X_i\) is also non-empty.
\end{lemma}

\begin{proof}
We induct on \(n\) (starting with the base case \(n = 1\); the claim is also vacuously true with \(n = 0\) but is not particularly interesting in that case).
When \(n = 1\) the claim follows from Lemma \ref{3.1.6}.
Now suppose inductively that the claim has already been proven for some \(n\);
we will now prove it for \(n++\).
Let \(X_1, \cdots, X_{n++}\) be a collection of non-empty sets.
By induction hypothesis, we can find an \(n\)-tuple \((x_i)_{1 \leq i \leq n}\) such that \(x_i \in X_i\) for all \(1 \leq i \leq n\).
Also, since \(X_{n++}\) is non-empty, by Lemma \ref{3.1.6} we may find an object \(a\) such that \(a \in X_{n++}\).
If we thus define the \(n++\)-tuple \((y_i)_{1 \leq i \leq n++}\) by setting \(y_i \coloneqq x_i\) when \(1 \leq i \leq n\) and \(y_i \coloneqq a\) when \(i = n++\) it is clear that \(y_i \in X_i\) for all \(1 \leq i \leq n++\), thus closing the induction.
\end{proof}

\begin{remark}\label{3.5.13}
It is intuitively plausible that this lemma should be extended to allow for an infinite number of choices, but this cannot be done automatically;
it requires an additional axiom, the \emph{axiom of choice}.
\end{remark}

\exercisesection

\begin{exercise}\label{ex 3.5.1}
Suppose we \emph{define} the ordered pair \((x, y)\) for any objects \(x\) and \(y\) by the formula \((x, y) \coloneqq \{\{x\}, \{x, y\}\}\)
(thus using several applications of Axiom \ref{3.3}).
Show that such a definition indeed obeys the Definition \ref{3.5.1}, and also whenever \(X\) and \(Y\) are sets, the Cartesian product \(X \times Y\) is also a set.
Thus this definition can be validly used as a definition of an ordered pair.
For an additional challenge, show that the alternate definition \((x, y) := \{x, \{x, y\}\}\) also verifies Definition \ref{3.5.1} and is thus also an acceptable definition of ordered pair.
\end{exercise}

\begin{proof}
Let \((x, y) = \{\{x\}, \{x, y\}\}\) and \((x', y') = \{\{x'\}, \{x', y'\}\}\) be two ordered pairs.
We want to show that \((x, y) = (x', y') \iff ((x = x') \land (y = y'))\).
We first prove the necessary condition.
If \((x, y) = (x', y')\), then \(\{\{x\}, \{x, y\}\} = \{\{x'\}, \{x', y'\}\}\) is true, and both \(\{x\} \in \{\{x'\}, \{x', y'\}\}\) and \(\{x, y\} \in \{\{x'\}, \{x', y'\}\}\) are true.
Now we divide into four cases.
    \begin{enumerate}
        \item If \((\{x\} = \{x'\}) \land (\{x, y\} = \{x'\})\), then \((x = x') \land (y = x')\).
        But \(\{x', y'\} \in \{\{x\}, \{x, y\}\}\), so \((\{x', y'\} = \{x\}) \lor (\{x', y'\} = \{x, y\})\).
        If \(\{x', y'\} = \{x\}\), then \(y' = x = x' = y\).
        Otherwise \(\{x', y'\} = \{x, y\}\), then \((y' = x) \lor (y' = y)\), and both can derive \(y' = y\).
        \item If \((\{x\} = \{x'\}) \land (\{x, y\} = \{x', y'\})\), then \((x = x') \land ((y = x') \lor (y = y'))\).
        If \(y = x'\), then we need to show that \(y = y'\).
        But \(\{x', y'\} \in \{\{x\}, \{x, y\}\}\), so \((\{x', y'\} = \{x\}) \lor (\{x', y'\} = \{x, y\})\).
        If \(\{x', y'\} = \{x\}\), then \(y' = x = x' = y\).
        Otherwise \(\{x', y'\} = \{x, y\}\), then \((y' = x) \lor (y' = y)\), and both can derive \(y' = y\).
        \item If \((\{x\} = \{x', y'\}) \land (\{x, y\} = \{x'\})\), then \((x = x') \land (x = y') \land (y = x')\), so \(y = x' = x = y'\).
        \item If \((\{x\} = \{x', y'\}) \land (\{x, y\} = \{x', y'\})\), then \((x = x') \land (x = y') \land ((y = x') \lor (y = y'))\).
        If \(y = x'\), then \(y = x' = x = y'\).
    \end{enumerate}
So \((x, y) = (x', y') \implies ((x = x') \land (y = y'))\).
Now we prove the sufficient condition.
If \((x = x') \land (y = y')\), then \((\{x\} = \{x'\}) \land (\{x, y\} = \{x', y'\})\).
So \(\{\{x\}, \{x, y\}\} = \{\{x'\}, \{x', y'\}\}\) is true, or equivalently \((x, y) = (x', y')\).
Since we have proved both the necessary and the sufficient conditions, we conclude that the given definition of the ordered pair satisfied the constrain \((x, y) = (x', y') \iff ((x = x') \land (y = y'))\).

Next we prove that \(X \times Y\) is a set with the given ordered pair definition.
By Axiom \ref{3.3}, \(\{x\}\) is a set for all \(x \in X\), and \(\{x, y\}\) is also a set for all \(x \in X\) and for all \(y \in Y\).
Again by Axiom \ref{3.3}, \(\{\{x\}, \{x, y\}\}\) is a set for all \(x \in X\) and for all \(y \in Y\).
By Axiom \ref{3.6}, \(\{\{\{x\}, \{x, y\}\} : x \in X \land y \in Y\}\) is a set.
So \(X \times Y = \{(x, y) : x \in X \land y \in Y\}\) is a set with the given ordered pair definition.
\end{proof}

\begin{proof}{(additional challenge)}
Let \((x, y) = \{x, \{x, y\}\}\) and \((x', y') = \{x', \{x', y'\}\}\) be two ordered pairs.
We want to show that \((x, y) = (x', y') \iff ((x = x') \land (y = y'))\).
We first prove the necessary condition.
\((x, y) = (x', y') \iff \{x, \{x, y\}\} = \{x', \{x', y'\}\}\), so both \(x \in \{x', \{x', y'\}\}\) and \(x' \in \{x, \{x, y\}\}\) are true.
Suppose for sake of contradition that \(x \neq x'\).
Then \(x = \{x', y'\}\) is true, and \(x' \in x\) is true.
But \(x' \in \{x, \{x, y\}\} \implies x \in x'\), we get both \((x \in x') \land (x' \in x)\), contradict to Axiom \ref{3.9} and Exercise \ref{ex 3.2.2}.
So \(x = x'\) must be true.
Now we need to show that \(y = y'\).
\(\{x, y\} \in \{x', \{x', y'\}\} \implies \{x, y\} = \{x', y'\}\) (because \(\{x, y\} = x' \implies x \in x' = x\), contradict to Axiom \ref{3.9}) and Exercise \ref{ex 3.2.2}.
So \(((y = x') \lor (y = y')) \land ((y' = x) \lor (y' = y))\) is true, and \(((y = x') \land (y' = x)) \lor (y = y')\) is true.
If \((y = x') \land (y' = x)\), then \(y = x' = x = y'\).
Thus \((x, y) = (x', y') \implies ((x = x') \land (y = y'))\).
Now we prove the sufficient condition.
If \((x = x') \land (y = y')\), then \((x = x') \land (\{x, y\} = \{x', y'\})\).
So \(\{x, \{x, y\}\} = \{x', \{x', y'\}\}\) is true, or equivalently \((x, y) = (x', y')\).
Since we have proved both the necessary and the sufficient conditions, we conclude that the given definition of the ordered pair satisfied the constrain \((x, y) = (x', y') \iff ((x = x') \land (y = y'))\).

Next we prove that \(X \times Y\) is a set with the given ordered pair definition.
By Axiom \ref{3.3}, \(\{x, y\}\) is a set for all \(x \in X\) and for all \(y \in Y\).
Again by Axiom \ref{3.3}, \(\{x, \{x, y\}\}\) is a set for all \(x \in X\) and for all \(y \in Y\).
By Axiom \ref{3.6}, \(\{\{x, \{x, y\}\} : x \in X \land y \in Y\}\) is a set.
So \(X \times Y = \{(x, y) : x \in X \land y \in Y\}\) is a set with the given ordered pair definition.
\end{proof}

\begin{exercise}\label{ex 3.5.2}
Suppose we \emph{define} an ordered \(n\)-tuple to be a surjective function \(x : \{i \in \mathbf{N} : 1 \leq i \leq n\} \to X\) whose range is some arbitrary set \(X\) (so different ordered \(n\)-tuples are allowed to have different ranges);
we then write \(x_i\) for \(x(i)\), and also write \(x\) as \((x_i)_{1 \leq i \leq n}\).
Using this definition, verify that we have \((x_i)_{1 \leq i \leq n} = (y_i)_{1 \leq i \leq n}\) if and only if \(x_i = y_i\) for all \(1 \leq i \leq n\).
Also, show that if \((X_i)_{1 \leq i \leq n}\) are an ordered \(n\)-tuple of sets, then the Cartesian product, as defined in Definition \ref{3.5.7}, is indeed a set.
\end{exercise}

\begin{proof}
We first prove \((x_i)_{1 \leq i \leq n} = (y_i)_{1 \leq i \leq n} \iff x_i = y_i \ \forall\ 1 \leq i \leq n\).
We first prove the necessary condition.
By the given definition, \(x = (x_i)_{1 \leq i \leq n} = (y_i)_{1 \leq i \leq n} = y\).
Since \(x = y\), \(\forall\ i \in \{i \in \mathbf{N}: 1 \leq i \leq n\}\), \(x(i) = y(i)\), but by the definition \(x_i = x(i) = y(i) = y_i\), so \(x_i = y_i \ \forall\ i \in \{i \in \mathbf{N} : 1 \leq i \leq n\}\).
Now we prove the sufficient condition.
\(\forall\ i \in \{i \in \mathbf{N} : 1 \leq i \leq n\}\), \(x_i = y_i\).
But by the definition \(x(i) = x_i = y_i = y(i)\), so \(x = y\) is true.
Again by definition, \(x = y\) means \((x_i)_{1 \leq i \leq n} = (y_i)_{1 \leq i \leq n}\).
Since we have proved both the necessary and sufficient conditions, we conclude that \((x_i)_{1 \leq i \leq n} = (y_i)_{1 \leq i \leq n} \iff x_i = y_i \ \forall\ 1 \leq i \leq n\).

If \((X_i)_{1 \leq i \leq n}\) are an ordered \(n\)-tuple of sets, by Axiom \ref{3.10}, we can consider a set of all functions \(i \mapsto x_i\) from the domain \(\{1 \leq i \leq n\}\) to the range \(\bigcup_{1 \leq i \leq n} X_i\).
We denote such set as \(F\).
Then by Exercise \ref{ex 3.4.7}, there exist a set of all partial function \(P = \{f : A \to B \mid (A \subseteq \{1 \leq i \leq n\}) \land (B \subseteq \bigcup_{1 \leq i \leq n} X_i\})\).
Then by Axiom \ref{3.5}, there exist a set \(\{f \in P \mid (f = i \mapsto x_i) \land (x_i \in X_i) \ \forall\ 1 \leq i \leq n\}\).
Using the given definition, we can rewrite such set as \(\{(x_i)_{1 \leq i \leq n} : x_i \in X_i \text{ for all } 1 \leq i \leq n\}\), which is the same definition as \(\prod_{1 \leq i \leq n} X_i\).
So the Cartesian product is itself a set.
\end{proof}

\begin{exercise}\label{ex 3.5.3}
Show that the definitions of equality for ordered pair and ordered \(n\)-tuple obey the reflexivity, symmetry, and transitivity axioms.
\end{exercise}

\begin{proof}
We first prove the reflexivity.
Let \((x_i)_{1 \leq i \leq n}\) be a \(n\)-tuple.
Then by Definition \ref{3.5.7}, \(x_i = x_i\) for all object \(x_i\), \(1 \leq i \leq n\), so \((x_i)_{1 \leq i \leq n} = (x_i)_{1 \leq i \leq n}\).

Next we prove the symmetry.
Let \((x_i)_{1 \leq i \leq n}\) and \((y_i)_{1 \leq i \leq n}\) be two \(n\)-tuples.
If \((x_i)_{1 \leq i \leq n} = (y_i)_{1 \leq i \leq n}\), then by Definition \ref{3.5.7}, \(x_i = y_i\), \(1 \leq i \leq n\), so \(y_i = x_i\), \(1 \leq i \leq n\).
Thus \((y_i)_{1 \leq i \leq n} = (x_i)_{1 \leq i \leq n}\).

Finally we prove the transitivity.
Let \((x_i)_{1 \leq i \leq n}\), \((y_i)_{1 \leq i \leq n}\) and \((z_i)_{1 \leq i \leq n}\) be three \(n\)-tuples.
If \((x_i)_{1 \leq i \leq n} = (y_i)_{1 \leq i \leq n}\) and \((y_i)_{1 \leq i \leq n} = (z_i)_{1 \leq i \leq n}\), then by Definition \ref{3.5.7}, \(x_i = y_i\) and \(y_i = z_i\), \(1 \leq i \leq n\), so \(x_i = z_i\), \(1 \leq i \leq n\).
Thus \((x_i)_{1 \leq i \leq n} = (z_i)_{1 \leq i \leq n}\).
\end{proof}

\begin{exercise}\label{ex 3.5.4}
Let \(A\), \(B\), \(C\) be sets.
Show that \(A \times (B \cup C) = (A \times B) \cup (A \times C)\), that \(A \times (B \cap C) = (A \times B) \cap (A \times C)\), and that \(A \times (B \setminus C) = (A \times B) \setminus (A \times C)\).
\end{exercise}

\begin{proof}
We first prove the union part.
\(\forall\ (a, d) \in A \times (B \cup C)\), \((a \in A) \land (d \in B \cup C)\).
If \(d \in B\), then \((a, d) \in A \times B\).
Similarly if \(d \in C\), then \((a, d) \in A \times C\).
Thus \((a, d) \in (A \times B) \cup (A \times C)\).
\(\forall\ (a', d') \in (A \times B) \cup (A \times C)\), \(((a', d') \in A \times B) \lor ((a', d') \in A \times C)\).
If \((a', d') \in A \times B\), then \((a' \in A) \land (d' \in B)\), so \(d' \in B \cup C\) is true.
Similarly if \((a', d') \in A \times C\), then \((a' \in A) \land (d' \in C)\), so \(d' \in B \cup C\) is true.
Thus \((a', d') \in A \times (B \cup C)\).
We conclude that \(A \times (B \cup C) = (A \times B) \cup (A \times C)\).

Next we prove the intersection part.
\(\forall\ (a, d) \in A \times (B \cap C)\), \((a \in A) \land (d \in B \cap C)\).
Because \((a \in A) \land (d \in B)\), so \((a, d) \in A \times B\).
Similarly because \((a \in A) \land (d \in C)\), so \((a, d) \in A \times C\).
Thus \((a, d) \in (A \times B) \cap (A \times C)\).
\(\forall\ (a', d') \in (A \times B) \cap (A \times C)\), \(((a', d') \in A \times B) \land ((a', d') \in A \times C)\).
Because \((a', d') \in A \times B\), so \((a' \in A) \land (d' \in B)\).
Similarly because \((a', d') \in A \times C\), so \((a' \in A) \land (d' \in C)\).
Thus \(d' \in B \cap C\), and \((a', d') \in A \times (B \cap C)\).

Now we prove the difference part.
\(\forall\ (a, d) \in A \times (B \setminus C)\), \((a \in A) \land (d \in B \setminus C)\).
Since \(d \in B \setminus C\), \((d \in B) \land (d \notin C)\).
So \((a \in A) \land (d \in B) \implies (a, d) \in A \times B\), and \((a \in A) \land (d \notin C) \implies (a, d) \notin A \times C\).
Thus \((a, d) \in (A \times B) \setminus (A \times C)\).
\(\forall\ (a', d') \in (A \times B) \setminus (A \times C)\), \(((a', d') \in A \times B) \land ((a', d') \notin A \times C)\).
Because \((a', d') \in A \times B\), so \((a' \in A) \land (d' \in B)\).
Also \(((a', d') \notin A \times C) \land (a' \in A)\), so \(d' \notin C\).
Thus \(d' \in B \setminus C\), and \((a', d') \in A \times (B \setminus C)\).
\end{proof}

\begin{exercise}\label{ex 3.5.5}
Let \(A\), \(B\), \(C\), \(D\) be sets.
Show that \((A \times B) \cap (C \times D) = (A \cap C) \times (B \cap D)\).
Is it true that \((A \times B) \cup (C \times D) = (A \cup C) \times (B \cup D)?\)
Is it true that \((A \times B) \setminus (C \times D) = (A \setminus C) \times (B \setminus D)?\)
\end{exercise}

\begin{proof}
We first prove the intersection part.
\(\forall\ (x, y) \in (A \times B) \cap (C \times D)\), \(((x, y) \in A \times B) \land ((x, y) \in C \times D)\).
So \((x \in A) \land (y \in B) \land (x \in C) \land (y \in D)\), and \((x \in A \cap C) \land (y \in B \cap D)\).
Thus \((x, y) \in (A \cap C) \times (B \cap D)\).
\(\forall\ (x', y') \in (A \cap C) \times (B \cap D)\), \((x' \in A \cap C) \land (y' \in B \cap D)\).
So \((x' \in A \land y' \in B) \land (x' \in C \land y' \in D)\), and \(((x', y') \in A \times B) \land ((x', y') \in C \times D)\).
Thus \((x', y') \in (A \times B) \cap (C \times D)\).
We conclude that \((A \times B) \cap (C \times D) = (A \cap C) \times (B \cap D)\).

Next we prove the union part.
Let \((a, d) \in A \times D\) and \((a \notin C) \land (d \notin B)\).
Because \(a \in A\), so \(a \in A \cup C\).
And because \(d \in D\), so \(d \in B \cup D\).
So \((a, d) \in (A \cup C) \times (B \cup D)\).
But because \(a \notin C\), so \((a, d) \notin C \times D\).
Also because \(d \notin B\), so \((a, d) \notin A \times B\).
Thus \((a, d) \notin (A \times B) \cup (C \times D)\).
We conclude that \((A \times B) \cup (C \times D) \neq (A \cup C) \times (B \cup D)\).

Now we prove the difference part.
Let \((a, b) \in A \times B\) and \((a \notin C) \land (b \in D)\).
Because \(a \notin C\), so \((a, b) \notin C \times D\), and \((a, b) \in (A \times B) \setminus (C \times D)\).
But \((b \in B) \land (b \in D)\), so \(b \notin B \setminus D\), and \((a, b) \notin (A \setminus C) \times (B \setminus D)\).
We conclude that \((A \times B) \setminus (C \times D) \neq (A \setminus C) \times (B \setminus D)\).
\end{proof}

\begin{exercise}\label{ex 3.5.6}
Let \(A\), \(B\), \(C\), \(D\) be non-empty sets.
Show that \(A \times B \subseteq C \times D\) if and only if \(A \subseteq C\) and \(B \subseteq D\), and that \(A \times B = C \times D\) if and only if \(A = C\) and \(B = D\).
What happens if the hypotheses that the \(A\), \(B\), \(C\), \(D\) are all non-empty are removed?
\end{exercise}

\begin{proof}
We first prove the subset part.
\(\forall\ (x, y) \in A \times B \subseteq C \times D\), \((x, y) \in A \times B \implies (x, y) \in C \times D\).
So \((x \in A \implies x \in C) \land (y \in B \implies y \in D)\), and \((A \subseteq C) \land (B \subseteq D)\).
Thus \(A \times B \subseteq C \times D \implies (A \subseteq C) \land (B \subseteq D)\).
\(\forall\ x' \in A \subseteq C\) and \(\forall\ y' \in B \subseteq D\), \(x' \in A \implies x' \in C\) and \(y' \in B \implies y' \in D\).
So \((x', y') \in A \times B \implies (x', y') \in C \times D\).
Thus \((A \subseteq C) \land (B \subseteq D) \implies (A \times B) \subseteq (C \times D)\).
We conclude that \(A \times B \subseteq C \times D \iff (A \subseteq C) \land (B \subseteq D)\).

Next we prove the equality part.
\(\forall\ (x, y) \in A \times B = C \times D\), \((x \in A \iff x \in C) \land (y \in B \iff y \in D)\).
But \(x \in A \iff x \in C\) means \(A = C\), similarly \(y \in B \iff y \in D)\) means \(B = D\).
Thus \(A \times B = C \times D \implies (A = C) \land (B = D)\).
\(\forall\ x' \in A = C\) and \(\forall\ y' \in B = D\), \((x' \in A \iff x' \in C) \land (y' \in B \iff y' \in D)\).
But \((x' \in A) \land (x' \in B) \iff (x', y') \in A \times B\) and \((x' \in C) \land (x' \in D) \iff (x', y') \in C \times D\), so \((x', y') \in A \times B \iff (x', y') \in C \times D\).
Thus \((A = C) \land (B = D) \implies A \times B = C \times D\).
We conclude that \(A \times B = C \times D \iff (A = C) \land (B = D)\).

Finally we show that if the hypothesis removed.
If \(A\), \(D\) are empty sets and \(B\), \(C\) are non-empty sets, then \(A \times B = \emptyset\) and \(C \times D = \emptyset\).
So \(\emptyset = A \times B \subseteq C \times D = \emptyset\), but \(B \subsetneq D = \emptyset\).
Thus the statement \(A \times B \subseteq C \times D \iff (A \subseteq C) \land (B \subseteq D)\) is false.
Also \(\emptyset = A \times B = C \times D = \emptyset\), but \(\emptyset = A \neq C\).
Thus the statement \(A \times B = C \times D \iff (A = C) \land (B = D)\) is false.
\end{proof}

\begin{exercise}\label{ex 3.5.7}
Let \(X\), \(Y\) be sets, and let \(\pi_{X \times Y \to X} : X \times Y \to X\) and \(\pi_{X \times Y \to Y} : X \times Y \to Y\) be the maps \(\pi_{X \times Y \to X}(x, y) \coloneqq x\) and \(\pi_{X \times Y \to Y}(x, y) \coloneqq y\);
these maps are known as the \emph{co-ordinate functions} on \(X \times Y\).
Show that for any functions \(f : Z \to X\) and \(g : Z \to Y\), there exists a unique function \(h : Z \to X \times Y\) such that \(\pi_{X \times Y \to X} \circ h = f\) and \(\pi_{X \times Y \to Y} \circ h = g\).
This function \(h\) is known as the \emph{direct sum} of \(f\) and \(g\) and is denoted \(h = f \oplus g\).
\end{exercise}

\begin{proof}
We first prove the existence.
Since \(h\) has domain \(Z\) and range \(X \times Y\), \(\pi_{X \times Y \to X} \circ h\) have domain \(Z\) and range \(X\), which is the same as \(f\).
Similarly \(\pi_{X \times Y \to Y} \circ h\) has domain \(Z\) and range \(Y\), which is the same as \(g\).
Therefore such \(h\) can exist.

Now we prove the uniqueness.
Suppose that there are two function \(h\) and \(h'\) statisfied the condition.
Then \(\forall\ z \in Z\), \(h(z) \in X \times Y, h'(z) \in X \times Y\).
Let \((x, y) = h(z)\) and \((x', y') = h'(z)\).
So \(\pi_{X \times Y \to X}(h(z)) = x \in X\), \(\pi_{X \times Y \to X}(h'(z)) = x' \in X\), \(\pi_{X \times Y \to Y}(h(z)) = y \in Y\), \(\pi_{X \times Y \to X}(h'(z)) = y' \in Y\).
But \(f(z) = \pi_{X \times Y \to X}(h(z)) = x\) and \(f(z) = \pi_{X \times Y \to X}(h'(z)) = x'\), so \(x = x'\).
Similarly \(g(z) = \pi_{X \times Y \to Y}(h(z)) = y\) and \(g(z) = \pi_{X \times Y \to Y}(h'(z)) = y'\), so \(y = y'\).
Thus \(\forall\ z \in Z\), \(h(z) = (x, y) = (x', y') = h'(z)\), so \(h = h'\).
\end{proof}

\begin{exercise}\label{ex 3.5.8}
Let \(X_1, \cdots, X_n\) be sets.
Show that the Cartesian product \(\prod_{i = 1}^n X_i\) is empty if and only if at least one of the \(X_i\) is empty.
\end{exercise}

\begin{proof}
By Definition \ref{3.5.7} \(\prod_{i = 1}^n X_i = \{(x_i)_{1 \leq i \leq n} : x_i \in X_i, \forall\ 1 \leq i \leq n\}\).
But if at least one of the \(X_i\) is empty, then there does not exist a \(x_i\) such that \(x_i \in X_i\).
Therefore \((x_i)_{1 \leq i \leq n}\) does not exist, which means \(\prod_{i = 1}^n X_i = \emptyset\).
\end{proof}

\begin{exercise}\label{ex 3.5.9}
Suppose that \(I\) and \(J\) are two sets, and for all \(\alpha \in I\) let \(A_{\alpha}\) be a set, and for all \(\beta \in J\) let \(B_{\beta}\) be a set.
Show that \((\bigcup_{\alpha \in I} A_{\alpha}) \cap (\bigcup_{\beta \in J} B_{\beta}) = \bigcup_{(\alpha, \beta) \in I \times J} (A_{\alpha} \cap B_{\beta})\).
\end{exercise}

\begin{proof}
\(\forall\ x \in (\bigcup_{\alpha \in I} A_{\alpha}) \cap (\bigcup_{\beta \in J} B_{\beta}) \iff (x \in \bigcup_{\alpha \in I} A_{\alpha}) \land (x \in \bigcup_{\beta \in J} B_{\beta}) \iff (\exists\ \alpha \in I, x \in A_{\alpha}) \land (\exists\ \beta \in J, x \in B_{\beta}) \iff \exists\ \alpha \in I \ \exists\ \beta \in J, x \in A_{\alpha} \cap B_{\beta} \iff \exists\ (\alpha, \beta) \in I \times J, x \in A_{\alpha} \cap B_{\beta} \iff x \in \bigcup_{(\alpha, \beta) \in I \times J} (A_{\alpha} \cap B_{\beta})\).
\end{proof}

\begin{exercise}\label{ex 3.5.10}
If \(f : X \to Y\) is a function, define the \emph{graph} of \(f\) to be the subset of \(X \times Y\) defined by \(\{(x, f(x)) : x \in X\}\).
Show that two functions \(f : X \to Y\), \(\tilde{f} : X \to Y\) are equal if and only if they have the same graph.
Conversely, if \(G\) is any subset of \(X \times Y\) with the property that for each \(x \in X\), the set \(\{y \in Y : (x, y) \in G\}\) has exactly one element (or in other words, \(G\) obeys the vertical line test), show that there is exactly one function \(f : X \to Y\) whose graph is equal to \(G\).
\end{exercise}

\begin{proof}
We first prove the equality.
\(f = \tilde{f} \iff \forall\ x \in X, f(x) = \tilde{f}(x) \iff \forall\ x \in X, (x, f(x)) = (x, \tilde{f}(x)) \iff \{(x, f(x)) : x \in X\} = \{(x, \tilde{f}(x)) : x \in X\}\).
Thus \(f = \tilde{f}\) iff they have the same graph.

Now we prove that when \(G\) obeys the vertical line test, there is exactly one function \(f\) whose graph is \(G\).
By the given condition, \(\forall\ x \in X\), \(\exists!\ y \in Y\) such that \((x, y) \in G\).
Let \(f : X \to Y\) be a function that \(\forall\ x \in X\), \(f(x) = y\).
If there is another \(f'\) satisfied that \(\forall\ x \in X\), \(f'(x) = y\), then \(f = f'\) by Definition \ref{3.3.7}.
Then the graph of \(f\) is \(\{(x, f(x)) : (x \in X) \land (f(x) = y)\}\), which is the same set as \(G\).
\end{proof}

\begin{exercise}\label{ex 3.5.11}
Show that Axiom \ref{3.10} can in fact be deduced from Lemma \ref{3.4.9} and the other axioms of set theory, and thus Lemma \ref{3.4.9} can be used as an alternate formulation of the power set axiom.
\end{exercise}

\begin{proof}
For any two sets \(X\) and \(Y\), there exists a set \(X \times Y\).
By Lemma \ref{3.4.9}, there exists a set \(A = \{a \mid a \subseteq X \times Y\}\).
By Axiom \ref{3.5}, there exists a set \(B = \{b \mid (b \in A) \land (\forall\ (x, y), (x, y') \in b, (x \in X) \land ((x, y) = (x, y') \implies y = y'))\}\).
By Exercise \ref{ex 3.5.10}, \(\forall\ G \in B\), there is exactly one \(f : X \to Y\) whose graph is equal to \(G\).
Then by Axiom \ref{3.6}, there exist a set \(X^Y = \{f : X \to Y \mid (G \in B) \land (\text{graph of } f = G)\}\).
\end{proof}

\begin{exercise}\label{ex 3.5.12}
Let \(f : \mathbf{N} \times X \to X\) be a function, and let \(c\) be a natural number.
Let \(X\) be an arbitrary set.
Show that there exists a function \(a : \mathbf{N} \to \mathbf{N}\) such that
\[
    a(0) = c
\]
and
\[
    a(n++) = f(n, a(n)) \text{ for all } n \in \mathbf{N},
\]
and furthermore that this function is unique.
For an additional challenge, prove this result without using any properties of the natural numbers other than the Peano axioms directly.
\end{exercise}

\begin{proof}
We claim that for every natural number \(N \in \mathbf{N}\), there exists a unique function \(a_N : \{n \in \mathbf{N} : n \leq N\} \to \mathbf{N}\) such that \(a_N(0) = c\) and \(a_N(n++) = f(n, a_{N}(n))\) for all \(n \in \mathbf{N}\) such that \(n < N\).

We prove the claim by using induction on \(N\).
For \(N = 0\), we need to show that such function \(a_0 : \{n \in \mathbf{N} : n \leq 0\} \to \mathbf{N}\) exists and is unique.
The domain of \(a_0\) is equal to \(\{0\}\) since \(0\) is the only natural number less than or equal to \(0\).
Then by defining \(a_0(0) = c\) where \(c \in \mathbf{N}\) we get a unique function \(a_0 : \{0\} \to \{c\}\) (if \(a'_0 : \{0\} \to \{c\}\), then by Definition \ref{3.3.7} \(a_0 = a'_0\)).
And because the domain of \(a_0\) is equal to \(\{0\}\), so there is no \(n \in \{0\}\) such that \((n \in \mathbf{N}) \land (n < 0)\), so \(a_0(n++) = f(n, a_0(n))\) is vacuously true.
Thus for \(N = 0\) the claim is true.
Suppose inductively that for \(N\) the claim is also true.
The function \(a_N : \{n \in \mathbf{N} : n \leq N\} \to \mathbf{N}\) exists and is unique, and \(a_N(0) = c\) and \(a_N(n++) = f(n, a_N(n))\) for all \(n \in \mathbf{N}\) such that \(n < N\).
Then for \(N++\), we can define \(a_{N++} : \{n \in \mathbf{N} : n \leq N++\} \to \mathbf{N}\) by setting \(a_{N++}(n) = a_N(n)\) when \((n \in \mathbf{N}) \land (n < N++)\) (which is unique by induction hypothesis) and \(a_{N++}(N++) = f(N, a_{N++}(N))\) when \((n \in \mathbf{N}) \land (n = N++)\) (which is also unique because \(N\) is unique by Axiom \ref{2.4} and \(a_{N++}(N) = a_N(N)\) is unique by induction hypothesis, so \(f(n, a(n))\) is unique by Definition \ref{3.3.1}).
So \(a_{N++}\) exists and is unique, and \(a_{N++}(0) = a_N(0) = c\) and \(a_{N++}(n++) = f(n, a_{N++}(n))\) for all \(n \in \mathbf{N}\) such that \(n < N++\).
Thus we conclude that the claim is true.

Now we prove the exercise.
\(\forall\ N \in \mathbf{N}\), we can define \(a(N)\) be the value of \(a_N(N)\), where \(a_N\) is the function in the claim, i.e., \(a(N) = a_{N}(N)\).
Then \(a(0) = a_0(0) = c\), and \(a(N++) = a_{N++}(N++) = f(N, a_{N++}(N)) = f(N, a_{N}(N)) = f(N, a(N))\).
Since all \(a_N\) exists and unique, thus such \(a\) exists and is unique.
\end{proof}

\begin{proof}{(additional challenge)}
We claim that for every natural number \(N \in \mathbf{N}\), there exists a unique pair \(A_N\), \(B_N\) of subsets of \(\mathbf{N}\) which obeys the following properties:
    \begin{enumerate}
        \item \(A_N \cap B_N = \emptyset\)
        \item \(A_N \cup B_N = \mathbf{N}\)
        \item \(0 \in A_N\)
        \item \(N++ \in B_N\)
        \item Whenever \(n \in B_N\), we have \(n++ \in B_N\)
        \item Whenever \(n \in A_N\) and \(n \neq N\), we have \(n++ \in A_N\)
    \end{enumerate}

We prove the claim by using induction on \(N\).
For \(N = 0\), by Axiom \ref{3.3} there exists a set \(\{0\}\), and let \(A_0 = \{0\}\).
Also by Axiom \ref{3.7} and \ref{3.5}, there exists a set \(\mathbf{N}\), and there also exists a set \(B_0 = \{n \in \mathbf{N} \mid n \neq 0\}\).
Then \(A_0 \cap B_0 = \emptyset\), \(A_0 \cup B_0 = \mathbf{N}\), \(0 \in A_0\), \(0++ = 1 \in B_0\), \(\forall\ n \in B_0\) whenever \(n \in B_0\), we have \(n++ \in B_0\).
And \(\forall\ n \in A_0\) whenever \(n \in A_0\) and \(n \neq 0\), we have \(n++ \in A_0\) is vacuously true because \(\{n \in A_0 : n \neq 0\} = \emptyset\).
Now we need to show that \(A_0\) and \(B_0\) is unique, so assume that there exists another sets \(A_0'\) and \(B_0'\) such that the above condition holds.
Then \(0 \in A_0'\) according to condition (c), and \(A_0' = \{0\}\) according to condition (f), so \(A_0 = A_0'\).
And because \(A_0' \cap B_0' = \{0\} \cap B_0' = \emptyset\), and \(A_0' \cup B_0' = \mathbf{N}\), and condition (e) is true, so \(B_0' = \{n \in \mathbf{N} : n \neq 0\} = B_0\).
Thus for \(N = 0\), there exists a unique pair \(A_0\) and \(B_0\) such that the above conditions hold.
Suppose inductively that there exists a unique pair \(A_N\) and \(B_N\) such that the above conditions hold.
Then for \(N++\) we can define \(A_{N++}\) and \(B_{N++}\) by setting \(A_{N++} = A_N \cup \{N++\}\) and \(B_{N++} = \{n \in B_N : n \neq N++\} = B_N \setminus \{N++\}\).
Thus we can check the above conditions still hold.
For condition (a), \(A_{N++} \cap B_{N++} = (A_N \cup \{N++\}) \cap (B_N \setminus \{N++\})\).
For all \(n \in A_{N++}\), \(n \in A_N\) or \(n = N++\).
If \(n \in A_N\), then \(n \notin B_N\) because \(A_N \cap B_N = \emptyset\) by induction hypothesis, so \(n \notin B_N \setminus \{N++\}\).
If \(n = N++\), then \(n \notin B_N \setminus \{N++\}\).
Thus \(A_{N++} \cap B_{N++} = \emptyset\), and condition (a) is true for \(A_{N++}\) and \(B_{N++}\).
For condition (b), \(A_{N++} \cup B_{N++} = (A_N \cup \{N++\}) \cup (B_N \setminus \{N++\}) = (A_N \cup (B_N \setminus \{N++\})) \cup (\{N++\} \cup (B_N \setminus \{N++\})) = (A_N \cup (B_N \setminus \{N++\})) \cup B_N = A_N \cup B_N = \mathbf{N}\) by induction hypothesis.
Thus condition (b) is true for \(A_{N++}\) and \(B_{N++}\).
For condition (c), \(0 \in A_{N++}\) because \(0 \in A_N\) and \(A_{N++} = A_N \cup \{N++\}\), so condition (c) is true for \(A_{N++}\).
For condition (d), we want to show that \((N++)++ \in B_N\).
By induction hypothesis, \(N \in A_N\) and \(N++ \in B_N\) is true, so \((N++)++ \in B_N\) is true.
Thus \((N++)++ \in B_N \setminus \{N++\} = B_{N++}\) is also true, so condition (e) is true for \(B_{N++}\).
For condition (f), since \(N \in A_N\) by induction hypothesis, \(N++ \in A_N \cup \{N++\} = A_{N++}\) is true.
And by (a) and (e), \((N++)++\) is not in \(A_N\).
So whenever \(n \in A_{N++}\) and \(n \neq N++\), we have \(n++ \in A_{N++}\).
Thus condition (f) is true for \(A_{N++}\).
Now we need to show that \(A_{N++}\) and \(B_{N++}\) is unique.
Because \(A_{N++} = A_N \cup \{N++\}\), and by induction hypothesis \(A_N\) is unique and \(\{N++\}\) is also unique, so \(A_{N++}\) is unique.
Similarly \(B_{N++}\) is unique.
Thus we close the induction, so for every \(N \in \mathbf{N}\) we have unique \(A_N\) and \(B_N\) such that the above condition is true.

Now we claim that for every natural number \(N \in \mathbf{N}\), there exists a unique function \(a_N : A_N \to \mathbf{N}\) such that \(a_N(0) = c\) and \(a_N(n++) = f(n, a_{N}(n))\) for all \(n \in \mathbf{N}\) such that \(n < N\).

With similar process of previous proof, we can show that the statement is true, and by defining \(a\) with similar argument we can show that \(a\) exist and is unique.
\end{proof}

\begin{exercise}\label{ex 3.5.13}
Suppose we have a set \(\mathbf{N}'\) of ``alternative natural numbers'', an ``alternative zero'' \(0'\), and an ``alternative increment operation'' which takes any alternative natural number \(n' \in N\) and returns another alternative natural number \(n'++' \in \mathbf{N}'\), such that the Peano axioms (Axioms \ref{2.1}-\ref{2.5}) all hold with the natural numbers, zero, and increment replaced by their alternative counterparts.
Show that there exists a bijection \(f : \mathbf{N} \to \mathbf{N}'\) from the natural numbers to the alternative natural numbers such that \(f(0) = 0'\), and such that for any \(n \in \mathbf{N}\) and \(n' \in \mathbf{N}'\), we have \(f(n) = n'\) if and only if \(f(n++) = n'++'\).
\end{exercise}

\begin{proof}
Define \(f: \mathbf{N} \to \mathbf{N}'\) be a function such that \(f(0) = 0'\) and \(f(n++) = f(n)++'\).
We need to show that such \(f\) exists and is bijective.

First we show that \(f\) exists, for which we need to show that \(\forall\ n \in \mathbf{N}\), there exists exactly one value \(f(n) \in \mathbf{N}'\).
And we use induction on \(n\).
For \(n = 0\), by the above definition \(f(0) = 0' \in \mathbf{N}'\), so the base case holds.
Suppose inductively that \(f(n) \in \mathbf{N}'\) and there is only one value for \(f(n)\) for some \(n\).
Then for \(n++\), we need to show that \(f(n++) \in \mathbf{N}'\) and there is only one value for \(f(n++)\).
By induction hypothesis, \(f(n) \in \mathbf{N}'\), and because by the given condition Axiom \ref{2.1}-\ref{2.5} all hold on \(\mathbf{N}'\), \(f(n)++' \in \mathbf{N}'\) and there is only one value for \(f(n)++'\).
So by the above definition \(f(n++) = f(n)++'\) exists and there is only one value for \(f(n++)\), this close the induction.

Next we show that \(f\) is injective.
Let \(a, b \in \mathbf{N}\) and \(f(a) = f(b)\).
We claim that \(f(a) = f(b) \implies a = b\), and we use induction on \(a\).
For \(a = 0\), \(f(a) = f(b) = f(0) = 0'\) by the above definition.
If \(b \neq 0\), then by Lemma \ref{2.2.10}, there exists exactly one natural number \(c\) such that \(c++ = b\).
So \(f(b) = f(c++) = f(c)++' = 0'\) by the above definition, but by Axiom \ref{2.3}, \(f(c)++' \neq 0'\), a contradiction.
Thus \(b = 0\), which means the base case holds.
Suppose inductively that \(f(a) = f(b) \implies a = b\) for some \(a\).
Then for \(a++\), we need to show that \(f(a++) = f(b++) \implies a++ = b++\).
By the above definition \(f(a++) = f(a)++' = f(b++) = f(b)++'\).
Because \(f(a)++' = f(b)++'\), by Axiom \ref{2.4}, \(f(a) = f(b)\).
By induction hypothesis, \(f(a) = f(b) \implies a = b\), thus \(a++ = b++\) by Axiom \ref{2.4}, and this close the induction.

Finally we show that \(f\) is surjective.
We need to show that \(\forall\ n' \in \mathbf{N}'\), \(\exists\ n \in \mathbf{N}\) such that \(f(n) = n'\).
We use induction on \(n'\).
For \(n' = 0'\), \(f(0) = 0'\) by the above definition, so the base case holds.
Suppose inductively that for some \(n' \in \mathbf{N}'\), \(\exists\ n \in \mathbf{N}\) such that \(f(n) = n'\).
Then for \(n'++'\), by induction hypothesis, \(n'++' = (n')++' = f(n)++'\)
By the above definition, \(f(n)++' = f(n++)\), this close the induction.
\end{proof}
\section{Cardinality of sets}\label{sec 3.6}

\begin{definition}[Equal cardinality]\label{3.6.1}
    We say that two sets \(X\) and \(Y\) have \emph{equal cardinality} iff there exists a bijection \(f : X \to Y\) from \(X\) to \(Y\).
\end{definition}

\setcounter{theorem}{2}
\begin{remark}\label{3.6.3}
    The fact that two sets have equal cardinality does not preclude one of the sets from containing the other.
    For instance, if \(X\) is the set of natural numbers and \(Y\) is the set of even natural numbers, then the map \(f : X \to Y\) defined by \(f(n) \coloneqq 2n\) is a bijection from \(X\) to \(Y\), and so \(X\) and \(Y\) have equal cardinality, despite \(Y\) being a subset of \(X\) and seeming intuitively as if it should only have ``half'' of the elements of \(X\).
\end{remark}

\begin{proposition}\label{3.6.4}
    Let \(X\), \(Y\), \(Z\) be sets.
    Then \(X\) has equal cardinality with \(X\).
    If \(X\) has equal cardinality with \(Y\), then \(Y\) has equal cardinality with \(X\).
    If \(X\) has equal cardinality with \(Y\) and \(Y\) has equal cardinality with \(Z\), then \(X\) has equal cardinality with \(Z\).
\end{proposition}

\begin{proof}
    We first show that Definition \ref{3.6.1} is reflexive.
    Suppose that \(X\) is a set.
    Let \(f : X \to X\) be a function where \(f = x \mapsto x\).
    By Axiom \ref{3.6} \(f\) is well-defined.
    Such \(f\) is injective since \(\forall\ x, x' \in X\), \(f(x) = f(x') \implies x = x'\), and \(f\) is also surjective since \(\forall\ x \in X\), \(\exists\ x \in X\) such that \(f(x) = x\).
    Thus \(f\) is bijective, and by Definition \ref{3.6.1} \(X\) has equal cardinality with \(X\).

    Next we show that Definition \ref{3.6.1} is symmetric.
    Suppose that \(X, Y\) are sets such that \(X\) has equal cardinality with \(Y\).
    Then by Definition \ref{3.6.1} there exists a function \(f : X \to Y\) such that \(f\) is bijective.
    Since \(f\) is bijective, by Exercise \ref{ex 3.3.6} \(f^{-1} : Y \to X\) is also bijective.
    Thus by Definition \ref{3.6.1} \(Y\) has equal cardinality with \(X\).

    Finally we show that Definition \ref{3.6.1} is transitive.
    Suppose that \(X, Y, Z\) are sets such that \(X\) has equal cardinality with \(Y\) and \(Y\) has equal cardinality with \(Z\).
    Then by Definition \ref{3.6.1} there exist two functions \(f : X \to Y\) and \(g : Y \to Z\) such that \(f\) and \(g\) are bijective.
    Since \(f\) and \(g\) are bijective, by Exercise \ref{ex 3.3.7} \(g \circ f : X \to Z\) is also bijective.
    Thus by Definition \ref{3.6.1} \(X\) has equal cardinality with \(Z\).
\end{proof}

\begin{definition}\label{3.6.5}
    Let \(n\) be a natural number.
    A set \(X\) is said to have \emph{cardinality} \(n\), iff it has equal cardinality with \(\{i \in \mathbf{N} : 1 \leq i \leq n\}\).
    We also say that \(X\) \emph{has \(n\) elements} iff it has cardinality \(n\).
\end{definition}

\begin{remark}\label{3.6.6}
    One can use the set \(\{i \in \mathbf{N} : i < n\}\) instead of \(\{i \in \mathbf{N} : 1 \leq i \leq n\}\), since these two sets clearly have equal cardinality.
\end{remark}

\setcounter{theorem}{7}
\begin{proposition}[Uniqueness of cardinality]\label{3.6.8}
    Let \(X\) be a set with some cardinality \(n\).
    Then \(X\) cannot have any other cardinality, i.e., \(X\) cannot have cardinality \(m\) for any \(m \neq n\).
\end{proposition}

\begin{proof}
    We induct on \(n\).
    First suppose that \(n = 0\).
    Then \(X\) must be empty, and so \(X\) cannot have any non-zero cardinality.
    Now suppose that the proposition is already proven for some \(n\);
    we now prove it for \(n++\).
    Let \(X\) have cardinality \(n++\);
    and suppose that \(X\) also has some other cardinality \(m \neq n++\).
    By Lemma \ref{3.6.9}, \(X\) is non-empty, and if \(x\) is any element of \(X\), then \(X \setminus \{x\}\) has cardinality \(n\) and also has cardinality \(p\), where \(p++ = m\), by Lemma \ref{3.6.9}.
    By induction hypothesis, this means that \(n = p\), which implies that \(p++ = m = n++\), a contradiction.
    This closes the induction.
\end{proof}

\begin{lemma}\label{3.6.9}
    Suppose that \(n \geq 1\), and \(X\) has cardinality \(n\).
    Then \(X\) is non-empty, and if \(x\) is any element of \(X\), then the set \(X \setminus \{x\}\) (i.e., \(X\) with the element \(x\) removed) has cardinality \(m\), where \(m++ = n\).
\end{lemma}

\begin{proof}
    If \(X\) is empty then it clearly cannot have the same cardinality as the non-empty set \(\{i \in \mathbf{N} : 1 \leq i \leq n\}\), as there is no bijection from the empty set to a non-empty set.
    Now let \(x\) be an element of \(X\).
    Since \(X\) has the same cardinality as \(\{i \in \mathbf{N} : 1 \leq i \leq n\}\), we thus have a bijection \(f\) from \(X\) to \(\{i \in \mathbf{N} : 1 \leq i \leq n\}\).
    In particular, \(f(x)\) is a natural number between \(1\) and \(n\).
    Now define the function \(g : X \setminus \{x\} \to \{i \in \mathbf{N} : 1 \leq i \leq m\}\) by the following rule: for any \(y \in X \setminus \{x\}\), we define \(g(y) \coloneqq f(y)\) if \(f(y) < f(x)\), and define \(g(y)++ \coloneqq f(y)\) if \(f(y) > f(x)\).
    (Note that \(f(y)\) cannot equal \(f(x)\) since \(y \neq x\) and \(f\) is a bijection.)
    It is easy to check that this map is also a bijection, and so \(X \setminus \{x\}\) has equal cardinality with \(\{i \in \mathbf{N} : 1 \leq i \leq m\}\).
    In particular \(X \setminus \{x\}\) has cardinality \(m\), as desired.
\end{proof}

\begin{definition}[Finite sets]\label{3.6.10}
    A set is \emph{finite} iff it has cardinality \(n\) for some natural number \(n\);
    otherwise, the set is called \emph{infinite}.
    If \(X\) is a finite set, we use \(\#(X)\) to denote the cardinality of \(X\).
\end{definition}

\setcounter{theorem}{11}
\begin{theorem}\label{3.6.12}
    The set of natural numbers \(\mathbf{N}\) is infinite.
\end{theorem}

\begin{proof}
    Suppose for sake of contradiction that the set of natural numbers \(\mathbf{N}\) was finite, so it had some cardinality \(\#(\mathbf{N}) = n\).
    Then there is a bijection \(f\) from \(\{i \in \mathbf{N} : 1 \leq i \leq n\}\) to \(\mathbf{N}\).
    One can show that the sequence \(f(1), f(2), \dots, f(n)\) is bounded, or more precisely that there exists a natural number \(M\) such that \(f(i) \leq M\) for all \(1 \leq i \leq n\) (Exercise \ref{ex 3.6.3}).
    But then the natural number \(M+1\) is not equal to any of the \(f(i)\), contradicting the hypothesis that \(f\) is a bijection.
\end{proof}

\begin{remark}\label{3.6.13}
    One can also use similar arguments to show that any unbounded set is infinite;
    for instance the rationals \(\mathbf{Q}\) and the reals \(\mathbf{R}\) are infinite.
    However, it is possible for some sets to be ``more'' infinite than others.
\end{remark}

\begin{proposition}[Cardinal arithmetic]\label{3.6.14}
    \leavevmode
    \begin{enumerate}
        \item Let \(X\) be a finite set, and let \(x\) be an object which is not an element of \(X\).
              Then \(X \cup \{x\}\) is finite and \(\#(X \cup \{x\}) = \#(X) + 1\).
        \item Let \(X\) and \(Y\) be finite sets.
              Then \(X \cup Y\) is finite and \(\#(X \cup Y) \leq \#(X) + \#(Y)\).
              If in addition \(X\) and \(Y\) are disjoint (i.e., \(X \cap Y = \emptyset\)), then \(\#(X \cup Y) = \#(X) + \#(Y)\).
        \item Let \(X\) be a finite set, and let \(Y\) be a subset of \(X\).
              Then \(Y\) is finite, and \(\#(Y) \leq \#(X)\).
              If in addition \(Y \neq X\) (i.e., \(Y\) is a proper subset of \(X\)), then we have \(\#(Y) < \#(X)\).
        \item If \(X\) is a finite set, and \(f : X \to Y\) is a function, then \(f(X)\) is a finite set with \(\#(f(X)) \leq \#(X)\).
              If in addition \(f\) is one-to-one, then \(\#(f(X)) = \#(X)\).
        \item Let \(X\) and \(Y\) be finite sets.
              Then Cartesian product \(X \times Y\) is finite and \(\#(X \times Y) = \#(X) \times \#(Y)\).
        \item Let \(X\) and \(Y\) be finite sets.
              Then the set \(Y^X\) (defined in Axiom \ref{3.10}) is finite and \(\#(Y^X) = \#(Y)^{\#(X)}\).
    \end{enumerate}
\end{proposition}

\begin{proof}{(a)}
    Suppose that \(X\) is a finite set and \(x \notin X\).
    By Definition \ref{3.6.10} \(\exists\ n \in \mathbf{N}\) such that \(\#(X) = n\).
    By Definition \ref{3.6.5} \(\exists\ f : X \to \{i \in \mathbf{N} : 1 \leq i \leq n\}\) such that \(f\) is bijective.
    Now we define a function \(g : X \cup \{x\} \to \{i \in \mathbf{N} : 1 \leq i \leq n + 1\}\) as follow:
    \[
        g(y) = \begin{cases}
            f(y)  & \text{if } y \in X \\
            n + 1 & \text{otherwise}
        \end{cases}
    \]

    Now we need to show that \(g\) is bijective.
    Since \(f\) is bijective, \(\forall\ i \in \{i \in \mathbf{N} : 1 \leq i \leq n\}, \exists\ y \in X\) such that \(f(y) = i\).
    With that and \(g(x) = n + 1\) we thus have \(g\) is surjective.
    For any \(y, y' \in X \cup \{x\}\), if \(g(y) = g(y')\), then we have two cases:
    \begin{itemize}
        \item If \(g(y) \in X\), then \(y = y'\) since \(f\) is bijective.
        \item If \(g(y) = x\), then \(y = y' = x\) by definition of \(g\).
    \end{itemize}
    For all cases above we have \(g(y) = g(y') \implies y = y'\), thus \(g\) is injective.
    Since \(g\) is bijective, we have \(\#(X \cup \{x\}) = n + 1 = \#(X) + 1\).
\end{proof}

\begin{proof}{(b)}
    Suppose that \(X, Y\) are finite sets.
    If \(X = \emptyset \lor X = Y\), then \(X \cup Y = Y\) is finite.
    Similarly if \(Y = \emptyset \lor Y = X\), then \(X \cup Y = X\) is finite.
    Thus assume that \(X \neq \emptyset \land Y \neq \emptyset\) and \(X, Y\) are distinct.
    By Definition \ref{3.6.10} \(\exists\ n \in \mathbf{N}\) such that \(\#(X) = n\).
    We use induction on \(n\) to show that \(X \cup Y\) is finite and \(\#(X \cup Y) \leq \#(X) + \#(Y)\).
    We start with \(n = 1\) since \(X \neq \emptyset\).
    For \(n = 1\), we have
    \begin{align*}
        \#(X \cup Y) & = \#(Y) + 1         & \text{(by Proposition \ref{3.6.14}(a))} \\
                     & = \#(X) + \#(Y)     & \text{(by Definition \ref{3.6.5})}      \\
                     & \leq \#(X) + \#(Y).
    \end{align*}
    Thus \(X \cup Y\) is finite and the base case holds.
    Suppose inductively that the statement is true for some \(\#(X) = n\).
    We show that the statement is still true for \(\#(X) = n++\).
    Let \(x \in X\).
    By Lemma \ref{3.6.9} we have \(\#(X \setminus \{x\}) = n\).
    If \(x \in Y\), then we have
    \begin{align*}
        \#(X \cup Y) & = \#((X \setminus \{x\}) \cup \{x\} \cup Y)   & \text{(by Proposition \ref{3.1.28}(g))} \\
                     & = \#((X \setminus \{x\}) \cup (\{x\} \cup Y)) & \text{(by Proposition \ref{3.1.28}(e))} \\
                     & = \#((X \setminus \{x\}) \cup Y)                                                        \\
                     & \leq \#(X \setminus \{x\}) + \#(Y)            & \text{(by induction hypothesis)}        \\
                     & < \#(X) + \#(Y).
    \end{align*}
    If \(x \notin Y\), then we have
    \begin{align*}
        \#(X \cup Y) & = \#((X \setminus \{x\}) \cup \{x\} \cup Y)  & \text{(by Proposition \ref{3.1.28}(g))}    \\
                     & = \#((X \setminus \{x\}) \cup Y \cup \{x\})  & \text{(by Proposition \ref{3.1.28}(d)(e))} \\
                     & = \#((X \setminus \{x\}) \cup Y) + 1         & \text{(by Proposition \ref{3.6.14}(a))}    \\
                     & \leq \#(X \setminus \{x\}) + \#(Y) + 1       & \text{(by induction hypothesis)}           \\
                     & = \#((X \setminus \{x\}) \cup \{x\}) + \#(Y) & \text{(by Proposition \ref{3.6.14}(a))}    \\
                     & = \#(X) + \#(Y).                             & \text{(by Proposition \ref{3.1.28}(g))}
    \end{align*}
    In either cases we have \(\#(X \cup Y) \leq \#(X) + \#(Y)\).
    Thus \(X \cup Y\) is finite and this closes the induction.

    Now suppose that \(X, Y\) are finite sets and \(X \cap Y = \emptyset\).
    By Definition \ref{3.6.10} \(\exists\ n \in \mathbf{N}\) such that \(\#(X) = n\).
    From proof above we already know that \(\#(X \cup Y) \leq \#(X) + \#(Y)\).
    We now use induction on \(n\) to show that \(\#(X \cup Y) = \#(X) + \#(Y)\).
    For \(n = 0\), we have
    \begin{align*}
        \#(X \cup Y) & = \#(Y)          & \text{(by Proposition \ref{3.1.28}(a))} \\
                     & = 0 + \#(Y)                                                \\
                     & = \#(X) + \#(Y). & \text{(by Definition \ref{3.6.5})}
    \end{align*}
    Thus the base case holds.
    Suppose inductively that the statement is true for some \(\#(X) = n\).
    We show that the statement is still true for \(\#(X) = n++\).
    Let \(x \in X\).
    By Lemma \ref{3.6.9} we have \(\#(X \setminus \{x\}) = n\).
    Since \(X \cap Y = \emptyset\), \(x \notin Y\).
    So we have
    \begin{align*}
        \#(X \cup Y) & = \#((X \setminus \{x\}) \cup \{x\} \cup Y)  & \text{(by Proposition \ref{3.1.28}(g))}    \\
                     & = \#((X \setminus \{x\}) \cup Y \cup \{x\})  & \text{(by Proposition \ref{3.1.28}(d)(e))} \\
                     & = \#((X \setminus \{x\}) \cup Y) + 1         & \text{(by Proposition \ref{3.6.14}(a))}    \\
                     & = \#(X \setminus \{x\}) + \#(Y) + 1          & \text{(by induction hypothesis)}           \\
                     & = \#((X \setminus \{x\}) \cup \{x\}) + \#(Y) & \text{(by Proposition \ref{3.6.14}(a))}    \\
                     & = \#(X) + \#(Y).                             & \text{(by Proposition \ref{3.1.28}(g))}
    \end{align*}
    This closes the induction.
\end{proof}

\begin{proof}{(c)}
    Suppose that \(X\) is a finite sets.
    By Definition \ref{3.6.10} \(\exists\ n \in \mathbf{N}\) such that \(\#(X) = n\).
    We use induction on \(n\) to show that \(\forall\ Y \subseteq X\), \(Y\) is finite and \(\#(Y) \leq \#(X)\).
    For \(n = 0\), we have
    \begin{align*}
                 & \forall\ Y \subseteq \emptyset                                      \\
        \implies & Y = \emptyset                  & \text{(by Axiom \ref{3.2})}        \\
        \implies & \#(Y) = 0 \leq 0 = \#(X).      & \text{(by Definition \ref{3.6.5})}
    \end{align*}
    Thus the base case holds.
    Suppose inductively that the statement is true for some \(\#(X) = n\).
    We show that the statement is still true for \(\#(X) = n++\).
    Let \(Y \subseteq X\).
    If \(Y = X\), then by Proposition \ref{3.6.8} \(\#(Y) = \#(X)\).
    If \(Y \neq X\), then \(\exists\ x \in X \setminus Y\) such that \(Y \subseteq X \setminus \{x\}\) and
    \begin{align*}
        \#(X) & = \#((X \setminus \{x\}) \cup \{x\}) & \text{(by Proposition \ref{3.1.28}(g))} \\
              & = \#(X \setminus \{x\}) + 1          & \text{(by Proposition \ref{3.6.14}(a))} \\
              & \geq \#(Y) + 1                       & \text{(by induction hypothesis)}        \\
              & > \#(Y).
    \end{align*}
    From all cases above we have \(\forall\ Y \subseteq X \implies \#(Y) \leq \#(X)\).
    Thus \(Y\) is finite and this closes the induction.

    We now use induction on \(n\) to show that \(\forall\ Y \subseteq X : Y \neq X \implies \#(Y) < \#(X)\).
    We start with \(n = 1\) since for \(n = 0\) we have \(X = \emptyset\) and \(\nexists\ Y : Y \subseteq X \land Y \neq X\).
    For \(n = 1\), we have
    \begin{align*}
                 & \forall\ Y : Y \subseteq X \land Y \neq X                                      \\
        \implies & Y = \emptyset                             & \text{(by Axiom \ref{3.3})}        \\
        \implies & \#(Y) = 0 < 1 = \#(X).                    & \text{(by Definition \ref{3.6.5})}
    \end{align*}
    Thus the base case holds.
    Suppose inductively that the statement is true for some \(\#(X) = n\).
    We show that the statement is still true for \(\#(X) = n++\).
    Let \(Y \subseteq X \land Y \neq X\).
    If \(Y = \emptyset\), then \(\#(Y) = 0 < n++ = \#(X)\).
    If \(Y \neq \emptyset\), then \(\exists\ x \in Y\) such that
    \begin{align*}
                 & \#(Y \setminus \{x\}) < \#(X \setminus \{x\})                           & \text{(by induction hypothesis)}        \\
        \implies & \#(Y \setminus \{x\}) + 1 < \#(X \setminus \{x\}) + 1                                                             \\
        \implies & \#((Y \setminus \{x\}) \cup \{x\}) < \#((X \setminus \{x\}) \cup \{x\}) & \text{(by Proposition \ref{3.6.14}(a))} \\
        \implies & \#(Y) < \#(X).                                                          & \text{(by Proposition \ref{3.1.28}(g))}
    \end{align*}
    This closes the induction.
\end{proof}

\begin{proof}{(d)}
    Suppose that \(X\) is a finite sets.
    By Definition \ref{3.6.10} \(\exists\ n \in \mathbf{N}\) such that \(\#(X) = n\).
    We use induction on \(n\) to show that for any set \(Y\) and any function \(f : X \to Y\) we have \(\#(f(X)) \leq \#(X)\).
    For \(n = 0\), let \(Y\) be arbitrary set and \(f : X \to Y\) be arbitrary function.
    Then we have
    \begin{align*}
                 & X = \emptyset                                                     \\
        \implies & f(X) = \emptyset             & \text{(by Definition \ref{3.4.1})} \\
        \implies & \#(f(X)) = 0 \leq 0 = \#(X). & \text{(by Definition \ref{3.6.5})}
    \end{align*}
    Thus the base case holds.
    Suppose inductively that the statement is true for some \(\#(X) = n\).
    We show that the statement is still true for \(\#(X) = n++\).
    Let \(x \in X\), \(Y\) be arbitrary set and \(f : X \to Y\) be arbitrary function.
    If \(f(X \setminus \{x\}) = f(X)\), then we have
    \begin{align*}
                 & \#(f(X \setminus \{x\})) \leq \#(X \setminus \{x\}) & \text{(by induction hypothesis)}        \\
        \implies & \#(f(X)) \leq \#(X \setminus \{x\})                                                           \\
        \implies & \#(f(X)) < \#(X \setminus \{x\}) + 1                                                          \\
        \implies & \#(f(X)) < \#((X \setminus \{x\}) \cup \{x\})       & \text{(by Proposition \ref{3.6.14}(a))} \\
        \implies & \#(f(X)) < \#(X).                                   & \text{(by Proposition \ref{3.1.28}(g))}
    \end{align*}
    If \(f(X \setminus \{x\}) \neq f(X)\), then we have
    \begin{align*}
        \#(f(X)) & = \#(f(X \setminus \{x\}) \cup \{f(x)\}) & \text{(by Exercise \ref{ex 3.4.3})}     \\
                 & = \#(f(X \setminus \{x\})) + 1           & \text{(by Proposition \ref{3.6.14}(a))} \\
                 & \leq \#(X \setminus \{x\}) + 1           & \text{(by induction hypothesis)}        \\
                 & = \#((X \setminus \{x\}) \cup \{x\})     & \text{(by Proposition \ref{3.6.14}(a))} \\
                 & = \#(X).                                 & \text{(by Proposition \ref{3.1.28}(g))}
    \end{align*}
    From all cases above we have \(\#(f(X)) \leq \#(X)\).
    This closes the induction.

    We now use induction on \(n\) to show that for any set \(Y\) and any one-to-one function \(f : X \to Y\) we have \(\#(f(X)) = \#(X)\).
    For \(n = 0\), let \(Y\) be arbitrary set and \(f : X \to Y\) be arbitrary one-to-one function.
    Then we have
    \begin{align*}
                 & X = \emptyset                                              \\
        \implies & f(X) = \emptyset      & \text{(by Definition \ref{3.4.1})} \\
        \implies & \#(f(X)) = 0 = \#(X). & \text{(by Definition \ref{3.6.5})}
    \end{align*}
    Thus the base case holds.
    Suppose inductively that the statement is true for some \(\#(X) = n\).
    We show that the statement is still true for \(\#(X) = n++\).
    Let \(x \in X\), \(Y\) be arbitrary set and \(f : X \to Y\) be arbitrary one-to-one function.
    Since \(f\) is one-to-one, we must have \(f(X \setminus \{x\}) \neq f(X)\) and
    \begin{align*}
        \#(f(X)) & = \#(f(X \setminus \{x\}) \cup \{f(x)\}) & \text{(by Exercise \ref{ex 3.4.3})}     \\
                 & = \#(f(X \setminus \{x\})) + 1           & \text{(by Proposition \ref{3.6.14}(a))} \\
                 & = \#(X \setminus \{x\}) + 1              & \text{(by induction hypothesis)}        \\
                 & = \#((X \setminus \{x\}) \cup \{x\})     & \text{(by Proposition \ref{3.6.14}(a))} \\
                 & = \#(X).                                 & \text{(by Proposition \ref{3.1.28}(g))}
    \end{align*}
    This closes the induction.
\end{proof}

\begin{proof}{(e)}
    Suppose that \(X, Y\) are finite sets.
    We first show that \(\forall\ x : \#(\{x\} \times Y) = \#(Y)\).
    By Definition \ref{3.6.1}, we only need to find a function \(f : \{x\} \times Y \to Y\) such that \(f\) is bijective.
    We now define \(f : \{x\} \times Y \to Y\) as \(f(x', y) = y\).
    We need to show that \(f\) is bijective.
    We start by showing \(f\) is injective.
    \begin{align*}
                 & \forall\ (x_1, y_1), (x_2, y_2) \in \{x\} \times Y : f(x_1, y_1) = f(x_2, y_2)                                      \\
        \implies & x_1 = x_2 \land y_1 = y_2                                                      & \text{(by Axiom \ref{3.3})}        \\
        \implies & (x_1, y_1) = (x_2, y_2).                                                       & \text{(by Definition \ref{3.5.1})}
    \end{align*}
    Thus \(f\) is injective.
    Now we show that \(f\) is surjective.
    \begin{align*}
                 & \forall\ y \in Y                                                                       \\
        \implies & (x, y) \in \{x\} \times Y                         & \text{(by Definition \ref{3.5.4})} \\
        \implies & \exists\ (x, y) \in \{x\} \times Y : f(x, y) = y.
    \end{align*}
    Thus \(f\) is surjective.
    Since \(f\) is both injective and surjective, \(f\) is bijective and thus by Definition \ref{3.6.1} we have \(\#(\{x\} \times Y) = \#(Y)\).

    Now we show that \(\#(X \times Y) = \#(X) \times \#(Y)\).
    By Definition \ref{3.6.10} \(\exists\ n \in \mathbf{N}\) such that \(\#(X) = n\).
    We use induction on \(n\) to show that \(\#(X \times Y) = \#(X) \times \#(Y)\).
    For \(n = 0\), we have
    \begin{align*}
        \#(X \times Y) & = \#(\emptyset \times Y) & \text{(by Definition \ref{3.6.5})} \\
                       & = \#(\emptyset)          & \text{(by Definition \ref{3.5.4})} \\
                       & = 0                      & \text{(by Definition \ref{3.6.5})} \\
                       & = \#(X) \times \#(Y).
    \end{align*}
    Thus the base case holds.
    Suppose inductively that the statement is true for some \(\#(X) = n\).
    We show that the statement is still true for \(\#(X) = n++\).
    Let \(x \in X\).
    Then we have
    \begin{align*}
        \#(X \times Y) & = \#\bigg(\big((X \setminus \{x\}) \cup \{x\}\big) \times Y\bigg)                                           \\
                       & = \#((X \setminus \{x\}) \times Y \cup \{x\} \times Y)            & \text{(by Exercise \ref{ex 3.5.4})}     \\
                       & = \#((X \setminus \{x\}) \times Y) + \#(\{x\} \times Y)           & \text{(by Proposition \ref{3.6.14}(b))} \\
                       & = \#(X \setminus \{x\}) \times \#(Y) + \#(\{x\} \times Y)         & \text{(by induction hypothesis)}        \\
                       & = \#(X \setminus \{x\}) \times \#(Y) + \#(Y)                      & \text{(from proof above)}               \\
                       & = (\#(X \setminus \{x\}) + 1) \times \#(Y)                                                                  \\
                       & = \#((X \setminus \{x\}) \cup \{x\}) \times \#(Y)                 & \text{(by Proposition \ref{3.6.14}(a))} \\
                       & = \#(X) \times \#(Y).                                             & \text{(by Proposition \ref{3.1.28}(g))}
    \end{align*}
    This closes the induction.
\end{proof}

\begin{proof}{(f)}
    Suppose that \(X, Y\) are finite sets.
    We first show that \(\forall\ x : \#(Y^{\{x\}}) = \#(Y)\).
    We define a function \(f : Y^{\{x\}} \to Y\) by setting \(\forall\ g \in Y^{\{x\}} : f(g) = g(x)\).
    We now show that \(f\) is bijective.
    We start by showing \(f\) is injective.
    \begin{align*}
                 & \forall\ g, g' \in Y^{\{x\}} : f(g) = f(g')                                      \\
        \implies & g(x) = g'(x)                                                                     \\
        \implies & \forall\ x' \in \{x\} : g(x') = g'(x').     & \text{(by Axiom \ref{3.3})}        \\
        \implies & g = g'.                                     & \text{(by Definition \ref{3.3.7})}
    \end{align*}
    Thus \(f\) is injective.
    Now we show that \(f\) is surjective.
    \begin{align*}
                 & \forall\ y \in Y, \exists\ (g : \{x\} \to Y) : g(x) = y & \text{(by Axiom \ref{3.6})}  \\
        \implies & g \in Y^{\{x\}}.                                        & \text{(by Axiom \ref{3.10})}
    \end{align*}
    Thus \(f\) is surjective.
    Since \(f\) is both injective and surjective, \(f\) is bijective and thus by Definition \ref{3.6.1} we have \(\#(Y^{\{x\}}) = \#(Y)\).

    Now we show that \(\#(Y^X) = \#(Y)^{\#(X)}\).
    By Definition \ref{3.6.10} \(\exists\ n \in \mathbf{N}\) such that \(\#(X) = n\).
    We use induction on \(n\) to show that \(\#(Y^X) = \#(Y)^{\#(X)}\).
    For \(n = 0\), by Definition \ref{3.6.5} we have \(X = \emptyset\) and
    \[
        \forall\ f, f' \in Y^\emptyset, \forall\ x \in \emptyset : f(x) = f'(x).
    \]
    Thus by Axiom \ref{3.3} \(Y^\emptyset\) is a singleton set.
    We can construct a bijection \(g : \{i \in \mathbf{N} : 1 \leq i \leq 1\} \to Y^\emptyset\) and thus by Definition \ref{3.6.5} \(\#(Y^\emptyset) = 1\).
    Again by Definition \ref{3.6.5} we have \(\#(X) = 0\), and thus by Definition \ref{2.3.11} we have \(\#(Y)^0 = 1\).
    So the base case holds.

    Suppose inductively that the statement is true for some \(\#(X) = n\).
    We show that the statement is still true for \(\#(X) = n++\).
    Let \(x \in X\).
    We define a function \(h : Y^X \to Y^{X \setminus \{x\}} \times Y^{\{x\}}\) as follow:
    \[
        \forall\ f \in Y^X : h(f) = \bigg(g : X \setminus \{x\} \to f(X \setminus \{x\}), g' : \{x\} \to f(\{x\})\bigg),
    \]
    where \(\forall\ x' \in X \setminus \{x\} : g(x') = f(x')\).
    We show that such \(h\) is bijective.
    We start by showing \(h\) is injective.
    \begin{align*}
                 & \forall\ f_1, f_2 \in Y^X : h(f_1) = h(f_2)                                                              \\
        \implies & (g_{f_1}, g_{f_1}') = (g_{f_2}, g_{f_2}')                                                                \\
        \implies & g_{f_1} = g_{f_2} \land g_{f_2} = g_{f_2}'                     & \text{(by Definition \ref{3.5.1})}      \\
        \implies & (\forall\ x' \in X \setminus \{x\} : g_{f_1}(x') = g_{f_2}(x') & \text{(by Definition \ref{3.3.7})}      \\
                 & \land (\forall\ x' \in \{x\} : g_{f_1}'(x') = g_{f_2}'(x')                                               \\
        \implies & (\forall\ x' \in X \setminus \{x\} : f_1(x') = f_2(x')                                                   \\
                 & \land (\forall\ x' \in \{x\} : f_1(x') = f_2(x')                                                         \\
        \implies & \forall\ x' \in X : f_1(x') = f_2(x')                          & \text{(by Proposition \ref{3.1.28}(g))} \\
        \implies & f_1 = f_2.                                                     & \text{(by Definition \ref{3.3.7})}
    \end{align*}
    Thus \(h\) is injective.
    Now we show that \(h\) is surjective.
    \(\forall\ (g, g') \in Y^{X \setminus \{x\}} \times Y^{\{x\}}\), we define a function \(k : X \to Y\) as follow:
    \[
        \forall\ x' \in X : k(x') = \begin{cases}
            g(x')  & \text{if } x' \in X \setminus \{x\} \\
            g'(x') & \text{if } x' \in \{x\}
        \end{cases}
    \]
    Then \(k \in Y^X\).
    Thus \(h\) is surjective.
    Since \(h\) is both injective and surjective, \(h\) is bijective, and we have \(\#(Y^X) = \#(Y^{(X \setminus \{x\})} \times Y^{\{x\}})\).
    We now finish our induction as follow:
    \begin{align*}
        \#(Y^X) & = \#(Y^{(X \setminus \{x\})} \times Y^{\{x\}})       & \text{(by proof above)}                 \\
                & = \#(Y^{(X \setminus \{x\})}) \times \#(Y^{\{x\}})   & \text{(by Proposition \ref{3.6.14}(e))} \\
                & = \#(Y)^{\#(X \setminus \{x\})} \times \#(Y^{\{x\}}) & \text{(by induction hypothesis)}        \\
                & = \#(Y)^{\#(X \setminus \{x\})} \times \#(Y)         & \text{(by proof above)}                 \\
                & = \#(Y)^{\#(X \setminus \{x\}) + 1}                  & \text{(by Definition \ref{2.3.11})}     \\
                & = \#(Y)^{\#((X \setminus \{x\}) \cup \{x\})}         & \text{(by Proposition \ref{3.6.14}(a))} \\
                & = \#(Y)^{\#(X)}.                                     & \text{(by Proposition \ref{3.1.28}(g))}
    \end{align*}
    This closes the induction.
\end{proof}

\begin{remark}\label{3.6.15}
    Proposition \ref{3.6.14} suggests that there is another way to define the arithmetic operations of natural numbers;
    not defined recursively as in Definitions \ref{2.2.1}, \ref{2.3.1}, \ref{2.3.11}, but instead using the notions of union, Cartesian product, and power set.
    This is the basis of \emph{cardinal arithmetic}, which is an alternative foundation to arithmetic than the Peano arithmetic we have developed here.
\end{remark}

\exercisesection

\begin{exercise}\label{ex 3.6.1}
    Prove Proposition \ref{3.6.4}.
\end{exercise}

\begin{proof}
    See Proposition \ref{3.6.4}.
\end{proof}

\begin{exercise}\label{ex 3.6.2}
    Show that a set \(X\) has cardinality \(0\) if and only if \(X\) is the empty set.
\end{exercise}

\begin{proof}
    \begin{align*}
             & \#(X) = 0                                                                                                                   \\
        \iff & \exists\ f : X \to \{i \in \mathbf{N} : 1 \leq i \leq 0\} \land f \text{ is bijective} & \text{(by Definition \ref{3.6.5})} \\
        \iff & \exists\ f : X \to \emptyset                                                           & \text{(by Axiom \ref{3.2})}        \\
        \iff & X = \emptyset.                                                                         & \text{(by Axiom \ref{3.6})}
    \end{align*}
\end{proof}

\begin{exercise}\label{ex 3.6.3}
    Let \(n\) be a natural number, and let \(f : \{i \in \mathbf{N} : 1 \leq i \leq n\} \to \mathbf{N}\) be a function.
    Show that there exists a natural number \(M\) such that \(f(i) \leq M\) for all \(1 \leq i \leq n\).
    Thus finite subsets of the natural numbers are bounded.
\end{exercise}

\begin{proof}
    Suppose that \(n \in \mathbf{N}\).
    We use induction on \(n\) to show that for any function \(f : \{i \in \mathbf{N} : 1 \leq i \leq n\} \to \mathbf{N}\), \(\exists\ M \in \mathbf{N}\) such that \(f(i) \leq M\).
    For \(n = 0\), for any function \(f : \{i \in \mathbf{N} : 1 \leq i \leq 0\} \to \mathbf{N}\) we have
    \begin{align*}
                 & f : \{i \in \mathbf{N} : 1 \leq i \leq 0\} \to \mathbf{N}                                        \\
        \implies & f : \emptyset \to \mathbf{N}                                       & \text{(by Axiom \ref{3.2})} \\
        \implies & \forall\ M \in \mathbf{N}, \forall\ i \in \emptyset : f(i) \leq M. & \text{(trivially true)}     \\
    \end{align*}
    Thus the base case holds.
    Suppose inductively that for some \(n\) the statement is true.
    Then for \(n++\), for any function \(f : \{i \in \mathbf{N} : 1 \leq i \leq n++\} \to \mathbf{N}\) we have
    \begin{itemize}
        \item By induction hypothesis, \(\exists\ M \in \mathbf{N}\) such that \(f(\{i \in \mathbf{N} : 1 \leq i \leq n\}) \leq M\).
        \item By Proposition \ref{2.2.13}, exactly one of \(M < f(n++)\), \(M = f(n++)\) or \(M > f(n++)\) is true.
    \end{itemize}
    If \(f(n++) \leq M\), then we have \(\forall\ i \in \{i \in \mathbf{N} : 1 \leq i \leq n++\} : f(i) \leq M\).
    If \(f(n++) > M\), then we can set \(M' = f(n++)\) and thus \(\forall\ i \in \{i \in \mathbf{N} : 1 \leq i \leq n++\} : f(i) \leq M'\).
    In all cases above we can conclude that \(\exists\ M \in \mathbf{N}\) such that \(\forall\ i \in \{i \in \mathbf{N} : 1 \leq i \leq n++\} : f(i) \leq M\).
    This closes the induction.
\end{proof}

\begin{exercise}\label{ex 3.6.4}
    Prove Proposition \ref{3.6.14}.
\end{exercise}

\begin{proof}
    See Proposition \ref{3.6.14}.
\end{proof}

\begin{exercise}\label{ex 3.6.5}
    Let \(A\) and \(B\) be sets.
    Show that \(A \times B\) and \(B \times A\) have equal cardinality by constructing an explicit bijection between the two sets.
    Then use Proposition \ref{3.6.14} to conclude an alternate proof of Lemma \ref{2.3.2}.
\end{exercise}

\begin{proof}
    Suppose that \(A, B\) are sets.
    By Definition \ref{3.5.4} we have \(A \times B, B \times A\) are sets.
    We define a function \(f : A \times B \to B \times A\) by setting \(\forall\ (a, b) \in A \times B : f(a, b) = (b, a)\).
    We now show that such \(f\) is bijective.
    We start by showing \(f\) is injective.
    \begin{align*}
                 & \forall\ (a, b), (a', b') \in A \times B : f(a, b) = f(a', b')                                      \\
        \implies & (b, a) = (b', a')                                                                                   \\
        \implies & b = b' \land a = a'                                            & \text{(by Definition \ref{3.5.1})} \\
        \implies & (a, b) = (a', b').                                             & \text{(by Definition \ref{3.5.1})}
    \end{align*}
    Thus \(f\) is injective.
    Now we show that \(f\) is surjective.
    This is true since
    \[
        \forall\ (b, a) \in B \times A, \exists\ (a, b) \in A \times B : f(a, b) = (b, a).
    \]
    Thus \(f\) is surjective.
    Since \(f\) is both injective and surjective, we conclude that \(f\) is bijective.
    Since \(f\) is bijective, by Definition \ref{3.6.1} we conclude that \(A \times B\) and \(B \times A\) have same cardinality.

    Now suppose that \(A, B\) are two finite set.
    By Definition \ref{3.6.5}, \(\exists\ n, m \in \mathbf{N}\) such that \(\#(A) = n \land \#(B) = m\).
    Then we have
    \begin{align*}
        \#(A \times B) & = \#(A) \times \#(B) & \text{(by Proposition \ref{3.6.14}(e))} \\
                       & = n \times m                                                   \\
                       & = \#(B \times A)     & \text{(by proof above)}                 \\
                       & = \#(B) \times \#(A) & \text{(by Proposition \ref{3.6.14}(e))} \\
                       & = m \times n.
    \end{align*}
    Thus Lemma \ref{2.3.2} is true.
\end{proof}

\begin{exercise}\label{ex 3.6.6}
    Let \(A, B, C\) be sets.
    Show that the sets \((A^B)^C\) and \(A^{B \times C}\) have equal cardinality by constructing an explicit bijection between the two sets.
    Conclude that \((a^b)^c = a^{bc}\) for any natural numbers \(a, b, c\).
    Use a similar argument to also conclude \(a^b \times a^c = a^{b+c}\).
\end{exercise}

\begin{proof}
    We first show that \((A^B)^C\) and \(A^{B \times C}\) have equal cardinality.
    Suppose that \(A, B, C\) are sets.
    By Definition \ref{3.5.4}, \(B \times C\) is a set.
    By Axiom \ref{3.10} \(A^B, (A^B)^C, A^{B \times C}\) are sets.
    We define a function \(f : (A^B)^C \to A^{B \times C}\) by setting \(\big(f(g)\big)(b, c) = \big(g(c)\big)(b)\) where \(b \in B\), \(c \in C\) and \(g : C \to A^B\).
    We now show that \(f\) is bijective.
    We start by showing that \(f\) is injective.
    \begin{align*}
                 & \forall\ h, h' \in (A^B)^C : f(h) = f(h')                                                                          \\
        \implies & \forall\ (b, c) \in B \times C : \big(f(h)\big)(b, c) = \big(f(h')\big)(b, c) & \text{(by Definition \ref{3.3.7})} \\
        \implies & \forall\ (b, c) \in B \times C : \big(h(c)\big)(b) = \big(h'(c)\big)(b)                                            \\
        \implies & \forall\ c \in C : h(c) = h'(c)                                               & \text{(by Definition \ref{3.3.7})} \\
        \implies & h = h'.                                                                       & \text{(by Definition \ref{3.3.7})}
    \end{align*}
    Thus \(f\) is injective.
    We now show that \(f\) is surjective.
    \(\forall\ h \in A^{B \times C}\), we define a function \(k : C \to A^B\) by setting \(h(b, c) = (k(c))(b)\) where \(b \in B\) and \(c \in C\).
    Then \(k \in (A^B)^C\) and thus \(f\) is surjective.
    Since \(f\) is both injective and surjective, we conclude that \(f\) is bijective.
    Since \(f\) is bijective, by Definition \ref{3.6.1} we conclude that \((A^B)^C\) and \(A^{B \times C}\) have same cardinality.

    Now we show that \(\forall\ a, b, c \in \mathbf{N} : (a^b)^c = a^{bc}\).
    Suppose that \(A, B, C\) are finite set.
    By Definition \ref{3.6.5}, \(\exists\ a, b, c \in \mathbf{N}\) such that \(\#(A) = a \land \#(B) = b \land \#(C) = c\).
    Then we have
    \begin{align*}
        \#((A^B)^C) & = \#(A^B)^{\#(C)}            & \text{(by Proposition \ref{3.6.14}(f))} \\
                    & = \#(A^B)^c                                                            \\
                    & = (\#(A)^{\#(B)})^c          & \text{(by Proposition \ref{3.6.14}(f))} \\
                    & = (a^b)^c                                                              \\
                    & = \#(A^{B \times C})         & \text{(by proof above)}                 \\
                    & = \#(A)^{\#(B \times C)}     & \text{(by Proposition \ref{3.6.14}(f))} \\
                    & = \#(A)^{\#(B) \times \#(C)} & \text{(by Proposition \ref{3.6.14}(e))} \\
                    & = a^{bc}.
    \end{align*}
    Thus we conclude that \(\forall\ a, b, c \in \mathbf{N} : (a^b)^c = a^{bc}\).

    Next we show that \(A^B \times A^C\) and \(A^{B \cup C}\) have equal cardinality if \(B \cap C = \emptyset\).
    Now suppose that \(A, B, C\) are sets where \(B \cap C = \emptyset\).
    By Axiom \ref{3.10} \(A^B, A^C, A^{B \cup C}\) are sets.
    By Definition \ref{3.5.4}, \(A^B \times A^C\) is a set.
    We define a function \(f : A^B \times A^C \to A^{B \cup C}\) by setting
    \[
        f(g, h)(x) = \begin{cases}
            g(x) & \text{if } x \in B \\
            h(x) & \text{if } x \in C
        \end{cases}
    \]
    where \(x \in B \cup C\), \(g : B \to A\) and \(h : C \to A\).
    We now show that \(f\) is bijective.
    We start by showing that \(f\) is injective.
    \begin{align*}
                 & \forall\ (g, h), (g', h') \in A^B \times A^C : f(g, h) = f(g', h')                                      \\
        \implies & \forall\ x \in B \cup C : f(g, h)(x) = f(g', h')(x)                & \text{(by Definition \ref{3.3.7})} \\
        \implies & (\forall\ x \in B : f(g, h)(x) = f(g', h')(x))                                                          \\
                 & \land (\forall\ x \in C : f(g, h)(x) = f(g', h')(x))               & \text{(by Axiom \ref{3.4})}        \\
        \implies & (\forall\ x \in B : g(x) = g'(x))                                                                       \\
                 & \land (\forall\ x \in C : h(x) = h'(x))                                                                 \\
        \implies & g = g' \land h = h'                                                & \text{(by Definition \ref{3.3.7})} \\
        \implies & (g, h) = (g', h').                                                 & \text{(by Definition \ref{3.5.1})}
    \end{align*}
    Thus \(f\) is injective.
    We now show that \(f\) is surjective.
    \(\forall\ k \in A^{B \cup C}\), we define functions \(g : B \to A\) and \(h : C \to A\) by setting
    \[
        k(x) = \begin{cases}
            g(x) & \text{if } x \in B \\
            h(x) & \text{if } x \in C
        \end{cases}
    \]
    Since \(g \in A^B \land h \in A^C\), by Definition \ref{3.5.4} we have \((g, h) \in A^B \times A^C\).
    Thus \(f\) is surjective.
    Since \(f\) is both injective and surjective, we conclude that \(f\) is bijective.
    Since \(f\) is bijective, by Definition \ref{3.6.1} we conclude that \(A^B \times A^C\) and \(A^{B \cup C}\) have same cardinality.

    Now we show that \(\forall\ a, b, c \in \mathbf{N} : a^b \times a^c = a^{b + c}\).
    Suppose that \(A, B, C\) are finite set where \(B \cap C = \emptyset\).
    By Definition \ref{3.6.5}, \(\exists\ a, b, c \in \mathbf{N}\) such that \(\#(A) = a \land \#(B) = b \land \#(C) = c\).
    Then we have
    \begin{align*}
        \#(A^B \times A^C) & = \#(A^B) \times \#(A^C)             & \text{(by Proposition \ref{3.6.14}(e))} \\
                           & = \#(A)^{\#(B)} \times \#(A)^{\#(C)} & \text{(by Proposition \ref{3.6.14}(f))} \\
                           & = a^b \times a^c                                                               \\
                           & = \#(A^{B \cup C})                   & \text{(by proof above)}                 \\
                           & = \#(A)^{\#(B \cup C)}               & \text{(by Proposition \ref{3.6.14}(f))} \\
                           & = \#(A)^{\#(B) + \#(C)}              & \text{(by Proposition \ref{3.6.14}(b))} \\
                           & = a^{b + c}.
    \end{align*}
    Thus we conclude that \(\forall\ a, b, c \in \mathbf{N} : a^b \times a^c = a^{b + c}\).
\end{proof}

\begin{exercise}\label{ex 3.6.7}
    Let \(A\) and \(B\) be sets.
    Let us say that \(A\) has \emph{lesser or equal} cardinality to \(B\) if there exists an injection \(f : A \to B\) from \(A\) to \(B\).
    Show that if \(A\) and \(B\) are finite sets, then \(A\) has lesser or equal cardinality to \(B\) if and only if \(\#(A) \leq \#(B)\).
\end{exercise}

\begin{proof}
    Suppose that \(A, B\) are finite sets.
    Then we have
    \begin{align*}
                 & A \text{ has lesser or equal cardinality to } B                                             \\
        \implies & \exists\ f : A \to B \land f \text{ is injective}                                           \\
        \implies & f(A) \subseteq B                                  & \text{(by Definition \ref{3.4.1})}      \\
                 & \land \#(f(A)) = \#(A)                            & \text{(by Proposition \ref{3.6.14}(d))} \\
        \implies & \#(A) = \#(f(A)) \leq \#(B).                      & \text{(by Proposition \ref{3.6.14}(c))}
    \end{align*}
    And
    \begin{align*}
                 & \#(A) \leq \#(B)                                                                                                                            \\
        \implies & \exists\ g : \{i \in \mathbf{N} : 1 \leq i \leq \#(A)\} \to A                                                                               \\
                 & \land g \text{ is bijective}                                                                          & \text{(by Definition \ref{3.6.5})}  \\
                 & \land \exists\ g' : \{i \in \mathbf{N} : 1 \leq i \leq \#(B)\} \to B                                                                        \\
                 & \land g' \text{ is bijective}                                                                         & \text{(by Definition \ref{3.6.5})}  \\
                 & \land \{i \in \mathbf{N} : 1 \leq i \leq \#(A)\} \subseteq \{i \in \mathbf{N} : 1 \leq i \leq \#(B)\}                                       \\
        \implies & g'(g^{-1}(A)) \subseteq B                                                                             & \text{(by Definition \ref{3.4.1})}  \\
                 & \land g' \circ g^{-1} \text{ is bijective}                                                            & \text{(by Exercise \ref{ex 3.3.2})} \\
        \implies & A \text{ has lesser or equal cardinality to } B.                                                      & \text{(by Definition \ref{3.3.20})}
    \end{align*}
    Thus we conclude that if \(A, B\) are finite sets, then \(A\) has lesser or equal cardinality to \(B\) iff \(\#(A) \leq \#(B)\).
\end{proof}

\begin{exercise}\label{ex 3.6.8}
    Let \(A\) and \(B\) be sets and \(A \neq \emptyset\) such that there exists an injection \(f : A \to B\) from \(A\) to \(B\) (i.e., \(A\) has lesser or equal cardinality to \(B\)).
    Show that there exists a surjection \(g : B \to A\) from \(B\) to \(A\).
\end{exercise}

\begin{proof}
    Suppose that \(A, B\) are sets, \(A \neq \emptyset\) and \(f : A \to B\) where \(f\) is injection.
    We now define a function \(g : B \to A\) as follow:
    \[
        \forall\ b \in B : \begin{cases}
            g(b) \in A \setminus \{a\} & \text{if } b \in f(A \setminus \{a\})  \\
            g(b) = a                   & \text{if } b \notin f(A) \lor b = f(a)
        \end{cases}
    \]
    where \(a \in A\) is a fixed value.
    We now show that \(g\) is surjective.
    \begin{align*}
                 & \forall\ a' \in A : (a' = a) \lor (a' \neq a)                                                       \\
        \implies & (\exists\ b \in B : b \notin f(A) \lor b = f(a'))     & \text{(\(a' = a \land f\) is injective)}    \\
                 & \lor (\exists\ b \in B : b \in f(A \setminus \{a'\})) & \text{(\(a' \neq a \land f\) is injective)} \\
        \implies & \exists\ b \in B : g(b) = a'.
    \end{align*}
    Thus \(g\) is surjective.
\end{proof}

\begin{exercise}\label{ex 3.6.9}
    Let \(A\) and \(B\) be finite sets.
    Show that \(A \cup B\) and \(A \cap B\) are also finite sets, and that \(\#(A) + \#(B) = \#(A \cup B) + \#(A \cap B)\).
\end{exercise}

\begin{proof}
    Suppose that \(A, B\) are finite sets.
    By Proposition \ref{3.6.14}(b) we have \(A \cup B\) is finite.
    By Exercise \ref{ex 3.1.7} we have \(A \cap B \subseteq A\), and thus by Proposition \ref{3.6.14}(c) we have \(A \cup B\) is finite.
    By Definition \ref{3.6.10} \(\exists\ n \in \mathbf{N}\) such that \(\#(A) = n\).
    We use induction on \(n\) to show that \(\#(A) + \#(B) = \#(A \cup B) + \#(A \cap B)\).
    For \(n = 0\), we have \(A = \emptyset\) and
    \begin{align*}
        \#(\emptyset) + \#(B) & = 0 + \#(B)                                    & \text{(by Definition \ref{3.6.5})}      \\
                              & = \#(B)                                                                                  \\
                              & = \#(\emptyset \cup B)                         & \text{(by Proposition \ref{3.1.28}(a))} \\
                              & = \#(\emptyset \cup B) + 0                                                               \\
                              & = \#(\emptyset \cup B) + \#(\emptyset)         & \text{(by Definition \ref{3.6.5})}      \\
                              & = \#(\emptyset \cup B) + \#(\emptyset \cap B). & \text{(by Proposition \ref{3.1.28}(a))}
    \end{align*}
    Thus the base case holds.
    Suppose inductively that for some \(n\) the statement is true.
    We now show that for \(n++\) the statement is also true.
    Since \(\#(A) = n++\), By Proposition \ref{3.6.8} we must have \(A \neq \emptyset\).
    Let \(a \in A\).
    Then we have
    \begin{align*}
         & \#(A) + \#(B)                                                                                                    \\
         & = \#((A \setminus \{a\}) \cup \{a\}) + \#(B)                           & \text{(by Proposition \ref{3.1.28}(g))} \\
         & = \#(A \setminus \{a\}) + 1 + \#(B)                                    & \text{(by Proposition \ref{3.6.14}(a))} \\
         & = \#((A \setminus \{a\}) \cup B) + \#((A \setminus \{a\}) \cap B) + 1. & \text{(by induction hypothesis)}
    \end{align*}
    We now divide into two cases:
    \begin{itemize}
        \item If \(a \in B\), then we have
              \begin{align*}
                   & \#((A \setminus \{a\}) \cup B) + \#((A \setminus \{a\}) \cap B) + 1                                                 \\
                   & = \#(A \cup B) + \#((A \setminus \{a\}) \cap B) + 1                                                                 \\
                   & = \#(A \cup B) + \#(((A \setminus \{a\}) \cap B) \cup \{a\})              & \text{(by Proposition \ref{3.6.14}(a))} \\
                   & = \#(A \cup B) + \#(((A \setminus \{a\}) \cup \{a\}) \cap (B \cup \{a\})) & \text{(by Proposition \ref{3.1.28}(f))} \\
                   & = \#(A \cup B) + \#(A \cap B).                                            & \text{(by Proposition \ref{3.1.28}(g))}
              \end{align*}
        \item If \(a \notin B\), then we have
              \begin{align*}
                   & \#((A \setminus \{a\}) \cup B) + \#((A \setminus \{a\}) \cap B) + 1                                           \\
                   & = \#((A \setminus \{a\}) \cup B) + \#(A \cap B) + 1                                                           \\
                   & = \#(((A \setminus \{a\}) \cup B) \cup \{a\}) + \#(A \cap B)        & \text{(by Proposition \ref{3.6.14}(a))} \\
                   & = \#(A \cup B) + \#(A \cap B).                                      & \text{(by Proposition \ref{3.1.28}(d))}
              \end{align*}
    \end{itemize}
    From all cases above we have \(\#(A) + \#(B) = \#(A \cup B) + \#(A \cap B)\) and this closes the induction.
\end{proof}

\begin{exercise}\label{ex 3.6.10}
    Let \(A_1, \dots, A_n\) be finite sets such that \(\#(\bigcup_{i \in \{1, \dots, n\}} A_i) > n\).
    Show that there exists \(i \in \{1, \dots, n\}\) such that \(\#(A_i) \geq 2\).
    (This is known as the \emph{pigeonhole principle}.)
\end{exercise}

\begin{proof}
    Suppose that \(n \in \mathbf{N}\), \(A_1, \dots, A_n\) are finite sets and \(\#(\bigcup_{i \in \{1, \dots, n\}} A_i) > n\).
    We use induction on \(n\) to show that \(\exists\ i \in \{1, \dots, n\} : \#(A_i) \geq 2\).
    We start with \(n = 1\) since for \(n = 0\) the statement is vacuously true.
    For \(n = 1\), we have
    \begin{align*}
                 & \#(\bigcup_{i \in \{1, \dots, 1\}} A_i) > 1                                \\
        \implies & \#(A_1) > 1                                 & \text{(by Axiom \ref{3.11})} \\
        \implies & \#(A_1) \geq 2.
    \end{align*}
    Thus the base case holds.
    Suppose inductively that for some \(n\) the statement is true.
    Then for \(n++\), we have
    \begin{align*}
                 & \#(\bigcup_{i \in \{1, \dots, n++\}} A_i) > n++                                                        \\
        \implies & \#((\bigcup_{i \in \{1, \dots, n\}} A_i) \cup A_{n++}) > n++ & \text{(by Axiom \ref{3.11})}            \\
        \implies & \#(\bigcup_{i \in \{1, \dots, n\}} A_i) + \#(A_{n++}) > n++. & \text{(by Proposition \ref{3.6.14}(b))}
    \end{align*}
    Now we split into three cases:
    \begin{itemize}
        \item If \(\#(A_{n++}) = 0\), then we have \(\#(\bigcup_{i \in \{1, \dots, n\}} A_i) > n++ > n\).
              By induction hypothesis, \(\exists\ i \in \{1, \dots, n\} : \#(A_i) \geq 2\).
              Then we have \(\exists\ i \in \{1, \dots, n++\} : \#(A_i) \geq 2\).
        \item If \(\#(A_{n++}) = 1\), then we have \(\#(\bigcup_{i \in \{1, \dots, n\}} A_i) + 1 > n++\).
              This means \(\#(\bigcup_{i \in \{1, \dots, n\}} A_i) > n\) and by induction hypothesis, \(\exists\ i \in \{1, \dots, n\} : \#(A_i) \geq 2\).
              Then we have \(\exists\ i \in \{1, \dots, n++\} : \#(A_i) \geq 2\).
        \item If \(\#(A_{n++}) > 1\), then \(\#(A_{n++}) \geq 2\) and we have \(\exists\ i \in \{1, \dots, n++\} : \#(A_i) \geq 2\).
    \end{itemize}
    From all cases above we have \(\exists\ i \in \{1, \dots, n++\} : \#(A_i) \geq 2\).
    This closes the induction.
\end{proof}
\chapter{Integers and rationals}

\section{The integers}

\begin{definition}[Integers]\label{4.1.1}
An \emph{integer} is an expression of the form \(a \text{-----} b\), where \(a\) and \(b\) are natural numbers.
Two integers are considered to be equal, \(a \text{-----} b = c \text{-----} d\), if and only if \(a + d = c + b\).
We let \(\mathds{Z}\) denote the set of all integers.
\end{definition}

\begin{note}
In the language of set theory, what we are doing here is starting with the space \(\mathds{N} \times \mathds{N}\) of ordered pairs \((a, b)\) of natural numbers.
Then we place an \emph{equivalence relation} \(\sim\) on these pairs by declaring \((a, b) \sim (c, d)\) iff \(a + d = c + b\).
The set-theoretic interpretation of the symbol \(a \text{-----} b\) is that it is the space of all pairs equivalent to \((a, b): a \text{-----} b \coloneqq \{(c, d) \in \mathds{N} \times \mathds{N} : (a, b) \sim (c, d)\}\).
However, this interpretation plays no role in how we manipulate the integers and we will not refer to it again.
A similar set-theoretic interpretation can be given to the construction of the rational numbers later in this chapter, or the real numbers in the next chapter.
\end{note}

\begin{additional corollary}\label{ac 4.1.1}
The definition of equality on the integers is reflexive, symmetric and transitive.
\end{additional corollary}

\begin{proof}
We first prove the reflexivity of the integers.
\(\forall\ (a, b) \in \mathds{N} \times \mathds{N}\), \(a + b = a + b\), so \(a \text{-----} b = a \text{-----} b\).

Next we prove the symmetry of the integers.
\(\forall\ (a, b), (c, d) \in \mathds{N} \times \mathds{N}\), if \(a + d = c + b\), then \(a \text{-----} b = c \text{-----} d\).
But \(a + d = c + b \implies c + b = a + d\), so \(c \text{-----} d = a \text{-----} b\).
Thus \(a \text{-----} b = c \text{-----} d \implies c \text{-----} d = a \text{-----} b\).

Finally we prove the transitivity of the integers.
\(\forall\ (a, b), (c, d), (e, f) \in \mathds{N} \times \mathds{N}\), if \(a + d = c + b\) and \(c + f = e + d\), then \(a \text{-----} b = c \text{-----} d\) and \(c \text{-----} d = e \text{-----} f\).
Because \(a + d = c + b\) and \(c + f = e + d\), so \(a + d + c + f = c + b + e + d\).
By Proposition \ref{2.2.6}, \(a + d + c + f = c + b + e + d \implies a + f = e + b\), so \(a \text{-----} b = e \text{-----} f\).
\end{proof}

\begin{definition}\label{4.1.2}
The sum of two integers, \((a \text{-----} b) + (c \text{-----} d)\), is defined by the formula
\[
    (a \text{-----} b) + (c \text{-----} d) \coloneqq (a + c) \text{-----} (b + d).
\]
The product of two integers, \((a \text{-----} b) \times (c \text{-----} d)\), is defined by the formula
\[
    (a \text{-----} b) \times (c \text{-----} d) \coloneqq (ac + bd) \text{-----} (ad + bc).
\]
\end{definition}

\begin{lemma}[Addition and multiplication are well-defined]\label{4.1.3}
Let \(a, b, a', b', c, d\) be natural numbers.
If \((a \text{-----} b) = (a' \text{-----} b')\), then \((a \text{-----} b) + (c \text{-----} d) = (a' \text{-----} b') + (c \text{-----} d)\) and \((a \text{-----} b) \times (c \text{-----} d) = (a' \text{-----} b') \times (c \text{-----} d)\), and also \((c \text{-----} d) + (a \text{-----} b) = (c \text{-----} d) + (a' \text{-----} b')\) and \((c \text{-----} d) \times (a \text{-----} b) = (c \text{-----} d) \times (a' \text{-----} b')\).
Thus addition and multiplication are well-defined operations (equal inputs give equal outputs).
\end{lemma}

\begin{proof}
To prove that \((a \text{-----} b) + (c \text{-----} d) = (a' \text{-----} b') + (c \text{-----} d)\), we evaluate both sides as \((a + c) \text{-----} (b + d)\) and \((a' + c) \text{-----} (b' + d)\).
Thus we need to show that \(a + c + b' + d = a' + c + b + d\).
But since \((a \text{-----} b) = (a' \text{-----} b')\), we have \(a + b' = a' + b\), and so by adding \(c + d\) to both sides we obtain the claim.
Now we show that \((a \text{-----} b) \times (c \text{-----} d) = (a' \text{-----} b') \times (c \text{-----} d)\).
Both sides evaluate to \((ac + bd) \text{-----} (ad + bc)\) and \((a'c + b'd) \text{-----} (a'd + b'c)\), so we have to show that \(ac + bd + a'd + b'c = a'c + b'd + ad + bc\).
But the left-hand side factors as \(c(a + b') + d(a' + b)\), while the right factors as \(c(a' + b) + d(a + b')\).
Since \(a + b' = a' + b\), the two sides are equal.
The other two identities are proven similarly.
\end{proof}

\begin{note}
The integers \(n \text{-----} 0\) behave in the same way as the natural numbers \(n\);
indeed one can check that \((n \text{-----} 0) + (m \text{-----} 0) = (n + m) \text{-----} 0\) and \((n \text{-----} 0) \times (m \text{-----} 0) = nm \text{-----} 0\).
Furthermore, \((n \text{-----} 0)\) is equal to \((m \text{-----} 0)\) if and only if \(n = m\).
(The mathematical term for this is that there is an \emph{isomorphism} between the natural numbers \(n\) and those integers of the form \(n \text{-----} 0\).)
Thus we may \emph{identify} the natural numbers with integers by setting \(n \equiv n \text{-----} 0\);
this does not affect our definitions of addition or multiplication or equality since they are consistent with each other.
Of course, if we set \(n\) equal to \(n \text{-----} 0\), then it will also be equal to any other integer which is equal to \(n \text{-----} 0\).
\end{note}

\begin{note}
We can now define incrementation on the integers by defining \(x++ \coloneqq x + 1\) for any integer \(x\);
this is of course consistent with our definition of the increment operation for natural numbers.
However, this is no longer an important operation for us, as it has been now superceded by the more general notion of addition.
\end{note}

\begin{definition}[Negation of integers]\label{4.1.4}
If \((a \text{-----} b)\) is an integer, we define the negation \(-(a \text{-----} b)\) to be the integer \((b \text{-----} a)\).
In particular if \(n = n \text{-----} 0\) is a positive natural number, we can define its negation \(-n = 0 \text{-----} n\).
\end{definition}

\begin{additional corollary}\label{ac 4.1.2}
The definition of negation on the integers is well-defined.
\end{additional corollary}

\begin{proof}
Let \(a, b, a', b' \in \mathds{N}\) and \(a \text{-----} b = a' \text{-----} b'\).
\begin{align*}
& a \text{-----} b = a' \text{-----} b' \\
\implies & a + b' = a' + b & \text{(By Definition \ref{4.1.1})} \\
\implies & b' + a = b + a' & \text{(By Proposition \ref{2.2.4})} \\
\implies & b' \text{-----} a' = b \text{-----} a & \text{(By Definition \ref{4.1.1})} \\
\implies & -(a' \text{-----} b') = -(a \text{-----} b) & \text{(By Definition \ref{4.1.4})} \\
\end{align*}
\end{proof}

\begin{lemma}[Trichotomy of integers]\label{4.1.5}
Let \(x\) be an integer.
Then exactly one of the following three statements is true:
\begin{enumerate*}
    \item \(x\) is zero.
    \item \(x\) is equal to a positive natural number \(n\).
    \item \(x\) is the negation \(-n\) of a positive natural number \(n\).
\end{enumerate*}
\end{lemma}

\begin{proof}
We first show that at least one of (a), (b), (c) is true.
By definition, \(x = a \text{-----} b\) for some natural numbers \(a, b\).
By Proposition \ref{2.2.13}, we have three cases: \(a > b\), \(a = b\), or \(a < b\).
If \(a > b\) then \(a = b + c\) for some positive natural number \(c\), which means that \(a \text{-----} b = c \text{-----} 0 = c\), which is (b).
If \(a = b\), then \(a \text{-----} b = a \text{-----} a = 0 \text{-----} 0 = 0\), which is (a).
If \(a < b\), then \(b > a\), so that \(b \text{-----} a = n\) for some natural number \(n\) by the previous reasoning, and thus \(a \text{-----} b = -n\), which is (c).
Now we show that no more than one of (a), (b), (c) can hold at a time.
By definition, a positive natural number is non-zero, so (a) and (b) cannot simultaneously be true.
If (a) and (c) were simultaneously true, then \(0 = -n\) for some positive natural \(n\);
thus \((0 \text{-----} 0) = (0 \text{-----} n)\), so that \(0 + n = 0 + 0\), so that \(n = 0\), a contradiction.
If (b) and (c) were simultaneously true, then \(n = -m\) for some positive \(n, m\), so that \((n \text{-----} 0) = (0 \text{-----} m)\), so that \(n + m = 0 + 0\), which contradicts Proposition \ref{2.2.8}.
Thus exactly one of (a), (b), (c) is true for any integer \(x\).
\end{proof}

\begin{note}
If \(n\) is a positive natural number, we call \(-n\) a \emph{negative integer}.
Thus every integer is positive, zero, or negative, but not more than one of these at a time.
\end{note}

\begin{note}
One could well ask why we don’t use Lemma \ref{4.1.5} to \emph{define} the integers;
i.e., why didn’t we just say an integer is anything which is either a positive natural number, zero, or the negative of a natural number.
The reason is that if we did so, the rules for adding and multiplying integers would split into many different cases (e.g., negative times positive equals positive; negative plus positive is either negative, positive, or zero, depending on which term is larger, etc.) and to verify all the properties would end up being much messier.
\end{note}

\begin{proposition}[Laws of algebra for integers]\label{4.1.6}
Let \(x\), \(y\), \(z\) be integers.
Then we have
\begin{align*}
    x + y &= y + x \\
    (x + y) + z &= x + (y + z) \\
    x + 0 = 0 + x &= x \\
    x + (-x) = (-x) + x &= 0 \\
    xy &= yx \\
    (xy)z &= x(yz) \\
    x1 = 1x &= x \\
    x(y + z) &= xy + xz \\
    (y + z)x &= yx + zx.
\end{align*}
\end{proposition}

\begin{proof}
There are two ways to prove these identities.
One is to use Lemma \ref{4.1.5} and split into a lot of cases depending on whether \(x, y, z\) are zero, positive, or negative.
This becomes very messy.
A shorter way is to write \(x = (a \text{-----} b), y = (c \text{-----} d)\), and \(z = (e \text{-----} f)\) for some natural numbers \(a, b, c, d, e, f\), and expand these identities in terms of \(a, b, c, d, e, f\) and use the algebra of the natural numbers.
This allows each identity to be proven in a few lines.

We first prove \(x + y = y + x\).
\begin{align*}
x + y &= (a \text{-----} b) + (c \text{-----} d) && \text{(by the given condition)} \\
&= (a + c) \text{-----} (b + d) && \text{(by Definition \ref{4.1.2})} \\
&= (c + a) \text{-----} (d + b) && \text{(by Proposition \ref{2.2.4})} \\
&= (c \text{-----} d) + (a \text{-----} b) && \text{(by Definition \ref{4.1.2})} \\
&= y + x. && \text{(by the given condition)}
\end{align*}

Next we prove \((x + y) + z = x + (y + z)\).
\begin{align*}
(x + y) + z &= ((a \text{-----} b) + (c \text{-----} d)) + (e \text{-----} f) && \text{(by the given condition)} \\
&= ((a + c) \text{-----} (b + d)) + (e \text{-----} f) && \text{(by Definition \ref{4.1.2})} \\
&= ((a + c) + e) \text{-----} ((b + d) + f) && \text{(by Definition \ref{4.1.2})} \\
&= (a + (c + e)) \text{-----} (b + (d + f)) && \text{(by Proposition \ref{2.2.5})} \\
&= (a \text{-----} b) + ((c + e) \text{-----} (d + f)) && \text{(by Definition \ref{4.1.2})} \\
&= (a \text{-----} b) + ((c \text{-----} d) + (e \text{-----} f)) && \text{(by Definition \ref{4.1.2})} \\
&= x + (y + z). && \text{(by the given condition)}
\end{align*}

Next we prove \(x + 0 = 0 + x = x\).
\begin{align*}
x + 0 &= (a \text{-----} b) + (0 \text{-----} 0) && \text{(by the given condition)} \\
&= (a + 0) \text{-----} (b + 0) && \text{(by Definition \ref{4.1.2})} \\
&= (a \text{-----} b) && \text{(by Lemma \ref{2.2.2})} \\
&= x && \text{(by the given condition)} \\
&= (0 + a) \text{-----} (0 + b) && \text{(by Definition \ref{2.2.1})} \\
&= (0 \text{-----} 0) + (a \text{-----} b) && \text{(by Definition \ref{4.1.2})} \\
&= 0 + x. && \text{(by the given condition)}
\end{align*}

Next we prove \(x + (-x) = (-x) + x = 0\).
\begin{align*}
x + (-x) &= (a \text{-----} b) + (-x) && \text{(by the given condition)} \\
&= (a \text{-----} b) + (b \text{-----} a) && \text{(by Definition \ref{4.1.4})} \\
&= (a + b) \text{-----} (b + a) && \text{(by Definition \ref{4.1.2})} \\
&= (b + a) \text{-----} (a + b) && \text{(by Proposition \ref{2.2.4})} \\
&= (b \text{-----} a) + (a \text{-----} b) && \text{(by Definition \ref{4.1.2})} \\
&= (-x) + x && \text{(by the given condition)} \\
&= (a + b) \text{-----} (a + b) && \text{(by Proposition \ref{2.2.4})} \\
&= (a \text{-----} a) + (b \text{-----} b) && \text{(by Definition \ref{4.1.2})} \\
&= (0 \text{-----} 0) + (0 \text{-----} 0) && \text{(by Definition \ref{4.1.1})} \\
&= (0 + 0) \text{-----} (0 + 0) && \text{(by Definition \ref{4.1.2})} \\
&= (0 \text{-----} 0) && \text{(by Definition \ref{2.2.1})} \\
&= 0. && \text{(by Lemma \ref{4.1.5})}
\end{align*}

Next we prove \(xy = yx\).
\begin{align*}
xy &= (a \text{-----} b) \times (c \text{-----} d) && \text{(by the given condition)} \\
&= (ac + bd) \text{-----} (ad + bc) && \text{(by Definition \ref{4.1.2})} \\
&= (ca + db) \text{-----} (da + cb) && \text{(by Lemma \ref{2.3.2})} \\
&= (ca + db) \text{-----} (cb + da) && \text{(by Proposition \ref{2.2.4})} \\
&= (c \text{-----} d) \times (a \text{-----} b) && \text{(by Definition \ref{4.1.2})} \\
&= yx. && \text{(by the given condition)}
\end{align*}

Next we prove \((xy)z = x(yz)\).
\begin{align*}
(xy)z &= ((a \text{-----} b) \times (c \text{-----} d)) \times (e \text{-----} f) && \text{(by the given condition)} \\
&= ((ac + bd) \text{-----} (ad + bc)) \times (e \text{-----} f) && \text{(by Definition \ref{4.1.2})} \\
&= ((ac + bd)e + (ad + bc)f) \\
&\quad \text{-----} ((ac + bd)f + (ad + bc)e) && \text{(by Definition \ref{4.1.2})} \\
&= ((ac)e + (bd)e + (ad)f + (bc)f) \\
&\quad \text{-----} ((ac)f + (bd)f + (ad)e + (bc)e) && \text{(by Proposition \ref{2.3.4})} \\
&= (a(ce) + b(de) + a(df) + b(cf)) \\
&\quad \text{-----} (a(cf) + b(df) + a(de) + b(ce)) && \text{(by Proposition \ref{2.3.5})} \\
&= (a(ce) + a(df) + b(cf) + b(de)) \\
&\quad \text{-----} (a(cf) + a(de) + b(ce) + b(df)) && \text{(by Proposition \ref{2.2.4})} \\
&= (a(ce + df) + b(cf + de)) \\
&\quad \text{-----} (a(cf + de) + b(ce + df)) && \text{(by Proposition \ref{2.3.4})} \\
&= (a \text{-----} b) \times ((ce + df) \text{-----} (cf + de)) && \text{(by Definition \ref{4.1.2})} \\
&= (a \text{-----} b) \times ((c \text{-----} d) \times (e \text{-----} f)) && \text{(by Definition \ref{4.1.2})} \\
&= x(yz). && \text{(by the given condition)}
\end{align*}

Next we prove \(x1 = 1x = x\).
We first show that \(\forall\ n \in \mathds{N}\), \(n1 = n\).
We use induction on \(n\).
For \(n=0\), \(0 \times 1 = 0\) by Lemma \ref{2.3.3}, so the base case holds.
Suppose inductively that \(n1 = n\) is true for some \(n\).
Then for \(n + 1\), by Definition \ref{2.3.1}, \((n + 1) \times 1 = n1 + 1\).
By induction hypothesis, \(n1 + 1 = n + 1\).
This close the induction.
Since \(n1 = 1n\) by Lemma \ref{2.3.2}, \(1n = n\).

Now we show that \(x1 = 1x = x\).
\begin{align*}
x1 &= (a \text{-----} b) \times (1 \text{-----} 0) && \text{(by the given condition)} \\
&= (a1 + b0) \text{-----} (a0 + b1) && \text{(by Definition \ref{4.1.2})} \\
&= (1a + 0b) \text{-----} (0a + 1b) && \text{(by Lemma \ref{2.3.2})} \\
&= (1a + 0b) \text{-----} (1b + 0a) && \text{(by Proposition \ref{2.2.4})} \\
&= (1 \text{-----} 0) + (a \text{-----} b) && \text{(by Definition \ref{4.1.2})} \\
&= 1x && \text{(by the given condition)} \\
&= (1a + 0) \text{-----} (1b + 0) && \text{(by Definition \ref{2.3.1})} \\
&= (1a) \text{-----} (1b) && \text{(by Lemma \ref{2.2.2})} \\
&= (a \text{-----} b) && \text{(by previous prove)} \\
&= x. && \text{(by the given condition)}
\end{align*}

Next we prove that \(x(y + z) = xy + xz\).
\begin{align*}
x(y + z) &= (a \text{-----} b) \times ((c \text{-----} d) + (e \text{-----} f)) && \text{(by the given condition)} \\
&= (a \text{-----} b) \times ((c + e) \text{-----} (d + f)) && \text{(by Definition \ref{4.1.2})} \\
&= (a(c + e) + b(d + f)) \text{-----} (a(d + f) + b(c + e)) && \text{(by Definition \ref{4.1.2})} \\
&= (ac + ae + bd + bf) \text{-----} (ad + af + bc + be) && \text{(by Proposition \ref{2.3.4})} \\
&= (ac + bd + ae + bf) \text{-----} (ad + bc + af + be) && \text{(by Proposition \ref{2.2.4})} \\
&= ((ac + bd) \text{-----} (ad + bc)) + ((ae + bf) \text{-----} (af + be)) && \text{(by Definition \ref{4.1.2})} \\
&= (a \text{-----} b) \times (c \text{-----} d) + (a \text{-----} b) \times (e \text{-----} f) && \text{(by Definition \ref{4.1.2})} \\
&= xy + xz. && \text{(by the given condition)}
\end{align*}

Finally we prove \((y + z)x = yx + zx\).
\begin{align*}
(y + z)x &= ((c \text{-----} d) + (e \text{-----} f)) \times (a \text{-----} b) && \text{(by the given condition)} \\
&= ((c + e) \text{-----} (d + f)) \times (a \text{-----} b) && \text{(by Definition \ref{4.1.2})} \\
&= ((c + e)a + (d + f)b) \text{-----} ((c + e)b + (d + f)a) && \text{(by Definition \ref{4.1.2})} \\
&= (ca + ea + db + fb) \text{-----} (cb + eb + da + fa) && \text{(by Proposition \ref{2.3.4})} \\
&= (ca + db + ea + fb) \text{-----} (cb + da + eb + fa) && \text{(by Proposition \ref{2.2.4})} \\
&= ((ca + db) \text{-----} (cb + da)) + ((ea + fb) \text{-----} (eb + fa)) && \text{(by Definition \ref{4.1.2})} \\
&= (c \text{-----} d) \times (a \text{-----} b) + (e \text{-----} f) \times (a \text{-----} b) && \text{(by Definition \ref{4.1.2})} \\
&= yx + zx. && \text{(by the given condition)}
\end{align*}
\end{proof}

\begin{remark}\label{4.1.7}
The above set of nine identities have a name; they are asserting that the integers form a \emph{commutative ring}.
(If one deleted the identity \(xy = yx\), then they would only assert that the integers form a \emph{ring}).
Note that some of these identities were already proven for the natural numbers, but this does not automatically mean that they also hold for the integers because the integers are a larger set than the natural numbers.
On the other hand, this proposition supercedes many of the propositions derived earlier for natural numbers.
\end{remark}

\begin{note}
We now define the operation of \emph{subtraction} \(x - y\) of two integers by the formula
\[
    x - y \coloneqq x + (-y).
\]
We do not need to verify the substitution axiom for this operation, since we have defined subtraction in terms of two other operations on integers, namely addition and negation, and we have already verified that those operations are well-defined.
\end{note}

\begin{note}
One can easily check now that if \(a\) and \(b\) are natural numbers, then
\[
    a - b = a + -b = (a \text{-----} 0) + (0 \text{-----} b) = a \text{-----} b,
\]
and so \(a \text{-----} b\) is just the same thing as \(a - b\).
Because of this we can now discard the ----- notation, and use the familiar operation of subtraction instead.
(As remarked before, we could not use subtraction immediately because it would be circular.)
\end{note}

\begin{proposition}[Integers have no zero divisors]\label{4.1.8}
Let \(a\) and \(b\) be integers such that \(ab = 0\).
Then either \(a = 0\) or \(b = 0\) (or both).
\end{proposition}

\begin{proof}
Let \(a = a_1 \text{-----} a_2\) and \(b = b_1 \text{-----} b_2\), where \(a_1, a_2, b_1, b_2 \in \mathds{N}\).
Then
\begin{align*}
ab &= (a_1 \text{-----} a_2) \times (b_1 \text{-----} b_2) \\
&= (a_1b_1 + a_2b_2) \text{-----} (a_1b_2 + a_2b_1) \\
&= 0 \text{-----} 0 \\
&= 0.
\end{align*}
Which means \(a_1b_1 + a_2b_2 = a_1b_2 + a_2b_1\).
    \begin{enumerate}
        \item If \(a \neq 0\), then \(a_1 \neq a_2\).
            \begin{enumerate}[label=(\roman*)]
                \item If \(b_1 = 0\), then \(a_1b_1 + a_2b_2 = a_1b_2 + a_2b_1 \implies a_2b_2 = a_1b_2\).
                If \(b_2 \neq 0\), by Corollary \ref{2.3.7}, \(a_2b_2 = a_1b_2 \implies a_2 = a_1\), a contradiction.
                So \(b_2 = 0\), which means \(b = b_1 \text{-----} b_2 = 0 \text{-----} 0 = 0\).
                \item If \(b_1 \neq 0\), then \(a_1b_1 \neq a_2b_1\) by Corollary \ref{2.3.7}.
                By Proposition \ref{2.2.13}, only one of the \(a_1 < a_2\) and \(a_1 > a_2\) is true.
                \begin{enumerate}[label=(\arabic*)]
                    \item If \(a_1 < a_2\), then \(a_2 = a_1 + d\), where \(d \in \mathds{N}\) and \(d\) is positive.
                    So
                        \begin{align*}
                            & a_1b_1 + a_2b_2 = a_1b_2 + a_2b_1 \\
                            \implies & a_1b_1 + (a_1 + d)b_2 = a_1b_2 + (a_1 + d)b_1 \\
                            \implies & a_1b_1 + a_1b_2 + db_2 = a_1b_2 + a_1b_1 + db_1 & \text{(by Proposition \ref{2.3.4})} \\
                            \implies & db_2 = db_1 & \text{(by Proposition \ref{2.2.6})} \\
                            \implies & b_2 = b_1. & \text{(by Corollary \ref{2.3.7})}
                        \end{align*}
                    Which means \(b = b_1 \text{-----} b_2 = 0 \text{-----} 0 = 0\).
                    \item If \(a_1 > a_2\), then \(a_1 = a_2 + d\), where \(d \in \mathds{N}\) and \(d\) is positive.
                    So
                        \begin{align*}
                            & a_1b_1 + a_2b_2 = a_1b_2 + a_2b_1 \\
                            \implies & (a_2 + d)b_1 + a_2b_2 = (a_2 + d)b_2 + a_2b_1 \\
                            \implies & a_2b_1 + db_1 + a_2b_2 = a_2b_2 + db_2 + a_2b_1 & \text{(by Proposition \ref{2.3.4})} \\
                            \implies & db_1 = db_2 & \text{(by Proposition \ref{2.2.6})} \\
                            \implies & b_1 = b_2. & \text{(by Corollary \ref{2.3.7})}
                        \end{align*}
                    Which means \(b = b_1 \text{-----} b_2 = 0 \text{-----} 0 = 0\).
                \end{enumerate}
            \end{enumerate}
        In all cases we get \(b = 0\).
        \item If \(a = 0\), then \(a_1 = a_2\) and \(a_1 \text{-----} a_2 = 0 \text{-----} 0\).
        So
        \begin{align*}
            ab &= (a_1 \text{-----} a_2) \times (b_1 \text{-----} b_2) \\
            &= (0 \text{-----} 0) \times (b_1 \text{-----} b_2) \\
            &= (0b_1 + 0b_2) \text{-----} (0b_2 + 0b_1) & \text{(by Definition \ref{4.1.2})} \\
            &= (0 + 0) \text{-----} (0 + 0) & \text{(by Definition \ref{2.3.1})} \\
            &= 0 \text{-----} 0 & \text{(by Definition \ref{2.2.1})} \\
            &= 0.
        \end{align*}
    \end{enumerate}
\end{proof}

\begin{corollary}[Cancellation law for integers]\label{4.1.9}
If \(a, b, c\) are integers such that \(ac = bc\) and \(c\) is non-zero, then \(a = b\).
\end{corollary}

\begin{proof}
By Proposition \ref{4.1.6}, \(ac = bc \implies ac - bc = 0 \implies (a - b)c = 0\).
By Proposition \ref{4.1.8}, either \(a - b = 0\) or \(c = 0\) is true, but \(c \neq 0\), so \(a - b = 0\), which means \(a = b\).
\end{proof}

\begin{definition}[Ordering of the integers]\label{4.1.10}
Let \(n\) and \(m\) be integers.
We say that \(n\) is greater than or equal to \(m\), and write \(n \geq m\) or \(m \leq n\), iff we have \(n = m + a\) for some natural number \(a\).
We say that \(n\) is strictly greater than \(m\), and write \(n > m\) or \(m < n\), iff \(n \geq m\) and \(n \neq m\).
\end{definition}

\begin{lemma}[Properties of order]\label{4.1.11}
Let \(a, b, c\) be integers.
\begin{enumerate}
    \item \(a > b\) if and only if \(a - b\) is a positive natural number.
    \item (Addition preserves order) If \(a > b\), then \(a + c > b + c\).
    \item (Positive multiplication preserves order) If \(a > b\) and \(c\) is positive, then \(ac > bc\).
    \item (Negation reverses order) If \(a > b\), then \(-a < -b\).
    \item (Order is transitive) If \(a > b\) and \(b > c\), then \(a > c\).
    \item (Order trichotomy) Exactly one of the statements \(a > b\), \(a < b\), or \(a = b\) is true.
\end{enumerate}
\end{lemma}

\begin{proof}{(a)}
We first prove that \(a > b\) implies \(a - b\) is a positive natural number.
By Definition \ref{4.1.10}, \(a > b \implies a = b + d\), where \(d \in \mathds{N}\) and \(a \neq b\).
Then \(a - b = a + (-b) = (b + d) + (-b) = (d + b) + (-b) = d + (b + (-b)) = d + 0 = d\) by Proposition \ref{4.1.6}.
Also because \(a \neq b\), \(d \neq 0\).
By Definition \ref{2.2.7}, \(d\) is positive natural number, so \(a - b\) is positive natural number.

Now we prove that \(a - b\) is a positive natural number implies \(a > b\).
Let \(a - b = d\).
By Proposition \ref{4.1.6}, \((a - b) + b = d + b \implies a = b + d\).
By Definition \ref{4.1.10}, \(a \geq b\).
Because \(d\) is a positive natural number, \(b \neq b + d = a\), so by Definition \ref{4.1.10}, \(a > b\).
\end{proof}

\begin{proof}{(b)}
By Lemma \ref{4.1.11}(a), \(a > b \implies a - b\) is positive.
Let \(a - b = d\).
So
\begin{align*}
& a - b = d \\
\implies & (a - b) + b = d + b & \text{(by Lemma \ref{4.1.3})} \\
\implies & a + (-b + b) = d + b & \text{(by Proposition \ref{4.1.6})} \\
\implies & a = d + b & \text{(by Proposition \ref{4.1.6})} \\
\implies & a = b + d & \text{(by Proposition \ref{4.1.6})} \\
\implies & a + c = (b + d) + c & \text{(by Lemma \ref{4.1.3})} \\
\implies & a + c = b + (d + c) & \text{(by Proposition \ref{4.1.6})} \\
\implies & a + c = b + (c + d) & \text{(by Proposition \ref{4.1.6})} \\
\implies & a + c = (b + c) + d & \text{(by Proposition \ref{4.1.6})} \\
\implies & (a + c) + (-(b + c)) = ((b + c) + d) + (-(b + c)) & \text{(by Lemma \ref{4.1.3})} \\
\implies & (a + c) - (b + c) = ((b + c) + d) + (-(b + c)) \\
\implies & (a + c) - (b + c) = (d + (b + c)) + (-(b + c)) & \text{(by Proposition \ref{4.1.6})} \\
\implies & (a + c) - (b + c) = d + ((b + c) + (-(b + c))) & \text{(by Proposition \ref{4.1.6})} \\
\implies & (a + c) - (b + c) = d + 0 & \text{(by Proposition \ref{4.1.6})} \\
\implies & (a + c) - (b + c) = d & \text{(by Proposition \ref{4.1.6})} \\
\implies & a + c > b + c. & \text{(by Lemma \ref{4.1.11}(a))} \\
\end{align*}
\end{proof}

\begin{proof}{(c)}
By Lemma \ref{4.1.11}(a), \(a > b \implies a - b\) is positive.
Let \(a - b = d\).
So
\begin{align*}
& a - b = d \\
\implies & (a - b) + b = d + b & \text{(by Lemma \ref{4.1.3})} \\
\implies & a + (-b + b) = d + b & \text{(by Proposition \ref{4.1.6})} \\
\implies & a = d + b & \text{(by Proposition \ref{4.1.6})} \\
\implies & a = b + d & \text{(by Proposition \ref{4.1.6})} \\
\implies & ac = (b + d)c & \text{(by Lemma \ref{4.1.3})} \\
\implies & ac = bc + dc & \text{(by Proposition \ref{4.1.6})} \\
\implies & ac + (-bc) = (bc + dc) + (-bc) & \text{(by Lemma \ref{4.1.3})} \\
\implies & ac - bc = (bc + dc) + (-bc) \\
\implies & ac - bc = (dc + bc) + (-bc) & \text{(by Proposition \ref{4.1.6})} \\
\implies & ac - bc = dc + (bc + (-bc)) & \text{(by Proposition \ref{4.1.6})} \\
\implies & ac - bc = dc + 0 & \text{(by Proposition \ref{4.1.6})} \\
\implies & ac - bc = dc. & \text{(by Proposition \ref{4.1.6})}
\end{align*}
By Lemma \ref{2.3.3}, \(dc\) is also positive.
Thus by Lemma \ref{4.1.11}(a), \(ac > bc\).
\end{proof}

\begin{proof}{(d)}
\begin{align*}
& a > b \\
\implies & a + (-a) > b + (-a) & \text{(by Lemma \ref{4.1.11}(b))} \\
\implies & 0 > b + (-a) & \text{(by Proposition \ref{4.1.6})} \\
\implies & (-b) + 0 > (-b) + (b + (-a)) & \text{(by Lemma \ref{4.1.11}(b))} \\
\implies & -b > (-b) + (b + (-a)) & \text{(by Proposition \ref{4.1.6})} \\
\implies & -b > ((-b) + b) + (-a) & \text{(by Proposition \ref{4.1.6})} \\
\implies & -b > 0 + (-a) & \text{(by Proposition \ref{4.1.6})} \\
\implies & -b > -a. & \text{(by Proposition \ref{4.1.6})} \\
\end{align*}
\end{proof}

\begin{proof}{(e)}
By Lemma \ref{4.1.11}(a), \(a > b \implies a - b\) is positive, \(b > c \implies b - c\) is positive.
Let \(a - b = d_{ab}\), \(b - c = d_{bc}\).
So
\begin{align*}
& b - c = d_{bc} \\
\implies & (b - c) + c = d_{bc} + c & \text{(by Lemma \ref{4.1.3})} \\
\implies & b + (-c + c) = d_{bc} + c & \text{(by Proposition \ref{4.1.6})} \\
\implies & b + 0 = d_{bc} + c & \text{(by Proposition \ref{4.1.6})} \\
\implies & b = d_{bc} + c & \text{(by Proposition \ref{4.1.6})} \\
\implies & b = c + d_{bc}. & \text{(by Proposition \ref{4.1.6})} \\
& a - b = d \\
\implies & (a - b) + b = d_{ab} + b & \text{(by Lemma \ref{4.1.3})} \\
\implies & a + (-b + b) = d_{ab} + b & \text{(by Proposition \ref{4.1.6})} \\
\implies & a + 0 = d_{ab} + b & \text{(by Proposition \ref{4.1.6})} \\
\implies & a = d_{ab} + b & \text{(by Proposition \ref{4.1.6})} \\
\implies & a = b + d_{ab} & \text{(by Proposition \ref{4.1.6})} \\
\implies & a = (c + d_{bc}) + d_{ab} & \text{(by the given condition)} \\
\implies & a = c + (d_{bc} + d_{ab}) & \text{(by Proposition \ref{4.1.6})} \\
\implies & (-c) + a = (-c) + (c + (d_{bc} + d_{ab})) & \text{(by Lemma \ref{4.1.3})} \\
\implies & a + (-c) = (-c) + (c + (d_{bc} + d_{ab})) & \text{(by Proposition \ref{4.1.6})} \\
\implies & a - c = (-c) + (c + (d_{bc} + d_{ab})) & \text{(by Proposition \ref{4.1.6})} \\
\implies & a - c = ((-c) + c) + (d_{bc} + d_{ab}) & \text{(by Proposition \ref{4.1.6})} \\
\implies & a - c = 0 + (d_{bc} + d_{ab}) & \text{(by Proposition \ref{4.1.6})} \\
\implies & a - c = d_{bc} + d_{ab} & \text{(by Proposition \ref{4.1.6})} \\
\end{align*}
By Proposition \ref{2.2.8}, \(d_{bc} + d_{ab}\) is also positive.
Thus by Definition \ref{4.1.11}(a), \(a > c\).
\end{proof}

\begin{proof}{(f)}
By Lemma \ref{4.1.5}, \(a - b\) can be exactly one of the following three statements:
\begin{enumerate}[label=(\roman*)]
    \item \(a - b = 0\).
    So
    \begin{align*}
        & a - b = 0 \\
        \iff & (a - b) + b = b & \text{(by Lemma \ref{4.1.3})} \\
        \iff & a + (-b + b) = b & \text{(by Proposition \ref{4.1.6})} \\
        \iff & a + 0 = b & \text{(by Proposition \ref{4.1.6})} \\
        \iff & a = b. & \text{(by Proposition \ref{4.1.6})}
    \end{align*}
    \item \(a - b\) is a positive natural number, which means \(a > b\) by Lemma \ref{4.1.11}(a).
    \item \(a - b = -d\), where \(d\) is a positive natural number.
    So
    \begin{align*}
        & a - b = -d \\
        \iff & a + (-b) = -d \\
        \iff & (a + (-b)) + b = (-d) + b & \text{(by Lemma \ref{4.1.3})} \\
        \iff & a + ((-b) + b) = (-d) + b & \text{(by Proposition \ref{4.1.6})} \\
        \iff & a + 0 = (-d) + b & \text{(by Proposition \ref{4.1.6})} \\
        \iff & a = (-d) + b & \text{(by Proposition \ref{4.1.6})} \\
        \iff & a + (-a) = ((-d) + b) + (-a) & \text{(by Lemma \ref{4.1.3})} \\
        \iff & 0 = ((-d) + b) + (-a) & \text{(by Proposition \ref{4.1.6})} \\
        \iff & d + 0 = d + (((-d) + b) + (-a)) & \text{(by Lemma \ref{4.1.3})} \\
        \iff & d = d + (((-d) + b) + (-a)) & \text{(by Proposition \ref{4.1.6})} \\
        \iff & d = (d + ((-d) + b)) + (-a) & \text{(by Proposition \ref{4.1.6})} \\
        \iff & d = ((d + (-d)) + b) + (-a) & \text{(by Proposition \ref{4.1.6})} \\
        \iff & d = (0 + b) + (-a) & \text{(by Proposition \ref{4.1.6})} \\
        \iff & d = b + (-a) & \text{(by Proposition \ref{4.1.6})} \\
        \iff & d = b - a \\
        \iff & b > a & \text{(by Lemma \ref{4.1.11}(a))} \\
        \iff & a < b. & \text{(by Definition \ref{4.1.10})} \\
    \end{align*}
\end{enumerate}
\end{proof}

\exercisesection

\begin{exercise}\label{ex 4.1.1}
Verify that the definition of equality on the integers is both reflexive and symmetric.
\end{exercise}

\begin{proof}
See Additional Corollary \ref{ac 4.1.1}.
\end{proof}

\begin{exercise}\label{ex 4.1.2}
Show that the definition of negation on the integers is well-defined in the sense that if \((a \text{-----} b) = (a' \text{-----} b')\), then \(-(a \text{-----} b) = -(a' \text{-----} b')\)
(so equal integers have equal negations).
\end{exercise}

\begin{proof}
See Additional Corollary \ref{ac 4.1.2}.
\end{proof}

\begin{exercise}\label{ex 4.1.3}
Show that \((-1) \times a = -a\) for every integer \(a\).
\end{exercise}

\begin{proof}
Let \(a = a_1 \text{-----} a_2\).
So
\begin{align*}
(-1) \times a &= (0 \text{-----} 1) \times (a_1 \text{-----} a_2) \\
&= (0a_1 + 1a_2) \text{-----} (0a_2 + 1a_1) & \text{(by Definition \ref{4.1.2})} \\
&= (0 + 1a_2) \text{-----} (0 + 1a_1) & \text{(by Definition \ref{2.3.1})} \\
&= 1a_2 \text{-----} 1a_1 & \text{(by Definition \ref{2.2.1})} \\
&= (0a_2 + a_2) \text{-----} (0a_1 + a_1) & \text{(by Definition \ref{2.3.1})} \\
&= (0 + a_2) \text{-----} (0 + a_1) & \text{(by Definition \ref{2.3.1})} \\
&= a_2 \text{-----} a_1 & \text{(by Definition \ref{2.2.1})} \\
&= -a.
\end{align*}
\end{proof}

\begin{exercise}\label{ex 4.1.4}
Prove the remaining identities in Proposition \ref{4.1.6}.
\end{exercise}

\begin{proof}
See Proposition \ref{4.1.6}.
\end{proof}

\begin{exercise}\label{ex 4.1.5}
Prove Proposition \ref{4.1.8}.
\end{exercise}

\begin{proof}
See Proposition \ref{4.1.8}.
\end{proof}

\begin{exercise}\label{ex 4.1.6}
Prove Corollary \ref{4.1.9}.
\end{exercise}

\begin{proof}
See Corollary \ref{4.1.9}.
\end{proof}

\begin{exercise}\label{ex 4.1.7}
Prove Lemma \ref{4.1.11}.
\end{exercise}

\begin{proof}
See Lemma \ref{4.1.11}.
\end{proof}

\begin{exercise}\label{ex 4.1.8}
Show that the principle of induction (Axiom \ref{2.5}) does not apply directly to the integers.
More precisely, give an example of a property \(P(n)\) pertaining to an integer \(n\) such that \(P(0)\) is true, and that \(P(n)\) implies \(P(n++)\) for all integers \(n\), but that \(P(n)\) is not true for all integers \(n\).
Thus induction is not as useful a tool for dealing with the integers as it is with the natural numbers.
(The situation becomes even worse with the rational and real numbers, which we shall define shortly.)
\end{exercise}

\begin{proof}
For sake of contradiction, we claim that \(\forall\ n \in \mathds{Z}\), \(n + 1 > 0\).
And we use Axiom \ref{2.5} to prove the above claim.
For \(n = 0\), \((0 + 1 = 1) \land (1 \neq 0) \implies 1 > 0\), so the base case holds.
Suppose inductively that for some \(n\) the statement \(n + 1 > 0\) is true.
Then for \(n++\), \((n++) + 1 = (n + 1) + 1\).
Because by induction hypothesis, \(n + 1 > 0\), so by Lemma \ref{4.1.11}, \((n + 1) > 0 \implies (n + 1) + 1 > 0 + 1 = 1\).
Again by Lemma \ref{4.1.11}, \(((n + 1) + 1 > 1) \land (1 > 0) \implies (n + 1) + 1 > 0\), and this close induction.

But if \(n = -1\), then \(-1 + 1 = 0\) by Proposition \ref{4.1.6}.
By Definition \ref{4.1.10}, \(0 \geq 0\), and By Lemma \ref{4.1.11}, \(0 > 0\) is false because \(0 = 0\) is true.
So \(n = -1\) does not satisified the above claim, which means induction is not as useful a tool for dealing with the integers as it is with the natural numbers.
\end{proof}
\section{The rationals}\label{sec 4.2}

\begin{definition}\label{4.2.1}
    A \emph{rational number} is an expression of the form \(a // b\), where \(a\) and \(b\) are integers and \(b\) is non-zero;
    \(a // 0\) is not considered to be a rational number.
    Two rational numbers are considered to be equal, \(a // b = c // d\), if and only if \(ad = cb\).
    The set of all rational numbers is denoted \(\Q\).
\end{definition}

\begin{note}
    There is no reasonable way we can divide by zero, since one cannot have both the identities \((a / b) \times b = a\) and \(c \times 0 = 0\) hold simultaneously if \(b\) is allowed to be zero and \(a\) is non-zero.
    Similarly, the identities \(a / a = 1\) and \(2 \times (a / a) = (2 \times a) / a\) cannot hold simultaneously if \(0 / 0\) is defined.
    However, we can eventually get a reasonable notion of dividing by a quantity which approaches zero
    - think of L'H\^opital's rule (see Section \ref{sec 10.5}), which suffices for doing things like defining differentiation.
\end{note}

\begin{additional corollary}\label{ac 4.2.1}
The definition of equality for the rational numbers is reflexive, symmetric and transitive.
\end{additional corollary}

\begin{proof}
    Let \(a // b\), \(c // d\), \(e // f\) be rational numbers where \(a, b, c, d, e, f \in \Z\) and \(b, d, f \neq 0\).
    Since
    \begin{align*}
                 & ab = ab          & \text{(by Lemma \ref{4.1.3})}      \\
        \implies & a // b = a // b, & \text{(by Definition \ref{4.2.1})}
    \end{align*}
    we know that Definition \ref{4.2.1} is reflexive.

    Now suppose that \(a // b = c // d\).
    Then we have
    \begin{align*}
                 & a // b = c // d                                       \\
        \implies & ad = cb          & \text{(by Definition \ref{4.2.1})} \\
        \implies & cb = ad          & \text{(by Lemma \ref{4.1.3})}      \\
        \implies & c // d = a // b. & \text{(by Definition \ref{4.2.1})}
    \end{align*}
    Thus Definition \ref{4.2.1} is symmetric.

    Finally suppose that \(a // b = c // d\) and \(c // d = e // f\).
    Then we have
    \begin{align*}
                 & (a // b = c // d) \land (c // d) \land (e // f)                                       \\
        \implies & (ad = cb) \land (cf = ed)                       & \text{(by Definition \ref{4.2.1})}  \\
        \implies & (adf = cbf) \land (cfb = edb)                   & \text{(by Lemma \ref{4.1.3})}       \\
        \implies & (afd = cbf) \land (cbf = ebd)                   & \text{(by Proposition \ref{4.1.6})} \\
        \implies & afd = ebd                                       & \text{(by Lemma \ref{4.1.3})}       \\
        \implies & af = eb                                         & \text{(by Corollary \ref{4.1.9})}   \\
        \implies & a // b = e // f.                                & \text{(by Definition \ref{4.2.1})}
    \end{align*}
    Thus Definition \ref{4.2.1} is transitive.
\end{proof}

\begin{definition}\label{4.2.2}
    If \(a // b\) and \(c // d\) are rational numbers, we define their sum
    \[
        (a // b) + (c // d) \coloneqq (ad + bc) // (bd)
    \]
    their product
    \[
        (a // b) \times (c // d) \coloneqq (ac) // (bd)
    \]
    and the negation
    \[
        -(a // b) \coloneqq (-a) // b.
    \]
\end{definition}

\begin{note}
    If \(b\) and \(d\) are non-zero, then \(bd\) is also non-zero, by Proposition \ref{4.1.8}, so the sum or product of two rational numbers remains a rational number.
\end{note}

\begin{lemma}\label{4.2.3}
    The sum, product, and negation operations on rational numbers are well-defined, in the sense that if one replaces \(a // b\) with another rational number \(a' // b'\) which is equal to \(a // b\), then the output of the above operations remains unchanged, and similarly for \(c // d\).
\end{lemma}

\begin{proof}
    We first show that the addition on rationals numbers is well-defined.
    Suppose \(a // b = a' // b'\), so that \(b\) and \(b'\) are non-zero and \(ab' = a'b\).
    We now show that \((a // b) + (c // d) = (a' // b') + (c // d)\).
    By Definition \ref{4.2.2}, the left-hand side is \((ad + bc) // bd\) and the right-hand side is \((a'd + b'c) // b'd\).
    So by Definition \ref{4.2.1}, we have to show that
    \[
        (ad + bc)b'd = (a'd + b'c)bd,
    \]
    which expands to
    \[
        ab'd^2 + bb'cd = a'bd^2 + bb'cd.
    \]
    But since \(ab' = a'b\), the claim follows.
    Similarly suppose \(c // d = c' // d'\), so that \(d\) and \(d'\) are non-zero and \(cd' = c'd\).
    We now show that \((a // b) + (c // d) = (a // b) + (c' // d')\).
    By Definition \ref{4.2.2}, the left-hand side is \((ad + bc) // bd\) and the right-hand side is \((ad' + bc') // bd'\).
    So by Definition \ref{4.2.1}, we have to show that
    \[
        (ad + bc)bd' = (ad' + bc')bd,
    \]
    which expands to
    \[
        abdd' + b^2cd' = abdd' + b^2c'd.
    \]
    But since \(cd' = c'd\), the claim follows.

    Now we show that the multiplication on rationals numbers is well-defined.
    Suppose \(a // b = a' // b'\), so that \(b\) and \(b'\) are non-zero and \(ab' = a'b\).
    We now show that \((a // b) \times (c // d) = (a' // b') \times (c // d)\).
    By Definition \ref{4.2.2}, the left-hand side is \((ac) // (bd)\) and the right-hand side is \((a'c) // (b'd)\).
    So by Definition \ref{4.2.1}, we have to show that
    \[
        (ac)(b'd) = (a'c)(bd),
    \]
    which is equivalent to
    \[
        ab'cd = a'bcd.
    \]
    But since \(ab' = a'b\), the claim follows.
    Similarly suppose \(c // d = c' // d'\), so that \(d\) and \(d'\) are non-zero and \(cd' = c'd\).
    We now show that \((a // b) \times (c // d) = (a // b) \times (c' // d')\).
    By Definition \ref{4.2.2}, the left-hand side is \((ac) // (bd)\) and the right-hand side is \((ac') // (bd')\).
    So by Definition \ref{4.2.1}, we have to show that
    \[
        (ac)(bd') = (ac')(bd),
    \]
    which is equivalent to
    \[
        abcd' = abc'd.
    \]
    But since \(cd' = c'd\), the claim follows.

    Finally we show that the negation on rationals numbers is well-defined.
    Suppose \(a // b = a' // b'\), so that \(b\) and \(b'\) are non-zero and \(ab' = a'b\).
    We now show that \(-(a // b) = -(a' // b')\).
    By Definition \ref{4.2.2}, the left-hand side is \((-a) // b\) and the right-hand side is \((-a') // b'\).
    So by Definition \ref{4.2.1}, we have to show that
    \[
        (-a)b' = (-a')b,
    \]
    which by Exercise \ref{ex 4.1.3} is equivalent to
    \[
        (-1)ab' = (-1)a'b.
    \]
    But since \(ab' = a'b\), the claim follows.
\end{proof}

\begin{note}
    The rational numbers \(a // 1\) behave in a manner identical to the integers \(a\):
    \begin{align*}
        (a // 1) + (b // 1)      & = (a + b) // 1; \\
        (a // 1) \times (b // 1) & = (ab // 1);    \\
        -(a // 1)                & = (-a) // 1.
    \end{align*}
    Also, \(a // 1\) and \(b // 1\) are only equal when \(a\) and \(b\) are equal.
    Because of this, we will identify \(a\) with \(a // 1\) for each integer \(a\): \(a \equiv a // 1\);
    the above identities then guarantee that the arithmetic of the integers is consistent with the arithmetic of the rationals.
    Thus just as we embedded the natural numbers inside the integers, we embed the integers inside the rational numbers.
    In particular, all natural numbers are rational numbers, for instance \(0\) is equal to \(0 // 1\) and \(1\) is equal to \(1 // 1\).
\end{note}

\begin{note}
    Observe that a rational number \(a // b\) is equal to \(0 = 0 // 1\) if and only if \(a \times 1 = b \times 0\), i.e., if the numerator \(a\) is equal to \(0\).
    Thus if \(a\) and \(b\) are non-zero then so is \(a // b\).
\end{note}

\begin{note}
    We now define a new operation on the rationals: reciprocal.
    If \(x = a // b\) is a non-zero rational (so that \(a, b \neq 0\)) then we define the \emph{reciprocal} \(x^{-1}\) of \(x\) to be the rational number \(x^{-1} \coloneqq b // a\).
\end{note}

\begin{additional corollary}\label{ac 4.2.2}
The reciprocal operation on rational numbers is consistent with Definition \ref{4.2.1}:
if two rational numbers \(a // b\), \(a' // b'\) are equal, then their reciprocals are also equal.
We however leave the reciprocal of \(0\) undefined.
\end{additional corollary}

\begin{proof}
    By Definition \ref{4.2.1} and the definition of reciprocal, we have \(a, a', b, b' \neq 0\).
    Then we have
    \begin{align*}
                 & a // b = a' // b'                                        \\
        \implies & ab' = a'b          & \text{(by Definition \ref{4.2.1})}  \\
        \implies & b'a = ba'          & \text{(by Proposition \ref{4.1.6})} \\
        \implies & b' // a' = b // a. & \text{(by Definition \ref{4.2.1})}
    \end{align*}
\end{proof}

\begin{note}
    In contrast to reciprocal, an operation such as ``numerator'' is not well-defined:
    the rationals \(3 // 4\) and \(6 // 8\) are equal, but have unequal numerators, so we have to be careful when referring to such terms as ``the numerator of \(x\)''.
\end{note}

\begin{proposition}[Laws of algebra for rationals]\label{4.2.4}
    Let \(x\), \(y\), \(z\) be rationals.
    Then the following laws of algebra hold:
    \begin{align*}
        x + y               & = y + x       \\
        (x + y) + z         & = x + (y + z) \\
        x + 0 = 0 + x       & = x           \\
        x + (-x) = (-x) + x & = 0           \\
        xy                  & = yx          \\
        (xy)z               & = x(yz)       \\
        x1 = 1x             & = x           \\
        x(y + z)            & = xy + xz     \\
        (y + z)x            & = yx + zx.
    \end{align*}
    If \(x\) is non-zero, we also have
    \[
        xx^{-1} = x^{-1}x = 1.
    \]
\end{proposition}

\begin{proof}
    To prove this identity, one writes \(x = a // b\), \(y = c // d\), \(z = e // f\) for some integers \(a\), \(c\), \(e\) and non-zero integers \(b\), \(d\), \(f\), and verifies each identity in turn using the algebra of the integers.

    First we show that \(x + y = y + x\).
    \begin{align*}
        x + y & = (a // b) + (c // d)                                       \\
              & = (ad + bc) // bd     & \text{(by Definition \ref{4.2.2})}  \\
              & = (bc + ad) // bd     & \text{(by Proposition \ref{4.1.6})} \\
              & = (cb + da) // db     & \text{(by Proposition \ref{4.1.6})} \\
              & = (c // d) + (a // b) & \text{(by Definition \ref{4.2.2})}  \\
              & = y + x.
    \end{align*}
    Thus the addition on rationals is commutative.

    Next we show that \((x + y) + z = x + (y + z)\).
    \begin{align*}
        (x + y) + z & = \big((a // b) + (c // d)\big) + (e // f)                                                 \\
                    & = \big((ad + bc) // bd\big) + (e // f)               & \text{(by Definition \ref{4.2.2})}  \\
                    & = \big((ad + bc)f + (bd)e\big) // \big((bd)f\big)    & \text{(by Definition \ref{4.2.2})}  \\
                    & = \big((ad)f + (bc)f + (bd)e\big) // \big((bd)f\big) & \text{(by Proposition \ref{4.1.6})} \\
                    & = \big(a(df) + b(cf) + b(de)\big) // \big(b(df)\big) & \text{(by Proposition \ref{4.1.6})} \\
                    & = \big(a(df) + b(cf + de)\big) // \big(b(df)\big)    & \text{(by Proposition \ref{4.1.6})} \\
                    & = (a // b) + \big((cf + de) // df\big)               & \text{(by Definition \ref{4.2.2})}  \\
                    & = (a // b) + \big((c // d) + (e // f)\big)           & \text{(by Definition \ref{4.2.2})}  \\
                    & = x + (y + z).
    \end{align*}
    Thus the addition on rationals is associative.

    Next we show that \(x + 0 = 0 + x = x\).
    Since the addition on rationals is commutative, we know that \(x + 0 = 0 + x\).
    Thus we only need to show that \(x + 0 = x\).
    \begin{align*}
        x + 0 & = (a // b) + (0 // 1)                                       \\
              & = (a1 + b0) // b1     & \text{(by Definition \ref{4.2.2})}  \\
              & = (a + 0) // b        & \text{(by Proposition \ref{4.1.6})} \\
              & = a // b              & \text{(by Proposition \ref{4.1.6})} \\
              & = x.
    \end{align*}
    Thus \(0\) is the additive identity on rationals.

    Next we show that \(x + (-x) = (-x) + x = 0\).
    Since the addition on rationals is commutative, we know that \(x + (-x) = (-x) + x\).
    Thus we only need to show that \(x + (-x) = 0\).
    \begin{align*}
        x + (-x) & = (a // b) + ((-a) // b) & \text{(by Definition \ref{4.2.2})}  \\
                 & = (ab + b(-a)) // b^2    & \text{(by Definition \ref{4.2.2})}  \\
                 & = (ab + (-a)b) // b^2    & \text{(by Proposition \ref{4.1.6})} \\
                 & = (ab + ((-1)a)b) // b^2 & \text{(by Exercise \ref{ex 4.1.3})} \\
                 & = (ab + (-1)(ab)) // b^2 & \text{(by Proposition \ref{4.1.6})} \\
                 & = (ab + (-(ab)) // b^2   & \text{(by Exercise \ref{ex 4.1.3})} \\
                 & = 0 // b^2               & \text{(by Proposition \ref{4.1.6})} \\
                 & = 0.
    \end{align*}
    Thus the additive inverse of rational \(x\) is \(-x\).

    Next we show that \(xy = yx\).
    \begin{align*}
        xy & = (a // b) \times (c // d)                                       \\
           & = ac // bd                 & \text{(by Definition \ref{4.2.2})}  \\
           & = ca // db                 & \text{(by Proposition \ref{4.1.6})} \\
           & = (c // d) \times (a // b) & \text{(by Definition \ref{4.2.2})}  \\
           & = yx.
    \end{align*}
    Thus the multiplication on rationals is commutative.

    Next we show that \((xy)z = x(yz)\).
    \begin{align*}
        (xy)z & = \big((a // b) \times (c // d)\big) \times (e // f)                                       \\
              & = (ac // bd) \times (e // f)                         & \text{(by Definition \ref{4.2.2})}  \\
              & = \big((ac)e\big) // \big((bd)f\big)                 & \text{(by Definition \ref{4.2.2})}  \\
              & = \big(a(ce)\big) // \big(b(df)\big)                 & \text{(by Proposition \ref{4.1.6})} \\
              & = (a // b) \times (ce // df)                         & \text{(by Definition \ref{4.2.2})}  \\
              & = (a // b) \times \big((c // d) \times (e // f)\big) & \text{(by Definition \ref{4.2.2})}  \\
              & = x(yz).
    \end{align*}
    Thus the multiplication on rationals is associative.

    Next we show that \(x1 = 1x = x\).
    Since the multiplication on rationals is commutative, we know that \(x1 = 1x\).
    Thus we only need to show that \(x1 = x\).
    \begin{align*}
        x1 & = (a // b) \times (1 // 1)                                       \\
           & = a1 // b1                 & \text{(by Definition \ref{4.2.2})}  \\
           & = a // b                   & \text{(by Proposition \ref{4.1.6})} \\
           & = x.
    \end{align*}
    Thus \(1\) is the multiplicative identity on rationals.

    Next we show that \(x(y + z) = xy + xz\).
    \begin{align*}
        x(y + z) & = (a // b) \times \big((c // d) + (e // f)\big)                                                                 \\
                 & = (a // b) \times \big((cf + de) // df\big)                               & \text{(by Definition \ref{4.2.2})}  \\
                 & = \big(a(cf + de)\big) // \big(b(df)\big)                                 & \text{(by Definition \ref{4.2.2})}  \\
                 & = \Big(b\big(a(cf + de)\big)\Big) // \big(b^2(df)\big)                    & \text{(by Lemma \ref{4.2.3})}       \\
                 & = \big((ba)(cf + de)\big) // \big(b^2(df)\big)                            & \text{(by Proposition \ref{4.1.6})} \\
                 & = \big((ba)(cf) + (ba)(de)\big) // \big(b^2(df)\big)                      & \text{(by Proposition \ref{4.2.6})} \\
                 & = \big((ab)(fc) + (ba)(ed)\big) // \big(b^2(fd)\big)                      & \text{(by Proposition \ref{4.2.6})} \\
                 & = \big(a(bf)c + b(ae)d\big) // \big(b(bf)d\big)                           & \text{(by Proposition \ref{4.2.6})} \\
                 & = \big((ac)(bf) + (bd)(ae)\big) // \big((bd)(bf)\big)                     & \text{(by Proposition \ref{4.2.6})} \\
                 & = (ac // bd) + (ae // bf)                                                 & \text{(by Definition \ref{4.2.2})}  \\
                 & = \big((a // b) \times (c // d)\big) + \big((a // b) \times (e // f)\big) & \text{(by Definition \ref{4.2.2})}  \\
                 & = xy + xz.
    \end{align*}
    Thus the multiplication and addition on rationals are left distributive.

    Next we show that \((y + z)x = yx + zx\).
    \begin{align*}
        (y + z)x & = x(y + z) & \text{(multiplication is commutative)}                     \\
                 & = xy + xz  & \text{(multiplication and addition are left distributive)} \\
                 & = yx + zx. & \text{(multiplication is commutative)}
    \end{align*}
    Thus the multiplication and addition on rationals are right distributive.

    Finally we show that \(xx^{-1} = x^{-1}x = 1\).
    Since the multiplication on rationals is commutative, we know that \(xx^{-1} = x^{-1}x\).
    Thus we only need to show that \(xx^{-1} = 1\).
    \begin{align*}
        xx^{-1} & = (a // b) \times (b // a)                                       \\
                & = ab // ba                 & \text{(by Definition \ref{4.2.2})}  \\
                & = ab // ab                 & \text{(by Proposition \ref{4.1.6})} \\
                & = 1 // 1                   & \text{(by Definition \ref{4.2.1})}  \\
                & = 1.
    \end{align*}
    Thus the multiplicative inverse of rational \(x\) is \(x^{-1}\).
\end{proof}

\begin{remark}\label{4.2.5}
    The above set (Proposition \ref{4.2.4}) of ten identities have a name;
    they are asserting that the rationals \(\Q\) form a \emph{field}.
    This is better than being a commutative ring because of the tenth identity \(xx^{-1} = x^{-1}x = 1\).
    Note that Proposition \ref{4.2.4} supercedes Proposition \ref{4.1.6}.
\end{remark}

\begin{note}
    We can now define the \emph{quotient} \(x / y\) of two rational numbers \(x\) and \(y\), \emph{provided that} \(y\) is non-zero, by the formula
    \[
        x / y \coloneqq x \times y^{-1}.
    \]
\end{note}

\begin{note}
    Using the above formula, it is easy to see that \(a / b = a // b\) for every integer \(a\) and every non-zero integer \(b\).
    Thus we can now discard the \(//\) notation, and use the more customary \(a / b\) instead of \(a // b\).
\end{note}

\begin{note}
    In a similar spirit, we define subtraction on the rationals by the formula
    \[
        x - y \coloneqq x + (-y),
    \]
    just as we did with the integers.
\end{note}

\begin{definition}\label{4.2.6}
    A rational number \(x\) is said to be \emph{positive} iff we have \(x = a / b\) for some positive integers \(a\) and \(b\).
    It is said to be \emph{negative} iff we have \(x = -y\) for some positive rational \(y\)
    (i.e., \(x = (-a) / b\) for some positive integers \(a\) and \(b\)).
\end{definition}

\begin{note}
    Thus for instance, every positive integer is a positive rational number, and every negative integer is a negative rational number, so our new definition is consistent with our old one.
\end{note}

\begin{additional corollary}\label{ac 4.2.3}
Let \(x = a / b\) be a rational number where \(a, b \in \Z\) and \(b \neq 0\).
Then
\[
    -x = (-a) / b = a / (-b) = (-1)(a / b) = (-1)x.
\]
\end{additional corollary}

\begin{proof}
    \begin{align*}
        -x & = -(a / b)                                                        \\
           & = (-a) / b                  & \text{(by Definition \ref{4.2.2})}  \\
           & = \big((-1)a\big) / b       & \text{(by Exercise \ref{ex 4.1.3})} \\
           & = \big((-1)a\big) / 1b      & \text{(by Proposition \ref{4.1.6})} \\
           & = ((-1) / 1) \times (a / b) & \text{(by Definition \ref{4.2.2})}  \\
           & = (-1)(a / b)                                                     \\
           & = (-1)x                                                           \\
           & = (1 / (-1)) \times (a / b) & \text{(by Definition \ref{4.2.1})}  \\
           & = 1a / (-1)b                & \text{(by Definition \ref{4.2.2})}  \\
           & = a / (-1)b                 & \text{(by Proposition \ref{4.1.6})} \\
           & = a / (-b).                 & \text{(by Exercise \ref{ex 4.1.3})}
    \end{align*}
\end{proof}

\begin{lemma}[Trichotomy of rationals]\label{4.2.7}
    Let \(x\) be a rational number.
    Then exactly one of the following three statements is true:
    \begin{enumerate}
        \item \(x\) is equal to \(0\).
        \item \(x\) is a positive rational number.
        \item \(x\) is a negative rational number.
    \end{enumerate}
\end{lemma}

\begin{proof}
    We first show that at least one of (a), (b), (c) is true.
    Let \(x = a / b\), where \(a, b \in \Z\) and \(b \neq 0\).
    By Lemma \ref{4.1.11}, \(a\) can only satisified one of the following three statements:
    \(a = 0\), \(a > 0\) and \(a < 0\).
    Similarly, \(b\) can only satisified one of the following two statements:
    \(b > 0\) and \(b < 0\).
    We first consider \(a\):
    \begin{itemize}
        \item If \(a = 0\), then \(x = 0 / b = 0\).
        \item If \(a > 0\), then we need to consider \(b\):
              \begin{itemize}
                  \item If \(b > 0\), then by Definition \ref{4.2.6}, \(x\) is positive.
                  \item If \(b < 0\), then by Definition \ref{4.1.4} \(b = -c\) for some \(c \in \Z^+\).
                        Thus by Additional Corollary \ref{ac 4.2.3} we have \(a / b = a / (-c) = (-a) / c\), which means \(x\) is negative by Definition \ref{4.2.6}.
              \end{itemize}
        \item If \(a < 0\), then by Definition \ref{4.1.4} \(a = -c\) for some \(c \in \Z^+\).
              Now we consider \(b\):
              \begin{itemize}
                  \item If \(b > 0\), then \(a / b = (-c) / b\), which means \(x\) is negative by Definition \ref{4.2.6}.
                  \item If \(b < 0\), then by Definition \ref{4.1.4} \(b = -d\) for some \(d \in \Z^+\).
                        Thus by Definition \ref{4.2.1} we have \(a / b = (-c) / (-d) = (-1) / (-1) \times (c / d) = c / d\), which means \(x\) is positive by Definition \ref{4.2.6}.
              \end{itemize}
    \end{itemize}
    From all cases above we conclude that at least one of (a), (b), (c) is true.

    Now we show that at most one of (a), (b), (c) is true.
    \begin{itemize}
        \item If \(x\) is both positive and \(0\), then by Definition \ref{4.2.6} and \ref{4.2.1}, \(x = a / b = 0 / 1\), where \(a, b \in \Z^+\).
              But \(a / b = 0 / 1\) means \(a = 0\), contradicted to \(a\) is positive.
        \item If \(x\) is both negative and \(0\), then by Definition \ref{4.2.6} and \ref{4.2.1}, \(x = (-a) / b = 0 / 1\), where \(a, b \in \Z^+\).
              But \((-a) / b = 0 / 1\) means \(-a = 0\), contradicted to \(a\) is positive.
        \item If \(x\) is both positive and negative, then by Definition \ref{4.2.6} and \ref{4.2.1}, \(x = a / b = (-c) / d\), where \(a, b, c, d \in \Z^+\).
              But \(a / b = (-c) / d\) means \(ad = b(-c) = b((-1)c) = (b(-1))c = ((-1)b)c = (-1)(bc)\).
              By Lemma \ref{2.3.3} we know that \(ad\) and \(bc\) are positive.
              But by Exercise \ref{ex 4.1.3} and Definition \ref{4.1.4} we know that \((-1)(bc) = -(bc)\) is negative, a contradiction.
    \end{itemize}
    From all cases above we conclude that no more than one of (a), (b), (c) is true at the same time.
\end{proof}

\begin{definition}[Ordering of the rationals]\label{4.2.8}
    Let \(x\) and \(y\) be rational numbers.
    We say that \(x > y\) iff \(x - y\) is a positive rational number, and \(x < y\) iff \(x - y\) is a negative rational number.
    We write \(x \geq y\) iff either \(x > y\) or \(x = y\), and similarly define \(x \leq y\) iff either \(x < y\) or \(x = y\).
\end{definition}

\begin{additional corollary}\label{ac 4.2.4}
If \(x\) and \(y\) are two positive rationals, then \(x + y\) is also a positive rational number.
If \(x\) and \(y\) are two negative rationals, then \(x + y\) is also a negative rational number.
\end{additional corollary}

\begin{proof}
    We first show that if \(x\) and \(y\) are two positive rationals, then \(x + y\) is also positive.
    By Definition \ref{4.2.6} we have \(x = a / b\) and \(y = c / d\) where \(a, b, c, d \in \Z^+\).
    Then by Definition \ref{4.2.2} we have \(x + y = (ad + bc) / bd\).
    By Lemma \ref{2.3.2} we know that \(ad, bc, bd \in \Z^+\).
    Since \(ad, bc \in \Z^+\), by Proposition \ref{2.2.8} we know that \(ad + bc \in \Z^+\).
    Thus by Definition \ref{4.2.6} we know that \(x + y = (ad + bc) / bd\) is a positive rational number.

    Now we show that if \(x\) and \(y\) are two negative rationals, then \(x + y\) is also negative.
    By Definition \ref{4.2.6} we have \(x = (-a) / b\) and \(y = (-c) / d\) where \(a, b, c, d \in \Z^+\).
    Then by Definition \ref{4.2.2} we have \(x + y = ((-a)d + b(-c)) / bd\).
    By Proposition \ref{4.2.4} and Additional Corollary \ref{ac 4.2.3} we have
    \[
        (-a)d + b(-c) = (-a)d + (-c)b = ((-1)a)d + ((-1)c)b = (-1)(ad + cb) = -(ad + cb).
    \]
    By Lemma \ref{2.3.2} we know that \(ad, cb, bd \in \Z^+\).
    Since \(ad, cb \in \Z^+\), by Proposition \ref{2.2.8} we have \(ad + cb \in \Z^+\).
    Thus by Definition \ref{4.1.4} we have \(-(ad + cb) \in \Z^-\) and by Definition \ref{4.2.6}, \(x + y = -(ad + cb) / bd\) is a negative rational number.
\end{proof}

\begin{additional corollary}\label{ac 4.2.5}
Let \(x\) and \(y\) be two rationals.
If \(x\) and \(y\) are positive, then \(xy\) is positive.
If \(x\) and \(y\) are negative, then \(xy\) is positive.
\end{additional corollary}

\begin{proof}
    We first show that if \(x\) and \(y\) are two positive rationals, then \(xy\) is a positive rational number.
    By Definition \ref{4.2.6} we know that \(x = a / b\) and \(y = c / d\) where \(a, b, c, d \in \Z^+\).
    By Definition \ref{4.2.2} we have \(xy = ac / bd\).
    By Lemma \ref{2.3.2} we have \(ac, bd \in \Z^+\), thus by Definition \ref{4.2.6} we know that \(xy\) is a positive rational number.

    Now we show that if \(x\) and \(y\) are two negative rationals, then \(xy\) is a positive rational number.
    By Definition \ref{4.2.6} we know that \(x = (-a) / b\) and \(y = (-c) / d\) where \(a, b, c, d \in \Z^+\).
    By Definition \ref{4.2.2} we have \(xy = (-a)(-c) / bd\).
    By Additional Corollary \ref{ac 4.1.5} we have \((-a)(-c) = ac\).
    By Lemma \ref{2.3.2} we have \(ac, bd \in \Z^+\), thus by Definition \ref{4.2.6} we know that \(xy\) is a positive rational number.
\end{proof}

\begin{additional corollary}\label{ac 4.2.6}
Let \(x\) and \(y\) be two rationals.
If \(x\) is negative and \(y\) is positive, then \(xy\) is negative.
If \(x\) is positive and \(y\) is negative, then \(xy\) is negative.
\end{additional corollary}

\begin{proof}
    By Proposition \ref{4.2.4} we know that \(xy = yx\), thus we only need to show that if \(x\) is negative and \(y\) is positive, then \(xy\) is negative.
    By Definition \ref{4.2.6} we know that \(x = (-a) / b\) and \(y = c / d\) where \(a, b, c, d \in \Z^+\).
    By Definition \ref{4.2.2} we have \(xy = (-a)c / bd\).
    By Additional corollary \ref{ac 4.1.3} we have \((-a)c = -(ac)\).
    By Lemma \ref{2.3.2} we know that \(ac, bd \in \Z^+\), thus by Definition \ref{4.2.6} we know that \(xy\) is a negative rational number.
\end{proof}

\begin{additional corollary}\label{ac 4.2.7}
\(x\) is a positive rational number if and only if \(x > 0\).
\(x\) is a negative rational number if and only if \(x < 0\).
\end{additional corollary}

\begin{proof}
    We first show that \(x\) is a positive rational number if and only if \(x > 0\).
    By Definition \ref{4.2.6} we know that \(x = a / b\) where \(a, b \in \Z^+\).
    Thus
    \begin{align*}
             & x = a / b \in \Q^+         & \text{(by Definition \ref{4.2.6})}              \\
        \iff & x - 0 = a / b - 0 \in \Q^+ & \text{(by Additional Corollary \ref{ac 4.2.3})} \\
        \iff & x > 0.                     & \text{(by Definition \ref{4.2.8})}
    \end{align*}

    Now we show that \(x\) is a negative rational number if and only if \(x < 0\).
    By Definition \ref{4.2.6} we know that \(x = (-a) / b\) where \(a, b \in \Z^+\).
    Thus
    \begin{align*}
             & x = (-a) / b \in \Q^-         & \text{(by Definition \ref{4.2.6})}              \\
        \iff & x - 0 = (-a) / b - 0 \in \Q^- & \text{(by Additional Corollary \ref{ac 4.2.3})} \\
        \iff & x < 0.                        & \text{(by Definition \ref{4.2.8})}
    \end{align*}
\end{proof}

\begin{proposition}[Basic properties of order on the rationals]\label{4.2.9}
    Let \(x\), \(y\), \(z\) be rational numbers.
    Then the following properties hold.
    \begin{enumerate}
        \item (Order trichotomy)
              Exactly one of the three statements \(x = y\), \(x < y\), or \(x > y\) is true.
        \item (Order is anti-symmetric)
              One has \(x < y\) if and only if \(y > x\).
        \item (Order is transitive)
              If \(x < y\) and \(y < z\), then \(x < z\).
        \item (Addition preserves order)
              If \(x < y\), then \(x + z < y + z\).
        \item (Positive multiplication preserves order)
              If \(x < y\) and \(z\) is positive, then \(xz < yz\).
    \end{enumerate}
\end{proposition}

\begin{proof}{(a)}
    By Lemma \ref{4.2.7} \(x - y\) is exactly one of the following three cases:
    \begin{enumerate}[label=(\Roman*)]
        \item \(x - y = 0\).
              Then by Proposition \ref{4.2.4} we have \(x = y\).
        \item \(x - y\) is positive.
              Then by Definition \ref{4.2.8} we have \(x > y\).
        \item \(x - y\) is negative.
              Then by Definition \ref{4.2.8} we have \(x < y\).
    \end{enumerate}
\end{proof}

\begin{proof}{(b)}
    Since
    \begin{align*}
        x - y & = (-1)(-1)(x - y)                 & \text{(by Definition \ref{4.2.2})}              \\
              & = (-1)(-1)\big(x + (-1)y\big)     & \text{(by Additional Corollary \ref{ac 4.2.3})} \\
              & = (-1)\big((-1)x + (-1)(-1)y\big) & \text{(by Proposition \ref{4.2.4})}             \\
              & = (-1)\big((-1)x + y\big)         & \text{(by Definition \ref{4.2.2})}              \\
              & = (-1)\big(y + (-1)x\big)         & \text{(by Proposition \ref{4.2.4})}             \\
              & = (-1)(y - x),                    & \text{(by Additional Corollary \ref{ac 4.2.3})}
    \end{align*}
    we have
    \begin{align*}
             & x < y                                                                                 \\
        \iff & x - y \text{ is negative}           & \text{(by Definition \ref{4.2.8})}              \\
        \iff & (-1)(x - y) \text{ is positive}     & \text{(by Additional Corollary \ref{ac 4.2.5})} \\
        \iff & (-1)(-1)(y - x) \text{ is positive} & \text{(by Additional Corollary \ref{ac 4.2.2})} \\
        \iff & y - x \text{ is positive}           & \text{(by Definition \ref{4.2.2})}              \\
        \iff & y > x.                              & \text{(by Definition \ref{4.2.8})}
    \end{align*}
\end{proof}

\begin{proof}{(c)}
    We have
    \begin{align*}
                 & (x < y) \land (y < z)                                                                                           \\
        \implies & (x - y \text{ is negative}) \land (y - z \text{ is negative}) & \text{(by Definition \ref{4.2.8})}              \\
        \implies & (x - y) + (y - z) \text{ is negative}                         & \text{(by Additional Corollary \ref{ac 4.2.4})} \\
        \implies & x + z \text{ is negative}                                     & \text{(by Proposition \ref{4.2.4})}             \\
        \implies & x < z.                                                        & \text{(by Definition \ref{4.2.8})}
    \end{align*}
\end{proof}

\begin{proof}{(d)}
    We have
    \begin{align*}
                 & x < y                                                                                 \\
        \implies & x - y \text{ is negative}           & \text{(by Definition \ref{4.2.8})}              \\
        \implies & x + z - z - y \text{ is negative}   & \text{(by Proposition \ref{4.2.4})}             \\
        \implies & x + z - y - z \text{ is negative}   & \text{(by Proposition \ref{4.2.4})}             \\
        \implies & x + z - (y + z) \text{ is negative} & \text{(by Additional Corollary \ref{ac 4.2.3})} \\
        \implies & x + z < y + z.                      & \text{(by Definition \ref{4.2.8})}
    \end{align*}
\end{proof}

\begin{proof}{(e)}
    We have
    \begin{align*}
                 & x < y                                                                          \\
        \implies & x - y \text{ is negative}    & \text{(by Definition \ref{4.2.8})}              \\
        \implies & (x - y)z \text{ is negative} & \text{(by Additional Corollary \ref{ac 4.2.6})} \\
        \implies & xz - yz \text{ is negative}  & \text{(by Proposition \ref{4.2.4})}             \\
        \implies & xz < yz.                     & \text{(by Definition \ref{4.2.8})}
    \end{align*}
\end{proof}

\begin{remark}\label{4.2.10}
    The above five properties in Proposition \ref{4.2.9}, combined with the field axioms in Proposition \ref{4.2.4}, have a name:
    they assert that the rationals \(\Q\) form an \emph{ordered field}.
    It is important to keep in mind that Proposition \ref{4.2.9}(e) only works when \(z\) is positive.
\end{remark}

\exercisesection

\begin{exercise}\label{ex 4.2.1}
    Show that the definition of equality for the rational numbers is reflexive, symmetric, and transitive.
\end{exercise}

\begin{proof}
    See Additional Corollary \ref{ac 4.2.1}.
\end{proof}

\begin{exercise}\label{ex 4.2.2}
    Prove the remaining components of Lemma \ref{4.2.3}.
\end{exercise}

\begin{proof}
    See Lemma \ref{4.2.3}.
\end{proof}

\begin{exercise}\label{ex 4.2.3}
    Prove the remaining components of Proposition \ref{4.2.4}.
\end{exercise}

\begin{proof}
    See Proposition \ref{4.2.4}.
\end{proof}

\begin{exercise}\label{ex 4.2.4}
    Prove Lemma \ref{4.2.7}.
\end{exercise}

\begin{proof}
    See Lemma \ref{4.2.7}.
\end{proof}

\begin{exercise}\label{ex 4.2.5}
    Prove Proposition \ref{4.2.9}.
\end{exercise}

\begin{proof}
    See Proposition \ref{4.2.9}.
\end{proof}

\begin{exercise}\label{ex 4.2.6}
    Show that if \(x\), \(y\), \(z\) are rational numbers such that \(x < y\) and \(z\) is negative, then \(xz > yz\).
\end{exercise}

\begin{proof}
    We have
    \begin{align*}
                 & x < y                                                                          \\
        \implies & x - y \text{ is negative}    & \text{(by Definition \ref{4.2.8})}              \\
        \implies & (x - y)z \text{ is positive} & \text{(by Additional Corollary \ref{ac 4.2.5})} \\
        \implies & xz - yz \text{ is positive}  & \text{(by Proposition \ref{4.2.4})}             \\
        \implies & xz > yz.                     & \text{(by Definition \ref{4.2.8})}
    \end{align*}
\end{proof}
\section{Absolute value and exponentiation}

\begin{definition}[Absolute value]\label{4.3.1}
If \(x\) is a rational number, the \emph{absolute value} \(\abs*{x}\) of \(x\) is defined as follows.
If \(x\) is positive, then \(\abs*{x} \coloneqq x\).
If \(x\) is negative, then \(\abs*{x} \coloneqq -x\).
If \(x\) is zero, then \(\abs*{x} \coloneqq 0\).
\end{definition}

\begin{definition}[Distance]\label{4.3.2}
Let \(x\) and \(y\) be rational numbers.
The quantity \(\abs*{x - y}\) is called the \emph{distance between \(x\) and \(y\)} and is sometimes denoted \(d(x, y)\), thus \(d(x, y) \coloneqq \abs*{x - y}\).
\end{definition}

\begin{proposition}[Basic properties of absolute value and distance]\label{4.3.3}
Let \(x\), \(y\), \(z\) be rational numbers.
\begin{enumerate}
    \item (Non-degeneracy of absolute value)
    We have \(\abs*{x} \geq 0\).
    Also, \(\abs*{x} = 0\) if and only if \(x\) is \(0\).
    \item (Triangle inequality for absolute value)
    We have \(\abs*{x + y} \leq \abs*{x} + \abs*{y}\).
    \item We have the inequalities \(-y \leq x \leq y\) if and only if \(y \geq \abs*{x}\).
    In particular, we have \(-\abs*{x} \leq x \leq \abs*{x}\).
    \item (Multiplicativity of absolute value)
    We have \(\abs*{xy} = \abs*{x} \abs*{y}\).
    In particular, \(\abs*{-x} = \abs*{x}\).
    \item (Non-degeneracy of distance)
    We have \(d(x, y) \geq 0\).
    Also, \(d(x, y) = 0\) if and only if \(x = y\).
    \item (Symmetry of distance)
    \(d(x, y) = d(y, x)\).
    \item (Triangle inequality for distance)
    \(d(x, z) \leq d(x, y) + d(y, z)\).
\end{enumerate}
\end{proposition}

\begin{proof}{(a)}
By Lemma \ref{4.2.7}, exactly one of the three statements is true:
\begin{enumerate}[label=(\roman*)]
    \item \(x = 0\).
    Then by Definition \ref{4.3.1}, \(\abs*{x} = 0\).
    \item \(x\) is a positive rational number.
    Then by Definition \ref{4.3.1}, \(\abs*{x} = x\), which is a positive rational number.
    \item \(x\) is a negative rational number.
    Then by Definition \ref{4.3.1}, \(\abs*{x} = -x\).
    By Additional Corollary \ref{ac 4.2.5}, \(-x = (-1)x\) is a positive rational number.
\end{enumerate}
So \(\abs*{x}\) is either \(0\) or a positive rational number, which by Definition \ref{4.2.8}, \(\abs*{x} \geq 0\).

Now we proof that \(\abs*{x} = 0 \iff x = 0\).
By Definition \ref{4.3.1}, \(x = 0 \implies \abs*{x} = 0\), so we only need to show that \(\abs*{x} = 0 \implies x = 0\).
By Lemma \ref{4.2.7}, exactly one of the following is true:
\(x = 0\), \(x\) is positive rational number or \(x\) is negative rational number.
If \(x\) is positive rational number, then \(\abs*{x} = x \neq 0\).
If \(x\) is negative rational number, then \(\abs*{x} = -x \neq 0\).
So \(x\) can only be \(0\), which means \(\abs*{x} = 0 \implies x = 0\).
\end{proof}

\begin{proof}{(b)}
By Lemma \ref{4.2.7}, \(x\) and \(y\) can both have three different cases.
\begin{enumerate}[label=(\Roman*)]
    \item For \(x = 0\) and
    \begin{enumerate}[label=(\roman*)]
        \item \(y = 0\).
        By Proposition \ref{4.2.4} and Definition \ref{4.3.1}, \(\abs*{0 + 0} = \abs*{0} = 0\), and \(\abs*{0} + \abs*{0} = 0 + 0 = 0\).
        Thus \(\abs*{x + y} = 0 = \abs*{x} + \abs*{y}\).
        \item \(y\) is positive.
        By Proposition \ref{4.2.4} and Definition \ref{4.3.1}, \(\abs*{0 + y} = \abs*{y} = y\), and \(\abs*{0} + \abs*{y} = 0 + y = y\).
        Thus \(\abs*{x + y} = y = \abs*{x} + \abs*{y}\).
        \item \(y\) is negative.
        By Proposition \ref{4.2.4} and Definition \ref{4.3.1}, \(\abs*{0 + y} = \abs*{y} = -y\), and \(\abs*{0} + \abs*{y} = 0 + (-y) = -y\).
        Thus \(\abs*{x + y} = -y = \abs*{x} + \abs*{y}\).
    \end{enumerate}
    \item For \(x\) is positive and
    \begin{enumerate}[label=(\roman*)]
        \item \(y = 0\).
        Which is just equivalent to the case \(x = 0\) and \(y\) is positive.
        \item \(y\) is positive.
        By Additional Corollary \ref{ac 4.2.4} and Definition \ref{4.3.1}, \(\abs*{x + y} = x + y\), and \(\abs*{x} + \abs*{y} = x + y\).
        Thus \(\abs*{x + y} = x + y = \abs*{x} + \abs*{y}\).
        \item \(y\) is negative.
        Let \(y = -a\), where \(a\) is a positive rational number.
        By Proposition \ref{4.2.9}, exactly one of the following three statements is true:
        \begin{enumerate}[label=(\arabic*)]
            \item \(x = a\).
            By Proposition \ref{4.2.4} and Definition \ref{4.3.1}, \((\abs*{x} + \abs*{-a}) - \abs*{x - a} = (\abs*{x} + \abs*{-a}) - \abs*{0} = (x + a) - 0 = x + a\).
            By Additional Corollary \ref{ac 4.2.4}, \(x + a\) is a positive rational number.
            Thus by Definition \ref{4.2.8}, \(\abs*{x + y} = 0 < x + a = \abs*{x} + \abs*{y}\).
            \item \(x > a\).
            By Definition \ref{4.2.8}, \(x - a\) is a positive rational number, so by Definition \ref{4.3.1}, \(\abs*{x - a} = x - a\).
            By Definition \ref{4.3.1}, \(\abs*{x} + \abs*{-a} = x + a\).
            By Proposition \ref{4.2.4} and Additional Corollary \ref{ac 4.2.3}, \((\abs*{x} + \abs*{y}) - \abs*{x + y} = (x + a) - (x - a) = 2a\).
            By Additional Corollary \ref{ac 4.2.5}, \(2a\) is a positive rational number.
            Thus by Definition \ref{4.2.8}, \(\abs*{x + y} = x - a < x + a = \abs*{x} + \abs*{y}\).
            \item \(x < a\).
            By Definition \ref{4.2.8}, \(x - a\) is a negative rational number, so by Definition \ref{4.3.1}, \(\abs*{x - a} = -(x - a) = a - x\).
            By Definition \ref{4.3.1}, \(\abs*{x} + \abs*{-a} = x + a\).
            By Proposition \ref{4.2.4} and Additional Corollary \ref{ac 4.2.3}, \((\abs*{x} + \abs*{y}) - \abs*{x + y} = (x + a) - (a - x) = 2x\).
            By Additional Corollary \ref{ac 4.2.5}, \(2x\) is a positive rational number.
            Thus by Definition \ref{4.2.8}, \(\abs*{x + y} = a - x < x + a = \abs*{x} + \abs*{y}\).
        \end{enumerate}
    \end{enumerate}
    \item For \(x\) is negative and
    \begin{enumerate}[label=(\roman*)]
        \item \(y = 0\).
        Which is just equivalent to the case \(x = 0\) and \(y\) is negative.
        \item \(y\) is positive.
        Which is just equivalent to the case \(x\) is positive and \(y\) is negative.
        \item \(y\) is negative.
        By Additional Corollary \ref{ac 4.2.4} and Definition \ref{4.3.1}, \(\abs*{x + y} = -(x + y)\).
        By Definition \ref{4.3.1}, Additional Corollary \ref{ac 4.2.3} and Proposition \ref{4.2.4}, \(\abs*{x} + \abs*{y} = (-x) + (-y) = (-1)x + (-1)y = (-1)(x + y) = -(x + y)\).
        Thus \(\abs*{x + y} = -(x + y) = \abs*{x} + \abs*{y}\).
    \end{enumerate}
\end{enumerate}
For all cases above, we get either \(\abs*{x + y} = \abs*{x} + \abs*{y}\) or \(\abs*{x + y} < \abs*{x} + \abs*{y}\).
So by Definition \ref{4.2.8}, \(\abs*{x + y} \leq \abs*{x} + \abs*{y}\).
\end{proof}

\begin{proof}{(c)}
We first prove that \(-y \leq x \leq y\) implies \(y \geq \abs*{x}\).
Because \(-y \leq y\), so \(y \geq 0\) by Proposition \ref{4.2.9} and \(-y \leq 0\) by Exercise \ref{ex 4.2.6}.
Again by Proposition \ref{4.2.9}, exactly one of the three statements is true:
\begin{enumerate}[label=(\Roman*)]
    \item \(x < 0\).
    Then we have
    \begin{align*}
        & -y \leq x < 0 \leq y \\
        \implies & (-y \leq x) \land (x < 0) \\
        \implies & (-x \leq y) \land (x < 0) & \text{(by Exercise \ref{ex 4.2.6})} \\
        \implies & (-x \leq y) \land (x \text{ is negative}) & \text{(by Addtitional Corollary \ref{ac 4.2.7})} \\
        \implies & (-x \leq y) \land (\abs*{x} = -x) & \text{(by Definition \ref{4.3.1})} \\
        \implies & \abs*{x} \leq y & \text{(by Addtitional Corollary \ref{ac 4.2.1})} \\
        \implies & y \geq \abs*{x}. & \text{(by Proposition \ref{4.2.9})}
    \end{align*}
    \item \(x = 0\).
    Then we have
    \begin{align*}
        & -y \leq x = 0 \leq y \\
        \implies & (-y \leq 0) \land (x = 0) \\
        \implies & (0 \leq y) \land (x = 0) & \text{(by Exercise \ref{ex 4.2.6})} \\
        \implies & (0 \leq y) \land (\abs*{x} = 0) & \text{(by Definition \ref{4.3.1})} \\
        \implies & \abs*{x} \leq y & \text{(by Addtitional Corollary \ref{ac 4.2.1})} \\
        \implies & y \geq \abs*{x}. & \text{(by Proposition \ref{4.2.9})}
    \end{align*}
    \item \(x > 0\).
    Then we have
    \begin{align*}
        & -y \leq 0 < x \leq y \\
        \implies & (0 < x) \land (x \leq y) \\
        \implies & (x \text{ is positive}) \land (x \leq y) & \text{(by Addtitional Corollary \ref{ac 4.2.7})} \\
        \implies & (\abs*{x} = x) \land (x \leq y) & \text{(by Definition \ref{4.3.1})} \\
        \implies & \abs*{x} \leq y & \text{(by Addtitional Corollary \ref{ac 4.2.1})} \\
        \implies & y \geq \abs*{x}. & \text{(by Proposition \ref{4.2.9})}
    \end{align*}
\end{enumerate}
For all cases above, we have \(y \geq \abs*{x}\).
Thus we conclude that \(-y \leq x \leq y \implies y \geq \abs*{x}\).

Now we prove that \(y \geq \abs*{x} \implies -y \leq x \leq y\).
By Proposition \ref{4.3.3} (a), \(\abs*{x} \geq 0\).
So we have \(y \geq \abs*{x} \geq 0\).
By Exercise \ref{ex 4.2.6}, \(y \geq 0 \implies 0 \geq -y\).
Then we have \(y \geq \abs*{x} \geq -y\) by Proposition \ref{4.2.9}.
By Lemma \ref{4.2.7}, exactly one of the three statements is true:
\begin{enumerate}
    \item \(x\) is positive.
    Then by Definition \ref{4.3.1}, \(\abs*{x} = x\).
    Thus we have \(y \geq x \geq -y\), or \(-y \leq x \leq y\) by Proposition \ref{4.2.9}.
    \item \(x = 0\).
    Then by Definition \ref{4.3.1}, \(\abs*{x} = 0\).
    Thus we have \(y \geq 0 \geq -y\), or \(-y \leq 0 \leq y\) by Proposition \ref{4.2.9}.
    \item \(x\) is negative.
    Then by Definition \ref{4.3.1}, \(\abs*{x} = -x\).
    Thus we have \(y \geq -x \geq -y\), or \(-y \leq x \leq y\) by Exercise \ref{ex 4.2.6}.
\end{enumerate}
For all cases above, we have \(-y \leq x \leq y\).
Thus we conclude that \(y \geq \abs*{x} \implies -y \leq x \leq y\).
Combine with proof above, we have \(-y \leq x \leq y \iff y \geq \abs*{x}\).
In particular, by replacing \(y\) with \(\abs*{x}\), we have \(-\abs*{x} \leq x \leq \abs*{x} \iff \abs*{x} \geq \abs*{x}\).
\end{proof}

\begin{proof}{(d)}
By Lemma \ref{4.2.7}, exactly one of the following three statements is true:
\begin{enumerate}[label=(\Roman*)]
    \item \(x = 0\).
    By Proposition \ref{4.2.4} and Definition \ref{4.3.1}, \(\abs*{xy} = \abs*{0y} = \abs*{0} = 0\) and \(\abs*{x}\abs*{y} = \abs*{0}\abs*{y} = 0\abs*{y} = 0\).
    So \(\abs*{xy} = 0 = \abs*{x}\abs*{y}\).
    \item \(x\) is a positive rational number.
    By Lemma \ref{4.2.7}, exactly one of the following three statements is true:
    \begin{enumerate}[label=(\roman*)]
        \item \(y = 0\).
        Which is just the same case as \(x = 0\).
        \item \(y\) is a positive rational number.
        By Additional Corollary \ref{ac 4.2.5}, \(xy\) is a positive rational number.
        Then by Definition \ref{4.3.1}, \(\abs*{xy} = xy\) and \(\abs*{x}\abs*{y} = xy\).
        So \(\abs*{xy} = xy = \abs*{x}\abs*{y}\).
        \item \(y\) is a negative rational number.
        By Additional Corollary \ref{ac 4.2.6}, \(xy\) is a negative rational number.
        Then by Definition \ref{4.3.1}, \(\abs*{xy} = -(xy)\).
        By Definition \ref{4.3.1}, Additional Corollary \ref{ac 4.2.3} and Proposition \ref{4.2.4}, \(\abs*{x}\abs*{y} = x(-y) = x((-1)y) = (x(-1))y = ((-1)x)y = (-1)(xy) = -(xy)\).
        So \(\abs*{xy} = -(xy) = \abs*{x}\abs*{y}\).
    \end{enumerate}
    \item \(x\) is a negative rational number.
    By Lemma \ref{4.2.7}, exactly one of the following three statements is true:
    \begin{enumerate}[label=(\roman*)]
        \item \(y = 0\).
        Which is just the same case as \(x = 0\).
        \item \(y\) is a positive rational number.
        Which is just the same case as \(x\) is a positive rational number and \(y\) is a negative rational number.
        \item \(y\) is a negative rational number.
        By Additional Corollary \ref{ac 4.2.5}, \(xy\) is a positive rational number.
        Then by Definition \ref{4.3.1}, \(\abs*{xy} = xy\).
        By Definition \ref{4.3.1}, Additional Corollary \ref{ac 4.2.3} and Proposition \ref{4.2.4}, \(\abs*{x}\abs*{y} = (-x)(-y) = ((-1)x)((-1)y) = (x(-1))((-1)y) = (x((-1)(-1)))y = (x1)y = xy\).
        So \(\abs*{xy} = xy = \abs*{x}\abs*{y}\).
    \end{enumerate}
\end{enumerate}
From all cases above, we can see that \(\abs*{xy} = \abs*{x}\abs*{y}\).

Now we show that \(\abs*{-x} = \abs*{x}\).
Using previous proof, let \(y = -1\), we get \(\abs*{x(-1)} = \abs*{x}\abs*{-1}\).
By Proposition \ref{4.2.4}, Definition \ref{4.3.1} and Additional Corollary \ref{4.2.3}, \(\abs*{x(-1)} = \abs*{(-1)x} = \abs*{-x}\).
Again by Proposition \ref{4.2.4}, Definition \ref{4.3.1} and Additional Corollary \ref{4.2.3}, \(\abs*{x}\abs*{-1} = \abs*{x}(-(-1)) = \abs*{x}((-1)(-1)) = \abs*{x}1 = \abs*{x}\).
Thus \(\abs*{-x} = \abs*{x}\).
\end{proof}

\begin{proof}{(e)}
We first show that \(d(x, y) \geq 0\).
By Definition \ref{4.3.2}, \(d(x, y) = \abs*{x - y}\).
By Proposition \ref{4.3.3}(a), \(\abs*{x - y} \geq 0\).
So \(d(x, y) = \abs*{x - y} \geq 0\).

Now we show that \(d(x, y) = 0 \iff x = y\).
\begin{align*}
& d(x, y) = 0 & \text{(by the given condition)} \\
\iff & \abs*{x - y} = 0 & \text{(by Definition \ref{4.3.2})} \\
\iff & x - y = 0 & \text{(by Proposition \ref{4.3.3}(a))} \\
\iff & (x - y) + y = 0 + y & \text{(by Proposition \ref{4.2.4})} \\
\iff & (x - y) + y = y & \text{(by Proposition \ref{4.2.4})} \\
\iff & (x + (-y)) + y = y \\
\iff & x + ((-y) + y) = y & \text{(by Proposition \ref{4.2.4})} \\
\iff & x + 0 = y & \text{(by Proposition \ref{4.2.4})} \\
\iff & x = y. & \text{(by Proposition \ref{4.2.4})}
\end{align*}
\end{proof}

\begin{proof}{(f)}
\begin{align*}
d(x, y) &= \abs*{x - y} & \text{(by Definition \ref{4.3.2})} \\
&= \abs*{x + (-y)} \\
&= \abs*{-(x + (-y))} & \text{(by Proposition \ref{4.3.3}(d))} \\
&= \abs*{(-1)(x + (-1)y)} & \text{(by Additional Corollary \ref{ac 4.2.3})} \\
&= \abs*{(-1)x + (-1)((-1)y)} & \text{(by Proposition \ref{4.2.4})} \\
&= \abs*{(-1)x + ((-1)(-1))y} & \text{(by Proposition \ref{4.2.4})} \\
&= \abs*{(-1)x + 1y} & \text{(by Proposition \ref{4.2.4})} \\
&= \abs*{(-1)x + y} & \text{(by Proposition \ref{4.2.4})} \\
&= \abs*{y + (-1)x} & \text{(by Proposition \ref{4.2.4})} \\
&= \abs*{y + -x} & \text{(by Additional Corollary \ref{ac 4.2.3})} \\
&= \abs*{y - x} \\
&= d(y, x). & \text{(by Definition \ref{4.3.2})}
\end{align*}
\end{proof}

\begin{proof}{(g)}
\begin{align*}
d(x, z) &= \abs*{x - z} & \text{(by Definition \ref{4.3.2})} \\
&= \abs*{x + (-z)} \\
&= \abs*{(x + (-z)) + 0} & \text{(by Proposition \ref{4.2.4})} \\
&= \abs*{(x + (-z)) + ((-y) + y)} & \text{(by Proposition \ref{4.2.4})} \\
&= \abs*{(x + (((-z) + (-y)) + y)} & \text{(by Proposition \ref{4.2.4})} \\
&= \abs*{(x + (((-y) + (-z)) + y)} & \text{(by Proposition \ref{4.2.4})} \\
&= \abs*{(x + (-y)) + ((-z) + y)} & \text{(by Proposition \ref{4.2.4})} \\
&= \abs*{(x + (-y)) + (y + (-z))} & \text{(by Proposition \ref{4.2.4})} \\
&= \abs*{(x - y) + (y - z)} \\
&\leq \abs*{x - y} + \abs*{y - z} & \text{(by Proposition \ref{4.3.3}(b))} \\
&= d(x, y) + d(y, z). & \text{(by Definition \ref{4.3.2})} \\
\end{align*}
\end{proof}

\begin{additional corollary}\label{ac 4.3.1}
Let \(x, y\) be rational numbers.
Then \(\abs*{x} - \abs*{y} \leq \abs*{x + y}\).
\end{additional corollary}

\begin{proof}
\begin{align*}
& \abs*{(x + y) + (-y)} \leq \abs*{x + y} + \abs*{-y} & \text{(by Proposition \ref{4.3.3})} \\
\implies & \abs*{x + (y + (-y))} \leq \abs*{x + y} + \abs*{-y}& \text{(by Proposition \ref{4.2.4})} \\
\implies & \abs*{x + 0} \leq \abs*{x + y} + \abs*{-y}& \text{(by Proposition \ref{4.2.4})} \\
\implies & \abs*{x} \leq \abs*{x + y} + \abs*{-y}& \text{(by Proposition \ref{4.2.4})} \\
\implies & \abs*{x} \leq \abs*{x + y} + \abs*{y}& \text{(by Proposition \ref{4.3.3})} \\
\implies & \abs*{x} + (-\abs*{y}) \leq (\abs*{x + y} + \abs*{y}) + (-\abs*{y}) & \text{(by Proposition \ref{4.2.9})} \\
\implies & \abs*{x} + (-\abs*{y}) \leq \abs*{x + y} + (\abs*{y} + (-\abs*{y})) & \text{(by Proposition \ref{4.2.4})} \\
\implies & \abs*{x} + (-\abs*{y}) \leq \abs*{x + y} + 0 & \text{(by Proposition \ref{4.2.4})} \\
\implies & \abs*{x} + (-\abs*{y}) \leq \abs*{x + y} & \text{(by Proposition \ref{4.2.4})} \\
\implies & \abs*{x} - \abs*{y} \leq \abs*{x + y}.
\end{align*}
\end{proof}

\begin{definition}[\(\varepsilon\)-closeness]\label{4.3.4}
Let \(\varepsilon > 0\) be a rational number, and let \(x\), \(y\) be rational numbers.
We say that \(y\) is \emph{\(\varepsilon\)-close} to \(x\) iff we have \(d(y, x) \leq \varepsilon\).
\end{definition}

\begin{remark}\label{4.3.5}
This definition is not standard in mathematics textbooks;
we will use it as ``scaffolding'' to construct the more important notions of limits (and of Cauchy sequences) later on, and once we have those more advanced notions we will discard the notion of \(\varepsilon\)-close.
\end{remark}

\begin{note}
We do not bother defining a notion of \(\varepsilon\)-close when \(\varepsilon\) is zero or negative, because if \(\varepsilon\) is zero then \(x\) and \(y\) are only \(\varepsilon\)-close when they are equal, and when \(\varepsilon\) is negative then \(x\) and \(y\) are never \(\varepsilon\)-close.
\end{note}

\begin{note}
In any event it is a long-standing tradition in analysis that the Greek letters \(\varepsilon\), \(\delta\) should only denote small positive numbers.
\end{note}

\setcounter{theorem}{6}
\begin{proposition}\label{4.3.7}
Let \(x, y, z, w\) be rational numbers.
(extended to cover the \(0\)-close case)
\begin{enumerate}
    \item If \(x = y\), then \(x\) is \(\varepsilon\)-close to \(y\) for every \(\varepsilon > 0\).
    Conversely, if \(x\) is \(\varepsilon\)-close to \(y\) for every \(\varepsilon > 0\), then we have \(x = y\).
    \item Let \(\varepsilon > 0\).
    If \(x\) is \(\varepsilon\)-close to \(y\), then \(y\) is \(\varepsilon\)-close to \(x\).
    \item Let \(\varepsilon, \delta > 0\).
    If \(x\) is \(\varepsilon\)-close to \(y\), and \(y\) is \(\delta\)-close to \(z\), then \(x\) and \(z\) are \((\varepsilon + \delta)\)-close.
    \item Let \(\varepsilon, \delta > 0\).
    If \(x\) and \(y\) are \(\varepsilon\)-close, and \(z\) and \(w\) are \(\delta\)-close, then \(x + z\) and \(y + w\) are \((\varepsilon + \delta)\)-close, and \(x - z\) and \(y - w\) are also \((\varepsilon + \delta)\)-close.
    \item Let \(\varepsilon > 0\).
    If \(x\) and \(y\) are \(\varepsilon\)-close, they are also \(\varepsilon'\)-close for every \(\varepsilon' > \varepsilon\).
    \item Let \(\varepsilon > 0\).
    If \(y\) and \(z\) are both \(\varepsilon\)-close to \(x\), and \(w\) is between \(y\) and \(z\) (i.e., \(y \leq w \leq z\) or \(z \leq w \leq y\)), then \(w\) is also \(\varepsilon\)-close to \(x\).
    \item Let \(\varepsilon > 0\).
    If \(x\) and \(y\) are \(\varepsilon\)-close, and \(z\) is non-zero, then \(xz\) and \(yz\) are \(\varepsilon\abs*{z}\)-close.
    \item Let \(\varepsilon, \delta > 0\).
    If \(x\) and \(y\) are \(\varepsilon\)-close, and \(z\) and \(w\) are \(\delta\)-close, then \(xz\) and \(yw\) are \((\varepsilon\abs*{z} + \delta\abs*{x} + \varepsilon\delta)\)-close.
\end{enumerate}
\end{proposition}

\begin{proof}{(a)}
We first prove that \(x = y\) implies \(x\) is \(\varepsilon\)-close to \(y\), \(\forall\ \varepsilon > 0\).
By Proposition \ref{4.3.3}, \(x = y \iff d(x, y) = 0\).
Then by the given condition, \(\forall\ \varepsilon > 0 = d(x, y)\), which means \(d(x, y) < \varepsilon\) by Proposition \ref{4.2.9}.
By Definition \ref{4.2.8}, \(d(x, y) < \varepsilon \implies d(x, y) \leq \varepsilon\).
Thus by Definition \ref{4.3.4}, \(x\) is \(\varepsilon\)-close to \(y\).

Now we prove that \(\forall\ \varepsilon > 0\), \(x\) is \(\varepsilon\)-close to \(y\) implies \(x = y\).
By the given condition, \(d(x, y) \leq \varepsilon\).
Suppose for sake of contradiction that \(x \neq y\), then \(d(x, y) > 0\) by Proposition \ref{4.3.3}.
Since \(d(x, y) > 0\), \(d(x, y) / 2 > 0\).
But by the given condition, \(\forall\ \varepsilon > 0\), \(d(x, y) \leq \varepsilon\).
Then we get \(d(x, y) \leq d(x, y) / 2\), and because \(d(x, y) \neq 0\), \(d(x, y) < d(x, y) / 2\), a contradiction.
So \(x = y\).
\end{proof}

\begin{proof}{(b)}
\begin{align*}
& x \text{ is \(\varepsilon\)-close to } y \\
\implies & d(x, y) \leq \varepsilon & \text{(by Definition \ref{4.3.4})} \\
\implies & d(y, x) \leq \varepsilon & \text{(by Proposition \ref{4.3.3}, \(d(x, y) = d(y, x)\))} \\
\implies & y \text{ is \(\varepsilon\)-close to } x. & \text{(by Definition \ref{4.3.4})}
\end{align*}
\end{proof}

\begin{proof}{(c)}
By Definition \ref{4.3.4}, \(x\) is \(\varepsilon\)-close to \(y\) implies \(d(x, y) \leq \varepsilon\), and \(y\) is \(\delta\)-close to \(z\) implies \(d(y, z) \leq \delta\).
So
\begin{align*}
& d(y, z) \leq \delta \\
\implies & \varepsilon + d(y, z) \leq \varepsilon + \delta. & \text{(by Proposition \ref{4.3.3})} \\
& d(x, y) \leq \varepsilon \\
\implies & d(x, y) + d(y, z) \leq \varepsilon + d(y, z) & \text{(by Proposition \ref{4.3.3})} \\
\implies & d(x, y) + d(y, z) \leq \varepsilon + \delta & \text{(by Proposition \ref{4.2.9})} \\
\implies & d(x, z) \leq d(x, y) + d(y, z) \leq \varepsilon + \delta & \text{(by Proposition \ref{4.3.3})} \\
\implies & x \text{ is \((\varepsilon + \delta)\)-close to } z. & \text{(by Definition \ref{4.3.4})}
\end{align*}
\end{proof}

\begin{proof}{(d)}
We first prove that If \(x\) and \(y\) are \(\varepsilon\)-close, and \(z\) and \(w\) are \(\delta\)-close, then \(x + z\) and \(y + w\) are \((\varepsilon + \delta)\)-close.
By Definition \ref{4.3.4}, \(x\) is \(\varepsilon\)-close to \(y\) implies \(d(x, y) \leq \varepsilon\), and \(z\) is \(\delta\)-close to \(w\) implies \(d(z, w) \leq \delta\).
Because
\begin{align*}
& d(z, w) \leq \delta \\
\implies & \varepsilon + d(z, w) \leq \varepsilon + \delta. & \text{(by Proposition \ref{4.2.9})} \\
& d(x, y) \leq \varepsilon \\
\implies & d(x, y) + d(z, w) \leq \varepsilon + d(z, w) & \text{(by Proposition \ref{4.2.9})} \\
\implies & d(x, y) + d(z, w) \leq \varepsilon + \delta. & \text{(by Proposition \ref{4.2.9})}
\end{align*}
So
\begin{align*}
& d(x + z, y + w) \\
&= \abs*{(x + z) - (y + w)} & \text{(by Definition \ref{4.3.2})} \\
&= \abs*{(x + z) + (-(y + w))} \\
&= \abs*{(x + z) + (-1)(y + w)} & \text{(by Additional Corollary \ref{ac 4.2.3})} \\
&= \abs*{(x + z) + ((-1)y + (-1)w)} & \text{(by Proposition \ref{4.2.4})} \\
&= \abs*{(x + (z + (-1)y)) + (-1)w} & \text{(by Proposition \ref{4.2.4})} \\
&= \abs*{(x + ((-1)y + z)) + (-1)w} & \text{(by Proposition \ref{4.2.4})} \\
&= \abs*{(x + (-1)y) + (z + (-1)w)} & \text{(by Proposition \ref{4.2.4})} \\
&= \abs*{(x + (-y)) + (z + (-w))} & \text{(by Additional Corollary \ref{ac 4.2.3})} \\
&= \abs*{(x - y) + (z - w)} \\
&\leq \abs*{x - y} + \abs*{z - w} & \text{(by Proposition \ref{4.3.3})} \\
&= d(x, y) + d(z, w) & \text{(by Definition \ref{4.3.2})} \\
&\leq \varepsilon + \delta. & \text{(by the given conditions)}
\end{align*}
Thus by Definition \ref{4.3.4}, \(x + z\) and \(y + w\) are \((\varepsilon + \delta)\)-close.

Now we prove that If \(x\) and \(y\) are \(\varepsilon\)-close, and \(z\) and \(w\) are \(\delta\)-close, then \(x - z\) and \(y - w\) are \((\varepsilon + \delta)\)-close.
By Definition \ref{4.3.4}, \(x\) is \(\varepsilon\)-close to \(y\) implies \(d(x, y) \leq \varepsilon\), and \(z\) is \(\delta\)-close to \(w\) implies \(d(z, w) \leq \delta\).
Again because
\begin{align*}
& d(z, w) \leq \delta \\
\implies & \varepsilon + d(z, w) \leq \varepsilon + \delta. & \text{(by Proposition \ref{4.2.9})} \\
& d(x, y) \leq \varepsilon \\
\implies & d(x, y) + d(z, w) \leq \varepsilon + d(z, w) & \text{(by Proposition \ref{4.2.9})} \\
\implies & d(x, y) + d(z, w) \leq \varepsilon + \delta. & \text{(by Proposition \ref{4.2.9})}
\end{align*}
So
\begin{align*}
& d(x - z, y - w) \\
&= \abs*{(x - z) - (y - w)} & \text{(by Definition \ref{4.3.2})} \\
&= \abs*{(x + (-z)) + (-(y + (-w)))} \\
&= \abs*{(x + (-1)z) + (-1)(y + (-1)w)} & \text{(by Additional Corollary \ref{ac 4.2.3})} \\
&= \abs*{(x + (-1)z) + ((-1)y + (-1)(-1)w)} & \text{(by Proposition \ref{4.2.4})} \\
&= \abs*{(x + (-1)z) + ((-1)y + 1w)} & \text{(by Proposition \ref{4.2.4})} \\
&= \abs*{(x + (-1)z) + ((-1)y + w)} & \text{(by Proposition \ref{4.2.4})} \\
&= \abs*{(x + (-z)) + ((-y) + w)} & \text{(by Additional Corollary \ref{ac 4.2.3})} \\
&= \abs*{(x + ((-z) + (-y))) + w} & \text{(by Proposition \ref{4.2.4})} \\
&= \abs*{(x + ((-y) + (-z))) + w} & \text{(by Proposition \ref{4.2.4})} \\
&= \abs*{(x + (-y)) + ((-z) + w)} & \text{(by Proposition \ref{4.2.4})} \\
&= \abs*{(x + (-y)) + (w + (-z))} & \text{(by Proposition \ref{4.2.4})} \\
&= \abs*{(x - y) + (w - z)} \\
&\leq \abs*{x - y} + \abs*{w - z} & \text{(by Proposition \ref{4.3.3})} \\
&= \abs*{x - y} + \abs*{-(w - z)} & \text{(by Proposition \ref{4.3.3})} \\
&= \abs*{x - y} + \abs*{-(w + (-z))} \\
&= \abs*{x - y} + \abs*{(-1)(w + (-1)z)} & \text{(by Additional Corollary \ref{ac 4.2.3})} \\
&= \abs*{x - y} + \abs*{(-1)w + (-1)((-1)z)} & \text{(by Proposition \ref{4.2.4})} \\
&= \abs*{x - y} + \abs*{(-1)w + ((-1)(-1))z} & \text{(by Proposition \ref{4.2.4})} \\
&= \abs*{x - y} + \abs*{(-1)w + 1z} & \text{(by Proposition \ref{4.2.4})} \\
&= \abs*{x - y} + \abs*{(-1)w + z} & \text{(by Proposition \ref{4.2.4})} \\
&= \abs*{x - y} + \abs*{z + (-1)w} & \text{(by Proposition \ref{4.2.4})} \\
&= \abs*{x - y} + \abs*{z + (-w)} & \text{(by Additional Corollary \ref{ac 4.2.3})} \\
&= \abs*{x - y} + \abs*{z - w} \\
&= d(x, y) + d(z, w) & \text{(by Definition \ref{4.3.2})} \\
&\leq \varepsilon + \delta. & \text{(by the given conditions)}
\end{align*}
Thus by Definition \ref{4.3.4}, \(x - z\) and \(y - w\) are \((\varepsilon + \delta)\)-close.
\end{proof}

\begin{proof}{(e)}
By the given condition, \(\forall\ \varepsilon'\), \(\varepsilon < \varepsilon'\).
And by Definition \ref{4.3.4}, \(x\) is \(\varepsilon\)-close to \(y\) implies \(d(x, y) \leq \varepsilon\).
If \(d(x, y) = \varepsilon\), then \(d(x, y) < \varepsilon'\).
If \(d(x, y) < \varepsilon\), then by Proposition \ref{4.2.9}, \(d(x, y) < \varepsilon'\).
Thus \(d(x, y) < \varepsilon'\), by Definition \ref{4.2.8}, \(d(x, y) \leq \varepsilon'\), which means \(x\) is \(\varepsilon'\)-close to \(y\) by Definition \ref{4.3.4}.
\end{proof}

\begin{proof}{(f)}
\begin{align*}
& y \text{ is } \varepsilon\text{-close to } x \\
\implies & d(y, x) \leq \varepsilon & \text{(by Definition \ref{4.3.4})} \\
\implies & \abs*{y - x} \leq \varepsilon & \text{(by Definition \ref{4.3.2})} \\
\implies & (-1)\abs*{y - x} \geq (-1)\varepsilon & \text{(by Exercise \ref{ex 4.2.6})} \\
\implies & -\abs*{y - x} \geq -\varepsilon. \\
& z \text{ is } \varepsilon\text{-close to } x \\
\implies & d(z, x) \leq \varepsilon & \text{(by Definition \ref{4.3.4})} \\
\implies & \abs*{z - x} \leq \varepsilon & \text{(by Definition \ref{4.3.2})} \\
\implies & (-1)\abs*{z - x} \geq (-1)\varepsilon & \text{(by Exercise \ref{ex 4.2.6})} \\
\implies & -\abs*{z - x} \geq -\varepsilon. \\
& y \leq w \leq z \\
\implies & y + (-x) \leq w + (-x) \leq z + (-x) & \text{(by Lemma \ref{4.2.9})} \\
\implies & y - x \leq w - x \leq z - x \\
\implies & -\abs*{y - x} \leq y - x \leq w - x \leq z - x \leq \abs*{z - x} & \text{(by Proposition \ref{4.3.3})} \\
\implies & -\abs*{y - x} \leq w - x \leq \abs*{z - x} \\
\implies & -\varepsilon \leq -\abs*{y - x} \leq w - x \leq \abs*{z - x} \leq \varepsilon & \text{(by the given conditions)} \\
\implies & -\varepsilon \leq w - x \leq \varepsilon \\
\implies & \varepsilon \geq \abs*{w - x} & \text{(by Proposition \ref{4.3.3})} \\
\implies & \abs*{w - x} \leq \varepsilon & \text{(by Proposition \ref{4.2.9})} \\
\implies & d(w, x) \leq \varepsilon & \text{(by Definition \ref{4.3.2})} \\
\implies & w \text{ is } \varepsilon\text{-close to } x. \\
& z \leq w \leq y \\
\implies & z + (-x) \leq w + (-x) \leq y + (-x) & \text{(by Lemma \ref{4.2.9})} \\
\implies & z - x \leq w - x \leq y - x \\
\implies & -\abs*{z - x} \leq z - x \leq w - x \leq y - x \leq \abs*{y - x} & \text{(by Proposition \ref{4.3.3})} \\
\implies & -\abs*{z - x} \leq w - x \leq \abs*{y - x} \\
\implies & -\varepsilon \leq -\abs*{z - x} \leq w - x \leq \abs*{y - x} \leq \varepsilon & \text{(by the given conditions)} \\
\implies & -\varepsilon \leq w - x \leq \varepsilon \\
\implies & \varepsilon \geq \abs*{w - x} & \text{(by Proposition \ref{4.3.3})} \\
\implies & \abs*{w - x} \leq \varepsilon & \text{(by Proposition \ref{4.2.9})} \\
\implies & d(w, x) \leq \varepsilon & \text{(by Definition \ref{4.3.2})} \\
\implies & w \text{ is } \varepsilon\text{-close to } x.
\end{align*}
\end{proof}

\begin{proof}{(g)}
By Proposition \ref{4.3.3}, \(\abs*{z} \geq 0\).
So
\begin{align*}
& x \text{ is } \varepsilon\text{-close to } y \\
\implies & d(x, y) \leq \varepsilon & \text{(by Definition \ref{4.3.4})} \\
\implies & \abs*{x - y} \leq \varepsilon & \text{(by Definition \ref{4.3.2})} \\
\implies & \abs*{x - y}\abs*{z} \leq \varepsilon\abs*{z} & \text{(by Proposition \ref{4.2.9})} \\
\implies & \abs*{(x - y)z} \leq \varepsilon\abs*{z} & \text{(by Proposition \ref{4.3.3})} \\
\implies & \abs*{(x + (-y))z} \leq \varepsilon\abs*{z} \\
\implies & \abs*{xz + (-y)z} \leq \varepsilon\abs*{z} & \text{(by Proposition \ref{4.2.4})} \\
\implies & \abs*{xz + ((-1)y)z} \leq \varepsilon\abs*{z} & \text{(by Additional Corollary \ref{ac 4.2.3})} \\
\implies & \abs*{xz + (-1)(yz)} \leq \varepsilon\abs*{z} & \text{(by Proposition \ref{4.2.4})} \\
\implies & \abs*{xz + (-(yz))} \leq \varepsilon\abs*{z} & \text{(by Additional Corollary \ref{ac 4.2.3})} \\
\implies & \abs*{xz - yz} \leq \varepsilon\abs*{z} \\
\implies & d(xz, yz) \leq \varepsilon\abs*{z} & \text{(by Definition \ref{4.3.2})} \\
\implies & xz \text{ is } \varepsilon\abs*{z}\text{-close to } yz. & \text{(by Definition \ref{4.3.4})} \\
\end{align*}
\end{proof}

\begin{proof}{(h)}
Let \(\varepsilon, \delta > 0\), and suppose that \(x\) and \(y\) are \(\varepsilon\)-close.
If we write \(a \coloneqq y - x\), then we have \(y = x + a\) and that \(\abs*{a} \leq \varepsilon\).
Similarly, if \(z\) and \(w\) are \(\delta\)-close, and we define \(b \coloneqq w - z\), then \(w = z + b\) and \(\abs*{b} \leq \delta\).

Since \(y = x + a\) and \(w = z + b\), we have
\[
    yw = (x + a)(z + b) = xz + az + xb + ab.
\]
Thus
\[
    \abs*{yw - xz} = \abs*{az + bx + ab} \leq \abs*{az} + \abs*{bx} + \abs*{ab} = \abs*{a}\abs*{z} + \abs*{b}\abs*{x} + \abs*{a}\abs*{b}.
\]
Since \(\abs*{a} \leq \varepsilon\) and \(\abs*{b} \leq \delta\), we thus have
\[
    \abs*{yw - xz} \leq \varepsilon\abs*{z} + \delta\abs*{x} + \varepsilon\delta
\]
and thus that \(yw\) and \(xz\) are \((\varepsilon\abs*{z} + \delta\abs*{x} + \varepsilon\delta)\)-close.
\end{proof}

\begin{remark}\label{4.3.8}
One should compare statements (a)-(c) of Proposition \ref{4.3.7} with the reflexive, symmetric, and transitive axioms of equality.
It is often useful to think of the notion of ``\(\varepsilon\)-close'' as an approximate substitute for that of equality in analysis.
\end{remark}

\begin{definition}[Exponentiation to a natural number]\label{4.3.9}
Let \(x\) be a rational number.
To raise \(x\) to the power \(0\), we define \(x^0 \coloneqq 1\);
in particular we define \(0^0 \coloneqq 1\).
Now suppose inductively that \(x^n\) has been defined for some natural number \(n\), then we define \(x^{n+1} \coloneqq x^n \times x\).
\end{definition}

\begin{proposition}[Properties of exponentiation, I]\label{4.3.10}
Let \(x\), \(y\) be rational numbers, and let \(n\), \(m\) be natural numbers.
\begin{enumerate}
    \item We have \(x^n x^m = x^{n + m}\), \((x^n)^m = x^{nm}\), and \((xy)^n = x^n y^n\).
    \item Suppose \(n > 0\).
    Then we have \(x^n = 0\) if and only if \(x = 0\).
    \item If \(x \geq y \geq 0\), then \(x^n \geq y^n \geq 0\).
    If \(x > y \geq 0\) and \(n > 0\), then \(x^n > y^n \geq 0\).
    \item We have \(\abs*{x^n} = \abs*{x}^n\).
\end{enumerate}
\end{proposition}

\begin{proof}{(a)}
We first prove that \(x^n x^m = x^{n + m}\).
We use induction on \(n\).
For \(n = 0\), \(x^0 x^m = 1 x^m = x^m\) by Definition \ref{4.3.9} and Proposition \ref{4.2.4}.
And \(x^{0 + m} = x^m\) by Proposition \ref{4.2.4}.
So \(x^0 x^m = x^{0 + m}\), and the base case holds.
Suppose inductively that for some \(n\), \(x^n x^m = x^{n + m}\).
Then for \(n++\), \(x^{n++} x^m = (x^n x) x^m = x^n (x x^m) = x^n (x^m x) = (x^n x^m)x\) by Definition \ref{4.3.9} and Proposition \ref{4.2.4}.
And \(x^{(n++) + m} = x^{(n + m)++} = x^{n + m} x = (x^n x^m)x\) by Definition \ref{2.2.1} and induction hypothesis.
So \(x^{n++} x^m = x^{(n++) + m}\), and this close the induction.

Next we prove that \((x^n)^m = x^{nm}\).
We use induction on \(m\).
For \(m = 0\), \((x^n)^0 = 1\) by Definition \ref{4.3.9}.
And \(x^{n0} = x^0 = 1\) by Additional Corollary \ref{ac 2.3.2} and Definition \ref{4.3.9}.
So \((x^n)^0 = x^{n0}\), and the base case holds.
Suppose inductively that for some \(m\), \((x^n)^m = x^{nm}\).
Then for \(m++\), \((x^n)^{m++} = (x^n)^m x^n = x^{nm} x^n\) by Definition \ref{4.3.9} and induction hypothesis.
And \(x^{n(m++)} = x^{nm + n} = x^{nm} x^n\) by Additional Corollary \ref{ac 2.3.3} and previous prove.
So \((x^n)^{m++} = x^{n(m++)}\), and this close the induction.

Finally we prove that \((xy)^n = x^n y^n\).
We use induction on \(n\).
For \(n = 0\), \((xy)^0 = 1\) by Definition \ref{4.3.9}.
And \(x^0 y^0 = 1 \times 1 = 1\) by Definition \ref{4.3.9}.
So \((xy)^0 = x^0 y^0\), and the base case holds.
Suppose inductively that for some \(n\), \((xy)^n = x^n y^n\).
Then for \(n++\), \((xy)^{n++} = (xy)^n (xy) = (x^n y^n)(xy)\) by Definition \ref{4.3.9} and induction hypothesis.
And \(x^{n++} y^{n++} = (x^n x)(y^n y) = (x^n (x y^n))y = (x^n (y^n x))y = (x^n y^n)(xy)\) by Definition \ref{4.3.9} and Proposition \ref{4.2.4}.
So \((xy)^{n++} = x^{n++} y^{n++}\), and this close the induction.
\end{proof}

\begin{proof}{(b)}
We first prove that \(x^n = 0\) implies \(x = 0\).
We use induction on \(n\) and begin with \(n = 1\).
For \(n = 1\), \(x^1 = x^0 x = 1x = x = 0\) by Definition \ref{4.3.9} and Proposition \ref{4.2.4}.
So \(x^1 = 0 \implies x = 0\), and the base case holds.
Suppose inductively that for some \(n\), \(x^n = 0 \implies x = 0\).
Then for \(n++\), \(x^{n++} = x^n x = 0\) by Definition \ref{4.3.9}.
Suppose for sake of contradiction that \(x \neq 0\).
Then \(x^n x = 0 \implies x^n = 0 / x = 0\).
But by induction hypothesis, \(x^n = 0 \implies x = 0\), a contradiction.
Thus \(x = 0\), which means \(x^{n++} = 0 \implies x = 0\), and this close the induction.

Now we prove that \(x = 0\) implies \(x^n = 0\).
Since \(n > 0\), \(x^n = x^{(n - 1) + 1} = x^{n - 1} x = x^{n - 1} 0 = 0\) by Proposition \ref{4.3.10}(a).
Thus \(x = 0 \implies x^n = 0\).
\end{proof}

\begin{proof}{(c)}
We first prove that \(x \geq y \geq 0\) implies \(x^n \geq y^n \geq 0\).
We use induction on \(n\).
For \(n = 0\), \(x^0 = 1\) and \(y^0 = 1\) by Definition \ref{4.3.9}, and \(1 \geq 1 \geq 0\), so the base case holds.
Suppose inductively that for some \(n\), \(x^n \geq y^n \geq 0\).
Then for \(n++\), \(x^{n++} = x^n x \geq y^n x\) and \(y^n x \geq y^n y = y^{n++}\) by Definition \ref{4.3.9}, induction hypothesis, Proposition \ref{4.2.9} and the given conditions.
So \(x^{n++} \geq y^{n++}\) by Proposition \ref{4.2.4}.
And \(y^{n++} = y^n y \geq y^n 0 = 0\) by Definition \ref{4.3.9} and the given conditions.
So \(y^{n++} \geq 0\).
Thus \(x^{n++} \geq y^{n++} \geq 0\), and this close the induction.

Now we prove that \(x > y \geq 0\) and \(n > 0\), then \(x^n > y^n \geq 0\).
We use induction on \(n\) and begin with \(n = 1\).
For \(n = 1\), \(x^1 = x^0 x = 1x = x\) and \(y^1 = y^0 y = 1y = y\) by Definition \ref{4.3.9} and Proposition \ref{4.2.4}.
So \(x^1 > y^1 \geq 0\) by the given conditions, and the base case holds.
Suppose inductively that for some \(n\), \(x^n > y^n \geq 0\).
Then for \(n++\), \(x^{n++} = x^n x > y^n x\) and \(y^n x > y^n y = y^{n++}\) by Definition \ref{4.3.9}, induction hypothesis, Proposition \ref{4.2.9} and the given conditions.
So \(x^{n++} > y^{n++}\) by Proposition \ref{4.2.4}.
And \(y^{n++} = y^n y \geq y^n 0 = 0\) by Definition \ref{4.3.9} and the given conditions.
So \(y^{n++} \geq 0\).
Thus \(x^{n++} > y^{n++} \geq 0\), and this close the induction.
\end{proof}

\begin{proof}{(d)}
We use induction on \(n\).
For \(n = 0\), \(\abs*{x^0} = \abs*{1} = 1\) and \(\abs*{x}^0 = 1\) by Definition \ref{4.3.9} and Definition \ref{4.3.1}.
So \(\abs*{x^0} = \abs*{x}^0\), and the base case holds.
Suppose inductively that for some \(n\), \(\abs*{x^n} = \abs*{x}^n\).
Then for \(n++\), \(\abs*{x^{n++}} = \abs*{x^n x} = \abs*{x^n}\abs*{x} = \abs*{x}^n \abs*{x}\) and \(\abs*{x}^{n++} = \abs*{x}^n \abs*{x}\) by Definition \ref{4.3.9}, Proposition \ref{4.3.3} and induction hypothesis.
So \(\abs*{x^{n++}} = \abs*{x}^{n++}\), and this close the induction.
\end{proof}

\begin{definition}[Exponentiation to a negative number]\label{4.3.11}
Let \(x\) be a non-zero rational number.
Then for any negative integer \(-n\), we define \(x^{-n} \coloneqq 1 / x^n\).
\end{definition}

\begin{proposition}[Properties of exponentiation, II]\label{4.3.12}
Let \(x\), \(y\) be nonzero rational numbers, and let \(n\), \(m\) be integers.
\begin{enumerate}
    \item We have \(x^n x^m = x^{n + m}\), \((x^n)^m = x^{nm}\), and \((xy)^n = x^n y^n\).
    \item If \(x \geq y > 0\), then \(x^n \geq y^n > 0\) if \(n\) is positive, and \(0 < x^n \leq y^n\) if \(n\) is negative.
    \item If \(x, y > 0\), \(n \neq 0\), and \(x^n = y^n\), then \(x = y\).
    \item We have \(\abs*{x^n} = \abs*{x}^n\).
\end{enumerate}
\end{proposition}

\begin{proof}{(a)}
We first prove that \(x^n x^m = x^{n + m}\).
By Lemma \ref{4.1.5}, exactly one of the following three statements is true:
\begin{enumerate}[label=(\Roman*)]
    \item \(n = 0\).
    Then \(x^0 x^m = 1x^m = x^m\) by Definition \ref{4.3.9} and Proposition \ref{4.2.4}.
    And \(x^{0 + m} = x^m\) by Proposition \ref{4.2.4}.
    So \(x^0 x^m = x^{0 + m}\).
    \item \(n\) is a positive integer.
    Again by Lemma \ref{4.1.5}, exactly one of the following three statements is true:
    \begin{enumerate}[label=(\roman*)]
        \item \(m = 0\).
        Then \(x^n x^0 = x^0 x^n\) and \(x^{n + 0} = x^{0 + n}\) by Proposition \ref{4.2.4} and Proposition \ref{4.1.6}, which is the same case as \(n = 0\).
        \item \(m\) is a positive integer.
        Then by Proposition \ref{4.3.10}, \(x^n x^m = x^{n + m}\).
        \item \(m\) is a negative integer.
        Again by Lemma \ref{4.1.5}, exactly one of the following three statements is true:
        \begin{enumerate}[label=(\arabic*)]
            \item \(n + m = 0\).
            Then \(n = -m\) and \(x^n x^m = x^{-m} x^m = 1\) by Proposition \ref{4.2.4}.
            And \(x^{n + m} = x^{(-m) + m} = x^0 = 1\) by Proposition \ref{4.2.4} and Definition \ref{4.3.9}.
            So \(x^{-m} x^m = x^{(-m) + m}\).
            \item \(n + m\) is a positive integer.
            By Proposition \ref{4.3.10}, \(x \neq 0 \implies x^{-m} \neq 0\).
            Then
            \begin{align*}
            & (x^n x^m) x^{-m} \\
            &= x^n (x^m x^{-m}) & \text{(by Proposition \ref{4.2.4})} \\
            &= x^n 1 & \text{(by Proposition \ref{4.2.4})} \\
            &= x^n & \text{(by Proposition \ref{4.2.4})} \\
            &= x^{n + 0} & \text{(by Proposition \ref{4.1.6})} \\
            &= x^{n + (m + (-m))} & \text{(by Proposition \ref{4.1.6})} \\
            &= x^{(n + m) + (-m)} & \text{(by Proposition \ref{4.1.6})} \\
            &= x^{n + m} x^{-m}. & \text{(by Proposition \ref{4.3.10})}
            \end{align*}
            So
            \begin{align*}
            & (x^n x^m) x^{-m} = x^{n + m} x^{-m} \\
            \implies & ((x^n x^m) x^{-m}) x^m = (x^{n + m} x^{-m}) x^m & \text{(by Lemma \ref{4.2.3})} \\
            \implies & (x^n x^m)(x^{-m} x^m) = x^{n + m} (x^{-m} x^m) & \text{(by Proposition \ref{4.2.4})} \\
            \implies & (x^n x^m)1 = x^{n + m} 1 & \text{(by Proposition \ref{4.2.4})} \\
            \implies & x^n x^m = x^{n + m}. & \text{(by Proposition \ref{4.2.4})} \\
            \end{align*}
            \item \(n + m\) is a negative integer.
            Then \(-(n + m)\) is a positive integer.
            By Proposition \ref{4.3.10}, \(x \neq 0 \implies x^n \neq 0\), \(x^{-m} \neq 0\) and \(x^{-(n + m)} \neq 0\).
            So
            \begin{align*}
            & x^n (x^{-n} x^{-m}) \\
            &= (x^n x^{-n})x^{-m} & \text{(by Proposition \ref{4.2.4})} \\
            &= 1x^{-m} & \text{(by Proposition \ref{4.2.4})} \\
            &= x^{-m} & \text{(by Proposition \ref{4.2.4})} \\
            &= x^{0 + (-m)} & \text{(by Proposition \ref{4.1.6})} \\
            &= x^{(n + (-n)) + (-m)} & \text{(by Proposition \ref{4.1.6})} \\
            &= x^{n + ((-n) + (-m))} & \text{(by Proposition \ref{4.1.6})} \\
            &= x^{n + ((-1)n + (-1)m)} & \text{(by Exercise \ref{ex 4.1.3})} \\
            &= x^{n + (-1)(n + m)} & \text{(by Proposition \ref{4.1.6})} \\
            &= x^{n + (-(n + m))} & \text{(by Exercise \ref{ex 4.1.3})} \\
            &= x^n x^{-(n + m)}. & \text{(by Proposition \ref{4.3.10})} \\
            \end{align*}
            And
            \begin{align*}
            & x^n (x^{-n} x^{-m}) = x^n x^{-(n + m)} \\
            \implies & x^{-n} (x^n (x^{-n} x^{-m})) = x^{-n} (x^n x^{-(n + m)}) & \text{(by Lemma \ref{4.2.3})} \\
            \implies & (x^{-n} x^n)(x^{-n} x^{-m}) = (x^{-n} x^n)x^{-(n + m)} & \text{(by Proposition \ref{4.2.4})} \\
            \implies & 1(x^{-n} x^{-m}) = 1x^{-(n + m)} & \text{(by Proposition \ref{4.2.4})} \\
            \implies & x^{-n} x^{-m} = x^{-(n + m)} & \text{(by Proposition \ref{4.2.4})} \\
            \implies & x^n (x^{-n} x^{-m}) = x^n x^{-(n + m)} & \text{(by Lemma \ref{4.2.3})} \\
            \implies & (x^n x^{-n}) x^{-m} = x^n x^{-(n + m)} & \text{(by Proposition \ref{4.2.4})} \\
            \implies & 1x^{-m} = x^n x^{-(n + m)} & \text{(by Proposition \ref{4.2.4})} \\
            \implies & x^{-m} = x^n x^{-(n + m)} & \text{(by Proposition \ref{4.2.4})} \\
            \implies & x^m x^{-m} = x^m (x^n x^{-(n + m)}) & \text{(by Lemma \ref{4.2.3})} \\
            \implies & 1 = x^m (x^n x^{-(n + m)}) & \text{(by Proposition \ref{4.2.4})} \\
            \implies & 1 = (x^m x^n) x^{-(n + m)} & \text{(by Proposition \ref{4.2.4})} \\
            \implies & 1x^{n + m} = ((x^m x^n) x^{-(n + m)}) x^{n + m} & \text{(by Lemma \ref{4.2.3})} \\
            \implies & 1x^{n + m} = (x^m x^n)(x^{-(n + m)} x^{n + m}) & \text{(by Proposition \ref{4.2.4})} \\
            \implies & 1x^{n + m} = (x^m x^n)1 & \text{(by Proposition \ref{4.2.4})} \\
            \implies & x^{n + m} = x^m x^n & \text{(by Proposition \ref{4.2.4})} \\
            \implies & x^{n + m} = x^n x^m. & \text{(by Proposition \ref{4.2.4})}
            \end{align*}
        \end{enumerate}
    \end{enumerate}
    \item \(n\) is a negative integer.
    Again by Lemma \ref{4.1.5}, exactly one of the following three statements is true:
    \begin{enumerate}[label=(\roman*)]
        \item \(m = 0\).
        Then \(x^n x^0 = x^0 x^n\) and \(x^{n + 0} = x^{0 + n}\) by Proposition \ref{4.2.4} and Proposition \ref{4.1.6}, which is the same case as \(n = 0\).
        \item \(m\) is a positive integer.
        Then \(x^n x^m = x^m x^n\) and \(x^{n + m} = x^{m + n}\) by Proposition \ref{4.2.4} and Proposition \ref{4.1.6}, which is the same case as \(n\) is positive and \(m\) is negative.
        \item \(m\) is a negative integer.
        Then \(-n, -m, -(n + m)\) are positive integers.
        By Proposition \ref{4.3.10}, \(x \neq 0 \implies x^{-n} \neq 0\), \(x^{-m} \neq 0\) and \(x^{-(n + m)} \neq 0\).
        So
        \begin{align*}
        & x^{-n} x^{-m} = x^{-(n + m)} & \text{(by Proposition \ref{4.3.10})} \\
        \implies & x^n (x^{-n} x^{-m}) = x^n x^{-(n + m)} & \text{(by Lemma \ref{4.2.3})} \\
        \implies & (x^n x^{-n}) x^{-m} = x^n x^{-(n + m)} & \text{(by Proposition \ref{4.2.4})} \\
        \implies & 1x^{-m} = x^n x^{-(n + m)} & \text{(by Proposition \ref{4.2.4})} \\
        \implies & x^{-m} = x^n x^{-(n + m)} & \text{(by Proposition \ref{4.2.4})} \\
        \implies & x^m x^{-m} = x^m (x^n x^{-(n + m)}) & \text{(by Lemma \ref{4.2.3})} \\
        \implies & 1 = x^m (x^n x^{-(n + m)}) & \text{(by Proposition \ref{4.2.4})} \\
        \implies & 1 = (x^m x^n) x^{-(n + m)} & \text{(by Proposition \ref{4.2.4})} \\
        \implies & 1x^{n + m} = ((x^m x^n) x^{-(n + m)})x^{n + m} & \text{(by Lemma \ref{4.2.3})} \\
        \implies & 1x^{n + m} = (x^m x^n)(x^{-(n + m)} x^{n + m}) & \text{(by Proposition \ref{4.2.4})} \\
        \implies & 1x^{n + m} = (x^m x^n)1 & \text{(by Proposition \ref{4.2.4})} \\
        \implies & x^{n + m} = x^m x^n & \text{(by Proposition \ref{4.2.4})} \\
        \implies & x^{n + m} = x^n x^m. & \text{(by Proposition \ref{4.2.4})}
        \end{align*}
    \end{enumerate}
\end{enumerate}
From all cases above, we can conclude that \(x^n x^m = x^{n + m}\).

Next we prove that \((x^n)^m = x^{nm}\).
By Lemma \ref{4.1.5}, exactly one of the following three statements is true:
\begin{enumerate}[label=(\Roman*)]
    \item \(n = 0\).
    Then by Definition \ref{4.3.9}, \((x^0)^m = 1^m\) and \(x^{0m} = x^0 = 1\).
    Again By Lemma \ref{4.1.5}, exactly one of the following three statements is true:
    \begin{enumerate}[label=(\roman*)]
        \item \(m = 0\).
        Then \(1^0 = 1\) by Definition \ref{4.3.9}, so \((x^0)^0 = x^{0 \times 0}\).
        \item \(m\) is a positive integer.
        We claim that \(1^m = 1\) by using induction on \(m\).
        For \(m = 0\), \(1^0 = 1\) by Definition \ref{4.3.9}, so the base case holds.
        Suppose inductively that for some \(m\), \(1^m = 1\).
        Then for \(m++\), \(1^{m++} = 1^m \times 1 = 1 \times 1 = 1\) by Definition \ref{4.3.9} and induction hypothesis, and this close the induction.
        So \((x^0)^m = x^{0m}\).
        \item \(m\) is a negative integer.
        Then \(-m\) is a positive integer, and \(1^m = 1 / 1^{-m}\) by Definition \ref{4.3.11}.
        From previous prove, we show that \(1^{-m} = 1\).
        So \((x^0)^m = x^{0m}\).
    \end{enumerate}
    \item \(n\) is a positive integer.
    Again by Lemma \ref{4.1.5}, exactly one of the following three statements is true:
    \begin{enumerate}[label=(\roman*)]
        \item \(m = 0\).
        Then by Definition \ref{4.3.9}, \((x^n)^0 = 1\) and \(x^{n0} = x^0 = 1\).
        So \((x^n)^0 = x^{n0}\).
        \item \(m\) is a positive integer.
        Then by Proposition \ref{4.3.10}, \((x^n)^m = x^{nm}\).
        \item \(m\) is a negative integer.
        Then \(-m\) is a positive integer.
        So
        \begin{align*}
        & (x^n)^{-m} = x^{n(-m)} & \text{(by Proposition \ref{4.3.10})} \\
        \implies & (x^n)^{-m} = x^{n((-1)m)} & \text{(by Additional Corollary \ref{ac 4.2.3})} \\
        \implies & (x^n)^{-m} = x^{(n(-1))m} & \text{(by Proposition \ref{4.1.6})} \\
        \implies & (x^n)^{-m} = x^{((-1)n)m} & \text{(by Proposition \ref{4.1.6})} \\
        \implies & (x^n)^{-m} = x^{(-1)(nm)} & \text{(by Proposition \ref{4.1.6})} \\
        \implies & (x^n)^{-m} = x^{-(nm)} & \text{(by Additional Corollary \ref{ac 4.2.3})} \\
        \implies & 1 / (x^n)^m = 1 / x^{nm} & \text{(by Definition \ref{4.3.11})} \\
        \implies & 1x^{nm} = 1(x^n)^m  & \text{(by Definition \ref{4.2.1})} \\
        \implies & x^{nm} = (x^n)^m.  & \text{(by Proposition \ref{4.2.4})}
        \end{align*}
    \end{enumerate}
    \item \(n\) is a negative integer.
    Then \(-n\) is a positive integer.
    Again by Lemma \ref{4.1.5}, exactly one of the following three statements is true:
    \begin{enumerate}[label=(\roman*)]
        \item \(m = 0\).
        Then by Definition \ref{4.3.9}, \((x^n)^0 = 1\) and \(x^{n0} = x^0 = 1\).
        So \((x^n)^0 = x^{n0}\).
        \item \(m\) is a positive integer.
        So
        \begin{align*}
        & (x^{-n})^m = x^{(-n)m} & \text{(by Proposition \ref{4.3.10})} \\
        \implies & (x^{-n})^m = x^{((-1)n)m} & \text{(by Additional Corollary \ref{ac 4.2.3})} \\
        \implies & (x^{-n})^m = x^{(-1)(nm)} & \text{(by Proposition \ref{4.1.6})} \\
        \implies & (x^{-n})^m = x^{-(nm)} & \text{(by Additional Corollary \ref{ac 4.2.3})} \\
        \implies & (1 / x^n)^m = 1 / x^{nm}. & \text{(by Definition \ref{4.3.11})}
        \end{align*}
        We claim that \((1 / x^n)^m = 1 / (x^n)^m\) by using induction on \(m\).
        For \(m = 0\), \((1 / x^n)^0 = 1\) and \(1 / (x^n)^0 = 1 / 1 = 1\) by Definition \ref{4.3.9}.
        So \((1 / x^n)^0 = 1 / (x^n)^0\), and the base case holds.
        Suppose inductively that for some \(m\), \((1 / x^n)^m = 1 / (x^n)^m\).
        Then for \(m++\), \((1 / x^n)^{m++} = (1 / x^n)^m \times (1 / x^n) = 1 / (x^n)^m \times (1 / x^n) = (1 \times 1) / ((x^n)^m \times x^n) = 1 / (x^n)^{m++}\) by Definition \ref{4.3.9}, induction hypothesis and Definition \ref{4.2.2}.
        This close the induction.
        So
        \begin{align*}
        & 1 / (x^n)^m = 1 / x^{nm} \\
        \implies & 1x^{nm} = 1(x^n)^m & \text{(by Definition \ref{4.2.1})} \\
        \implies & x^{nm} = (x^n)^m. & \text{(by Proposition \ref{4.2.4})}
        \end{align*}
        \item \(m\) is a negative integer.
        Then \(-m\) is a positive integer.
        So
        \begin{align*}
        & (x^n)^m \\
        &= 1 / (x^n)^{-m} & \text{(by Definition \ref{4.3.11})} \\
        &= 1 / x^{n(-m)} & \text{(from case above)} \\
        &= 1 / x^{n((-1)m)} & \text{(by Additional Corollary \ref{ac 4.2.3})} \\
        &= 1 / x^{(n(-1))m} & \text{(by Proposition \ref{4.2.4})} \\
        &= 1 / x^{((-1)n)m} & \text{(by Proposition \ref{4.2.4})} \\
        &= 1 / x^{(-1)(nm)} & \text{(by Proposition \ref{4.2.4})} \\
        &= 1 / x^{-(nm)} & \text{(by Additional Corollary \ref{ac 4.2.3})} \\
        &= x^{nm}. & \text{(by Definition \ref{4.3.11})}
        \end{align*}
    \end{enumerate}
\end{enumerate}
From all cases above, we can conclude that \((x^n)^m = x^{nm}\).

Finally we prove that \((xy)^n = x^n y^n\).
By Lemma \ref{4.1.5}, exactly one of the following three statements is true:
\begin{enumerate}[label=(\roman*)]
    \item \(n = 0\).
    Then by Proposition \ref{4.3.10}, \((xy)^0 = x^0 y^0\).
    \item \(n\) is a positive integer.
    Then by Proposition \ref{4.3.10}, \((xy)^n = x^n y^n\).
    \item \(n\) is a negative integer.
    Then \(-n\) is a positive integer.
    So
    \begin{align*}
    & (xy)^{-n} = x^{-n} y^{-n} & \text{(by Proposition \ref{4.3.10})} \\
    \implies & 1 / (xy)^n = (1 / x^n)(1 / y^n) & \text{(by Definition \ref{4.3.11})} \\
    \implies & 1 / (xy)^n = (1 \times 1) / (x^n y^n) & \text{(by Definition \ref{4.2.2})} \\
    \implies & 1 / (xy)^n = 1 / (x^n y^n) \\
    \implies & 1(x^n y^n) = 1(xy)^n & \text{(by Definition \ref{4.2.1})} \\
    \implies & (x^n y^n) = (xy)^n. & \text{(by Proposition \ref{4.2.4})}
    \end{align*}
\end{enumerate}
From all cases above, we can conclude that \((xy)^n = x^n y^n\).
\end{proof}

\begin{proof}{(b)}
By Definition \ref{4.2.8}, \(x \geq y > 0 \implies x \geq y \geq 0\).
If \(n\) is a positive integer, then by Proposition \ref{4.3.10}, \(x \geq y \geq 0 \implies x^n \geq y^n \geq 0\).
By the given conditions and Proposition \ref{4.3.10}, \(y \neq 0 \implies y^n \neq 0\).
So \(x \geq y > 0 \implies x^n \geq y^n > 0\) when \(n\) is a positive integer.

If \(n\) is a negative integer, then let \(n = -a\), where \(a\) is a positive integer.
By the given conditions we have two cases:
\begin{enumerate}[label=(\roman*)]
    \item \(x > y\).
    By Definition \ref{4.2.6}, \((x > 0 \implies 1 / x > 0) \land (y > 0 \implies 1 / y > 0)\).
    By Additional Corollary \ref{ac 4.2.5} and Definition \ref{4.2.2}, \((1 / x) \times (1 / y) = 1 / xy > 0\).
    So \(x > y \implies x \times (1 / xy) > y \times (1 / xy) \implies 1 / y > 1 / x\) by Proposition \ref{4.2.9}.
    From previous prove, we get \(1 / y \geq 1 / x > 0 \implies (1 / y)^a \geq (1 / x)^a > 0\).
    But by Definition \ref{4.3.11}, Proposition \ref{4.3.12}(a) and Additional Corollary \ref{ac 4.2.3}, \((1 / y)^a = (y^{-1})^a = y^{(-1)a} = y^{-a} = y^n\) and \((1 / x)^a = (x^{-1})^a = x^{(-1)a} = x^{-a} = x^n\).
    So \(x \geq y > 0 \implies y^n \geq x^n > 0\) when \(n\) is a negative integer.
    \item \(x = y\).
    Then \(x^a = y^a > 0\) by Proposition \ref{4.3.10}.
    By Definition \ref{4.2.6}, \((x^a > 0 \implies 1 / x^a > 0) \land (y^a > 0 \implies 1 / y^a > 0)\).
    But by Definition \ref{4.3.11}, \((1 / x^a = x^{-a} = x^n) \land (1 / y^a = y^{-a} = y^n)\).
    And by Definition \ref{4.2.8}, \(x^n = y^n \implies y^n \geq x^n\).
    So \(x \geq y > 0 \implies y^n \geq x^n > 0\) when \(n\) is a negative integer.
\end{enumerate}
From all cases above, we conclude that \(x \geq y > 0 \implies y^n \geq x^n > 0\) when \(n\) is a negative integer.
\end{proof}

\begin{proof}{(c)}
Suppose for sake of contradiction that \(x \neq y\)
Then by Proposition \ref{4.2.9}, exactly one of the following two statements is true:
\begin{enumerate}[label=(\Roman*)]
    \item \(x > y\).
    Then by Lemma \ref{4.1.5} and the given conditions, exactly one of the following two statements is true:
    \begin{enumerate}[label=(\roman*)]
        \item \(n\) is a positive integer.
        But by Proposition \ref{4.3.10}, \(x^n > y^n\), a contradiction.
        \item \(n\) is a negative integer.
        Let \(n = -a\), where \(a\) is a negative integer.
        By Definition \ref{4.2.6}, \((x > 0 \implies 1 / x > 0) \land (y > 0 \implies 1 / y > 0)\).
        By Additional Corollary \ref{ac 4.2.5} and Definition \ref{4.2.2}, \((1 / x) \times (1 / y) = 1 / xy > 0\).
        So \(x > y \implies x \times (1 / xy) > y \times (1 / xy) \implies 1 / y > 1 / x\) by Proposition \ref{4.2.9}.
        By Proposition \ref{4.3.10}, \((1 / y)^a > (1 / x)^a\).
        But by Definition \ref{4.3.11}, Proposition \ref{4.3.12}(a) and Additional Corollary \ref{ac 4.2.3}, \((1 / y)^a = (y^{-1})^a = y^{(-1)a} = y^{-a} = y^n\) and \((1 / x)^a = (x^{-1})^a = x^{(-1)a} = x^{-a} = x^n\).
        So \(x > y \implies y^n > x^n\), a contradiction.
    \end{enumerate}
    \item \(x < y\).
    By Proposition \ref{4.2.9}, \(x < y \implies y > x\), which just the same as \(x > y\).
\end{enumerate}
From all cases above we get a contradiction, so \(x = y\).
\end{proof}

\begin{proof}{(d)}
By Lemma \ref{4.1.5}, exactly one of the following three statements is true:
\begin{enumerate}[label=(\Roman*)]
    \item \(n = 0\).
    Then by Proposition \ref{4.3.10}, \(\abs*{x^0} = \abs*{x}^0\).
    \item \(n\) is a positive integer.
    Then by Proposition \ref{4.3.10}, \(\abs*{x^n} = \abs*{x}^n\).
    \item \(n\) is a negative integer.
    Then by Definition \ref{4.3.11}, \(\abs*{x^n} = \abs*{1 / x^{-n}}\) and \(\abs*{x}^n = 1 / \abs*{x}^{-n}\).
    By Lemma \ref{4.2.7}, exactly one of the following three statements is true:
    \begin{enumerate}[label=(\roman*)]
        \item \(x^{-n} = 0\).
        This case does not exist because \(x \neq 0 \implies x^{-n} \neq 0\).
        \item \(x^{-n}\) is a positive rational number.
        Then by Definition \ref{4.2.6} and Definition \ref{4.3.1}, \(1 / x^{-n}\) is a positive rational number and \(\abs*{1 / x^{-n}} = 1 / x^{-n} = 1 / \abs*{x^{-n}}\).
        And by Proposition \ref{4.3.10}, \(1 / \abs*{x}^{-n} = 1 / \abs*{x^{-n}}\).
        So \(\abs*{x^n} = \abs*{x}^n\).
        \item \(x^{-n}\) is a negative rational number.
        By Additional Corollary \ref{ac 4.2.3}, \(1 / x^{-n}\) is a negative rational number.
        By Definition \ref{4.3.1}, \(\abs*{1 / x^{-n}} = -(1 / x^{-n})\).
        Again by Additional Corollary \ref{ac 4.2.3}, \(-(1 / x^{-n}) = 1 / -(x^{-n})\).
        Again by Definition \ref{4.3.1}, \(1 / -(x^{-n}) = 1 / \abs*{x^{-n}}\).
        And by Proposition \ref{4.3.10}, \(1 / \abs*{x}^{-n} = 1 / \abs*{x^{-n}}\).
        So \(\abs*{x^n} = \abs*{x}^n\).
    \end{enumerate}
\end{enumerate}
From all cases above, we conclude that \(\abs*{x^n} = \abs*{x}^n\).
\end{proof}

\exercisesection

\begin{exercise}\label{ex 4.3.1}
Prove Proposition \ref{4.3.3}.
\end{exercise}

\begin{proof}
See Proposition \ref{4.3.3}.
\end{proof}

\begin{exercise}\label{ex 4.3.2}
Prove the remaining claims in Proposition \ref{4.3.7}.
\end{exercise}

\begin{proof}
See Proposition \ref{4.3.7}.
\end{proof}

\begin{exercise}\label{ex 4.3.3}
Prove Proposition \ref{4.3.10}.
\end{exercise}

\begin{proof}
See Proposition \ref{4.3.10}.
\end{proof}

\begin{exercise}
Prove Proposition \ref{4.3.12}.
\end{exercise}

\begin{proof}
See Proposition \ref{4.3.12}.
\end{proof}

\begin{exercise}\label{ex 4.3.5}
Prove that \(2^N \geq N\) for all positive integers \(N\).
\end{exercise}

\begin{proof}
We use induction on \(N\) and begin with \(N = 1\).
For \(N = 1\), \(2^1 = 2^0 \times 2 = 1 \times 2 = 2 \geq 1\) by Definition \ref{4.3.9}, so the base case holds.
Suppose inductively that for some \(N\), \(2^N \geq N\).
Then for \(N++\),
\begin{align*}
& (2N = N + N) \land (N \text{ is a positive integer}) \\
\implies & N < 2N & \text{(by Definition \ref{2.2.11})} \\
\implies & N++ \leq 2N. & \text{(by Proposition \ref{2.2.12})} \\
& 2^N \geq N & \text{(by induction hypothesis)} \\
\implies & N \leq 2^N & \text{(by Lemma \ref{4.2.3})} \\
\implies & 2N \leq 2 \times 2^N & \text{(by Lemma \ref{4.2.3})} \\
\implies & 2N \leq 2^N \times 2 & \text{(by Proposition \ref{4.2.4})} \\
\implies & 2N \leq 2^{N++} & \text{(by Definition \ref{4.3.9})} \\
\implies & N++ \leq 2^{N++} & \text{(by Proposition \ref{4.2.9})} \\
\implies & 2^{N++} \geq N++. & \text{(by Proposition \ref{4.2.9})}
\end{align*}
This close the induction.
\end{proof}
\section{Gaps in the rational numbers}\label{sec 4.4}

\begin{additional corollary}[Euclidean algorithm]\label{ac 4.4.1}
Let \(n \in \mathbf{Z}\) and let \(q \in \mathbf{Z}^+\).
Then \(\exists!\ m, r \in \mathbf{Z}\) such that \(0 \leq r < q\) and \(n = mq + r\).
\end{additional corollary}

\begin{proof}
    We first show that there exists at least one pairs of \(m, r \in \mathbf{Z}\) satisfy the statement.
    By Lemma \ref{4.1.5} exactly one of the following two statements is true:
    \begin{enumerate}
        \item \(n \geq 0\).
              Then by Proposition \ref{2.3.9} we know that such \(m, r \in \mathbf{Z}\) exist.
        \item \(n < 0\).
              Then by Additional Corollary \ref{ac 4.2.5} we have \(-n > 0\) and
              \begin{align*}
                           & \exists\ m, r \in \mathbf{Z} : (0 \leq r < q) \land (-n = mq + r) & \text{(by Proposition \ref{2.3.9})}             \\
                  \implies & (q > q - r > 0) \land (-n = mq + r)                               & \text{(by Lemma \ref{4.1.11}(a))}               \\
                  \implies & (q > q - r > 0) \land (n = (-m)q - r)                             & \text{(by Additional Corollary \ref{ac 4.1.3})} \\
                  \implies & n = (-m)q - q + q - r                                             & \text{(by Proposition \ref{4.1.6})}             \\
                  \implies & n = (-m - 1)q + (q - r).                                          & \text{(by Proposition \ref{4.1.6})}
              \end{align*}
              Since \(-m - 1 \in \mathbf{Z}\) and \(q > q - r > 0\), by setting \(m' = -m - 1\) and \(r' = q - r\) we see that \(n = m'q + r'\) satisfy the statement.
    \end{enumerate}
    From all cases above we conclude that at least one pairs of \(m, r \in \mathbf{Z}\) satisfy the statement.

    Now we show the uniqueness of such \(m, r\).
    Let \(m, m', r, r' \in \mathbf{Z}\) such that
    \[
        (n = mq + r = m'q + r') \land (0 \leq r < q) \land (0 \leq r' < q).
    \]
    Suppose for sake of contradiction that \(r \neq r'\).
    By Lemma \ref{4.1.5} exactly one of the following two statements is true:
    \begin{enumerate}
        \item \(r > r'\).
              Let \(a = r - r'\)
              By Lemma \ref{4.1.11}(a) we know that \(a \in \mathbf{Z}^+\).
              Then we have
              \begin{align*}
                           & mq + r = m'q + r'                                                         \\
                  \implies & r - r' = (m' - m)q      & \text{(by Proposition \ref{4.1.6})}             \\
                  \implies & m' - m > 0              & \text{(by Additional Corollary \ref{ac 4.2.6})} \\
                  \implies & m' - m \in \mathbf{Z}^+                                                   \\
                  \implies & (m' - m)q \geq q        & \text{(by Lemma \ref{2.2.12}(d))}               \\
                  \implies & r - r' \geq q                                                             \\
                  \implies & r \geq q + r' \geq q    & \text{(by Lemma \ref{4.1.11}(b))}
              \end{align*}
              which contradict to \(r < q\).
        \item \(r < r'\).
              By Definition \ref{4.1.10} we have \(r' > r\).
              Using similar argument above we derive a contradiction.
    \end{enumerate}
    From all cases we derive contradictions, thus we must have \(r = r'\).
    This means
    \begin{align*}
                 & mq + r = m'q + r'                                       \\
        \implies & mq + r = m'q + r                                        \\
        \implies & mq = m'q          & \text{(by Proposition \ref{4.1.6})} \\
        \implies & m = m'.           & \text{(by Corollary \ref{4.1.9})}
    \end{align*}
    Thus we conclude that for every \(n \in \mathbf{Z}\) and \(q \in \mathbf{Z}^+\), \(\exists!\ m, r \in \mathbf{Z}\) such that \(n = mq + r\) and \(0 \leq r < q\).
\end{proof}

\begin{proposition}[Interspersing of integers by rationals]\label{4.4.1}
    Let \(x\) be a rational number.
    Then there exists an integer \(n\) such that \(n \leq x < n + 1\).
    In fact, this integer is unique (i.e., for each \(x\) there is only one \(n\) for which \(n \leq x < n + 1\)).
    In particular, there exists a natural number \(N\) such that \(N > x\)
    (i.e., there is no such thing as a rational number which is larger than all the natural numbers).
\end{proposition}

\begin{proof}
    By Definition \ref{4.2.1} we know that \(x = a / b\) where \(a, b \in \mathbf{Z}\) and \(b > 0\).
    Since \(a \in \mathbf{Z}\) and \(b \in \mathbf{Z}^+\), by Additional Corollary \ref{ac 4.4.1} we know that \(\exists!\ m, r \in \mathbf{Z}\) such that \(a = mb + r\) and \(0 \leq r < b\).
    Then we have
    \begin{align*}
                 & (a = mb + r) \land (0 \leq r < b)                                                                           \\
        \implies & (x = \frac{a}{b} = m + \frac{r}{b}) \land (0 \leq r < b)           & \text{(by Proposition \ref{4.2.4})}    \\
        \implies & (x = \frac{a}{b} = m + \frac{r}{b}) \land (0 \leq \frac{r}{b} < 1) & \text{(by Proposition \ref{4.2.9}(e))} \\
        \implies & m \leq x = m + \frac{r}{b} < m + 1.                                & \text{(by Proposition \ref{4.2.9}(d))}
    \end{align*}
    Note that such \(m\) is unique by Additional Corollary \ref{4.4.2}.
    If \(m + 1 < 0\), then by setting \(N = 0\) we have \(x < N\).
    If \(m + 1 \geq 0\), then by setting \(N = m + 1\) we again have \(x < N\).
    Thus we conclude that \(\forall\ x \in \mathbf{Q}\), \(\exists\ N \in \mathbf{N}\) such that \(x < N\).
\end{proof}

\begin{remark}\label{4.4.2}
    The integer \(n\) for which \(n \leq x < n + 1\) is sometimes referred to as the \emph{integer part} of \(x\) and is sometimes denoted \(n = \floor*{x}\).
\end{remark}

\begin{proposition}[Interspersing of rationals by rationals]\label{4.4.3}
    If \(x\) and \(y\) are two rationals such that \(x < y\), then there exists a third rational \(z\) such that \(x < z < y\).
\end{proposition}

\begin{proof}
    We set \(z \coloneqq (x + y) / 2\).
    Since \(x < y\), and \(1 / 2 = 1 // 2\) is positive, we have from Proposition \ref{4.2.9} that \(x / 2 < y / 2\).
    If we add \(y / 2\) to both sides using Proposition \ref{4.2.9} we obtain \(x / 2 + y / 2 < y / 2 + y / 2\), i.e., \(z < y\).
    If we instead add \(x / 2\) to both sides we obtain \(x / 2 + x / 2 < y / 2 + x / 2\), i.e., \(x < z\).
    Thus \(x < z < y\) as desired.
\end{proof}

\begin{note}
    Despite the rationals having this denseness property, they are still incomplete;
    there are still an infinite number of ``gaps'' or ``holes'' between the rationals, although this denseness property does ensure that these holes are in some sense infinitely small.
\end{note}

\begin{additional corollary}\label{ac 4.4.2}
Let \(n, m\) be two natural numbers.
Define \(n\) to be even if \(n = 2m\), and odd if \(n = 2m + 1\).
Then every natural number is either even or odd, but not both.
\end{additional corollary}

\begin{proof}
    We use induction on \(n\).
    For \(n = 0\), by Definition \ref{2.3.1} and Lemma \ref{2.3.2} we have \(0 = 0 \times 2 = 2 \times 0\).
    By Axiom \ref{2.3} we have \(0 \neq 2m + 1\).
    Thus \(0\) is even and is not odd, so the base case holds.
    Suppose inductively that for some \(n \geq 0\), \(\exists\ m \in \mathbf{N}\) such that either \(n = 2m\) or \(n = 2m + 1\) is true, but not both.
    Then for \(n + 1\), by induction hypothesis we can split into two cases:
    \begin{enumerate}
        \item If \(n = 2m\), then \(n + 1 = 2m + 1\), which means \(n + 1\) is odd.
        \item If \(n = 2m + 1\), then by Proposition \ref{2.2.5} and Proposition \ref{2.3.4} we have \(n + 1 = 2m + 2 = 2(m + 1)\), which means \(n + 1\) is even.
    \end{enumerate}
    By induction hypothesis the two cases can not be true at the same time, thus \(n + 1\) is also either even or odd, but not both.
    This close the induction.
\end{proof}

\begin{additional corollary}\label{ac 4.4.3}
Let \(n\) be a natural number.
If \(n\) is even, then \(n^2\) is also even.
If \(n\) is odd, then \(n^2\) is also odd.
\end{additional corollary}

\begin{proof}
    We first show that \(n\) is even implies \(n^2\) is even.
    Since \(n\) is even, by Additional Corollary \ref{ac 4.4.2} \(\exists\ m \in \mathbf{N}\) such that \(n = 2m\).
    Then we have
    \begin{align*}
        n^2 & = (2m)^2                                                 \\
            & = (2m)(2m)         & \text{(by Definition \ref{2.3.11})} \\
            & = 2\big(m(2m)\big) & \text{(by Proposition \ref{2.3.5})}
    \end{align*}
    and thus by Additional Corollary \ref{ac 4.4.2} \(n^2\) is even.

    Now we show that \(n\) is odd implies \(n^2\) is odd.
    Since \(n\) is odd, by Additional Corollary \ref{ac 4.4.2} \(\exists\ m \in \mathbf{N}\) such that \(n = 2m + 1\).
    Then we have
    \begin{align*}
        n^2 & = (2m + 1)^2                                                         \\
            & = (2m)^2 + 2(2m)1 + 1^2        & \text{(by Exercise \ref{ex 2.3.4})} \\
            & = (2m)(2m) + 2(2m) + 1         & \text{(by Definition \ref{2.3.11})} \\
            & = 2\big(m(2m)\big) + 2(2m) + 1 & \text{(by Proposition \ref{2.3.5})} \\
            & = 2\big(m(2m) + 2m\big) + 1    & \text{(by Proposition \ref{2.3.4})}
    \end{align*}
    and thus by Additional Corollary \ref{ac 4.4.2} \(n^2\) is odd.
\end{proof}

\begin{proposition}\label{4.4.4}
    There does not exist any rational number \(x\) for which \(x^2 = 2\).
\end{proposition}

\begin{proof}
    Suppose for sake of contradiction that we had a rational number \(x\) for which \(x^2 = 2\).
    Clearly \(x\) is not zero.
    We may assume that \(x\) is positive, for if \(x\) were negative then we could just replace \(x\) by \(-x\)
    (since \(x^2 = (-x)^2\)).
    Thus \(x = p / q\) for some positive integers \(p, q\), so \((p / q)^2 = 2\), which we can rearrange as \(p^2 = 2q^2\).
    Define a natural number \(p\) to be even if \(p = 2k\) for some natural number \(k\), and odd if \(p = 2k + 1\) for some natural number \(k\).
    By Additional Corollary \ref{ac 4.4.2}, every natural number is either even or odd, but not both.
    By Additional Corollary \ref{ac 4.4.3}, if \(p\) is odd, then \(p^2\) is also odd, which contradicts \(p^2 = 2q^2\).
    Thus \(p\) is even, i.e., \(p = 2k\) for some natural number \(k\).
    Since \(p\) is positive, \(k\) must also be positive.
    Inserting \(p = 2k\) into \(p^2 = 2q^2\) we obtain \(4k^2 = 2q^2\), so that \(q^2 = 2k^2\).

    To summarize, we started with a pair \((p, q)\) of positive integers such that \(p^2 = 2q^2\), and ended up with a pair \((q, k)\) of positive integers such that \(q^2 = 2k^2\).
    Since \(p^2 = 2q^2\), by Proposition \ref{2.2.12} we have \(p^2 = q^2 + q^2 \implies p^2 > q^2\).
    If \(p < q\), then by Proposition \ref{2.3.6}, \(p^2 < pq\) and \(pq < q^2\).
    So by Proposition \ref{2.2.12}, \(p^2 < q^2\), a contradiction.
    Thus we have \(q < p\).
    If we rewrite \(p' \coloneqq q\) and \(q' \coloneqq k\), we thus can pass from one solution \((p, q)\) to the equation \(p^2 = 2q^2\) to a new solution \((p', q')\) to the same equation which has a smaller value of \(p\).
    But then we can repeat this procedure again and again, obtaining a sequence \((p'', q'')\), \((p''', q''')\), etc. of solutions to \(p^2 = 2q^2\), each one with a smaller value of \(p\) than the previous, and each one consisting of positive integers.
    But this contradicts the principle of infinite descent (see Exercise \ref{ex 4.4.2}).
    This contradiction shows that we could not have had a rational \(x\) for which \(x^2 = 2\).
\end{proof}

\begin{proposition}\label{4.4.5}
    For every rational number \(\varepsilon > 0\), there exists a non-negative rational number \(x\) such that \(x^2 < 2 < (x + \varepsilon)^2\).
\end{proposition}

\begin{proof}
    Let \(\varepsilon > 0\) be rational.
    Suppose for sake of contradiction that there is no non-negative rational number \(x\) for which \(x^2 < 2 < (x + \varepsilon)^2\).
    This means that whenever \(x\) is non-negative and \(x^2 < 2\), we must also have \((x + \varepsilon)^2 < 2\)
    (note that \((x + \varepsilon)^2\) cannot equal \(2\), by Proposition \ref{4.4.4}).
    Since \(0^2 < 2\), we thus have \(\varepsilon^2 < 2\), which then implies \((2\varepsilon)^2 < 2\), and indeed a simple induction shows that \((n\varepsilon)^2 < 2\) for every natural number \(n\).
    (Note that \(n\varepsilon\) is non-negative for every natural number \(n\) by Additional Corollary \ref{ac 4.2.5})
    But, by Proposition \ref{4.4.1} we can find an integer \(n\) such that \(n > 2 / \varepsilon\), which implies that \(n\varepsilon > 2\), which implies that \((n\varepsilon)^2 > 4 > 2\), contradicting the claim that \((n\varepsilon)^2 < 2\) for all natural numbers \(n\).
    This contradiction gives the proof.
\end{proof}

\begin{note}
    Proposition \ref{4.4.5} indicates that, while the set \(\mathbf{Q}\) of rationals does not actually have \(\sqrt{2}\) as a member, we can get as close as we wish to \(\sqrt{2}\).
    For instance, the sequence of rationals
    \[
        1.4, 1.41, 1.414, 1.4142, 1.41421, \dots
    \]
    seem to get closer and closer to \(\sqrt{2}\), as their squares indicate:
    \[
        1.96, 1.9881, 1.99396, 1.99996164, 1.9999899241, \dots
    \]
    Thus it seems that we can create a square root of \(2\) by taking a ``limit'' of a sequence of rationals.
    This is how we shall construct the real numbers in the next chapter.
\end{note}

\begin{note}
    There is another way to construct the real numbers, using something called ``Dedekind cuts'', which we will not pursue here.
    One can also proceed using infinite decimal expansions, but there are some sticky issues when doing so, e.g., one has to make \(0.999\dots\) equal to \(1.000\dots\), and this approach, despite being the most familiar, is actually more complicated than other approaches.
\end{note}

\exercisesection

\begin{exercise}\label{ex 4.4.1}
    Prove Proposition \ref{4.4.1}.
\end{exercise}

\begin{proof}
    See Proposition \ref{4.4.1}.
\end{proof}

\begin{exercise}\label{ex 4.4.2}
    A definition: a sequence \(a_0, a_1, a_2, \dots\) of numbers (natural numbers, integers, rationals, or reals) is said to be in \emph{infinite descent} if we have \(a_n > a_{n + 1}\) for all natural numbers \(n\)
    (i.e., \(a_0 > a_1 > a_2 > \dots\)).
    \begin{enumerate}
        \item Prove the \emph{principle of infinite descent}:
              that it is not possible to have a sequence of \emph{natural numbers} which is in infinite descent.
        \item Does the principle of infinite descent work if the sequence \(a_1, a_2, a_3, \dots\) is allowed to take integer values instead of natural number values?
              What about if it is allowed to take positive rational values instead of natural numbers?
              Explain.
    \end{enumerate}
\end{exercise}

\begin{proof}{(a)}
    Suppose for sake of contradiction that there exists a sequence of natural numbers \(a_0, a_1, a_2, \dots\) is in infinite descent.
    Since all the \(a_n\) are natural numbers, \(a_n \geq 0\) for all \(n \in \mathbf{N}\).
    Now we use induction on \(k\) to show in fact that \(a_n \geq k\) for all \(k \in \mathbf{N}\) and all \(n \in \mathbf{N}\).
    For \(k = 0\), because \(a_n \geq 0\) for all \(n \in \mathbf{N}\), so the base case holds.
    Suppose inductively that for some \(k\), \(a_n \geq k\) for all \(n \in \mathbf{N}\).
    Then for \(k++\), we want to show that \(a_n \geq k++\) for all \(n \in \mathbf{N}\).
    By induction hypothesis, \(\forall\ m \in \mathbf{N}\) such that \(a_m \geq k\), which also implies \(a_{m++} \geq k\).
    And because the sequence is in infinite descent, \(a_m > a_{m++}\).
    So \(a_m > a_{m++} \geq k \implies a_m > k\) by Proposition \ref{2.2.12}.
    Again by Proposition \ref{2.2.12}, \(a_m > k \implies a_m \geq k++\).
    This close the induction.

    Now we show that such sequence does not exist.
    Because \(\forall\ k \in \mathbf{N}\), \(a_n \geq k \ \forall\ n \in \mathbf{N}\).
    We set \(k = a_0\).
    Then \(a_n \geq a_0\), which contradicts to the sequence which is in infinite descent.
    So such sequence does not exist.
\end{proof}

\begin{proof}{(b)}
    By setting \(a_n = -n \ \forall\ n \in \mathbf{N}\), we can always have \(a_n > a_{n + 1}\).
    So the principle of infinite descent does not work on integers.

    Similarly, by setting \(a_n = 1 / n \ \forall\ n \in \mathbf{N}\), we can always have \(a_n > a_{n + 1}\).
    So the principle of infinite descent does not work on rationals.
\end{proof}

\begin{exercise}\label{ex 4.4.3}
    Fill in the gaps marked (why?) in the proof of Proposition \ref{4.4.4}.
\end{exercise}

\begin{proof}
    See Proposition \ref{4.4.4}.
\end{proof}
\chapter{The real numbers}

\begin{note}
We defined the natural numbers using the five Peano axioms, and postulated that such a number system existed;
this is plausible, since the natural numbers correspond to the very intuitive and fundamental notion of \emph{sequential counting}.
\end{note}

\begin{note}
The symbols \(\mathds{N}\), \(\mathds{Q}\), and \(\mathds{R}\) stand for ``natural'', ``quotient'', and ``real'' respectively.
\(\mathds{Z}\) stands for ``Zahlen'', the German word for numbers.
There is also the \emph{complex numbers} \(\mathds{C}\), which obviously stands for ``complex''.
\end{note}

\begin{note}
\emph{Formal} means ``having the form of'';
at the beginning of our construction the expression \(a \text{-----} b\) did not actually \emph{mean} the difference \(a - b\), since the symbol \text{-----} was meaningless.
It only had the \emph{form} of a difference.
Later on we defined subtraction and verified that the formal difference was equal to the actual difference, so this eventually became a non-issue, and our symbol for formal differencing was discarded.
Somewhat confusingly, this use of the term ``formal'' is unrelated to the notions of a formal argument and an informal argument.
\end{note}

\begin{note}
There is a fundamental area of mathematics where the rational number system does not suffice - that of \emph{geometry}
(the study of lengths, areas, etc.).
For instance, a right-angled triangle with both sides equal to \(1\) gives a hypotenuse of \(\sqrt{2}\), which is an \emph{irrational} number, i.e., not a rational number;
see Proposition \ref{4.4.4}.
Things get even worse when one starts to deal with the sub-field of geometry known as \emph{trigonometry}, when one sees numbers such as \(\pi\) or \(\cos(1)\), which turn out to be in some sense ``even more'' irrational than \(\sqrt{2}\).
(These numbers are known as \emph{transcendental numbers}, but to discuss this further would be far beyond the scope of this text.)
Thus, in order to have a number system which can adequately describe geometry
- or even something as simple as measuring lengths on a line
- one needs to replace the rational number system with the real number system.
\end{note}

\begin{note}
In the constructions of integers and rationals, the task was to introduce one more \emph{algebraic} operation to the number system
- e.g., one can get integers from naturals by introducing subtraction, and get the rationals from the integers by introducing division.
But to get the reals from the rationals is to pass from a ``discrete'' system to a ``continuous'' one, and requires the introduction of a somewhat different notion
- that of a \emph{limit}.
\end{note}

\begin{note}
The limit is a concept which on one level is quite intuitive, but to pin down rigorously turns out to be quite difficult.
(Even such great mathematicians as Euler and Newton had difficulty with this concept.
It was only in the nineteenth century that mathematicians such as Cauchy and Dedekind figured out how to deal with limits rigorously.)
\end{note}

\begin{note}
The procedure we give here of obtaining the real numbers as the limit of sequences of rational numbers may seem rather complicated.
However, it is in fact an instance of a very general and useful procedure, that of \emph{completing} one metric space to form another.
\end{note}

\section{Cauchy sequences}\label{sec 5.1}

\begin{definition}[Sequences]\label{5.1.1}
    Let \(m\) be an integer.
    A \emph{sequence \((a_n)_{n = m}^{\infty}\) of rational numbers} is any function from the set \(\{n \in \mathbf{Z} : n \geq m\}\) to \(\mathbf{Q}\), i.e., a mapping which assigns to each integer \(n\) greater than or equal to \(m\), a rational number \(a_n\).
    More informally, a sequence \((a_n)_{n = m}^{\infty}\) of rational numbers is a collection of rationals \(a_m, a_{m + 1}, a_{m + 2}, \dots\).
\end{definition}

\setcounter{theorem}{2}
\begin{definition}[\(\varepsilon\)-steadiness]\label{5.1.3}
    Let \(\varepsilon > 0\).
    A sequence \((a_n)_{n = 0}^{\infty}\) is said to be \emph{\(\varepsilon\)-steady} iff each pair \(a_j\), \(a_k\) of sequence elements is \(\varepsilon\)-close for every natural number \(j, k\).
    In other words, the sequence \(a_0, a_1, a_2, \dots\) is \(\varepsilon\)-steady iff \(d(a_j, a_k) \leq \varepsilon\) for all \(j, k\).
\end{definition}

\begin{remark}\label{5.1.4}
    Definition \ref{5.1.3} is not standard in the literature;
    we will not need it outside of this section;
    similarly for the concept of ``eventual \(\varepsilon\)-steadiness'' below.
    We have defined \(\varepsilon\)-steadiness for sequences whose index starts at \(0\), but clearly we can make a similar notion for sequences whose indices start from any other number:
    a sequence \(a_N, a_{N + 1}, \dots\) is \(\varepsilon\)-steady if one has \(d(a_j, a_k) \leq \varepsilon\) for all \(j, k \geq N\).
\end{remark}

\begin{note}
    The notion of \(\varepsilon\)-steadiness of a sequence is simple, but does not really capture the \emph{limiting} behavior of a sequence, because it is too sensitive to the initial members of the sequence.
    So we need a more robust notion of steadiness that does not care about the initial members of a sequence.
\end{note}

\setcounter{theorem}{5}
\begin{definition}[Eventual \(\varepsilon\)-steadiness]\label{5.1.6}
    Let \(\varepsilon > 0\).
    A sequence \((a_n)_{n = 0}^{\infty}\) is said to be \emph{eventually \(\varepsilon\)-steady} iff the sequence \(a_N, a_{N + 1}, a_{N + 2}, \dots\) is \(\varepsilon\)-steady for some natural number \(N \geq 0\).
    In other words, the sequence \(a_0, a_1, a_2, \dots\) is eventually \(\varepsilon\)-steady iff there exists an \(N \geq 0\) such that \(\abs*{a_j - a_k} \leq \varepsilon\) for all \(j, k \geq N\).
\end{definition}

\setcounter{theorem}{7}
\begin{definition}[Cauchy sequences]\label{5.1.8}
    A sequence \((a_n)_{n = 0}^{\infty}\) of rational numbers is said to be a \emph{Cauchy sequence} iff for every rational \(\varepsilon > 0\), the sequence \((a_n)_{n = 0}^{\infty}\) is eventually \(\varepsilon\)-steady.
    In other words, the sequence \(a_0, a_1, a_2, \dots\) is a Cauchy sequence iff for every \(\varepsilon > 0\), there exists an \(N \geq 0\) such that \(\abs*{a_j - a_k} \leq \varepsilon\) for all \(j, k \geq N\).
\end{definition}

\begin{remark}\label{5.1.9}
    At present, the parameter \(\varepsilon\) is restricted to be a positive rational;
    we cannot take \(\varepsilon\) to be an arbitrary positive real number, because the real numbers have not yet been constructed.
    However, once we do construct the real numbers, we shall see that Definition \ref{5.1.8} will not change if we require \(\varepsilon\) to be real instead of rational.
    In other words, we will eventually prove that a sequence is eventually \(\varepsilon\)-steady for every rational \(\varepsilon > 0\) if and only if it is eventually \(\varepsilon\)-steady for every real \(\varepsilon > 0\).
    This rather subtle distinction between a rational \(\varepsilon\) and a real \(\varepsilon\) turns out not to be very important in the long run, and the reader is advised not to pay too much attention as to what type of number \(\varepsilon\) should be.
\end{remark}

\setcounter{theorem}{10}
\begin{proposition}\label{5.1.11}
    The sequence \(a_1, a_2, a_3, \dots\) defined by \(a_n \coloneqq 1 / n\) (i.e., the sequence \(1, 1 / 2, 1 / 3, \dots\)) is a Cauchy sequence.
\end{proposition}

\begin{proof}
    We have to show that for every \(\varepsilon > 0\), the sequence \(a_1, a_2, \dots\) is eventually \(\varepsilon\)-steady.
    So let \(\varepsilon > 0\) be arbitrary.
    We now have to find a number \(N \geq 1\) such that the sequence \(a_N, a_{N + 1}, \dots\) is \(\varepsilon\)-steady.
    Let us see what this means.
    This means that \(d(a_j, a_k) \leq \varepsilon\) for every \(j, k \geq N\), i.e.
    \[
        \abs*{1 / j - 1 / k} \leq \varepsilon \text{ for every } j, k \geq N.
    \]
    Now since \(j, k \geq N\), we know that \(0 < 1 / j, 1 / k \leq 1 / N\), so that
    \begin{align*}
                 &
        \begin{cases}
            0 \leq \frac{1}{j} \leq \frac{1}{N} \\
            0 \leq \frac{1}{k} \leq \frac{1}{N} \\
        \end{cases}
        \\
        \implies &
        \begin{cases}
            0 \leq \frac{1}{j} \leq \frac{1}{N}   \\
            \frac{-1}{N} \leq \frac{-1}{k} \leq 0 \\
        \end{cases}
                 & \text{(by Exercise \ref{ex 4.2.6})}                                                                \\
        \implies & \frac{-1}{N} \leq \frac{1}{j} - \frac{1}{k} \leq \frac{1}{N} & \text{(by Proposition \ref{4.2.9})} \\
        \implies & \abs*{\frac{1}{j} - \frac{1}{k}} \leq \frac{1}{N}.           & \text{(by Proposition \ref{4.3.3})}
    \end{align*}
    So in order to force \(\abs*{1 / j - 1 / k}\) to be less than or equal to \(\varepsilon\), it would be sufficient for \(1 / N\) to be less than \(\varepsilon\).
    So all we need to do is choose an \(N\) such that \(1 / N\) is less than \(\varepsilon\), or in other words that \(N\) is greater than \(1 / \varepsilon\).
    But this can be done thanks to Proposition \ref{4.4.1}.
\end{proof}

\begin{note}
    As you can see, verifying from first principles (i.e., without using any of the machinery of limits, etc.) that a sequence is a Cauchy sequence requires some effort, even for a sequence as simple as \(1 / n\).
    The part about selecting an \(N\) can be particularly difficult for beginners
    - one has to think in reverse, working out what conditions on \(N\) would suffice to force the sequence \(a_N, a_{N + 1}, a_{N + 2}, \dots\) to be \(\varepsilon\)-steady, and then finding an \(N\) which obeys those conditions.
    Later we will develop some limit laws which allow us to determine when a sequence is Cauchy more easily.
\end{note}

\begin{definition}[Bounded sequences]\label{5.1.12}
    Let \(M \geq 0\) be rational.
    A finite sequence \(a_1, a_2, \dots, a_n\) is \emph{bounded by \(M\)} iff \(\abs*{a_i} \leq M\) for all \(1 \leq i \leq n\).
    An infinite sequence \((a_n)_{n = 1}^{\infty}\) is \emph{bounded by \(M\)} iff \(\abs*{a_i} \leq M\) for all \(i \geq 1\).
    A sequence is said to be \emph{bounded} iff it is bounded by \(M\) for some rational \(M \geq 0\).
\end{definition}

\setcounter{theorem}{13}
\begin{lemma}[Finite sequences are bounded]\label{5.1.14}
    Every finite sequence \(a_1, a_2, \dots, a_n\) is bounded.
\end{lemma}

\begin{proof}
    We prove this by induction on \(n\).
    When \(n = 1\) the sequence \(a_1\) is clearly bounded, for if we choose \(M \coloneqq \abs*{a_1}\) then clearly we have \(\abs*{a_i} \leq M\) for all \(1 \leq i \leq n\).
    Now suppose that we have already proved the lemma for some \(n \geq 1\);
    we now prove it for \(n + 1\), i.e., we prove every sequence \(a_1, a_2, \dots, a_{n + 1}\) is bounded.
    By the induction hypothesis we know that \(a_1, a_2, \dots, a_n\) is bounded by some \(M \geq 0\);
    in particular, it must be bounded by \(M + \abs*{a_{n + 1}}\).
    On the other hand, \(a_{n + 1}\) is also bounded by \(M + \abs*{a_{n + 1}}\).
    Thus \(a_1, a_2, \dots, a_n, a_{n++}\) is bounded by \(M + \abs*{a_{n + 1}}\), and is hence bounded.
    This closes the induction.
\end{proof}

\begin{note}
    While this argument shows that every finite sequence is bounded, no matter how long the finite sequence is, it does not say anything about whether an infinite sequence is bounded or not;
    infinity is not a natural number.
\end{note}

\begin{lemma}[Cauchy sequences are bounded]\label{5.1.15}
    Every Cauchy sequence \((a_n)_{n = 1}^{\infty}\) is bounded.
\end{lemma}

\begin{proof}
    Let \(n \in \mathbf{N}\) and \((a_n)_{n = 1}^{\infty}\) be a rational Cauchy sequence.
    Since \((a_n)_{n = 1}^{\infty}\) is a Cauchy sequence, by Definition \ref{5.1.8} \(\forall\ \varepsilon \in \mathbf{Q}^+\), we know that \((a_n)_{n = 1}^{\infty}\) is eventually \(\varepsilon\)-steady.
    In particular, \((a_n)_{n = 1}^{\infty}\) is eventually \(1\)-steady.
    By Definition \ref{5.1.6}, \(\exists\ N \in \mathbf{Z}^+\) such that \((a_n)_{n = N}^{\infty}\) is \(1\)-steady.
    Then we can split \((a_n)_{n = 1}^{\infty}\) into two sequences:
    \begin{itemize}
        \item A finite sequence \((a_n)_{n = 1}^{N - 1}\).
              Then by Lemma \ref{5.1.14} \((a_n)_{n = 1}^{N - 1}\) is bounded by some \(M \in \mathbf{Q} \setminus \mathbf{Q}^-\).
        \item An infinite sequence \((a_n)_{n = N}^{\infty}\).
              Since \((a_n)_{n = N}^\infty\) is \(1\)-steady, we have
              \begin{align*}
                           & \forall\ j \in \mathbf{Z}^+ \land j \geq N, \abs*{a_j - a_N} \leq 1           & \text{(by Definition \ref{5.1.3})}     \\
                  \implies & \abs*{a_j - a_N} + \abs*{a_N} \leq 1 + \abs*{a_N}                             & \text{(by Proposition \ref{4.2.9}(d))} \\
                  \implies & \abs*{a_j - a_N + a_N} \leq \abs*{a_j - a_N} + \abs*{a_N} \leq 1 + \abs*{a_N} & \text{(by Proposition \ref{4.3.3}(b))} \\
                  \implies & \abs*{a_j} \leq 1 + \abs*{a_N}.
              \end{align*}
              Thus by Definition \ref{5.1.12} \((a_n)_{n = N}^\infty\) is bounded by \(1 + \abs*{a_N}\).
    \end{itemize}
    Now let \(M' = M + 1 + \abs*{a_N}\).
    Then we have
    \begin{align*}
                 & \begin{cases}
            \abs*{a_n} \leq M              & \text{if } 1 \leq n \leq N - 1 \\
            \abs*{a_n} \leq 1 + \abs*{a_N} & \text{if } n \geq N
        \end{cases}                                                        \\
        \implies & \begin{cases}
            \abs*{a_n} \leq M \leq M'              & \text{if } 1 \leq n \leq N - 1 \\
            \abs*{a_n} \leq 1 + \abs*{a_N} \leq M' & \text{if } n \geq N
        \end{cases}               & \text{(by Proposition \ref{4.2.9}(c))} \\
        \implies & \forall\ n \geq 1, \abs*{a_n} \leq M'                                             \\
        \implies & (a_n)_{n = 1}^\infty \text{ is bounded}. & \text{(by Definition \ref{5.1.12})}
    \end{align*}
\end{proof}

\exercisesection

\begin{exercise}\label{ex 5.1.1}
    Prove Lemma \ref{5.1.15}.
\end{exercise}

\begin{proof}
    See Lemma \ref{5.1.15}.
\end{proof}
\section{Equivalent Cauchy sequences}

\begin{definition}[\(\varepsilon\)-close sequences]\label{5.2.1}
Let \((a_n)_{n = 0}^{\infty}\) and \((b_n)_{n = 0}^{\infty}\) be two sequences, and let \(\varepsilon > 0\).
We say that the sequence \((a_n)_{n = 0}^{\infty}\) is \emph{\(\varepsilon\)-close} to \((b_n)_{n = 0}^{\infty}\) iff \(a_n\) is \(\varepsilon\)-close to \(b_n\) for each \(n \in \mathds{N}\).
In other words, the sequence \(a_0, a_1, a_2, \dots\) is \(\varepsilon\)-close to the sequence \(b_0, b_1, b_2, \dots\) iff \(|a_n - b_n| \leq \varepsilon\) for all \(n = 0, 1, 2, \dots\).
\end{definition}

\setcounter{theorem}{2}
\begin{definition}[\(Eventually \varepsilon\)-close sequences]\label{5.2.3}
Let \((a_n)_{n = 0}^{\infty}\) and \((b_n)_{n = 0}^{\infty}\) be two sequences, and let \(\varepsilon > 0\).
We say that the sequence \((a_n)_{n = 0}^{\infty}\) is \emph{eventually \(\varepsilon\)-close} to \((b_n)_{n = 0}^{\infty}\) iff there exists an \(N \geq 0\) such that the sequences \((a_n)_{n = N}^{\infty}\) and \((b_n)_{n = N}^{\infty}\) are \(\varepsilon\)-close.
In other words, \(a_0, a_1, a_2, \dots\) is eventually \(\varepsilon\)-close to \(b_0, b_1, b_2, \dots\) iff there exists an \(N \geq 0\) such that \(|a_n - b_n| \leq \varepsilon\) for all \(n \geq N\).
\end{definition}

\begin{remark}\label{5.2.4}
Again, the notations for \(\varepsilon\)-close sequences and eventually \(\varepsilon\)-close sequences are not standard in the literature, and we will not use them outside of this section.
\end{remark}

\setcounter{theorem}{5}
\begin{definition}[Equivalent sequences]\label{5.2.6}
Two sequences \((a_n)_{n = 0}^{\infty}\) and \((b_n)_{n = 0}^{\infty}\) are \emph{equivalent} iff for each rational \(\varepsilon > 0\), the sequences \((a_n)_{n = 0}^{\infty}\) and \((b_n)_{n = 0}^{\infty}\) are eventually \(\varepsilon\)-close.
In other words, \(a_0, a_1, a_2, \dots\) and \(b_0, b_1, b_2, \dots\) are equivalent iff for every rational \(\varepsilon > 0\), there exists an \(N \geq 0\) such that \(|a_n - b_n| \leq \varepsilon\) for all \(n \geq N\).
\end{definition}

\begin{remark}\label{5.2.7}
As with Definition \ref{5.1.8}, the quantity \(\varepsilon > 0\) is currently restricted to be a positive rational, rather than a positive real.
However, we shall eventually see that it makes no difference whether \(\varepsilon\) ranges over the positive rationals or positive reals.
\end{remark}

\begin{proposition}\label{5.2.8}
Let \((a_n)_{n = 1}^{\infty}\) and \((b_n)_{n = 1}^{\infty}\) be the sequences \(a_n = 1 + 10^{-n}\) and \(b_n = 1 - 10^{-n}\).
Then the sequences \(a_n, b_n\) are equivalent.
\end{proposition}

\begin{proof}
We need to prove that for every \(\varepsilon > 0\), the two sequences \((a_n)_{n = 1}^{\infty}\) and \((b_n)_{n = 1}^{\infty}\) are eventually \(\varepsilon\)-close to each other.
So we fix an \(\varepsilon > 0\).
We need to find an \(N > 0\) such that \((a_n)_{n = 1}^{\infty}\) and \((b_n)_{n = 1}^{\infty}\) are \(\varepsilon\)-close;
in other words, we need to find an \(N > 0\) such that
\[
    |a_n - b_n| \leq \varepsilon \text{ for all } n \geq N.
\]
However, we have
\[
    |a_n - b_n| = |(1 + 10^{-n}) - (1 - 10^{-n})| = 2 \times 10^{-n}.
\]
Since \(10^{-n}\) is a decreasing function of \(n\) (i.e., \(10^{-m} < 10^{-n}\) whenever \(m > n\);
this is easily proven by induction), and \(n \geq N\), we have \(2 \times 10^{-n} \leq 2 \times 10^{-N}\).
Thus we have
\[
    |a_n - b_n| \leq 2 \times 10^{-N} \text{ for all } n \geq N.
\]
Thus in order to obtain \(|a_n - b_n| \leq \varepsilon\) for all \(n \geq N\), it will be sufficient to choose \(N\) so that \(2 \times 10^{-N} \leq \varepsilon\).
This is easy to do using logarithms, but we have not yet developed logarithms yet, so we will use a cruder method.
First, we observe \(10^N\) is always greater than \(N\) for any \(N \geq 1\) (see Exercise \ref{ex 4.3.5}).
Thus \(10^{-N} \leq 1 / N\), and so \(2 \times 10^{-N} \leq 2 / N\).
Thus to get \(2 \times 10^{-N} \leq \varepsilon\), it will suffice to choose \(N\) so that \(2 / N \leq \varepsilon\), or equivalently that \(N \geq 2 / \varepsilon\).
But by Proposition \ref{4.4.1} we can always choose such an \(N\), and the claim follows.
\end{proof}

\begin{remark}\label{5.2.9}
Proposition \ref{5.2.8}, in decimal notation, asserts that
\[
    1.0000 \dots = 0.9999 \dots.
\]
\end{remark}

\exercisesection

\begin{exercise}\label{ex 5.2.1}
Show that if \((a_n)_{n = 1}^{\infty}\) and \((b_n)_{n = 1}^{\infty}\) are equivalent sequences of rationals, then \((a_n)_{n = 1}^{\infty}\) is a Cauchy sequence if and only if \((b_n)_{n = 1}^{\infty}\) is a Cauchy sequence.
\end{exercise}

\begin{proof}
Let \((a_n)_{n = 1}^{\infty}\) be a Cauchy sequence.
By Definition \ref{5.1.8}, \(\forall\ \varepsilon > 0\) and \(\varepsilon \in \mathds{Q}\), \(\exists\ N_1 \geq 1\) and \(N_1 \in \mathds{N}\) such that
\[
    |a_j - a_k| \leq \varepsilon \ \forall\ j, k \geq N_1
\]
where \(j, k \in \mathds{N}\).
Since \((a_n)_{n = 1}^{\infty}\) and \((b_n)_{n = 1}^{\infty}\) are equivalent sequences, by Definition \ref{5.2.6}, \(\forall\ \varepsilon > 0\) and \(\varepsilon \in \mathds{Q}\), \(\exists\ N_2 \geq 1\) and \(N_2 \in \mathds{N}\) such that
\[
    |a_m - b_m| \leq \varepsilon \ \forall\ m \geq N_2
\]
where \(m \in \mathds{N}\).
Let \(N = N_1 + N_2\).
Since \(N > N_1\) and \(N > N_2\) by Proposition \ref{2.2.11}, we have
\[
    |a_j - a_k| \leq \varepsilon \ \forall\ j, k \geq N
\]
and
\[
    |a_m - b_m| \leq \varepsilon \ \forall\ m \geq N.
\]
Since \(\varepsilon > 0\), \(\varepsilon / 3 > 0\) by Additional Corollary \ref{ac 4.2.5}, then we have
\[
    |a_j - a_k| \leq \varepsilon / 3 \ \forall\ j, k \geq N
\]
and
\[
    |a_m - b_m| \leq \varepsilon / 3 \ \forall\ m \geq N.
\]
So \(\forall\ \varepsilon > 0\) and \(\forall\ j, k \geq N\),
\begin{align*}
|b_j - b_k| &= |b_j + (-b_k)| \\
&= |(b_j + (-b_k)) + 0| & \text{(by Proposition \ref{4.2.4})} \\
&= |(b_j + (-b_k)) + ((-a_k) + a_k)| & \text{(by Proposition \ref{4.2.4})} \\
&= |b_j + ((-b_k) + ((-a_k) + a_k))| & \text{(by Proposition \ref{4.2.4})} \\
&= |b_j + (((-a_k) + a_k) + (-b_k))| & \text{(by Proposition \ref{4.2.4})} \\
&= |b_j + ((-a_k) + (a_k + (-b_k)))| & \text{(by Proposition \ref{4.2.4})} \\
&= |(b_j + (-a_k)) + (a_k + (-b_k))| & \text{(by Proposition \ref{4.2.4})} \\
&= |0 + ((b_j + (-a_k)) + (a_k + (-b_k)))| & \text{(by Proposition \ref{4.2.4})} \\
&= |(a_j + (-a_j)) + ((b_j + (-a_k)) + (a_k + (-b_k)))| & \text{(by Proposition \ref{4.2.4})} \\
&= |((a_j + (-a_j)) + (b_j + (-a_k))) + (a_k + (-b_k))| & \text{(by Proposition \ref{4.2.4})} \\
&= |(a_j + (((-a_j) + b_j) + (-a_k))) + (a_k + (-b_k))| & \text{(by Proposition \ref{4.2.4})} \\
&= |(a_j + ((-a_k) + ((-a_j) + b_j))) + (a_k + (-b_k))| & \text{(by Proposition \ref{4.2.4})} \\
&= |((a_j + (-a_k)) + ((-a_j) + b_j)) + (a_k + (-b_k))| & \text{(by Proposition \ref{4.2.4})} \\
&\leq |(a_j + (-a_k)) + ((-a_j) + b_j)| + |a_k + (-b_k)| & \text{(by Proposition \ref{4.3.3})} \\
&\leq (|a_j + (-a_k)| + |(-a_j) + b_j|) + |a_k + (-b_k)| & \text{(by Proposition \ref{4.3.3})} \\
&= (|a_j + (-a_k)| + |a_j + (-b_j)|) + |a_k + (-b_k)| & \text{(by Proposition \ref{4.3.3})} \\
&= (|a_j - a_k| + |a_j - b_j|) + |a_k - b_k| \\
&\leq (\varepsilon / 3 + \varepsilon / 3) + \varepsilon / 3 & \text{(by Proposition \ref{4.3.3})} \\
&= \varepsilon.
\end{align*}
By Definition \ref{5.1.8}, \((b_n)_{n = 1}^{\infty}\) is also a Cauchy sequence.
Similar proof can show that \((b_n)_{n = 1}^{\infty}\) is a Cauchy sequence implies \((a_n)_{n = 1}^{\infty}\) is also a Cauchy sequence.
Thus we finished the proof.
\end{proof}

\begin{exercise}\label{ex 5.2.2}
Let \(\varepsilon > 0\).
Show that if \((a_n)_{n = 1}^{\infty}\) and \((b_n)_{n = 1}^{\infty}\) are eventually \(\varepsilon\)-close, then \((a_n)_{n = 1}^{\infty}\) is bounded if and only if \((b_n)_{n = 1}^{\infty}\) is bounded.
\end{exercise}
\chapter{Limits of sequences}

\section{Convergence and limit laws}\label{sec 6.1}

\begin{definition}[Distance between two real numbers]\label{6.1.1}
    Given two real numbers \(x\) and \(y\), we define their distance \(d(x, y)\) to be \(d(x, y) \coloneqq \abs*{x - y}\).
\end{definition}

\begin{note}
    Clearly Definition \ref{6.1.1} is consistent with Definition \ref{4.3.2}.
    Further, Proposition \ref{4.3.3} works just as well for real numbers as it does for rationals, because the real numbers obey all the rules of algebra that the rationals do.
\end{note}

\begin{definition}[\(\varepsilon\)-close real numbers]\label{6.1.2}
    Let \(\varepsilon > 0\) be a real number.
    We say that two real numbers \(x, y\) are \emph{\(\varepsilon\)-close} iff we have \(d(y, x) \leq \varepsilon\).
\end{definition}

\begin{note}
    Again, it is clear that Definition \ref{6.1.2} is consistent with Definition \ref{4.3.4}.
\end{note}

\begin{note}
    Now let \((a_n)_{n = m}^\infty\) be a sequence of \emph{real} numbers;
    i.e., we assign a real number \(a_n\) for every integer \(n \geq m\).
    The starting index \(m\) is some integer;
    usually this will be \(1\), but in some cases we will start from some index other than \(1\).
    (The choice of label used to index this sequence is unimportant; we could use for instance \((a_k)_{k = m}^{\infty}\) and this would represent exactly the same sequence as \((a_n)_{n = m}^{\infty}\).)
    We can define the notion of a Cauchy sequence in the same manner as before.
\end{note}

\begin{definition}[Cauchy sequences of reals]\label{6.1.3}
    Let \(\varepsilon > 0\) be a real number.
    A sequence \((a_n)_{n = N}^\infty\) of real numbers starting at some integer index \(N\) is said to be \emph{\(\varepsilon\)-steady} iff \(a_j\) and \(a_k\) are \(\varepsilon\)-close for every \(j, k \geq N\).
    A sequence \((a_n)_{n = m}^\infty\) starting at some integer index \(m\) is said to be \emph{eventually \(\varepsilon\)-steady} iff there exists an \(N \geq m\) such that \((a_n)_{n = N}^\infty\) is \(\varepsilon\)-steady.
    We say that \((a_n)_{n = m}^\infty\) is a \emph{Cauchy sequence} iff it is eventually \(\varepsilon\)-steady for every \(\varepsilon > 0\).
\end{definition}

\begin{note}
    To put it another way, a sequence \((a_n)_{n = m}^\infty\) of real numbers is a Cauchy sequence if, for every real \(\varepsilon > 0\), there exists an \(N \geq m\) such that \(\abs*{a_n - a_n'} \leq \varepsilon\) for all \(n, n' \geq N\).
    These definitions are consistent with the corresponding definitions for rational numbers (Definitions \ref{5.1.3}, \ref{5.1.6}, \ref{5.1.8}), although verifying consistency for Cauchy sequences takes a little bit of care.
\end{note}

\begin{proposition}\label{6.1.4}
    Let \((a_n)_{n = m}^\infty\) be a sequence of rational numbers starting at some integer index \(m\).
    Then \((a_n)_{n = m}^\infty\) is a Cauchy sequence in the sense of Definition \ref{5.1.8} if and only if it is a Cauchy sequence in the sense of Definition \ref{6.1.3}.
\end{proposition}

\begin{proof}
    Suppose first that \((a_n)_{n = m}^\infty\) is a Cauchy sequence in the sense of Definition \ref{6.1.3};
    then it is eventually \(\varepsilon\)-steady for every \emph{real} \(\varepsilon > 0\).
    In particular, it is eventually \(\varepsilon\)-steady for every \emph{rational} \(\varepsilon > 0\), which makes it a Cauchy sequence in the sense of Definition \ref{5.1.8}.

    Now suppose that \((a_n)_{n = m}^\infty\) is a Cauchy sequence in the sense of Definition \ref{5.1.8};
    then it is eventually \(\varepsilon'\)-steady for every \emph{rational} \(\varepsilon' > 0\).
    If \(\varepsilon > 0\) is a real number, then there exists a \emph{rational} \(\varepsilon' > 0\) which is smaller than \(\varepsilon\), by Proposition \ref{5.4.12}.
    Since \(\varepsilon'\) is rational, we know that \((a_n)_{n = m}^\infty\) is eventually \(\varepsilon'\)-steady;
    since \(\varepsilon' < \varepsilon\), this implies that \((a_n)_{n = m}^\infty\) is eventually \(\varepsilon\)-steady.
    Since \(\varepsilon\) is an arbitrary positive real number, we thus see that \((a_n)_{n = m}^\infty\) is a Cauchy sequence in the sense of Definition \ref{6.1.3}.
\end{proof}

\begin{note}
    Because of Proposition \ref{6.1.4}, we will no longer care about the distinction between Definition \ref{5.1.8} and Definition \ref{6.1.3}, and view the concept of a Cauchy sequence as a single unified concept.
\end{note}

\begin{definition}[Convergence of sequences]\label{6.1.5}
    Let \(\varepsilon > 0\) be a real number, and let \(L\) be a real number.
    A sequence \((a_n)_{n = N}^\infty\) of real numbers is said to be \emph{\(\varepsilon\)-close to \(L\)} iff \(a_n\) is \(\varepsilon\)-close to \(L\) for every \(n \geq N\), i.e., we have \(\abs*{a_n - L} \leq \varepsilon\) for every \(n \geq N\).
    We say that a sequence \((a_n)_{n = m}^\infty\) is \emph{eventually \(\varepsilon\)-close to \(L\)} iff there exists an \(N \geq m\) such that \((a_n)_{n = N}^\infty\) is \(\varepsilon\)-close to \(L\).
    We say that a sequence \((a_n)_{n = m}^\infty\) \emph{converges to \(L\)} iff it is eventually \(\varepsilon\)-close to \(L\) for every real \(\varepsilon > 0\).
\end{definition}

\setcounter{theorem}{6}
\begin{proposition}[Uniqueness of limits]\label{6.1.7}
    Let \((a_n)_{n = m}^\infty\) be a real sequence starting at some integer index \(m\), and let \(L \neq L'\) be two distinct real numbers.
    Then it is not possible for \((a_n)_{n = m}^\infty\) to converge to \(L\) while also converging to \(L'\).
\end{proposition}

\begin{proof}
    Suppose for sake of contradiction that \((a_n)_{n = m}^\infty\) was converging to both \(L\) and \(L'\).
    Let \(\varepsilon = \abs*{L - L'} / 3\);
    note that \(\varepsilon\) is positive since \(L \neq L'\).
    Since \((a_n)_{n = m}^\infty\) converges to \(L\), we know that \((a_n)_{n = m}^\infty\) is eventually \(\varepsilon\)-close to \(L\);
    thus there is an \(N \geq m\) such that \(d(a_n, L) \leq \varepsilon\) for all \(n \geq N\).
    Similarly, there is an \(M \geq m\) such that \(d(a_n, L') \leq \varepsilon\) for all \(n \geq M\).
    In particular, if we set \(n \coloneqq \max(N, M)\), then we have \(d(a_n, L) \leq \varepsilon\) and \(d(a_n, L') \leq \varepsilon\), hence by the triangle inequality \(d(L, L') \leq 2\varepsilon = 2\abs*{L - L'} / 3\).
    But then we have \(\abs*{L - L'} \leq 2\abs*{L - L'} / 3\), which contradicts the fact that \(\abs*{L - L'} > 0\).
    Thus it is not possible to converge to both \(L\) and \(L'\).
\end{proof}

\begin{definition}[Limits of sequences]\label{6.1.8}
    If a sequence \((a_n)_{n = m}^\infty\) converges to some real number \(L\), we say that \((a_n)_{n = m}^\infty\) is \emph{convergent} and that its \emph{limit} is \(L\);
    we write
    \[
        L = \lim_{n \to \infty} a_n
    \]
    to denote this fact.
    If a sequence \((a_n)_{n = m}^\infty\) is not converging to any real number \(L\), we say that the sequence \((a_n)_{n = m}^\infty\) is \emph{divergent} and we leave \(\lim_{n \to \infty} a_n\) undefined.
\end{definition}

\begin{note}
    Proposition \ref{6.1.7} ensures that a sequence can have at most one limit.
    Thus, if the limit exists, it is a single real number, otherwise it is undefined.
\end{note}

\begin{remark}\label{6.1.9}
    The notation \(\lim_{n \to \infty} a_n\) does not give any indication about the starting index \(m\) of the sequence, but the starting index is irrelevant.
    Thus in the rest of this discussion we shall not be too careful as to where these sequences start, as we shall be mostly focused on their limits.
\end{remark}

\begin{note}
    We sometimes use the phrase ``\(a_n \to x\) as \(n \to \infty\)'' as an alternate way of writing the statement ``\((a_n)_{n = m}^\infty\) converges to \(x\)''.
    Bear in mind, though, that the individual statements \(a_n \to x\) and \(n \to \infty\) do not have any rigorous meaning;
    this phrase is just a convention, though of course a very suggestive one.
\end{note}

\begin{remark}\label{6.1.10}
    The exact choice of letter used to denote the index (in this case \(n\)) is irrelevant:
    the phrase \(\lim_{n \to \infty} a_n\) has exactly the same meaning as \(\lim_{k \to \infty} a_k\), for instance.
    Sometimes it will be convenient to change the label of the index to avoid conflicts of notation;
    for instance, we might want to change \(n\) to \(k\) because \(n\) is simultaneously being used for some other purpose, and we want to reduce confusion.
\end{remark}

\begin{proposition}\label{6.1.11}
    We have \(\lim_{n \to \infty} 1 / n = 0\).
\end{proposition}

\begin{proof}
    We have to show that the sequence \((a_n)_{n = 1}^\infty\) converges to \(0\), where \(a_n \coloneqq 1 / n\).
    In other words, for every \(\varepsilon > 0\), we need to show that the sequence \((a_n)_{n = 1}^\infty\) is eventually \(\varepsilon\)-close to \(0\).
    So, let \(\varepsilon > 0\) be an arbitrary real number.
    We have to find an \(N\) such that \(\abs*{a_n - 0} \leq \varepsilon\) for every \(n \geq N\).
    But if \(n \geq N\), then
    \[
        \abs*{a_n - 0} = \abs*{1 / n - 0} = 1 / n \leq 1 / N.
    \]
    Thus, if we pick \(N > 1 / \varepsilon\) (which we can do by the Archimedean principle), then \(1 / N < \varepsilon\), and so \((a_n)_{n = 1}^\infty\) is \(\varepsilon\)-close to \(0\).
    Thus \((a_n)_{n = 1}^\infty\) is eventually \(\varepsilon\)-close to \(0\).
    Since \(\varepsilon\) was arbitrary, \((a_n)_{n = 1}^\infty\) converges to \(0\).
\end{proof}

\begin{proposition}[Convergent sequences are Cauchy]\label{6.1.12}
    Suppose that \((a_n)_{n = m}^\infty\) is a convergent sequence of real numbers.
    Then \((a_n)_{n = m}^\infty\) is also a Cauchy sequence.
\end{proposition}

\begin{proof}
    Let \((a_n)_{n = m}^\infty\) be a sequence of real numbers converges to \(L\).
    Then by Definition \ref{6.1.5} \(\forall\ \varepsilon \in \mathbf{R}^+\), \(\exists\ N \in \mathbf{N}\) and \(N \geq m\) such that \(\abs*{a_n - L} \leq \varepsilon\) for every \(n \geq N\).
    In particular, we have \(\abs*{a_n - L} \leq \varepsilon / 2\).
    Let \(n' \in \mathbf{N}\) and \(n' \geq N\).
    Then we have
    \begin{align*}
        \abs*{a_n - a_{n'}} & = \abs*{a_n - a_{n'} + L - L}           \\
                            & = \abs*{(a_n - L) + (L - a_{n'})}       \\
                            & \leq \abs*{a_n - L} + \abs*{L - a_{n'}} \\
                            & = \abs*{a_n - L} + \abs*{a_{n'} - L}    \\
                            & \leq \varepsilon / 2 + \varepsilon / 2  \\
                            & = \varepsilon.
    \end{align*}
    Since \(\varepsilon\) is arbitrary, by Definition \ref{6.1.3} we know that \((a_n)_{n = m}^\infty\) is a Cauchy sequence.
\end{proof}

\setcounter{theorem}{14}
\begin{proposition}[Formal limits are genuine limits]\label{6.1.15}
    Suppose that \((a_n)_{n = 1}^\infty\) is a Cauchy sequence of rational numbers.
    Then \((a_n)_{n = 1}^\infty\) converges to \(\text{LIM}_{n \to \infty} a_n\), i.e.
    \[
        \text{LIM}_{n \to \infty} a_n = \lim_{n \to \infty} a_n.
    \]
\end{proposition}

\begin{proof}
    Let \((a_n)_{n = m}^\infty\) be a Cauchy sequence of rationals, and let \(L = \text{LIM}_{n \to \infty} a_n\).
    By Definition \ref{5.3.1} we know that \(L \in \mathbf{R}\).
    Thus by Proposition \ref{6.1.4} and Definition \ref{6.1.5} we can ask whether \((a_n)_{n = m}^\infty\) converges to \(L\).
    Suppose for sake of contradiction that sequence \(a_n\) is not eventually \(\varepsilon\)-close to \(L\) for every \(\varepsilon \in \mathbf{R}^+\).
    Then \(\exists\ \varepsilon \in \mathbf{R}^+\) such that \(\forall\ n \in \mathbf{N}\) and \(n \geq m\), we have \(\abs*{a_n - L} > \varepsilon\).
    Since \(\varepsilon > 0\), we know that \(\abs*{a_n - L} > 0\), now we split into two cases:
    \begin{itemize}
        \item If \(a_n - L > 0\), then we have
              \begin{align*}
                           & \forall\ n \geq m, \abs*{a_n - L} > \varepsilon                                       \\
                  \implies & a_n - L > \varepsilon                                                                 \\
                  \implies & a_n > L + \varepsilon                                                                 \\
                  \implies & \text{LIM}_{n \to \infty} a_n > L + \varepsilon & \text{(by Exercise \ref{ex 5.4.8})} \\
                  \implies & L > L + \varepsilon                                                                   \\
                  \implies & 0 > \varepsilon.
              \end{align*}
              But this contradict to \(\varepsilon > 0\).
        \item If \(a_n - L < 0\), then we have
              \begin{align*}
                           & \forall\ n \geq m, \abs*{a_n - L} > \varepsilon                                       \\
                  \implies & L - a_n > \varepsilon                                                                 \\
                  \implies & a_n < L - \varepsilon                                                                 \\
                  \implies & \text{LIM}_{n \to \infty} a_n < L - \varepsilon & \text{(by Exercise \ref{ex 5.4.8})} \\
                  \implies & L < L - \varepsilon                                                                   \\
                  \implies & \varepsilon < 0.
              \end{align*}
              But this contradict to \(\varepsilon > 0\).
    \end{itemize}
    From all cases above we derived contradictions.
    Thus such \(\varepsilon\) does not exist, and therefor we must have \((a_n)_{n = m}^\infty\) eventually \(\varepsilon\)-close to \(L\) for every \(\varepsilon \in \mathbf{R}^+\).
    By Definition \ref{6.1.5} this means \(\lim_{n \to \infty} a_n = L\).
\end{proof}

\begin{definition}[Bounded sequences]\label{6.1.16}
    A sequence \((a_n)_{n = m}^\infty\) of reals is \emph{bounded by} a real number \(M\) iff we have \(\abs*{a_n} \leq M\) for all \(n \geq m\).
    We say that \((a_n)_{n = m}^\infty\) is bounded iff it is \emph{bounded} by \(M\) for some real number \(M > 0\).
\end{definition}

\begin{note}
    Definition \ref{6.1.16} is consistent with Definition \ref{5.1.12}.
\end{note}

\begin{note}
    Recall from Lemma \ref{5.1.15} that every Cauchy sequence of rational numbers is bounded.
    An inspection of the proof of that Lemma shows that the same argument works for real numbers;
    every Cauchy sequence of real numbers is bounded.
\end{note}

\begin{corollary}\label{6.1.17}
    Every convergent sequence of real numbers is bounded.
\end{corollary}

\begin{proof}
    From Proposition \ref{6.1.12} we have every convergent sequence of real numbers is a Cauchy sequence.
    And by Lemma \ref{5.1.15} every Cauchy sequence is bounded.
    Thus every convergent sequence of real numbers is bounded.
\end{proof}

\setcounter{theorem}{18}
\begin{theorem}[Limit Laws]\label{6.1.19}
    Let \((a_n)_{n = m}^\infty\) and \((b_n)_{n = m}^\infty\) be convergent sequences of real numbers, and let \(x, y\) be the real numbers \(x \coloneqq \lim_{n \to \infty} a_n\) and \(y \coloneqq \lim_{n \to \infty} b_n\).
    \begin{enumerate}
        \item The sequence \((a_n + b_n)_{n = m}^\infty\) converges to \(x + y\);
              in other words,
              \[
                  \lim_{n \to \infty} (a_n + b_n) = \lim_{n \to \infty} a_n + \lim_{n \to \infty} b_n.
              \]
        \item The sequence \((a_n b_n)_{n = m}^\infty\) converges to \(xy\);
              in other words,
              \[
                  \lim_{n \to \infty} (a_n b_n) = (\lim_{n \to \infty} a_n)(\lim_{n \to \infty} b_n).
              \]
        \item For any real number \(c\), the sequence \((c a_n)_{n = m}^\infty\) converges to \(cx\);
              in other words,
              \[
                  \lim_{n \to \infty} (c a_n) = c(\lim_{n \to \infty} a_n).
              \]
        \item The sequence \((a_n - b_n)_{n = m}^\infty\) converges to \(x - y\);
              in other words,
              \[
                  \lim_{n \to \infty} (a_n - b_n) = \lim_{n \to \infty} a_n - \lim_{n \to \infty} b_n.
              \]
        \item Suppose that \(y \neq 0\), and that \(b_n \neq 0\) for all \(n \geq m\).
              Then the sequence \((b_n^{-1})_{n = m}^\infty\) converges to \(y^{-1}\);
              in other words,
              \[
                  \lim_{n \to \infty} b_n^{-1} = (\lim_{n \to \infty} b_n)^{-1}.
              \]
        \item Suppose that \(y \neq 0\), and that \(b_n \neq 0\) for all \(n \geq m\).
              Then the sequence \((a_n / b_n)_{n = m}^\infty\) converges to \(x / y\);
              in other words,
              \[
                  \lim_{n \to \infty} \frac{a_n}{b_n} = \frac{\lim_{n \to \infty} a_n}{\lim_{n \to \infty} b_n}.
              \]
        \item The sequence \((\max(a_n, b_n))_{n = m}^\infty\) converges to \(\max(x, y)\);
              in other words,
              \[
                  \lim_{n \to \infty} \max(a_n, b_n) = \max(\lim_{n \to \infty} a_n, \lim_{n \to \infty} b_n).
              \]
        \item The sequence \((\min(a_n, b_n))_{n = m}^\infty\) converges to \(\min(x, y)\);
              in other words,
              \[
                  \lim_{n \to \infty} \min(a_n, b_n) = \min(\lim_{n \to \infty} a_n, \lim_{n \to \infty} b_n).
              \]
    \end{enumerate}
\end{theorem}

\begin{proof}{(a)}
    By Definition \ref{6.1.8} \(\forall\ \varepsilon \in \mathbf{R}^+\), \(\exists\ N_a \in \mathbf{N}\) such that \(\abs*{a_n - x} \leq \varepsilon / 2\) for every \(n \geq N_a\).
    Similarly \(\exists\ N_b \in \mathbf{N}\) such that \(\abs*{b_n - y} \leq \varepsilon / 2\) for every \(n \geq N_b\).
    Let \(N = \max(N_a, N_b)\).
    Then we have \(\forall\ n \geq N\),
    \begin{align*}
        \abs*{a_n + b_n - (x + y)} & = \abs*{(a_n - x) + (b_n - y)}         \\
                                   & \leq \abs*{a_n - x} + \abs*{b_n - y}   \\
                                   & \leq \varepsilon / 2 + \varepsilon / 2 \\
                                   & = \varepsilon.
    \end{align*}
    Thus by Definition \ref{6.1.5} \((a_n + b_n)_{n = m}^\infty\) converges to \(x + y\).
    And by Definition \ref{6.1.8} we have \(\lim_{n \to \infty} (a_n + b_n) = x + y = \lim_{n \to \infty} a_n + \lim_{n \to \infty} b_n\).
\end{proof}

\begin{proof}{(b)}
    By Corollary \ref{6.1.17}, \(\exists\ A, B \in \mathbf{R}^+\) such that \(\abs*{a_n} \leq A\) and \(\abs*{b_n} \leq B\) for every \(n \geq m\).
    By Definition \ref{6.1.8} \(\forall\ \varepsilon \in \mathbf{R}^+\), \(\exists\ N_a \in \mathbf{N}\) such that \(\abs*{a_n - x} \leq \varepsilon / 2B\) for every \(n \geq N_a\).
    Similarly \(\exists\ N_b \in \mathbf{N}\) such that \(\abs*{b_n - y} \leq \varepsilon / 2A\) for every \(n \geq N_b\).
    Let \(N = \max(N_a, N_b)\).
    Then we have
    \begin{align*}
        \abs*{a_n b_n - x y} & = \abs*{a_n b_n - x y + x b_n - x b_n}                                 \\
                             & = \abs*{a_n b_n - x b_n + x b_n - x y}                                 \\
                             & = \abs*{b_n(a_n - x) + x(b_n - y)}                                     \\
                             & \leq \abs*{b_n(a_n - x)} + \abs*{x(b_n - y)}                           \\
                             & = \abs*{b_n}\abs*{a_n - x} + \abs*{x}\abs*{b_n - y}                    \\
                             & \leq B \times \frac{\varepsilon}{2B} + A \times \frac{\varepsilon}{2A} \\
                             & = \varepsilon.
    \end{align*}
    Thus by Definition \ref{6.1.5} \((a_n b_n)_{n = m}^\infty\) converges to \(x y\).
    And by Definition \ref{6.1.8} we have \(\lim_{n \to \infty} (a_n b_n) = x y = (\lim_{n \to \infty} a_n)(\lim_{n \to \infty} b_n)\).
\end{proof}

\begin{proof}{(c)}
    Let \((b_n)_{n = m}^\infty\) be a sequence where \(b_n = c \) for every \(n \geq m\).
    Then we have \(\lim_{n \to \infty} c = c\) and
    \begin{align*}
        \lim_{n \to \infty} (c a_n) & = \lim_{n \to \infty} (b_n a_n)                                                            \\
                                    & = (\lim_{n \to \infty} b_n)(\lim_{n \to \infty} a_n) & \text{(by Theorem \ref{6.1.19}(b))} \\
                                    & = c(\lim_{n \to \infty} a_n).
    \end{align*}
\end{proof}

\begin{proof}{(d)}
    We have
    \begin{align*}
        \lim_{n \to \infty} (a_n - b_n) & = \lim_{n \to \infty} (a_n + (-1)(b_n))                     & \text{(by Proposition \ref{5.3.11})} \\
                                        & = \lim_{n \to \infty} a_n + \lim_{n \to \infty} ((-1)(b_n)) & \text{(by Theorem \ref{6.1.19}(a))}  \\
                                        & = \lim_{n \to \infty} a_n + (-1)(\lim_{n \to \infty} b_n)   & \text{(by Theorem \ref{6.1.19}(c))}  \\
                                        & = \lim_{n \to \infty} a_n - \lim_{n \to \infty} b_n.        & \text{(by Proposition \ref{5.3.11})}
    \end{align*}
\end{proof}

\begin{proof}{(e)}
    We first show that \((b_n)_{n = m}^\infty\) is bounded away from zero.
    Since \(\lim_{n \to \infty} b_n \neq 0\), we must have some \(M \in \mathbf{R}^+\) such that \(\abs*{b_n - 0} > M\) for every \(n \in \mathbf{N}\).
    Otherwise we would have \(\lim_{n \to \infty} b_n = 0\), which is a contradiction by Proposition \ref{6.1.7}.
    Since \(\abs*{b_n - 0} = \abs*{b_n} > M > 0\) for every \(n \geq m\), we know that \((b_n)_{n = m}^\infty\) is bounded away from zero.

    Now we show that \(\lim_{n \to \infty} b_n^{-1} = (\lim_{n \to \infty} b_n)^{-1}\).
    By Definition \ref{6.1.8} \(\forall\ \varepsilon \in \mathbf{R}^+\), \(\exists\ N \in \mathbf{N}\) such that \(\abs*{b_n - y} \leq \varepsilon M \abs*{y}\) for every \(n \geq N\).
    (\(M\) is derived from the claim above).
    So
    \begin{align*}
        \abs*{b_n^{-1} - y^{-1}} & = \abs*{\frac{1}{b_n} - \frac{1}{y}}                                          \\
                                 & = \abs*{\frac{y - b_n}{b_n y}}                                                \\
                                 & = \abs*{y - b_n}\frac{1}{\abs*{b_n}\abs*{y}}                                  \\
                                 & < \abs*{y - b_n}\frac{1}{M\abs*{y}}           & \text{(From the claim above)} \\
                                 & \leq \varepsilon M\abs*{y}\frac{1}{M\abs*{y}}                                 \\
                                 & = \varepsilon.
    \end{align*}
    Thus by Definition \ref{6.1.5} \((b_n^{-1})_{n = m}^\infty\) converges to \(y^{-1}\).
    And by Definition \ref{6.1.8} we have \(\lim_{n \to \infty} (b_n^{-1}) = y^{-1} = (\lim_{n \to \infty} b_n)^{-1}\).
\end{proof}

\begin{proof}{(f)}
    We have
    \begin{align*}
        \lim_{n \to \infty} \frac{a_n}{b_n} & = \lim_{n \to \infty} a_n b_n^{-1}                                                               \\
                                            & = (\lim_{n \to \infty} a_n)(\lim_{n \to \infty} b_n^{-1})  & \text{(by Theorem \ref{6.1.19}(b))} \\
                                            & = (\lim_{n \to \infty} a_n)(\lim_{n \to \infty} b_n)^{-1}  & \text{(by Theorem \ref{6.1.19}(e))} \\
                                            & = \frac{\lim_{n \to \infty} a_n}{\lim_{n \to \infty} b_n}.
    \end{align*}
\end{proof}

\begin{proof}{(g)}
    By Definition \ref{6.1.8} \(\forall\ \varepsilon \in \mathbf{R}^+\), \(\exists\ N_a \in \mathbf{N}\) such that \(\abs*{a_n - x} \leq \varepsilon\) for every \(n \geq N_a\).
    Similarly \(\exists\ N_b \in \mathbf{N}\) such that \(\abs*{b_n - y} \leq \varepsilon\) for every \(n \geq N_b\).
    Let \(N = \max(N_a, N_b)\).
    Then we have \(\abs*{a_n - x} \leq \varepsilon \land \abs*{b_n - y} \leq \varepsilon\) for every \(n \geq N\).
    Now we split into two cases:
    \begin{itemize}
        \item If \(x = y\), then we have \(\max(x, y) = x\) and for every \(n \geq N\),
              \begin{align*}
                           & (\abs*{a_n - x} < \varepsilon) \land (\abs*{b_n - x} < \varepsilon) \\
                  \implies & \abs*{\max(a_n, b_n) - x} < \varepsilon                             \\
                  \implies & \abs*{\max(a_n, b_n) - \max(x, y)} < \varepsilon                    \\
                  \implies & \lim_{n \to \infty} \max(a_n, b_n) = \max(x, y) = x.
              \end{align*}
        \item If \(x \neq y\), then we have either \(x < y\) or \(x > y\).
              Without loss of generality suppose that \(x < y\).
              Since \(x < y\), we have \(y - x > 0\).
              Since we have \(\abs*{a_n - x} \leq \varepsilon\) and \(\abs*{b_n - y} \leq \varepsilon\) for every positive real number \(\varepsilon\), we also have \(\abs*{a_n - x} \leq (y - x) / 2\) and \(\abs*{b_n - y} \leq (y - x) / 2\).
              So \(\forall\ n \geq N\), we have
              \begin{align*}
                           & (\abs*{a_n - x} \leq \frac{y - x}{2}) \land (\abs*{b_n - y} \leq \frac{y - x}{2})                               \\
                  \implies & (-\frac{y - x}{2} \leq a_n - x \leq \frac{y - x}{2}) \land (-\frac{y - x}{2} \leq b_n - y \leq \frac{y - x}{2}) \\
                  \implies & (a_n - x \leq \frac{y - x}{2}) \land (-\frac{y - x}{2} \leq b_n - y)                                            \\
                  \implies & (a_n \leq \frac{y - x}{2} + x) \land (y - \frac{y - x}{2} \leq b_n)                                             \\
                  \implies & (a_n \leq \frac{x + y}{2}) \land (\frac{x + y}{2} \leq b_n)                                                     \\
                  \implies & a_n \leq \frac{x + y}{2} \leq b_n.
              \end{align*}
              This means \(\forall\ n \geq N, \max(a_n, b_n) = b_n\).
              Thus \(\forall\ n \geq N\), we have
              \begin{align*}
                           & \abs*{\max(a_n, b_n) - \max(x, y)} = \abs*{b_n - y} \leq \varepsilon                                      \\
                  \implies & \lim_{n \to \infty} \max(a_n, b_n) = \max(x, y) = \max(\lim_{n \to \infty} a_n, \lim_{n \to \infty} b_n).
              \end{align*}
    \end{itemize}
\end{proof}

\begin{proof}{(h)}
    We have \(\min(x, y) = -\max(-x, -y)\) and
    \begin{align*}
        \lim_{n \to \infty} \min(a_n, b_n) & = \lim_{n \to \infty} -\max(-a_n, -b_n)                                       \\
                                           & = -\lim_{n \to \infty} \max(-a_n, -b_n) & \text{(by Theorem \ref{6.1.19}(c))} \\
                                           & = -\max(-x, -y)                         & \text{(by Theorem \ref{6.1.19}(g))} \\
                                           & = \min(x, y).
    \end{align*}
\end{proof}

\exercisesection

\begin{exercise}\label{ex 6.1.1}
    Let \((a_n)_{n = m}^\infty\) be a sequence of real numbers, such that \(a_{n + 1} > a_n\) for each natural number \(n\).
    Prove that whenever \(n\) and \(m\) are natural numbers such that \(m > n\), then we have \(a_m > a_n\).
    (We refer to these sequences as \emph{increasing} sequences.)
\end{exercise}

\begin{proof}
    Let \(E = \{z \in \mathbf{N} : n \leq z \leq m\}\).
    Then \(E\) is finite (since \(\#(E) = m - n + 1\)) and non-empty (since \(n, m \in E\)).
    Let \((a_z)_{z = n}^m\) be a sequence by mapping \(z \in E\) to \(a_z\).
    So \((a_z)_{z = n}^m\) is a finite sequence, and the elements in sequence \((a_z)_{z = n}^m\) are \(\{a_n, a_{n + 1}, \dots, a_{m - 1}, a_m\}\).
    By the given conditions we have \(a_{z + 1} > a_z\) for each natural number \(z\).
    Thus we have \(a_n < a_{n + 1} < \dots < a_{m - 1} < a_m\), and by Proposition \ref{5.4.7} we have \(a_n < a_m\).
\end{proof}

\begin{exercise}\label{ex 6.1.2}
    Let \((a_n)_{n = m}^\infty\) be a sequence of real numbers, and let \(L\) be a real number.
    Show that \((a_n)_{n = m}^\infty\) converges to \(L\) if and only if, given any real \(\varepsilon > 0\), one can find an \(N \geq m\) such that \(\abs*{a_n - L} \leq \varepsilon\) for all \(n \geq N\).
\end{exercise}

\begin{proof}
    \begin{align*}
             & (a_n)_{n = m}^\infty \text{ converges to } L                                                                                                           \\
        \iff & \forall\ \varepsilon \in \mathbf{R}^+, (a_n)_{n = m}^\infty \text{ is eventually } \varepsilon\text{-close to } L & \text{(by Definition \ref{6.1.5})} \\
        \iff & \forall\ \varepsilon \in \mathbf{R}^+, \exists\ N \in \mathbf{N} \land N \geq m :                                                                      \\
             & (a_n)_{n = N}^\infty \text{ is } \varepsilon\text{-close to } L                                                   & \text{(by Definition \ref{6.1.5})} \\
        \iff & \forall\ \varepsilon \in \mathbf{R}^+, \exists\ N \in \mathbf{N} \land N \geq m :                                                                      \\
             & \forall\ n \geq N, \abs*{a_n - L} \leq \varepsilon.                                                               & \text{(by Definition \ref{6.1.5})}
    \end{align*}
\end{proof}

\begin{exercise}\label{ex 6.1.3}
    Let \((a_n)_{n = m}^\infty\) be a sequence of real numbers, let \(c\) be a real number, and let \(m' \geq m\) be an integer.
    Show that \((a_n)_{n = m}^\infty\) converges to \(c\) if and only if \((a_n)_{n = m'}^\infty\) converges to \(c\).
\end{exercise}

\begin{proof}
    If \((a_n)_{n = m'}^\infty\) converges to \(c\) for all \(m' \in \mathbf{N}\) and \(m' \geq m\), then obviously \((a_n)_{n = m}^\infty\) converges to \(c\).
    So we only need to show that if \((a_n)_{n = m}^\infty\) converges to \(c\), then \((a_n)_{n = m'}^\infty\) converges to \(c\) for all \(m' \in \mathbf{N}\) and \(m' \geq m\).
    Let \(N \in \mathbf{N}\).
    Then we have
    \begin{align*}
                 & (a_n)_{n = m}^\infty \text{ converges to } c                                                                                                           \\
        \implies & \forall\ \varepsilon \in \mathbf{R}^+, (a_n)_{n = m}^\infty \text{ is eventually } \varepsilon\text{-close to } c & \text{(by Definition \ref{6.1.5})} \\
        \implies & \forall\ \varepsilon \in \mathbf{R}^+, \exists\ N \geq m :                                                                                             \\
                 & (a_n)_{n = N}^\infty \text{ is } \varepsilon\text{-close to } c                                                   & \text{(by Definition \ref{6.1.5})} \\
        \implies & \forall\ \varepsilon \in \mathbf{R}^+, \exists\ N \geq m' \geq m :                                                                                     \\
                 & (a_n)_{n = N}^\infty \text{ is } \varepsilon\text{-close to } c                                                                                        \\
        \implies & (a_n)_{n = m}^\infty \text{ converges to } c \text{ for all } m' \geq m
    \end{align*}
    Thus if \((a_n)_{n = m}^\infty\) converges to \(c\), then \((a_n)_{n = m'}^\infty\) converges to \(c\) for all \(m' \geq m\).
\end{proof}

\begin{exercise}\label{ex 6.1.4}
    Let \((a_n)_{n = m}^\infty\) be a sequence of real numbers, let \(c\) be a real number, and let \(k \geq 0\) be a non-negative integer.
    Show that \((a_n)_{n = m}^\infty\) converges to \(c\) if and only if \((a_{n + k})_{n = m}^\infty\) converges to \(c\).
\end{exercise}

\begin{proof}
    Since \((a_{n + k})_{n = m}^\infty = (a_n)_{n = m + k}^\infty\) and \(m + k \geq m\), by Exercise \ref{ex 6.1.3} we have \((a_n)_{n = m}^\infty\) converges to \(c\) if and only if \((a_n)_{n = m + k}^\infty\) converges to \(c\).
    Thus \((a_n)_{n = m}^\infty\) converges to \(c\) if and only if \((a_{n + k})_{n = m}^\infty\) converges to \(c\).
\end{proof}

\begin{exercise}\label{ex 6.1.5}
    Prove Proposition \ref{6.1.12}.
\end{exercise}

\begin{proof}
    See Proposition \ref{6.1.12}.
\end{proof}

\begin{exercise}\label{ex 6.1.6}
    Prove Proposition \ref{6.1.15}.
\end{exercise}

\begin{proof}
    See Proposition \ref{6.1.15}.
\end{proof}

\begin{exercise}\label{ex 6.1.7}
    Show that Definition \ref{6.1.16} is consistent with Definition \ref{5.1.12}
    (i.e., prove an analogue of Proposition \ref{6.1.4} for bounded sequences instead of Cauchy sequences).
\end{exercise}

\begin{proof}
    First suppose that \((a_n)_{n = m}^\infty\) is a bounded sequence in the sense of Definition \ref{6.1.16};
    then \(\exists\ M \in \mathbf{R}^+\) such that \(\abs*{a_n} \leq M\) for every \(n \geq m\).
    By Proposition \ref{5.4.12}, \(\exists\ N \in \mathbf{N}\) such that \(M \leq N\).
    Since \(N \in \mathbf{N}\), we also have \(N \in \mathbf{Q}\).
    Thus \(\abs*{a_n} \leq N\) for every \(n \geq m\), and \((a_n)_{n = m}^\infty\) is a bounded sequence in the sense of Definition \ref{5.1.12}.

    Now suppose that \((a_n)_{n = m}^\infty\) is a bounded sequence in the sense of Definition \ref{5.1.12};
    then \(\exists\ M \in \mathbf{Q}^+\) such that \(\abs*{a_n} \leq M\) for every \(n \geq m\).
    Since \(M\) is also a real number, we see that \((a_n)_{n = m}^\infty\) is a bounded sequence in the sense of Definition \ref{6.1.16}.
\end{proof}

\begin{exercise}\label{ex 6.1.8}
    Proof Theorem \ref{6.1.19}.
\end{exercise}

\begin{proof}
    See Theorem \ref{6.1.19}.
\end{proof}

\begin{exercise}\label{ex 6.1.9}
    Explain why Theorem \ref{6.1.19}(f) fails when the limit of the denominator is \(0\).
\end{exercise}

\begin{proof}
    Suppose for sake of contradiction that Theorem \ref{6.1.19}(f) works when denominator is \(0\).
    Let \((a_n)_{n = 1}^\infty = (b_n)_{n = 1}^\infty = 1 / n\).
    Then we have
    \[
        \lim_{n \to \infty} a_n / b_n = \lim_{n \to \infty} \frac{1 / n}{1 / n} = \lim_{n \to \infty} 1 = 1.
    \]
    But by Proposition \ref{6.1.11} we also have
    \[
        \frac{\lim_{n \to \infty} a_n}{\lim_{n \to \infty} b_n} = \frac{0}{0}
    \]
    which is undefined.
    Thus Theorem \ref{6.1.19}(f) fails when denominator is \(0\).
\end{proof}

\begin{exercise}\label{ex 6.1.10}
    Show that the concept of equivalent Cauchy sequence, as defined in Definition \ref{5.2.6}, does not change if \(\varepsilon\) is required to be positive real instead of positive rational.
    More precisely, if \((a_n)_{n = 0}^\infty\) and \((b_n)_{n = 0}^\infty\) are sequences of reals, show that \((a_n)_{n = 0}^\infty\) and \((b_n)_{n = 0}^\infty\) are eventually \(\varepsilon\)-close for every rational \(\varepsilon > 0\) if and only if they are eventually \(\varepsilon\)-close for every real \(\varepsilon > 0\).
\end{exercise}

\begin{proof}
    Suppose first that \((a_n)_{n = 0}^\infty\) and \((b_n)_{n = 0}^\infty\) are eventually \(\varepsilon\)-close \(\forall\ \varepsilon \in \mathbf{Q}^+\).
    Let \(\varepsilon' \in \mathbf{R}^+\).
    By Proposition \ref{5.4.12} \(\exists\ \varepsilon \in \mathbf{Q}^+\) such that \(\varepsilon \leq \varepsilon'\).
    Since \(\varepsilon \in \mathbf{Q}^+\), we know that \((a_n)_{n = 0}^\infty\) and \((b_n)_{n = 0}^\infty\) are eventually \(\varepsilon\)-close.
    Thus \((a_n)_{n = 0}^\infty\) and \((b_n)_{n = 0}^\infty\) are eventually \(\varepsilon'\)-close.
    Since \(\varepsilon'\) is arbitrary, we have \((a_n)_{n = 0}^\infty\) and \((b_n)_{n = 0}^\infty\) are eventually \(\varepsilon'\)-close \(\forall\ \varepsilon' \in \mathbf{R}^+\).

    Now suppose that \((a_n)_{n = 0}^\infty\) and \((b_n)_{n = 0}^\infty\) are eventually \(\varepsilon'\)-close \(\forall\ \varepsilon' \in \mathbf{R}^+\).
    This implies that \((a_n)_{n = 0}^\infty\) and \((b_n)_{n = 0}^\infty\) are eventually \(\varepsilon\)-close for \(\forall\ \varepsilon \in \mathbf{Q}^+\).
    Thus we conclude that \((a_n)_{n = 0}^\infty\) and \((b_n)_{n = 0}^\infty\) are eventually \(\varepsilon\)-close \(\forall\ \varepsilon \in \mathbf{Q}^+\) if and only if they are eventually \(\varepsilon'\)-close \(\forall\ \varepsilon' \in \mathbf{R}^+\).
\end{proof}
\section{The Extended real number system}\label{sec 6.2}

\begin{definition}[Extended real number system]\label{6.2.1}
The \emph{extended real number system \(\mathbf{R}^*\)} is the real line \(\mathbf{R}\) with two additional elements attached, called \(+\infty\) and \(-\infty\).
These elements are distinct from each other and also distinct from every real number.
An extended real number \(x\) is called \emph{finite} iff it is a real number, and \emph{infinite} iff it is equal to \(+\infty\) or \(-\infty\).
(This definition is not directly related to the notion of finite and infinite sets in Section \ref{sec 3.6}, though it is of course similar in spirit.)
\end{definition}

\begin{definition}[Negation of extended reals]\label{6.2.2}
The operation of negation \(x \to -x\) on \(\mathbf{R}\), we now extend to \(\mathbf{R}^*\) by defining \(-(+\infty) \coloneqq -\infty\) and \(-(-\infty) \coloneqq +\infty\).
\end{definition}

\begin{note}
Thus every extended real number \(x\) has a negation, and \(-(-x)\) is always equal to \(x\).
\end{note}

\begin{definition}[Ordering of extended reals]\label{6.2.3}
Let \(x\) and \(y\) be extended real numbers.
We say that \(x \leq y\), i.e., \(x\) is less than or equal to \(y\), iff one of the following three statements is true:
\begin{enumerate}
    \item \(x\) and \(y\) are real numbers, and \(x \leq y\) as real numbers.
    \item \(y = +\infty\).
    \item \(x = -\infty\).
\end{enumerate}
We say that \(x < y\) if we have \(x \leq y\) and \(x \neq y\).
We sometimes write \(x < y\) as \(y > x\), and \(x \leq y\) as \(y \geq x\).
\end{definition}

\setcounter{theorem}{4}
\begin{proposition}\label{6.2.5}
Let \(x, y, z\) be extended real numbers.
Then the following statements are true:
\begin{enumerate}
    \item (Reflexivity)
    We have \(x \leq x\).
    \item (Trichotomy)
    Exactly one of the statements \(x < y\), \(x = y\), or \(x > y\) is true.
    \item (Transitivity)
    If \(x \leq y\) and \(y \leq z\), then \(x \leq z\).
    \item (Negation reverses order) If \(x \leq y\), then \(-y \leq -x\).
\end{enumerate}
\end{proposition}

\begin{proof}{(a)}
By Proposition \ref{5.4.7}, we already have \(x \leq x\) when \(x \in \mathbf{R}\).
So we only need to consider the cases \(x = +\infty\) or \(x = -\infty\).
By Definition \ref{6.2.3}, we have \(x \leq +\infty \ \forall\ x \in \mathbf{R}^*\).
So we have \(+\infty \leq +\infty\).
Again by Definition \ref{6.2.3}, we have \(-\infty \leq x \ \forall\ x \in \mathbf{R}^*\).
So we have \(-\infty \leq -\infty\).
Thus we conclude that \(x \leq x \ \forall\ x \in \mathbf{R}^*\).
\end{proof}

\begin{proof}{(b)}
By Proposition \ref{5.4.7}, we already have exactly one of the statements \(x < y\), \(x = y\), or \(x > y\) is true when \(x, y \in \mathbf{R}\).
So we only need to consider the cases \(x = +\infty\), \(x = -\infty\), \(y = +\infty\), or \(y = -\infty\).
\begin{enumerate}[label=(\Roman*)]
    \item If \(x = +\infty\), then by Definition \ref{6.2.3} we have \(x \geq y \ \forall\ y \in \mathbf{R}^*\).
    \begin{enumerate}[label=(\roman*)]
        \item If \(y = +\infty\), then we have \(x = y\).
        \item If \(y \in \mathbf{R}\), then by Definition \ref{6.2.1} \(x \neq y\).
        Thus by Definition \ref{6.2.3} we have \(x > y\).
        \item If \(y = -\infty\), then by Definition \ref{6.2.1} \(x \neq y\).
        Thus by Definition \ref{6.2.3} we have \(x > y\).
    \end{enumerate}
    \item If \(x = -\infty\), then by Definition \ref{6.2.3} we have \(x \leq y \ \forall\ y \in \mathbf{R}^*\).
    \begin{enumerate}[label=(\roman*)]
        \item If \(y = +\infty\), then by Definition \ref{6.2.1} \(x \neq y\).
        Thus by Definition \ref{6.2.3} we have \(x < y\).
        \item If \(y \in \mathbf{R}\), then by Definition \ref{6.2.1} \(x \neq y\).
        Thus by Definition \ref{6.2.3} we have \(x < y\).
        \item If \(y = -\infty\), then we have \(x = y\).
    \end{enumerate}
    \item If \(y = +\infty\), then by Definition \ref{6.2.3} we have \(x \leq y \ \forall\ x \in \mathbf{R}^*\).
    \begin{enumerate}[label=(\roman*)]
        \item If \(x = +\infty\), then we have \(x = y\).
        \item If \(x \in \mathbf{R}\), then by Definition \ref{6.2.1} \(x \neq y\).
        Thus by Definition \ref{6.2.3} we have \(x < y\).
        \item If \(x = -\infty\), then by Definition \ref{6.2.1} \(x \neq y\).
        Thus by Definition \ref{6.2.3} we have \(x < y\).
    \end{enumerate}
    \item If \(y = -\infty\), then by Definition \ref{6.2.3} we have \(x \geq y \ \forall\ x \in \mathbf{R}^*\).
    \begin{enumerate}[label=(\roman*)]
        \item If \(x = +\infty\), then by Definition \ref{6.2.1} \(x \neq y\).
        Thus by Definition \ref{6.2.3} we have \(x > y\).
        \item If \(x \in \mathbf{R}\), then by Definition \ref{6.2.1} \(x \neq y\).
        Thus by Definition \ref{6.2.3} we have \(x > y\).
        \item If \(x = -\infty\), then we have \(x = y\).
    \end{enumerate}
\end{enumerate}
From all cases above we show that exactly one of the statements \(x < y\), \(x = y\), or \(x > y\) is true.
Thus we finish the proof.
\end{proof}

\begin{proof}{(c)}
By Proposition \ref{5.4.7}, we already have \((x \leq y) \land (y \leq z) \implies x \leq z\) when \(x, y, z \in \mathbf{R}\).
So we only need to consider the cases \(x = +\infty\), \(x = -\infty\), \(y = +\infty\), \(y = -\infty\), \(z = +\infty\), or \(z = -\infty\).
\begin{enumerate}[label=(\Roman*)]
    \item If \(x = +\infty\), then by Definition \ref{6.2.3} \((x = +\infty) \land (x \leq y) \implies y = +\infty\).
    Similarly \((y = +\infty) \land (y \leq z) \implies z = +\infty\).
    Thus we have \(x = +\infty = z\), and by Proposition \ref{6.2.5}(a) we have \(x \leq z\).
    \item If \(x = -\infty\), then by Definition \ref{6.2.3} \(x \leq z \ \forall\ z \in \mathbf{R}^*\).
    \item If \(y = +\infty\), then by Definition \ref{6.2.3} \((y = +\infty) \land (y \leq z) \implies z = +\infty\).
    Again by Definition \ref{6.2.3}, we have \(x \leq +\infty = z \ \forall\ x \in \mathbf{R}^*\).
    \item If \(y = -\infty\), then by Definition \ref{6.2.3} \((y = -\infty) \land (x \leq y) \implies x = -\infty\).
    Again by Definition \ref{6.2.3}, we have \(x = -\infty \leq z \ \forall\ z \in \mathbf{R}^*\).
    \item If \(z = +\infty\), then by Definition \ref{6.2.3}, we have \(x \leq +\infty = z \ \forall\ x \in \mathbf{R}^*\).
    \item If \(z = -\infty\), then by Definition \ref{6.2.3} \((z = -\infty) \land (y \leq z) \implies y = -\infty\).
    Similarly \((y = -\infty) \land (x \leq y) \implies x = -\infty\).
    Thus we have \(x = -\infty = z\), and by Proposition \ref{6.2.5}(a) we have \(x \leq z\).
\end{enumerate}
From all cases above we show that \((x \leq y) \land (y \leq z) \implies x \leq z\).
Thus we finish the proof.
\end{proof}

\begin{proof}{(d)}
By Proposition \ref{5.4.7}, we already have \(x \leq y \implies -y \leq -x\) when \(x, y \in \mathbf{R}\).
So we only need to consider the cases \(x = +\infty\), \(x = -\infty\), \(y = +\infty\), or \(y = -\infty\).
\begin{enumerate}[label=(\Roman*)]
    \item If \(x = +\infty\), then by Definition \ref{6.2.3} \((x = +\infty) \land (x \leq y) \implies y = +\infty\).
    And by Definition \ref{6.2.2} we have \(-x = -\infty = -y\).
    Thus by Proposition \ref{6.2.5}(a) we have \(-y \leq -x\).
    \item If \(x = -\infty\), then by Definition \ref{6.2.2} \(-x = +\infty\).
    And by Definition \ref{6.2.3} we have \(-y \leq -x \ \forall\ -y \in \mathbf{R}^*\).
    \item If \(y = +\infty\), then by Definition \ref{6.2.2} \(-y = -\infty\).
    And by Definition \ref{6.2.3} we have \(-y \leq -x \ \forall\ -x \in \mathbf{R}^*\).
    \item If \(y = -\infty\), then by Definition \ref{6.2.3} \((y = -\infty) \land (x \leq y) \implies x = -\infty\).
    And by Definition \ref{6.2.2} we have \(-x = +\infty = -y\).
    Thus by Proposition \ref{6.2.5}(a) we have \(-y \leq -x\).
\end{enumerate}
From all cases above we show that \(x \leq y \implies -y \leq -x\).
Thus we finish the proof.
\end{proof}

\begin{note}
One could also introduce other operations on the extended real number system, such as addition, multiplication, etc.
However, this is somewhat dangerous as these operations will almost certainly fail to obey the familiar rules of algebra.
For instance, to define addition it seems reasonable (given one’s intuitive notion of infinity) to set \(+\infty + 5 = +\infty\) and \(+\infty + 3 = +\infty\), but then this implies that \(+\infty + 5 = +\infty + 3\), while \(5 \neq 3\).
So things like the cancellation law begin to break down once we try to operate involving infinity.
To avoid these issues we shall simply not define any arithmetic operations on the extended real number system other than negation and order.
\end{note}

\begin{definition}[Supremum of sets of extended reals]\label{6.2.6}
Let \(E\) be a subset of \(\mathbf{R}^*\).
Then we define the \emph{supremum} \(\sup(E)\) or \emph{least upper bound} of \(E\) by the following rule.
\begin{enumerate}
    \item If \(E\) is contained in \(\mathbf{R}\) (i.e., \(+\infty\) and \(-\infty\) are not elements of \(E\)), then we let \(\sup(E)\) be as defined in Definition \ref{5.5.10}.
    \item If \(E\) contains \(+\infty\), then we set \(\sup(E) \coloneqq +\infty\).
    \item If \(E\) does not contain \(+\infty\) but does contain \(-\infty\), then we set \(\sup(E) \coloneqq \sup(E \setminus \{-\infty\})\)
    (which is a subset of \(\mathbf{R}\) and thus falls under case (a)).
\end{enumerate}
We also define the \emph{infimum} \(\inf(E)\) of \(E\) (also known as the \emph{greatest lower bound} of \(E\)) by the formula
\[
    \inf(E) \coloneqq -\sup(-E)
\]
where \(-E\) is the set \(-E \coloneqq \{-x : x \in E\}\).
\end{definition}

\setcounter{theorem}{9}
\begin{example}\label{6.2.10}
Let \(E\) be the empty set.
Then \(\sup(E) = -\infty\) and \(\inf(E) = +\infty\).
(Because \(-E\) is also empty, so \(\sup(-E) = -\infty\), thus \(\inf(E) = -\sup(-E) = -(-\infty) = +\infty\))
This is the only case in which the supremum can be less than the infimum.
We show that by let \(\sup(E)\) be three different value, namely \(+\infty\), \(-\infty\), or arbitrary real number.
\begin{enumerate}[label=(\Roman*)]
    \item If \(\sup(E) = +\infty\), then we can further divide into two cases:
    \begin{enumerate}[label=(\roman*)]
        \item If \(-\infty \in E\), then \(+\infty \in -E\).
        So \(\inf(E) = -\sup(-E) = -(+\infty) = -\infty\), and we have \(\inf(E) < \sup(E)\).
        \item If \(-\infty \not\in E\), then \(+\infty \not\in -E\).
        So \(\inf(E) \in \mathbf{R}\), and we have \(\inf(E) < \sup(E)\).
    \end{enumerate}
    \item If \(\sup(E) = -\infty\), then \(\inf(E) = +\infty\).
    So we have \(\inf(E) > \sup(E)\).
    \item If \(\sup(E) \in \mathbf{R}\), then we can further divide into two cases:
    \begin{enumerate}[label=(\roman*)]
        \item If \(-\infty \in E\), then \(+\infty \in -E\).
        So \(\inf(E) = -\sup(-E) = -(+\infty) = -\infty\), and we have \(\inf(E) < \sup(E)\).
        \item If \(-\infty \not\in E\), then \(+\infty \not\in -E\).
        So \(\inf(E) \in \mathbf{R}\), and we have
        \begin{align*}
            & (x \in E) \land (x \leq \sup(E)) \\
            \implies & -x \geq -\sup(E) \\
            \implies & \sup(-E) \geq -x \geq -\sup(E) \\
            \implies & \inf(E) = -\sup(-E) \leq x \leq \sup(E).
        \end{align*}
    \end{enumerate}
\end{enumerate}
Thus only when \(\sup(E) = -\infty\) we have \(\inf(E) > \sup(E)\).
\end{example}

\begin{note}
One can intuitively think of the supremum of \(E\) as follows.
Imagine the real line with \(+\infty\) somehow on the far right, and \(-\infty\) on the far left.
Imagine a piston at \(+\infty\) moving leftward until it is stopped by the presence of a set \(E\);
the location where it stops is the supremum of \(E\).
Similarly if one imagines a piston at \(-\infty\) moving rightward until it is stopped by the presence of \(E\), the location where it stops is the infimum of \(E\).
In the case when \(E\) is the empty set, the pistons pass through each other, the supremum landing at \(-\infty\) and the infimum landing at \(+\infty\).
\end{note}

\begin{theorem}\label{6.2.11}
Let \(E\) be a subset of \(\mathbf{R}^*\).
Then the following statements are true.
\begin{enumerate}
    \item For every \(x \in E\) we have \(x \leq \sup(E)\) and \(x \geq \inf(E)\).
    \item Suppose that \(M \in \mathbf{R}^*\) is an upper bound for \(E\), i.e., \(x \leq M\) for all \(x \in E\).
    Then we have \(\sup(E) \leq M\).
    \item Suppose that \(M \in \mathbf{R}^*\) is a lower bound for \(E\), i.e., \(x \geq M\) for all \(x \in E\).
    Then we have \(\inf(E) \geq M\).
\end{enumerate}
\end{theorem}

\begin{proof}{(a)}
We first show that \(\forall\ x \in E\), \(x \leq \sup(E)\).
Suppose first that \(E = \emptyset\).
Then the statements \(\forall\ x \in \emptyset\), \(x \leq \sup(E)\) is vacuously true.
Now suppose that \(E \neq \emptyset\).
We split into two different cases:
\begin{enumerate}[label=(\Roman*)]
    \item If \(+\infty \not\in E\), then we can further split into two cases:
    \begin{enumerate}[label=(\roman*)]
        \item If \(E \neq \{-\infty\}\), let \(E' = E \setminus \{-\infty\}\), so \(E' \neq \emptyset\).
        By Theorem \ref{5.5.9} we have \(x \leq \sup(E')\).
        And by Definition \ref{6.2.6} we have \(\sup(E) = \sup(E')\), so \(x \leq \sup(E)\).
        \item If \(E = \{-\infty\}\), then by Definition \ref{6.2.6} we have \(\sup(E) = \sup(\emptyset)\).
        And we already show that \(\forall\ x \in \emptyset\), \(x \leq \sup(E)\) is vacuously true.
    \end{enumerate}
    \item If \(+\infty \in E\), then \(\sup(E) = +\infty\), so by Definition \ref{6.2.3} \(x \leq \sup(E)\).
\end{enumerate}
Thus we conclude that \(\forall\ x \in E\), \(x \leq \sup(E)\).

Now we show that \(\forall\ x \in E\), \(x \geq \inf(E)\).
Suppose first that \(E \neq \emptyset\).
From above proof we have \(x \leq \sup(E)\).
So
\begin{align*}
& x \leq \sup(E) \\
\implies & -x \geq -\sup(E) \\
\implies & \sup(-E) \geq -x \geq -\sup(E) \\
\implies & \inf(E) = -\sup(-E) \leq x \leq \sup(E).
\end{align*}
Now suppose that \(E = \emptyset\).
So the statements \(\forall\ x \in \emptyset\), \(x \geq \inf(E)\) is vacuously true.
Thus we conclude that \(\forall\ x \in E\), \(x \geq \inf(E)\).
\end{proof}

\begin{proof}{(b)}
Suppose first that \(E = \emptyset\).
Then by Example \ref{6.2.10} and Definition \ref{6.2.3} we have \(\sup(E) = -\infty \leq M\).
Now suppose that \(E \neq \emptyset\).
We split into two different cases:
\begin{enumerate}[label=(\Roman*)]
    \item If \(+\infty \not\in E\), then we can further split into two cases:
    \begin{enumerate}[label=(\roman*)]
        \item If \(E \neq \{-\infty\}\), let \(E' = E \setminus \{-\infty\}\), so \(E' \neq \emptyset\).
        By Theorem \ref{5.5.9} and Definition \ref{5.5.5} we have \(x \leq \sup(E') \leq M\).
        And by Definition \ref{6.2.6} we have \(\sup(E) = \sup(E')\), so \(x \leq \sup(E) \leq M\).
        \item If \(E = \{-\infty\}\), then by Definition \ref{6.2.6} we have \(\sup(E) = \sup(\emptyset)\).
        And we already show that \(\sup(E) = -\infty \leq M\).
    \end{enumerate}
    \item If \(+\infty \in E\), then \(\sup(E) = +\infty\).
    By the given condition we have \(\sup(E) \in E \implies \sup(E) \leq M\).
\end{enumerate}
Thus we conclude that \(\sup(E) \leq M\).
\end{proof}

\begin{proof}{(c)}
Suppose first that \(E = \emptyset\).
Then by Example \ref{6.2.10} and Definition \ref{6.2.3} we have \(\inf(E) = +\infty \geq M\).
Now suppose that \(E \neq \emptyset\).
We split into two different cases:
\begin{enumerate}[label=(\Roman*)]
    \item If \(-\infty \not\in E\), then \(+\infty \not\in -E\).
    We can further split into two cases:
    \begin{enumerate}[label=(\roman*)]
        \item If \(-E \neq \{-\infty\}\), let \(-E' = -E \setminus \{-\infty\}\), so \(-E' \neq \emptyset\).
        By the given condition we have \(-M \geq -x \in -E'\), which means \(-M \geq \sup(-E') \geq -x\) by Theorem \ref{5.5.9}.
        And by Definition \ref{6.2.6} we have \(\sup(-E) = \sup(-E')\), so \(M \leq -\sup(-E) = \inf(E) \leq x\).
        \item If \(-E = \{-\infty\}\), then by Definition \ref{6.2.6} we have \(\sup(-E) = \sup(\emptyset)\).
        And we already show that \(\inf(E) = -\sup(-E) = -(-\infty) = +\infty \geq M\).
    \end{enumerate}
    \item If \(-\infty \in E\), then \(\sup(-E) = +\infty\).
    By the given condition we have \(\inf(E) = -\sup(-E) = -(+\infty) = -\infty \in E \implies \inf(E) \geq M\).
\end{enumerate}
Thus we conclude that \(\inf(E) \geq M\).
\end{proof}

\exercisesection

\begin{exercise}\label{ex 6.2.1}
Prove Proposition \ref{6.2.5}.
\end{exercise}

\begin{proof}
See Proposition \ref{6.2.5}.
\end{proof}

\begin{exercise}\label{ex 6.2.2}
Prove Theorem \ref{6.2.11}.
\end{exercise}

\begin{proof}
See Theorem \ref{6.2.11}.
\end{proof}
\chapter{Series}\label{ch 7}

\section{Finite series}\label{sec 7.1}

\begin{definition}[Finite series]\label{7.1.1}
  Let \(m, n\) be integers, and let \((a_i)_{i = m}^n\) be a finite sequence of real numbers, assigning a real number \(a_i\) to each integer \(i\) between \(m\) and \(n\) inclusive (i.e., \(m \leq i \leq n\)).
  Then we define the finite sum (or finite series) \(\sum_{i = m}^n a_i\) by the recursive formula
  \begin{align*}
     & \sum_{i = m}^n a_i \coloneqq 0 \text{ whenever } n < m ;                                                      \\
     & \sum_{i = m}^{n + 1} a_i \coloneqq \Bigg(\sum_{i = m}^n a_i\Bigg) + a_{n + 1} \text{ whenever } n \geq m - 1.
  \end{align*}
\end{definition}

\begin{note}
  we sometimes express \(\sum_{i = m}^n a_i\) less formally as
  \[
    \sum_{i = m}^n a_i = a_m + a_{m + 1} + \dots + a_n.
  \]
\end{note}

\begin{remark}\label{7.1.2}
  The difference between ``sum'' and ``series'' is a subtle linguistic one.
  Strictly speaking, a series is an \emph{expression} of the form \(\sum_{i = m}^n a_i\);
  this series is mathematically (but not semantically) equal to a real number, which is then the \emph{sum} of that series.
  For instance, \(1 + 2 + 3 + 4 + 5\) is a series, whose sum is \(15\);
  if one were to be very picky about semantics, one would not consider \(15\) a series and one would not consider \(1 + 2 + 3 + 4 + 5\) a sum, despite the two expressions having the same value.
  However, we will not be very careful about this distinction as it is purely linguistic and has no bearing on the mathematics;
  the expressions \(1 + 2 + 3 + 4 + 5\) and \(15\) are the same number, and thus \emph{mathematically} interchangeable, in the sense of the axiom of substitution, even if they are not semantically interchangeable.
\end{remark}

\begin{remark}\label{7.1.3}
  Note that the variable \(i\) (sometimes called the \emph{index of summation}) is a \emph{bound variable} (sometimes called a \emph{dummy variable});
  the expression \(\sum_{i = m}^n a_i\) does not actually depend on any quantity named \(i\).
  In particular, one can replace the index of summation \(i\) with any other symbol, and obtain the same sum:
  \[
    \sum_{i = m}^n a_i = \sum_{j = m}^n a_j.
  \]
\end{remark}

\begin{lemma}\label{7.1.4}
  \mbox{}
  \begin{enumerate}
    \item Let \(m \leq n < p\) be integers, and let \(a_i\) be a real number assigned to each integer \(m \leq i \leq p\).
          Then we have
          \[
            \sum_{i = m}^n a_i + \sum_{i = n + 1}^p a_i = \sum_{i = m}^p a_i.
          \]
    \item Let \(m \leq n\) be integers, \(k\) be another integer, and let \(a_i\) be a real number assigned to each integer \(m \leq i \leq n\).
          Then we have
          \[
            \sum_{i = m}^n a_i = \sum_{j = m + k}^{n + k} a_{j - k}.
          \]
    \item Let \(m \leq n\) be integers, and let \(a_i, b_i\) be real numbers assigned to each integer \(m \leq i \leq n\).
          Then we have
          \[
            \sum_{i = m}^n (a_i + b_i) = \Bigg(\sum_{i = m}^n a_i\Bigg) + \Bigg(\sum_{i = m}^n b_i\Bigg).
          \]
    \item Let \(m \leq n\) be integers, and let \(a_i\) be a real number assigned to each integer \(m \leq i \leq n\), and let \(c\) be another real number.
          Then we have
          \[
            \sum_{i = m}^n (ca_i) = c\Bigg(\sum_{i = m}^n a_i\Bigg).
          \]
    \item (Triangle inequality for finite series)
          Let \(m \leq n\) be integers, and let \(a_i\) be a real number assigned to each integer \(m \leq i \leq n\).
          Then we have
          \[
            \abs{\sum_{i = m}^n a_i} \leq \sum_{i = m}^n \abs{a_i}.
          \]
    \item (Comparison test for finite series) Let \(m \leq n\) be integers, and let \(a_i\), \(b_i\) be real numbers assigned to each integer \(m \leq i \leq n\).
          Suppose that \(a_i \leq b_i\) for all \(m \leq i \leq n\).
          Then we have
          \[
            \sum_{i = m}^n a_i \leq \sum_{i = m}^n b_i
          \]
  \end{enumerate}
\end{lemma}

\begin{proof}{(a)}
  Let \(k = p - m\).
  By hypothesis we know that \(k > 0\).
  Now we use induction on \(k\) to show that \cref{7.1.4}(a) is true and we start with \(k = 1\).
  For \(k = 1\), we have \(p = m + 1\) and by \cref{7.1.1} we have
  \[
    \sum_{i = m}^n a_i + \sum_{i = n + 1}^p a_i = \sum_{i = m}^m a_i + \sum_{i = m + 1}^p a_i = a_m + a_{m + 1} = \sum_{i = m}^p a_i.
  \]
  Thus the base case holds.
  Suppose inductively that for some \(k \geq 1\) \cref{7.1.4}(a) is true.
  Then for \(k + 1 = p - m\), we have \(p - 1 = k + m\) and
  \begin{align*}
    \sum_{i = m}^n a_i + \sum_{i = n + 1}^p a_i & = \Bigg(\sum_{i = m}^n a_i\Bigg) + \Bigg(\sum_{i = n + 1}^{p - 1} a_i\Bigg) + a_p & \text{(by \cref{7.1.1})}         \\
                                                & = \Bigg(\sum_{i = m}^{p - 1} a_i\Bigg) + a_p                                      & \text{(by induction hypothesis)} \\
                                                & = \sum_{i = m}^p a_i.                                                             & \text{(by \cref{7.1.1})}
  \end{align*}
  This closes the induction.
\end{proof}

\begin{proof}{(b)}
  Let \(p = n - m\).
  By hypothesis we know that \(p \geq 0\).
  Now we use induction on \(p\) to show that \cref{7.1.4}(b) is true.
  For \(p = 0\), we have \(n = m\) and
  \begin{align*}
    \sum_{j = m + k}^{m + k} a_{j - k} & = \Bigg(\sum_{j = m + k}^{m + k - 1} a_{j - k}\Bigg) + a_{m + k - k} & \text{(by \cref{7.1.1})} \\
                                       & = 0 + a_{m + k - k}                                                  & \text{(by \cref{7.1.1})} \\
                                       & = 0 + a_m                                                                                       \\
                                       & = \Bigg(\sum_{i = m}^{m - 1} a_i\Bigg) + a_m                         & \text{(by \cref{7.1.1})} \\
                                       & = \sum_{i = m}^m a_i.                                                & \text{(by \cref{7.1.1})}
  \end{align*}
  So the base case holds.
  Suppose inductively that for some \(p \geq 0\) \cref{7.1.4}(b) is true.
  Then for \(p + 1 = n - m\), we have \(p = n - m - 1\) and
  \begin{align*}
    \sum_{j = m + k}^{n + k} a_{j - k} & = \Bigg(\sum_{j = m + k}^{n + k - 1} a_{j - k}\Bigg) + a_{n + k - k} & \text{(by \cref{7.1.1})}         \\
                                       & = \Bigg(\sum_{j = m + k}^{n + k - 1} a_{j - k}\Bigg) + a_n                                              \\
                                       & = \Bigg(\sum_{i = m}^{n - 1} a_i\Bigg) + a_n                         & \text{(by induction hypothesis)} \\
                                       & = \sum_{i = m}^n a_i.                                                & \text{(by \cref{7.1.1})}
  \end{align*}
  This closes the induction.
\end{proof}

\begin{proof}{(c)}
  Let \(p = n - m\).
  By hypothesis we know that \(p \geq 0\).
  Now we use induction on \(p\) to show that \cref{7.1.4}(c) is true.
  For \(p = 0\), we have \(n = m\) and
  \begin{align*}
    \sum_{i = m}^m (a_i + b_i) & = \Bigg(\sum_{i = m}^{m - 1} (a_i + b_i)\Bigg) + a_m + b_m                                & \text{(by \cref{7.1.1})} \\
                               & = 0 + a_m + b_m                                                                           & \text{(by \cref{7.1.1})} \\
                               & = \Bigg(\sum_{i = m}^{m - 1} a_i\Bigg) + \Bigg(\sum_{i = m}^{m - 1} b_i\Bigg) + a_m + b_m & \text{(by \cref{7.1.1})} \\
                               & = \Bigg(\sum_{i = m}^m a_i\Bigg) + \Bigg(\sum_{i = m}^m b_i\Bigg).                        & \text{(by \cref{7.1.1})}
  \end{align*}
  So the base case holds.
  Suppose inductively that for some \(p \geq 0\) \cref{7.1.4}(c) is true.
  Then for \(p + 1 = n - m\), we have \(p = n - m - 1\) and
  \begin{align*}
    \sum_{i = m}^n (a_i + b_i) & = \Bigg(\sum_{i = m}^{n - 1} (a_i + b_i)\Bigg) + a_n + b_n                                & \text{(by \cref{7.1.1})}         \\
                               & = \Bigg(\sum_{i = m}^{n - 1} a_i\Bigg) + \Bigg(\sum_{i = m}^{n - 1} b_i\Bigg) + a_n + b_n & \text{(by induction hypothesis)} \\
                               & = \Bigg(\sum_{i = m}^n a_i\Bigg) + \Bigg(\sum_{i = m}^n b_i\Bigg).                        & \text{(by \cref{7.1.1})}
  \end{align*}
  This closes the induction.
\end{proof}

\begin{proof}{(d)}
  Let \(p = n - m\).
  By hypothesis we know that \(p \geq 0\).
  Now we use induction on \(p\) to show that \cref{7.1.4}(d) is true.
  For \(p = 0\), we have \(n = m\) and
  \begin{align*}
    \sum_{i = m}^m ca_i & = \Bigg(\sum_{i = m}^{m - 1} ca_i\Bigg) + ca_m             & \text{(by \cref{7.1.1})} \\
                        & = 0 + ca_m                                                 & \text{(by \cref{7.1.1})} \\
                        & = c \times 0 + ca_m                                                                   \\
                        & = c \Bigg(\sum_{i = m}^{m - 1} a_i\Bigg) + ca_m            & \text{(by \cref{7.1.1})} \\
                        & = c \Bigg(\Bigg(\sum_{i = m}^{m - 1} a_i\Bigg) + a_m\Bigg)                            \\
                        & = c \Bigg(\sum_{i = m}^m a_i\Bigg).                        & \text{(by \cref{7.1.1})}
  \end{align*}
  So the base case holds.
  Suppose inductively that for some \(p \geq 0\) \cref{7.1.4}(d) is true.
  Then for \(p + 1 = n - m\), we have \(p = n - m - 1\) and
  \begin{align*}
    \sum_{i = m}^n ca_i & = \Bigg(\sum_{i = m}^{n - 1} ca_i\Bigg) + ca_n             & \text{(by \cref{7.1.1})}         \\
                        & = c \Bigg(\sum_{i = m}^{n - 1} a_i\Bigg) + ca_n            & \text{(by induction hypothesis)} \\
                        & = c \Bigg(\Bigg(\sum_{i = m}^{n - 1} a_i\Bigg) + a_n\Bigg)                                    \\
                        & = c \Bigg(\sum_{i = m}^n a_i\Bigg).                        & \text{(by \cref{7.1.1})}
  \end{align*}
  This closes the induction.
\end{proof}

\begin{proof}{(e)}
  Let \(p = n - m\).
  By hypothesis we know that \(p \geq 0\).
  Now we use induction on \(p\) to show that \cref{7.1.4}(e) is true.
  For \(p = 0\), we have \(n = m\) and
  \begin{align*}
    \abs{\sum_{i = m}^m a_i} & = \abs{\Bigg(\sum_{i = m}^{m - 1} a_i\Bigg) + a_m}       & \text{(by \cref{7.1.1})} \\
                             & = \abs{0 + a_m}                                          & \text{(by \cref{7.1.1})} \\
                             & = 0 + \abs{a_m}                                                                     \\
                             & = \Bigg(\sum_{i = m}^{m - 1} \abs{a_i}\Bigg) + \abs{a_m} & \text{(by \cref{7.1.1})} \\
                             & = \sum_{i = m}^m \abs{a_i}.                              & \text{(by \cref{7.1.1})}
  \end{align*}
  So the base case holds.
  Suppose inductively that for some \(p \geq 0\) \cref{7.1.4}(e) is true.
  Then for \(p + 1 = n - m\), we have \(p = n - m - 1\) and
  \begin{align*}
    \abs{\sum_{i = m}^n a_i} & = \abs{\Bigg(\sum_{i = m}^{n - 1} a_i\Bigg) + a_n} & \text{(by \cref{7.1.1})}         \\
                             & \leq \abs{\sum_{i = m}^{n - 1} a_i} + \abs{a_n}                                       \\
                             & \leq \sum_{i = m}^{n - 1} \abs{a_i} + \abs{a_n}    & \text{(by induction hypothesis)} \\
                             & = \sum_{i = m}^n \abs{a_i}.                        & \text{(by \cref{7.1.1})}
  \end{align*}
  This closes the induction.
\end{proof}

\begin{proof}{(f)}
  Let \(p = n - m\).
  By hypothesis we know that \(p \geq 0\).
  Now we use induction on \(p\) to show that \cref{7.1.4}(f) is true.
  For \(p = 0\), we have \(n = m\) and
  \begin{align*}
    \sum_{i = m}^m a_i & = \Bigg(\sum_{i = m}^{m - 1} a_i\Bigg) + a_m & \text{(by \cref{7.1.1})} \\
                       & = 0 + a_m                                    & \text{(by \cref{7.1.1})} \\
                       & \leq 0 + b_m                                 & \text{(by hypothesis)}   \\
                       & = \Bigg(\sum_{i = m}^{m - 1} b_i\Bigg) + b_m & \text{(by \cref{7.1.1})} \\
                       & = \sum_{i = m}^m b_i.                        & \text{(by \cref{7.1.1})} \\
  \end{align*}
  So the base case holds.
  Suppose inductively that for some \(p \geq 0\) \cref{7.1.4}(f) is true.
  Then for \(p + 1 = n - m\), we have \(p = n - m - 1\) and
  \begin{align*}
    \sum_{i = m}^n a_i & = \Bigg(\sum_{i = m}^{n - 1} a_i\Bigg) + a_n    & \text{(by \cref{7.1.1})}         \\
                       & \leq \Bigg(\sum_{i = m}^{n - 1} b_i\Bigg) + a_n & \text{(by induction hypothesis)} \\
                       & \leq \Bigg(\sum_{i = m}^{n - 1} b_i\Bigg) + b_n & \text{(by hypothesis)}           \\
                       & = \sum_{i = m}^n b_i.                           & \text{(by \cref{7.1.1})}         \\
  \end{align*}
  This closes the induction.
\end{proof}

\begin{remark}\label{7.1.5}
  In the future we may omit some of the parentheses in series expressions, for instance we may write \(\sum_{i = m}^n (a_i + b_i)\) simply as \(\sum_{i = m}^n a_i + b_i\).
  This is reasonably safe from being mis-interpreted, because the alternative interpretation \((\sum_{i = m}^n a_i) + b_i\) does not make any sense
  (the index \(i\) in \(b_i\) is meaningless outside of the summation, since \(i\) is only a dummy variable).
\end{remark}

\begin{definition}[Summations over finite sets]\label{7.1.6}
  Let \(X\) be a finite set with \(n\) elements (where \(n \in \N\)), and let \(f : X \to \R\) be a function from \(X\) to the real numbers
  (i.e., \(f\) assigns a real number \(f(x)\) to each element \(x\) of \(X\)).
  Then we can define the finite sum \(\sum_{x \in X} f(x)\) as follows.
  We first select any bijection \(g\) from \(\{i \in \N : 1 \leq i \leq n\}\) to \(X\);
  such a bijection exists since \(X\) is assumed to have \(n\) elements.
  We then define
  \[
    \sum_{x \in X} f(x) \coloneqq \sum_{i = 1}^n f(g(i)).
  \]
  In some cases we would like to define the sum \(\sum_{x \in X} f(x)\) when \(f : Y \to \R\) is defined on a larger set \(Y\) than \(X\).
  In such cases we use exactly the same definition as is given above.
\end{definition}

\setcounter{theorem}{7}
\begin{proposition}[Finite summations are well-defined]\label{7.1.8}
  Let \(X\) be a finite set with \(n\) elements (where \(n \in \N\)), let \(f : X \to \R\) be a function, and let \(g : \{i \in \N : 1 \leq i \leq n\} \to X\) and \(h : \{i \in \N : 1 \leq i \leq n\} \to X\) be bijections.
  Then we have
  \[
    \sum_{i = 1}^n f(g(i)) = \sum_{i = 1}^n f(h(i)).
  \]
\end{proposition}

\begin{proof}
  We use induction on \(n\);
  more precisely, we let \(P(n)\) be the assertion that ``For any set \(X\) of \(n\) elements, any function \(f : X \to \R\), and any two bijections \(g, h\) from \(\{i \in \N : 1 \leq i \leq n\}\) to \(X\), we have \(\sum_{i = 1}^n f(g(i)) = \sum_{i = 1}^n f(h(i))\)''.
  (More informally, \(P(n)\) is the assertion that \cref{7.1.8} is true for that value of \(n\).)
  We want to prove that \(P(n)\) is true for all natural numbers \(n\).

  We first check the base case \(P(0)\).
  In this case \(\sum_{i = 1}^0 f(g(i))\) and \(\sum_{i = 1}^0 f(h(i))\) both equal to \(0\), by definition of finite series (\cref{7.1.1}), so we are done.

  Now suppose inductively that \(P(n)\) is true;
  we now prove that \(P(n + 1)\) is true.
  Thus, let \(X\) be a set with \(n + 1\) elements, let \(f : X \to \R\) be a function, and let \(g\) and \(h\) be bijections from \(\{i \in N : 1 \leq i \leq n + 1\}\) to \(X\).
  We have to prove that
  \[
    \sum_{i = 1}^{n + 1} f(g(i)) = \sum_{i = 1}^{n + 1} f(h(i)). \tag{7.1}\label{eq 7.1}
  \]
  Let \(x \coloneqq g(n + 1)\);
  thus \(x\) is an element of \(X\).
  By definition of finite series (\cref{7.1.1}), we can expand the left-hand side of \eqref{eq 7.1} as
  \[
    \sum_{i = 1}^{n + 1} f(g(i)) = \Bigg(\sum_{i = 1}^n f(g(i))\Bigg) + f(x).
  \]
  Now let us look at the right-hand side of \eqref{eq 7.1}.
  Ideally we would like to have \(h(n + 1)\) also equal to \(x\)
  - this would allow us to use the inductive hypothesis \(P(n)\) much more easily
  - but we cannot assume this.
  However, since \(h\) is a bijection, we do know that there is \emph{some} index \(j\), with \(1 \leq j \leq n + 1\), for which \(h(j) = x\).
  We now use \cref{7.1.4} and the definition of finite series (\cref{7.1.1}) to write
  \begin{align*}
    \sum_{i = 1}^{n + 1} f(h(i)) & = \Bigg(\sum_{i = 1}^j f(h(i))\Bigg) + \Bigg(\sum_{i = j + 1}^{n + 1} f(h(i))\Bigg)                 \\
                                 & = \Bigg(\sum_{i = 1}^{j - 1} f(h(i))\Bigg) + f(h(j)) + \Bigg(\sum_{i = j + 1}^{n + 1} f(h(i))\Bigg) \\
                                 & = \Bigg(\sum_{i = 1}^{j - 1} f(h(i))\Bigg) + f(x) + \Bigg(\sum_{i = j}^n f(h(i + 1))\Bigg).
  \end{align*}
  We now define the function \(\tilde{h} : \{i \in \N : 1 \leq i \leq n\} \to X - \{x\}\) by setting \(\tilde{h}(i) \coloneqq h(i)\) when \(i < j\) and \(\tilde{h}(i) \coloneqq h(i + 1)\) when \(i \geq j\).
  We can thus write the right-hand side of \eqref{eq 7.1} as
  \[
    = \Bigg(\sum_{i = 1}^{j - 1} f(\tilde{h}(i))\Bigg) + f(x) + \Bigg(\sum_{i = j}^n f(\tilde{h}(i))\Bigg) = \Bigg(\sum_{i = 1}^n f(\tilde{h}(i))\Bigg) + f(x)
  \]
  where we have used \cref{7.1.4} once again.
  Thus to finish the proof of \eqref{eq 7.1} we have to show that
  \[
    \sum_{i = 1}^n f(g(i)) = \sum_{i = 1}^n f(\tilde{h}(i)). \tag{7.2}\label{eq 7.2}
  \]
  But the function \(g\) (when restricted to \(\{i \in \N : 1 \leq i \leq n\}\)) is a bijection from \(\{i \in \N : 1 \leq i \leq n\} \to X - \{x\}\).
  The function \(\tilde{h}\) is also a bijection from \(\{i \in \N : 1 \leq i \leq n\} \to X - \{x\}\) (cf. \cref{3.6.9}).
  Since \(X - \{x\}\) has \(n\) elements (by \cref{3.6.9}), the claim \eqref{eq 7.2} then follows directly from the induction hypothesis \(P(n)\).
\end{proof}

\begin{remark}\label{7.1.9}
  The issue is somewhat more complicated when summing over infinite sets;
  See \cref{sec 8.2}.
\end{remark}

\begin{remark}\label{7.1.10}
  Suppose that \(X\) is a set, that \(P(x)\) is a property pertaining to an element \(x\) of \(X\), and \(f : \{y \in X : P(y) \text{ is true}\} \to \R\) is a function.
  Then we will often abbreviate
  \[
    \sum_{x \in \{y \in X : P(y) \text{ is true}\}} f(x)
  \]
  as \(\sum_{x \in X : P(x) \text{ is true}} f(x)\) or even as \(\sum_{P(x) \text{ is true}} f(x)\) when there is no change of confusion.
\end{remark}

\begin{proposition}[Basic properties of summation over finite sets]\label{7.1.11}
  \mbox{}
  \begin{enumerate}
    \item If \(X\) is empty, and \(f : X \to \R\) is a function (i.e., \(f\) is the empty function), we have
          \[
            \sum_{x \in X} f(x) = 0.
          \]
    \item If \(X\) consists of a single element, \(X = \{x_0\}\), and \(f : X \to \R\) is a function, we have
          \[
            \sum_{x \in X} f(x) = f(x_0).
          \]
    \item (Substitution, part I) If \(X\) is a finite set, \(f : X \to \R\) is a function, and \(g : Y \to X\) is a bijection, then
          \[
            \sum_{x \in X} f(x) = \sum_{y \in Y} f(g(y)).
          \]
    \item (Substitution, part II) Let \(n \leq m\) be integers, and let \(X\) be the set \(X \coloneqq \{i \in \Z : n \leq i \leq m\}\).
          If \(a_i\) is a real number assigned to each integer \(i \in X\), then we have
          \[
            \sum_{i = n}^m a_i = \sum_{i \in X} a_i.
          \]
    \item Let \(X, Y\) be disjoint finite sets (so \(X \cap Y = \emptyset\)), and \(f : X \cup Y \to \R\) is a function.
          Then we have
          \[
            \sum_{z \in X \cup Y} f(z) = \Bigg(\sum_{x \in X} f(x)\Bigg) + \Bigg(\sum_{y \in Y} f(y)\Bigg).
          \]
    \item (Linearity, part I) Let \(X\) be a finite set, and let \(f : X \to \R\) and \(g : X \to \R\) be functions.
          Then
          \[
            \sum_{x \in X} (f(x) + g(x)) = \sum_{x \in X} f(x) + \sum_{x \in X} g(x).
          \]
    \item (Linearity, part II) Let \(X\) be a finite set, let \(f : X \to \R\) be a function, and let \(c\) be a real number.
          Then
          \[
            \sum_{x \in X} cf(x) = c \sum_{x \in X} f(x).
          \]
    \item (Monotonicity) Let \(X\) be a finite set, and let \(f : X \to \R\) and \(g : X \to \R\) be functions such that \(f(x) \leq g(x)\) for all \(x \in \mathbf{X}\).
          Then we have
          \[
            \sum_{x \in X} f(x) \leq \sum_{x \in X} g(x).
          \]
    \item (Triangle inequality) Let \(X\) be a finite set, and let \(f : X \to \R\) be a function, then
          \[
            \abs{\sum_{x \in X} f(x)} \leq \sum_{x \in X} \abs{f(x)}.
          \]
  \end{enumerate}
\end{proposition}

\begin{proof}{(a)}
  Let \(g : \{i \in \N : 1 \leq i \leq 0\} \to \emptyset\) be a function.
  Then \(g\) is a bijection and
  \begin{align*}
    \sum_{x \in X} f(x) & = \sum_{i = 1}^0 f(g(i)) & \text{(by \cref{7.1.6})} \\
                        & = 0.                     & \text{(by \cref{7.1.1})}
  \end{align*}
\end{proof}

\begin{proof}{(b)}
  Let \(g : \{1\} \to \{x_0\}\) be a function.
  Then \(g\) is a bijection and
  \begin{align*}
    \sum_{x \in X} f(x) & = \sum_{i = 1}^1 f(g(i))                       & \text{(by \cref{7.1.6})} \\
                        & = \bigg(\sum_{i = 1}^0 f(g(i))\bigg) + f(g(1)) & \text{(by \cref{7.1.1})} \\
                        & = 0 + f(g(1))                                  & \text{(by \cref{7.1.1})} \\
                        & = f(x_0).
  \end{align*}
\end{proof}

\begin{proof}{(c)}
  Let \(h : \{i \in \N : 1 \leq i \leq \#(Y)\} \to Y\) be a bijection.
  Since \(X\) is finite and \(g\) is a bijection between \(X\) and \(Y\), we know that \(Y\) is finite and thus such \(h\) is well-defined.
  Then we know that \(g \circ h : \{i \in \N : 1 \leq i \leq \#(Y)\} \to X\) is also a bijection and
  \begin{align*}
    \sum_{x \in X} f(x) & = \sum_{i = 1}^{\#(Y)} f((g \circ h)(i)) & \text{(by \cref{7.1.6})} \\
                        & = \sum_{i = 1}^{\#(Y)} f(g(h(i)))                                   \\
                        & = \sum_{i = 1}^{\#(Y)} (f \circ g)(h(i))                            \\
                        & = \sum_{y \in Y} (f \circ g)(y)          & \text{(by \cref{7.1.6})} \\
                        & = \sum_{y \in Y} f(g(y)).
  \end{align*}
\end{proof}

\begin{proof}{(d)}
  Let \(f : X \to \{a_i \in \R : n \leq i \leq m\}\) be a function where \(f = i \mapsto a_i\).
  Let \(g : \{i \in \N : 1 \leq i \leq m - n + 1\} \to X\) be a function where \(g = i \mapsto i + n - 1\).
  Then \(g\) is a bijection and
  \begin{align*}
    \sum_{i \in X} a_i & = \sum_{i \in X} f(i)                                                                          \\
                       & = \sum_{i = 1}^{m - n + 1} f(g(i))                               & \text{(by \cref{7.1.6})}    \\
                       & = \sum_{i = 1}^{m - n + 1} f(i + n - 1)                                                        \\
                       & = \sum_{i = 1}^{m - n + 1} a_{i + n - 1}                                                       \\
                       & = \sum_{i = 1 + n - 1}^{m - n + 1 + n - 1} a_{i + n - 1 - n + 1} & \text{(by \cref{7.1.4}(b))} \\
                       & = \sum_{i = n}^m a_i.
  \end{align*}
\end{proof}

\begin{proof}{(e)}
  Let \(g : \{i \in \N : 1 \leq i \leq \#(X)\} \to X\) and \(h : \{i \in \N : 1 \leq i \leq \#(Y)\} \to Y\) be bijections.
  Since \(X, Y\) are finite, we know that \(g, h\) are well-defined and \(X \cup Y\) is finite.
  Let \(k : \{i \in \N : 1 \leq i \leq \#(X \cup Y)\} \to X \cup Y\) be a bijection where
  \[
    k(i) = \begin{cases}
      g(i)         & \text{if } 1 \leq i \leq \#(X)                  \\
      h(i - \#(X)) & \text{if } \#(X) + 1 \leq i \leq \#(X) + \#(Y).
    \end{cases}
  \]
  Since \(X \cup Y\) is finite, we know that \(k\) is well-defined and \(\#(X \cup Y) = \#(X) + \#(Y)\).
  Then we have
  \begin{align*}
    \sum_{z \in X \cup Y} f(z) & = \sum_{i = 1}^{\#(X \cup Y)} f(k(i))                                                & \text{(by \cref{7.1.6})}    \\
                               & = \sum_{i = 1}^{\#(X)} f(k(i)) + \sum_{i = \#(X) + 1}^{\#(X \cup Y)} f(k(i))         & \text{(by \cref{7.1.4}(a))} \\
                               & = \sum_{i = 1}^{\#(X)} f(g(i)) + \sum_{i = \#(X) + 1}^{\#(X \cup Y)} f(h(i - \#(X)))                               \\
                               & = \sum_{i = 1}^{\#(X)} f(g(i)) + \sum_{i = 1}^{\#(Y)} f(h(i))                        & \text{(by \cref{7.1.4}(b))} \\
                               & = \sum_{x \in X} f(x) + \sum_{y \in Y} f(y).                                         & \text{(by \cref{7.1.6})}
  \end{align*}
\end{proof}

\begin{proof}{(f)}
  Let \(h : \{i \in \N : 1 \leq i \leq \#(X)\} \to X\) be a bijection.
  Since \(X\) is finite, we know that \(h\) is well-defined and
  \begin{align*}
    \sum_{x \in X} (f(x) + g(x)) & = \sum_{x \in X} (f + g)(x)                                                                 \\
                                 & = \sum_{i = 1}^{\#(X)} (f + g)(h(i))                          & \text{(by \cref{7.1.6})}    \\
                                 & = \sum_{i = 1}^{\#(X)} (f(h(i)) + g(h(i)))                                                  \\
                                 & = \sum_{i = 1}^{\#(X)} f(h(i)) + \sum_{i = 1}^{\#(X)} g(h(i)) & \text{(by \cref{7.1.4}(c))} \\
                                 & = \sum_{x \in X} f(x) + \sum_{x \in X} g(x).                  & \text{(by \cref{7.1.6})}
  \end{align*}
\end{proof}

\begin{proof}{(g)}
  Let \(g : \{i \in \N : 1 \leq i \leq \#(X)\} \to X\) be a bijection.
  Since \(X\) is finite, we know that \(g\) is well-defined and
  \begin{align*}
    \sum_{x \in X} cf(x) & = \sum_{x \in X} (cf)(x)                                        \\
                         & = \sum_{i = 1}^{\#(X)} (cf)(g(i)) & \text{(by \cref{7.1.6})}    \\
                         & = \sum_{i = 1}^{\#(X)} cf(g(i))                                 \\
                         & = c \sum_{i = 1}^{\#(X)} f(g(i))  & \text{(by \cref{7.1.4}(d))} \\
                         & = c \sum_{x \in X} f(x).          & \text{(by \cref{7.1.6})}
  \end{align*}
\end{proof}

\begin{proof}{(h)}
  Let \(h : \{i \in \N : 1 \leq i \leq \#(X)\} \to X\) be a bijection.
  Since \(X\) is finite, we know that \(h\) is well-defined and
  \begin{align*}
    \sum_{x \in X} f(x) & = \sum_{i = 1}^{\#(X)} f(h(i))    & \text{(by \cref{7.1.6})}    \\
                        & \leq \sum_{i = 1}^{\#(X)} g(h(i)) & \text{(by \cref{7.1.4}(f))} \\
                        & = \sum_{x \in X} g(x).            & \text{(by \cref{7.1.6})}
  \end{align*}
\end{proof}

\begin{proof}{(i)}
  Let \(g : \{i \in \N : 1 \leq i \leq \#(X)\} \to X\) be a bijection.
  Since \(X\) is finite, we know that \(g\) is well-defined and
  \begin{align*}
    \abs{\sum_{x \in X} f(x)} & = \abs{\sum_{i = 1}^{\#(X)} f(g(i))}    & \text{(by \cref{7.1.6})}    \\
                              & \leq \sum_{i = 1}^{\#(X)} \abs{f(g(i))} & \text{(by \cref{7.1.4}(e))} \\
                              & = \sum_{x \in X} \abs{f(x)}.            & \text{(by \cref{7.1.6})}
  \end{align*}
\end{proof}

\begin{remark}\label{7.1.12}
  The substitution rule in \cref{7.1.11}(c) can be thought of as making the substitution \(x \coloneqq g(y)\) (hence the name).
  Note that the assumption that \(g\) is a bijection is essential.
  From \cref{7.1.11}(c) and (d) we see that
  \[
    \sum_{i = n}^m a_i = \sum_{i = n}^m a_{f(i)}
  \]
  for any bijection \(f\) from the set \(\{i \in \Z : n \leq i \leq m\}\) to itself.
  Informally, this means that we can rearrange the elements of a finite sequence at will and still obtain the same value.
\end{remark}

\begin{lemma}\label{7.1.13}
  Let \(X, Y\) be finite sets, and let \(f : X \times Y \to \R\) be a function.
  Then
  \[
    \sum_{x \in X} \bigg(\sum_{y \in Y} f(x, y)\bigg) = \sum_{(x, y) \in X \times Y} f(x, y).
  \]
\end{lemma}

\begin{proof}
  Let \(n\) be the number of elements in \(X\).
  We will use induction on \(n\) (cf. \cref{7.1.8});
  i.e., we let \(P(n)\) be the assertion that \cref{7.1.13} is true for any set \(X\) with \(n\) elements, and any finite set \(Y\) and any function \(f : X \times Y \to \R\).
  We wish to prove \(P(n)\) for all natural numbers \(n\).

  The base case \(P(0)\) is easy, following from \cref{7.1.11}(a).
  Now suppose that \(P(n)\) is true;
  we now show that \(P(n + 1)\) is true.
  Let \(X\) be a set with \(n + 1\) elements.
  In particular, by \cref{3.6.9}, we can write \(X = X' \cup \{x_0\}\), where \(x_0\) is an element of \(X\) and \(X' \coloneqq X - \{x_0\}\) has \(n\) elements.
  Then by \cref{7.1.11}(e) we have
  \[
    \sum_{x \in X} \bigg(\sum_{y \in Y} f(x, y)\bigg) = \sum_{x \in X'} \bigg(\sum_{y \in Y} f(x, y)\bigg) + \bigg(\sum_{y \in Y} f(x_0, y)\bigg);
  \]
  by the induction hypothesis this is equal to
  \[
    \sum_{(x, y) \in X' \times Y} f(x, y) + \bigg(\sum_{y \in Y} f(x_0, y)\bigg).
  \]
  By \cref{7.1.11}(c) this is equal to
  \[
    \sum_{(x, y) \in X' \times Y} f(x, y) + \bigg(\sum_{(x, y) \in \{x_0\} \times Y} f(x, y)\bigg).
  \]
  By \cref{7.1.11}(e) this is equal to
  \[
    \sum_{(x, y) \in X \times Y} f(x, y)
  \]
  as desired.
\end{proof}

\begin{corollary}[Fubini's theorem for finite series]\label{7.1.14}
  Let \(X, Y\) be finite sets, and let \(f : X \times Y \to \R\) be a function.
  Then
  \begin{align*}
    \sum_{x \in X} \bigg(\sum_{y \in Y} f(x, y)\bigg) & = \sum_{(x, y) \in X \times Y} f(x, y)               \\
                                                      & = \sum_{(y, x) \in Y \times X} f(x, y)               \\
                                                      & = \sum_{y \in Y} \bigg(\sum_{x \in X} f(x, y)\bigg).
  \end{align*}
\end{corollary}

\begin{proof}
  In light of \cref{7.1.13}, it suffices to show that
  \[
    \sum_{(x, y) \in X \times Y} f(x, y) = \sum_{(y, x) \in Y \times X} f(x, y).
  \]
  But this follows from \cref{7.1.11}(c) by applying the bijection \(h : Y \times X \to X \times Y\) defined by \(h(y, x) \coloneqq (x, y)\).
\end{proof}

\begin{remark}\label{7.1.15}
  We anticipate something interesting to happen when we move from finite sums to infinite sums.
  However, see \cref{8.2.2}.
\end{remark}

\begin{additional corollary}[Products over finite sets]\label{ac 7.1.1}
Let \(m, n\) be integers, and let \((a_i)_{i = m}^n\) be a finite sequence of real numbers, assigning a real number \(a_i\) to each integer \(i\) between \(m\) and \(n\) inclusive (i.e., \(m \leq i \leq n\)).
Then we define the finite product \(\prod_{i = m}^n a_i\) by the recursive formula
\begin{align*}
   & \prod_{i = m}^n a_i \coloneqq 1 \text{ whenever } n < m;                                                             \\
   & \prod_{i = m}^{n + 1} a_i \coloneqq \Bigg(\prod_{i = m}^n a_i\Bigg) \times a_{n + 1} \text{ whenever } n \geq m - 1.
\end{align*}
\end{additional corollary}

\begin{additional corollary}\label{ac 7.1.2}
\mbox{}
\begin{enumerate}
  \item Let \(m \leq n < p\) be integers, and let \(a_i\) be a real number assigned to each integer \(m \leq i \leq p\).
        Then we have
        \[
          \prod_{i = m}^n a_i \times \prod_{i = n + 1}^p a_i = \prod_{i = m}^p a_i.
        \]
  \item Let \(m \leq n\) be integers, \(k\) be another integer, and let \(a_i\) be a real number assigned to each integer \(m \leq i \leq n\).
        Then we have
        \[
          \prod_{i = m}^n a_i = \prod_{j = m + k}^{n + k} a_{j - k}.
        \]
  \item Let \(m \leq n\) be integers, and let \(a_i, b_i\) be real numbers assigned to each integer \(m \leq i \leq n\).
        Then we have
        \[
          \prod_{i = m}^n (a_i \times b_i) = \Bigg(\prod_{i = m}^n a_i\Bigg) \times \Bigg(\prod_{i = m}^n b_i\Bigg).
        \]
  \item Let \(m \leq n\) be integers, and let \(a_i\) be a real number assigned to each integer \(m \leq i \leq n\), and let \(c\) be another real number.
        Then we have
        \[
          \prod_{i = m}^n (ca_i) = c^{n - m + 1} \Bigg(\prod_{i = m}^n a_i\Bigg).
        \]
  \item Let \(m \leq n\) be integers, and let \(a_i\) be a real number assigned to each integer \(m \leq i \leq n\).
        Then we have
        \[
          \abs{\prod_{i = m}^n a_i} = \prod_{i = m}^n \abs{a_i}.
        \]
\end{enumerate}
\end{additional corollary}

\begin{proof}{(a)}
  Let \(k = p - m\).
  By hypothesis we know that \(k > 0\).
  Now we use induction on \(k\) to show that \cref{ac 7.1.2}(a) is true and we start with \(k = 1\).
  For \(k = 1\), we have \(p = m + 1\) and by \cref{ac 7.1.1} we have
  \[
    \prod_{i = m}^n a_i \times \prod_{i = n + 1}^p a_i = \prod_{i = m}^m a_i \times \prod_{i = m + 1}^p a_i = a_m \times a_{m + 1} = \prod_{i = m}^p a_i.
  \]
  Thus the base case holds.
  Suppose inductively that for some \(k \geq 1\) \cref{ac 7.1.2}(a) is true.
  Then for \(k + 1 = p - m\), we have \(p - 1 = k + m\) and
  \begin{align*}
     & \prod_{i = m}^n a_i \times \prod_{i = n + 1}^p a_i                                                                               \\
     & = \Bigg(\prod_{i = m}^n a_i\Bigg) \times \Bigg(\prod_{i = n + 1}^{p - 1} a_i\Bigg) \times a_p & \text{(by \cref{ac 7.1.1})}      \\
     & = \Bigg(\prod_{i = m}^{p - 1} a_i\Bigg) \times a_p                                            & \text{(by induction hypothesis)} \\
     & = \prod_{i = m}^p a_i.                                                                        & \text{(by \cref{ac 7.1.1})}
  \end{align*}
  This closes the induction.
\end{proof}

\begin{proof}{(b)}
  Let \(p = n - m\).
  By hypothesis we know that \(p \geq 0\).
  Now we use induction on \(p\) to show that \cref{ac 7.1.2}(b) is true.
  For \(p = 0\), we have \(n = m\) and
  \begin{align*}
    \prod_{j = m + k}^{m + k} a_{j - k} & = \Bigg(\prod_{j = m + k}^{m + k - 1} a_{j - k}\Bigg) \times a_{m + k - k} & \text{(by \cref{ac 7.1.1})} \\
                                        & = 1 \times a_{m + k - k}                                                   & \text{(by \cref{ac 7.1.1})} \\
                                        & = 1 \times a_m                                                                                           \\
                                        & = \Bigg(\prod_{i = m}^{m - 1} a_i\Bigg) \times a_m                         & \text{(by \cref{ac 7.1.1})} \\
                                        & = \prod_{i = m}^m a_i.                                                     & \text{(by \cref{ac 7.1.1})}
  \end{align*}
  So the base case holds.
  Suppose inductively that for some \(p \geq 0\) \cref{ac 7.1.2}(b) is true.
  Then for \(p + 1 = n - m\), we have \(p = n - m - 1\) and
  \begin{align*}
    \prod_{j = m + k}^{n + k} a_{j - k} & = \Bigg(\prod_{j = m + k}^{n + k - 1} a_{j - k}\Bigg) \times a_{n + k - k} & \text{(by \cref{ac 7.1.1})}      \\
                                        & = \Bigg(\prod_{j = m + k}^{n + k - 1} a_{j - k}\Bigg) \times a_n                                              \\
                                        & = \Bigg(\prod_{i = m}^{n - 1} a_i\Bigg) \times a_n                         & \text{(by induction hypothesis)} \\
                                        & = \prod_{i = m}^n a_i.                                                     & \text{(by \cref{ac 7.1.1})}
  \end{align*}
  This closes the induction.
\end{proof}

\begin{proof}{(c)}
  Let \(p = n - m\).
  By hypothesis we know that \(p \geq 0\).
  Now we use induction on \(p\) to show that \cref{ac 7.1.2}(c) is true.
  For \(p = 0\), we have \(n = m\) and
  \begin{align*}
    \prod_{i = m}^m (a_i \times b_i) & = \Bigg(\prod_{i = m}^{m - 1} (a_i \times b_i)\Bigg) \times a_m \times b_m                                 & \text{(by \cref{ac 7.1.1})} \\
                                     & = 1 \times a_m \times b_m                                                                                  & \text{(by \cref{ac 7.1.1})} \\
                                     & = \Bigg(\prod_{i = m}^{m - 1} a_i\Bigg) \times \Bigg(\prod_{i = m}^{m - 1} b_i\Bigg) \times a_m \times b_m & \text{(by \cref{ac 7.1.1})} \\
                                     & = \Bigg(\prod_{i = m}^m a_i\Bigg) \times \Bigg(\prod_{i = m}^m b_i\Bigg).                                  & \text{(by \cref{ac 7.1.1})}
  \end{align*}
  So the base case holds.
  Suppose inductively that for some \(p \geq 0\) \cref{ac 7.1.2}(c) is true.
  Then for \(p + 1 = n - m\), we have \(p = n - m - 1\) and
  \begin{align*}
    \prod_{i = m}^n (a_i \times b_i) & = \Bigg(\prod_{i = m}^{n - 1} (a_i \times b_i)\Bigg) \times a_n \times b_n                                 & \text{(by \cref{ac 7.1.1})}      \\
                                     & = \Bigg(\prod_{i = m}^{n - 1} a_i\Bigg) \times \Bigg(\prod_{i = m}^{n - 1} b_i\Bigg) \times a_n \times b_n & \text{(by induction hypothesis)} \\
                                     & = \Bigg(\prod_{i = m}^n a_i\Bigg) \times \Bigg(\prod_{i = m}^n b_i\Bigg).                                  & \text{(by \cref{ac 7.1.1})}
  \end{align*}
  This closes the induction.
\end{proof}

\begin{proof}{(d)}
  Let \(p = n - m\).
  By hypothesis we know that \(p \geq 0\).
  Now we use induction on \(p\) to show that \cref{ac 7.1.2}(d) is true.
  For \(p = 0\), we have \(n = m\) and
  \begin{align*}
    \prod_{i = m}^m ca_i & = \Bigg(\prod_{i = m}^{m - 1} ca_i\Bigg) \times ca_m & \text{(by \cref{ac 7.1.1})} \\
                         & = 1 \times ca_m                                      & \text{(by \cref{ac 7.1.1})} \\
                         & = c \times a_m                                                                     \\
                         & = c \Bigg(\prod_{i = m}^m a_i\Bigg)                  & \text{(by \cref{ac 7.1.1})} \\
                         & = c^{m - m + 1} \Bigg(\prod_{i = m}^m a_i\Bigg).
  \end{align*}
  So the base case holds.
  Suppose inductively that for some \(p \geq 0\) \cref{ac 7.1.2}(d) is true.
  Then for \(p + 1 = n - m\), we have \(p = n - m - 1\) and
  \begin{align*}
    \prod_{i = m}^n ca_i & = \Bigg(\prod_{i = m}^{n - 1} ca_i\Bigg) \times ca_n                         & \text{(by \cref{ac 7.1.1})}      \\
                         & = c^{n - 1 - m + 1} \Bigg(\prod_{i = m}^{n - 1} a_i\Bigg) \times ca_n        & \text{(by induction hypothesis)} \\
                         & = c^{n - m + 1} \Bigg(\Bigg(\prod_{i = m}^{n - 1} a_i\Bigg) \times a_n\Bigg)                                    \\
                         & = c^{n - m + 1} \Bigg(\prod_{i = m}^n a_i\Bigg).                             & \text{(by \cref{ac 7.1.1})}
  \end{align*}
  This closes the induction.
\end{proof}

\begin{proof}{(e)}
  Let \(p = n - m\).
  By hypothesis we know that \(p \geq 0\).
  Now we use induction on \(p\) to show that \cref{ac 7.1.2}(e) is true.
  For \(p = 0\), we have \(n = m\) and
  \begin{align*}
    \abs{\prod_{i = m}^m a_i} & = \abs{\Bigg(\prod_{i = m}^{m - 1} a_i\Bigg) \times a_m}       & \text{(by \cref{ac 7.1.1})} \\
                              & = \abs{1a_m}                                                   & \text{(by \cref{ac 7.1.1})} \\
                              & = \abs{1}\abs{a_m}                                                                           \\
                              & = \Bigg(\prod_{i = m}^{m - 1} \abs{a_i}\Bigg) \times \abs{a_m} & \text{(by \cref{ac 7.1.1})} \\
                              & = \prod_{i = m}^m \abs{a_i}.                                   & \text{(by \cref{ac 7.1.1})}
  \end{align*}
  So the base case holds.
  Suppose inductively that for some \(p \geq 0\) \cref{ac 7.1.2}(e) is true.
  Then for \(p + 1 = n - m\), we have \(p = n - m - 1\) and
  \begin{align*}
    \abs{\prod_{i = m}^n a_i} & = \abs{\Bigg(\prod_{i = m}^{n - 1} a_i\Bigg) \times a_n}       & \text{(by \cref{ac 7.1.1})}      \\
                              & = \abs{\prod_{i = m}^{n - 1} a_i} \times \abs{a_n}                                                \\
                              & = \Bigg(\prod_{i = m}^{n - 1} \abs{a_i}\Bigg) \times \abs{a_n} & \text{(by induction hypothesis)} \\
                              & = \prod_{i = m}^n \abs{a_i}.                                   & \text{(by \cref{ac 7.1.1})}
  \end{align*}
  This closes the induction.
\end{proof}

\begin{additional corollary}\label{ac 7.1.3}
Let \(X\) be a finite set with \(n\) elements (where \(n \in \N\)), and let \(f : X \to \R\) be a function from \(X\) to the real numbers
(i.e., \(f\) assigns a real number \(f(x)\) to each element \(x\) of \(X\)).
Then we can define the finite product \(\prod_{x \in X} f(x)\) as follows.
We first select any bijection \(g\) from \(\{i \in \N : 1 \leq i \leq n\}\) to \(X\);
such a bijection exists since \(X\) is assumed to have \(n\) elements.
We then define
\[
  \prod_{x \in X} f(x) \coloneqq \prod_{i = 1}^n f(g(i))
\]
\end{additional corollary}

\begin{additional corollary}[Finite products are well-defined]\label{ac 7.1.4}
Let \(X\) be a finite set with \(n\) elements (where \(n \in \N\)), let \(f : X \to \R\) be a function, and let \(g : \{i \in \N : 1 \leq i \leq n\} \to X\) and \(h : \{i \in \N : 1 \leq i \leq n\} \to X\) be bijections.
Then we have
\[
  \prod_{i = 1}^n f(g(i)) = \prod_{i = 1}^n f(h(i)).
\]
\end{additional corollary}

\begin{proof}
  Let \(P(n)\) be the assertion that ``For any set \(X\) of \(n\) elements, any function \(f : X \to \R\), and any two bijections \(g, h\) from \(\{i \in \N : 1 \leq i \leq n\}\) to \(X\), we have \(\prod_{i = 1}^n f(g(i)) = \prod_{i = 1}^n f(h(i))\)''.
  (More informally, \(P(n)\) is the assertion that \cref{ac 7.1.4} is true for that value of \(n\).)
  We use induction on \(n\);

  We first check the base case \(P(0)\).
  In this case \(\prod_{i = 1}^0 f(g(i))\) and \(\prod_{i = 1}^0 f(h(i))\) both equal to \(1\), by \cref{ac 7.1.1}, so we are done.

  Now suppose inductively that \(P(n)\) is true;
  we now prove that \(P(n + 1)\) is true.
  Thus, let \(X\) be a set with \(n + 1\) elements, let \(f : X \to \R\) be a function, and let \(g\) and \(h\) be bijections from \(\{i \in N : 1 \leq i \leq n + 1\}\) to \(X\).
  We have to prove that
  \[
    \prod_{i = 1}^{n + 1} f(g(i)) = \prod_{i = 1}^{n + 1} f(h(i)). \tag{ac 7.1}\label{eq ac 7.1}
  \]
  Let \(x \coloneqq g(n + 1)\);
  thus \(x\) is an element of \(X\).
  By \cref{ac 7.1.1}, we can expand the left-hand side of \eqref{eq ac 7.1} as
  \[
    \prod_{i = 1}^{n + 1} f(g(i)) = \Bigg(\prod_{i = 1}^n f(g(i))\Bigg) \times f(x).
  \]
  Now let us look at the right-hand side of \eqref{eq ac 7.1}.
  Since \(h\) is a bijection, we do know that there is \emph{some} index \(j\), with \(1 \leq j \leq n + 1\), for which \(h(j) = x\).
  We now use \cref{ac 7.1.1,ac 7.1.2} to write
  \begin{align*}
    \prod_{i = 1}^{n + 1} f(h(i)) & = \Bigg(\prod_{i = 1}^j f(h(i))\Bigg) \times \Bigg(\prod_{i = j + 1}^{n + 1} f(h(i))\Bigg)                      \\
                                  & = \Bigg(\prod_{i = 1}^{j - 1} f(h(i))\Bigg) \times f(h(j)) \times \Bigg(\prod_{i = j + 1}^{n + 1} f(h(i))\Bigg) \\
                                  & = \Bigg(\prod_{i = 1}^{j - 1} f(h(i))\Bigg) \times f(x) \times \Bigg(\prod_{i = j}^n f(h(i + 1))\Bigg).
  \end{align*}
  We now define the function \(\tilde{h} : \{i \in \N : 1 \leq i \leq n\} \to X - \{x\}\) by setting \(\tilde{h}(i) \coloneqq h(i)\) when \(i < j\) and \(\tilde{h}(i) \coloneqq h(i + 1)\) when \(i \geq j\).
  We can thus write the right-hand side of \eqref{eq ac 7.1} as
  \[
    = \Bigg(\prod_{i = 1}^{j - 1} f(\tilde{h}(i))\Bigg) \times f(x) \times \Bigg(\prod_{i = j}^n f(\tilde{h}(i))\Bigg) = \Bigg(\prod_{i = 1}^n f(\tilde{h}(i))\Bigg) \times f(x)
  \]
  where we have used \cref{ac 7.1.2} once again.
  Thus to finish the proof of \eqref{eq ac 7.1} we have to show that
  \[
    \prod_{i = 1}^n f(g(i)) = \prod_{i = 1}^n f(\tilde{h}(i)). \tag{ac 7.2}\label{eq ac 7.2}
  \]
  But the function \(g\) (when restricted to \(\{i \in \N : 1 \leq i \leq n\}\)) is a bijection from \(\{i \in \N : 1 \leq i \leq n\} \to X - \{x\}\).
  The function \(\tilde{h}\) is also a bijection from \(\{i \in \N : 1 \leq i \leq n\} \to X - \{x\}\) (cf. \cref{3.6.9}).
  Since \(X - \{x\}\) has \(n\) elements (by \cref{3.6.9}), the claim \eqref{eq ac 7.2} then follows directly from the induction hypothesis \(P(n)\).
\end{proof}

\begin{additional corollary}[Basic properties of product over finite sets]\label{ac 7.1.5}
\mbox{}
\begin{enumerate}
  \item If \(X\) is empty, and \(f : X \to \R\) is a function (i.e., \(f\) is the empty function), we have
        \[
          \prod_{x \in X} f(x) = 1.
        \]
  \item If \(X\) consists of a single element, \(X = \{x_0\}\), and \(f : X \to \R\) is a function, we have
        \[
          \prod_{x \in X} f(x) = f(x_0).
        \]
  \item (Substitution, part I) If \(X\) is a finite set, \(f : X \to \R\) is a function, and \(g : Y \to X\) is a bijection, then
        \[
          \prod_{x \in X} f(x) = \prod_{y \in Y} f(g(y)).
        \]
  \item (Substitution, part II) Let \(n \leq m\) be integers, and let \(X\) be the set \(X \coloneqq \{i \in \Z : n \leq i \leq m\}\).
        If \(a_i\) is a real number assigned to each integer \(i \in X\), then we have
        \[
          \prod_{i = n}^m a_i = \prod_{i \in X} a_i.
        \]
  \item Let \(X, Y\) be disjoint finite sets (so \(X \cap Y = \emptyset\)), and \(f : X \cup Y \to \R\) is a function.
        Then we have
        \[
          \prod_{z \in X \cup Y} f(z) = \Bigg(\prod_{x \in X} f(x)\Bigg) \times \Bigg(\prod_{y \in Y} f(y)\Bigg).
        \]
  \item Let \(X\) be a finite set, and let \(f : X \to \R\) and \(g : X \to \R\) be functions.
        Then
        \[
          \prod_{x \in X} (f(x) \times g(x)) = \prod_{x \in X} f(x) \times \prod_{x \in X} g(x).
        \]
  \item Let \(X\) be a finite set, let \(f : X \to \R\) be a function, and let \(c\) be a real number.
        Then
        \[
          \prod_{x \in X} cf(x) = c^{\#(X)} \prod_{x \in X} f(x).
        \]
  \item Let \(X\) be a finite set, and let \(f : X \to \R\) be a function, then
        \[
          \abs{\prod_{x \in X} f(x)} = \prod_{x \in X} \abs{f(x)}.
        \]
\end{enumerate}
\end{additional corollary}

\begin{proof}{(a)}
  Let \(g : \{i \in \N : 1 \leq i \leq 0\} \to \emptyset\) be a function.
  Then \(g\) is a bijection and
  \begin{align*}
    \prod_{x \in X} f(x) & = \prod_{i = 1}^0 f(g(i)) & \text{(by \cref{ac 7.1.3})} \\
                         & = 1.                      & \text{(by \cref{ac 7.1.1})}
  \end{align*}
\end{proof}

\begin{proof}{(b)}
  Let \(g : \{1\} \to \{x_0\}\) be a function.
  Then \(g\) is a bijection and
  \begin{align*}
    \prod_{x \in X} f(x) & = \prod_{i = 1}^1 f(g(i))                            & \text{(by \cref{ac 7.1.3})} \\
                         & = \bigg(\prod_{i = 1}^0 f(g(i))\bigg) \times f(g(1)) & \text{(by \cref{ac 7.1.1})} \\
                         & = 1 \times f(g(1))                                   & \text{(by \cref{ac 7.1.1})} \\
                         & = f(x_0).
  \end{align*}
\end{proof}

\begin{proof}{(c)}
  Let \(h : \{i \in \N : 1 \leq i \leq \#(Y)\} \to Y\) be a bijection.
  Since \(X\) is finite and \(g\) is a bijection between \(X\) and \(Y\), we know that \(Y\) is finite and thus such \(h\) is well-defined.
  Then we know that \(g \circ h : \{i \in \N : 1 \leq i \leq \#(Y)\} \to X\) is also a bijection and
  \begin{align*}
    \prod_{x \in X} f(x) & = \prod_{i = 1}^{\#(Y)} f((g \circ h)(i)) & \text{(by \cref{ac 7.1.3})} \\
                         & = \prod_{i = 1}^{\#(Y)} f(g(h(i)))                                      \\
                         & = \prod_{i = 1}^{\#(Y)} (f \circ g)(h(i))                               \\
                         & = \prod_{y \in Y} (f \circ g)(y)          & \text{(by \cref{ac 7.1.3})} \\
                         & = \prod_{y \in Y} f(g(y)).
  \end{align*}
\end{proof}

\begin{proof}{(d)}
  Let \(f : X \to \{a_i \in \R : n \leq i \leq m\}\) be a function where \(f = i \mapsto a_i\).
  Let \(g : \{i \in \N : 1 \leq i \leq m - n + 1\} \to X\) be a function where \(g = i \mapsto i + n - 1\).
  Then \(g\) is a bijection and
  \begin{align*}
    \prod_{i \in X} a_i & = \prod_{i \in X} f(i)                                                                             \\
                        & = \prod_{i = 1}^{m - n + 1} f(g(i))                               & \text{(by \cref{ac 7.1.3})}    \\
                        & = \prod_{i = 1}^{m - n + 1} f(i + n - 1)                                                           \\
                        & = \prod_{i = 1}^{m - n + 1} a_{i + n - 1}                                                          \\
                        & = \prod_{i = 1 + n - 1}^{m - n + 1 + n - 1} a_{i + n - 1 - n + 1} & \text{(by \cref{ac 7.1.2}(b))} \\
                        & = \prod_{i = n}^m a_i.
  \end{align*}
\end{proof}

\begin{proof}{(e)}
  Let \(g : \{i \in \N : 1 \leq i \leq \#(X)\} \to X\) and \(h : \{i \in \N : 1 \leq i \leq \#(Y)\} \to Y\) be bijections.
  Since \(X, Y\) are finite, we know that \(g, h\) are well-defined and \(X \cup Y\) is finite.
  Let \(k : \{i \in \N : 1 \leq i \leq \#(X \cup Y)\} \to X \cup Y\) be a bijection where
  \[
    k(i) = \begin{cases}
      g(i)         & \text{if } 1 \leq i \leq \#(X)                  \\
      h(i - \#(X)) & \text{if } \#(X) + 1 \leq i \leq \#(X) + \#(Y).
    \end{cases}
  \]
  Since \(X \cup Y\) is finite, we know that \(k\) is well-defined and \(\#(X \cup Y) = \#(X) + \#(Y)\).
  Then we have
  \begin{align*}
     & \prod_{z \in X \cup Y} f(z)                                                                                                  \\
     & = \prod_{i = 1}^{\#(X \cup Y)} f(k(i))                                                      & \text{(by \cref{ac 7.1.3})}    \\
     & = \prod_{i = 1}^{\#(X)} f(k(i)) \times \prod_{i = \#(X) + 1}^{\#(X \cup Y)} f(k(i))         & \text{(by \cref{ac 7.1.2}(a))} \\
     & = \prod_{i = 1}^{\#(X)} f(g(i)) \times \prod_{i = \#(X) + 1}^{\#(X \cup Y)} f(h(i - \#(X)))                                  \\
     & = \prod_{i = 1}^{\#(X)} f(g(i)) \times \prod_{i = 1}^{\#(Y)} f(h(i))                        & \text{(by \cref{ac 7.1.2}(b))} \\
     & = \prod_{x \in X} f(x) \times \prod_{y \in Y} f(y).                                         & \text{(by \cref{ac 7.1.3})}
  \end{align*}
\end{proof}

\begin{proof}{(f)}
  Let \(h : \{i \in \N : 1 \leq i \leq \#(X)\} \to X\) be a bijection.
  Since \(X\) is finite, we know that \(h\) is well-defined and
  \begin{align*}
     & \prod_{x \in X} (f(x) \times g(x))                                                                    \\
     & = \prod_{x \in X} (f \times g)(x)                                                                     \\
     & = \prod_{i = 1}^{\#(X)} (f \times g)(h(i))                           & \text{(by \cref{ac 7.1.3})}    \\
     & = \prod_{i = 1}^{\#(X)} (f(h(i)) \times g(h(i)))                                                      \\
     & = \prod_{i = 1}^{\#(X)} f(h(i)) \times \prod_{i = 1}^{\#(X)} g(h(i)) & \text{(by \cref{ac 7.1.2}(c))} \\
     & = \prod_{x \in X} f(x) \times \prod_{x \in X} g(x).                  & \text{(by \cref{ac 7.1.3})}
  \end{align*}
\end{proof}

\begin{proof}{(g)}
  Let \(g : \{i \in \N : 1 \leq i \leq \#(X)\} \to X\) be a bijection.
  Since \(X\) is finite, we know that \(g\) is well-defined and
  \begin{align*}
    \prod_{x \in X} cf(x) & = \prod_{x \in X} (cf)(x)                                                  \\
                          & = \prod_{i = 1}^{\#(X)} (cf)(g(i))        & \text{(by \cref{ac 7.1.3})}    \\
                          & = \prod_{i = 1}^{\#(X)} cf(g(i))                                           \\
                          & = c^{\#(X)} \prod_{i = 1}^{\#(X)} f(g(i)) & \text{(by \cref{ac 7.1.2}(d))} \\
                          & = c^{\#(X)} \prod_{x \in X} f(x).         & \text{(by \cref{ac 7.1.3})}
  \end{align*}
\end{proof}

\begin{proof}{(h)}
  Let \(g : \{i \in \N : 1 \leq i \leq \#(X)\} \to X\) be a bijection.
  Since \(X\) is finite, we know that \(g\) is well-defined and
  \begin{align*}
    \abs{\prod_{x \in X} f(x)} & = \abs{\prod_{i = 1}^{\#(X)} f(g(i))} & \text{(by \cref{ac 7.1.3})}    \\
                               & = \prod_{i = 1}^{\#(X)} \abs{f(g(i))} & \text{(by \cref{ac 7.1.2}(e))} \\
                               & = \prod_{x \in X} \abs{f(x)}.         & \text{(by \cref{ac 7.1.3})}
  \end{align*}
\end{proof}

\begin{additional corollary}\label{ac 7.1.6}
Let \(X, Y\) be finite sets, and let \(f : X \times Y \to \R\) be a function.
Then
\[
  \prod_{x \in X} \bigg(\prod_{y \in Y} f(x, y)\bigg) = \prod_{(x, y) \in X \times Y} f(x, y).
\]
\end{additional corollary}

\begin{proof}
  Let \(n\) be the number of elements in \(X\).
  We will use induction on \(n\) (cf. \cref{ac 7.1.4});
  i.e., we let \(P(n)\) be the assertion that \cref{ac 7.1.6} is true for any set \(X\) with \(n\) elements, and any finite set \(Y\) and any function \(f : X \times Y \to \R\).
  We wish to prove \(P(n)\) for all natural numbers \(n\).

  The base case \(P(0)\) is easy, following from \cref{ac 7.1.5}(a).
  Now suppose that \(P(n)\) is true;
  we now show that \(P(n + 1)\) is true.
  Let \(X\) be a set with \(n + 1\) elements.
  In particular, by \cref{3.6.9}, we can write \(X = X' \cup \{x_0\}\), where \(x_0\) is an element of \(X\) and \(X' \coloneqq X - \{x_0\}\) has \(n\) elements.
  Then by \cref{7.1.5}(e) we have
  \[
    \prod_{x \in X} \bigg(\prod_{y \in Y} f(x, y)\bigg) = \prod_{x \in X'} \bigg(\prod_{y \in Y} f(x, y)\bigg) \times \bigg(\prod_{y \in Y} f(x_0, y)\bigg);
  \]
  by the induction hypothesis this is equal to
  \[
    \prod_{(x, y) \in X' \times Y} f(x, y) \times \bigg(\prod_{y \in Y} f(x_0, y)\bigg).
  \]
  By \cref{7.1.11}(c) this is equal to
  \[
    \prod_{(x, y) \in X' \times Y} f(x, y) \times \bigg(\prod_{(x, y) \in \{x_0\} \times Y} f(x, y)\bigg).
  \]
  By \cref{7.1.11}(e) this is equal to
  \[
    \prod_{(x, y) \in X \times Y} f(x, y)
  \]
  as desired.
\end{proof}

\begin{additional corollary}\label{ac 7.1.7}
Let \(X, Y\) be finite sets, and let \(f : X \times Y \to \R\) be a function.
Then
\begin{align*}
  \prod_{x \in X} \bigg(\prod_{y \in Y} f(x, y)\bigg) & = \prod_{(x, y) \in X \times Y} f(x, y)                \\
                                                      & = \prod_{(y, x) \in Y \times X} f(x, y)                \\
                                                      & = \prod_{y \in Y} \bigg(\prod_{x \in X} f(x, y)\bigg).
\end{align*}
\end{additional corollary}

\begin{proof}
  In light of \cref{ac 7.1.6}, it suffices to show that
  \[
    \prod_{(x, y) \in X \times Y} f(x, y) = \prod_{(y, x) \in Y \times X} f(x, y).
  \]
  But this follows from \cref{ac 7.1.5}(c) by applying the bijection \(h : Y \times X \to X \times Y\) defined by \(h(y, x) \coloneqq (x, y)\).
\end{proof}

\exercisesection

\begin{exercise}\label{ex 7.1.1}
  Prove \cref{7.1.4}.
\end{exercise}

\begin{proof}
  See \cref{7.1.4}.
\end{proof}

\begin{exercise}\label{ex 7.1.2}
  Prove \cref{7.1.11}.
\end{exercise}

\begin{proof}
  See \cref{7.1.11}.
\end{proof}

\begin{exercise}\label{ex 7.1.3}
  Form a definition for the finite products \(\prod_{i = 1}^n a_i\) and \(\prod_{x \in X} f(x)\).
  Which of the above result for finite series have analoges for finite products?
\end{exercise}

\begin{proof}
  See \crefrange{ac 7.1.1}{ac 7.1.7}.
\end{proof}

\begin{exercise}\label{ex 7.1.4}
  Define the \emph{factorial function} \(n!\) for natural numbers \(n\) by the recursive definition \(0! \coloneqq 1\) and \((n + 1)! \coloneqq n! \times (n + 1)\).
  If \(x\) and \(y\) are real numbers, prove the \emph{binomial formula}
  \[
    (x + y)^n = \sum_{j = 0}^n \frac{n!}{j!(n - j)!} x^j y^{n - j}
  \]
  for all natural numbers \(n\).
\end{exercise}

\begin{proof}
  We use induction on \(n\).
  For \(n = 0\), we have
  \begin{align*}
    (x + y)^0 & = 1                                                                                                                      \\
              & = \frac{0!}{0!(0 - 0)!} x^0 y^{0 - 0}                                                         & \text{(by definition)}   \\
              & = \sum_{j = 0}^{-1} \frac{0!}{j!(0 - j)!} x^j y^{0 - j} + \frac{0!}{0!(0 - 0)!} x^0 y^{0 - 0} & \text{(by \cref{7.1.1})} \\
              & = \sum_{j = 0}^0 \frac{0!}{j!(0 - j)!} x^j y^{0 - j}                                          & \text{(by \cref{7.1.1})}
  \end{align*}
  So the base case holds.
  Suppose inductively that for some \(n \geq 0\) the statement holds.
  Then for \(n + 1\), we have
  \begin{align*}
    (x + y)^{n + 1} & = (x + y)^n \times (x + y)                                                                                                    \\
                    & = \bigg(\sum_{j = 0}^n \frac{n!}{j!(n - j)!} x^j y^{n - j}\bigg) \times (x + y)            & \text{(by induction hypothesis)} \\
                    & = \bigg(\sum_{j = 0}^n \frac{n!}{j!(n - j)!} x^{j + 1} y^{n - j}\bigg)                                                        \\
                    & \quad + \bigg(\sum_{j = 0}^n \frac{n!}{j!(n - j)!} x^j y^{n + 1 - j}\bigg)                                                    \\
                    & = \bigg(\sum_{j = 0}^{n - 1} \frac{n!}{j!(n - j)!} x^{j + 1} y^{n - j}\bigg)               & \text{(by \cref{7.1.1})}         \\
                    & \quad + \bigg(\frac{n!}{n!0!} x^{n + 1} y^0\bigg)                                                                             \\
                    & \quad + \bigg(\sum_{j = 1}^n \frac{n!}{j!(n - j)!} x^j y^{n + 1 - j}\bigg)                                                    \\
                    & \quad + \bigg(\frac{n!}{0!n!} x^0 y^{n + 1}\bigg)                                                                             \\
                    & = \bigg(\sum_{j = 0}^{n - 1} \frac{n!}{j!(n - j)!} x^{j + 1} y^{n - j}\bigg) + x^{n + 1}   & \text{(by definition)}           \\
                    & \quad + \bigg(\sum_{j = 1}^n \frac{n!}{j!(n - j)!} x^j y^{n + 1 - j}\bigg) + y^{n + 1}                                        \\
                    & = \bigg(\sum_{j = 1}^n \frac{n!}{(j - 1)!(n + 1 - j)!} x^j y^{n + 1 - j}\bigg) + x^{n + 1} & \text{(by \cref{7.1.4}(b))}      \\
                    & \quad + \bigg(\sum_{j = 1}^n \frac{n!}{j!(n - j)!} x^j y^{n + 1 - j}\bigg) + y^{n + 1}
  \end{align*}
  and
  \begin{align*}
     & \bigg(\sum_{j = 1}^n \frac{n!}{(j - 1)!(n + 1 - j)!} x^j y^{n + 1 - j}\bigg)                                                                                                     \\
     & \quad + \bigg(\sum_{j = 1}^n \frac{n!}{j!(n - j)!} x^j y^{n + 1 - j}\bigg)                                                                                                       \\
     & = \sum_{j = 1}^n \bigg(\frac{n!}{(j - 1)!(n + 1 - j)!} x^j y^{n + 1 - j} + \frac{n!}{j!(n - j)!} x^j y^{n + 1 - j}\bigg)                           & \text{(by \cref{7.1.4}(c))} \\
     & = \sum_{j = 1}^n \bigg(\frac{j \times n!}{j!(n + 1 - j)!} x^j y^{n + 1 - j} + \frac{(n + 1 - j) \times n!}{j!(n + 1 - j)!} x^j y^{n + 1 - j}\bigg)                               \\
     & = \sum_{j = 1}^n \bigg(\frac{j \times n! + (n + 1 - j) \times n!}{j!(n + 1 - j)!} x^j y^{n + 1 - j}\bigg)                                                                        \\
     & = \sum_{j = 1}^n \bigg(\frac{(n + 1)!}{j!(n + 1 - j)!} x^j y^{n + 1 - j}\bigg).
  \end{align*}
  We also have
  \begin{align*}
     & \sum_{j = 0}^{n + 1} \frac{(n + 1)!}{j!(n + 1 - j)!} x^j y^{n + 1 - j}                                                                                    \\
     & = \frac{(n + 1)!}{(n + 1)! 0!} x^{n + 1} y^0 + \bigg(\sum_{j = 0}^n \frac{(n + 1)!}{j!(n + 1 - j)!} x^j y^{n + 1 - j}\bigg) & \text{(by \cref{7.1.1})}    \\
     & = x^{n + 1} + \bigg(\sum_{j = 0}^n \frac{(n + 1)!}{j!(n + 1 - j)!} x^j y^{n + 1 - j}\bigg)                                  & \text{(by definition)}      \\
     & = x^{n + 1} + \bigg(\sum_{j = 0}^0 \frac{(n + 1)!}{j!(n + 1 - j)!} x^j y^{n + 1 - j}\bigg)                                  & \text{(by \cref{7.1.4}(a))} \\
     & \quad + \bigg(\sum_{j = 1}^n \frac{(n + 1)!}{j!(n + 1 - j)!} x^j y^{n + 1 - j}\bigg)                                                                      \\
     & = x^{n + 1} + \frac{(n + 1)!}{0! (n + 1)!} x^0 y^{n + 1}                                                                    & \text{(by \cref{7.1.1})}    \\
     & \quad + \bigg(\sum_{j = 1}^n \frac{(n + 1)!}{j!(n + 1 - j)!} x^j y^{n + 1 - j}\bigg)                                                                      \\
     & = x^{n + 1} + y^{n + 1} + \bigg(\sum_{j = 1}^n \frac{(n + 1)!}{j!(n + 1 - j)!} x^j y^{n + 1 - j}\bigg).                     & \text{(by definition)}
  \end{align*}
  Thus we have
  \[
    (x + y)^{n + 1} = \sum_{j = 0}^{n + 1} \frac{(n + 1)!}{j!(n + 1 - j)!} x^j y^{n + 1 - j}.
  \]
  and this closes the induction.
\end{proof}

\begin{exercise}\label{ex 7.1.5}
  Let \(X\) be a finite set, let \(m\) be an integer, and for each \(x \in X\) let \((a_n(x))_{n = m}^\infty\) be a convergent sequence of real numbers.
  Show that the sequence \((\sum_{x \in X} a_n(x))_{n = m}^\infty\) is convergent, and
  \[
    \lim_{n \to \infty} \sum_{x \in X} a_n(x) = \sum_{x \in X} \lim_{n \to \infty} a_n(x).
  \]
  Thus we may always interchange finite sums with convergent limits.
  Things however get trickier with infinite sums.
\end{exercise}

\begin{proof}
  Let \(k = \#(X)\).
  We use induction on \(k\).
  For \(k = 0\), we have \(X = \emptyset\).
  So
  \begin{align*}
    \lim_{n \to \infty} \sum_{x \in X} a_n(x) & = \lim_{n \to \infty} 0                      & \text{(by \cref{7.1.11})} \\
                                              & = 0                                                                      \\
                                              & = \sum_{x \in X} \lim_{n \to \infty} a_n(x). & \text{(by \cref{7.1.11})}
  \end{align*}
  Thus the base case holds.
  Suppose inductively that for some \(k \geq 0\) the statement is true.
  Then for \(k + 1\), we have to show that the statement is also true.
  Let \(x_0 \in X\) and \(X' = X \setminus \{x_0\}\).
  So \(\#(X') = \#(X) - 1 = n\), and we have
  \begin{align*}
     & \lim_{n \to \infty} \sum_{x \in X} a_n(x)                                                                                                                 \\
     & = \lim_{n \to \infty} \sum_{x \in \{x_0\} \cup X'} a_n(x)                                                                                                 \\
     & = \lim_{n \to \infty} \bigg(\sum_{x \in \{x_0\}} a_n(x) + \sum_{x \in X'} a_n(x)\bigg)                                 & \text{(by \cref{7.1.11}(e))}     \\
     & = \bigg(\lim_{n \to \infty} \sum_{x \in \{x_0\}} a_n(x)\bigg) + \bigg(\lim_{n \to \infty} \sum_{x \in X'} a_n(x)\bigg) & \text{(by \cref{6.1.19}(a))}     \\
     & = \bigg(\lim_{n \to \infty} a_n(x_0)\bigg) + \bigg(\lim_{n \to \infty} \sum_{x \in X'} a_n(x)\bigg)                    & \text{(by \cref{7.1.11}(b))}     \\
     & = \bigg(\sum_{x \in \{x_0\}} \lim_{n \to \infty} a_n(x)\bigg) + \bigg(\lim_{n \to \infty} \sum_{x \in X'} a_n(x)\bigg) & \text{(by \cref{7.1.11}(b))}     \\
     & = \bigg(\sum_{x \in \{x_0\}} \lim_{n \to \infty} a_n(x)\bigg) + \bigg(\sum_{x \in X'} \lim_{n \to \infty} a_n(x)\bigg) & \text{(by induction hypothesis)} \\
     & = \bigg(\sum_{x \in \{x_0\} \cup X'} \lim_{n \to \infty} a_n(x)\bigg)                                                  & \text{(by \cref{7.1.11}(e))}     \\
     & = \sum_{x \in X} \lim_{n \to \infty} a_n(x).
  \end{align*}
  This closes the induction.
\end{proof}
\section{Infinite series}\label{sec 7.2}

\begin{definition}[Formal infinite series]\label{7.2.1}
A (formal) infinite series is any expression of the form
\[
    \sum_{n = m}^\infty a_n,
\]
where \(m\) is an integer, and \(a_n\) is a real number for any integer \(n \geq m\).
\end{definition}

\begin{note}
We sometimes write this series as
\[
    a_m + a_{m + 1} + a_{m + 2} + \dots.
\]
\end{note}

\begin{note}
At present, this series is only defined formally;
we have not set this sum equal to any real number;
the notation \(a_m + a_{m + 1} + a_{m + 2} + \dots\) is of course designed to look very suggestively like a sum, but is not actually a finite sum because of the ``\(\dots\)'' symbol.
To rigorously define what the series actually sums to, we need another definition.
\end{note}

\begin{definition}[Convergence of series]\label{7.2.2}
Let \(\sum_{n = m}^\infty a_n\) be a formal infinite series.
For any integer \(N \geq m\), we define the \emph{\(N^{\text{th}}\) partial sum} \(S_N\) of this series to be \(S_N \coloneqq \sum_{n = m}^N a_n\);
of course, \(S_N\) is a real number.
If the sequence \((S_N)_{N = m}^\infty\) converges to some limit \(L\) as \(N \to \infty\), then we say that the infinite series \(\sum_{n = m}^\infty a_n\) is \emph{convergent}, and \emph{converges to \(L\)};
we also write \(L = \sum_{n = m}^\infty a_n\), and say that \(L\) is the \emph{sum} of the infinite series \(\sum_{n = m}^\infty a_n\).
If the partial sums \(S_N\) diverge, then we say that the infinite series \(\sum_{n = m}^\infty a_n\) is \emph{divergent}, and we do not assign any real number value to that series.
\end{definition}

\begin{remark}\label{7.2.3}
Note that Proposition \ref{6.1.7} shows that if a series converges, then it has a unique sum, so it is safe to talk about \emph{the} sum \(L = \sum_{n = m}^\infty a_n\) of a convergent series.
\end{remark}

\setcounter{theorem}{4}
\begin{proposition}\label{7.2.5}
Let \(\sum_{n = m}^\infty a_n\) be a formal series of real numbers.
Then \(\sum_{n = m}^\infty a_n\) converges if and only if, for every real number \(\varepsilon > 0\), there exists an integer \(N \geq m\) such that
\[
    \abs*{\sum_{n = p + 1}^q a_n} \leq \varepsilon \text{ for all } p, q \geq N.
\]
\end{proposition}

\begin{proof}
We first show that if \(\sum_{n = m}^\infty a_n\) converges, then \(\forall\ \varepsilon \in \mathbf{R}\) and \(\varepsilon > 0\), \(\exists\ N \in \mathbf{N}\) and \(N \geq m\) such that \(\abs*{\sum_{n = p + 1}^q a_n} \leq \varepsilon \ \forall\ p, q \in \mathbf{N}\) and \(p, q \geq N\).
Let \(k \in \mathbf{N}\) and let \(S_k = \sum_{n = m}^k a_n\) be the \(k^{\text{th}}\) partial sum of \((a_n)_{n = m}^\infty\).
Since \((S_k)_{k = m}^\infty\) converges, by Theorem \ref{6.4.18} \((S_k)_{k = m}^\infty\) is a Cauchy sequence.
So we have \(\forall\ \varepsilon > 0\), \(\exists\ N \geq m\) such that \(\abs*{S_q - S_p} \leq \varepsilon \ \forall\ p, q \geq N\).
We now divide into two cases:
\begin{enumerate}
    \item If \(p > q\), then \(p + 1 > q\).
    By Definition \ref{7.1.1} \(\abs*{\sum_{n = p + 1}^q a_n} = \abs*{0} = 0 \leq \varepsilon\).
    \item If \(p \leq q\), then
    \begin{align*}
    & \abs*{S_q - S_p} \leq \varepsilon \\
    \implies & \abs*{\bigg(\sum_{n = m}^q a_n\bigg) - \bigg(\sum_{n = m}^p a_n\bigg)} \leq \varepsilon \\
    \implies & \abs*{\bigg(\sum_{n = m}^p a_n\bigg) + \bigg(\sum_{n = p + 1}^q a_n\bigg) - \bigg(\sum_{n = m}^p a_n\bigg)} \leq \varepsilon & \text{(by Lemma \ref{7.1.4})} \\
    \implies & \abs*{\sum_{n = p + 1}^q a_n} \leq \varepsilon.
    \end{align*}
\end{enumerate}
From all cases above we have \(\abs*{\sum_{n = p + 1}^q a_n} \leq \varepsilon \ \forall\ p, q \geq N\).
Thus we conclude that if \(\sum_{n = m}^\infty a_n\) converges, then \(\forall\ \varepsilon \in \mathbf{R}\) and \(\varepsilon > 0\), \(\exists\ N \in \mathbf{N}\) and \(N \geq m\) such that \(\abs*{\sum_{n = p + 1}^q a_n} \leq \varepsilon \ \forall\ p, q \in \mathbf{N}\) and \(p, q \geq N\).

Now we show that if \(\forall\ \varepsilon \in \mathbf{R}\) and \(\varepsilon > 0\), \(\exists\ N \in \mathbf{N}\) and \(N \geq m\) such that \(\abs*{\sum_{n = p + 1}^q a_n \leq \varepsilon} \ \forall\ p, q \in \mathbf{N}\) and \(p, q \geq N\), then \(\sum_{n = m}^\infty a_n\) converges.
Since
\[
    \abs*{\sum_{n = p + 1}^q a_n} \leq \varepsilon \ \forall\ p, q \geq N,
\]
we can choose some \(p < q\) such that
\begin{align*}
& \abs*{\sum_{n = p + 1}^q a_n} \leq \varepsilon \\
\implies & \abs*{\sum_{n = p + 1}^q a_n + \sum_{n = m}^p a_n - \sum_{n = m}^p a_n} \leq \varepsilon \\
\implies & \abs*{\sum_{n = m}^q a_n - \sum_{n = m}^p a_n} \leq \varepsilon & \text{(by Lemma \ref{7.1.4})} \\
\implies & \abs*{S_q - S_p} \leq \varepsilon.
\end{align*}
This means \((S_k)_{k = m}^\infty\) is a Cauchy Sequence, so by Theorem \ref{6.4.18} \((S_k)_{k = m}^\infty\) converges, and therefore \(\sum_{n = m}^\infty a_n\) converges.
Thus we conclude that \(\sum_{n = m}^\infty a_n\) converges if and only if \(\forall\ \varepsilon \in \mathbf{R}\) and \(\varepsilon > 0\), \(\exists\ N \in \mathbf{N}\) and \(N \geq m\) such that \(\abs*{\sum_{n = p + 1}^q a_n} \leq \varepsilon \ \forall\ p, q \in \mathbf{N}\) and \(p, q \geq N\).
\end{proof}

\begin{corollary}[Zero test]\label{7.2.6}
Let \(\sum_{n = m}^\infty a_n\) be a convergent series of real numbers.
Then we must have \(\lim_{n \to \infty} a_n = 0\).
To put this another way, if \(\lim_{n \to \infty} a_n\) is non-zero or divergent, then the series \(\sum_{n = m}^\infty a_n\) is divergent.
\end{corollary}

\begin{proof}
\begin{align*}
& \sum_{n = m}^\infty a_n \text{ converges} \\
\implies & \forall\ \varepsilon \in \mathbf{R} \land \varepsilon > 0, \exists\ N \in \mathbf{N} \land N \geq m : \\
& \abs*{\sum_{n = p + 1}^q a_n} \leq \varepsilon \ \forall\ p, q \in \mathbf{N} \land p, q \geq N & \text{(by Proposition \ref{7.2.5})} \\
\implies & \forall\ \varepsilon > 0, \ \exists\ N \geq m : \abs*{\sum_{n = p + 1}^{p + 2} a_n} \leq \varepsilon \ \forall\ p \geq N \\
\implies & \forall\ \varepsilon > 0, \ \exists\ N \geq m : \abs*{a_{p + 1}} \leq \varepsilon \ \forall\ p \geq N & \text{(by Lemma \ref{7.1.4})} \\
\implies & \forall\ \varepsilon > 0, \ \exists\ N \geq m : \abs*{a_{p + 1} - 0} \leq \varepsilon \ \forall\ p \geq N \\
\implies & \lim_{n \to \infty} a_n = 0. & \text{(by Definition \ref{6.1.8})}
\end{align*}
\end{proof}

\begin{note}
If a sequence \((a_n)_{n = m}^\infty\) \emph{does} converge to zero, then the series \(\sum_{n = m}^\infty a_n\) may or may not be convergent;
it depends on the series.
\end{note}

\setcounter{theorem}{7}
\begin{definition}[Absolute convergence]\label{7.2.8}
Let \(\sum_{n = m}^\infty a_n\) be a formal series of real numbers.
We say that this series is \emph{absolutely convergent} iff the series \(\sum_{n = m}^\infty \abs*{a_n}\) is convergent.
\end{definition}

\begin{note}
In order to distinguish convergence from absolute convergence, we sometimes refer to the former as \emph{conditional convergence}.
\end{note}

\begin{proposition}[Absolute convergence test]\label{7.2.9}
Let \(\sum_{n = m}^\infty a_n\) be a formal series of real numbers.
If this series is absolutely convergent, then it is also conditionally convergent.
Furthermore, in this case we have the triangle inequality
\[
    \abs*{\sum_{n = m}^\infty a_n} \leq \sum_{n = m}^\infty \abs*{a_n}.
\]
\end{proposition}

\begin{proof}
We first show that if \(\sum_{n = m}^\infty a_n\) is absolutely convergent, then it is also conditionally convergent.
\begin{align*}
& \sum_{n = m}^\infty \abs*{a_n} \text{ converge} \\
\implies & \forall\ \varepsilon \in \mathbf{R} \land \varepsilon > 0, \exists\ N \in \mathbf{N} \land N \geq m : \\
& \abs*{\sum_{n = p + 1}^q \abs*{a_n}} \leq \varepsilon\ \forall\ p, q \in \mathbf{N} \land p, q \geq N & \text{(by Proposition \ref{7.2.5})} \\
\implies & \forall\ \varepsilon > 0, \ \exists\ N \geq m : \\
& \sum_{n = p + 1}^q \abs*{a_n} \leq \varepsilon\ \forall\ p, q \geq N \\
\implies & \forall\ \varepsilon > 0, \ \exists\ N \geq m : \\
& \abs*{\sum_{n = p + 1}^q a_n} \leq \sum_{n = p + 1}^q \abs*{a_n} \leq \varepsilon\ \forall\ p, q \geq N & \text{(by Lemma \ref{7.1.4})} \\
\implies & \sum_{n = m}^\infty a_n \text{ converge}. & \text{(by Proposition \ref{7.2.5})}
\end{align*}

Now we show that the triangle inequality is true.
Let \(N \in \mathbf{N} \land N \geq m\).
Let \((S_N)_{N = m}^\infty\) be a sequence where \(S_N = \abs*{\sum_{n = m}^N a_n}\).
Let \((T_N)_{N = m}^\infty\) be a sequence where \(T_N = \sum_{n = m}^N \abs*{a_n}\).
Since \(\lim_{N \to \infty} T_N\) exists, from proof above we know that \(\lim_{N \to \infty} S_N\) also exists.
So
\begin{align*}
& \abs*{\sum_{n = m}^N a_n} \leq \sum_{n = m}^N \abs*{a_n} \ \forall\ N \geq m & \text{(by Lemma \ref{7.1.4})} \\
\implies & S_N \leq T_N \ \forall\ N \geq m \\
\implies & \lim_{N \to \infty} S_N \leq \lim_{N \to \infty} T_N & \text{(by Lemma \ref{6.4.13})} \\
\implies & \abs*{\sum_{n = m}^\infty a_n} \leq \sum_{n = m}^\infty \abs*{a_n}. & \text{(by Definition \ref{7.2.2})}
\end{align*}
\end{proof}

\begin{remark}\label{7.2.10}
The converse to this proposition is not true;
there exist series which are conditionally convergent but not absolutely convergent.
\end{remark}

\begin{remark}\label{7.2.11}
We consider the class of conditionally convergent series to include the class of absolutely convergent series as a subclass.
Thus when we say a statement such as ``\(\sum_{n = m}^\infty a_n\) is conditionally convergent'', this does not automatically mean that \(\sum_{n = m}^\infty a_n\) is not absolutely convergent.
If we wish to say that a series is conditionally convergent but not absolutely convergent, then we will instead use a phrasing such as ``\(\sum_{n = m}^\infty a_n\) is \emph{only} conditionally convergent'', or ``\(\sum_{n = m}^\infty a_n\) converges conditionally, but not absolutely''.
\end{remark}

\begin{proposition}[Alternating series test]\label{7.2.12}
Let \((a_n)_{n = m}^\infty\) be a sequence of real numbers which are non-negative and decreasing, thus \(a_n \geq 0\) and \(a_n \geq a_{n + 1}\) for every \(n \geq m\).
Then the series \(\sum_{n = m}^\infty (-1)^n a_n\) is convergent if and only if the sequence \(a_n\) converges to \(0\) as \(n \to \infty\).
\end{proposition}

\begin{proof}
From the zero test (Corollary \ref{7.2.6}), we know that if \(\sum_{n = m}^\infty (-1)^n a_n\) is a convergent series, then the sequence \(((-1)^n a_n)_{n = m}^\infty\) converges to \(0\), which implies that \(a_n\) also converges to \(0\), since \((-1)^n a_n\) and \(a_n\) have the same distance from \(0\).
\begin{align*}
& \forall\ \varepsilon \in \mathbf{R} \land \varepsilon > 0, \exists\ N \in \mathbf{N} \land N \geq m : \\
& \abs*{a_n - 0} = \abs*{a_n} = \abs*{(-1)^n a_n} = \abs*{(-1)^n a_n - 0} \leq \varepsilon \ \forall\ n \geq N
\end{align*}

Now suppose conversely that \(a_n\) converges to \(0\).
For each \(N\), let \(S_N\) be the partial sum \(S_N \coloneqq \sum_{n = m}^N (-1)^n a_n\);
our job is to show that \(S_N\) converges.
Observe that
\begin{align*}
S_{N + 2} &= S_N + (-1)^{N + 1} a_{N + 1} + (-1)^{N + 2} a_{N + 2} \\
&= S_N + (-1)^{N + 1} (a_{N + 1} - a_{N + 2}).
\end{align*}
But by hypothesis, \((a_{N + 1} - a_{N + 2})\) is non-negative.
Thus we have \(S_{N + 2} \geq S_N\) when \(N\) is odd and \(S_{N + 2} \leq S_N\) if \(N\) is even.

Now suppose that \(N\) is even.
From the above discussion and induction we see that \(S_{N + 2k} \leq S_N\) for all natural numbers \(k\).
Also we have \(S_{N + 2k + 1} \geq S_{N + 1} = S_N - a_{N + 1}\).
Finally, we have \(S_{N + 2k + 1} = S_{N + 2k} - a_{N + 2k + 1} \leq S_{N + 2k}\).
Thus we have
\[
    S_N - a_{N + 1} \leq S_{N + 2k + 1} \leq S_{N + 2k} \leq S_N
\]
for all \(k\).
In particular, we have
\[
    S_N - a_{N + 1} \leq S_n \leq S_N \ \forall\ n \geq N.
\]
In particular, the sequence \((S_n)_{n = m}^\infty\) is eventually \(a_{N + 1}\)-steady.
But the sequence \((a_N)_{N = m}^\infty\) converges to \(0\) as \(N \to \infty\), thus this implies that \((S_n)_{n = m}^\infty\) is eventually \(\varepsilon\)-steady for every \(\varepsilon > 0\).
Thus \((S_n)_{n = m}^\infty\) converges, and so the series \(\sum_{n = m}^\infty (-1)^n a_n\) is convergent.
\end{proof}

\begin{note}
Lack of absolute convergence does not imply lack of conditional convergence, even though absolute convergence implies conditional convergence.
\end{note}

\setcounter{theorem}{13}
\begin{proposition}[Series law]\label{6.2.14}
    \mbox{}
    \begin{enumerate}
        \item If \(\sum_{n = m}^\infty a_n\) is a series of real numbers converging to \(x\), and \(\sum_{n = m}^\infty b_n\) is a series of real numbers converging to \(y\), then \(\sum_{n = m}^\infty (a_n + b_n)\) is also a convergent series, and converges to \(x + y\).
        In particular, we have
        \[
            \sum_{n = m}^\infty (a_n + b_n) = \sum_{n = m}^\infty a_n + \sum_{n = m}^\infty b_n.
        \]
        \item If \(\sum_{n = m}^\infty a_n\) is a series of real numbers converging to \(x\), and \(c\) is a real number, then \(\sum_{n = m}^\infty (c a_n)\) is also a convergent series, and converges to \(cx\).
        In particular, we have
        \[
            \sum_{n = m}^\infty (c a_n) = c \sum_{n = m}^\infty a_n.
        \]
        \item Let \(\sum_{n = m}^\infty a_n\) be a series of real numbers, and let \(k \geq 0\) be an integer.
        If one of the two series \(\sum_{n = m}^\infty a_n\) and \(\sum_{n = m + k}^\infty a_n\) are convergent, then the other one is also, and we have the identity
        \[
            \sum_{n = m}^\infty a_n = \sum_{n = m}^{m + k - 1} a_n + \sum_{n = m + k}^\infty a_n.
        \]
        \item Let \(\sum_{n = m}^\infty a_n\) be a series of real numbers converging to \(x\), and let \(k\) be an integer.
        Then \(\sum_{n = m + k}^\infty a_{n - k}\) also converges to \(x\).
    \end{enumerate}
\end{proposition}

\begin{proof}{(a)}
Let \(A_N = \sum_{n = m}^N a_n\) be the \(N^{\text{th}}\) partial sum of \(x\), and let \(B_N = \sum_{n = m}^N b_n\) be the \(N^{\text{th}}\) partial sum of \(y\).
So
\begin{align*}
x + y &= \sum_{n = m}^\infty a_n + \sum_{n = m}^\infty b_n \\
&= \lim_{N \to \infty} A_N + \lim_{N \to \infty} B_N & \text{(by Definition \ref{7.2.2})} \\
&= \lim_{N \to \infty} (A_N + B_N) & \text{(by Theorem \ref{6.1.19})} \\
&= \sum_{n = m}^\infty (a_n + b_n). & \text{(by Definition \ref{7.2.2})}
\end{align*}
\end{proof}

\begin{proof}{(b)}
Let \(S_N = \sum_{n = m}^N a_n\) be the \(N^{\text{th}}\) partial sum of \(x\).
So
\begin{align*}
cx &= c \sum_{n = m}^\infty a_n \\
&= c \lim_{N \to \infty} S_N & \text{(by Definition \ref{7.2.2})} \\
&= \lim_{N \to \infty} (c S_N) & \text{(by Theorem \ref{6.1.19})} \\
&= \sum_{n = m}^\infty (c a_n). & \text{(by Definition \ref{7.2.2})}
\end{align*}
\end{proof}

\begin{proof}{(c)}
We first show that \(\sum_{n = m}^\infty a_n\) converges if and only if \(\sum_{n = m + k}^\infty a_n\) converges.
\begin{align*}
& \sum_{n = m}^\infty a_n \text{ converges} \\
\iff & \forall\ \varepsilon \in \mathbf{R} \land \varepsilon > 0, \exists\ N \in \mathbf{N} \land N \geq m : \\
& \abs*{\sum_{n = p + 1}^q a_n} \leq \varepsilon \ \forall\ p, q \in \mathbf{N} \land p, q \geq N & \text{(by Proposition \ref{7.2.5})} \\
\iff & \forall\ \varepsilon \in \mathbf{R} \land \varepsilon > 0, \exists\ N \in \mathbf{N} \land N \geq m : \\
& \abs*{\sum_{n = p + k + 1}^q a_n} \leq \varepsilon \ \forall\ p, q \in \mathbf{N} \land p, q \geq N \\
\iff & \forall\ \varepsilon \in \mathbf{R} \land \varepsilon > 0, \exists\ N \in \mathbf{N} \land N \geq m + k : \\
& \abs*{\sum_{n = p + 1}^q a_n} \leq \varepsilon \ \forall\ p, q \in \mathbf{N} \land p, q \geq N \\
\iff & \sum_{n = m + k}^\infty a_n \text{ converges}. & \text{(by Proposition \ref{7.2.5})}
\end{align*}

Now we show that \(\sum_{n = m}^\infty a_n = \sum_{n = m}^{n + k - 1} a_n + \sum_{n = m + k}^\infty a_n\).
Let \(A_N = \sum_{n = m}^N a_n\) be the \(N^{\text{th}}\) partial sum of \(\sum_{n = m}^\infty a_n\), and let \(B_N = \sum_{n = m + k}^N a_n\) be the \(N^{\text{th}}\) partial sum of \(\sum_{n = m + k}^\infty a_n\).
So
\begin{align*}
\sum_{n = m}^{m + k - 1} a_n + \sum_{n = m + k}^\infty a_n &= A_{m + k - 1} + \sum_{n = m + k}^\infty a_n \\
&= A_{m + k - 1} + \lim_{N \to \infty} B_N & \text{(by Definition \ref{7.2.2})} \\
&= \lim_{N \to \infty} A_{m + k - 1} + \lim_{N \to \infty} B_N \\
&= \lim_{N \to \infty} (A_{m + k - 1} + B_N) & \text{(by Theorem \ref{6.1.19})} \\
&= \lim_{N \to \infty} (\sum_{n = m}^{m + k - 1} a_n + \sum_{n = m + k}^N a_n) \\
&= \lim_{N \to \infty} \sum_{n = m}^N a_n = \lim_{N \to \infty} A_N & \text{(by Lemma \ref{7.1.4})} \\
&= \sum_{n = m}^\infty a_n. & \text{(by Definition \ref{7.2.2})}
\end{align*}
\end{proof}

\begin{proof}{(d)}
Let \(A_N = \sum_{n = m}^N a_n\) be the \(N^{\text{th}}\) partial sum of \(x\), and let \(B_N = \sum_{n = m + k}^N b_{n - k}\) be the \(N^{\text{th}}\) partial sum of \(\sum_{n = m + k}^\infty a_{n - k}\).
So
\begin{align*}
x &= \sum_{n = m}^\infty a_n = \lim_{N \to \infty} A_N = \lim_{N \to \infty} \sum_{n = m}^N a_n & \text{(by Definition \ref{7.2.2})} \\
&= \lim_{N' \to \infty} \sum_{n = m}^{N' - k} a_n & (N' = N + k) \\
&= \lim_{N' \to \infty} \sum_{n = m + k}^{N' - k + k} a_{n - k} & \text{(by Lemma \ref{7.1.4})} \\
&= \lim_{N' \to \infty} \sum_{n = m + k}^{N'} a_{n - k} = \lim_{N' \to \infty} B_{N'} \\
&= \sum_{n = m + k}^\infty a_{n - k}. & \text{(by Definition \ref{7.2.2})}
\end{align*}
\end{proof}

\section{Sums of non-negative numbers}\label{sec 7.3}

\begin{note}
When all the terms in a series are non-negative, there is no distinction between conditional convergence and absolute convergence.
\end{note}

\begin{proposition}\label{7.3.1}
Let \(\sum_{n = m}^\infty a_n\) be a formal series of non-negative real numbers.
Then this series is convergent if and only if there is a real number \(M\) such that
\[
    \sum_{n = m}^N a_n \leq M \text{ for all integers } N \geq m.
\]
\end{proposition}

\begin{proof}
Suppose \(\sum_{n = m}^\infty a_n\) is a series of non-negative numbers.
Then the partial sums \(S_N \coloneqq \sum_{n = m}^N a_n\) are are increasing, i.e., \(S_{N + 1} \geq S_N\) for all \(N \geq m\).
From Proposition \ref{6.3.8} and Corollary \ref{6.1.17}, we thus see that the sequence \((S_N)_{n = m}^\infty\) is convergent if and only if it has an upper bound \(M\).
\end{proof}

\begin{corollary}[Comparison test]\label{7.3.2}
Let \(\sum_{n = m}^\infty a_n\) and \(\sum_{n = m}^\infty b_n\) be two formal series of real numbers, and suppose that \(\abs*{a_n} \leq b_n\) for all \(n \geq m\).
Then if \(\sum_{n = m}^\infty b_n\) is convergent, then \(\sum_{n = m}^\infty a_n\) is absolutely convergent, and in fact
\[
    \abs*{\sum_{n = m}^\infty a_n} \leq \sum_{n = m}^\infty \abs*{a_n} \leq \sum_{n = m}^\infty b_n.
\]
\end{corollary}

\begin{proof}
Let \(A_N = \sum_{n = m}^N \abs*{a_n}\) be the \(N^\text{th}\) partial sum of \(\sum_{n = m}^\infty \abs*{a_n}\), and \(B_N = \sum_{n = m}^N b_n\) be the \(N^\text{th}\) partial sum of \(\sum_{n = m}^\infty b_n\).
Since \((B_N)_{N = m}^\infty\) converges, by Proposition \ref{7.3.1} \(\exists\ M \in \mathbf{R}\) such that \(\sum_{n = m}^N b_n \leq M \ \forall\ N \geq m\).
\begin{align*}
& \sum_{n = m}^\infty b_n \text{ converges} \\
\implies & \exists\ M \in \mathbf{R} : \sum_{n = m}^N b_n \leq M \ \forall\ N \in \mathbf{N} \land N \geq m & \text{(by Proposition \ref{7.3.1})} \\
\implies & \exists\ M \in \mathbf{R} : \sum_{n = m}^N \abs*{a_n} \leq \sum_{n = m}^N b_n \leq M \ \forall\ N \geq m & \text{(by the given condition)} \\
\implies & \exists\ M \in \mathbf{R} : \abs*{\sum_{n = m}^N a_n} \leq \sum_{n = m}^N \abs*{a_n} \leq \sum_{n = m}^N b_n \leq M \ \forall\ N \geq m & \text{(by Lemma \ref{7.1.4})} \\
\implies & \abs*{\sum_{n = m}^\infty a_n} \text{ converges} \land \sum_{n = m}^\infty \abs*{a_n} \text{ converges} & \text{(by Proposition \ref{7.3.1})} \\
\implies & \abs*{\sum_{n = m}^\infty a_n} \leq \sum_{n = m}^\infty \abs*{a_n} \leq \sum_{n = m}^\infty b_n. & \text{(by Theorem \ref{6.1.19})}
\end{align*}
\end{proof}
\section{Rearrangement of series}\label{sec 7.4}

\begin{note}
    One feature of finite sums is that no matter how one rearranges the terms in a sequence, the total sum is the same.
    A more rigorous statement of this, involving bijections, has already appeared earlier, see Remark \ref{7.1.12}.
\end{note}

\begin{proposition}\label{7.4.1}
    Let \(\sum_{n = 0}^\infty a_n\) be a convergent series of non-negative real numbers, and let \(f : \mathbf{N} \to \mathbf{N}\) be a bijection.
    Then \(\sum_{m = 0}^\infty a_{f(m)}\) is also convergent, and has the same sum:
    \[
        \sum_{n = 0}^\infty a_n = \sum_{m = 0}^\infty a_{f(m)}.
    \]
\end{proposition}

\begin{proof}
    We introduce the partial sums \(S_N \coloneqq \sum_{n = 0}^N a_n\) and \(T_M \coloneqq \sum_{m = 0}^M a_{f(m)}\).
    We know that the sequences \((S_N)_{N = 0}^\infty\) and \((T_M)_{M = 0}^\infty\) are increasing.
    Write \(L \coloneqq \sup(S_N)_{N = 0}^\infty\) and \(L' \coloneqq \sup(T_M)_{M = 0}^\infty\).
    By Proposition \ref{6.3.8} we know that \(L\) is finite, and in fact \(L = \sum_{n = 0}^\infty a_n\);
    by Proposition \ref{6.3.8} again we see that we will thus be done as soon as we can show that \(L' = L\).

    Fix \(M\), and let \(Y\) be the set \(Y \coloneqq \{m \in \mathbf{N} : m \leq M\}\).
    Note that \(f\) is a bijection between \(Y\) and \(f(Y)\).
    By Proposition \ref{7.1.11}, we have
    \[
        T_M = \sum_{m = 0}^M a_{f(m)} = \sum_{m \in Y} a_{f(m)} = \sum_{n \in f(Y)} a_n.
    \]
    The sequence \((f(m))_{m = 0}^M\) is finite, hence bounded, i.e., there exists an \(N\) such that \(f(m) \leq N\) for all \(m \leq M\).
    In particular \(f(Y)\) is a subset of \(\{n \in \mathbf{N} : n \leq N\}\), and so by Proposition \ref{7.1.11} again (and the assumption that all the \(a_n\) are non-negative)
    \[
        T_M = \sum_{n \in f(Y)} a_n \leq \sum_{n \in \{n \in \mathbf{N} : n \leq N\}} a_n = \sum_{n = 0}^N a_n = S_N.
    \]
    But since \((S_N)_{N = 0}^\infty\) has a supremum of \(L\), we thus see that \(S_N \leq L\), and hence that \(T_M \leq L\) for all \(M\).
    Since \(L'\) is the least upper bound of \((T_M)_{M = 0}^\infty\), this implies that \(L' \leq L\).

    A very similar argument (using the inverse \(f^{-1}\) instead of \(f\)) shows that every \(S_N\) is bounded above by \(L'\), and hence \(L \leq L'\).

    Fix \(N\), and let \(X\) be the set \(X \coloneqq \{n \in \mathbf{N} : n \leq N\}\).
    Note that \(f^{-1}\) is a bijection between \(X\) and \(f^{-1}(X)\).
    By Proposition \ref{7.1.11}, we have
    \[
        S_N = \sum_{n = 0}^N a_n = \sum_{n \in X} a_n = \sum_{m \in f^{-1}(X)} a_{f(m)}.
    \]
    The sequence \((f^{-1}(n))_{n = 0}^N\) is finite, hence bounded, i.e., there exists an \(M\) such that \(f^{-1}(n) \leq M\) for all \(n \leq N\).
    In particular \(f^{-1}(X)\) is a subset of \(\{m \in \mathbf{N} : m \leq M\}\), and so by Proposition \ref{7.1.11} again (and the assumption that all the \(a_n\) are non-negative)
    \[
        S_N = \sum_{m \in f^{-1}(X)} a_{f(m)} \leq \sum_{m \in \{m \in \mathbf{N} : m \leq M\}} a_{f(m)} = \sum_{m = 0}^M a_{f(m)} = T_M.
    \]
    But since \((T_M)_{M = 0}^\infty\) has a supremum of \(L'\), we thus see that \(T_M \leq L'\), and hence that \(S_N \leq L'\) for all \(N\).
    Since \(L\) is the least upper bound of \((S_N)_{N = 0}^\infty\), this implies that \(L \leq L'\).

    Combining these two inequalities we obtain \(L = L'\), as desired.
\end{proof}

\setcounter{theorem}{2}
\begin{proposition}[Rearrangement of series]\label{7.4.3}
    Let \(\sum_{n = 0}^\infty a_n\) be an absolutely convergent series of real numbers, and let \(f : \mathbf{N} \to \mathbf{N}\) be a bijection.
    Then \(\sum_{m = 0}^\infty a_{f(m)}\) is also absolutely convergent, and has the same sum:
    \[
        \sum_{n = 0}^\infty a_n = \sum_{m = 0}^\infty a_{f(m)}.
    \]
\end{proposition}

\begin{proof}
    We apply Proposition \ref{7.4.1} to the infinite series \(\sum_{n = 0}^\infty \abs*{a_n}\), which by hypothesis is a convergent series of non-negative numbers.
    If we write \(L \coloneqq \sum_{n = 0}^\infty \abs*{a_n}\), then by Proposition \ref{7.4.1} we know that \(\sum_{m = 0}^\infty \abs*{a_{f(m)}}\) also converges to \(L\).

    Now write \(L' \coloneqq \sum_{n = 0}^\infty a_n\).
    We have to show that \(\sum_{m = 0}^\infty a_{f(m)}\) also converges to \(L'\).
    In other words, given any \(\varepsilon > 0\), we have to find an \(M\) such that \(\sum_{m = 0}^{M'} a_{f(m)}\) is \(\varepsilon\)-close to \(L'\) for every \(M' \geq M\).

    Since \(\sum_{n = 0}^\infty \abs*{a_n}\) is convergent, we can use Proposition \ref{7.2.5} and find an \(N_1\) such that \(\sum_{n = p + 1}^q \abs*{a_n} \leq \varepsilon / 2\) for all \(p, q \geq N_1\).
    Since \(\sum_{n = 0}^\infty a_n\) converges to \(L'\), the partial sums \(\sum_{n = 0}^N a_n\) also converge to \(L'\), and so there exists \(N \geq N_1\) such that \(\sum_{n = 0}^N a_n\) is \(\varepsilon / 2\)-close to \(L'\).

    Now the sequence \((f^{-1}(n))_{n = 0}^N\) is finite, hence bounded, so there exists an \(M\) such that \(f^{-1}(n) \leq M\) for all \(0 \leq n \leq N\).
    In particular, for any \(M' \geq M\), the set \(\{f(m) : m \in \mathbf{N}; m \leq M'\}\) contains \(\{n \in \mathbf{N} : n \leq N\}\).
    So by Proposition \ref{7.1.11}, for any \(M' \geq M\),
    \[
        \sum_{m = 0}^{M'} a_{f(m)} = \sum_{n \in \{f(m) : m \in \mathbf{N}; m \leq M'\}} a_n = \sum_{n = 0}^N a_n + \sum_{n \in X} a_n
    \]
    where \(X\) is the set
    \[
        X = \{f(m) : m \in \mathbf{N}; m \leq M'\} \setminus \{n \in \mathbf{N} : n \leq N\}.
    \]
    The set \(X\) is finite, and is therefore bounded by some natural number \(q\);
    we must therefore have
    \[
        X \subseteq \{n \in \mathbf{N} : N + 1 \leq n \leq q\}.
    \]
    Thus
    \[
        \abs*{\sum_{n \in X} a_n} \leq \sum_{n \in X} \abs*{a_n} \leq \sum_{n = N + 1}^q \abs*{a_n} \leq \varepsilon / 2
    \]
    by our choice of \(N\).
    Thus \(\sum_{m = 0}^{M'} a_{f(m)}\) is \(\varepsilon / 2\)-close to \(\sum_{n = 0}^N a_n\), which as mentioned before is \(\varepsilon / 2\)-close to \(L'\).
    Thus \(\sum_{m = 0}^{M'} a_{f(m)}\) is \(\varepsilon\)-close to \(L'\) for all \(M' \geq M\), as desired.
\end{proof}

\begin{note}
    There is in fact a surprising result of Riemann, which shows that a series which is conditionally convergent but not absolutely convergent can in fact be rearranged to converge to \emph{any} value
    (or rearranged to diverge).
\end{note}

\begin{note}
    To summarize, rearranging series is safe when the series is absolutely convergent, but is somewhat dangerous otherwise.
    (This is not to say that rearranging a series that is not absolutely convergent necessarily gives you the wrong answer
    - for instance, in theoretical physics one often performs similar maneuvres, and one still (usually) obtains a correct answer at the end
    - but doing so is risky, unless it is backed by a rigorous result such as Proposition \ref{7.4.3}.)
\end{note}

\exercisesection

\begin{exercise}\label{ex 7.4.1}
    Let \(\sum_{n = 0}^\infty a_n\) be an absolutely convergent series of real numbers.
    Let \(f : \mathbf{N} \to \mathbf{N}\) be an increasing function (i.e., \(f(n + 1) > f(n)\) for all \(n \in \mathbf{N}\)).
    Show that \(\sum_{n = 0}^\infty a_{f(n)}\) is also an absolutely convergent series.
\end{exercise}

\begin{proof}
    Let \(S_N = \sum_{n = 0}^N \abs*{a_n}\) and \(T_N = \sum_{n = 0}^N \abs*{a_{f(n)}}\).
    Since \(\sum_{n = 0}^\infty a_n\) is absolutely convergent and \(S_N\) is an increasing sequence, by Proposition \ref{6.3.8} we have \(\lim_{N \to \infty} S_N = \sup(S_N)_{N = 0}^\infty\).
    Since \((f(n))_{n = 0}^N\) is a finite sequence, it is bounded by some \(M \in \mathbf{N}\), thus \(\{f(n) : n \in \mathbf{N} \land n \leq N\} \subseteq \{n \in \mathbf{N} : n \leq M\}\).
    Now we have
    \[
        T_N = \sum_{n = 0}^N \abs*{a_{f(n)}} = \sum_{n \in \{f(n) : n \in \mathbf{N} \land n \leq N\}} \abs*{a_n} \leq \sum_{n = 0}^M a_n = S_M \leq \sup(S_M)_{M = 0}^\infty = \lim_{M \to \infty} S_M,
    \]
    which means \(T_N\) is bounded.
    Since \((T_N)_{N = 0}^\infty\) is an increasing sequence and is bounded, by Proposition \ref{6.3.8} \((T_N)_{N = 0}^\infty\) converges, and thus \(\sum_{n = 0}^\infty a_{f(n)}\) is absolutely convergent.
\end{proof}
\chapter{Infinite sets}\label{ch 8}

\section{Countability}\label{sec 8.1}

\begin{note}
    From Theorem \ref{3.6.12} we know that the set \(\mathbf{N}\) of natural numbers is infinite.
    The set \(\mathbf{N} - \{0\}\) is also infinite, thanks to Proposition \ref{3.6.14}(a), and is a proper subset of \(\mathbf{N}\).
    However, the set \(\mathbf{N} - \{0\}\), despite being ``smaller'' than \(\mathbf{N}\), still has the same cardinality as \(\mathbf{N}\), because the function \(f : \mathbf{N} \to \mathbf{N} - \{0\}\) defined by \(f(n) \coloneqq n + 1\), is a bijection from \(\mathbf{N}\) to \(\mathbf{N} - \{0\}\).
    This is one characteristic of infinite sets.
\end{note}

\begin{definition}[Countable sets]\label{8.1.1}
    A set \(X\) is said to be \emph{countably infinite} (or just \emph{countable}) iff it has equal cardinality with the natural numbers \(\mathbf{N}\).
    A set \(X\) is said to be \emph{at most countable} iff it is either countable or finite.
    We say that a set is \emph{uncountable} if it is infinite but not countable.
\end{definition}

\begin{remark}\label{8.1.2}
    Countably infinite sets are also called \emph{denumerable} sets.
\end{remark}

\begin{example}\label{8.1.3}
    The even natural numbers \(\{2n : n \in \mathbf{N}\}\), since the function \(f(n) \coloneqq 2n\) provides a bijection between \(\mathbf{N}\) and the even natural numbers.
\end{example}

\begin{note}
    Let \(X\) be a countable set.
    Then, by definition, we know that there exists a bijection \(f : \mathbf{N} \to X\).
    Thus, every element of \(X\) can be written in the form \(f(n)\) for exactly one natural number \(n\).
    Informally, we thus have
    \[
        X = \{f(0), f(1), f(2), f(3), \dots\}.
    \]
    Thus, a countable set can be arranged in a sequence, so that we have a zeroth element \(f(0)\), followed by a first element \(f(1)\), then a second element \(f(2)\), and so forth, in such a way that all these elements \(f(0), f(1), f(2), \dots\) are all distinct, and together they fill out all of \(X\).
    (This is why these sets are called \emph{countable};
    because we can literally count them one by one, starting from \(f(0)\), then \(f(1)\), and so forth.)
\end{note}

\begin{proposition}[Well ordering principle]\label{8.1.4}
    Let \(X\) be a non-empty subset of the natural numbers \(\mathbf{N}\).
    Then there exists exactly one element \(n \in X\) such that \(n \leq m\) for all \(m \in X\).
    In other words, every non-empty set of natural numbers has a minimum element.
\end{proposition}

\begin{proof}
    Suppose for sake of contradiction that \(X\) has no minimum element.
    In other words, we have
    \[
        \forall\ n \in \mathbf{N}, \forall\ m \in X : n \leq m \implies n \notin X.
    \]
    We now use induction to show that \(\forall\ n \in \mathbf{N}\), \(n \notin X\).
    For \(n = 0\), we have
    \begin{align*}
                 & X \subseteq \mathbf{N}                                                               \\
        \implies & (\forall\ m \in X \implies m \in \mathbf{N}) & \text{(by Definition \ref{3.1.15})}   \\
        \implies & (\forall\ m \in X \implies 0 \leq m)         & \text{(by Axiom \ref{2.3})}           \\
        \implies & 0 \notin X.                                  & \text{(\(X\) has no minimum element)}
    \end{align*}
    Thus the base case holds.
    Suppose inductively that \(n \notin X\) is true for some \(n \geq 0\).
    Then for \(n + 1\), we have
    \begin{align*}
                 & n \notin X                    & \text{(by induction hypothesis)}        \\
        \implies & \forall\ m \in X : m \geq n                                             \\
        \implies & \forall\ m \in X : m > n      & (m \neq n)                              \\
        \implies & \forall\ m \in X : m \geq n++ & \text{(by Proposition \ref{2.2.12}(e))} \\
        \implies & n++ \notin X.                 & \text{(\(X\) has no minimum element)}
    \end{align*}
    This close the induction.

    Since \(\forall\ n \in \mathbf{N} \implies n \notin X\), we must have \(X = \emptyset\), a contradiction.
    This means \(\exists\ n \in \mathbf{N}\) such that \(\forall\ m \in X : n \leq m \land n \in X\).
    We now show that such \(n\) is unique.
    Suppose that \(\exists\ n, n' \in X\) such that \(\forall\ m \in X : n \leq m \land n' \leq m\).
    Then we have \(n \leq n' \land n' \leq n\), which means \(n = n'\).
\end{proof}

\begin{note}
    We will refer to the element \(n\) given by the well-ordering principle as the \emph{minimum} of \(X\), and write it as \(\min(X)\).
    This minimum is clearly the same as the infimum of \(X\), as defined in Definition \ref{5.5.10}.
\end{note}

\begin{proposition}\label{8.1.5}
    Let \(X\) be an infinite subset of the natural numbers \(\mathbf{N}\).
    Then there exists a unique bijection \(f : \mathbf{N} \to X\) which is increasing, in the sense that \(f(n + 1) > f(n)\) for all \(n \in N\).
    In particular, \(X\) has equal cardinality with \(\mathbf{N}\) and is hence countable.
\end{proposition}

\begin{proof}
    We now define a sequence \(a_0, a_1, a_2, \dots\) of natural numbers recursively by the formula
    \[
        a_n \coloneqq \min\{x \in X : x \neq a_m \ \forall\ m < n\}.
    \]
    Intuitively speaking, \(a_0\) is the smallest element of \(X\);
    \(a_1\) is the second smallest element of \(X\), i.e., the smallest element of \(X\) once \(a_0\) is removed;
    \(a_2\) is the third smallest element of \(X\);
    and so forth.
    Observe that in order to define \(a_n\), one only needs to know the values of \(a_m\) for all \(m < n\), so this definition is recursive.
    Also, since \(X\) is infinite, the set \(\{x \in X : x \neq a_m \ \forall\ m < n\}\) is infinite, hence non-empty.
    (If it is finite, then its union with the set \(\{a_0, \dots, a_{n - 1}\}\) is also finite, contradict to \(X\) is infinite)
    Thus by the well-ordering principle (Proposition \ref{8.1.5}), the minimum, \(\min\{x \in X : x \neq a_m \ \forall\ m < n\}\) is always well-defined.

    One can show that \(a_n\) is an increasing sequence, i.e.
    \[
        a_0 < a_1 < a_2 < \dots
    \]
    and in particular that \(a_n \neq a_m\) for all \(n \neq m\).
    (If \(a_n \geq a_{n + 1}\), then \(a_{n + 1} = \min\{x \in X : x \neq a_m \ \forall\ m < n\}\), a contradiction)
    Also, we have \(a_n \in X\) for each natural number \(n\) (by Proposition \ref{8.1.4}).

    Now define the function \(f : \mathbf{N} \to X\) by \(f(n) \coloneqq a_n\).
    From the previous paragraph we know that \(f\) is one-to-one.
    Now we show that \(f\) is onto.
    In other words, we claim that for every \(x \in X\), there exists an \(n\) such that \(a_n = x\).

    Let \(x \in X\).
    Suppose for sake of contradiction that \(a_n \neq x\) for every natural number \(n\).
    Then this implies that \(x\) is an element of the set \(\{x \in X : x \neq a_m \ \forall\ m < n\}\) for all \(n\).
    By definition of \(a_n\), this implies that \(x \geq a_n\) for every natural number \(n\).
    However, since \(a_n\) is an increasing sequence, we have \(a_n \geq n\), and hence \(x \geq n\) for every natural number \(n\).
    In particular we have \(x \geq x + 1\), which is a contradiction.
    Thus we must have \(a_n = x\) for some natural number \(n\), and hence \(f\) is onto.

    Since \(f : \mathbf{N} \to X\) is both one-to-one and onto, it is a bijection.
    We have thus found at least one increasing bijection \(f\) from \(\mathbf{N}\) to \(X\).
    Now suppose for sake of contradiction that there was at least one other increasing bijection \(g\) from \(\mathbf{N}\) to \(X\) which was not equal to \(f\).
    Then the set \(\{n \in \mathbf{N} : g(n) \neq f(n)\}\) is non-empty, and define \(m \coloneqq \min\{n \in \mathbf{N} : g(n) \neq f(n)\}\), thus in particular \(g(m) \neq f(m) = a_m\), and \(g(n) = f(n) = a_n\) for all \(n < m\).
    But we then must have
    \[
        g(m) = \min\{x \in X : x \neq a_t \ \forall\ t < m\} = a_m,
    \]
    a contradiction.
    Thus there is no other increasing bijection from \(\mathbf{N}\) to \(X\) other than \(f\).
\end{proof}

\begin{corollary}\label{8.1.6}
    All subsets of the natural numbers are at most countable.
\end{corollary}

\begin{proof}
    Since finite sets are at most countable by definition, combine with Proposition \ref{8.1.5} we thus have all subsets of the natural numbers are at most countable.
\end{proof}

\begin{corollary}\label{8.1.7}
    If \(X\) is an at most countable set, and \(Y\) is a subset of \(X\), then \(Y\) is at most countable.
\end{corollary}

\begin{proof}
    If \(X\) is finite then this follows from Proposition \ref{3.6.14}(c), so assume \(X\) is countable.
    Then there is a bijection \(f : X \to \mathbf{N}\) between \(X\) and \(\mathbf{N}\).
    Since \(Y\) is a subset of \(X\), and \(f\) is a bijection from \(X\) and \(\mathbf{N}\), then when we restrict \(f\) to \(Y\), we obtain a bijection between \(Y\) and \(f(Y)\).
    Thus \(f(Y)\) has equal cardinality with \(Y\).
    But \(f(Y)\) is a subset of \(\mathbf{N}\), and hence at most countable by Corollary \ref{8.1.6}.
    Hence \(Y\) is also at most countable.
\end{proof}

\begin{proposition}\label{8.1.8}
    Let \(Y\) be a set, and let \(f : \mathbf{N} \to Y\) be a function.
    Then \(f(\mathbf{N})\) is at most countable.
\end{proposition}

\begin{proof}
    If \(f(\mathbf{N})\) is finite then by Definition \ref{8.1.1} it is at most countable.
    So assume that \(f(\mathbf{N})\) is infinite.
    Let \(A\) be the set
    \[
        A = \{n \in \mathbf{N} : f(m) \neq f(n) \ \forall\ 0 \leq m < n\}.
    \]
    So \(A \subseteq \mathbf{N}\).
    We now show that \(f_A : A \to f(\mathbf{N})\) is a bijection.

    Let \(m, n \in A \land m \neq n\).
    By the definition of \(A\) we know that \(f_A(m) \neq f_A(n)\).
    So \(f_A\) is injective.

    Suppose for sake of contradiction that \(f_A\) is not surjective.
    Then \(\exists\ y \in f(\mathbf{N})\) such that \(f_A(n) \neq y \ \forall\ n \in A\).
    But \(f_A(n) \neq y\) means \(\exists\ m \in A\) such that \(f(m) = y\), a contradiction.
    So \(f_A\) is surjective, and thus is bijective.

    Since \(A \subseteq \mathbf{N}\), by Proposition \ref{8.1.5} \(\exists\ g : \mathbf{N} \to A\) where \(g\) is bijective.
    This means \(f_A \circ g : \mathbf{N} \to f(\mathbf{N})\) is bijective.
    So by Definition \ref{8.1.1} \(f(\mathbf{N})\) is countable, and thus at most countable.
\end{proof}

\begin{corollary}\label{8.1.9}
    Let \(X\) be a countable set, and let \(f : X \to Y\) be a function.
    Then \(f(X)\) is at most countable.
\end{corollary}

\begin{proof}
    By Definition \ref{8.1.1} \(\exists\ g : \mathbf{N} \to X\) such that \(g\) is a bijection.
    So we have \(f \circ g : \mathbf{N} \to Y\) and by Proposition \ref{8.1.8} \((f \circ g)(\mathbf{N})\) is at most countable.
    But
    \[
        (f \circ g)(\mathbf{N}) = f(g(\mathbf{N})) = f(X).
    \]
    So \(f(X)\) is at most countable.
\end{proof}

\begin{proposition}\label{8.1.10}
    Let \(X\) be a countable set, and let \(Y\) be a countable set.
    Then \(X \cup Y\) is a countable set.
\end{proposition}

\begin{proof}
    By Definition \ref{8.1.1} \(\exists\ f : \mathbf{N} \to X\) and \(g : \mathbf{N} \to Y\) such that \(f\) and \(g\) are bijections.
    Let \(h : \mathbf{N} \to X \cup Y\) by setting \(h(2n) = f(n)\) and \(h(2n + 1) = g(n)\) for every natural number \(n\).
    We now show that \(h(\mathbf{N}) = X \cup Y\).
    \begin{align*}
             & z \in h(\mathbf{N})          \\
        \iff & z = h(2n) \lor z = h(2n + 1) \\
        \iff & z = f(n) \lor z = g(n)       \\
        \iff & z \in X \lor z \in Y         \\
        \iff & z \in X \cup Y.
    \end{align*}
    Then by Corollary \ref{8.1.9} we have \(h(\mathbf{N}) = X \cup Y\) is at most countable.
    But since \(X\) and \(Y\) are infinite sets, \(X \cup Y\) can not be finite, thus \(X \cup Y\) is countable.
\end{proof}

\begin{note}
    To summarize, any subset or image of a countable set is at most countable, and any finite union of countable sets is still countable.
\end{note}

\begin{corollary}\label{8.1.11}
    The integers \(\mathbf{Z}\) are countable.
\end{corollary}

\begin{proof}
    We already know that the set \(\mathbf{N} = \{0, 1, 2, 3, \dots\}\) of natural numbers are countable.
    The set \(-\mathbf{N}\) defined by
    \[
        -\mathbf{N} \coloneqq \{-n : n \in \mathbf{N}\} = \{0, -1, -2, -3, \dots\}
    \]
    is also countable, since the map \(f(n) \coloneqq -n\) is a bijection between \(\mathbf{N}\) and this set.
    Since the integers are the union of \(\mathbf{N}\) and \(-\mathbf{N}\), the claim follows from Proposition \ref{8.1.10}.
\end{proof}

\begin{note}
    To establish countability of the rationals, we need to relate countability with Cartesian products.
    In particular, we need to show that the set \(\mathbf{N} \times \mathbf{N}\) is countable.
\end{note}

\begin{lemma}\label{8.1.12}
    The set
    \[
        A \coloneqq \{(n, m) \in \mathbf{N} \times \mathbf{N} : 0 \leq m \leq n\}
    \]
    is countable.
\end{lemma}

\begin{proof}
    Define the sequence \(a_0, a_1, a_2, \dots\) recursively by setting \(a_0 \coloneqq 0\), and \(a_{n + 1} \coloneqq a_n + n + 1\) for all natural numbers \(n\).
    Thus
    \[
        a_0 = 0; a_1 = 0 + 1; a_2 = 0 + 1 + 2; a_3 = 0 + 1 + 2 + 3; \dots
    \]
    By induction one can show that \(a_n\) is increasing, i.e., that \(a_n > a_m\) whenever \(n > m\).

    Now define the function \(f : A \to \mathbf{N}\) by
    \[
        f(n, m) \coloneqq a_n + m.
    \]
    We claim that \(f\) is one-to-one.
    In other words, if \((n, m)\) and \((n', m')\) are any two distinct elements of \(A\), then we claim that \(f(n, m) \neq f(n', m')\).

    To prove this claim, let \((n, m)\) and \((n', m')\) be two distinct elements of \(A\).
    There are three cases: \(n' = n\), \(n' > n\), and \(n' < n\).
    First suppose that \(n' = n\).
    Then we must have \(m \neq m'\), otherwise \((n, m)\) and \((n', m')\) would not be distinct.
    Thus \(a_n + m \neq a_n + m'\), and hence \(f(n, m) \neq f(n', m')\), as desired.

    Now suppose that \(n' > n\).
    Then \(n' \geq n + 1\), and hence
    \[
        f(n', m') = a_{n'} + m' \geq a_{n'} \geq a_{n + 1} = a_n + n + 1.
    \]
    But since \((n, m) \in A\), we have \(m \leq n < n + 1\), and hence
    \[
        f(n', m') \geq a_n + n + 1 > a_n + m = f(n, m),
    \]
    and thus \(f(n', m') \neq f(n, m)\).

    The case \(n' < n\) is proven similarly, by switching the roles of \(n\) and \(n'\) in the previous argument.
    Thus we have shown that \(f\) is one-to-one.
    Thus \(f\) is a bijection from \(A\) to \(f(A)\), and so \(A\) has equal cardinality with \(f(A)\).
    But \(f(A)\) is a subset of \(\mathbf{N}\), and hence by Corollary \ref{8.1.6} \(f(A)\) is at most countable.
    Therefore \(A\) is at most countable.
    But, \(A\) is clearly not finite.
    (if \(A\) was finite, then every subset of \(A\) would be finite, and in particular \(\{(n, 0) : n \in \mathbf{N}\}\) would be finite, but this is clearly countably infinite, a contradiction.)
    Thus, \(A\) must be countable.
\end{proof}

\begin{corollary}\label{8.1.13}
    The set \(\mathbf{N} \times \mathbf{N}\) is countable.
\end{corollary}

\begin{proof}
    We already know that the set
    \[
        A \coloneqq \{(n, m) \in \mathbf{N} \times \mathbf{N} : 0 \leq m \leq n\}
    \]
    is countable.
    This implies that the set
    \[
        B \coloneqq \{(n, m) \in \mathbf{N} \times \mathbf{N} : 0 \leq n \leq m\}
    \]
    is also countable, since the map \(f : A \to B\) given by \(f(n, m) \coloneqq (m, n)\) is a bijection from \(A\) to \(B\).
    But since \(\mathbf{N} \times \mathbf{N}\) is the union of \(A\) and \(B\), the claim then follows from Proposition \ref{8.1.10}.
\end{proof}

\begin{corollary}\label{8.1.14}
    If \(X\) and \(Y\) are countable, then \(X \times Y\) is countable.
\end{corollary}

\begin{proof}
    By Definition \ref{8.1.1} \(\exists\ f : \mathbf{N} \to X\) and \(g : \mathbf{N} \to Y\) such that \(f\) and \(g\) are bijections.
    Let \(h : \mathbf{N} \times \mathbf{N} \to X \times Y\) by setting \(h(x, y) = (f(x), g(y))\).
    If \(n, n', m, m' \in \mathbf{N}\) and \((n, m) \neq (n', m')\), then
    \[
        h(n, m) = (f(n), g(m)) \neq (f(n'), g(m')) = h(n', m')
    \]
    since \(f, g\) are bijections, so \(h\) is injective.
    Again since \(f, g\) are bijections, \(\forall\ x \in X \land \ \forall\ y \in Y\), \(\exists\ n, m \in \mathbf{N}\) such that \(x = f(n) \land y = g(m)\).
    So \(h\) is surjective, and thus is bijective.

    Since \(h\) is bijective, \(\mathbf{N} \times \mathbf{N}\) and \(X \times Y\) has the same cardinality.
    But by Corollary \ref{8.1.13} we know that \(\mathbf{N} \times \mathbf{N}\) is countable.
    So by Proposition \ref{3.6.3} \(X \times Y\) has the same cardinality as \(\mathbf{N}\), thus is countable.
\end{proof}

\begin{corollary}\label{8.1.15}
    The rationals \(\mathbf{Q}\) are countable.
\end{corollary}

\begin{proof}
    We already know that the integers \(\mathbf{Z}\) are countable, which implies that the non-zero integers \(\mathbf{Z} - \{0\}\) are countable.
    By Corollary \ref{8.1.14}, the set
    \[
        \mathbf{Z} \times (\mathbf{Z} - \{0\}) = \{(a, b) : a, b \in \mathbf{Z}, b \neq 0\}
    \]
    is thus countable.
    If one lets \(f : \mathbf{Z} \times (\mathbf{Z} - \{0\}) \to \mathbf{Q}\) be the function \(f(a, b) \coloneqq a / b\)
    (note that \(f\) is well-defined since we prohibit \(b\) from being equal to \(0\)), we see from Corollary \ref{8.1.9} that \(f(\mathbf{Z} \times (\mathbf{Z} - \{0\}))\) is at most countable.
    But we have \(f(\mathbf{Z} \times (\mathbf{Z} - \{0\})) = \mathbf{Q}\)
    (This is basically the definition of the rationals \(\mathbf{Q}\)).
    Thus \(\mathbf{Q}\) is at most countable.
    However, \(\mathbf{Q}\) cannot be finite, since it contains the infinite set \(\mathbf{N}\).
    Thus \(\mathbf{Q}\) is countable.
\end{proof}

\begin{remark}\label{8.1.16}
    Because the rationals are countable, we know \emph{in principle} that it is possible to arrange the rational numbers as a sequence:
    \[
        \mathbf{Q} = \{a_0, a_1, a_2, a_3, \dots\}
    \]
    such that every element of the sequence is different from every other element, and that the elements of the sequence exhaust \(\mathbf{Q}\)
    (i.e., every rational number turns up as one of the elements \(a_n\) of the sequence).
    However, it is quite difficult (though not impossible) to actually try and come up with an explicit sequence \(a_0, a_1, \dots\) which does this.
\end{remark}

\exercisesection

\begin{exercise}\label{ex 8.1.1}
    Let \(X\) be a set.
    Show that \(X\) is infinite if and only if there exists a proper subset \(Y \subsetneq X\) of \(X\) which has the same cardinality as \(X\).
\end{exercise}

\begin{proof}
    This shall be done once I learn Axiom of choice.
\end{proof}

\begin{exercise}\label{ex 8.1.2}
    Prove Proposition \ref{8.1.4}.
\end{exercise}

\begin{proof}
    See Proposition \ref{8.1.4}.
\end{proof}

\begin{exercise}\label{ex 8.1.3}
    Fill in the gaps marked in Proposition \ref{8.1.5}.
\end{exercise}

\begin{proof}
    See Proposition \ref{8.1.5}.
\end{proof}

\begin{exercise}\label{ex 8.1.4}
    Prove Proposition \ref{8.1.8}.
\end{exercise}

\begin{proof}
    See Proposition \ref{8.1.8}.
\end{proof}

\begin{exercise}\label{ex 8.1.5}
    Use Proposition \ref{8.1.8} to prove Corollary \ref{8.1.9}.
\end{exercise}

\begin{proof}
    See Corollary \ref{8.1.9}.
\end{proof}

\begin{exercise}\label{ex 8.1.6}
    Let \(A\) be a set.
    Show that \(A\) is at most countable if and only if there exists an injective map \(f : A \to \mathbf{N}\) from \(A\) to \(\mathbf{N}\).
\end{exercise}

\begin{proof}
    We first show that if \(A\) is at most countable, then there exists an injective map \(f : A \to \mathbf{N}\) from \(A\) to \(\mathbf{N}\).
    By Definition \ref{8.1.1} \(A\) is either finite or countable.
    \begin{enumerate}
        \item If \(A\) is finite, then by Definition \ref{3.6.10} \(\exists\ f : A \to \{i \in \mathbf{N} : 1 \leq i \leq \mathbf{N}\}\) where \(f\) is a bijection.
              Since \(f\) is a bijection and \(\{i \in \mathbf{N} : 1 \leq i \leq \mathbf{N}\} \subseteq \mathbf{N}\), we have \(f : A \to \mathbf{N}\) is injective.
        \item If \(A\) is countable, then by Definition \ref{8.1.1} \(\exists\ f : A \to \mathbf{N}\) such that \(f\) is a bijection, and hence \(f\) is injective.
    \end{enumerate}
    From all cases above we can conclude that if \(A\) is at most countable then there exists an injective map \(f : A \to \mathbf{N}\) from \(A\) to \(\mathbf{N}\).

    Now we show that if there exists an injective map \(f : A \to \mathbf{N}\) from \(A\) to \(\mathbf{N}\), then \(A\) is at most countable.
    Since \(f(A) \subseteq \mathbf{N}\), by Corollary \ref{8.1.6} \(f(A)\) is at most countable.
    Since \(f\) is bijective from \(A\) to \(f(A)\), \(A\) and \(f(A)\) have equal cardinality, and thus \(A\) is at most countable.
\end{proof}

\begin{exercise}\label{ex 8.1.7}
    Prove Proposition \ref{8.1.10}.
\end{exercise}

\begin{proof}
    See Proposition \ref{8.1.10}.
\end{proof}

\begin{exercise}\label{ex 8.1.8}
    Use Proposition \ref{8.1.13} to prove Corollary \ref{8.1.14}.
\end{exercise}

\begin{proof}
    See Corollary \ref{8.1.14}.
\end{proof}

\begin{exercise}\label{ex 8.1.9}
    Suppose that \(I\) is an at most countable set, and for each \(\alpha \in I\), let \(A_{\alpha}\) be an at most countable set.
    Show that the set \(\bigcup_{\alpha \in I} A_{\alpha}\) is also at most countable.
    In particular, countable unions of countable sets are countable.
\end{exercise}

\begin{proof}
    This shall be done once I learn Axiom of choice.
\end{proof}

\begin{exercise}\label{ex 8.1.10}
    Find a bijection \(f : \mathbf{N} \to \mathbf{Q}\) from the natural numbers to the rationals.
\end{exercise}

\begin{proof}
    Helped needed.
\end{proof}
\chapter{Continuous functions on R}\label{ch 9}

\begin{note}
    Roughly speaking a set is discrete if each element is separated from the rest of the set by some non-zero distance, whereas a set is a \emph{continuum} if it is connected and contains no ``holes''.
\end{note}

\section{Subsets of the real line}\label{sec 9.1}

\begin{definition}[Intervals]\label{9.1.1}
    Let \(a, b \in \mathbf{R}^*\) be extended real numbers.
    We define the \emph{closed interval} \([a, b]\) by
    \[
        [a, b] \coloneqq \{x \in \mathbf{R}^* : a \leq x \leq b\},
    \]
    the \emph{half-open intervals} \([a, b)\) and \((a, b]\) by
    \[
        [a, b) \coloneqq \{x \in \mathbf{R}^* : a \leq x < b\}; (a, b] \coloneqq \{x \in \mathbf{R}^* : a < x \leq b\},
    \]
    and the \emph{open interval} \((a, b)\) by
    \[
        (a, b) \coloneqq \{x \in \mathbf{R}^* : a < x < b\}.
    \]
    We call \(a\) the \emph{left endpoint} of these intervals, and \(b\) the \emph{right endpoint}.
\end{definition}

\begin{remark}\label{9.1.2}
    Once again, we are overloading the parenthesis notation;
    for instance, we are now using \((2, 3)\) to denote both an open interval from \(2\) to \(3\), as well as an ordered pair in the Cartesian plane \(\mathbf{R}^2 \coloneqq \mathbf{R} \times \mathbf{R}\).
    This can cause some genuine ambiguity, but the reader should still be able to resolve which meaning of the parentheses is intended from context.
    In some texts, this issue is resolved by using reversed brackets instead of parenthesis, thus for instance \([a, b)\) would now be \([a, b[\), \((a, b]\) would be \(]a, b]\), and \((a, b)\) would be \(]a, b[\).
\end{remark}

\begin{note}
    We sometimes refer to an interval in which one endpoint is infinite (either \(+\infty\) or \(-\infty\)) as \emph{half-infinite} intervals, and intervals in which both endpoints are infinite as \emph{doubly-infinite} intervals;
    all other intervals are \emph{bounded intervals}.
    Thus the positive and negative real axes are half-infinite intervals, and \(\mathbf{R}\) and \(\mathbf{R}^*\) are infinite intervals.
\end{note}

\setcounter{theorem}{3}
\begin{example}\label{9.1.4}
    If \(a > b\) then all four of the intervals \([a, b], [a, b), (a, b]\), and \((a, b)\) are the empty set (by trichotomy, see Proposition \ref{5.4.7}(a)).
    If \(a = b\), then the three intervals \([a, b), (a, b]\), and \((a, b)\) are the empty set, while \([a, b]\) is just the singleton set \(\{a\}\).
    Because of this, we call these intervals \emph{degenerate};
    most (but not all) of our analysis will be restricted to non-degenerate intervals.
\end{example}

\begin{definition}[\(\varepsilon\)-adherent points]\label{9.1.5}
    Let \(X\) be a subset of \(\mathbf{R}\), let \(\varepsilon > 0\), and let \(x \in \mathbf{R}\).
    We say that \(x\) is \emph{\(\varepsilon\)-adherent to \(X\)} iff there exists a \(y \in X\) which is \(\varepsilon\)-close to \(x\)
    (i.e., \(\abs*{x - y} \leq \varepsilon\)).
\end{definition}

\begin{remark}\label{9.1.6}
    The terminology ``\(\varepsilon\)-adherent'' is not standard in the literature.
    However, we shall shortly use it to define the notion of an adherent point, which is standard.
\end{remark}

\setcounter{theorem}{7}
\begin{definition}[Adherent points]\label{9.1.8}
    Let \(X\) be a subset of \(\mathbf{R}\), and let \(x \in \mathbf{R}\).
    We say that \(x\) is an \emph{adherent point} of \(X\) iff it is \(\varepsilon\)-adherent to \(X\) for every \(\varepsilon > 0\).
\end{definition}

\setcounter{theorem}{9}
\begin{definition}[Closure]\label{9.1.10}
    Let \(X\) be a subset of \(\mathbf{R}\).
    The \emph{closure} of \(X\), sometimes denoted \(\overline{X}\) is defined to be the set of all the adherent points of \(X\).
\end{definition}

\begin{lemma}[Elementary properties of closures]\label{9.1.11}
    Let \(X\) and \(Y\) be arbitrary subsets of \(\mathbf{R}\).
    Then \(X \subseteq \overline{X}\), \(\overline{X \cup Y} = \overline{X} \cup \overline{Y}\), and \(\overline{X \cap Y} \subseteq \overline{X} \cap \overline{Y}\).
    If \(X \subseteq Y\), then \(\overline{X} \subseteq \overline{Y}\).
\end{lemma}

\begin{proof}
    Let \(\varepsilon \in \mathbf{R}^+\).
    Since
    \begin{align*}
                 & \forall\ x \in X                                                        \\
        \implies & \abs*{x - x} = 0 \leq \varepsilon                                       \\
        \implies & x \in \overline{X},               & \text{(by Definition \ref{9.1.10})}
    \end{align*}
    by Definition \ref{3.1.15} we have \(X \subseteq \overline{X}\).
    Since
    \begin{align*}
                 & \forall\ x \in \overline{X \cup Y}                                                            \\
        \implies & \exists\ y \in X \cup Y : \abs*{x - y} \leq \varepsilon & \text{(by Definition \ref{9.1.10})} \\
        \implies & y \in X \lor y \in Y                                    & \text{(by Axiom \ref{3.4})}         \\
        \implies & x \in \overline{X} \lor x \in \overline{Y}              & \text{(by Definition \ref{9.1.10})} \\
        \implies & x \in \overline{X} \cup \overline{Y}                    & \text{(by Axiom \ref{3.4})}
    \end{align*}
    and
    \begin{align*}
                 & \forall\ x \in \overline{X} \cup \overline{Y}                                                 \\
        \implies & x \in \overline{X} \lor x \in \overline{Y}              & \text{(by Axiom \ref{3.4})}         \\
        \implies & (\exists\ y \in X : \abs*{x - y} \leq \varepsilon)                                            \\
                 & \lor (\exists\ y \in Y : \abs*{x - y} \leq \varepsilon) & \text{(by Definition \ref{9.1.10})} \\
        \implies & \exists\ y \in X \cup Y : \abs*{x - y} \leq \varepsilon & \text{(by Axiom \ref{3.4})}         \\
        \implies & x \in \overline{X \cup Y},                              & \text{(by Definition \ref{9.1.10})}
    \end{align*}
    by Proposition \ref{3.1.18} we have \(\overline{X \cup Y} = \overline{X} \cup \overline{Y}\).
    Since
    \begin{align*}
                 & \forall\ x \in \overline{X \cap Y}                                                            \\
        \implies & \exists\ y \in X \cap Y : \abs*{x - y} \leq \varepsilon & \text{(by Definition \ref{9.1.10})} \\
        \implies & y \in X \land y \in Y                                   & \text{(by Definition \ref{3.1.23})} \\
        \implies & x \in \overline{X} \land x \in \overline{Y}             & \text{(by Definition \ref{9.1.10})} \\
        \implies & x \in \overline{X} \cap \overline{Y},                   & \text{(by Definition \ref{3.1.23})}
    \end{align*}
    by Definition \ref{3.1.15} we have \(\overline{X \cap Y} \subseteq \overline{X} \cap \overline{Y}\).
    Now suppose that \(X \subseteq Y\).
    Then we have
    \begin{align*}
                 & \forall\ x \in \overline{X}                                                            \\
        \implies & \exists\ y \in X : \abs*{x - y} \leq \varepsilon & \text{(by Definition \ref{9.1.10})} \\
        \implies & y \in Y                                          & (X \subseteq X)                     \\
        \implies & x \in \overline{Y}.                              & \text{(by Definition \ref{9.1.10})}
    \end{align*}
    Thus by Definition \ref{3.1.15} we have \(\overline{X} \subseteq \overline{Y}\).
    And we conclude that \(X \subseteq Y \implies \overline{X} \subseteq \overline{Y}\).
\end{proof}

\begin{lemma}[Closures of intervals]\label{9.1.12}
    Let \(a < b\) be real numbers, and let \(I\) be any one of the four intervals \((a, b)\), \((a, b]\), \([a, b)\), or \([a, b]\).
    Then the closure of \(I\) is \([a, b]\).
    Similarly, the closure of \((a, \infty)\) or \([a, \infty)\) is \([a, \infty)\), while the closure of \((-\infty, a)\) or \((-\infty, a]\) is \((-\infty, a]\).
    Finally, the closure of \((-\infty, \infty)\) is \((-\infty, \infty)\).
\end{lemma}

\begin{proof}
    First let us show that every element of \([a, b]\) is adherent to \((a, b)\).
    Let \(x \in [a, b]\).
    If \(x \in (a, b)\) then it is definitely adherent to \((a, b)\).
    This is true since \(\forall\ \varepsilon \in \mathbf{R}^+\) we have \(\abs*{x - x} \leq \varepsilon\).
    If \(x = b\) then \(x\) is also adherent to \((a, b)\).
    Otherwise \(\exists\ \varepsilon \in \mathbf{R}^+\) such that
    \[
        \forall\ y \in (a, b), \abs*{x - y} > \varepsilon.
    \]
    But this means
    \begin{align*}
                 & \abs*{x - y} > \varepsilon                                                               \\
        \implies & \abs*{b - y} > \varepsilon                        & (x = b)                              \\
        \implies & b - y > \varepsilon                               & (y < b)                              \\
        \implies & b - \varepsilon > y                                                                      \\
        \implies & b > b - \varepsilon > y > a                       & (\varepsilon > 0 \land y \in (a, b)) \\
        \implies & b - \varepsilon \in (a, b)                        & \text{(by Definition \ref{9.1.1})}   \\
        \implies & 0 = \abs*{b - (b - \varepsilon)} \leq \varepsilon                                        \\
        \implies & \abs*{x - (b - \varepsilon)} \leq \varepsilon,
    \end{align*}
    a contradiction.
    Thus \(x = b\) implies \(x\) is also adherent to \((a, b)\).
    Similarly when \(x = a\).
    Thus every point in \([a, b]\) is adherent to \((a, b)\).

    Now we show that every point \(x\) that is adherent to \((a, b)\) lies in \([a, b]\).
    Suppose for sake of contradiction that \(x\) does not lie in \([a, b]\), then either \(x > b\) or \(x < a\).
    If \(x > b\) then \(x\) is not \((x - b)\)-adherent to \((a, b)\), and is hence not an adherent point to \((a, b)\)
    (by setting \(\varepsilon = x - b\) we have \(\forall\ y \in (a, b)\), \(\abs*{x - y} = x - y > x - b = \varepsilon\)).
    Similarly, if \(x < a\), then \(x\) is not \((a - x)\)-adherent to \((a, b)\), and is hence not an adherent point to \((a, b)\).
    This contradiction shows that \(x\) is in fact in \([a, b]\) as claimed.

    Using similar arguments we can show that every point in \([a, b]\) is also adherent to \((a, b]\) and \([a, b)\), and every point \(x\) that is adherent to \((a, b]\) and \([a, b)\) lies in \([a, b]\).

    Now we show that the closure of \((a, \infty)\) or \([a, \infty)\) is \([a, \infty)\).
    From the proof above we know that the closure of \((a, \infty)\) and \([a, \infty)\) is \([a, \infty]\).
    But since \(\forall\ x \in (a, \infty)\) the statement \(\abs*{x - \infty} \leq \varepsilon\) is undefined, thus we can only have \([a, \infty)\) as the closure of \((a, \infty)\) and \([a, \infty)\).
                    Similar arguments show that the closure of \((-\infty, a)\) or \((-\infty, a]\) is \((-\infty, a]\), and the closure of \((-\infty, \infty)\) is \((-\infty, \infty)\).
\end{proof}

\begin{lemma}\label{9.1.13}
    The closure of \(\mathbf{N}\) is \(\mathbf{N}\).
    The closure of \(\mathbf{Z}\) is \(\mathbf{Z}\).
    The closure of \(\mathbf{Q}\) is \(\mathbf{R}\), and the closure of \(\mathbf{R}\) is \(\mathbf{R}\).
    The closure of the empty set \(\emptyset\) is \(\emptyset\).
\end{lemma}

\begin{proof}
    We first show that \(\overline{\mathbf{N}} = \mathbf{N}\).
    Let \(\overline{\mathbf{N}}\) be the closure of \(\mathbf{N}\).
    By Lemma \ref{9.1.11} we have \(\mathbf{N} \subseteq \overline{\mathbf{N}}\).
    Since
    \begin{align*}
                 & \forall\ n \in \mathbf{N}                                      \\
        \implies & 0 \leq n < \infty                                              \\
        \implies & n \in [0, \infty),        & \text{(by Definition \ref{9.1.1})}
    \end{align*}
    by Definition \ref{3.1.15} we have \(\mathbf{N} \subseteq [0, \infty)\), thus
    \begin{align*}
                 & \mathbf{N} \subseteq [0, \infty)                                                        \\
        \implies & \overline{\mathbf{N}} \subseteq \overline{[0, \infty)} & \text{(by Lemma \ref{9.1.11})} \\
        \implies & \overline{\mathbf{N}} \subseteq [0, \infty)            & \text{(by Lemma \ref{9.1.12})} \\
        \implies & \overline{\mathbf{N}} \subseteq \mathbf{R}.
    \end{align*}
    Now we show that \(\overline{\mathbf{N}} \subseteq \mathbf{N}\).
    Suppose for sake of contradiction that \(\overline{\mathbf{N}} \not\subseteq \mathbf{N}\), i.e., \(\exists\ x \in \overline{\mathbf{N}}\) such that \(x \notin \mathbf{N}\).
    Then we have \(x > 0\) and by Proposition \ref{5.4.12} \(\exists\ n \in \mathbf{N} : n < x < n + 1\).
    Let \(\varepsilon = \min(x - n, n + 1 - x) / 2\).
    By Definition \ref{9.1.10}, \(\exists\ m \in \mathbf{N}\) such that \(\abs*{x - m} \leq \varepsilon\).
    We now split into three cases:
    \begin{enumerate}
        \item If \(m = n\), then we have \(x - n \geq \min(x - n, n + 1 - x) > \varepsilon\), a contradiction.
        \item If \(m < n\), then we have \(x - m > x - n \geq \min(x - n, n + 1 - x) > \varepsilon\), a contradiction.
        \item If \(m > n\), then we have \(m \geq n + 1\) and \(m - x \geq n + 1 - x \geq \min(x - n, n + 1 - x) > \varepsilon\), a contradiction.
    \end{enumerate}
    From all cases above we derived contradictions.
    Thus such \(m\) does not exists and by Definition \ref{9.1.10} \(x \notin \overline{\mathbf{N}}\).
    So we have \(\overline{\mathbf{N}} \subseteq \mathbf{N}\).
    Since \(\mathbf{N} \subseteq \overline{\mathbf{N}} \land \overline{\mathbf{N}} \subseteq \mathbf{N}\), by Proposition \ref{3.1.18} we have \(\mathbf{N} = \overline{\mathbf{N}}\).

    Next we show that \(\overline{\mathbf{Z}} = \mathbf{Z}\).
    Let \(\overline{\mathbf{Z}}\) be the closure of \(\mathbf{Z}\).
    Let \(\mathbf{Z}^- = \{z \in \mathbf{Z} : z < 0\}\).
    Then we have
    \begin{align*}
        \overline{\mathbf{Z}} & = \overline{\mathbf{N} \cup \mathbf{Z}^-}                                             \\
                              & = \overline{\mathbf{N}} \cup \overline{\mathbf{Z}^-} & \text{(by Lemma \ref{9.1.11})} \\
                              & = \mathbf{N} \cup \overline{\mathbf{Z}^-}            & \text{(from proof above)}
    \end{align*}
    Thus to show that \(\mathbf{Z} = \overline{\mathbf{Z}}\) it is suffice to show that \(\mathbf{Z}^- = \overline{\mathbf{Z}^-}\).
    By Lemma \ref{9.1.11} we have \(\mathbf{Z}^- \subseteq \overline{\mathbf{Z}^-}\).
    We now to show that \(\overline{\mathbf{Z}^-} \subseteq \mathbf{Z}^-\).
    Suppose for sake of contradiction that \(\overline{\mathbf{Z}^-} \not\subseteq \mathbf{Z}^-\), i.e., \(\exists\ x \in \overline{\mathbf{Z}^-}\) such that \(x \notin \mathbf{Z}^-\).
    Then we have \(x < 0\) and by Proposition \ref{5.4.12} \(\exists\ n \in \mathbf{Z}^- : n < x < n + 1\).
    Let \(\varepsilon = \min(x - n, n + 1 - x) / 2\).
    By Definition \ref{9.1.10}, \(\exists\ m \in \mathbf{Z}^-\) such that \(\abs*{x - m} \leq \varepsilon\).
    We now split into three cases:
    \begin{enumerate}
        \item If \(m = n\), then we have \(x - n \geq \min(x - n, n + 1 - x) > \varepsilon\), a contradiction.
        \item If \(m < n\), then we have \(x - m > x - n \geq \min(x - n, n + 1 - x) > \varepsilon\), a contradiction.
        \item If \(m > n\), then we have \(m \geq n + 1\) and \(m - x \geq n + 1 - x \geq \min(x - n, n + 1 - x) > \varepsilon\), a contradiction.
    \end{enumerate}
    From all cases above we derived contradictions.
    Thus such \(m\) does not exists and by Definition \ref{9.1.10} \(x \notin \overline{\mathbf{Z}^-}\).
    So we have \(\overline{\mathbf{Z}^-} \subseteq \mathbf{Z}^-\).
    Since \(\mathbf{Z}^- \subseteq \overline{\mathbf{Z}^-} \land \overline{\mathbf{Z}^-} \subseteq \mathbf{Z}^-\), by Proposition \ref{3.1.18} we have \(\mathbf{Z}^- = \overline{\mathbf{Z}^-}\), and thus \(\mathbf{Z} = \overline{\mathbf{Z}}\).

    Next we show that \(\overline{\mathbf{Q}} = \mathbf{R}\).
    Let \(\overline{\mathbf{Q}}\) be the closure of \(\mathbf{Q}\).
    We have
    \begin{align*}
                 & \mathbf{Q} \subseteq \mathbf{R}                                                                   \\
        \implies & \mathbf{Q} \subseteq (-\infty, \infty)                       & \text{(by Definition \ref{9.1.1})} \\
        \implies & \overline{\mathbf{Q}} \subseteq \overline{(-\infty, \infty)} & \text{(by Lemma \ref{9.1.11})}     \\
        \implies & \overline{\mathbf{Q}} \subseteq (-\infty, \infty)            & \text{(by Lemma \ref{9.1.12})}     \\
        \implies & \overline{\mathbf{Q}} \subseteq \mathbf{R}.
    \end{align*}
    Since
    \begin{align*}
                 & \forall\ x \in \mathbf{R}                                                                                            \\
        \implies & \forall\ \varepsilon \in \mathbf{R}^+ : x - \varepsilon < x < x + \varepsilon                                        \\
        \implies & \exists\ q \in \mathbf{Q} : x - \varepsilon < q < x + \varepsilon             & \text{(by Proposition \ref{5.4.14})} \\
        \implies & \abs*{x - q} < \varepsilon                                                                                           \\
        \implies & x \in \overline{\mathbf{Q}},                                                  & \text{(by Definition \ref{9.1.10})}
    \end{align*}
    by Definition \ref{3.1.15} we have \(\mathbf{R} \subseteq \overline{\mathbf{Q}}\).
    Since \(\mathbf{R} \subseteq \overline{\mathbf{Q}} \land \overline{\mathbf{Q}} \subseteq \mathbf{R}\), by Proposition \ref{3.1.18} we have \(\mathbf{R} = \overline{\mathbf{Q}}\).

    Next we show that \(\overline{\mathbf{R}} = \mathbf{R}\).
    Since
    \begin{align*}
             & \forall\ x \in \mathbf{R}                                              \\
        \iff & -\infty < x < \infty                                                   \\
        \iff & x \in (\infty, \infty)            & \text{(by Definition \ref{9.1.1})} \\
        \iff & x \in \overline{(\infty, \infty)} & \text{(by Lemma \ref{9.1.12})}     \\
        \iff & x \in \overline{\mathbf{R}},
    \end{align*}
    we know that \(\overline{\mathbf{R}} = \mathbf{R}\).

    Finally we show that \(\overline{\emptyset} = \emptyset\).
    Suppose for sake of contradiction that \(\overline{\emptyset} \neq \emptyset\).
    Let \(x \in \overline{\emptyset}\)
    Then by Definition \ref{9.1.10} \(\forall\ \varepsilon \in \mathbf{R}^+\), \(\exists\ y \in \emptyset\) such that \(\abs*{x - y} \leq \varepsilon\), a contradiction.
    Thus \(\overline{\emptyset} = \emptyset\).
\end{proof}

\begin{lemma}\label{9.1.14}
    Let \(X\) be a subset of \(\mathbf{R}\), and let \(x \in \mathbf{R}\).
    Then \(x\) is an adherent point of \(X\) if and only if there exists a sequence \((a_n)_{n = 0}^\infty\), consisting entirely of elements in \(X\), which converges to \(x\).
\end{lemma}

\begin{proof}
    We first show that if \(x\) is an adherent point of \(X\), then there exists a sequence \((a_n)_{n = 0}^\infty\) such that \(\forall\ n \in \mathbf{N}\), \(a_n \in X\) and \(\lim_{n \to \infty} a_n = x\).
    For each \(n \in \mathbf{N}\) let \(A_n\) be a set where
    \[
        A_n = \{y \in X : \abs*{x - y} \leq \frac{1}{n}\}.
    \]
    We know by Definition \ref{9.1.10} that \(A_n \neq \emptyset\).
    By axiom of choice (Axiom \ref{8.1}) we know \(\prod_{n \in \mathbf{N}} A_n \neq \emptyset\).
    Let \(f \in \prod_{n \in \mathbf{N}} A_n\).
    We can define a sequence \((a_n)_{n = 0}^\infty\) by setting \(a_n = f(n)\).
    Then we have
    \begin{align*}
                 & \forall\ n \in \mathbf{N}                                                                                           \\
        \implies & 0 \leq \abs*{x - a_n} \leq \frac{1}{n}                                                                              \\
        \implies & \lim_{n \to \infty} \abs*{x - a_n} = 0                                         & \text{(by Corollary \ref{6.4.14})} \\
        \implies & \lim_{n \to \infty} \max(x - a_n, a_n - x) = 0                                                                      \\
        \implies & \max(\lim_{n \to \infty} x - a_n, \lim_{n \to \infty} a_n - x) = 0             & \text{(by Theorem \ref{6.1.19})}   \\
        \implies & \max\big(x - (\lim_{n \to \infty} a_n), (\lim_{n \to \infty} a_n) - x\big) = 0 & \text{(by Theorem \ref{6.1.19})}   \\
        \implies & x = \lim_{n \to \infty} a_n.
    \end{align*}

    Now we show that if there exists a sequence \((a_n)_{n = 0}^\infty\) such that \(\forall\ n \in \mathbf{N}\), \(a_n \in X\) and \(\lim_{n \to \infty} a_n = x\), then \(x\) is an adherent point of \(X\).
    Since \(\lim_{n \to \infty} a_n = x\), by Proposition \ref{6.4.5} \(x\) is the only limit point of \((a_n)_{n = m}^\infty\).
    So we have
    \begin{align*}
                 & \forall\ \varepsilon \in \mathbf{R}^+, \exists\ n \in \mathbf{N} : \abs*{x - a_n} \leq \varepsilon & \text{(by Definition \ref{6.4.1})}  \\
        \implies & \forall\ \varepsilon \in \mathbf{R}^+, \exists\ a_n \in X : \abs*{x - a_n} \leq \varepsilon                                              \\
        \implies & x \in \overline{X}.                                                                                & \text{(by Definition \ref{9.1.10})}
    \end{align*}
    And we conclude that \(x\) is an adherent point of \(X\) if and only if there exists a sequence \((a_n)_{n = 0}^\infty\) such that \(\forall\ n \in \mathbf{N}\), \(a_n \in X\) and \(\lim_{n \to \infty} a_n = x\).
\end{proof}

\begin{definition}\label{9.1.15}
    A subset \(E \subseteq \mathbf{R}\) is said to be \emph{closed} if \(\overline{E} = E\), or in other words that \(E\) contains all of its adherent points.
\end{definition}

\begin{example}\label{9.1.16}
    From Lemma \ref{9.1.12} we see that if \(a < b\) are real numbers, then \([a, b]\), \([a, +\infty)\), \((-\infty, a]\), and \((-\infty, +\infty)\) are closed, while \((a, b)\), \((a, b]\), \([a, b)\), \((a, +\infty)\), and \((-\infty, a)\) are not.
    From Lemma \ref{9.1.13} we see that \(\mathbf{N}\), \(\mathbf{Z}\), \(\mathbf{R}\), \(\emptyset\) are closed, while \(\mathbf{Q}\) is not.
\end{example}

\begin{corollary}\label{9.1.17}
    Let \(X\) be a subset of \(\mathbf{R}\).
    If \(X\) is closed, and \((a_n)_{n = 0}^\infty\) is a convergent sequence consisting of elements in \(X\), then \(\lim_{n \to \infty} a_n\) also lies in \(X\).
    Conversely, if it is true that every convergent sequence \((a_n)_{n = 0}^\infty\) of elements in \(X\) has its limit in \(X\) as well, then \(X\) is necessarily closed.
\end{corollary}

\begin{proof}
    We first show that if \(X\) is closed, and \((a_n)_{n = 0}^\infty\) is a convergent sequence consisting of elements in \(X\), then \(\lim_{n \to \infty} a_n\) also lies in \(X\).
    Let \(x = \lim_{n \to \infty} a_n\).
    Then we have
    \begin{align*}
                 & \forall\ \varepsilon \in \mathbf{R}^+, \exists\ n \in \mathbf{N} : \abs*{x - a_{n'}} \leq \varepsilon \ \forall\ n' \in \mathbf{N} \land n' \geq n                                       \\
        \implies & \forall\ \varepsilon \in \mathbf{R}^+, \exists\ a_n \in X : \abs*{x - a_n} \leq \varepsilon                                                                                              \\
        \implies & x \in \overline{X}                                                                                                                                 & \text{(by Definition \ref{9.1.10})} \\
        \implies & x \in X.                                                                                                                                           & \text{(by Definition \ref{9.1.15})}
    \end{align*}

    Now we show that if every convergent sequence \((a_n)_{n = 0}^\infty\) of elements in \(X\) has its limit in \(X\) as well, then \(X\) is closed.
    By Lemma \ref{9.1.11} we have \(X \subseteq \overline{X}\).
    Since
    \begin{align*}
                 & \forall\ x \in \overline{X}                                                                                                                   \\
        \implies & \exists\ (a_n)_{n = 0}^\infty : (\forall\ n \in \mathbf{N}, a_n \in X) \land (\lim_{n \to \infty} a_n = x) & \text{(by Lemma \ref{9.1.14})}   \\
        \implies & x \in X,                                                                                                   & \text{(by the given hypothesis)}
    \end{align*}
    by Definition \ref{3.1.15} we have \(\overline{X} \subseteq X\).
    Since \(X \subseteq \overline{X} \land \overline{X} \subseteq X\), by Proposition \ref{3.1.8} we have \(X = \overline{X}\), and thus by Definition \ref{9.1.15} \(X\) is closed.
\end{proof}

\begin{definition}[Limit points]\label{9.1.18}
    Let \(X\) be a subset of the real line.
    We say that \(x\) is a \emph{limit point} (or a \emph{cluster point}) of \(X\) iff it is an adherent point of \(X \setminus \{x\}\).
    We say that \(x\) is an \emph{isolated point} of \(X\) if \(x \in X\) and there exists some \(\varepsilon > 0\) such that \(\abs*{x - y} > \varepsilon\) for all \(y \in X \setminus \{x\}\).
\end{definition}

\setcounter{theorem}{19}
\begin{remark}\label{9.1.20}
    From Lemma \ref{9.1.14} we see that \(x\) is a limit point of \(X\) iff there exists a sequence \((a_n)_{n = 0}^\infty\), consisting entirely of elements in \(X\) that are distinct from \(x\), and such that \((a_n)_{n = 0}^\infty\) converges to \(x\).
    It turns out that the set of adherent points splits into the set of limit points and the set of isolated points.
\end{remark}

\begin{lemma}\label{9.1.21}
    Let \(I\) be an interval (possibly infinite), i.e., \(I\) is a set of the form \((a, b)\), \((a, b]\), \([a, b)\), \([a, b]\), \((a, +\infty)\), \([a, +\infty)\), \((-\infty, a)\), or \((-\infty, a]\), with \(a < b\) in the first four cases.
    Then every element of \(I\) is a limit point of \(I\).
\end{lemma}

\begin{proof}
    We show this for the case \(I = [a, b]\);
    the other cases are similar.
    Let \(x \in I\);
    we have to show that \(x\) is a limit point of \(I\).
    There are three cases: \(x = a\), \(a < x < b\), and \(x = b\).
    If \(x = a\), then consider the sequence \((x + \frac{1}{n})_{n = N}^\infty\).
    This sequence converges to \(x\), and will lie inside \(I - \{a\} = (a, b]\) if \(N\) is chosen large enough (by Proposition \ref{5.4.14}).
    Thus by Remark \ref{9.1.20} we see that \(x = a\) is a limit point of \([a, b]\).
    A similar argument works when \(a < x < b\).
    When \(x = b\) one has to use the sequence \((x - \frac{1}{n})_{n = N}^\infty\) instead.
    This sequence converges to \(x\), and will lie inside \(I - \{b\} = [a, b)\) if \(N\) is chosen large enough (by Proposition \ref{5.4.14}).
    Thus by Remark \ref{9.1.20} we see that \(x = b\) is a limit point of \([a, b]\).
\end{proof}

\begin{definition}[Bounded sets]\label{9.1.22}
    A subset \(X\) of the real line is said to be \emph{bounded} if we have \(X \subseteq [-M, M]\) for some real number \(M > 0\).
\end{definition}

\begin{example}\label{9.1.23}
    For any real numbers \(a, b\), the interval \([a, b]\) is bounded, because it is contained inside \([-M, M]\), where \(M \coloneqq \max(\abs*{a}, \abs*{b})\).
    However, the half-infinite interval \([0, +\infty)\) is unbounded.
    In fact, no half-infinite interval or doubly infinite interval can be bounded.
    The sets \(\mathbf{N}\), \(\mathbf{Z}\), \(\mathbf{Q}\), and \(\mathbf{R}\) are all unbounded.
\end{example}

\begin{theorem}[Heine-Borel theorem for the line]\label{9.1.24}
    Let \(X\) be a subset of \(\mathbf{R}\).
    Then the following two statements are equivalent:
    \begin{enumerate}
        \item \(X\) is closed and bounded.
        \item Given any sequence \((a_n)_{n = 0}^\infty\) of real numbers which takes values in \(X\) (i.e., \(a_n \in X\) for all \(n\)), there exists a subsequence \((a_{n_j})_{j = 0}^\infty\) of the original sequence, which converges to some number \(L\) in \(X\).
    \end{enumerate}
\end{theorem}

\begin{proof}
    We first show that statement (a) implies statement (b).
    Suppose that \(X\) is a set such that \(X\) is closed and bounded.
    Let \((a_n)_{n = 0}^\infty\) be a sequence where \(\forall\ n \in \mathbf{N}\), \(a_n \in X\).
    Since \(X\) is bounded, by Definition \ref{9.1.22} \(\exists\ M \in \mathbf{R}^+\) such that \(X \subseteq [-M, M]\), thus \((a_n)_{n = 0}^\infty\) is also bounded by \(M\), i.e., \(\forall\ n \in \mathbf{N}\), \(\abs*{a_n} \leq M\).
    By Bolzano-Weierstrass theorem (Theorem \ref{6.6.8}) we know that there exists a subsequence \((a_{n_j})_{j = 0}^\infty\) of \((a_n)_{n = 0}^\infty\) such that \((a_{n_j})_{j = 0}^\infty\) converges.
    Since \(X\) is closed, by Corollary \ref{9.1.17} we know that \(\lim_{j \to \infty} a_{n_j} \in X\).

    Now we show that statement (b) implies statement (a).
    Suppose for sake of contradiction that statement (b) does not implies statement (a).
    Then \(X\) is not closed or \(X\) is unbounded.
    Since given any sequence \((a_n)_{n = 0}^\infty\) we can always find a subsequence \((a_{n_j})_{j = 0}^\infty\) such that \(\lim_{j \to \infty} a_{n_j} \in X\), we know that if \((a_n)_{n = 0}^\infty\) converges then \(\lim_{n \to \infty} a_n \in X\).
    Thus we have every convergent sequence \((a_n)_{n = 0}^\infty\) have its limit in \(X\), and by Corollary \ref{9.1.17} we know that \(X\) is closed.
    Then we must have \(X\) is unbounded, i.e., \(\nexists\ M \in \mathbf{R}^+\) such that \(X \subseteq [-M, M]\).
    Now we define \(X_n = \{x \in X : \abs*{x} > n\}\) for every \(n \in \mathbf{N}\).
    We know that \(X_n \neq \emptyset\) since \(X_n\) is unbounded.
    By axiom of choice (Axiom \ref{8.1}) we know that \(\prod_{n \in \mathbf{N}} X_n \neq \emptyset\).
    Let \(f \in \prod_{n \in \mathbf{N}} X_n\).
    We can define a sequence \((a_n)_{n = 0}^\infty\) by setting \(a_n = f(n)\).
    By hypothesis we know that there exists a subsequence \((a_{n_j})_{j = 0}^\infty\) such that \(L = \lim_{j \to \infty} a_{n_j} \in X\).
    But this means
    \begin{align*}
                 & \forall\ n \in \mathbf{N} \land n \geq \abs*{L} + 1         & \text{(by Proposition \ref{5.4.12})} \\
        \implies & \abs*{a_n} > n \geq \abs*{L} + 1                            & (a_n \in X_n)                        \\
        \implies & \forall\ n_j > \abs*{L} + 1 : \abs*{a_{n_j}} > \abs*{L} + 1                                        \\
        \implies & \abs*{a_{n_j}} - \abs*{L} > 1                                                                      \\
        \implies & \abs*{a_{n_j} - L} \geq \abs*{a_{n_j}} - \abs*{L} > 1                                              \\
        \implies & \lim_{j \to \infty} a_{n_j} \neq L,
    \end{align*}
    a contradiction.
    Thus \(X\) is closed and bounded.
\end{proof}
\section{The algebra of real-valued functions}\label{sec 9.2}

\begin{note}
    We can take any one of the previous functions \(f : \mathbf{R} \to \mathbf{R}\) defined on all of \(\mathbf{R}\), and restrict the domain to a smaller set \(X \subseteq \mathbf{R}\), creating a new function, sometimes called \(f|_X\), from \(X\) to \(\mathbf{R}\).
    This is the same function as the original function \(f\), but is only defined on a smaller domain.
    (Thus \(f|_X(x) \coloneqq f(x)\) when \(x \in X\), and \(f|_X(x)\) is undefined when \(x \notin X\).)
\end{note}

\begin{note}
    If \(X\) is a subset of \(\mathbf{R}\), and \(f : X \to \mathbf{R}\) is a function, we can form the graph \(\{(x, f(x)) : x \in X\}\) of the function \(f\);
    this is a subset of \(X \times \mathbf{R}\), and hence a subset of the Euclidean plane \(\mathbf{R}^2 = \mathbf{R} \times \mathbf{R}\).
    One can certainly study a function through its graph, by using the geometry of the plane \(\mathbf{R}^2\)
    (e.g., employing such concepts as tangent lines, area, and so forth).
    We however will pursue a more ``analytic'' approach, in which we rely instead on the properties of the real numbers to analyze these functions.
    The two approaches are complementary;
    the geometric approach offers more visual intuition, while the analytic approach offers rigour and precision.
    Both the geometric intuition and the analytic formalism become useful when extending analysis of functions of one variable to functions of many variables
    (or possibly even infinitely many variables).
\end{note}

\begin{definition}[Arithmetic operations on functions]\label{9.2.1}
    Given two functions \(f : X \to \mathbf{R}\) and \(g : X \to \mathbf{R}\), we can define their sum \(f + g : X \to \mathbf{R}\) by the formula
    \[
        (f + g)(x) \coloneqq f(x) + g(x),
    \]
    their difference \(f - g : X \to \mathbf{R}\) by the formula
    \[
        (f - g)(x) \coloneqq f(x) - g(x),
    \]
    their maximum \(\max(f, g) : X \to \mathbf{R}\) by
    \[
        \max(f, g)(x) \coloneqq \max(f(x), g(x)),
    \]
    their minimum \(\min(f, g) : X \to \mathbf{R}\) by
    \[
        \min(f, g)(x) \coloneqq \min(f(x), g(x)),
    \]
    their product \(fg : X \to \mathbf{R}\) (or \(f \cdot g : X \to \mathbf{R}\)) by the formula
    \[
        (fg)(x) \coloneqq f(x)g(x),
    \]
    and (provided that \(g(x) = 0\) for all \(x \in X\)) the quotient \(f / g : X \to \mathbf{R}\) by the formula
    \[
        (f / g)(x) \coloneqq f(x) / g(x).
    \]
    Finally, if \(c\) is a real number, we can define the function \(cf : X \to \mathbf{R}\) (or \(c \cdot f : X \to \mathbf{R}\)) by the formula
    \[
        (cf)(x) \coloneqq c \times f(x).
    \]
\end{definition}
\section{Limiting values of functions}\label{sec 9.3}

\begin{definition}[\(\varepsilon\)-closeness]\label{9.3.1}
    Let \(X\) be a subset of \(\mathbf{R}\), let \(f : X \to \mathbf{R}\) be a function, let \(L\) be a real number, and let \(\varepsilon > 0\) be a real number.
    We say that the function \(f\) is \emph{\(\varepsilon\)-close to \(L\)} iff \(f(x)\) is \(\varepsilon\)-close to \(L\) for every \(x \in X\).
\end{definition}

\setcounter{theorem}{2}
\begin{definition}[Local \(\varepsilon\)-closeness]\label{9.3.3}
    Let \(X\) be a subset of \(\mathbf{R}\), let \(f : X \to \mathbf{R}\) be a function, let \(L\) be a real number, \(x_0\) be an adherent point of \(X\), and \(\varepsilon > 0\) be a real number.
    We say that \(f\) is \emph{\(\varepsilon\)-close to \(L\) near \(x_0\)} iff there exists a \(\delta > 0\) such that \(f\) becomes \(\varepsilon\)-close to \(L\) when restricted to the set \(\{x \in X : \abs*{x - x_0} < \delta\}\).
\end{definition}

\setcounter{theorem}{5}
\begin{definition}[Convergence of functions at a point]\label{9.3.6}
    Let \(X\) be a subset of \(\mathbf{R}\), let \(f : X \to \mathbf{R}\) be a function, let \(E\) be a subset of \(X\), \(x_0\) be an adherent point of \(E\), and let \(L\) be a real number.
    We say that \emph{\(f\) converges to \(L\) at \(x_0\) in \(E\)}, and write \(\lim_{x \to x_0 ; x \in E} f(x) = L\), iff \(f\), after restricting to \(E\), is \(\varepsilon\)-close to \(L\) near \(x_0\) for every \(\varepsilon > 0\).
    If \(f\) does not converge to any number \(L\) at \(x_0\), we say that \emph{\(f\) diverges at \(x_0\)}, and leave \(\lim_{x \to x_0 ; x \in E} f(x)\) undefined.
\end{definition}

\begin{note}
    In other words, we have \(\lim_{x \to x_0 ; x \in E} f(x) = L\) iff for every \(\varepsilon > 0\), there exists a \(\delta > 0\) such that \(\abs*{f(x) - L} \leq \varepsilon\) for all \(x \in E\) such that \(\abs*{x - x_0} < \delta\).
\end{note}

\begin{remark}\label{9.3.7}
    In many cases we will omit the set \(E\) from the above notation (i.e., we will just say that \(f\) converges to \(L\) at \(x_0\), or that \(\lim_{x \to x_0} f(x) = L\)), although this is slightly dangerous.
    For instance, it sometimes makes a difference whether \(E\) actually contains \(x_0\) or not.
    To give an example, if \(f : \mathbf{R} \to \mathbf{R}\) is the function defined by setting \(f(x) = 1\) when \(x = 0\) and \(f(x) = 0\) when \(x \neq 0\), then one has \(\lim_{x \to 0 ; x \in \mathbf{R} \setminus \{0\}} f(x) = 0\), but \(\lim_{x \to 0 ; x \in \mathbf{R}} f(x)\) is undefined.
    Some authors only define the limit \(\lim_{x \to x_0 ; x \in E} f(x)\) when \(E\) does not contain \(x_0\) (so that \(x_0\) is now a limit point of \(E\) rather than an adherent point), or would use \(\lim_{x \to x_0 ; x \in E} f(x)\) to denote what we would call \(\lim_{x \to x_0 ; x \in E \setminus \{x_0\}} f(x)\), but we have chosen a slightly more general notation, which allows the possibility that \(E\) contains \(x_0\).
\end{remark}

\setcounter{theorem}{8}
\begin{proposition}\label{9.3.9}
    Let \(X\) be a subset of \(\mathbf{R}\), let \(f : X \to \mathbf{R}\) be a function, let \(E\) be a subset of \(X\), let \(x_0\) be an adherent point of \(E\), and let \(L\) be a real number.
    Then the following two statements are logically equivalent:
    \begin{enumerate}
        \item \(f\) converges to \(L\) at \(x_0\) in \(E\).
        \item For every sequence \((a_n)_{n = 0}^\infty\) which consists entirely of elements of \(E\) and converges to \(x_0\), the sequence \((f(a_n))_{n = 0}^\infty\) converges to \(L\).
    \end{enumerate}
\end{proposition}

\begin{proof}
    We first show that statement (a) implies statement (b).
    Since \(f\) converges to \(L\) at \(x_0\) in \(E\), by Definition \ref{9.3.6} we have
    \[
        \forall\ \varepsilon \in \mathbf{R}^+, \exists\ \delta \in \mathbf{R}^+ : \big(\forall\ x \in E, \abs*{x - x_0} < \delta \implies \abs*{f(x) - L} \leq \varepsilon\big).
    \]
    Now we fix \(\varepsilon\), and we have some \(\delta\) satisfying the statement above, we also fix such \(\delta\).
    Let \((a_n)_{n = 0}^\infty\) be a sequence which consists entirely of elements of \(E\) and \(\lim_{n \to \infty} a_n = x_0\).
    Such sequence exists since Lemma \ref{9.1.14}.
    By Definition \ref{6.1.5} we have
    \[
        \forall\ \varepsilon' \in \mathbf{R}^+, \exists\ N \in \mathbf{N} : \forall\ n \geq N, \abs*{a_n - x_0} \leq \varepsilon'.
    \]
    In particular, we have
    \[
        \exists\ N \in \mathbf{N} : \forall\ n \geq N, \abs*{a_n - x_0} \leq \frac{\delta}{2} < \delta.
    \]
    Since \((a_n)_{n = 0}^\infty\) consists entirely of elements of \(E\), we have
    \[
        \abs*{a_n - x_0} < \delta \implies \abs*{f(a_n) - L} \leq \varepsilon.
    \]
    Since \(\varepsilon\) is arbitrary, we have
    \[
        \forall\ \varepsilon \in \mathbf{R}^+, \exists\ N \in \mathbf{N} : \forall\ n \geq N, \abs*{f(a_n) - L} \leq \varepsilon
    \]
    and by Definition \ref{6.1.5} we have \(\lim_{n \to \infty} f(a_n) = L\).

    Now we show that statement (b) implies statement (a).
    Suppose for sake of contradiction that \(f\) does not converge to \(L\) at \(x_0\) in \(E\).
    Then by Definition \ref{9.3.6} we have
    \[
        \exists\ \varepsilon \in \mathbf{R}^+ : \forall\ \delta \in \mathbf{R}^+, \Big(\forall\ x \in E, (\abs*{x - x_0} < \delta) \land \big(\abs*{f(x) - L} > \varepsilon\big)\Big).
    \]
    Let \((a_n)_{n = 0}^\infty\) be a sequence which consists entirely of elements of \(E\) and \(\lim_{n \to \infty} a_n = x_0\).
    By hypothesis we have \(\lim_{n \to \infty} f(a_n) = L\).
    By Definition \ref{6.1.5} the following two statements are true:
    \begin{align*}
         & \exists\ N_1 \in \mathbf{N} : \forall\ n_1 \geq N_1, \abs*{a_{n_1} - x_0} \leq \frac{\delta}{2} < \delta \\
         & \exists\ N_2 \in \mathbf{N} : \forall\ n_2 \geq N_2, \abs*{f(a_{n_2}) - L} \leq \varepsilon
    \end{align*}
    Let \(N = \max(N_1, N_2)\).
    Then we have
    \[
        \forall\ n \geq N, (\abs*{a_n - x_0} < \delta) \land \big(\abs*{f(a_n) - L} \leq \varepsilon\big).
    \]
    But \(a_n \in E\), so this contradict to \(\abs*{f(a_n) - L} > \varepsilon\).
    Thus \(\lim_{x \to x_0 ; x \in E} f(x) = L\).
\end{proof}

\begin{note}
    In view of Proposition \ref{9.3.9}, we will sometimes write ``\(f(x) \to L\) as \(x \to x_0\) in \(E\)'' or ``\(f\) has a limit \(L\) at \(x_0\) in \(E\)'' instead of ``\(f\) converges to \(L\) at \(x_0\)'', or ``\(\lim_{x \to x_0} f(x) = L\)''.
\end{note}

\begin{remark}\label{9.3.10}
    With the notation of Proposition \ref{9.3.9}, we have the following corollary:
    if \(\lim_{x \to x_0 ; x \in E} f(x) = L\), and \(\lim_{n \to \infty} a_n = x_0\), then \(\lim_{n \to \infty} f(a_n) = L\).
\end{remark}

\begin{remark}\label{9.3.11}
    We only consider limits of a function \(f\) at \(x_0\) in the case when \(x_0\) is an adherent point of \(E\).
    When \(x_0\) is not an adherent point then it is not worth it to define the concept of a limit.
\end{remark}

\begin{remark}\label{9.3.12}
    The variable \(x\) used to denote a limit is a dummy variable;
    we could replace it by any other variable and obtain exactly the same limit.
    For instance, if \(\lim_{x \to x_0 ; x \in E} f(x) = L\), then \(\lim_{y \to x_0 ; y \in E} f(y) = L\), and conversely.
    (Since \(x \in \mathbf{R}\).)
\end{remark}

\begin{corollary}\label{9.3.13}
    Let \(X\) be a subset of \(\mathbf{R}\), let \(E\) be a subset of \(X\), let \(x_0\) be an adherent point of \(E\), and let \(f : X \to \mathbf{R}\) be a function.
    Then \(f\) can have at most one limit at \(x_0\) in \(E\).
\end{corollary}

\begin{proof}
    Suppose for sake of contradiction that there are two distinct numbers \(L\) and \(L'\) such that \(f\) has a limit \(L\) at \(x_0\) in \(E\), and such that \(f\) also has a limit \(L'\) at \(x_0\) in \(E\).
    Since \(x_0\) is an adherent point of \(E\), we know by Lemma \ref{9.1.14} that there is a sequence \((a_n)_{n = 0}^\infty\) consisting of elements in \(E\) which converges to \(x_0\).
    Since \(f\) has a limit \(L\) at \(x_0\) in \(E\), we thus see by Proposition \ref{9.3.9}, that \((f(a_n))_{n = 0}^\infty\) converges to \(L\).
    But since \(f\) also has a limit \(L'\) at \(x_0\) in \(E\), we see that \((f(a_n))_{n = 0}^\infty\) also converges to \(L'\).
    But this contradicts the uniqueness of limits of sequences (Proposition \ref{6.1.7}).
\end{proof}

\begin{proposition}[Limit laws for functions]\label{9.3.14}
    Let \(X\) be a subset of \(R\), let \(E\) be a subset of \(X\), let \(x_0\) be an adherent point of \(E\), and let \(f : X \to \mathbf{R}\) and \(g : X \to \mathbf{R}\) be functions.
    Suppose that \(f\) has a limit \(L\) at \(x_0\) in \(E\), and \(g\) has a limit \(M\) at \(x_0\) in \(E\).
    Then \(f + g\) has a limit \(L + M\) at \(x_0\) in \(E\), \(f - g\) has a limit \(L - M\) at \(x_0\) in \(E\), \(\max(f, g\)) has a limit \(\max(L, M)\) at \(x_0\) in \(E\), \(\min(f, g)\) has a limit \(\min(L, M)\) at \(x_0\) in \(E\) and \(fg\) has a limit \(LM\) at \(x_0\) in \(E\).
    If \(c\) is a real number, then \(cf\) has a limit \(cL\) at \(x_0\) in \(E\).
    Finally, if \(g\) is non-zero on \(E\) (i.e., \(g(x) \neq 0\) for all \(x \in E\)) and \(M\) is non-zero, then \(f / g\) has a limit \(L / M\) at \(x_0\) in \(E\).
\end{proposition}

\begin{proof}
    Since \(x_0\) is an adherent point of \(E\), we know by Lemma \ref{9.1.14} that there is a sequence \((a_n)_{n = 0}^\infty\) consisting of elements in \(E\), which converges to \(x_0\).
    Since \(f\) has a limit \(L\) at \(x_0\) in \(E\), we thus see by Proposition \ref{9.3.9}, that \((f(a_n))_{n = 0}^\infty\) converges to \(L\).
    Similarly \((g(a_n))_{n = 0}^\infty\) converges to \(M\).

    By the limit laws for sequences (Theorem \ref{6.1.19}) we conclude that \(((f + g)(a_n))_{n = 0}^\infty\) converges to \(L + M\).
    By Proposition \ref{9.3.9} again, this implies that \(f + g\) has a limit \(L + M\) at \(x_0\) in \(E\)
    (since \((a_n)_{n = 0}^\infty\) was an arbitrary sequence in \(E\) converging to \(x_0\)).

    Similarly, By the limit laws for sequences (Theorem \ref{6.1.19}) we conclude that \(((f - g)(a_n))_{n = 0}^\infty\) converges to \(L - M\).
    By Proposition \ref{9.3.9} again, this implies that \(f - g\) has a limit \(L - M\) at \(x_0\) in \(E\).

    Similarly, By the limit laws for sequences (Theorem \ref{6.1.19}) we conclude that \((\max(f, g)(a_n))_{n = 0}^\infty\) converges to \(\max(L, M)\).
    By Proposition \ref{9.3.9} again, this implies that \(\max(f, g)\) has a limit \(\max(L, M)\) at \(x_0\) in \(E\).

    Similarly, By the limit laws for sequences (Theorem \ref{6.1.19}) we conclude that \((\min(f, g)(a_n))_{n = 0}^\infty\) converges to \(\min(L, M)\).
    By Proposition \ref{9.3.9} again, this implies that \(\min(f, g)\) has a limit \(\min(L, M)\) at \(x_0\) in \(E\).

    Similarly, By the limit laws for sequences (Theorem \ref{6.1.19}) we conclude that \(((fg)(a_n))_{n = 0}^\infty\) converges to \(LM\).
    By Proposition \ref{9.3.9} again, this implies that \(fg\) has a limit \(LM\) at \(x_0\) in \(E\).

    Similarly, By the limit laws for sequences (Theorem \ref{6.1.19}) we conclude that \(((cf)(a_n))_{n = 0}^\infty\) converges to \(cL\).
    By Proposition \ref{9.3.9} again, this implies that \(cf\) has a limit \(cL\) at \(x_0\) in \(E\).

    Now suppose that \(\forall\ x \in E, g(x) \neq 0\) and \(M \neq 0\).
    Using similar arguments above we know by the limit laws for sequences (Theorem \ref{6.1.19}) that \(((f / g)(a_n))_{n = 0}^\infty\) converges to \(L / M\).
    By Proposition \ref{9.3.9} again, this implies that \(f / g\) has a limit \(L / M\) at \(x_0\) in \(E\).
\end{proof}

\begin{remark}\label{9.3.15}
    One can phrase Proposition \ref{9.3.14} more informally as saying that
    \begin{align*}
        \lim_{x \to x_0} (f \pm g)(x)  & = \lim_{x \to x_0} f(x) \pm \lim_{x \to x_0} g(x)              \\
        \lim_{x \to x_0} \max(f, g)(x) & = \max\bigg(\lim_{x \to x_0} f(x), \lim_{x \to x_0} g(x)\bigg) \\
        \lim_{x \to x_0} \min(f, g)(x) & = \min\bigg(\lim_{x \to x_0} f(x), \lim_{x \to x_0} g(x)\bigg) \\
        \lim_{x \to x_0} (fg)(x)       & = \lim_{x \to x_0} f(x) \lim_{x \to x_0} g(x)                  \\
        \lim_{x \to x_0} (f / g)(x)    & = \frac{\lim_{x \to x_0} f(x)}{\lim_{x \to x_0} g(x)}
    \end{align*}
    (where we have dropped the restriction \(x \in E\) for brevity)
    but bear in mind that these identities are only true when the right-hand side makes sense, and furthermore for the final identity we need \(g\) to be non-zero, and also \(\lim_{x \to x_0} g(x)\) to be non-zero.
\end{remark}

\begin{note}
    If \(f\) converges to \(L\) at \(x_0\) in \(X\), and \(Y\) is any subset of \(X\) such that \(x_0\) is still an adherent point of \(Y\), then \(f\) will also converge to \(L\) at \(x_0\) in \(Y\)
    (Since \(Y \subseteq X \subseteq \mathbf{R}\)).
    Thus convergence on a large set implies convergence on a smaller set.
    The converse, however, is not true.
\end{note}

\begin{example}\label{9.3.16}
    Consider the \emph{signum function} \(\text{sgn} : \mathbf{R} \to \mathbf{R}\), defined by
    \[
        \text{sgn}(x) \coloneqq \begin{cases}
            1  & \text{if } x > 0 \\
            0  & \text{if } x = 0 \\
            -1 & \text{if } x < 0
        \end{cases}
    \]
    Then \(\lim_{x \to 0 ; x \in (0, \infty)} \text{sgn}(x) = 1\), whereas \(\lim_{x \to 0 ; x \in (-\infty, 0)} = -1\) and \(\lim_{x \to 0 ; x \in \mathbf{R}} \text{sgn}(x)\) is undefined.
    Thus it is sometimes dangerous to drop the set \(X\) from the notation of limit.
    However, in many cases it is safe to do so.
\end{example}

\begin{example}\label{9.3.17}
    Let \(f(x)\) be the function
    \[
        f(x) \coloneqq \begin{cases}
            1 & \text{if } x = 0     \\
            0 & \text{if } x \neq 0.
        \end{cases}
    \]
    Then \(\lim_{x \to 0 ; x \in \mathbf{R} \setminus \{0\}} f(x) = 0\), but \(\lim_{x \to 0 ; x \in \mathbf{R}} f(x)\) is undefined.
    (When this happens, we say that \(f\) has a ``removable singularity'' or ``removable discontinuity'' at \(0\).
    Because of such singularities, it is sometimes the convention when writing \(\lim_{x \to x_0} f(x)\) to automatically exclude \(x_0\) from the set;
    for instance, in some textbook, \(\lim_{x \to x_0} f(x)\) is used as shorthand for \(\lim_{x \to x_0 ; x \in X \setminus \{x_0\}} f(x)\).)
\end{example}

\begin{note}
    On the other hand, the limit at \(x_0\) should only depend on the values of the function near \(x_0\);
    the values away from \(x_0\) are not relevant.
\end{note}

\begin{proposition}[Limits are local]\label{9.3.18}
    Let \(X\) be a subset of \(\mathbf{R}\), let \(E\) be a subset of \(X\), let \(x_0\) be an adherent point of \(E\), let \(f : X \to \mathbf{R}\) be a function, and let \(L\) be a real number.
    Let \(\delta > 0\).
    Then we have
    \[
        \lim_{x \to x_0 ; x \in E} f(x) = L
    \]
    if and only if
    \[
        \lim_{x \to x_0 ; x \in E \cap (x_0 - \delta, x_0 + \delta)} f(x) = L.
    \]
\end{proposition}

\begin{proof}
    We first show that \(\lim_{x \to x_0 ; x \in E} f(x) = L \implies \lim_{x \to x_0 ; x \in E \cap (x_0 - \delta, x_0 + \delta)} f(x) = L\).
    This is true since \(E \cap (x_0 - \delta, x_0 + \delta) \subseteq E\).

    Now we show that \(\lim_{x \to x_0 ; x \in E \cap (x_0 - \delta, x_0 + \delta)} f(x) = L \implies \lim_{x \to x_0 ; x \in E} f(x) = L\).
    This is true since by Definition \ref{9.3.6} we have
    \begin{align*}
                 & \lim_{x \to x_0 ; x \in E \cap (x_0 - \delta, x_0 + \delta)} f(x) = L                                                                          \\
        \implies & \forall\ \varepsilon \in \mathbf{R}^+, \exists\ \delta' \in \mathbf{R}^+ :                                                                     \\
                 & \bigg(\forall\ x \in E \cap (x_0 - \delta, x_0 + \delta), \abs*{x - x_0} < \delta' \implies \abs*{f(x) - L} \leq \varepsilon\bigg)             \\
        \implies & \forall\ \varepsilon \in \mathbf{R}^+, \exists\ \delta' \in \mathbf{R}^+ :                                                                     \\
                 & \bigg(\forall\ x \in E \land x \in (x_0 - \delta, x_0 + \delta) \land \abs*{x - x_0} < \delta' \implies \abs*{f(x) - L} \leq \varepsilon\bigg) \\
        \implies & \forall\ \varepsilon \in \mathbf{R}^+, \exists\ \delta' \in \mathbf{R}^+ :                                                                     \\
                 & \bigg(\forall\ x \in E \land (x_0 - \delta < x < x_0 + \delta) \land \abs*{x - x_0} < \delta' \implies \abs*{f(x) - L} \leq \varepsilon\bigg)  \\
        \implies & \forall\ \varepsilon \in \mathbf{R}^+, \exists\ \delta' \in \mathbf{R}^+ :                                                                     \\
                 & \bigg(\forall\ x \in E \land (-\delta < x - x_0 < \delta) \land \abs*{x - x_0} < \delta' \implies \abs*{f(x) - L} \leq \varepsilon\bigg)       \\
        \implies & \forall\ \varepsilon \in \mathbf{R}^+, \exists\ \delta' \in \mathbf{R}^+ :                                                                     \\
                 & \bigg(\forall\ x \in E \land \abs*{x - x_0} < \delta \land \abs*{x - x_0} < \delta' \implies \abs*{f(x) - L} \leq \varepsilon\bigg)            \\
        \implies & \forall\ \varepsilon \in \mathbf{R}^+, \exists\ \delta' \in \mathbf{R}^+ :                                                                     \\
                 & \bigg(\forall\ x \in E \land \abs*{x - x_0} < \min(\delta, \delta') \implies \abs*{f(x) - L} \leq \varepsilon\bigg)                            \\
        \implies & \lim_{x \to x_0 ; x \in E} f(x) = L.
    \end{align*}
\end{proof}

\begin{note}
    Informally, Proposition \ref{9.3.18} asserts that
    \[
        \lim_{x \to x_0 ; x \in E} f(x) = \lim_{x \to x_0 ; x \in E \cap (x_0 - \delta, x_0 + \delta)} f(x).
    \]
    Thus the limit of a function at \(x_0\), if it exists, only depends on the values of \(f\) near \(x_0\);
    the values far away do not actually influence the limit.
\end{note}

\setcounter{theorem}{21}
\begin{definition}[Limit superior and limi inferior]\label{9.3.22}
    Let \(X\) be a subset of \(\mathbf{R}\), let \(f : X \to \mathbf{R}\) be a function, let \(E\) be a subset of \(X\), and let \(x_0\) be an adherent point of \(E\).
    We define \emph{limit superior at \(x_0\) in \(E\)} as
    \[
        \limsup_{x \to x_0 ; x \in E} f(x) = \inf\Bigg\{\sup\bigg\{f(x) : x \in E \land \abs*{x - x_0} < \delta\bigg\} : \delta \in \mathbf{R}^+\Bigg\}
    \]
    and define \emph{limit inferior at \(x_0\) in \(E\)} as
    \[
        \liminf_{x \to x_0 ; x \in E} f(x) = \sup\Bigg\{\inf\bigg\{f(x) : x \in E \land \abs*{x - x_0} < \delta\bigg\} : \delta \in \mathbf{R}^+\Bigg\}.
    \]
\end{definition}

\exercisesection

\begin{exercise}\label{ex 9.3.1}
    Prove Proposition \ref{9.3.9}.
\end{exercise}

\begin{proof}
    See Proposition \ref{9.3.9}.
\end{proof}

\begin{exercise}\label{ex 9.3.2}
    Prove the remaining claims in Proposition \ref{9.3.14}.
\end{exercise}

\begin{proof}
    See Proposition \ref{9.3.14}.
\end{proof}

\begin{exercise}\label{ex 9.3.3}
    Prove Proposition \ref{9.3.18}.
\end{exercise}

\begin{proof}
    See Proposition \ref{9.3.18}.
\end{proof}

\begin{exercise}\label{ex 9.3.4}
    Propose a definition for limit superior \(\limsup_{x \to x_0 ; x \in E} f(x)\) and limit inferior \(\liminf_{x \to x_0 ; x \in E} f(x)\), and then propose an analogue of Proposition \ref{9.3.9} for your definition.
    (For an additional challenge: prove that analogue.)
\end{exercise}

\begin{proof}
    Need helped.
\end{proof}

\begin{exercise}[Continuous version of squeeze test]\label{ex 9.3.5}
    Let \(X\) be a subset of \(\mathbf{R}\), let \(E\) be a subset of \(X\), let \(x_0\) be an adherent point of \(E\), and let \(f : X \to \mathbf{R}\), \(g : X \to \mathbf{R}\), \(h : X \to \mathbf{R}\) be functions such that \(f(x) \leq g(x) \leq h(x)\) for all \(x \in E\).
    If we have \(\lim_{x \to x_0 ; x \in E} f(x) = \lim_{x \to x_0 ; x \in E} h(x) = L\) for some real number \(L\), show that \(\lim_{x \to x_0 ; x \in E} g(x) = L\).
\end{exercise}

\begin{proof}
    Since \(\lim_{x \to x_0 ; x \in E} f(x) = L\), by Definition \ref{9.3.6} we have
    \[
        \forall\ \varepsilon \in \mathbf{R}^+, \exists\ \delta_1 \in \mathbf{R}^+ : \bigg(\forall\ x \in E, \abs*{x - x_0} < \delta_1 \implies \abs*{f(x) - L} \leq \varepsilon\bigg).
    \]
    Similarly we have
    \[
        \forall\ \varepsilon \in \mathbf{R}^+, \exists\ \delta_2 \in \mathbf{R}^+ : \bigg(\forall\ x \in E, \abs*{x - x_0} < \delta_2 \implies \abs*{h(x) - L} \leq \varepsilon\bigg).
    \]
    Let \(\delta = \min(\delta_1, \delta_2)\).
    Then we have
    \[
        \forall\ \varepsilon \in \mathbf{R}^+, \exists\ \delta \in \mathbf{R}^+ : \bigg(\forall\ x \in E, \abs*{x - x_0} < \delta \implies \abs*{f(x) - L} \leq \varepsilon \land \abs*{h(x) - L} \leq \varepsilon\bigg).
    \]
    Since \(f(x) \leq g(x) \leq h(x)\), we have
    \begin{align*}
                 & \big(x \in E\big) \land \big(\abs*{x - x_0} < \delta\big)                                                                      \\
        \implies & \big(f(x) \leq g(x) \leq h(x)\big) \land \big(\abs*{f(x) - L} < \varepsilon\big) \land \big(\abs*{h(x) - L} < \varepsilon\big) \\
        \implies & -\varepsilon \leq f(x) - L \leq g(x) - L \leq h(x) - L \leq \varepsilon                                                        \\
        \implies & \abs*{g(x) - L} \leq \varepsilon.
    \end{align*}
    But this means
    \[
        \forall\ \varepsilon \in \mathbf{R}^+, \exists\ \delta \in \mathbf{R}^+ : \bigg(\forall\ x \in E, \abs*{g(x) - L} \leq \varepsilon\bigg)
    \]
    and thus by Definition \ref{9.3.6} \(\lim_{x \to x_0 ; x \in E} g(x) = L\).
\end{proof}
\section{Continuous functions}\label{sec 9.4}

\begin{definition}[Continuity]\label{9.4.1}
    Let \(X\) be a subset of \(\mathbf{R}\), and let \(f : X \to \mathbf{R}\) be a function.
    Let \(x_0\) be an element of \(X\).
    We say that \(f\) is \emph{continuous at \(x_0\)} iff we have
    \[
        \lim_{x \to x_0 ; x \in X} f(x) = f(x_0);
    \]
    in other words, the limit of \(f(x)\) as \(x\) converges to \(x_0\) in \(X\) exists and is equal to \(f(x_0)\).
    We say that \(f\) is \emph{continuous on \(X\)} (or simply \emph{continuous}) iff \(f\) is continuous at \(x_0\) for every \(x_0 \in X\).
    We say that \(f\) is \emph{discontinuous at \(x_0\)} iff it is not continuous at \(x_0\).
\end{definition}

\begin{note}
    Restricting the domain of a function can make a discontinuous function continuous again.
\end{note}

\setcounter{theorem}{6}
\begin{proposition}[Equivalent formulations of continuity]\label{9.4.7}
    Let \(X\) be a subset of \(\mathbf{R}\), let \(f : X \to \mathbf{R}\) be a function, and let \(x_0\) be an element of \(X\).
    Then the following four statements are logically equivalent:
    \begin{enumerate}
        \item \(f\) is continuous at \(x_0\).
        \item For every sequence \((a_n)_{n = 0}^\infty\) consisting of elements of \(X\) with \(\lim_{n \to \infty} a_n = x_0\), we have \(\lim_{n \to \infty} f(a_n) = f(x_0)\).
        \item For every \(\varepsilon > 0\), there exists a \(\delta > 0\) such that \(\abs*{f(x) - f(x_0)} < \varepsilon\) for all \(x \in X\) with \(\abs*{x - x_0} < \delta\).
        \item For every \(\varepsilon > 0\), there exists a \(\delta > 0\) such that \(\abs*{f(x) - f(x_0)} \leq \varepsilon\) for all \(x \in X\) with \(\abs*{x - x_0} \leq \delta\).
    \end{enumerate}
\end{proposition}

\begin{proof}
    We first show that the statement (a) and the statement (b) are equivalent.
    By Definition \ref{9.4.1}, \(f\) is continuous at \(x_0\) iff \(f\) converges to \(f(x_0)\) at \(x_0\) in \(X\).
    Thus by Proposition \ref{9.3.9} we know that the statement (a) and the statement (b) are equivalent.

    Next we show that the statement (a) and the statement (c) are equivalent.
    By Definition \ref{9.4.1}, \(f\) is continuous at \(x_0\) iff \(f\) converges to \(f(x_0)\) at \(x_0\) in \(X\).
    By Definition \ref{9.3.6} this is equivalent to the statement
    \[
        \forall\ \varepsilon' \in \mathbf{R}^+, \exists\ \delta \in \mathbf{R}^+ : \bigg(\forall\ x \in X, \abs*{x - x_0} < \delta \implies \abs*{f(x) - f(x_0)} \leq \varepsilon'\bigg).
    \]
    Let \(\varepsilon \in \mathbf{R}^+ \land \varepsilon > \varepsilon'\).
    Then we have
    \[
        \forall\ \varepsilon \in \mathbf{R}^+, \exists\ \delta \in \mathbf{R}^+ : \bigg(\forall\ x \in X, \abs*{x - x_0} < \delta \implies \abs*{f(x) - f(x_0)} \leq \varepsilon' < \varepsilon\bigg).
    \]
    Thus the statement (a) and the statement (c) are equivalent.

    Finally we show that the statement (a) and the statement (d) are equivalent.
    By Definition \ref{9.4.1}, \(f\) is continuous at \(x_0\) iff \(f\) converges to \(f(x_0)\) at \(x_0\) in \(X\).
    By Definition \ref{9.3.6} this is equivalent to the statement
    \[
        \forall\ \varepsilon \in \mathbf{R}^+, \exists\ \delta' \in \mathbf{R}^+ : \bigg(\forall\ x \in X, \abs*{x - x_0} < \delta' \implies \abs*{f(x) - f(x_0)} \leq \varepsilon\bigg).
    \]
    Let \(\delta \in \mathbf{R}^+ \land \abs*{x - x_0} \leq \delta < \delta'\).
    By Proposition \ref{5.4.14} we know such \(\delta\) exists.
    Then we have
    \[
        \forall\ \varepsilon \in \mathbf{R}^+, \exists\ \delta \in \mathbf{R}^+ : \bigg(\forall\ x \in X, \abs*{x - x_0} \leq \delta \implies \abs*{f(x) - f(x_0)} \leq \varepsilon\bigg).
    \]
    Thus the statement (a) and the statement (d) are equivalent.
\end{proof}

\begin{remark}\label{9.4.8}
    A particularly useful consequence of Proposition \ref{9.4.7} is the following:
    if \(f\) is continuous at \(x_0\), and \(a_n \to x_0\) as \(n \to \infty\), then \(f(a_n) \to f(x_0)\) as \(n \to \infty\)
    (provided that all the elements of the sequence \((a_n)_{n = 0}^\infty\) lie in the domain of \(f\), of course).
    Thus continuous functions are very useful in computing limits.
\end{remark}

\begin{proposition}[Arithmetic preserves continuity]\label{9.4.9}
    Let \(X\) be a subset of \(\mathbf{R}\), and let \(f : X \to \mathbf{R}\) and \(g : X \to \mathbf{R}\) be functions.
    Let \(x_0 \in X\).
    Then if \(f\) and \(g\) are both continuous at \(x_0\), then the functions \(f + g\), \(f - g\), \(\max(f, g)\), \(\min(f, g)\) and \(fg\) are also continuous at \(x_0\).
    If \(g\) is non-zero on \(X\), then \(f / g\) is also continuous at \(x_0\).
\end{proposition}

\begin{proof}
    By Proposition \ref{9.3.14}, we have
    \begin{align*}
        \lim_{x \to x_0 ; x \in X} f(x) + g(x)      & = \lim_{x \to x_0 ; x \in X} f(x) + \lim_{x \to x_0 ; x \in X} g(x)                                                            \\
                                                    & = f(x_0) + g(x_0);                                                                       & \text{(by Definition \ref{9.4.1})}  \\
        \lim_{x \to x_0 ; x \in X} f(x) - g(x)      & = \lim_{x \to x_0 ; x \in X} f(x) - \lim_{x \to x_0 ; x \in X} g(x)                                                            \\
                                                    & = f(x_0) - g(x_0);                                                                       & \text{(by Definition \ref{9.4.1})}  \\
        \lim_{x \to x_0 ; x \in X} \max(f(x), g(x)) & = \max(\lim_{x \to x_0 ; x \in X} f(x), \lim_{x \to x_0 ; x \in X} g(x))                                                       \\
                                                    & = \max(f(x_0), g(x_0));                                                                  & \text{(by Definition \ref{9.4.1})}  \\
        \lim_{x \to x_0 ; x \in X} \min(f(x), g(x)) & = \min(\lim_{x \to x_0 ; x \in X} f(x), \lim_{x \to x_0 ; x \in X} g(x))                                                       \\
                                                    & = \min(f(x_0), g(x_0));                                                                  & \text{(by Definition \ref{9.4.1})}  \\
        \lim_{x \to x_0 ; x \in X} f(x) g(x)        & = \bigg(\lim_{x \to x_0 ; x \in X} f(x)\bigg)\bigg(\lim_{x \to x_0 ; x \in X} g(x)\bigg)                                       \\
                                                    & = f(x_0) g(x_0);                                                                         & \text{(by Definition \ref{9.4.1})}  \\
        \lim_{x \to x_0 ; x \in X} f(x) / g(x)      & = \lim_{x \to x_0 ; x \in X} f(x) / \lim_{x \to x_0 ; x \in X} g(x)                      & \text{(\(g\) is non-zero on \(X\))} \\
                                                    & = f(x_0) / g(x_0).                                                                       & \text{(by Definition \ref{9.4.1})}  \\
    \end{align*}
\end{proof}
\section{Left and right limits}\label{sec 9.5}

\begin{definition}[Left and right limits]\label{9.5.1}
    Let \(X\) be a subset of \(\mathbf{R}\), \(f : X \to \mathbf{R}\) be a function, and let \(x_0\) be a real number.
    If \(x_0\) is an adherent point of \(X \cap (x_0, \infty)\), then we define the \emph{right limit} \(f(x_0+)\) of \(f\) at \(x_0\) by the formula
    \[
        f(x_0+) \coloneqq \lim_{x \to x_0 ; x \in X \cap (x_0, \infty)} f(x),
    \]
    provided of course that this limit exists.
    Similarly, if \(x_0\) is an adherent point of \(X \cap (-\infty, x_0)\), then we define the \emph{left limit} \(f(x_0-)\) of \(f\) at \(x_0\) by the formula
    \[
        f(x_0-) \coloneqq \lim_{x \to x_0 ; x \in X \cap (-\infty, x_0)} f(x),
    \]
    again provided that the limit exists.
    (Thus in many cases \(f(x_0+)\) and \(f(x_0-)\) will not be defined.)
    Sometimes we use the shorthand notations
    \begin{align*}
        \lim_{x \to x_0+} f(x) & \coloneqq \lim_{x \to x_0 ; x \in X \cap (x_0, \infty)} f(x); \\
        \lim_{x \to x_0-} f(x) & \coloneqq \lim_{x \to x_0 ; x \in X \cap (-\infty, x_0)} f(x)
    \end{align*}
    when the domain \(X\) of \(f\) is clear from context.
\end{definition}

\begin{note}
    From Proposition \ref{9.4.7} we see that if the right limit \(f(x_0+)\) exists, and \((a_n)_{n = 0}^\infty\) is a sequence in \(X\) converging to \(x_0\) from the right (i.e., \(a_n > x_0\) for all \(n \in \mathbf{N}\)), then \(\lim_{n \to \infty} f(a_n) = f(x_0+)\).
    Similarly, if \((b_n)_{n = 0}^\infty\) is a sequence converging to \(x_0\) from the left (i.e., \(a_n < x_0\) for all \(n \in \mathbf{N}\)) then \(\lim_{n \to \infty} f(a_n) = f(x_0-)\).
\end{note}

\begin{additional corollary}\label{ac 9.5.1}
    Let \(x_0\) be an adherent point of both \(X \cap (x_0, \infty)\) and \(X \cap (-\infty, x_0)\).
    If \(f\) is continuous at \(x_0\), then \(f(x_0+)\) and \(f(x_0-)\) both exists and are equal to \(f(x_0)\).
\end{additional corollary}

\begin{proof}
    Since \(f\) is continuous at \(x_0\), by Definition \ref{9.4.1} we know that
    \[
        \forall\ \varepsilon \in \mathbf{R}^+, \exists\ \delta \in \mathbf{R}^+ : \bigg(\forall\ x \in X, \abs*{x - x_0} < \delta \implies \abs*{f(x) - f(x_0)} < \varepsilon\bigg).
    \]
    Since \(X \cap (-\infty, x_0) \subseteq X\) and \(X \cap (x_0, \infty) \subseteq X\), we know that the following two statements are true:
    \begin{align*}
        &\forall\ \varepsilon \in \mathbf{R}^+, \exists\ \delta \in \mathbf{R}^+ : \\
        &\bigg(\forall\ x \in X \cap (-\infty, x_0), \abs*{x - x_0} < \delta \implies \abs*{f(x) - f(x_0)} < \varepsilon\bigg) \\
        \implies & \lim_{x \to x_0 ; x \in X \cap (-\infty, x_0)} f(x) = f(x_0). \\
        &\forall\ \varepsilon \in \mathbf{R}^+, \exists\ \delta \in \mathbf{R}^+ : \\
        &\bigg(\forall\ x \in X \cap (x_0, \infty), \abs*{x - x_0} < \delta \implies \abs*{f(x) - f(x_0)} < \varepsilon\bigg) \\
        \implies & \lim_{x \to x_0 ; x \in X \cap (x_0, \infty)} f(x) = f(x_0).
    \end{align*}
    Thus by Definition \ref{9.5.1} we have \(f(x_0+) = f(x_0-) = f(x_0)\).
\end{proof}

\setcounter{theorem}{2}
\begin{proposition}\label{9.5.3}
    Let \(X\) be a subset of \(\mathbf{R}\) containing a real number \(x_0\), and suppose that \(x_0\) is an adherent point of both \(X \cap (x_0, \infty)\) and \(X \cap (-\infty, x_0)\).
    Let \(f : X \to \mathbf{R}\) be a function.
    If \(f(x_0+)\) and \(f(x_0-)\) both exist and are both equal to \(f(x_0)\), then \(f\) is continuous at \(x_0\).
\end{proposition}

\begin{proof}
    Let us write \(L \coloneqq f(x_0)\).
    Then by hypothesis we have
    \[
        \lim_{x \to x_0 ; x \in X \cap (x_0, \infty)} f(x) = L
    \]
    and
    \[
        \lim_{x \to x_0 ; x \in X \cap (-\infty, x_0)} f(x) = L.
    \]
    Let \(\varepsilon > 0\) be given.
    From the first statement above and Proposition \ref{9.4.7} (applied to the restriction of \(f\) to \(X \cap (x_0, +\infty)\)), we know that there exists a \(\delta_+ > 0\) such that \(\abs*{f(x) - L} < \varepsilon\) for all \(x \in X \cap(x_0, \infty)\) for which \(\abs*{x - x_0} < \delta_+\).
    From the second statement above we similarly know that there exists a \(\delta_- > 0\) such that \(\abs*{f(x) - L} < \varepsilon\) for all \(x \in X \cap (-\infty, x_0)\) for which \(\abs*{x - x_0} < \delta_-\).
    Now let \(\delta \coloneqq \min(\delta_-, \delta_+)\);
    then \(\delta > 0\), and suppose that \(x \in X\) is such that \(\abs*{x - x_0} < \delta\).
    Then there are three cases:
    \(x > x_0\), \(x = x_0\), and \(x < x_0\), but in all three cases we know that \(\abs*{f(x) - L} < \varepsilon\).
    (If \(x > x_0\), then \(x \in X \cap (x_0, \infty)\), thus \(\abs*{x - x_0} < \delta \leq \delta_+ \implies \abs*{f(x) - L} < \varepsilon\).
    Similarly, if \(x < x_0\), then \(x \in X \cap (-\infty, x_0)\), thus \(\abs*{x - x_0} < \delta \leq \delta_- \implies \abs*{f(x) - L} < \varepsilon\).
    If \(x = x_0\), then we have \(\abs*{x_0 - x_0} = 0 < \delta\) and \(\abs*{f(x_0) - f(x_0)} = 0 < \varepsilon\).)
    By Proposition \ref{9.4.7} we thus have that \(f\) is continuous at \(x_0\), as desired.
\end{proof}

\begin{note}
    When both \(f(x_0+), f(x_0-)\) exist and \(f(x_0+) \neq f(x_0-)\), we say that \(f\) has a \emph{jump discontinuity} at \(x_0\).
    When both \(f(x_0+), f(x_0-)\) exist and \(f(x_0+) = f(x_0-) \neq f(x_0)\), we say that \(f\) has a \emph{removable discontinuity} (or \emph{removable singularity}) at \(x_0\).
\end{note}

\begin{remark}\label{9.5.4}
    Jump discontinuities and removable discontinuities are not the only way a function can be discontinuous.
    Another way is for a function to go to infinity at the discontinuity:
    for instance, the function \(f : \mathbf{R} \setminus \{0\} \to \mathbf{R}\) defined by \(f(x) \coloneqq 1 / x\) has a discontinuity at \(0\) which is neither a jump discontinuity or a removable singularity;
    informally, \(f(x)\) converges to \(+\infty\) when \(x\) approaches \(0\) from the right, and converges to \(-\infty\) when \(x\) approaches \(0\) from the left.
    These types of singularities are sometimes known as \emph{asymptotic discontinuities}.
    There are also \emph{oscillatory discontinuities}, where the function remains bounded but still does not have a limit near \(x_0\).
    For instance, the function \(f : \mathbf{R} \to \mathbf{R}\) defined by
    \[
        f(x) \coloneqq \begin{cases}
            1 & \text{if } x \in \mathbf{Q} \\
            0 & \text{if } x \notin \mathbf{Q}
        \end{cases}
    \]
    has an oscillatory discontinuity at \(0\) (and in fact at any other real number also).
    This is because the function does not have left or right limits at \(0\), despite the fact that the function is bounded.
\end{remark}

\begin{note}
    The study of discontinuities is also called \emph{singularities}.
\end{note}

\exercisesection

\begin{exercise}\label{ex 9.5.1}
    Let \(E\) be a subset of \(\mathbf{R}\), let \(f : E \to \mathbf{R}\) be a function, and let \(x_0\) be an adherent point of \(E\).
    Write down a definition of what it would mean for the limit \(\lim_{x \to x_0 ; x \in E} f(x)\) to exist and equal \(+\infty\) or \(-\infty\).
    If \(f : \mathbf{R} \setminus \{0\} \to \mathbf{R}\) is the function \(f(x) \coloneqq 1 / x\), use your definition to conclude \(f(0+) = +\infty\) and \(f(0-) = -\infty\).
    Also, state and prove some analogue of Proposition \ref{9.3.9} when \(L = +\infty\) or \(L = -\infty\).
\end{exercise}

\begin{definition}
    We define \(\lim_{x \to x_0 ; x \in E} f(x) = \infty\) iff
    \[
        \forall\ \varepsilon \in \mathbf{R}^+, \exists\ \delta \in \mathbf{R}^+ : \bigg(\forall\ x \in E, \abs*{x - x_0} < \delta \implies f(x) > \varepsilon\bigg).
    \]
    And define \(\lim_{x \to x_0 ; x \in E} f(x) = -\infty\) iff
    \[
        \forall\ \varepsilon \in \mathbf{R}^+, \exists\ \delta \in \mathbf{R}^+ : \bigg(\forall\ x \in E, \abs*{x - x_0} < \delta \implies f(x) < -\varepsilon\bigg).
    \]

    Now we show that \(\lim_{x \to 0 ; x \in \mathbf{R} \cap (0, \infty)} 1 / x = \infty\) and \(\lim_{x \to 0 ; x \in \mathbf{R} \cap (-\infty, 0)} 1 / x = -\infty\).
    Let \(\varepsilon \in \mathbf{R}^+\).
    \(\forall\ x \in \mathbf{R} \cap (0, \infty)\), we have
    \[
        x = \abs*{x} = \abs*{x - 0} < 1 / \varepsilon \implies 1 / x < \varepsilon.
    \]
    By letting \(\delta = 1 / \varepsilon\) we have
    \[
        \forall\ \varepsilon \in \mathbf{R}^+, \exists\ \delta \in \mathbf{R}^+ : \bigg(\forall\ x \in \mathbf{R} \cap (0, \infty), \abs*{x - 0} < \delta \implies 1 / x > \varepsilon\bigg).
    \]
    Thus by our definition \(\lim_{x \to 0 ; x \in \mathbf{R} \cap (0, \infty)} 1 / x = \infty\).
    Similarly, \(\forall\ x \in \mathbf{R} \cap (-\infty, 0)\), we have
    \[
        -x = \abs*{x} = \abs*{x - 0} < 1 / \varepsilon \implies 1 / x < -\varepsilon.
    \]
    By letting \(\delta = 1 / \varepsilon\) we have
    \[
        \forall\ \varepsilon \in \mathbf{R}^+, \exists\ \delta \in \mathbf{R}^+ : \bigg(\forall\ x \in \mathbf{R} \cap (-\infty, 0), \abs*{x - 0} < \delta \implies 1 / x < -\varepsilon\bigg).
    \]
    Thus by our definition \(\lim_{x \to 0 ; x \in \mathbf{R} \cap (-\infty, 0)} 1 / x = -\infty\).

    Now we define an analogue of Proposition \ref{9.3.9} on \(\infty\).
    The following two statements are equivalent:
    \begin{enumerate}
        \item \(\lim_{x \to x_0 ; x \in E} f(x) = \infty\).
        \item For every sequence \((a_n)_{n = 0}^\infty\) which consists entirely of elements of \(E\) and converges to \(x_0\), the sequence \((f(a_n))_{n = 0}^\infty\) converges to \(\infty\).
    \end{enumerate}
    We first show that the first statement above implies the second statement above.
    Since \(\lim_{x \to x_0 ; x \in E} f(x) = \infty\), by our definition we have
    \[
        \forall\ \varepsilon \in \mathbf{R}^+, \exists\ \delta \in \mathbf{R}^+ : \bigg(\forall\ x \in E, \abs*{x - x_0} < \delta \implies f(x) > \varepsilon\bigg).
    \]
    Since \(\lim_{n \to \infty} a_n = x_0\), \(\exists\ n \in \mathbf{N}\) such that \(\forall\ N \geq n\) we have \(\abs*{a_N - x_0} \leq \delta / 2 < \delta\).
    But since \(a_N \in E\), we have \(\abs*{a_N - x_0} < \delta \implies f(a_N) > \varepsilon\).
    Thus \(\forall\ \varepsilon \in \mathbf{R}^+\), \(\exists\ n \in \mathbf{N}\) such that \(\forall\ N \geq n\) we have \(f(a_N) > \varepsilon\), which means \(\lim_{n \to \infty} f(a_n) = \infty\)
    (Since all element in \((f(a_n)_{n = 0}^\infty)\) are eventually positive and only bounded above by \(\infty\)).
    Next we show that the second statement above implies the first statement above.
    Suppose for sake of contradiction that \(\lim_{x \to x_0 ; x \in E} f(x) \neq \infty\).
    Then \(\exists\ \varepsilon \in \mathbf{R}^+\) such that \(\forall\ \delta \in \mathbf{R}^+\), we have \(\abs*{x - x_0} < \delta \land f(x) \leq \varepsilon\).
    Since \(\lim_{n \to \infty} a_n = x_0\), we know that \(\exists\ n_1 \in \mathbf{N}\) such that \(\forall\ N \geq n_1\) we have \(\abs*{a_N - x_0} < \delta\).
    Since \(\lim_{n \to \infty} f(a_n) = \infty\), we know that \(\exists\ n_2 \in \mathbf{N}\) such that \(\forall\ N \geq n_2\) we have \(f(a_N) > \varepsilon\).
    Let \(n = \max(n_1, n_2)\).
    Then we have \(\forall\ N \geq n\), \(\abs*{a_N - x_0} < \delta \land f(a_N) > \varepsilon\), a contradiction.
    Thus \(\lim_{x \to x_0 ; x \in E} f(x) = \infty\).

    Finally we define an analogue of Proposition \ref{9.3.9} on \(-\infty\).
    The following two statements are equivalent:
    \begin{enumerate}
        \item \(\lim_{x \to x_0 ; x \in E} f(x) = -\infty\).
        \item For every sequence \((a_n)_{n = 0}^\infty\) which consists entirely of elements of \(E\) and converges to \(x_0\), the sequence \((f(a_n))_{n = 0}^\infty\) converges to \(-\infty\).
    \end{enumerate}
    We first show that the first statement above implies the second statement above.
    Since \(\lim_{x \to x_0 ; x \in E} f(x) = -\infty\), by our definition we have
    \[
        \forall\ \varepsilon \in \mathbf{R}^+, \exists\ \delta \in \mathbf{R}^+ : \bigg(\forall\ x \in E, \abs*{x - x_0} < \delta \implies f(x) < -\varepsilon\bigg).
    \]
    Since \(\lim_{n \to -\infty} a_n = x_0\), \(\exists\ n \in \mathbf{N}\) such that \(\forall\ N \geq n\) we have \(\abs*{a_N - x_0} \leq \delta / 2 < \delta\).
    But since \(a_N \in E\), we have \(\abs*{a_N - x_0} < \delta \implies f(a_N) < -\varepsilon\).
    Thus \(\forall\ \varepsilon \in \mathbf{R}^+\), \(\exists\ n \in \mathbf{N}\) such that \(\forall\ N \geq n\) we have \(f(a_N) < -\varepsilon\), which means \(\lim_{n \to -\infty} f(a_n) = -\infty\)
    (Since all element in \((f(a_n)_{n = 0}^\infty)\) are eventually negative and only bounded below by \(-\infty\)).
    Next we show that the second statement above implies the first statement above.
    Suppose for sake of contradiction that \(\lim_{x \to x_0 ; x \in E} f(x) \neq -\infty\).
    Then \(\exists\ \varepsilon \in \mathbf{R}^+\) such that \(\forall\ \delta \in \mathbf{R}^+\), we have \(\abs*{x - x_0} < \delta \land f(x) \geq -\varepsilon\).
    Since \(\lim_{n \to -\infty} a_n = x_0\), we know that \(\exists\ n_1 \in \mathbf{N}\) such that \(\forall\ N \geq n_1\) we have \(\abs*{a_N - x_0} < \delta\).
    Since \(\lim_{n \to -\infty} f(a_n) = -\infty\), we know that \(\exists\ n_2 \in \mathbf{N}\) such that \(\forall\ N \geq n_2\) we have \(f(a_N) < -\varepsilon\).
    Let \(n = \max(n_1, n_2)\).
    Then we have \(\forall\ N \geq n\), \(\abs*{a_N - x_0} < \delta \land f(a_N) < -\varepsilon\), a contradiction.
    Thus \(\lim_{x \to x_0 ; x \in E} f(x) = -\infty\).
\end{definition}
\section{The maximum principle}\label{sec 9.6}

\begin{definition}\label{9.6.1}
    Let \(X\) be a subset of \(\mathbf{R}\), and let \(f : X \to \mathbf{R}\) be a function.
    We say that \(f\) is \emph{bounded from above} iff there exists a real number \(M\) such that \(f(x) \leq M\) for all \(x \in X\).
    We say that \(f\) is \emph{bounded from below} iff there exists a real number \(M\) such that \(f(x) \geq -M\) for all \(x \in X\).
    We say that \(f\) is \emph{bounded} iff there exists a real number \(M\) such that \(\abs*{f(x)} \leq M\) for all \(x \in X\).
\end{definition}

\begin{remark}\label{9.6.2}
    A function is bounded if and only if it is bounded both from above and below.
    Also, a function \(f : X \to \mathbf{R}\) is bounded if and only if its image \(f(X)\) is a bounded set in the sense of Definition \ref{9.1.22}.
\end{remark}

\begin{lemma}\label{9.6.3}
    Let \(a < b\) be real numbers, and let \(f : [a, b] \to \mathbf{R}\) be a function continuous on \([a, b]\).
    Then \(f\) is a bounded function.
\end{lemma}

\begin{proof}
    Suppose for sake of contradiction that \(f\) is not bounded.
    Thus for every real number \(M\) there exists an element \(x \in [a, b]\) such that \(\abs*{f(x)} \geq M\).

    In particular, for every natural number \(n\), the set \(\{x \in [a, b] : \abs*{f(x)} \geq n\}\) is non-empty.
    We can thus choose a sequence \((x_n)_{n = 0}^\infty\) in \([a, b]\) such that \(\abs*{f(x_n)} \geq n\) for all \(n\).
    This sequence lies in \([a, b]\), and so by Theorem \ref{9.1.24} there exists a subsequence \((x_{n_j})_{j = 0}^\infty\) which converges to some limit \(L \in [a, b]\), where \(n_0 < n_1 < n_2 < \dots\) is an increasing sequence of natural numbers.
    In particular, we see that \(n_j \geq j\) for all \(j \in \mathbf{N}\) (use induction).

    Since \(f\) is continuous on \([a, b]\), it is continuous at \(L\), and in particular we see that
    \[
        \lim_{j \to \infty} f(x_{n_j}) = f(L).
    \]
    Thus the sequence \((f(x_{n_j}))_{j = 0}^\infty\) is convergent, and hence it is bounded.
    On the other hand, we know from the construction that \(\abs*{f(x_{n_j})} \geq n_j \geq j\) for all \(j\), and hence the sequence \((f(x_{n_j}))_{j = 0}^\infty\) is not bounded, a contradiction.
\end{proof}

\begin{remark}\label{9.6.4}
    There are two things about the proof of Lemma \ref{9.6.3} that are worth noting.
    Firstly, it shows how useful the Heine-Borel theorem (Theorem \ref{9.1.24}) is.
    Secondly, it is an indirect proof;
    it doesn't say \emph{how} to find the bound for \(f\), but it shows that having \(f\) unbounded leads to a contradiction.
\end{remark}

\begin{definition}[Maxima and minima]\label{9.6.5}
    Let \(X\) be a subset of \(\mathbf{R}\), and let \(f : X \to \mathbf{R}\) be a function, and let \(x_0 \in X\).
    We say that \emph{\(f\) attains its maximum at \(x_0\)} if we have \(f(x_0) \geq f(x)\) for all \(x \in X\)
    (i.e., the value of \(f\) at the point \(x_0\) is larger than or equal to the value of \(f\) at any other point in \(X\)).
    We say that \emph{\(f\) attains its minimum at \(x_0\)} if we have \(f(x_0) \leq f(x)\) for all \(x \in X\).
\end{definition}

\begin{remark}\label{9.6.6}
    If a function attains its maximum somewhere, then it must be bounded from above.
    Similarly if it attains its minimum somewhere, then it must be bounded from below.
    These notions of maxima and minima are \emph{global}.
\end{remark}
\section{The intermediate value theorem}\label{sec 9.7}

\begin{theorem}[Intermediate value theorem]\label{9.7.1}
    Let \(a < b\), and let \(f : [a, b] \to \mathbf{R}\) be a continuous function on \([a, b]\).
    Let \(y\) be a real number between \(f(a)\) and \(f(b)\), i.e., either \(f(a) \leq y \leq f(b)\) or \(f(a) \geq y \geq f(b)\).
    Then there exists \(c \in [a, b]\) such that \(f(c) = y\).
\end{theorem}

\begin{proof}
    We have two cases: \(f(a) \leq y \leq f(b)\) or \(f(a) \geq y \geq f(b)\).
    We will assume the former, that \(f(a) \leq y \leq f(b)\);
    the latter is proven similarly.

    If \(y = f(a)\) or \(y = f(b)\) then the claim is easy, as one can simply set \(c = a\) or \(c = b\), so we will assume that \(f(a) < y < f(b)\).
    Let \(E\) denote the set
    \[
        E \coloneqq \{x \in [a, b] : f(x) < y\}.
    \]
    Clearly \(E\) is a subset of \([a, b]\), and is hence bounded.
    Also, since \(f(a) < y\), we see that \(a\) is an element of \(E\), so \(E\) is non-empty.
    By the least upper bound principle, the supremum
    \[
        c \coloneqq \sup(E)
    \]
    is thus finite.
    Since \(E\) is bounded by \(b\), we know that \(c \leq b\);
    since \(E\) contains \(a\), we know that \(c \geq a\).
    Thus we have \(c \in [a, b]\).
    To complete the proof we now show that \(f(c) = y\).
    The idea is to work from the left of \(c\) to show that \(f(c) \leq y\), and to work from the right of \(c\) to show that \(f(c) \geq y\).

    Let \(n \geq 1\) be an integer.
    The number \(c - \frac{1}{n}\) is less than \(c = \sup(E)\) and hence cannot be an upper bound for \(E\).
    Thus there exists a point, call it \(x_n\), which lies in \(E\) and which is greater than \(c - \frac{1}{n}\).
    Also \(x_n \leq c\) since \(c\) is an upper bound for \(E\).
    Thus
    \[
        c - \frac{1}{n} \leq x_n \leq c.
    \]
    By the squeeze test (Corollary \ref{6.4.14}) we thus have \(\lim_{n \to \infty} x_n = c\).
    Since \(f\) is continuous at \(c\), this implies that \(\lim_{n \to \infty} f(x_n) = f(c)\).
    But since \(x_n\) lies in \(E\) for every \(n\), we have \(f(x_n) < y\) for every \(n\).
    By the comparison principle (Lemma \ref{6.4.13}) we thus have \(f(c) \leq y\).
    Since \(f(b) > f(c)\), we conclude \(c \neq b\).

    Since \(c \neq b\) and \(c \in [a, b]\), we must have \(c < b\).
    In particular there is an \(N > 0\) such that \(c + \frac{1}{n} < b\) for all \(n > N\)
    (since \(c + \frac{1}{n}\) converges to \(c\) as \(n \to \infty\)).
    Since \(c\) is the supremum of \(E\) and \(c + \frac{1}{n} > c\), we thus have \(c + \frac{1}{n} \notin E\) for all \(n > N\).
    Since \(c + \frac{1}{n} \in [a, b]\), we thus have \(f(c + \frac{1}{n}) \geq y\) for all \(n \geq N\).
    But \(c + \frac{1}{n}\) converges to \(c\), and \(f\) is continuous at \(c\), thus \(f(c) \geq y\).
    But we already knew that \(f(c) \leq y\), thus \(f(c) = y\), as desired.
\end{proof}

\begin{note}
    The intermediate value theorem says that if \(f\) takes the values \(f(a)\) and \(f(b)\), then it must also take all the values in between.
    If \(f\) is not assumed to be continuous, then the intermediate value theorem no longer applies.
    If a function is discontinuous, it can ``jump'' past intermediate values;
    however continuous functions cannot do so.
\end{note}

\begin{remark}\label{9.7.2}
    A continuous function may take an intermediate value multiple times.
\end{remark}

\begin{remark}\label{9.7.3}
    The intermediate value theorem gives another way to show that one can take \(n^{\text{th}}\) roots of a number.
    For instance, to construct the square root of \(2\), consider the function \(f : [0, 2] \to \mathbf{R}\) defined by \(f(x) = x^2\).
    This function is continuous, with \(f(0) = 0\) and \(f(2) = 4\).
    Thus there exists a \(c \in [0, 2]\) such that \(f(c) = 2\), i.e., \(c^2 = 2\).
    (This argument does not show that there is just one square root of \(2\), but it does prove that there is \emph{at least} one square root of \(2\).)
\end{remark}

\begin{corollary}[Images of continuous functions]\label{9.7.4}
    Let \(a < b\), and let \(f : [a, b] \to \mathbf{R}\) be a continuous function on \([a, b]\).
    Let \(M \coloneqq \sup_{x \in [a, b]} f(x)\) be the maximum value of \(f\), and let \(m \coloneqq \inf_{x \in [a, b]} f(x)\) be the minimum value.
    Let \(y\) be a real number between \(m\) and \(M\) (i.e., \(m \leq y \leq M\)).
    Then there exists a \(c \in [a, b]\) such that \(f(c) = y\).
    Furthermore, we have \(f([a, b]) = [m, M]\).
\end{corollary}

\begin{proof}
    We first show that \(\exists\ c \in [a, b]\) such that \(f(c) = y\).
    By maximum principle (Proposition \ref{9.6.7}) we know that \(\exists\ x_M, x_m \in [a, b]\) such that \(f(x_M) = M\) and \(f(x_m) = m\).
    We have either \(x_m \leq x_M\) or \(x_m \geq x_M\).
    Without the loss of generality suppose that \(x_m \leq x_M\).
    Then we have \([x_m, x_M] \subseteq [a, b]\) and by Exercise \ref{ex 9.4.6} we know that \(f\) is continuous on \([x_m, x_M]\).
    Since \(m \leq y \leq M\), by Theorem \ref{9.7.1} \(\exists\ c \in [x_m, x_M]\) such that \(f(c) = y\).
    Since \([x_m, x_M] \subseteq [a, b]\), we have \(c \in [a, b]\).

    Now we show that \(f([a, b]) = [m, M]\).
    Since
    \begin{align*}
                 & \big(M = \sup_{x \in [a, b]} f(x)\big) \land \big(m = \inf_{x \in [a, b]} f(x)\big)                                       \\
        \implies & \forall y \in f([a, b], m \leq y \leq M                                             & \text{(by Definition \ref{9.6.5})}  \\
        \implies & \forall y \in f([a, b], y \in [m, M]                                                & \text{(by Definition \ref{9.1.1})}  \\
        \implies & f([a, b]) \subseteq [m, M]                                                          & \text{(by Definition \ref{3.1.15})}
    \end{align*}
    and
    \begin{align*}
                 & \forall y \in [m, M], \exists\ c \in [a, b] : f(c) = y & \text{(by proof above)}             \\
        \implies & \forall y \in [m, M], y \in f([a, b])                                                        \\
        \implies & [m, M] \subseteq f([a, b]),                            & \text{(by Definition \ref{3.1.15})}
    \end{align*}
    by Proposition \ref{3.1.18} we know that \(f([a, b]) = [m ,M]\).
\end{proof}

\exercisesection

\begin{exercise}\label{ex 9.7.1}
    Prove Corollary \ref{9.7.4}.
\end{exercise}

\begin{proof}
    See Corollary \ref{9.7.4}.
\end{proof}

\begin{exercise}\label{ex 9.7.2}
    Let \(f : [0, 1] \to [0, 1]\) be a continuous function.
    Show that there exists a real number \(x\) in \([0, 1]\) such that \(f(x) = x\).
    This point \(x\) is known as a \emph{fixed point} of \(f\), and this result is a basic example of a \emph{fixed point theorem}, which play an important role in certain types of analysis.
\end{exercise}

\begin{proof}
    Let \(F : [0, 1] \to \mathbf{R}\) be a function where \(F(x) = f(x) - x\).
    Since \(x \mapsto x\) is continuous on \([0, 1]\), by Proposition \ref{9.4.9} \(F\) is continuous on \([0, 1]\).
    Since \(f([0, 1]) \subseteq [0, 1]\), we know that \(0 \leq f(0)\) and \(f(1) \leq 1\).
    Thus
    \begin{align*}
                 & \big(0 \leq f(0)\big) \land \big(f(1) \leq 1\big)                                                         \\
        \implies & \big(0 \leq F(0) = f(0) - 0\big) \land \big(F(1) = f(1) - 1 \leq 0\big)                                   \\
        \implies & F(1) \leq 0 \leq F(0)                                                                                     \\
        \implies & \exists\ x \in [0, 1] : F(x) = 0                                        & \text{(by Theorem \ref{9.7.1})} \\
        \implies & \exists\ x \in [0, 1] : f(x) - x = 0                                                                      \\
        \implies & \exists\ x \in [0, 1] : f(x) = x.
    \end{align*}
\end{proof}
\section{Monotonic functions}\label{sec 9.8}

\begin{definition}[Monotonic functions]\label{9.8.1}
    Let \(X\) be a subset of \(\mathbf{R}\), and let \(f : X \to \mathbf{R}\) be a function.
    We say that \(f\) is \emph{monotone increasing} iff \(f(y) \geq f(x)\) whenever \(x, y \in X\) and \(y > x\).
    We say that \(f\) is \emph{strictly monotone increasing} iff \(f(y) > f(x)\) whenever \(x, y \in X\) and \(y > x\).
    Similarly, we say \(f\) is \emph{monotone decreasing} iff \(f(y) \leq f(x)\) whenever \(x, y \in X\) and \(y > x\), and \emph{strictly monotone decreasing} iff \(f(y) < f(x)\) whenever \(x, y \in X\) and \(y > x\).
    We say that \(f\) is \emph{monotone} if it is monotone increasing or monotone decreasing, and \emph{strictly monotone} if it is strictly monotone increasing or strictly monotone decreasing.
\end{definition}

\begin{note}
    If a function is strictly monotone on a domain \(X\), it is automatically monotone as well on the same domain \(X\).
    Constant functions, when restricted to an arbitrary domain \(X \subseteq \mathbf{R}\), are both monotone increasing and monotone decreasing, but is not strictly monotone
    (unless \(X\) consists of at most one point).
\end{note}

\begin{note}
    Continuous functions are not necessarily monotone, and monotone functions are not necessarily continuous.
\end{note}

\setcounter{theorem}{2}
\begin{proposition}\label{9.8.3}
    Let \(a < b\) be real numbers, and let \(f : [a, b] \to \mathbf{R}\) be a function which is both continuous and strictly monotone increasing.
    Then \(f\) is a bijection from \([a, b]\) to \([f(a), f(b)]\), and the inverse \(f^{-1} : [f(a), f(b)] \to [a, b]\) is also continuous and strictly monotone increasing.
\end{proposition}

\begin{proof}
    We first show that \(f\) is bijective from \([a, b]\) to \([f(a), f(b)]\).
    Since \(a < b\) and \(f\) is strictly monotone increasing, by Definition \ref{9.8.1} we know that \(f(a) < f(b)\).
    In particular, \(\forall\ c \in (a, b)\), we have \(a < c < b\), and thus \(f(a) < f(c) < f(b)\).
    By Definition \ref{9.6.5}, this means \(f\) attains its minimum at \(a\) and attains its maximum at \(b\).
    By Corollary \ref{9.7.4}, we know that \(f([a, b]) = [f(a), f(b)]\), thus \(f\) is surjective from \([a, b]\) to \([f(a), f(b)]\).
    Since \(f\) is strictly monotone increasing, by Definition \ref{9.8.1} \(x \neq y \implies f(x) \neq f(y)\), thus \(f\) is injective from \([a, b]\) to \([f(a), f(b)]\).
    Since \(f\) is both injective and surjective from \([a, b]\) to \([f(a), f(b)]\), we know that \(f\) is bijective from \([a, b]\) to \([f(a), f(b)]\).

    Next we show that \(f^{-1}\) is continuous.
    Let \(y_0 \in [f(a), f(b)]\).
    By Lemma \ref{9.1.12} we know that \([f(a), f(b)]\) is closed, so \(y_0\) is an adherent point of \([f(a), f(b)]\).
    Since \(f\) is bijective from \([a, b]\) to \([f(a), f(b)]\), we know that \(\exists!\ x_0 \in [a, b]\) such that \(f(x_0) = y_0\) and thus \(f^{-1}(y_0) = x_0\).
    Again by Lemma \ref{9.1.12} we know that \([a, b]\) is closed, so \(x_0\) is an adherent point of \([a, b]\).
    To show that \(f^{-1}\) is continuous at \(y_0\), by Definition \ref{9.4.1} we need to show that
    \[
        \lim_{y \to y_0 ; y \in [f(a), f(b)]} f^{-1}(y) = f^{-1}(y_0) = x_0.
    \]
    By Definition \ref{9.3.6}, it suffice to show that
    \[
        \forall\ \varepsilon \in \mathbf{R}^+, \exists\ \delta \in \mathbf{R}^+ : \bigg(\forall\ y \in [f(a), f(b)], \abs*{y - y_0} < \delta \implies \abs*{f^{-1}(y) - x_0} < \varepsilon\bigg).
    \]
    Now fix \(\varepsilon\).
    Let \(x_L = \max(x_0 - \varepsilon, a)\) and \(x_H = \min(x_0 + \varepsilon, b)\).
    Then \(x_L, x_H \in [a, b]\) and \(f(x_L), f(x_H)\) are well-defined.
    Since \(x_L \leq x_0 \leq x_H\) and \(f\) is strictly monotone increasing, we have \(f(x_L) < y_0 < f(x_H)\) and thus \(f(x_L) - y_0 < 0 < f(x_H) - y_0\).
    Let \(\delta = \min(-(f(x_L) - y_0), f(x_H) - y_0)\).
    Then we have
    \begin{align*}
                 & \forall\ y \in [f(a), f(b)] \land \abs*{y - y_0} < \delta                                           \\
        \implies & -\delta < y - y_0 < \delta                                                                          \\
        \implies & f(x_L) - y_0 \leq -\delta < y - y_0 < \delta \leq f(x_H) - y_0                                      \\
        \implies & f(x_L) \leq y \leq f(x_H)                                                                           \\
        \implies & \exists\ x \in [x_L, x_H] : f(x) = y \land x_L \leq x \leq x_H    & \text{(by Theorem \ref{9.7.1})} \\
        \implies & x_0 - \varepsilon \leq x_L \leq x \leq x_H \leq x_0 + \varepsilon                                   \\
        \implies & -\varepsilon \leq x - x_0 \leq \varepsilon                                                          \\
        \implies & \abs*{x - x_0} \leq \varepsilon                                                                     \\
        \implies & \abs*{f^{-1}(y) - x_0} \leq \varepsilon.
    \end{align*}
    Since \(\varepsilon\) is arbitrary, \(f^{-1}\) is continuous at \(y_0\).
    Since \(y_0\) is arbitrary, \(f^{-1}\) is continuous on \([f(a), f(b)]\).

    Next we show that \(f^{-1}\) is strictly monotone increasing.
    Let \(y_1, y_2 \in [f(a), f(b)]\) and \(y_1 < y_2\).
    We want to show that \(f^{-1}(y_1) < f^{-1}(y_2)\).
    Suppose for sake of contradiction that \(f^{-1}(y_1) \geq f^{-1}(y_2)\).
    Since \(f\) is strictly monotone increasing, we know that
    \begin{align*}
                 & f^{-1}(y_1) \geq f^{-1}(y_2)                                            \\
        \implies & f(f^{-1}(y_1)) \geq f(f^{-1}(y_2)) & \text{(by Definition \ref{9.8.1})} \\
        \implies & y_1 \geq y_2.
    \end{align*}
    But this contradict to \(y_1 < y_2\), thus \(f^{-1}(y_1) < f^{-1}(y_2)\).

    Next we show that if the continuity assumption is dropped, then the proposition is false.
    Let \(f : [0, 1] \to \mathbf{R}\) be a function
    \[
        f(x) = \begin{cases}
            x     & \text{if } x \in [0, 0.5)  \\
            x + 1 & \text{if } x \in [0.5, 1].
        \end{cases}
    \]
    Then \(f\) is not continuous and \(f([0, 1]) = [0, 0.5) \cup [1.5, 2] \neq [0, 2]\).

    Next we show that if strict monotonicity is replaced just by monotonicity, then the proposition is false.
    Let \(f : [0, 1] \to \mathbf{R}\) be a function where \(f(x) = 1\).
    Then \(f\) is not bijective.

    Next we deal with strictly monotone decreasing functions.
    Let \(a < b\) be real numbers, and let \(g : [a, b] \to \mathbf{R}\) be a function which is both continuous and strictly monotone decreasing.
    Then we claim that \(g\) is a bijection from \([a, b]\) to \([g(b), g(a)]\), and the inverse \(g^{-1} : [g(b), g(a)] \to [a, b]\) is also continuous and strictly monotone decreasing.
    To proof the claim, We first show that \(g\) is bijective from \([a, b]\) to \([g(b), g(a)]\).
    Since \(a < b\) and \(g\) is strictly monotone decreasing, by Definition \ref{9.8.1} we know that \(g(a) > g(b)\).
    In particular, \(\forall\ c \in (a, b)\), we have \(a < c < b\), and thus \(g(a) > g(c) > g(b)\).
    By Definition \ref{9.6.5}, this means \(g\) attains its minimum at \(b\) and attains its maximum at \(a\).
    By Corollary \ref{9.7.4}, we know that \(g([a, b]) = [g(b), g(a)]\), thus \(g\) is surjective from \([a, b]\) to \([g(b), g(a)]\).
    Since \(g\) is strictly monotone decreasing, by Definition \ref{9.8.1} \(x \neq y \implies g(x) \neq g(y)\), thus \(g\) is injective from \([a, b]\) to \([g(b), g(a)]\).
    Since \(g\) is both injective and surjective from \([a, b]\) to \([g(b), g(a)]\), we know that \(g\) is bijective from \([a, b]\) to \([g(b), g(a)]\).

    Next we show that \(g^{-1}\) is continuous.
    Let \(y_0 \in [g(b), g(a)]\).
    By Lemma \ref{9.1.12} we know that \([g(b), g(a)]\) is closed, so \(y_0\) is an adherent point of \([g(b), g(a)]\).
    Since \(g\) is bijective from \([a, b]\) to \([g(b), g(a)]\), we know that \(\exists!\ x_0 \in [a, b]\) such that \(g(x_0) = y_0\) and thus \(g^{-1}(y_0) = x_0\).
    Again by Lemma \ref{9.1.12} we know that \([a, b]\) is closed, so \(x_0\) is an adherent point of \([a, b]\).
    To show that \(g^{-1}\) is continuous at \(y_0\), by Definition \ref{9.4.1} we need to show that
    \[
        \lim_{y \to y_0 ; y \in [g(b), g(a)]} g^{-1}(y) = g^{-1}(y_0) = x_0.
    \]
    By Definition \ref{9.3.6}, it suffice to show that
    \[
        \forall\ \varepsilon \in \mathbf{R}^+, \exists\ \delta \in \mathbf{R}^+ : \bigg(\forall\ y \in [g(b), g(a)], \abs*{y - y_0} < \delta \implies \abs*{g^{-1}(y) - x_0} < \varepsilon\bigg).
    \]
    Now fix \(\varepsilon\).
    Let \(x_L = \max(x_0 - \varepsilon, a)\) and \(x_H = \min(x_0 + \varepsilon, b)\).
    Then \(x_L, x_H \in [a, b]\) and \(f(x_L), f(x_H)\) are well-defined.
    Since \(x_L \leq x_0 \leq x_H\) and \(g\) is strictly monotone decreasing, we have \(g(x_L) > y_0 > g(x_H)\) and thus \(g(x_L) - y_0 > 0 > g(x_H) - y_0\).
    Let \(\delta = \min(-(g(x_H) - y_0), g(x_L) - y_0)\).
    Then we have
    \begin{align*}
                 & \forall\ y \in [g(b), g(a)] \land \abs*{y - y_0} < \delta                                           \\
        \implies & -\delta < y - y_0 < \delta                                                                          \\
        \implies & g(x_H) - y_0 \leq -\delta < y - y_0 < \delta \leq g(x_L) - y_0                                      \\
        \implies & g(x_H) \leq y \leq g(x_L)                                                                           \\
        \implies & \exists\ x \in [x_L, x_H] : g(x) = y \land x_L \leq x \leq x_H    & \text{(by Theorem \ref{9.7.1})} \\
        \implies & x_0 - \varepsilon \leq x_L \leq x \leq x_H \leq x_0 + \varepsilon                                   \\
        \implies & -\varepsilon \leq x - x_0 \leq \varepsilon                                                          \\
        \implies & \abs*{x - x_0} \leq \varepsilon                                                                     \\
        \implies & \abs*{g^{-1}(y) - x_0} \leq \varepsilon.
    \end{align*}
    Since \(\varepsilon\) is arbitrary, \(g^{-1}\) is continuous at \(y_0\).
    Since \(y_0\) is arbitrary, \(g^{-1}\) is continuous on \([g(b), g(a)]\).

    Finally we show that \(g^{-1}\) is strictly monotone decreasing.
    Let \(y_1, y_2 \in [g(b), g(a)]\) and \(y_1 < y_2\).
    We want to show that \(g^{-1}(y_1) > g^{-1}(y_2)\).
    Suppose for sake of contradiction that \(g^{-1}(y_1) \leq g^{-1}(y_2)\).
    Since \(g\) is strictly monotone decreasing, we know that
    \begin{align*}
                 & g^{-1}(y_1) \leq g^{-1}(y_2)                                            \\
        \implies & g(g^{-1}(y_1)) \geq g(g^{-1}(y_2)) & \text{(by Definition \ref{9.8.1})} \\
        \implies & y_1 \geq y_2.
    \end{align*}
    But this contradict to \(y_1 < y_2\), thus \(g^{-1}(y_1) > g^{-1}(y_2)\).
\end{proof}

\begin{example}\label{9.8.4}
    Let \(n\) be a positive integer and \(R > 0\).
    Since the function \(f(x) \coloneqq x^n\) is strictly increasing on the interval \([0, R]\), we see from Proposition \ref{9.8.3} that this function is a bijection from \([0, R]\) to \([0, R^n]\), and hence there is an inverse from \([0, R^n]\) to \([0, R]\).
    This can be used to give an alternate means to construct the \(n^\text{th}\) root \(x^{1 / n}\) of a number \(x \in [0, R]\) than what was done in Lemma \ref{5.6.5}.
\end{example}

\exercisesection

\begin{exercise}\label{ex 9.8.1}
    Explain why the maximum principle remains true if the hypothesis that \(f\) is continuous is replaced with \(f\) being monotone, or with \(f\) being strictly monotone.
\end{exercise}

\begin{proof}
    Let \(a < b\) be real numbers, and let \(f : [a, b] \to \mathbf{R}\) be a function.
    Suppose that \(f\) is monotone.
    Then we have
    \begin{align*}
                 & \forall\ c \in [a, b]                                                                                           \\
        \implies & a \leq c \leq b                                                            & \text{(by Definition \ref{9.1.1})} \\
        \implies & \big(f(a) \leq f(c) \leq f(b)\big) \lor \big(f(a) \geq f(c) \geq f(b)\big) & \text{(by Definition \ref{9.8.1})} \\
    \end{align*}
    Thus \(f\) attains its maximum at \(f(b)\) and attains its minimum at \(f(a)\) when \(f\) is monotone increasing;
    \(f\) attains its maximum at \(f(a)\) and attains its minimum at \(f(b)\) when \(f\) is monotone decreasing.
    The same argument holds when \(f\) is strictly monotone.
\end{proof}

\begin{exercise}\label{ex 9.8.2}
    Give an example to show that the intermediate value theorem becomes false if the hypothesis that \(f\) is continuous is replaced with \(f\) being monotone, or with \(f\) being strictly monotone.
\end{exercise}

\begin{proof}
    Let \(f : [0, 1] \to \mathbf{R}\) be a function where
    \[
        f(x) = \begin{cases}
            x     & \text{if } x \in [0, 0.5); \\
            x + 1 & \text{if } x \in [0.5, 1].
        \end{cases}
    \]
    Then \(f\) is strictly monotone and thus monotone.
    Since \(\forall\ y \in [0.5, 1.5)\), \(\nexists\ x \in \mathbf{N}\) such that \(f(x) = y\), the intermediate value theorem (Theorem \ref{9.7.1}) does not hold.
\end{proof}

\begin{exercise}\label{ex 9.8.3}
    Let \(a < b\) be real numbers, and let \(f : [a, b] \to \mathbf{R}\) be a function which is both continuous and one-to-one.
    Show that \(f\) is strictly monotone.
\end{exercise}

\begin{proof}
    Since \(a < b\) and \(f\) is injective, we cannot have \(f(a) = f(b)\).
    So we have two cases:
    \begin{enumerate}
        \item \(f(a) < f(b)\).
              Suppose for sake of contradiction that \(f\) is not strictly monotone.
              Then \(\exists\ c \in [a, b]\) such that
              \[
                  \bigg(f(c) \geq f(a) \land f(c) \geq f(b)\bigg) \lor \bigg(f(c) \leq f(a) \land f(c) \leq f(b)\bigg).
              \]
              Since \(f\) is injective, we cannot have \(c = a \land c = b\), thus \(f(c) \neq f(a) \lor f(c) \neq f(b)\).
              Since \(f(a) < f(b)\), we have \(f(c) \neq f(a) \land f(c) \neq f(b)\).
              Then we have
              \begin{align*}
                           & f(c) > f(a) \land f(c) > f(b)                                                                   \\
                  \implies & \exists\ d \in (f(a), f(c)) \cap (f(b), f(c)) :                                                 \\
                           & \bigg(f(c) > d > f(a)\bigg) \land \bigg(f(c) > d > f(b)\bigg)                                   \\
                  \implies & \big(\exists\ c_1 \in (a, c) : f(c_1) = d\big)                & \text{(by Theorem \ref{9.7.1})} \\
                           & \land \big(\exists\ c_2 \in (c, b) : f(c_2) = d\big)          & \text{(by Theorem \ref{9.7.1})} \\
              \end{align*}
              and
              \begin{align*}
                           & f(c) < f(a) \land f(c) < f(b)                                                                   \\
                  \implies & \exists\ d \in (f(c), f(a)) \cap (f(c), f(b)) :                                                 \\
                           & \bigg(f(c) < d < f(a)\bigg) \land \bigg(f(c) < d < f(b)\bigg)                                   \\
                  \implies & \big(\exists\ c_1 \in (a, c) : f(c_1) = d\big)                & \text{(by Theorem \ref{9.7.1})} \\
                           & \land \big(\exists\ c_2 \in (c, b) : f(c_2) = d\big).         & \text{(by Theorem \ref{9.7.1})} \\
              \end{align*}
              But \(f\) is injective, in both case we have \(c_1 \neq c_2\) and \(f(c_1) = f(c_2)\), a contradiction.
              Thus \(f\) is strictly monotone.
              Since \(f(a) < f(b)\), by Definition \ref{9.8.1} \(f\) is strictly monotone increasing.
        \item \(f(a) > f(b)\).
              Suppose for sake of contradiction that \(f\) is not strictly monotone.
              Then \(\exists\ c \in [a, b]\) such that
              \[
                  \bigg(f(c) \geq f(a) \land f(c) \geq f(b)\bigg) \lor \bigg(f(c) \leq f(a) \land f(c) \leq f(b)\bigg).
              \]
              Since \(f\) is injective, we cannot have \(c = a \land c = b\), thus \(f(c) \neq f(a) \lor f(c) \neq f(b)\).
              Since \(f(a) > f(b)\), we have \(f(c) \neq f(a) \land f(c) \neq f(b)\).
              Then we have
              \begin{align*}
                           & f(c) > f(a) \land f(c) > f(b)                                                                   \\
                  \implies & \exists\ d \in (f(a), f(c)) \cap (f(b), f(c)) :                                                 \\
                           & \bigg(f(c) > d > f(a)\bigg) \land \bigg(f(c) > d > f(b)\bigg)                                   \\
                  \implies & \big(\exists\ c_1 \in (a, c) : f(c_1) = d\big)                & \text{(by Theorem \ref{9.7.1})} \\
                           & \land \big(\exists\ c_2 \in (c, b) : f(c_2) = d\big)          & \text{(by Theorem \ref{9.7.1})} \\
              \end{align*}
              and
              \begin{align*}
                           & f(c) < f(a) \land f(c) < f(b)                                                                   \\
                  \implies & \exists\ d \in (f(c), f(a)) \cap (f(c), f(b)) :                                                 \\
                           & \bigg(f(c) < d < f(a)\bigg) \land \bigg(f(c) < d < f(b)\bigg)                                   \\
                  \implies & \big(\exists\ c_1 \in (a, c) : f(c_1) = d\big)                & \text{(by Theorem \ref{9.7.1})} \\
                           & \land \big(\exists\ c_2 \in (c, b) : f(c_2) = d\big).         & \text{(by Theorem \ref{9.7.1})} \\
              \end{align*}
              But \(f\) is injective, in both case we have \(c_1 \neq c_2\) and \(f(c_1) = f(c_2)\), a contradiction.
              Thus \(f\) is strictly monotone.
              Since \(f(a) > f(b)\), by Definition \ref{9.8.1} \(f\) is strictly monotone decreasing.
    \end{enumerate}
    In all cases above we have \(f\) is either strictly monotone increasing or strictly monotone decreasing.
    Thus by Definition \ref{9.8.1} we know that \(f\) is strictly monotone.
\end{proof}

\begin{exercise}\label{ex 9.8.4}
    Prove Proposition \ref{9.8.3}.
    Is the proposition still true if the continuity assumption is dropped, or if strict monotonicity is replaced just by monotonicity?
    How should one modify the proposition to deal with strictly monotone decreasing functions instead of strictly monotone increasing functions?
\end{exercise}

\begin{proof}
    See Proposition \ref{9.8.3}.
\end{proof}

\begin{exercise}\label{ex 9.8.5}
    In this exercise we give an example of a function which has a discontinuity at every rational point, but is continuous at every irrational.
    Since the rationals are countable, we can write them as \(\mathbf{Q} = \{q(0), q(1), q(2), \dots\}\), where \(q : \mathbf{N} \to \mathbf{Q}\) is a bijection from \(\mathbf{N}\) to \(\mathbf{Q}\).
    Now define a function \(g : \mathbf{Q} \to \mathbf{R}\) by setting \(g(q(n)) \coloneqq 2^{-n}\) for each natural number \(n\);
    thus \(g\) maps \(q(0)\) to \(1\), \(q(1)\) to \(2^{-1}\), etc.
    Since \(\sum_{n = 0}^\infty 2^{-n}\) is absolutely convergent, we see that \(\sum_{r \in \mathbf{Q}} g(r)\) is also absolutely convergent.
    Now define the function \(f : \mathbf{R} \to \mathbf{R}\) by
    \[
        f(x) \coloneqq \sum_{r \in \mathbf{Q} : r < x} g(r).
    \]
    Since \(\sum_{r \in \mathbf{Q}} g(r)\) is absolutely convergent, we know that \(f(x)\) is well-defined for every real number \(x\).
    \begin{enumerate}
        \item Show that \(f\) is strictly monotone increasing.
        \item Show that for every rational number \(r\), \(f\) is discontinuous at \(r\).
        \item Show that for every irrational number \(x\), \(f\) is continuous at \(x\).
    \end{enumerate}
\end{exercise}

\begin{proof}{(a)}
    Let \(a, b \in \mathbf{R}\) and \(a < b\).
    By Proposition \ref{5.4.14}, \(\exists\ c \in \mathbf{Q}\) such that \(a < c < b\).
    Then we have
    \begin{align*}
        f(b) & = \sum_{r \in \mathbf{Q} : r < b} g(r)                                                                                                                                      \\
             & = \sum_{r \in \mathbf{Q} : r < a} g(r) + \sum_{r \in \mathbf{Q} : a \leq r < c} g(r) + \sum_{r \in \mathbf{Q} : c \leq r < b} g(r) & \text{(by Proposition \ref{8.2.6}(c))} \\
             & = f(a) + \sum_{r \in \mathbf{Q} : a \leq r < c} g(r) + g(c) + \sum_{r \in \mathbf{Q} : c < r < b} g(r)                             & \text{(by Proposition \ref{8.2.6}(c))} \\
             & > f(a).
    \end{align*}
    and by Definition \ref{9.8.1} \(f\) is strictly monotone increasing.
\end{proof}

\begin{proof}{(b)}
    Let \(\gamma \in \mathbf{Q}\).
    Since \(q\) is bijective, \(\exists!\ n \in \mathbf{N}\) such that \(q(n) = \gamma\).
    Then \(\forall\ x \in (\gamma, \infty)\), we have
    \begin{align*}
        f(x) & = \sum_{r \in \mathbf{Q} : r < x} g(r)                                                                                                           \\
             & = \sum_{r \in \mathbf{Q} : r < \gamma} g(r) + g(\gamma) + \sum_{r \in \mathbf{Q} : \gamma < r < x} g(r) & \text{(by Proposition \ref{8.2.6}(c))} \\
             & = f(\gamma) + g(\gamma) + \sum_{r \in \mathbf{Q} : \gamma < r < x} g(r)                                                                          \\
             & > f(\gamma) + g(\gamma)                                                                                                                          \\
             & = f(\gamma) + 2^{-n}.
    \end{align*}
    Since \(f\) is strictly monotone increasing, we have
    \begin{align*}
                 & x \in (\gamma, \infty)                                                                            \\
        \implies & x > \gamma                                                  & \text{(by Definition \ref{9.1.1})}  \\
        \implies & f(x) > f(\gamma)                                            & \text{(by Definition \ref{9.8.1})}  \\
        \implies & f(x) - f(\gamma) > 2^{-n}                                                                         \\
        \implies & \abs*{f(x) - f(\gamma)} = f(x) - f(\gamma) > 2^{-n}                                               \\
        \implies & f(\gamma^+) \neq f(\gamma)                                  & \text{(by Definition \ref{9.5.1})}  \\
        \implies & \lim_{r \to \gamma ; r \in \mathbf{R}} f(x) \neq f(\gamma). & \text{(by Proposition \ref{9.5.3})}
    \end{align*}
\end{proof}

\begin{proof}{(c)}
    Let \(n \in \mathbf{N}\), let \(E_n\) be a set where
    \[
        E_n = \{r \in \mathbf{Q} : g(r) \geq 2^{-n}\},
    \]
    and let \(f_n : \mathbf{R} \to \mathbf{R}\) be a function
    \[
        f_n(x) = \sum_{r \in E_n : r < x} g(r) = \sum_{r \in \mathbf{Q} : r < x \land g(r) \geq 2^{-n}} g(r).
    \]
    Since \(q\) is bijective, there are at most \(n + 1\) rationals satisfying \(r \in \mathbf{Q} \land g(r) \geq 2^{-n}\), thus \(E_n\) is finite and non-empty.
    Since \(\mathbf{R} \setminus \mathbf{Q} \subseteq \mathbf{R}\), by Lemma \ref{9.1.11} \(\forall\ x_0 \in \mathbf{R} \setminus \mathbf{Q}\), \(x_0\) is an adherent point of \(\mathbf{R} \setminus \mathbf{Q}\).
    Let \(\varepsilon \in \mathbf{R}^+\) and let \(\delta = \min\{\abs*{r - x_0} : r \in E_n\}\).
    Since \(x_0 \notin \mathbf{Q}\), we have \(\delta > 0\).
    Then we have
    \begin{align*}
                 & \forall\ x \in \mathbf{R} \setminus \mathbf{Q}, \abs*{x - x_0} < \delta   \\
        \implies & \big(\forall\ r \in E_n, \abs*{x - x_0} < \delta \leq \abs*{r - x_0}\big) \\
        \implies & \big(\forall\ r \in E_n, \abs*{x - x_0} < \abs*{r - x_0}\big)             \\
        \implies & \big(\forall\ r \in E_n, -\abs*{r - x_0} < x - x_0 < \abs*{r - x_0}\big)  \\
        \implies & \big(\forall\ r \in E_n \land r < x_0, r - x_0 < x - x_0 < x_0 - r\big)   \\
        \implies & \big(\forall\ r \in E_n, r < x_0 \land r < x\big)                         \\
        \implies & \{r \in E_n : r < x\} = \{r \in E_n : r < x_0\}                           \\
        \implies & f_n(x) = f_n(x_0)                                                         \\
        \implies & 0 = \abs*{f_n(x) - f_n(x_0)} < \varepsilon.
    \end{align*}
    Since \(\varepsilon\) is arbitrary, by Definition \ref{9.3.6} we have \(\lim_{x \to x_0 ; x \in \mathbf{R} \setminus \mathbf{Q}} f_n(x) = f_n(x_0)\), and by Definition \ref{9.4.1} \(f_n\) is continuous at \(x_0\).

    Now we show that \(f\) is continuous at \(x_0\).
    We have
    \begin{align*}
         & \forall\ x \in \mathbf{R} \setminus \mathbf{Q}, \abs*{f(x) - f_n(x)}                                                                                 \\
         & = \abs*{\sum_{r \in \mathbf{Q} : r < x} g(r) - \sum_{r \in E_n : r < x} g(r)}                                                                        \\
         & = \abs*{\sum_{r \in \mathbf{Q} : r < x} g(r) - \sum_{r \in \mathbf{Q} : r < x \land g(r) \geq 2^{-n}} g(r)}                                          \\
         & = \abs*{\sum_{r \in \mathbf{Q} : r < x \land g(r) < 2^{-n}} g(r)}                                           & \text{(by Proposition \ref{8.2.6}(c))} \\
         & = \sum_{r \in \mathbf{Q} : r < x \land g(r) < 2^{-n}} g(r)                                                                                           \\
         & = \sum_{r \in \mathbf{Q} : r < x \land g(r) \leq 2^{-(n + 1)}} g(r)                                                                                  \\
         & \leq \sum_{r \in \mathbf{Q} : g(r) \leq 2^{-(n + 1)}} g(r)                                                                                           \\
         & \leq \sum_{k = n + 1}^\infty 2^{-k}                                                                                                                  \\
         & = 2^{-(n + 1)} \bigg(\sum_{k = 0}^\infty 2^{-k}\bigg)                                                                                                \\
         & = 2^{-n}.                                                                                                   & \text{(by Lemma \ref{7.3.3})}
    \end{align*}
    By Proposition \ref{5.4.14}, \(\forall\ \varepsilon \in \mathbf{R}^+\), \(\exists\ n \in \mathbf{N}\) such that \(2^{-n} < \varepsilon / 2\).
    From the proof above we also have
    \[
        \forall\ x \in \mathbf{R} \setminus \mathbf{Q}, \abs*{x - x_0} < \delta \implies f_n(x) = f_n(x_0) \\
    \]
    Combine the results above we have
    \begin{align*}
         & \abs*{f(x) - f(x_0)}                               \\
         & = \abs*{f(x) - f_n(x) + f_n(x) - f(x_0)}           \\
         & \leq \abs*{f(x) - f_n(x)} + \abs*{f_n(x) - f(x_0)} \\
         & = \abs*{f(x) - f_n(x)} + \abs*{f_n(x_0) - f(x_0)}  \\
         & \leq 2^{-n} + 2^{-n}                               \\
         & < \varepsilon / 2 + \varepsilon / 2                \\
         & = \varepsilon.
    \end{align*}
    Since \(\varepsilon\) is arbitrary, by Definition \ref{9.3.6} we have \(\lim_{x \to x_0 ; x \in \mathbf{R} \setminus \mathbf{Q}} f(x) = f(x_0)\), and by Definition \ref{9.4.1} \(f\) is continuous at \(x_0\).
\end{proof}
\section{Uniform continuity}\label{sec 9.9}

\setcounter{theorem}{1}
\begin{definition}[Uniform continuity]\label{9.9.2}
    Let \(X\) be a subset of \(\mathbf{R}\), and let \(f : X \to \mathbf{R}\) be a function.
    We say that \(f\) is \emph{uniformly continuous} on \(X\) if, for every \(\varepsilon > 0\), there exists a \(\delta > 0\) such that \(f(x)\) and \(f(x_0)\) are \(\varepsilon\)-close whenever \(x, x_0 \in X\) are two points in \(X\) which are \(\delta\)-close.
\end{definition}

\begin{remark}\label{9.9.3}
    This definition should be compared with the notion of continuity.
    From Proposition \ref{9.4.7}(c), we know that a function \(f\) is \emph{continuous} if for every \(\varepsilon > 0\), and every \(x_0 \in X\), there is a \(\delta > 0\) such that \(f(x)\) and \(f(x_0)\) are \(\varepsilon\)-close whenever \(x \in X\) is \(\delta\)-close to \(x_0\).
    The difference between uniform continuity and continuity is that in uniform continuity one can take a single \(\delta\) which works for all \(x_0 \in X\);
    for ordinary continuity, each \(x_0 \in X\) might use a different \(\delta\).
    Thus every uniformly continuous function is continuous, but not conversely.
\end{remark}

\setcounter{theorem}{4}
\begin{definition}[Equivalent sequences]\label{9.9.5}
    Let \(m\) be an integer, let \((a_n)_{n = m}^\infty\) and \((b_n)_{n = m}^\infty\) be two sequences of real numbers, and let \(\varepsilon > 0\) be given.
    We say that \((a_n)_{n = m}^\infty\) is \(\varepsilon\)-close to \((b_n)_{n = m}^\infty\) iff \(a_n\) is \(\varepsilon\)-close to \(b_n\) for each \(n \geq m\).
    We say that \((a_n)_{n = m}^\infty\) is eventually \(\varepsilon\)-close to \((b_n)_{n = m}^\infty\) iff there exists an \(N \geq m\) such that the sequences \((a_n)_{n = N}^\infty\) and \((b_n)_{n = N}^\infty\) are \(\varepsilon\)-close.
    Two sequences \((a_n)_{n = m}^\infty\) and \((b_n)_{n = m}^\infty\) are equivalent iff for each \(\varepsilon > 0\), the sequences \((a_n)_{n = m}^\infty\) and \((b_n)_{n = m}^\infty\) are eventually \(\varepsilon\)-close.
\end{definition}

\begin{remark}\label{9.9.6}
    One could debate whether \(\varepsilon\) should be assumed to be rational or real, but a minor modification of Proposition \ref{6.1.4} shows that this does not make any difference to the above definitions.
\end{remark}

\begin{lemma}\label{9.9.7}
    Let \((a_n)_{n = 1}^\infty\) and \((b_n)_{n = 1}^\infty\) be sequences of real numbers
    (not necessarily bounded or convergent).
    Then \((a_n)_{n = 1}^\infty\) and \((b_n)_{n = 1}^\infty\) are equivalent if and only if \(\lim_{n \to \infty} (a_n - b_n) = 0\).
\end{lemma}

\begin{proof}
    \begin{align*}
             & (a_n)_{n = 1}^\infty = (b_n)_{n = 1}^\infty                                                                              \\
        \iff & \forall\ \varepsilon \in \mathbf{R}^+, \exists\ N \in \mathbf{N} \land N \geq 1 :   & \text{(by Definition \ref{9.9.5})} \\
             & (\forall\ n \in \mathbf{N} \land n \geq N, \abs*{a_n - b_n} \leq \varepsilon)                                            \\
        \iff & \forall\ \varepsilon \in \mathbf{R}^+, \exists\ N \in \mathbf{N} \land N \geq 1 :                                        \\
             & (\forall\ n \in \mathbf{N} \land n \geq N, \abs*{(a_n - b_n) - 0} \leq \varepsilon)                                      \\
        \iff & \lim_{n \to \infty} (a_n - b_n) = 0.                                                & \text{(by Definition \ref{6.1.5})}
    \end{align*}
\end{proof}

\begin{proposition}\label{9.9.8}
    Let \(X\) be a subset of \(\mathbf{R}\), and let \(f : X \to \mathbf{R}\) be a function.
    Then the following two statements are logically equivalent:
    \begin{enumerate}
        \item \(f\) is uniformly continuous on \(X\).
        \item Whenever \((x_n)_{n = 0}^\infty\) and \((y_n)_{n = 0}^\infty\) are two equivalent sequences consisting of elements of \(X\), the sequences \((f(x_n))_{n = 0}^\infty\) and \((f(y_n))_{n = 0}^\infty\) are also equivalent.
    \end{enumerate}
\end{proposition}

\begin{proof}
    We first show that the first statement implies the second statement.
    By Definition \ref{9.9.2}, \(f\) is uniformly continuous on \(X\) iff
    \[
        \forall\ \varepsilon \in \mathbf{R}^+, \exists\ \delta \in \mathbf{R}^+ : \bigg(\forall\ x, y \in X, \abs*{x - y} < \delta \implies \abs*{f(x) - f(y)} \leq \varepsilon\bigg).
    \]
    By Definition \ref{9.9.5}, \((x_n)_{n = 0}^\infty = (y_n)_{n = 0}^\infty\) iff
    \[
        \forall\ \varepsilon \in \mathbf{R}^+, \exists\ N \in \mathbf{N} : \bigg(\forall\ n \in \mathbf{N} \land n \geq N, \abs*{x_n - y_n} \leq \varepsilon\bigg).
    \]
    In particular, we have
    \[
        \exists\ N \in \mathbf{N} : \bigg(\forall\ n \in \mathbf{N} \land n \geq N, \abs*{x_n - y_n} \leq \delta / 2 < \delta\bigg).
    \]
    Since \(\forall\ n \geq 0\) we have \(x_n, y_n \in X\), by replacing \(x, y\) with \(x_n, y_n\) we see that
    \[
        \forall\ \varepsilon \in \mathbf{R}^+, \exists\ \delta \in \mathbf{R}^+ : \bigg(\forall\ x_n, y_n \in X, \abs*{x_n - y_n} < \delta \implies \abs*{f(x_n) - f(y_n)} \leq \varepsilon\bigg).
    \]
    Thus we have
    \[
        \forall\ \varepsilon \in \mathbf{R}^+, \exists\ N \in \mathbf{N} : \bigg(\forall\ n \in \mathbf{N} \land n \geq N, \abs*{f(x_n) - f(y_n)} \leq \varepsilon\bigg)
    \]
    and by Definition \ref{9.9.5} \((f(x_n))_{n = 0}^\infty = (f(y_n))_{n = 0}^\infty\).

    Now we show that the second statement implies the first statement.
    We have \(\forall\ n \in \mathbf{N}\), \(x_n, y_n \in X\) and \((x_n)_{n = 0}^\infty = (y_n)_{n = 0}^\infty \implies (f(x_n))_{n = 0}^\infty = (f(y_n))_{n = 0}^\infty\).
    Suppose for sake of contradiction that \(f\) is not uniformly continuous on \(X\).
    Then we have
    \[
        \exists\ \varepsilon \in \mathbf{R}^+ : \forall\ \delta \in \mathbf{R}^+, \bigg(\forall\ x, y \in X, \abs*{x - y} < \delta \land \abs*{f(x) - f(y)} > \varepsilon\bigg).
    \]
    By replacing \(x, y\) with \(x_n, y_n\) we have
    \[
        \exists\ \varepsilon \in \mathbf{R}^+ : \forall\ \delta \in \mathbf{R}^+, \bigg(\forall\ x_n, y_n \in X, \abs*{x_n - y_n} < \delta \land \abs*{f(x_n) - f(y_n)} > \varepsilon\bigg).
    \]
    Since \((x_n)_{n = 0}^\infty = (y_n)_{n = 0}^\infty\), we have
    \[
        \exists\ N_1 \in \mathbf{N} : \bigg(\forall\ n \in \mathbf{N} \land n \geq N_1, \abs*{x_n - y_n} \leq \delta / 2 < \delta\bigg).
    \]
    Since \((f(x_n))_{n = 0}^\infty = (f(y_n))_{n = 0}^\infty\), we have
    \[
        \exists\ N_2 \in \mathbf{N} : \bigg(\forall\ n \in \mathbf{N} \land n \geq N_2, \abs*{f(x_n) - f(y_n)} \leq \varepsilon\bigg).
    \]
    Let \(N = \max(N_1, N_2)\).
    Then we have
    \[
        \forall\ n \in \mathbf{N} \land n \geq N, \abs*{x_n - y_n} < \delta \land \abs*{f(x_n) - f(y_n)} \leq \varepsilon,
    \]
    a contradiction.
    Thus \(f\) is uniformly continuous on \(X\).
\end{proof}

\begin{remark}\label{9.9.9}
    The reader should compare Proposition \ref{9.9.8} with Proposition \ref{9.3.9}.
    Proposition \ref{9.3.9} asserted that if \(f\) was continuous, then \(f\) maps convergent sequences to convergent sequences.
    In contrast, Proposition \ref{9.9.8} asserts that if \(f\) is \emph{uniformly} continuous, then \(f\) maps \emph{equivalent} pairs of sequences to equivalent pairs of sequences.
    To see how the two Propositions are connected, observe from Lemma \ref{9.9.7} that \((x_n)_{n = 0}^\infty\) will converge to \(x_*\) if and only if the sequences \((x_n)_{n = 0}^\infty\) and \((x_*)_{n = 0}^\infty\) are equivalent.
\end{remark}

\setcounter{theorem}{11}
\begin{proposition}\label{9.9.12}
    Let \(X\) be a subset of \(\mathbf{R}\), and let \(f : X \to \mathbf{R}\) be a uniformly continuous function.
    Let \((x_n)_{n = 0}^\infty\) be a Cauchy sequence consisting entirely of elements in \(X\).
    Then \((f(x_n))_{n = 0}^\infty\) is also a Cauchy sequence.
\end{proposition}

\begin{proof}
    Since \(f\) is uniformly continuous, by Definition \ref{9.9.2} we have
    \[
        \forall\ \varepsilon \in \mathbf{R}^+, \exists\ \delta \in \mathbf{R}^+ : \bigg(\forall\ x, y \in X, \abs*{x - y} < \delta \implies \abs*{f(x) - f(y)} \leq \varepsilon\bigg).
    \]
    Since \((x_n)_{n = 0}^\infty\) is a Cauchy sequence, by Definition \ref{6.1.3} we have
    \[
        \forall\ \varepsilon \in \mathbf{R}^+, \exists\ N \in \mathbf{N} : \bigg(\forall\ i, j \in \mathbf{N} \land i, j \geq N, \abs*{x_i - x_j} \leq \varepsilon\bigg).
    \]
    In particular, we have
    \[
        \exists\ N \in \mathbf{N} : \bigg(\forall\ i, j \in \mathbf{N} \land i, j \geq N, \abs*{x_i - x_j} \leq \delta / 2 < \delta\bigg).
    \]
    Since \(x_i, x_j \in X\), we have
    \[
        \abs*{x_i - x_j} < \delta \implies \abs*{f(x_i) - f(x_j)} \leq \varepsilon.
    \]
    Thus we have
    \[
        \forall\ \varepsilon \in \mathbf{R}^+, \exists\ N \in \mathbf{N} : \bigg(\forall\ i, j \in \mathbf{N} \land i, j \geq N, \abs*{f(x_i) - f(x_j)} \leq \varepsilon\bigg).
    \]
    and by Definition \ref{6.1.3} \((f(x_n))_{n = 0}^\infty\) is a Cauchy sequence.
\end{proof}

\setcounter{theorem}{13}
\begin{corollary}\label{9.9.14}
    Let \(X\) be a subset of \(\mathbf{R}\), let \(f : X \to \mathbf{R}\) be a uniformly continuous function, and let \(x_0\) be an adherent point of \(X\).
    Then the limit \(\lim_{x \to x_0 ; x \in X} f(x)\) exists
    (in particular, it is a real number ).
\end{corollary}

\begin{proof}
    Since \(x_0\) is an adherent point of \(X\), by Lemma \ref{9.1.14} there exists a sequence \((a_n)_{n = 0}^\infty\), consisting entirely of elements in \(X\), which converges to \(x_0\).
    Since \(\lim_{n \to \infty} a_n = x_0\), by Proposition \ref{6.1.12} \((a_n)_{n = 0}^\infty\) is a Cauchy sequence.
    Since \(f\) is uniformly continuous, by Proposition \ref{9.9.12} \((f(a_n))_{n = 0}^\infty\) is also a Cauchy sequence.
    Let \(L = \lim_{n \to \infty} f(a_n)\).
    By Proposition \ref{9.9.8}, we know that if \((a_n)_{n = 0}^\infty\) and \((b_n)_{n = 0}^\infty\) are equivalent, then \((f(a_n))_{n = 0}^\infty\) and \((f(b_n))_{n = 0}^\infty\) are equivalent.
    By Proposition \ref{6.1.7}, we know that \(\lim_{n \to \infty} f(b_n) = L\).
    Since \((b_n)_{n = 0}^\infty\) is arbitrary, by Proposition \ref{9.3.9} we know that \(f\) converges to \(L\) at \(x_0\) in \(X\), and thus \(\lim_{x \to x_0 ; x \in X} f(x) = L\) exists.

    Now we show that \(f : (0, 2) \to \mathbf{R}\) defined by \(f(x) = 1 / x\) is not uniformly continuous.
    Suppose for sake of contradiction that \(f\) is uniformly continuous.
    Since \(0\) is an adherent point of \((0, 2)\), we know that \(\lim_{x \to 0 ; x \in (0, 2)} f(x)\) exists.
    Since \(f\) is continuous, \(f(0) = 1 / 0\) must exist, a contradiction.
    Thus \(f\) is not uniformly continuous.
\end{proof}

\begin{proposition}\label{9.9.15}
    Let \(X\) be a subset of \(\mathbf{R}\), and let \(f : X \to \mathbf{R}\) be a uniformly continuous function.
    Suppose that \(E\) is a bounded subset of \(X\).
    Then \(f(E)\) is also bounded.
\end{proposition}

\begin{proof}
    Suppose for sake of contradiction that \(f(E)\) is not bounded.
    Thus for every real number \(M\) there exists an element \(x \in E\) such that \(\abs*{f(x)} \geq M\).

    In particular, for every natural number \(n\), the set \(\{x \in E : \abs*{f(x)} \geq n\}\) is non-empty.
    We can thus choose a sequence \((x_n)_{n = 0}^\infty\) in \(E\) such that \(\abs*{f(x_n)} \geq n\) for all \(n\).
    Since \((x_n)_{n = 0}^\infty\) in \(E\), by Bolzano-Weierstrass theorem (Theorem \ref{6.6.8}) there exists a subsequence \((x_{n_j})_{j = 0}^\infty\) which converges, where \(n_0 < n_1 < n_2 < \dots\) is an increasing sequence of natural numbers.
    In particular, we see that \(n_j \geq j\) for all \(j \in \mathbf{N}\) (use induction).

    Let \(\lim_{n \to \infty} x_n = x_*\).
    Since \((x_n)_{n = 0}^\infty\) in \(E\), by Lemma \ref{9.1.14} we know that \(x_*\) is an adherent point of \(E\).
    Since \(f\) is continuous on \(X\), by Exercise \ref{ex 9.4.6} \(f\) is continuous on \(E\).
    In particular, \(f\) is uniformly continuous on \(E\).
    Thus by Corollary \ref{9.9.14} we know that \(\lim_{x \to x_* ; x \in E} f(x)\) exists.
    By Proposition \ref{9.3.9} we see that
    \[
        \lim_{j \to \infty} f(x_{n_j}) = \lim_{x \to x_* ; x \in E} f(x).
    \]
    Thus the sequence \((f(x_{n_j}))_{j = 0}^\infty\) is convergent, and hence it is bounded.
    On the other hand, we know from the construction that \(\abs*{f(x_{n_j})} \geq n_j \geq j\) for all \(j\), and hence the sequence \((f(x_{n_j}))_{j = 0}^\infty\) is not bounded, a contradiction.
\end{proof}
\chapter{Differentiation of functions}\label{ch 10}

\section{Basic definitions}\label{sec 10}

\begin{note}
    We can now define derivatives analyti cally, using limits, in contrast to the geometric definition of derivatives, which uses tangents.
    The advantage of working analytically is that
    (a) we do not need to know the axioms of geometry, and
    (b) these definitions can be modified to handle functions of several variables, or functions whose values are vectors instead of scalar.
    Furthermore, one's geometric intuition becomes difficult to rely on once one has more than three dimensions in play.
    (Conversely, one can use one's experience in analytic rigour to extend one's geometric intuition to such abstract settings;
    as mentioned earlier, the two viewpoints complement rather than oppose each other.)
\end{note}

\begin{definition}[Differentiability at a point]\label{10.1.1}
    Let \(X\) be a subset of \(\mathbf{R}\), and let \(x_0 \in X\) be an element of \(X\) which is also a limit point of \(X\).
    Let \(f : X \to \mathbf{R}\) be a function.
    If the limit
    \[
        \lim_{x \to x_0 ; x \in X \setminus \{x_0\}} \frac{f(x) - f(x_0)}{x - x_0}
    \]
    converges to some real number \(L\), then we say that \(f\) is \emph{differentiable at \(x_0\) on \(X\) with derivative \(L\)}, and write \(f'(x_0) \coloneqq L\).
    If the limit does not exist, or if \(x_0\) is not an element of \(X\) or not a limit point of \(X\), we leave \(f'(x_0)\) undefined, and say that \(f\) is \emph{not differentiable at \(x_0\) on \(X\)}.
\end{definition}

\begin{remark}\label{10.1.2}
    Note that we need \(x_0\) to be a limit point in order for \(x_0\) to be adherent to \(X \setminus \{x_0\}\), otherwise the limit
    \[
        \lim_{x \to x_0 ; x \in X \setminus \{x_0\}} \frac{f(x) - f(x_0)}{x - x_0}
    \]
    would automatically be undefined.
    In particular, we do not define the derivative of a function at an isolated point;
    In practice, the domain \(X\) will almost always be an interval, and so by Lemma \ref{9.1.21} all elements \(x_0\) of \(X\) will automatically be limit points and we will not have to care much about these issues.
\end{remark}

\setcounter{theorem}{3}
\begin{remark}\label{10.1.4}
    This point is trivial, but it is worth mentioning:
    if \(f : X \to \mathbf{R}\) is differentiable at \(x_0\), and \(g : X \to \mathbf{R}\) is equal to \(f\) (i.e., \(g(x) = f(x)\) for all \(x \in X\)), then \(g\) is also differentiable at \(x_0\) and \(g'(x_0) = f'(x_0)\).
    However, if two functions \(f\) and \(g\) merely have the same value at \(x_0\), i.e., \(g(x_0) = f(x_0)\), this does not imply that \(g'(x_0) = f'(x_0)\).
    Thus there is a big difference between two functions being equal on their whole domain, and merely being equal at one point.
\end{remark}

\begin{remark}\label{10.1.5}
    One sometimes writes \(\frac{df}{dx}\) instead of \(f'\).
    This notation is of course very familiar and convenient, but one has to be a little careful, because it is only safe to use as long as \(x\) is the only variable used to represent the input for \(f\);
    otherwise one can get into all sorts of trouble.
    Because of this possible source of confusion, we will refrain from using the notation \(\frac{df}{dx}\) whenever it could possibly lead to confusion.
    (This confusion becomes even worse in the calculus of several variables, and the standard notation of \(\frac{\partial f}{\partial x}\) can lead to some serious ambiguities.
    There are ways to resolve these ambiguities, most notably by introducing the notion of differentiation along vector fields, but this is beyond the scope of this text.)
\end{remark}

\begin{example}\label{10.1.6}
    Let \(f : \mathbf{R} \to \mathbf{R}\) be the function \(f(x) \coloneqq \abs*{x}\), and let \(x_0 = 0\).
    To see whether \(f\) is differentiable at \(0\) on \(\mathbf{R}\), we compute the limit
    \[
        \lim_{x \to 0 ; x \in \mathbf{R} \setminus \{0\}} \frac{f(x) - f(0)}{x - 0} = \lim_{x \to 0 ; x \in \mathbf{R} \setminus \{0\}} \frac{\abs*{x}}{x}.
    \]
    Now we take left limits and right limits.
    The right limit is
    \[
        \lim_{x \to 0 ; x \in (0, \infty)} \frac{\abs*{x}}{x} = \lim_{x \to 0 ; x \in (0, \infty)} \frac{x}{x} = \lim_{x \to 0 ; x \in (0, \infty)} 1 = 1,
    \]
    while the left limit is
    \[
        \lim_{x \to 0 ; x \in (-\infty, 0)} \frac{\abs*{x}}{x} = \lim_{x \to 0 ; x \in (-\infty, 0)} \frac{-x}{x} = \lim_{x \to 0 ; x \in (-\infty, 0)} -1 = -1,
    \]
    and these limits do not match.
    Thus \(\lim_{x \to 0 ; x \in (0, \infty)} \frac{\abs*{x}}{x}\) does not exist, and \(f\) is not differentiable at \(0\) on \(\mathbf{R}\).
    However, if one restricts \(f\) to \([0, \infty)\), then the restricted function \(f|_{[0, \infty)}\) \emph{is} differentiable at \(0\) on \([0, \infty)\), with derivative \(1\):
    \[
    \lim_{x \to 0 ; x \in [0, \infty) \setminus \{0\}} \frac{f(x) - f(0)}{x - 0} = \lim_{x \to 0 ; x \in (0, \infty)} \frac{\abs*{x}}{x} = 1.
        \]
        Similarly, when one restricts \(f\) to \((-\infty, 0]\), the restricted function \(f|_{(-\infty, 0]}\) is differentiable at \(0\) on \((-\infty, 0]\), with derivative \(-1\).
    Thus even when a function is not differentiable, it is sometimes possible to restore the differentiability by restricting the domain of the function.
\end{example}

\begin{proposition}[Newton's approximation]\label{10.1.7}
    Let \(X\) be a subset of \(\mathbf{R}\), let \(x_0 \in X\) be a limit point of \(X\), let \(f : X \to \mathbf{R}\) be a function, and let \(L\) be a real number.
    Then the following statements are logically equivalent:
    \begin{enumerate}
        \item \(f\) is differentiable at \(x_0\) on \(X\) with derivative \(L\).
        \item For every \(\varepsilon > 0\), there exists a \(\delta > 0\) such that \(f(x)\) is \(\varepsilon \abs*{x - x_0}\)-close to \(f(x_0) + L(x - x_0)\) whenever \(x \in X\) is \(\delta\)-close to \(x_0\), i.e., we have
              \[
                  \abs*{f(x) - (f(x_0) + L(x - x_0))} \leq \varepsilon \abs*{x - x_0}
              \]
              whenever \(x \in X\) and \(\abs*{x - x_0} \leq \delta\).
    \end{enumerate}
\end{proposition}

\begin{proof}
    We first show that the first statement implies the second statement.
    Since \(f\) is differentiable at \(x_0\) on \(X\) with derivative \(L\), by Definition \ref{10.1.1} we have
    \[
        \lim_{x \to x_0 ; x \in X \setminus \{x_0\}} \frac{f(x) - f(x_0)}{x - x_0} = L.
    \]
    By Definition \ref{9.3.6} this means
    \[
        \forall\ \varepsilon \in \mathbf{R}^+, \exists\ \delta \in \mathbf{R}^+ : \bigg(\forall\ x \in X \setminus \{x_0\}, \abs*{x - x_0} < \delta \implies \abs*{\frac{f(x) - f(x_0)}{x - x_0} - L} \leq \varepsilon\bigg).
    \]
    Thus we have
    \begin{align*}
                 & \forall\ x \in X \setminus \{x_0\}, \abs*{x - x_0} \leq \delta / 2 < \delta             \\
        \implies & \abs*{\frac{f(x) - f(x_0)}{x - x_0} - L} \leq \varepsilon                               \\
        \implies & \abs*{\frac{f(x) - f(x_0)}{x - x_0} - L} \abs*{x - x_0} \leq \varepsilon \abs*{x - x_0} \\
        \implies & \abs*{(f(x) - f(x_0)) - L(x - x_0)} \leq \varepsilon \abs*{x - x_0}                     \\
        \implies & \abs*{f(x) - \big(f(x_0) + L(x - x_0)\big)} \leq \varepsilon \abs*{x - x_0}.
    \end{align*}
    If \(x = x_0\), then we have
    \begin{align*}
                 & 0 = \abs*{x_0 - x_0} \leq \delta / 2 < \delta                                              \\
        \implies & 0 = \abs*{f(x_0) - \big(f(x_0) + L(x_0 - x_0)\big)} \leq \varepsilon \abs*{x_0 - x_0} = 0.
    \end{align*}
    Thus we have \(\forall\ \varepsilon \in \mathbf{R}^+\), \(\exists\ \delta \in \mathbf{R}^+\) such that
    \[
        \forall\ x \in X, \abs*{x - x_0} \leq \delta / 2 \implies \abs*{f(x) - \big(f(x_0) + L(x - x_0)\big)} \leq \varepsilon \abs*{x - x_0}.
    \]

    Now we show that the second statement implies the first statement.
    By hypothesis we have \(\forall\ \varepsilon \in \mathbf{R}^+\), \(\exists\ \delta \in \mathbf{R}^+\) such that
    \[
        \forall\ x \in X, \abs*{x - x_0} \leq \delta \implies \abs*{f(x) - \big(f(x_0) + L(x - x_0)\big)} \leq \varepsilon \abs*{x - x_0}.
    \]
    In particular, we have
    \[
        \forall\ x \in X \setminus \{x_0\}, \abs*{x - x_0} \leq \delta \implies \abs*{f(x) - \big(f(x_0) + L(x - x_0)\big)} \leq \varepsilon \abs*{x - x_0}.
    \]
    Thus we have
    \begin{align*}
                 & \forall\ x \in X \setminus \{x_0\}, \abs*{x - x_0} \leq \delta                      \\
        \implies & \abs*{f(x) - \big(f(x_0) + L(x - x_0)\big)} \leq \varepsilon \abs*{x - x_0}         \\
        \implies & \frac{\abs*{f(x) - \big(f(x_0) + L(x - x_0)\big)}}{\abs*{x - x_0}} \leq \varepsilon \\
        \implies & \abs*{\frac{f(x) - f(x_0)}{x - x_0} - L} \leq \varepsilon.
    \end{align*}
    By Definition \ref{9.3.6} this means
    \[
        \lim_{x \to x_0 ; x \in X \setminus \{x_0\}} \frac{f(x) - f(x_0)}{x - x_0} = L
    \]
    and by Definition \ref{10.1.1} we know that \(f\) is differentiable at \(x_0\) on \(X\) with derivative \(L\).
\end{proof}

\begin{remark}\label{10.1.8}
    Newton's approximation is of course named after the great scientist and mathematician Isaac Newton (1642 -- 1727), one of the founders of differential and integral calculus.
\end{remark}

\begin{remark}\label{10.1.9}
    We can phrase Proposition \ref{10.1.7} in a more informal way:
    if \(f\) is differentiable at \(x_0\), then one has the approximation \(f(x) \approx f(x_0) + f'(x_0)(x - x_0)\), and conversely.
\end{remark}

\begin{proposition}[Differentiability implies continuity]\label{10.1.10}
    Let \(X\) be a subset of \(\mathbf{R}\), let \(x_0 \in X\) be a limit point of \(X\), and let \(f : X \to \mathbf{R}\) be a function.
    If \(f\) is differentiable at \(x_0\), then \(f\) is also continuous at \(x_0\).
\end{proposition}

\begin{proof}
    Since \(f\) is differentiable at \(x_0\), by Definition \ref{10.1.1} we have
    \[
        L = \lim_{x \to x_0 ; x \in X \setminus \{x_0\}} \frac{f(x) - f(x_0)}{x - x_0}
    \]
    for some \(L \in \mathbf{R}\).
    By Proposition \ref{10.1.7}, we have \(\forall\ \varepsilon \in \mathbf{R}^+\), \(\exists\ \delta \in \mathbf{R}^+\) such that
    \begin{align*}
                 & \forall\ x \in X, \abs*{x - x_0} \leq \delta                                                                \\
        \implies & \abs*{f(x) - (f(x_0) + L(x - x_0))} \leq \varepsilon \abs*{x - x_0}                                         \\
        \implies & \abs*{f(x) - (f(x_0) + L(x - x_0))} + \abs*{L(x - x_0)} \leq \varepsilon \abs*{x - x_0} + \abs*{L(x - x_0)} \\
        \implies & \abs*{f(x) - (f(x_0) + L(x - x_0))} + \abs*{L(x - x_0)} \leq (\varepsilon + \abs*{L}) \abs*{x - x_0}        \\
        \implies & \abs*{f(x) - (f(x_0) + L(x - x_0)) + L(x - x_0)}                                                            \\
                 & \leq \abs*{f(x) - (f(x_0) + L(x - x_0))} + \abs*{L(x - x_0)}                                                \\
                 & \leq (\varepsilon + \abs*{L}) \abs*{x - x_0}                                                                \\
        \implies & \abs*{f(x) - f(x_0)} \leq (\varepsilon + \abs*{L}) \abs*{x - x_0}.
    \end{align*}
    Let \(\delta' = \min(\delta, \varepsilon / (\varepsilon + \abs*{L}))\).
    Then we have
    \begin{align*}
                 & \forall\ x \in X, \abs*{x - x_0} < \delta' \leq \delta                                                                                                                       \\
        \implies & \abs*{f(x) - f(x_0)} \leq (\varepsilon + \abs*{L}) \abs*{x - x_0}                                                                                                            \\
        \implies & \abs*{f(x) - f(x_0)} \leq (\varepsilon + \abs*{L}) \abs*{x - x_0} \leq (\varepsilon + \abs*{L}) \delta' \leq (\varepsilon + \abs*{L}) \varepsilon / (\varepsilon + \abs*{L}) \\
        \implies & \abs*{f(x) - f(x_0)} \leq \varepsilon.
    \end{align*}
    Thus by Definition \ref{9.3.6} we have \(\lim_{x \to x_0 ; x \in X} f(x) = f(x_0)\), and by Definition \ref{9.4.1} \(f\) is continuous at \(x_0\).
\end{proof}

\begin{definition}[Differentiability on a domain]\label{10.1.11}
    Let \(X\) be a subset of \(\mathbf{R}\), and let \(f : X \to \mathbf{R}\) be a function.
    We say that \(f\) is \emph{differentiable on} \(X\) if, for every limit point \(x_0 \in X\), the function \(f\) is differentiable at \(x_0\) on \(X\).
\end{definition}

\begin{corollary}\label{10.1.12}
    Let \(X\) be a subset of \(\mathbf{R}\), and let \(f : X \to \mathbf{R}\) be a function which is differentiable on \(X\).
    Then \(f\) is also continuous on \(X\).
\end{corollary}

\begin{proof}
    By Lemma \ref{9.1.11} we know that \(\forall\ x_0 \in X\), \(x_0\) is an adherent point.
    By Exercise \ref{ex 9.1.9} we know that \(x_0\) is either a limit point or an isolated point.
    By Proposition \ref{10.1.10} and Definition \ref{10.1.11} we know that if \(x_0\) is a limit point then \(f\) is continuous at \(x_0\).
    So we only need to show that if \(x_0\) is an isolated point, then \(f\) is also continuous at \(x_0\).
    Suppose that \(x_0\) is an isolated point of \(X\).
    By Definition \ref{9.1.18} we know that \(\exists\ \varepsilon' \in \mathbf{R}^+\) such that \(\abs*{x - x_0} > \varepsilon'\) for all \(x \in X \setminus \{x_0\}\).
    To show that \(f\) is continuous at \(x_0\), by Definition \ref{9.4.1} and Definition \ref{9.3.6} we need to show that
    \[
        \forall\ \varepsilon \in \mathbf{R}^+, \exists\ \delta \in \mathbf{R}^+ : \bigg(\forall\ x \in X, \abs*{x - x_0} < \delta \implies \abs*{f(x) - f(x_0)} \leq \varepsilon\bigg).
    \]
    Let \(\delta = \varepsilon'\).
    Since \(x_0\) is an isolated point, the only \(x \in X\) satisfying \(\abs*{x - x_0} < \delta\) is \(x_0\).
    Thus we have \(0 = \abs*{f(x_0) - f(x_0)} \leq \varepsilon\) and \(\lim_{x \to x_0 ; x \in X} f(x) = f(x_0)\).
\end{proof}

\begin{theorem}[Differential calculus]\label{10.1.13}
    Let \(X\) be a subset of \(\mathbf{R}\), let \(x_0 \in X\) be a limit point of \(X\), and let \(f : X \to \mathbf{R}\) and \(g : X \to \mathbf{R}\) be functions.
    \begin{enumerate}
        \item If \(f\) is a constant function, i.e., there exists a real number \(c\) such that \(f(x) = c\) for all \(x \in X\), then \(f\) is differentiable at \(x_0\) and \(f'(x_0) = 0\).
        \item If \(f\) is the identity function, i.e., \(f(x) = x\) for all \(x \in X\), then \(f\) is differentiable at \(x_0\) and \(f'(x_0) = 1\).
        \item (Sum rule)
              If \(f\) and \(g\) are differentiable at \(x_0\), then \(f + g\) is also differentiable at \(x_0\), and \((f + g)'(x_0) = f'(x_0) + g'(x_0)\).
        \item (Product rule)
              If \(f\) and \(g\) are differentiable at \(x_0\), then \(fg\) is also differentiable at \(x_0\), and \((fg)'(x_0) = f'(x_0)g(x_0) + f(x_0)g'(x_0)\).
        \item If \(f\) is differentiable at \(x_0\) and \(c\) is a real number, then \(cf\) is also differentiable at \(x_0\), and \((cf)'(x_0) = cf'(x_0)\).
        \item (Difference rule)
              If \(f\) and \(g\) are differentiable at \(x_0\), then \(f - g\) is also differentiable at \(x_0\), and \((f - g)'(x_0) = f'(x_0) - g'(x_0)\).
        \item If \(g\) is differentiable at \(x_0\), and \(g\) is non-zero on \(X\) (i.e., \(g(x) \neq 0\) for all \(x \in X\)), then \(1 / g\) is also differentiable at \(x_0\), and \((\frac{1}{g})'(x_0) = -\frac{g'(x_0)}{g(x_0)^2}\).
        \item (Quotient rule)
              If \(f\) and \(g\) are differentiable at \(x_0\), and \(g\) is non-zero on \(X\), then \(f / g\) is also differentiable at \(x_0\), and
              \[
                  (\frac{f}{g})'(x_0) = \frac{f'(x_0) g(x_0) - f(x_0) g'(x_0)}{g(x_0)^2}.
              \]
    \end{enumerate}
\end{theorem}

\begin{proof}{(a)}
    We have \(\forall\ \varepsilon \in \mathbf{R}^+\), \(\forall\ \delta \in \mathbf{R}^+\) such that
    \begin{align*}
                 & \forall\ x \in X \setminus \{x_0\}, \abs*{x - x_0} < \delta                                       \\
        \implies & \abs*{\frac{f(x) - f(x_0)}{x - x_0} - 0} = \abs*{\frac{c - c}{x - x_0} - 0} = 0 \leq \varepsilon.
    \end{align*}
    Thus by Definition \ref{9.3.6} we have
    \[
        \lim_{x \to x_0 ; x \in X \setminus \{x_0\}} \frac{f(x) - f(x_0)}{x - x_0} = 0
    \]
    and by Definition \ref{10.1.1} we have \(f'(x_0) = 0\).
\end{proof}

\begin{proof}{(b)}
    We have \(\forall\ \varepsilon \in \mathbf{R}^+\), \(\forall\ \delta \in \mathbf{R}^+\) such that
    \begin{align*}
                 & \forall\ x \in X \setminus \{x_0\}, \abs*{x - x_0} < \delta                                         \\
        \implies & \abs*{\frac{f(x) - f(x_0)}{x - x_0} - 1} = \abs*{\frac{x - x_0}{x - x_0} - 1} = 0 \leq \varepsilon.
    \end{align*}
    Thus by Definition \ref{9.3.6} we have
    \[
        \lim_{x \to x_0 ; x \in X \setminus \{x_0\}} \frac{f(x) - f(x_0)}{x - x_0} = 1
    \]
    and by Definition \ref{10.1.1} we have \(f'(x_0) = 1\).
\end{proof}

\begin{proof}{(c)}
    By Definition \ref{10.1.1} and Proposition \ref{9.3.14} we have
    \begin{align*}
          & f'(x_0) + g'(x_0)                                                                                                                                       \\
        = & \lim_{x \to x_0 ; x \in X \setminus \{x_0\}} \frac{f(x) - f(x_0)}{x - x_0} + \lim_{x \to x_0 ; x \in X \setminus \{x_0\}} \frac{g(x) - g(x_0)}{x - x_0} \\
        = & \lim_{x \to x_0 ; x \in X \setminus \{x_0\}} \frac{f(x) - f(x_0)}{x - x_0} + \frac{g(x) - g(x_0)}{x - x_0}                                              \\
        = & \lim_{x \to x_0 ; x \in X \setminus \{x_0\}} \frac{f(x) - f(x_0) + g(x) - g(x_0)}{x - x_0}                                                              \\
        = & \lim_{x \to x_0 ; x \in X \setminus \{x_0\}} \frac{f(x) + g(x) - (f(x_0) + g(x_0))}{x - x_0}                                                            \\
        = & \lim_{x \to x_0 ; x \in X \setminus \{x_0\}} \frac{(f + g)(x) - (f + g)(x_0)}{x - x_0}                                                                  \\
        = & (f + g)'(x_0).
    \end{align*}
    Thus \(f + g\) is differentiable at \(x_0\) and \((f + g)'(x_0) = f'(x_0) + g'(x_0)\).
\end{proof}

\begin{proof}{(d)}
    By Definition \ref{10.1.1} and Proposition \ref{9.3.14} we have
    \begin{align*}
          & f'(x_0) g(x_0) + f(x_0) g'(x_0)                                                                                                                                                                         \\
        = & \bigg(\lim_{x \to x_0 ; x \in X \setminus \{x_0\}} \frac{f(x) - f(x_0)}{x - x_0}\bigg) \bigg(\lim_{x \to x_0 ; x \in X \setminus \{x_0\}} g(x)\bigg)                                                    \\
          & + f(x_0) \bigg(\lim_{x \to x_0 ; x \in X \setminus \{x_0\}} \frac{g(x) - g(x_0)}{x - x_0}\bigg)                                                                                                         \\
        = & \bigg(\lim_{x \to x_0 ; x \in X \setminus \{x_0\}} \frac{(f(x) - f(x_0)) g(x)}{x - x_0}\bigg) + \bigg(\lim_{x \to x_0 ; x \in X \setminus \{x_0\}} \frac{f(x_0) (g(x) - g(x_0))}{x - x_0}\bigg)         \\
        = & \bigg(\lim_{x \to x_0 ; x \in X \setminus \{x_0\}} \frac{f(x) g(x) - f(x_0) g(x)}{x - x_0}\bigg) + \bigg(\lim_{x \to x_0 ; x \in X \setminus \{x_0\}} \frac{f(x_0) g(x) - f(x_0) g(x_0)}{x - x_0}\bigg) \\
        = & \lim_{x \to x_0 ; x \in X \setminus \{x_0\}} \bigg(\frac{f(x) g(x) - f(x_0) g(x)}{x - x_0} + \frac{f(x_0) g(x) - f(x_0) g(x_0)}{x - x_0}\bigg)                                                          \\
        = & \lim_{x \to x_0 ; x \in X \setminus \{x_0\}} \frac{f(x) g(x) - f(x_0) g(x) + f(x_0) g(x) - f(x_0) g(x_0)}{x - x_0}                                                                                      \\
        = & \lim_{x \to x_0 ; x \in X \setminus \{x_0\}} \frac{f(x) g(x) - f(x_0) g(x_0)}{x - x_0}                                                                                                                  \\
        = & \lim_{x \to x_0 ; x \in X \setminus \{x_0\}} \frac{(fg)(x) - (fg)(x_0)}{x - x_0}                                                                                                                        \\
        = & (fg)'(x_0).
    \end{align*}
    Thus \(fg\) is differentiable at \(x_0\) and \((fg)'(x_0) = f'(x_0) g(x_0) + f(x_0) g'(x_0)\).
\end{proof}

\begin{proof}{(e)}
    By Definition \ref{10.1.1} and Proposition \ref{9.3.14} we have
    \begin{align*}
          & cf'(x_0)                                                                                 \\
        = & c \bigg(\lim_{x \to x_0 ; x \in X \setminus \{x_0\}} \frac{f(x) - f(x_0)}{x - x_0}\bigg) \\
        = & \lim_{x \to x_0 ; x \in X \setminus \{x_0\}} \bigg(c \frac{f(x) - f(x_0)}{x - x_0}\bigg) \\
        = & \lim_{x \to x_0 ; x \in X \setminus \{x_0\}} \frac{cf(x) - cf(x_0)}{x - x_0}             \\
        = & \lim_{x \to x_0 ; x \in X \setminus \{x_0\}} \frac{(cf)(x) - (cf)(x_0)}{x - x_0}         \\
        = & (cf)'(x_0).
    \end{align*}
    Thus \(cf\) is differentiable at \(x_0\) and \((cf)'(x_0) = cf'(x_0)\).
\end{proof}

\begin{proof}{(f)}
    \begin{align*}
          & f'(x_0) - g'(x_0)                                             \\
        = & f'(x_0) + (-g'(x_0))                                          \\
        = & f'(x_0) + ((-g)'(x_0)) & \text{(by Theorem \ref{10.1.13}(e))} \\
        = & (f + (-g))'(x_0)       & \text{(by Theorem \ref{10.1.13}(c))} \\
        = & (f - g)'(x_0).         & \text{(by Definition \ref{9.2.1})}
    \end{align*}
    Thus \(f - g\) is differentiable at \(x_0\) and \((f - g)'(x_0) = f'(x_0) - g'(x_0)\).
\end{proof}

\begin{proof}{(g)}
    By Definition \ref{10.1.1} and Proposition \ref{9.3.14} we have
    \begin{align*}
          & -\frac{g'(x_0)}{g(x_0)^2}                                                                                                                                                     \\
        = & \bigg(\lim_{x \to x_0 ; x \in X \setminus \{x_0\}} \frac{g(x) - g(x_0)}{x - x_0}\bigg) \bigg(\frac{-1}{g(x_0)^2}\bigg)                                                        \\
        = & \bigg(\lim_{x \to x_0 ; x \in X \setminus \{x_0\}} \frac{g(x) - g(x_0)}{x - x_0}\bigg) \bigg(\frac{-g(x_0)}{g(x_0) g(x_0)^2}\bigg)                                            \\
        = & \bigg(\lim_{x \to x_0 ; x \in X \setminus \{x_0\}} \frac{g(x) - g(x_0)}{x - x_0}\bigg) \bigg(\lim_{x \to x_0 ; x \in X \setminus \{x_0\}} \frac{-g(x_0)}{g(x) g(x_0)^2}\bigg) \\
        = & \lim_{x \to x_0 ; x \in X \setminus \{x_0\}} \Bigg(\bigg(\frac{g(x) - g(x_0)}{x - x_0}\bigg) \bigg(\frac{-g(x_0)}{g(x) g(x_0)^2}\bigg)\Bigg)                                  \\
        = & \lim_{x \to x_0 ; x \in X \setminus \{x_0\}} \frac{\frac{g(x_0)(g(x_0) - g(x))}{g(x) g(x_0)^2}}{x - x_0}                                                                      \\
        = & \lim_{x \to x_0 ; x \in X \setminus \{x_0\}} \frac{\frac{g(x_0) - g(x)}{g(x) g(x_0)}}{x - x_0}                                                                                \\
        = & \lim_{x \to x_0 ; x \in X \setminus \{x_0\}} \frac{\frac{1}{g(x)} - \frac{1}{g(x_0)}}{x - x_0}                                                                                \\
        = & (\frac{1}{g})'(x_0).
    \end{align*}
    Thus \(1 / g\) is differentiable at \(x_0\) and \((1 / g)'(x_0) = -\frac{g'(x_0)}{g(x_0)^2}\).
\end{proof}

\begin{proof}{(h)}
    \begin{align*}
          & (\frac{f}{g})'(x_0)                                                                                      \\
        = & (f \cdot \frac{1}{g})'(x_0)                                       & \text{(by Definition \ref{9.2.1})}   \\
        = & f'(x_0) \frac{1}{g}(x_0) + f(x_0) (\frac{1}{g})'(x_0)             & \text{(by Theorem \ref{10.1.13}(d))} \\
        = & \frac{f'(x_0)}{g(x_0)} + f(x_0) (\frac{1}{g})'(x_0)               & \text{(by Definition \ref{9.2.1})}   \\
        = & \frac{f'(x_0)}{g(x_0)} + f(x_0) \frac{-g'(x_0)}{g(x_0)^2}         & \text{(by Theorem \ref{10.1.13}(g))} \\
        = & \frac{f'(x_0) g(x_0)}{g(x_0)^2} - \frac{f(x_0) g'(x_0)}{g(x_0)^2}                                        \\
        = & \frac{f'(x_0) g(x_0) - f(x_0) g'(x_0)}{g(x_0)^2}.                                                        \\
    \end{align*}
    Thus \(f / g\) is differentiable at \(x_0\) and \((f / g)'(x_0) = \frac{f'(x_0) g(x_0) - f(x_0) g'(x_0)}{g(x_0)^2}\).
\end{proof}

\begin{remark}\label{10.1.14}
    The product rule is also known as the \emph{Leibniz rule}, after Gottfried Leibniz (1646 -- 1716), who was the other founder of differential and integral calculus besides Newton.
\end{remark}

\begin{note}
    The trick of adding and subtracting an intermediate term is sometimes known as the ``middle-man trick'' and is very useful in analysis.
\end{note}

\begin{theorem}[Chain rule]\label{10.1.15}
    Let \(X, Y\) be subsets of \(\mathbf{R}\), let \(x_0 \in X\) be a limit point of \(X\), and let \(y_0 \in Y\) be a limit point of \(Y\).
    Let \(f : X \to Y\) be a function such that \(f(x_0) = y_0\), and such that \(f\) is differentiable at \(x_0\).
    Suppose that \(g : Y \to \mathbf{R}\) is a function which is differentiable at \(y_0\).
    Then the function \(g \circ f : X \to \mathbf{R}\) is differentiable at \(x_0\), and
    \[
        (g \circ f)'(x_0) = g'(y_0) f'(x_0)
    \]
\end{theorem}

\begin{proof}
    Since \(f\) is differentiable at \(x_0\), by Proposition \ref{10.1.10} we know that \(f\) is continuous at \(x_0\), so we have \(\forall\ \varepsilon_1 \in \mathbf{R}^+\), \(\exists\ \delta_1 \in \mathbf{R}^+\) such that
    \[
        \forall\ x \in X, \abs*{x - x_0} \leq \delta_1 \implies \abs*{f(x) - f(x_0)} \leq \varepsilon_1.
    \]
    Since \(f\) is differentiable at \(x_0\) and \(g\) is differentiable at \(y_0\), by Newton's approximation (Proposition \ref{10.1.7}), we have \(\forall\ \varepsilon_2, \varepsilon_3 \in \mathbf{R}^+\), \(\exists\ \delta_2, \delta_3 \in \mathbf{R}^+\) such that
    \[
        \forall\ x \in X, \abs*{x - x_0} \leq \delta_2 \implies \abs*{f(x) - (f(x_0) + f'(x_0)(x - x_0))} \leq \varepsilon_2 \abs*{x - x_0}
    \]
    and
    \[
        \forall\ y \in Y, \abs*{y - y_0} \leq \delta_3 \implies \abs*{g(y) - (g(y_0) + g'(y_0)(y - y_0))} \leq \varepsilon_3 \abs*{y - y_0}.
    \]
    In particular, we have
    \begin{align*}
                 & \forall\ x \in X, \abs*{x - x_0} \leq \delta_2                                 \\
        \implies & \abs*{f(x) - (f(x_0) + f'(x_0)(x - x_0))} \leq \varepsilon_2 \abs*{x - x_0}    \\
        \implies & \abs*{f(x) - f(x_0)}                                                           \\
                 & \leq \abs*{f(x) - (f(x_0) + f'(x_0)(x - x_0))} + \abs*{f'(x_0)(x - x_0)}       \\
                 & \leq \varepsilon_2 \abs*{x - x_0} + \abs*{f'(x_0)(x - x_0)}                    \\
        \implies & \abs*{f(x) - y_0} \leq \varepsilon_2 \abs*{x - x_0} + \abs*{f'(x_0)(x - x_0)}.
    \end{align*}
    Let \(\delta = \min(\delta_2, \delta_3)\) and \(\varepsilon_1 = \delta\).
    Then we have
    \begin{align*}
                 & \forall\ x \in X, \abs*{x - x_0} \leq \delta_1                                                                         \\
        \implies & \abs*{f(x) - f(x_0)} \leq \delta                                                                                       \\
        \implies & \abs*{f(x) - y_0} \leq \delta                                                                                          \\
        \implies & \abs*{g(f(x)) - \Big(g(y_0) + g'(y_0)\big(f(x) - y_0\big)\Big)} \leq \varepsilon_3 \abs*{f(x) - y_0}                   \\
        \implies & \abs*{g(f(x)) - \Big(g(y_0) + g'(y_0)\big(f(x) - y_0\big)\Big)}                                                        \\
                 & \leq \varepsilon_3 \abs*{f(x) - y_0}                                                                                   \\
                 & \leq \varepsilon_3 \big(\varepsilon_2 \abs*{x - x_0} + \abs*{f'(x_0)(x - x_0)}\big)                                    \\
                 & = \varepsilon_3 \varepsilon_2 \abs*{x - x_0} + \varepsilon_3 \abs*{f'(x_0)(x - x_0)}                                   \\
        \implies & \abs*{g(f(x)) - \Big(g(y_0) + g'(y_0)\big(f(x) - y_0\big)\Big)}                                                        \\
                 & + \abs*{g'(y_0) \Big(f(x) - \big(y_0 + f'(x_0)(x - x_0)\big)\Big)}                                                     \\
                 & \leq \varepsilon_3 \varepsilon_2 \abs*{x - x_0} + \varepsilon_3 \abs*{f'(x_0)(x - x_0)} + \varepsilon_2 \abs*{x - x_0} \\
        \implies & \abs*{g(f(x)) - \big(g(y_0) + g'(y_0) f'(x_0) (x - x_0)\big)}                                                          \\
                 & \leq \big(\varepsilon_3 \varepsilon_2 + \varepsilon_3 \abs*{f'(x_0)} + \varepsilon_2\big) \abs*{x - x_0}               \\
                 & = \Big(\varepsilon_3 \big(\varepsilon_2 + \abs*{f'(x_0)}\big) + \varepsilon_2\Big) \abs*{x - x_0}.
    \end{align*}
    Let \(\varepsilon \in \mathbf{R}^+\), let \(\varepsilon_2 = \frac{\varepsilon}{2}\) and let \(\varepsilon_3 = \frac{\varepsilon}{2(\varepsilon_2 + \abs*{f'(x_0)})}\).
    Then we have
    \begin{align*}
                 & \forall\ x \in X, \abs*{x - x_0} \leq \delta_1                                                                                              \\
        \implies & \abs*{g(f(x)) - \big(g(y_0) + g'(y_0) f'(x_0) (x - x_0)\big)}                                                                               \\
                 & \leq \Big(\varepsilon_3 \big(\varepsilon_2 + \abs*{f'(x_0)}\big) + \varepsilon_2\Big) \abs*{x - x_0}                                        \\
                 & = \bigg(\frac{\varepsilon \big(\varepsilon_2 + \abs*{f'(x_0)}\big)}{2(\varepsilon_2 + \abs*{f'(x_0)})} + \varepsilon_2\bigg) \abs*{x - x_0} \\
                 & = (\frac{\varepsilon}{2} + \varepsilon_2) \abs*{x - x_0}                                                                                    \\
                 & = (\frac{\varepsilon}{2} + \frac{\varepsilon}{2}) \abs*{x - x_0}                                                                            \\
                 & = \varepsilon \abs*{x - x_0}                                                                                                                \\
        \implies & \abs*{(g \circ f)(x) - \big((g \circ f)(x_0) + g'(y_0) f'(x_0) (x - x_0)\big)}                                                              \\
                 & \leq \varepsilon \abs*{x - x_0}
    \end{align*}
    and thus by Newton's approximation (Proposition \ref{10.1.7}) we have \((g \circ f)'(x_0) = g'(y_0) f'(x_0)\).
\end{proof}

\setcounter{theorem}{16}
\begin{remark}\label{10.1.17}
    If one writes \(y\) for \(f(x)\), and \(z\) for \(g(y)\), then the chain rule can be written in the more visually appealing manner \(\frac{dz}{dx} = \frac{dz}{dy} \frac{dy}{dx}\).
    However, this notation can be misleading (for instance it blurs the distinction between dependent variable and independent variable, especially for \(y\)), and leads one to believe that the quantities \(dz, dy, dx\) can be manipulated like real numbers.
    However, these quantities are not real numbers (in fact, we have not assigned any meaning to them at all), and treating them as such can lead to problems in the future.
    For instance, if \(f\) depends on \(x_1\) and \(x_2\), which depend on \(t\), then chain rule for several variables asserts that \(\frac{df}{dt} = \frac{\partial f}{\partial x_1} \frac{dx_1}{dt} + \frac{\partial f}{\partial x_2} \frac{dx_2}{dt}\), but this rule might seem suspect if one treated \(df, dt\), etc. as real numbers.
    It is possible to think of \(dy, dx\), etc. as ``infinitesimal real numbers'' if one knows what one is doing, but for those just starting out in analysis, I would not recommend this approach, especially if one wishes to work rigorously.
    (There is a way to make all of this rigorous, even for the calculus of several variables, but it requires the notion of a tangent vector, and the derivative map, both of which are beyond the scope of this text.)
\end{remark}

\exercisesection

\begin{exercise}\label{ex 10.1.1}
    Suppose that \(X\) is a subset of \(\mathbf{R}\), \(x_0\) is a limit point of \(X\), and \(f : X \to \mathbf{R}\) is a function which is differentiable at \(x_0\).
    Let \(Y \subseteq X\) be such that \(x_0 \in Y\), and \(x_0\) is also a limit point of \(Y\).
    Prove that the restricted function \(f|_Y : Y \to \mathbf{R}\) is also differentiable at \(x_0\), and has the same derivative as \(f\) at \(x_0\).
    Explain why this does not contradict the discussion in Remark \ref{10.1.2}.
\end{exercise}

\begin{proof}
    Since \(f\) is differentiable at \(x_0\), by Newton's approximation (Proposition \ref{10.1.7}) we have \(\forall\ \varepsilon \in \mathbf{R}^+\), \(\exists\ \delta \in \mathbf{R}^+\) such that
    \[
        \forall\ x \in X, \abs*{x - x_0} \leq \delta \implies \abs*{f(x) - (f(x_0) + f'(x_0)(x - x_0))} \leq \varepsilon \abs*{x - x_0}.
    \]
    Since \(Y \subseteq X\), we have
    \begin{align*}
                 & \forall\ x \in Y \land \abs*{x - x_0} \leq \delta                                \\
        \implies & x \in X \land \abs*{x - x_0} \leq \delta                                         \\
        \implies & \abs*{f(x) - (f(x_0) + f'(x_0)(x - x_0))} \leq \varepsilon \abs*{x - x_0}        \\
        \implies & \abs*{f|_Y(x) - (f|_Y(x_0) + f'(x_0)(x - x_0))} \leq \varepsilon \abs*{x - x_0}.
    \end{align*}
    Thus by Newton's approximation (Proposition \ref{10.1.7}) we know that \(f|_Y'(x_0) = f'(x_0)\).
    This does not contradict to Remark \ref{10.1.2} since \(x_0\) is a limit point of \(Y\) implies \(x_0\) is also a limit point of \(X\).
\end{proof}

\begin{exercise}\label{ex 10.1.2}
    Prove Proposition \ref{10.1.7}.
\end{exercise}

\begin{proof}
    See Proposition \ref{10.1.7}.
\end{proof}

\begin{exercise}\label{ex 10.1.3}
    Prove Proposition \ref{10.1.10}.
\end{exercise}

\begin{proof}
    See Proposition \ref{10.1.10}.
\end{proof}

\begin{exercise}\label{ex 10.1.4}
    Prove Theorem \ref{10.1.13}.
\end{exercise}

\begin{proof}
    See Theorem \ref{10.1.13}.
\end{proof}

\begin{exercise}\label{ex 10.1.5}
    Let \(n\) be a natural number, and let \(f : \mathbf{R} \to \mathbf{R}\) be the function \(f(x) \coloneqq x^n\).
    Show that \(f\) is differentiable on \(\mathbf{R}\) and \(f'(x) = n x^{n - 1}\) for all \(x \in \mathbf{R}\) with the convention that \(n x^{n - 1} = 0\) when \(n = 0\).
\end{exercise}

\begin{proof}
    We use induction on \(n\) to show that \(\forall\ n \in \mathbf{N}\), \(f_n(x) = x^n\) is differentiable on \(\mathbf{R}\) and \(f_n'(x) = n x^{n - 1}\).
    For \(n = 0\), we have \(f_0(x) = x^0 = 1\), and by Theorem \ref{10.1.13}(a) we know that \(f_0\) is differentiable on \(\mathbf{R}\) and \(\forall\ x \in X\), \(f_0'(x) = 0\).
    Thus the base case holds.
    Suppose inductively that for some \(n \geq 0\) we have \(f_n(x) = x^n\) is differentiable on \(\mathbf{R}\) and \(f_n'(x) = n x^{n - 1}\).
    Then for \(n + 1\), we have \(f_{n + 1}(x) = x^{n + 1} = x^n \cdot x^1 = f_n(x) f_1(x) = (f_n \cdot f_1)(x)\) and
    \begin{align*}
          & (f_n \cdot f_1)'(x)                                                        \\
        = & f_n'(x) f_1(x) + f_n(x) f_1'(x)     & \text{(by Theorem \ref{10.1.13}(d))} \\
        = & (n x^{n - 1})(x^1) + f_n(x) f_1'(x) & \text{(by induction hypothesis)}     \\
        = & (n x^{n - 1})(x^1) + (x^n)(1 x^0)   & \text{(by Theorem \ref{10.1.13}(b))} \\
        = & n x^n + x^n                                                                \\
        = & (n + 1) x^n.
    \end{align*}
    This close the induction.
\end{proof}

\begin{exercise}\label{ex 10.1.6}
    Let \(n\) be a \emph{negative} integer, and let \(f : \mathbf{R} \setminus \{0\} \to \mathbf{R}\) be the function \(f(x) \coloneqq x^n\).
    Show that \(f\) is differentiable on \(\mathbf{R} \setminus \{0\}\) and \(f'(x) = n x^{n - 1}\) for all \(x \in \mathbf{R} \setminus \{0\}\).
\end{exercise}

\begin{proof}
    Let \(x \in \mathbf{R} \setminus \{0\}\).
    Since \(n \in \mathbf{Z}^-\), \(-n \in \mathbf{Z}^+\).
    Then we have \((1 / f)(x) = 1 / x^n = x^{-n}\) and \(f(x) = (1 / (1 / f))(x)\).
    Thus \(f\) is differentiable at \(x\) and
    \begin{align*}
        f'(x) & = (1 / (1 / f))'(x)                                                        \\
              & = -\frac{(1 / f)'(x)}{(1 / f)(x)^2} & \text{(by Theorem \ref{10.1.13}(g))} \\
              & = -\frac{(-n) x^{-n - 1}}{x^{-2n}}  & \text{(by Exercise \ref{ex 10.1.5})} \\
              & = n x^{n - 1}.
    \end{align*}
\end{proof}

\begin{exercise}\label{ex 10.1.7}
    Prove Theorem \ref{10.1.15}.
\end{exercise}

\begin{proof}
    See Theorem \ref{10.1.15}.
\end{proof}
\section{Local maxima, local minima, and derivatives}\label{sec 10.2}

\begin{definition}[Local maxima and minima]\label{10.2.1}
    Let \(X\) be a subset of \(\mathbf{R}\), and let \(f : X \to \mathbf{R}\) be a function, and let \(x_0 \in X\).
    We say that \(f\) attains a \emph{local maximum} at \(x_0\) iff there exists a \(\delta > 0\) such that the restriction \(f|_{X \cap (x_0 - \delta, x_0 + \delta)}\) of \(f\) to \(X \cap (x_0 - \delta, x_0 + \delta)\) attains a maximum at \(x_0\).
    We say that \(f\) attains a \emph{local minimum} at \(x_0\) iff there exists a \(\delta > 0\) such that the restriction \(f|_{X \cap (x_0 - \delta, x_0 + \delta)}\) of \(f\) to \(X \cap (x_0 - \delta, x_0 + \delta)\) attains a minimum at \(x_0\).
\end{definition}

\begin{remark}\label{10.2.2}
    If \(f\) attains a maximum at \(x_0\), we sometimes say that \(f\) attains a \emph{global} maximum at \(x_0\), in order to distinguish it from the local maxima defined in Definition \ref{10.2.1}.
    Note that if \(f\) attains a global maximum at \(x_0\), then it certainly also attains a local maximum at this \(x_0\), and similarly for minima.
\end{remark}

\setcounter{theorem}{4}
\begin{remark}\label{10.2.5}
    If \(f : X \to \mathbf{R}\) attains a local maximum at a point \(x_0\) in \(X\), and \(Y \subseteq X\) is a subset of \(X\) which contains \(x_0\), then the restriction \(f|_Y : Y \to \mathbf{R}\) also attains a local maximum at \(x_0\).
    Similarly for minima.
\end{remark}

\begin{proposition}[Local extrema are stationary]\label{10.2.6}
    Let \(a < b\) be real numbers, and let \(f : (a, b) \to \mathbf{R}\) be a function.
    If \(x_0 \in (a, b)\), \(f\) is differentiable at \(x_0\), and \(f\) attains either a local maximum or local minimum at \(x_0\), then \(f'(x_0) = 0\).
\end{proposition}

\begin{proof}
    Suppose \(f\) attains local maximum at \(x_0\).
    Then by Definition \ref{10.2.1} we know that
    \[
        \exists\ \delta \in \mathbf{R}^+ : \forall\ x \in (a, b) \cap (x_0 - \delta, x_0 + \delta), f(x) \leq f(x_0).
    \]
    Since \(f\) is differentiable at \(x_0\), by Definition \ref{10.1.1} we know that
    \[
        \lim_{x \to x_0 ; x \in (a, b) \setminus \{x_0\}} \frac{f(x) - f(x_0)}{x - x_0} = f'(x_0)
    \]
    By Definition \ref{9.3.6} we know that \(\forall\ \varepsilon \in \mathbf{R}^+\), \(\exists\ \delta' \in \mathbf{R}^+\) such that
    \[
        \forall\ x \in (a, b) \setminus \{x_0\}, \abs*{x - x_0} < \delta' \implies \abs*{\frac{f(x) - f(x_0)}{x - x_0} - f'(x_0)} \leq \varepsilon.
    \]
    In particular, we have
    \[
        \forall\ x \in (a, b) \cap (x_0, x_0 + \delta), \abs*{x - x_0} < \delta' \implies \abs*{\frac{f(x) - f(x_0)}{x - x_0} - f'(x_0)} \leq \varepsilon
    \]
    and
    \[
        \forall\ x \in (a, b) \cap (x_0 - \delta, x_0), \abs*{x - x_0} < \delta' \implies \abs*{\frac{f(x) - f(x_0)}{x - x_0} - f'(x_0)} \leq \varepsilon.
    \]
    Thus by Definition \ref{9.3.6} we must have
    \[
        \lim_{x \to x_0 ; x \in (a, b) \cap (x_0, x_0 + \delta)} \frac{f(x) - f(x_0)}{x - x_0} = \lim_{x \to x_0 ; x \in (a, b) \cap (x_0 - \delta, x_0)} \frac{f(x) - f(x_0)}{x - x_0} = f'(x_0).
    \]
    Since
    \begin{align*}
                 & \forall\ x \in (a, b) \cap (x_0, x_0 + \delta)                                                                                       \\
        \implies & f(x) \leq f(x_0)                                                                                                                     \\
        \implies & f(x) - f(x_0) \leq 0                                                                                                                 \\
        \implies & \frac{f(x) - f(x_0)}{x - x_0} \leq 0                                                                                                 \\
        \implies & \lim_{x \to x_0 ; x \in (a, b) \cap (x_0, x_0 + \delta)} \frac{f(x) - f(x_0)}{x - x_0} \leq 0 & \text{(by Proposition \ref{9.3.14})} \\
        \implies & f'(x_0) \leq 0
    \end{align*}
    and
    \begin{align*}
                 & \forall\ x \in (a, b) \cap (x_0 - \delta, x_0)                                                                                        \\
        \implies & f(x) \leq f(x_0)                                                                                                                      \\
        \implies & f(x) - f(x_0) \leq 0                                                                                                                  \\
        \implies & \frac{f(x) - f(x_0)}{x - x_0} \geq 0                                                                                                  \\
        \implies & \lim_{x \to x_0 ; x \in (a, b) \cap (x_0 - \delta, x_0)} \frac{f(x) - f(x_0)}{x - x_0} \geq 0, & \text{(by Proposition \ref{9.3.14})} \\
        \implies & f'(x_0) \geq 0,
    \end{align*}
    we must have \(f'(x_0) = 0\).
    Similar arguments work for the case \(f\) attains local minimum at \(x_0\).
\end{proof}

\begin{note}
    Note that \(f\) must be differentiable for this proposition to work.
    Also, this proposition does not work if the open interval \((a, b)\) is replaced by a closed interval \([a, b]\).
    For instance, the function \(f : [1, 2] \to \mathbf{R}\) defined by \(f(x) \coloneqq x\) has a local maximum at \(x_0 = 2\) and a local minimum \(x_0 = 1\) (in fact, these local extrema are global extrema), but at both points the derivative is \(f'(x_0) = 1\), not \(f'(x_0) = 0\).
    Thus the endpoints of an interval can be local maxima or minima even if the derivative is not zero there.
    Finally, the converse of this proposition is false.
\end{note}

\begin{theorem}[Rolle's theorem]\label{10.2.7}
    Let \(a < b\) be real numbers, and let \(g : [a, b] \to \mathbf{R}\) be a continuous function which is differentiable on \((a, b)\).
    Suppose also that \(g(a) = g(b)\).
    Then there exists an \(x \in (a, b)\) such that \(g'(x) = 0\).
\end{theorem}

\begin{proof}
    Since \(g\) is continuous on \([a, b]\), by Proposition \ref{9.6.7} \(g\) attains its maximum at some point \(x_{\max} \in [a, b]\), and also attains its minimum at some point \(x_{\min} \in [a, b]\).
    If \(x_{\min} \in \{a, b\} \land x_{\max} \in \{a, b\}\), then by Definition \ref{9.6.5} we have \(\forall\ x \in [a, b]\), \(g(x) = g(a) = g(b)\), and by Theorem \ref{10.1.13}(a) we know that \(g'(x) = 0\).
    So suppose that at least one of \(x_{\min}, x_{\max} \notin \{a, b\}\), i.e., \(x_{\min} \in (a, b) \lor x_{\max} \in (a, b)\).
    If \(x_{\min} \in (a, b)\), then by Proposition \ref{10.2.6} we know that \(f'(x_{\min}) = 0\).
    Similarly, if \(x_{\max} \in (a, b)\), then by Proposition \ref{10.2.6} we know that \(f'(x_{\max}) = 0\).
    Thus there exists an \(x \in (a, b)\) such that \(g'(x) = 0\).
\end{proof}

\begin{remark}\label{10.2.8}
    Note that we only assume \(f\) is differentiable on the open interval \((a, b)\), though of course the theorem also holds if we assume \(f\) is differentiable on the closed interval \([a, b]\), since this is larger than \((a, b)\).
\end{remark}

\begin{corollary}[Mean value theorem]\label{10.2.9}
    Let \(a < b\) be real numbers, and let \(f : [a, b] \to \mathbf{R}\) be a function which is continuous on \([a, b]\) and differentiable on \((a, b)\).
    Then there exists an \(x \in (a, b)\) such that \(f'(x) = \frac{f(b) - f(a)}{b - a}\).
\end{corollary}

\begin{proof}
    Let \(g : [a, b] \to \mathbf{R}\) be a function where \(g(x) = f(x) - \frac{f(a) - f(b)}{a - b} x\).
    Since \(a < b\), we know that \(g\) is well-defined.
    Since \(f\) is differentiable on \((a, b)\), we know that
    \begin{align*}
                 & x \text{ is differentiable on } (a, b)                                  & \text{(by Theorem \ref{10.1.13}(b))} \\
        \implies & \frac{f(a) - f(b)}{a - b} x \text{ is differentiable on } (a, b)        & \text{(by Theorem \ref{10.1.13}(e))} \\
        \implies & f(x) - \frac{f(a) - f(b)}{a - b} x \text{ is differentiable on } (a, b) & \text{(by Theorem \ref{10.1.13}(f))} \\
        \implies & g(x) \text{ is differentiable on } (a, b)                                                                      \\
        \implies & g'(x) = f'(x) - \frac{f(a) - f(b)}{a - b}.
    \end{align*}
    Since
    \[
        g(a) = f(a) - \frac{f(a) - f(b)}{a - b} a = \frac{af(a) - bf(a) - af(a) + af(b)}{a - b} = \frac{af(b) - bf(a)}{a - b}
    \]
    and
    \[
        g(b) = f(b) - \frac{f(a) - f(b)}{a - b} b = \frac{af(b) - bf(b) - bf(a) + bf(b)}{a - b} = \frac{af(b) - bf(a)}{a - b},
    \]
    we have \(g(a) = g(b)\) and by Theorem \ref{10.2.7} \(\exists\ x_0 \in (a, b)\) such that \(g'(x_0) = 0\).
    Thus
    \begin{align*}
                 & g'(x_0) = 0                             \\
        \implies & f'(x_0) - \frac{f(a) - f(b)}{a - b} = 0 \\
        \implies & f'(x_0) = \frac{f(a) - f(b)}{a - b}.
    \end{align*}
\end{proof}

\exercisesection

\begin{exercise}\label{ex 10.2.1}
    Prove Proposition \ref{10.2.6}.
\end{exercise}

\begin{proof}
    See Proposition \ref{10.2.6}.
\end{proof}

\begin{exercise}\label{ex 10.2.2}
    Give an example of a function \(f : (-1, 1) \to \mathbf{R}\) which is continuous and attains a global maximum at \(0\), but which is not differentiable at \(0\).
    Explain why this does not contradict Proposition \ref{10.2.6}.
\end{exercise}

\begin{proof}
    Let \(f(x) = -\abs*{x}\).
    Then \(f\) is continuous and attains a global maximum at \(0\), but which is not differentiable at \(0\).
    The fact that \(f\) is not differentiable at \(0\) does not contradict to Proposition \ref{10.2.6}.
\end{proof}

\begin{exercise}\label{ex 10.2.3}
    Give an example of a function \(f : (-1, 1) \to \mathbf{R}\) which is differentiable, and whose derivative equals \(0\) at \(0\), but such that \(0\) is neither a local minimum nor a local maximum.
    Explain why this does not contradict Proposition \ref{10.2.6}.
\end{exercise}

\begin{proof}
    Let \(f(x) = x^3\).
    Then by Exercise \ref{ex 10.1.5} we know that \(f'(x) = 3x^2\).
    Then we have \(f'(0) = 0\), \(f(0) = 0\), \(f(-1) = -1\) and \(f'(1) = 1\).
    Thus \(0\) is neither a local minimum nor a local maximum.
    This does not contradict to Proposition \ref{10.2.6} since \(0\) is not given to be a local minimum or local maximum.
\end{proof}

\begin{exercise}\label{ex 10.2.4}
    Prove Theorem \ref{10.2.7}.
\end{exercise}

\begin{proof}
    See Theorem \ref{10.2.7}.
\end{proof}

\begin{exercise}\label{ex 10.2.5}
    Use Theorem \ref{10.2.7} to prove Corollary \ref{10.2.9}.
\end{exercise}

\begin{proof}
    See Corollary \ref{10.2.9}.
\end{proof}

\begin{exercise}\label{ex 10.2.6}
    Let \(M > 0\), and let \(f : [a, b] \to \mathbf{R}\) be a function which is continuous on \([a, b]\) and differentiable on \((a, b)\), and such that \(\abs*{f'(x)} \leq M\) for all \(x \in (a, b)\) (i.e., the derivative of \(f\) is bounded).
    Show that for any \(x, y \in [a, b]\) we have the inequality \(\abs*{f(x) - f(y)} \leq M \abs*{x - y}\).
    Functions which obey the bound \(\abs*{f(x) - f(y)} \leq M \abs*{x - y}\) are known as \emph{Lipschitz continuous functions} with \emph{Lipschitz constant} \(M\);
    thus this exercise shows that functions with bounded derivative are Lipschitz continuous.
\end{exercise}

\begin{proof}
    Let \(x, y \in [a, b]\).
    If \(x = y\), then we have \(0 = \abs*{f(x) - f(y)} \leq M \abs*{x - y} = 0\).
    So suppose that \(x \neq y\).
    We have either \(x < y\) or \(x > y\).
    \begin{enumerate}
        \item If \(x < y\), then \([x, y] \subseteq [a, b]\) and \((x, y) \subseteq (a, b)\).
              By Exercise \ref{ex 9.4.6} we know that \(f|_{[x, y]}\) is continuous on \([x, y]\).
              By Exercise \ref{ex 10.1.1} we know that \(f|_{[x, y]}\) is differentiable on \((x, y)\).
              By mean value theorem (Corollary \ref{10.2.9}) we know that \(\exists\ c \in (x, y)\) such that
              \[
                  f'(c) = \frac{f(y) - f(x)}{y - x}.
              \]
              Since \(c \in (x, y)\), we have \(c \in (a, b)\) and by hypothesis
              \begin{align*}
                           & \abs*{f'(c)} = \abs*{\frac{f(y) - f(x)}{y - x}} \leq M \\
                  \implies & \abs*{f(y) - f(x)} \leq M \abs*{y - x}                 \\
                  \implies & \abs*{f(x) - f(y)} \leq M \abs*{x - y}.
              \end{align*}
        \item If \(x > y\), then \([y, x] \subseteq [a, b]\) and \((y, x) \subseteq (a, b)\).
              By Exercise \ref{ex 9.4.6} we know that \(f|_{[y, x]}\) is continuous on \([y, x]\).
              By Exercise \ref{ex 10.1.1} we know that \(f|_{[y, x]}\) is differentiable on \((y, x)\).
              By mean value theorem (Corollary \ref{10.2.9}) we know that \(\exists\ c \in (y, x)\) such that
              \[
                  f'(c) = \frac{f(x) - f(y)}{x - y}.
              \]
              Since \(c \in (y, x)\), we have \(c \in (a, b)\) and by hypothesis
              \begin{align*}
                           & \abs*{f'(c)} = \abs*{\frac{f(x) - f(y)}{x - y}} \leq M \\
                  \implies & \abs*{f(x) - f(y)} \leq M \abs*{x - y}.
              \end{align*}
    \end{enumerate}
    From all cases above we have \(\abs*{f(x) - f(y)} \leq M \abs*{x - y}\).
    Thus \(\forall\ x, y \in [a, b]\), we must have \(\abs*{f(x) - f(y)} \leq M \abs*{x - y}\).
\end{proof}

\begin{exercise}\label{ex 10.2.7}
    Let \(f : \mathbf{R} \to \mathbf{R}\) be a differentiable function such that \(f'\) is bounded.
    Show that \(f\) is uniformly continuous.
\end{exercise}

\begin{proof}
    Since \(f'\) is bounded, by Definition \ref{9.6.1} \(\exists\ M \in \mathbf{R}^+\) such that \(\forall\ x \in \mathbf{R}\), \(\abs*{f(x)} \leq M\).
    By Exercise \ref{ex 10.2.6} we know that \(\forall\ x, y \in \mathbf{R}\), \(\abs*{f(x) - f(y)} \leq M \abs*{x - y}\).
    Then we have \(\forall\ \varepsilon \in \mathbf{R}^+\), \(\exists\ \delta = \varepsilon / M\) such that
    \begin{align*}
                 & \forall\ x, y \in \mathbf{R}, \abs*{x - y} \leq \delta                                          \\
        \implies & \abs*{f(x) - f(y)} \leq M \abs*{x - y} \leq M \delta \leq M \frac{\varepsilon}{M} = \varepsilon
    \end{align*}
    and by Definition \ref{9.9.2} \(f\) is uniformly continuous.
\end{proof}
\section{Monotone functions and derivatives}\label{sec 10.3}

\begin{proposition}\label{10.3.1}
    Let \(X\) be a subset of \(\mathbf{R}\), let \(x_0 \in X\) be a limit point of \(X\), and let \(f : X \to \mathbf{R}\) be a function.
    If \(f\) is monotone increasing and \(f\) is differentiable at \(x_0\), then \(f'(x_0) \geq 0\).
    If f is monotone decreasing and \(f\) is differentiable at \(x_0\), then \(f'(x_0) \leq 0\).
\end{proposition}

\begin{proof}
    First suppose that \(f\) is monotone increasing.
    Since \(f\) is differentiable at \(x_0\), by Definition \ref{10.1.1} we have
    \[
        f'(x_0) = \lim_{x \to x_0 ; x \in X \setminus \{x_0\}} \frac{f(x) - f(x_0)}{x - x_0}.
    \]
    Since \(x, x_0 \in \mathbf{R}\) and \(x \in X \setminus \{x_0\}\), we have either \(x < x_0\) and \(x > x_0\).
    \begin{enumerate}
        \item If \(x < x_0\), then we have
              \begin{align*}
                           & (x - x_0 < 0) \land \big(f(x) - f(x_0) \leq 0\big) & \text{(by Definition \ref{9.8.1})} \\
                  \implies & \frac{f(x) - f(x_0)}{x - x_0} \geq 0.                                                   \\
              \end{align*}
        \item If \(x > x_0\), then we have
              \begin{align*}
                           & (x - x_0 > 0) \land \big(f(x) - f(x_0) \geq 0\big) & \text{(by Definition \ref{9.8.1})} \\
                  \implies & \frac{f(x) - f(x_0)}{x - x_0} \geq 0.                                                   \\
              \end{align*}
              From all cases above we have \(x \in X \setminus \{x_0\} \implies \frac{f(x) - f(x_0)}{x - x_0} \geq 0\).
              Thus by Proposition \ref{9.3.14} we have \(f'(x_0) \geq 0\).
    \end{enumerate}

    Now suppose that \(f\) is monotone decreasing.
    Since \(f\) is differentiable at \(x_0\), by Definition \ref{10.1.1} we have
    \[
        f'(x_0) = \lim_{x \to x_0 ; x \in X \setminus \{x_0\}} \frac{f(x) - f(x_0)}{x - x_0}.
    \]
    Since \(x, x_0 \in \mathbf{R}\) and \(x \in X \setminus \{x_0\}\), we have either \(x < x_0\) and \(x > x_0\).
    \begin{enumerate}
        \item If \(x < x_0\), then we have
              \begin{align*}
                           & (x - x_0 < 0) \land \big(f(x) - f(x_0) \geq 0\big) & \text{(by Definition \ref{9.8.1})} \\
                  \implies & \frac{f(x) - f(x_0)}{x - x_0} \leq 0.                                                   \\
              \end{align*}
        \item If \(x > x_0\), then we have
              \begin{align*}
                           & (x - x_0 > 0) \land \big(f(x) - f(x_0) \leq 0\big) & \text{(by Definition \ref{9.8.1})} \\
                  \implies & \frac{f(x) - f(x_0)}{x - x_0} \leq 0.                                                   \\
              \end{align*}
              From all cases above we have \(x \in X \setminus \{x_0\} \implies \frac{f(x) - f(x_0)}{x - x_0} \leq 0\).
              Thus by Proposition \ref{9.3.14} we have \(f'(x_0) \leq 0\).
    \end{enumerate}
\end{proof}

\begin{remark}\label{10.3.2}
    We have to assume that \(f\) is differentiable at \(x_0\);
    There exist monotone functions which are not always differentiable, and of course if \(f\) is not differentiable at \(x_0\) we cannot possibly conclude that \(f'(x_0) \geq 0\) or \(f'(x_0) \leq 0\).
\end{remark}
\section{Inverse functions and derivatives}\label{sec 10.4}

\begin{lemma}\label{10.4.1}
    Let \(f : X \to Y\) be an invertible function, with inverse \(f^{-1} : Y \to X\).
    Suppose that \(x_0 \in X\) and \(y_0 \in Y\) are such that \(y_0 = f(x_0)\)
    (which also implies that \(x_0 = f^{-1}(y_0)\)).
    If \(f\) is differentiable at \(x_0\), and \(f^{-1}\) is differentiable at \(y_0\), then
    \[
        (f^{-1})'(y_0) = \frac{1}{f'(x_0)}.
    \]
\end{lemma}

\begin{proof}
    From the chain rule (Theorem \ref{10.1.15}) we have
    \[
        (f^{-1} \circ f)'(x_0) = (f^{-1})'(y_0) f'(x_0).
    \]
    But \(f^{-1} \circ f\) is the identity function on \(X\), and hence by Theorem \ref{10.1.13}(b) \((f^{-1} \circ f)'(x_0) = 1\).
    The claim follows.
\end{proof}

\begin{note}
    As a particular corollary of Lemma \ref{10.4.1}, we see that if \(f\) is differentiable at \(x_0\) with \(f'(x_0) = 0\), then \(f^{-1}\) cannot be differentiable at \(y_0 = f(x_0)\), since \(1 / f'(x_0)\) is undefined in that case.
\end{note}

\begin{note}
    If one writes \(y = f(x)\), so that \(x = f^{-1}(y)\), then one can write the conclusion of Lemma \ref{10.4.1} in the more appealing form \(dx / dy = 1 / (dy / dx)\).
    However, as mentioned before, this way of writing things, while very convenient and easy to remember, can be misleading and cause errors if applied too carelessly (especially when one begins to work in the calculus of several variables).
\end{note}

\begin{note}
    Lemma \ref{10.4.1} seems to answer the question of how to differentiate the inverse of a function, however it has one significant drawback:
    the lemma only works if one assumes a \emph{priori} that \(f^{-1}\) is differentiable.
    Thus, if one does not already know that \(f^{-1}\) is differentiable, one cannot use Lemma \ref{10.4.1} to compute the derivative of \(f^{-1}\) .
\end{note}
\chapter{The Riemann integral}\label{ch 11}

\begin{note}
    In Chapter \ref{ch 10} we reviewed \emph{differentiation} - one of the two pillars of single variable calculus.
    The other pillar is, of course, \emph{integration}, which is the focus of the current chapter.
    More precisely, we will turn to the \emph{definite integral}, the integral of a function on a fixed interval, as opposed to the \emph{indefinite integral}, otherwise known as the \emph{antiderivative}.
    These two are of course linked by the \emph{Fundamental theorem of calculus}.
\end{note}

\begin{note}
    To actually \emph{define} this integral \(\int_I f\) is somewhat delicate (especially if one does not want to assume any axioms concerning geometric notions such as area), and not all functions \(f\) are integrable.
    It turns out that there are at least two ways to define this integral:
    the \emph{Riemann integral}, named after Georg Riemann (1826 -- 1866), which suffices for most applications, and the \emph{Lebesgue integral}, named after Henri Lebesgue (1875 -- 1941), which supercedes the Riemann integral and works for a much larger class of functions.
    There is also the \emph{Riemann-Steiltjes integral} \(\int_I f(x) d \alpha(x)\), a generalization of the Riemann integral due to Thomas Stieltjes (1856 -- 1894).
\end{note}

\section{Partitions}\label{sec 11.1}

\begin{definition}\label{11.1.1}
    Let \(X\) be a subset of \(\mathbf{R}\).
    We say that \(X\) is \emph{connected} iff \(X\) is nonempty and the following property is true:
    whenever \(x, y\) are elements in \(X\) such that \(x < y\), the bounded interval \([x, y]\) is a subset of \(X\)
    (i.e., every number between \(x\) and \(y\) is also in \(X\)).
\end{definition}

\setcounter{theorem}{3}
\begin{lemma}\label{11.1.4}
    Let \(X\) be a subset of the real line.
    Then the following two statements are logically equivalent:
    \begin{enumerate}
        \item \(X\) is bounded and either connected or empty.
        \item \(X\) is a bounded interval.
    \end{enumerate}
\end{lemma}

\begin{proof}
    Both statements are logically equivalent when \(X = \emptyset\) (which is vacuously true).
    So suppose that \(X \neq \emptyset\).

    We first show that \(X\) is bounded and connected implies \(X\) is a bounded interval.
    Since \(X\) is bounded, by Theorem \ref{5.5.9} we know that \(\inf(X), \sup(X) \in \mathbf{R}\).
    Thus \(X \subseteq [\inf(X), \sup(X)]\).
    Now we split into four cases:
    \begin{itemize}
        \item If \(\sup(X) \in X\) and \(\inf(X) \in X\), then by Definition \ref{11.1.1} \(X\) is connected implies \([\inf(X), \sup(X)] \subseteq X\).
              Thus by Proposition \ref{3.1.18} we have \(X = [\inf(X), \sup(X)]\).
        \item If \(\sup(X) \in X\) and \(\inf(X) \notin X\), then we claim that \(\big(\inf(X), \sup(X)] \subseteq X\).
              This is true since \(X\) is connected and by Definition \ref{11.1.1} we have \(\big(a, \sup(X)] \subseteq X\) for every \(a \in X\).
        \item If \(\sup(X) \notin X\) and \(\inf(X) \in X\), then we claim that \([\inf(X), \sup(X)\big) \subseteq X\).
              This is true since \(X\) is connected and by Definition \ref{11.1.1} we have \([\inf(X), b\big) \subseteq X\) for every \(b \in X\).
        \item If \(\sup(X) \notin X\) and \(\inf(X) \notin X\), then we claim that \(\big(\inf(X), \sup(X)\big) \subseteq X\).
              This is true since \(X\) is connected and by Definition \ref{11.1.1} we have \((a, b) \subseteq X\) for every \(a, b \in X\) and \(a < b\).
    \end{itemize}
    From all cases above we conclude that \(X\) is a bounded interval.

    Now we show that \(X\) is a bounded interval implies \(X\) is bounded and connected.
    Obviously \(X\) is bounded.
    Let \(a, b \in \mathbf{R}\).
    Then \(X\) can be one of \((a, b), [a, b], (a, b], [a, b)\), and by Definition \ref{11.1.1} all of which are connected.
\end{proof}

\begin{remark}\label{11.1.5}
    Recall that intervals are allowed to be singleton points, or even the empty set.
\end{remark}

\begin{corollary}\label{11.1.6}
    If \(I\) and \(J\) are bounded intervals, then the intersection \(I \cap J\) is also a bounded interval.
\end{corollary}

\begin{proof}
    If \(I \cap J = \emptyset\), then \(I \cap J\) is bounded interval.
    So suppose that \(I \cap J \neq \emptyset\).
    Since \(I, J\) are bounded intervals, by Lemma \ref{11.1.4} we know that \(I, J\) are bounded and connected.
    Since \(I, J\) are bounded, \(\exists\ M_1, M_2 \in \mathbf{R}\) such that \(I \subseteq [-M_1, M_1]\) and \(J \subseteq [-M_2, M_2]\).
    Let \(M = \min(M_1, M_2)\).
    Then we have \(I \cap J \subseteq [-M, M]\) and thus \(I \cap J\) is bounded.
    Let \(x, y \in I \cap J\) and \(x < y\).
    Since \(I\) is connected and \(I \cap J \subseteq I\), we have \([x, y] \subseteq I\).
    Similarly since \(J\) is connected and \(I \cap J \subseteq J\), we have \([x, y] \subseteq J\).
    Thus \([x, y] \subseteq I \cap J\) and by Definition \ref{11.1.1} \(I \cap J\) is connected.
    Since \(I \cap J\) is bounded and connected, by Lemma \ref{11.1.4} \(I \cap J\) is bounded interval.
\end{proof}

\setcounter{theorem}{7}
\begin{definition}[Length of intervals]\label{11.1.8}
    If \(I\) is a bounded interval, we define the \emph{length} of \(I\), denoted \(\abs*{I}\) as follows.
    If \(I\) is one of the intervals \([a, b]\), \((a, b)\), \([a, b)\), or \((a, b]\) for some real numbers \(a < b\), then we define \(\abs*{I} \coloneqq b - a\).
    Otherwise, if \(I\) is a point or the empty set, we define \(\abs*{I} = 0\).
\end{definition}

\setcounter{theorem}{9}
\begin{definition}[Partitions]\label{11.1.10}
    Let \(I\) be a bounded interval.
    A \emph{partition} of \(I\) is a finite set \(\mathbf{P}\) of bounded intervals contained in \(I\), such that every \(x\) in \(I\) lies in exactly one of the bounded intervals \(J\) in \(\mathbf{P}\).
\end{definition}

\begin{remark}\label{11.1.11}
    Note that a partition is a set of intervals, while each interval is itself a set of real numbers.
    Thus a partition is a set consisting of other sets.
\end{remark}

\setcounter{theorem}{12}
\begin{theorem}[Length is finitely additive]\label{11.1.13}
    Let \(I\) be a bounded interval, \(n\) be a natural number, and let \(\mathbf{P}\) be a partition of \(I\) of cardinality \(n\).
    Then
    \[
        \abs*{I} = \sum_{J \in \mathbf{P}} \abs*{J}.
    \]
\end{theorem}

\begin{proof}
    We prove this by induction on \(n\).
    More precisely, we let \(P(n)\) be the property that whenever \(I\) is a bounded interval, and whenever \(\mathbf{P}\) is a partition of \(I\) with cardinality \(n\), that \(\abs*{I} = \sum_{J \in \mathbf{P}} \abs*{J}\).

    The base case \(P(0)\) is trivial;
    the only way that \(I\) can be partitioned into an empty partition is if \(I\) is itself empty, at which point the claim is easy.
    The case \(P(1)\) is also very easy;
    the only way that \(I\) can be partitioned into a singleton set \(\{J\}\) is if \(J = I\), at which point the claim is again very easy.

    Now suppose inductively that \(P(n)\) is true for some \(n \geq 1\), and now we prove \(P(n + 1)\).
    Let \(I\) be a bounded interval, and let \(\mathbf{P}\) be a partition of \(I\) of cardinality \(n + 1\).

    If \(I\) is the empty set or a point, then all the intervals in \(\mathbf{P}\) must also be either the empty set or a point, and so every interval has length zero and the claim is trivial.
    Thus we will assume that \(I\) is an interval of the form \((a, b)\), \((a, b]\), \([a, b)\), or \([a, b]\).

            Let us first suppose that \(b \in I\), i.e., \(I\) is either \((a, b]\) or \([a, b]\).
    Since \(b \in I\), we know that one of the intervals \(K\) in \(\mathbf{P}\) contains \(b\).
    Since \(K\) is contained in \(I\), it must therefore be of the form \((c, b]\), \([c, b]\), or \(\{b\}\) for some real number \(c\), with \(a \leq c \leq b\) (in the latter case of \(K = \{b\}\), we set \(c \coloneqq b\)).
    In particular, this means that the set \(I \setminus K\) is also an interval of the form \([a, c]\), \((a, c)\), \((a, c]\), \([a, c)\) when \(c > a\), or a point or empty set when \(a = c\).
    Either way, we easily see that
    \[
        \abs*{I} = \abs*{K} + \abs*{I \setminus K}.
    \]
    On the other hand, since \(\mathbf{P}\) forms a partition of \(I\), we see that \(\mathbf{P} \setminus \{K\}\) forms a partition of \(I \setminus K\).
    By the induction hypothesis, we thus have
    \[
        \abs*{I \setminus K} = \sum_{J \in \mathbf{P} \setminus \{K\}} \abs*{J}.
    \]
    Combining these two identities (and using the laws of addition for finite sets, see Proposition \ref{7.1.11}(e)) we obtain
    \[
        \abs*{I} = \sum_{J \in \mathbf{P}} \abs*{J}
    \]
    as desired.

    Now suppose that \(b \notin I\), i.e., \(I\) is either \((a, b)\) or \([a, b)\).
    Then one of the intervals \(K\) also is of the form \((c, b)\) or \([c, b)\) (see Exercise \ref{ex 11.1.3}).
            In particular, this means that the set \(I \setminus K\) is also an interval of the form \([a, c]\), \((a, c)\), \((a, c]\), \([a, c)\) when \(c > a\), or a point or empty set when \(a = c\).
    The rest of the argument then proceeds as above.
\end{proof}

\begin{definition}[Finer and coarser partitions]\label{11.1.14}
    Let \(I\) be a bounded interval, and let \(\mathbf{P}\) and \(\mathbf{P}'\) be two partitions of \(I\).
    We say that \(\mathbf{P}'\) is \emph{finer} than \(\mathbf{P}\) (or equivalently, that \(\mathbf{P}\) is \emph{coarser} than \(\mathbf{P}'\)) if for every \(J\) in \(\mathbf{P}'\), there exists a \(K\) in \(\mathbf{P}\) such that \(J \subseteq K\).
\end{definition}

\begin{note}
    There is no such thing as a ``finest'' partition of some interval \(I\).
    (recall all partitions are assumed to be finite.)
    We do not compare partitions of different intervals.
\end{note}

\setcounter{theorem}{15}
\begin{definition}[Common refinement]\label{11.1.16}
    Let \(I\) be a bounded interval, and let \(\mathbf{P}\) and \(\mathbf{P}'\) be two partitions of \(I\).
    We define the \emph{common refinement} \(\mathbf{P} \# \mathbf{P}'\) of \(\mathbf{P}\) and \(\mathbf{P}'\) to be the set
    \[
        \mathbf{P} \# \mathbf{P}' \coloneqq \{K \cap J : K \in \mathbf{P} \land J \in \mathbf{P}'\}.
    \]
\end{definition}

\setcounter{theorem}{17}
\begin{lemma}\label{11.1.18}
    Let \(I\) be a bounded interval, and let \(\mathbf{P}\) and \(\mathbf{P}'\) be two partitions of \(I\).
    Then \(\mathbf{P} \# \mathbf{P}'\) is also a partition of \(I\), and is both finer than \(\mathbf{P}\) and finer than \(\mathbf{P}'\).
\end{lemma}

\begin{proof}
    Let \(x \in I\).
    Then by Definition \ref{11.1.10} we know that \(\exists!\ K \in \mathbf{P}\) such that \(x \in K\).
    Similarly \(\exists!\ J \in \mathbf{P}'\) such that \(x \in J\), thus \(x \in K \cap J\).
    By Definition \ref{11.1.16} we know that \(K \cap J \in \mathbf{P} \# \mathbf{P}'\).
    Thus we have
    \[
        I \subseteq \bigcup \big(\mathbf{P} \# \mathbf{P}'\big).
    \]

    Let \(S \in \mathbf{P} \# \mathbf{P}'\).
    By Definition \ref{11.1.16} we know that \(\exists\ K \in \mathbf{P}\) and \(\exists\ J \in \mathbf{P}'\) such that \(S = K \cap J\).
    Since \(S = K \cap J\), we have \(S \subseteq I\), thus
    \[
        \bigcup \big(\mathbf{P} \# \mathbf{P}'\big) \subseteq I.
    \]
    Combining the result above we have
    \[
        I = \bigcup \big(\mathbf{P} \# \mathbf{P}'\big).
    \]
    Since \(K, J\) are bounded interval, by Corollary \ref{11.1.6} we know that \(S = K \cap J\) is a bounded interval.

    We now claim that \(\forall\ x \in I\), \(\exists!\ S \in \mathbf{P} \# \mathbf{P}'\) such that \(x \in S\).
    So suppose for sake of contradiction that \(\exists\ S_1, S_2 \in \mathbf{P} \# \mathbf{P}'\) such that \(S_1 \neq S_2\), \(x \in S_1\) and \(x \in S_2\).
    By Definition \ref{11.1.16} we know that \(\exists\ K_1 \in \mathbf{P}\) and \(\exists\ J_1 \in \mathbf{P}'\) such that \(S_1 = K_1 \cap J_1\).
    Similarly \(\exists\ K_2 \in \mathbf{P}\) and \(\exists\ J_2 \in \mathbf{P}'\) such that \(S_2 = K_2 \cap J_2\).
    Since \(x \in S_1\), \(x \in K_1\).
    Since \(x \in S_2\), \(x \in K_2\).
    But by Definition \ref{11.1.10} we know that \(K_1 = K_2\), similar argument holds for \(J_1 = J_2\).
    Thus we must have \(S_1 = S_2\), a contradiction.

    We already show that \(\mathbf{P} \# \mathbf{P}' = I\);
    \(\forall\ S \in \mathbf{P} \# \mathbf{P}'\), \(S\) is a bounded interval;
    \(\forall\ x \in I\), \(\exists!\ S \in \mathbf{P} \# \mathbf{P}'\) such that \(x \in I\).
    To show that \(\mathbf{P} \# \mathbf{P}'\) is a partition of \(I\), by Definition \ref{11.1.10} we also need to show that \(\mathbf{P} \# \mathbf{P}'\) is a finite set.
    Let \(f : \mathbf{P} \times \mathbf{P}' \to \mathbf{P} \# \mathbf{P}'\) be a function where
    \[
        \forall\ K \in \mathbf{P} \land \forall\ J \in \mathbf{P}', f(K, J) = K \cap J.
    \]
    From the proof above we can see that \(f\) is surjective.
    By Definition \ref{11.1.10} we have \(\#(\mathbf{P}), \#(\mathbf{P}')\) are finite.
    Thus by Proposition \ref{3.6.14}(e) and Exercise \ref{ex 8.4.3} we have
    \[
        \#(\mathbf{P} \times \mathbf{P}') = \#(\mathbf{P}) \times \#(\mathbf{P}') \geq \#(\mathbf{P} \# \mathbf{P}')
    \]
    and \(\mathbf{P} \# \mathbf{P}'\) is finite, and we conclude that \(\mathbf{P} \# \mathbf{P}'\) is a partition of \(I\).

    From the proof above we have \(\forall\ S \in \mathbf{P} \# \mathbf{P}'\), \(\exists!\ K \in \mathbf{P}\) and \(\exists!\ J \in \mathbf{P}'\) such that \(S = K \cap J\).
    Then we have \(S \subseteq K\) and \(S \subseteq J\).
    Thus by Definition \ref{11.1.14} \(\mathbf{P} \# \mathbf{P}'\) is both finer than \(\mathbf{P}\) and finer than \(\mathbf{P}'\)
\end{proof}

\exercisesection

\begin{exercise}\label{ex 11.1.1}
    Prove Lemma \ref{11.1.4}.
\end{exercise}

\begin{proof}
    See Lemma \ref{11.1.4}.
\end{proof}

\begin{exercise}\label{ex 11.1.2}
    Prove Corollary \ref{11.1.6}.
\end{exercise}

\begin{proof}
    Prove Corollary \ref{11.1.6}.
\end{proof}

\begin{exercise}\label{ex 11.1.3}
    Let \(I\) be a bounded interval of the form \(I = (a, b)\) or \(I = [a, b)\) for some real numbers \(a < b\).
    Let \(I_1, \dots, I_n\) be a partition of \(I\).
    Prove that one of the intervals \(I_j\) in this partition is of the form \(I_j = (c, b)\) or \(I_j = [c, b)\) for some \(a \leq c \leq b\).
\end{exercise}

\begin{proof}
    Let \(\mathbf{P} = \{I_1, \dots, I_n\}\).
    If \(c = b\), then \((c, b) = \emptyset\), and thus by Definition \ref{11.1.10} \(\mathbf{P} \cup \emptyset\) is a partition of \(I\).
    So we only need to proof the cases where \(a \leq c < b\).
    Suppose for sake of contradiction that every interval \(I_j\) in the partition is not of the form \((c, b)\) or \([c, b)\).
    By Definition \ref{11.1.10} this means \(\forall\ x \in I_j\), we have \(x \geq b\) or \(x < c\).
    Since \(I = (a, b)\) or \(I = [a, b)\), we know that \(x < b\), thus we must have \(x < c\).
    This means \(\sup(I_j) \leq c < b\).
    But we know that \(\sup(I) = b\).
    So we have \(\forall\ x \in I\), \(\exists!\ I_j \in \mathbf{P}\) such that \(x \in I_j\), and \(x \leq \sup(I_j) < b = \sup(I)\).
    Thus \(\sup(I) \neq b\), a contradiction.
\end{proof}

\begin{exercise}\label{ex 11.1.4}
    Prove Lemma \ref{11.1.18}.
\end{exercise}

\begin{proof}
    Prove Lemma \ref{11.1.18}.
\end{proof}
\section{Piecewise constant functions}\label{sec 11.2}

\begin{definition}[Constant functions]\label{11.2.1}
    Let \(X\) be a subset of \(\mathbf{R}\), and let \(f : X \to \mathbf{R}\) be a function.
    We say that \(f\) is \emph{constant} iff there exists a real number \(c\) such that \(f(x) = c\) for all \(x \in X\).
    If \(E\) is a subset of \(X\), we say that \(f\) is \emph{constant on} \(E\) if the restriction \(f|_E\) of \(f\) to \(E\) is constant, in other words there exists a real number \(c\) such that \(f(x) = c\) for all \(x \in E\).
    We refer to \(c\) as the \emph{constant value} of \(f\) on \(E\).
\end{definition}

\begin{remark}\label{11.2.2}
    If \(E\) is a non-empty set, then a function \(f\) which is constant on \(E\) can have only one constant value;
    However, if \(E\) is empty, every real number \(c\) is a constant value for \(f\) on \(E\).
\end{remark}

\begin{definition}[Piecewise constant functions I]\label{11.2.3}
    Let \(I\) be a bounded interval, let \(f : I \to \mathbf{R}\) be a function, and let \(\mathbf{P}\) be a partition of \(I\).
    We say that \(f\) is \emph{piecewise constant with respect to \(\mathbf{P}\)} if for every \(J \in \mathbf{P}\), \(f\) is constant on \(J\).
\end{definition}

\setcounter{theorem}{4}
\begin{definition}[Piecewise constant functions II]\label{11.2.5}
    Let \(I\) be a bounded interval, and let \(f : I \to \mathbf{R}\) be a function.
    We say that \(f\) is \emph{piecewise constant on \(I\)} if there exists a partition \(\mathbf{P}\) of \(I\) such that \(f\) is piecewise constant with respect to \(\mathbf{P}\).
\end{definition}

\setcounter{theorem}{6}
\begin{lemma}\label{11.2.7}
    Let \(I\) be a bounded interval, let \(\mathbf{P}\) be a partition of \(I\), and let \(f : I \to \mathbf{R}\) be a function which is piecewise constant with respect to \(\mathbf{P}\).
    Let \(\mathbf{P}'\) be a partition of \(I\) which is finer than \(\mathbf{P}\).
    Then \(f\) is also piecewise constant with respect to \(\mathbf{P}'\).
\end{lemma}

\begin{proof}
    Let \(K \in \mathbf{P}'\).
    Since \(\mathbf{P}'\) is finer than \(\mathbf{P}\), by Definition \ref{11.1.14} \(\exists\ J \in \mathbf{P}\) such that \(K \subseteq J\).
    Since \(f\) is piecewise constant with respect to \(\mathbf{P}\), by Definition \ref{11.2.3} we know that \(\forall\ x \in J\), \(f(x)\) is constant.
    Thus \(\forall\ x \in K\), \(x \in J\) and \(f(x)\) is constant.
    By Definition \ref{11.2.3} \(f\) is piecewise constant with respect to \(\mathbf{P}'\).
\end{proof}

\begin{lemma}\label{11.2.8}
    Let \(I\) be a bounded interval, and let \(f : I \to \mathbf{R}\) and \(g : I \to \mathbf{R}\) be piecewise constant functions on \(I\).
    Then the functions \(f + g\), \(f - g\), \(\max(f, g)\), \(\min(f, g)\) and \(fg\) are also piecewise constant functions on \(I\).
    Here of course \(\max(f, g) : I \to \mathbf{R}\) is the function \(\max(f, g)(x) \coloneqq \max(f(x), g(x))\).
    If \(g\) does not vanish anywhere on \(I\) (i.e., \(g(x) \neq 0\) for all \(x \in I\)) then \(f / g\) is also a piecewise constant function on \(I\).
\end{lemma}

\begin{proof}
    Since \(f\) is piecewise constant function on \(I\), by Definition \ref{11.2.5} \(\exists\ \mathbf{P}\) such that \(\mathbf{P}\) is a partition of \(I\) and \(f\) is piecewise constant with respect to \(\mathbf{P}\).
    Similarly since \(g\) is piecewise constant function on \(I\), by Definition \ref{11.2.5} \(\exists\ \mathbf{P}'\) such that \(\mathbf{P}'\) is a partition of \(I\) and \(g\) is piecewise constant with respect to \(\mathbf{P}'\).
    By Lemma \ref{11.1.18} we know that \(\mathbf{P} \# \mathbf{P}'\) is also a partition of \(I\) and \(\mathbf{P} \# \mathbf{P}'\) is both finer than \(\mathbf{P}\) and finer than \(\mathbf{P}'\).
    By Lemma \ref{11.2.7} we know that both \(f\) and \(g\) are piecewise constant with respect to \(\mathbf{P} \# \mathbf{P}'\).

    Now we show that \(f, g\) remain piecewise constant functions on \(I\) after algebraic operation.
    Since \(\forall\ J \in \mathbf{P} \# \mathbf{P}'\), we have \(\forall\ x \in J\), \(f(x)\) is constant and \(g(x)\) is constant.
    Thus we know that \(f(x) + g(x)\), \(f(x) - g(x)\), \(\max(f(x), g(x))\), \(\min(f(x), g(x))\) and \(f(x) g(x)\) are constant.
    If \(g(x) \neq 0\), then we also have \(f(x) / g(x)\) is constant.
    Thus by Definition \ref{11.2.3} \(f + g\), \(f - g\), \(\max(f, g)\), \(\min(f, g)\), \(fg\) is piecewise constant with respect to \(\mathbf{P} \# \mathbf{P}'\), and when \(g(x) \neq 0\) we have \(f / g\) is piecewise constant with respect to \(\mathbf{P} \# \mathbf{P}'\).
    By Definition \ref{11.2.5} \(f + g\), \(f - g\), \(\max(f, g)\), \(\min(f, g)\), \(fg\) is piecewise constant on \(I\), and when \(g(x) \neq 0\) we have \(f / g\) is piecewise constant on \(I\).
\end{proof}

\begin{definition}[Piecewise constant integral I]\label{11.2.9}
    Let \(I\) be a bounded interval, let \(\mathbf{P}\) be a partition of \(I\).
    Let \(f : I \to \mathbf{R}\) be a function which is piecewise constant with respect to \(\mathbf{P}\).
    Then we define the \emph{piecewise constant integral} \(p.c. \int_{[\mathbf{P}]} f\) of \(f\) with respect to the partition \(\mathbf{P}\) by the formula
    \[
        p.c. \int_{[\mathbf{P}]} f \coloneqq \sum_{J \in \mathbf{P}} c_J \abs*{J},
    \]
    where for each \(J\) in \(\mathbf{P}\), we let \(c_J\) be the constant value of \(f\) on \(J\).
\end{definition}

\begin{remark}\label{11.2.10}
    This definition seems like it could be ill-defined, because if \(J\) is empty then every number \(c_J\) can be the constant value of \(f\) on \(J\), but fortunately in such cases \(\abs*{J}\) is zero and so the choice of \(c_J\) is irrelevant.
    The notation \(p.c. \int_{[\mathbf{P}]} f\) is rather artificial, but we shall only need it temporarily, en route to a more useful definition.
    Note that since \(\mathbf{P}\) is finite, the sum \(\sum_{J \in \mathbf{P}} c_J \abs*{J}\) is always well-defined
    (it is never divergent or infinite).
\end{remark}

\begin{remark}\label{11.2.11}
    The piecewise constant integral corresponds intuitively to one's notion of area, given that the area of a rectangle ought to be the product of the lengths of the sides.
    (Of course, if \(f\) is negative somewhere, then the ``area'' \(c_J \abs*{J}\) would also be negative.)
\end{remark}

\setcounter{theorem}{12}
\begin{proposition}[Piecewise constant integral is independent of partition]\label{11.2.13}
    Let \(I\) be a bounded interval, and let \(f : I \to \mathbf{R}\) be a function.
    Suppose that \(\mathbf{P}\) and \(\mathbf{P}'\) are partitions of \(I\) such that \(f\) is piecewise constant both with respect to \(\mathbf{P}\) and with respect to \(\mathbf{P}'\).
    Then \(p.c. \int_{[\mathbf{P}]} f = p.c. \int_{[\mathbf{P}']} f\).
\end{proposition}

\begin{proof}
    By Lemma \ref{11.1.18} we know that \(\mathbf{P} \# \mathbf{P}'\) is a partition of \(I\) and is both finer than \(\mathbf{P}\) and finer than \(\mathbf{P}'\), thus by Definition \ref{11.2.9} we have
    \[
        p.c. \int_{[\mathbf{P} \# \mathbf{P}']} f = \sum_{J \in \mathbf{P} \# \mathbf{P}'} c_J \abs*{J}.
    \]
    By Theorem \ref{11.1.13}, we know that
    \[
        \abs*{I} = \sum_{J \in \mathbf{P}} \abs*{J} = \sum_{J \in \mathbf{P} \# \mathbf{P}'} \abs*{J}.
    \]
    By Definition \ref{11.1.14}, \(\forall\ S \in \mathbf{P} \# \mathbf{P}'\), \(\exists\ K \in \mathbf{P}\) such that \(S \subseteq K\).
    Now we fix such \(K\) and let \(\mathbf{P}_K\) be the set
    \[
        \mathbf{P}_K = \{S \in \mathbf{P} \# \mathbf{P}' : S \subseteq K\}.
    \]
    We claim that \(\mathbf{P}_K\) is a partition of \(K\).
    We know that \(\mathbf{P}_K\) is finite since \(\mathbf{P}_K \subseteq \mathbf{P} \# \mathbf{P}'\) and \(\mathbf{P} \# \mathbf{P}'\) is finite.
    Since \(\mathbf{P} \# \mathbf{P}'\) is a partition of \(I\), by Definition \ref{11.1.10} \(\forall\ S \in \mathbf{P}_K\), \(S\) is a bounded interval, and if \(S' \in \mathbf{P}_K\) and \(S \neq S'\) then \(S \cap S' = \emptyset\).
    Let \(x \in K\).
    By Definition \ref{11.1.10} we must have \(x \in I\), and \(\exists!\ K' \in \mathbf{P}'\) such that \(x \in K'\).
    By Definition \ref{11.1.16} we know that \(K \cap K' \in \mathbf{P} \# \mathbf{P}'\).
    Since \(K \cap K' \subseteq K\), we have \(x \in \bigcup \mathbf{P}_K\), so \(K \subseteq \bigcup \mathbf{P}_K\).
    By the definition of \(\mathbf{P}_K\) we know that \(\bigcup \mathbf{P}_K \subseteq K\), thus by Proposition \ref{3.1.18} we have \(K = \bigcup \mathbf{P}_K\) and by Definition \ref{11.1.10} \(\mathbf{P}_K\) is a partition of \(K\).

    Now we show that \(\bigcup_{K \in \mathbf{P}} \mathbf{P}_K = \mathbf{P} \# \mathbf{P}'\).
    By the definition of \(\mathbf{P}_K\) we have \(\bigcup_{K \in \mathbf{P}} \mathbf{P}_K \subseteq \mathbf{P} \# \mathbf{P}'\).
    Let \(S \in \mathbf{P} \# \mathbf{P}'\).
    Since \(\mathbf{P} \# \mathbf{P}'\) is finer than \(\mathbf{P}\), by Definition \ref{11.1.14} \(\exists\ K \in \mathbf{P}\) such that \(S \subseteq K\).
    Thus \(S \in \mathbf{P}_K\) and \(\mathbf{P} \# \mathbf{P}' \subseteq \bigcup_{K \in \mathbf{P}} \mathbf{P}_K\).
    Again by Proposition \ref{3.1.18} we have \(\bigcup_{K \in \mathbf{P}} \mathbf{P}_K = \mathbf{P} \# \mathbf{P}'\).

    Since \(f\) is piecewise constant with respect to \(\mathbf{P}\), by Lemma \ref{11.2.7} we know that \(f\) is piecewise constant with respect to \(\mathbf{P} \# \mathbf{P}'\).
    So we have
    \begin{align*}
        p.c. \int_{[\mathbf{P} \# \mathbf{P}']} f & = \sum_{J \in \mathbf{P} \# \mathbf{P}'} c_J \abs*{J}                        & \text{(by Definition \ref{11.2.9})}     \\
                                                  & = \sum_{J \in \bigcup_{K \in \mathbf{P}} \mathbf{P}_K} c_J \abs*{J}                                                    \\
                                                  & = \sum_{K \in \mathbf{P}} \sum_{J \in \mathbf{P}_K} c_J \abs*{J}             & \text{(by Proposition \ref{7.1.11}(e))} \\
                                                  & = \sum_{K \in \mathbf{P}} \sum_{J \in \mathbf{P}_K} c_K \abs*{J}             & (J \subseteq K)                         \\
                                                  & = \sum_{K \in \mathbf{P}} c_K \bigg(\sum_{J \in \mathbf{P}_K} \abs*{J}\bigg)                                           \\
                                                  & = \sum_{K \in \mathbf{P}} c_K \abs*{K}                                       & \text{(by Theorem \ref{11.1.13})}       \\
                                                  & = p.c. \int_{[\mathbf{P}]} f.                                                & \text{(by Definition \ref{11.2.9})}
    \end{align*}
    Using similar arguments we can show that \(p.c. \int_{[\mathbf{P}']} f = p.c. \int_{[\mathbf{P} \# \mathbf{P}']} f\).
    Thus we have \(p.c. \int_{[\mathbf{P}]} f = p.c. \int_{[\mathbf{P}']} f\).
\end{proof}

\begin{definition}[Piecewise constant integral II]\label{11.2.14}
    Let \(I\) be a bounded interval, and let \(f : I \to \mathbf{R}\) be a piecewise constant function on \(I\).
    We define the \emph{piecewise constant integral} \(p.c. \int_I f\) by the formula
    \[
        p.c. \int_I f \coloneqq p.c. \int_{[\mathbf{P}]} f,
    \]
    where \(\mathbf{P}\) is any partition of \(I\) with respect to which \(f\) is piecewise constant.
    (Note that Proposition \ref{11.2.13} tells us that the precise choice of this partition is irrelevant.)
\end{definition}

\setcounter{theorem}{15}
\begin{theorem}[Laws of integration]\label{11.2.16}
    Let \(I\) be a bounded interval, and let \(f : I \to \mathbf{R}\) and \(g : I \to \mathbf{R}\) be piecewise constant functions on \(I\).
    \begin{enumerate}
        \item We have \(p.c. \int_I (f + g) = p.c. \int_I f + p.c. \int_I g\).
        \item For any real number \(c\), we have \(p.c. \int_I (cf) = c (p.c. \int_I f)\).
        \item We have \(p.c. \int_I (f - g) = p.c. \int_I f - p.c. \int_I g\).
        \item If \(f(x) \geq 0\) for all \(x \in I\), then \(p.c. \int_I f \geq 0\).
        \item If \(f(x) \geq g(x)\) for all \(x \in I\), then \(p.c. \int_I f \geq p.c. \int_I g\).
        \item If \(f\) is the constant function \(f(x) = c\) for all \(x \in I\), then \(p.c. \int_I f = c \abs*{I}\).
        \item Let \(J\) be a bounded interval containing \(I\) (i.e., \(I \subseteq J\)), and let \(F : J \to \mathbf{R}\) be the function
              \[
                  F(x) \coloneqq \begin{cases}
                      f(x) & \text{if } x \in I    \\
                      0    & \text{if } x \notin I
                  \end{cases}
              \]
              Then \(F\) is piecewise constant on \(J\), and \(p.c. \int_I F = p.c. \int_I f\).
        \item Suppose that \(\{J, K\}\) is a partition of \(I\) into two intervals \(J\) and \(K\).
              Then the function \(f|_J : J \to \mathbf{R}\) and \(f|_K : K \to \mathbf{R}\) are piecewise constant on \(J\) and \(K\) respectively, and we have
              \[
                  p.c. \int_I f = p.c. \int_I f|_J + p.c. \int_I f|_K.
              \]
    \end{enumerate}
\end{theorem}

\begin{proof}{(a)}
    Since \(f, g\) are both piecewise constant on \(I\), by Lemma \ref{11.2.8} \(f + g\) is also piecewise constant on \(I\).
    By Definition \ref{11.2.3}, \(\exists\ \mathbf{P}_f, \mathbf{P}_g\) such that \(\mathbf{P}_f, \mathbf{P}_g\) are partitions of \(I\), \(f\) is piecewise constant with respect to \(\mathbf{P}_f\) and \(g\) is piecewise constant with respect to \(\mathbf{P}_g\).
    Let \(\mathbf{P} = \mathbf{P}_f \# \mathbf{P}_g\).
    Then by Lemma \ref{11.1.18} we know that \(\mathbf{P}\) is also a partition of \(I\) and by Lemma \ref{11.2.7} \(f, g\) are piecewise constant with respect to \(\mathbf{P}\).
    Now let \(J \in \mathbf{P}\), let \(f_J \in \mathbf{R}\) be the constant value of \(f\) on \(J\) and \(g_J \in \mathbf{R}\) be the constant value of \(g\) on \(J\).
    Then by Definition \ref{11.2.1} \(f_J + g_J\) is also a constant of \(f + g\) on \(J\).
    Thus we have \(f + g\) is piecewise constant with respect to \(\mathbf{P}\) and
    \begin{align*}
        p.c. \int_I f + p.c. \int_I g & = p.c. \int_{[\mathbf{P}]} f + p.c. \int_{[\mathbf{P}]} g                     & \text{(by Definition \ref{11.2.14})}    \\
                                      & = \sum_{J \in \mathbf{P}} f_J \abs*{J} + \sum_{J \in \mathbf{P}} g_J \abs*{J} & \text{(by Definition \ref{11.2.9})}     \\
                                      & = \sum_{J \in \mathbf{P}} (f_J + g_J) \abs*{J}                                & \text{(by Proposition \ref{7.1.11}(f))} \\
                                      & = p.c. \int_{[\mathbf{P}]} (f_J + g_J)                                        & \text{(by Definition \ref{11.2.9})}     \\
                                      & = p.c. \int_I (f_J + g_J).                                                    & \text{(by Definition \ref{11.2.14})}
    \end{align*}
\end{proof}

\begin{proof}{(b)}
    Since \(f\) is piecewise constant on \(I\), by Lemma \ref{11.2.8} \(cf\) is also piecewise constant on \(I\) (since \(c\) is constant on \(I\)).
    By Definition \ref{11.2.3}, \(\exists\ \mathbf{P}\) such that \(\mathbf{P}\) is a partition of \(I\) and \(f\) is piecewise constant with respect to \(\mathbf{P}\).
    Now let \(J \in \mathbf{P}\) and let \(f_J \in \mathbf{R}\) be the constant value of \(f\) on \(J\).
    Then by Definition \ref{11.2.1} \(c f_J\) is also a constant of \(cf\) on \(J\).
    Thus we have \(cf\) is piecewise constant with respect to \(\mathbf{P}\) and
    \begin{align*}
        c (p.c. \int_I f) & = c (p.c. \int_{[\mathbf{P}]} f)           & \text{(by Definition \ref{11.2.14})}    \\
                          & = c (\sum_{J \in \mathbf{P}} f_J \abs*{J}) & \text{(by Definition \ref{11.2.9})}     \\
                          & = \sum_{J \in \mathbf{P}} c f_J \abs*{J}   & \text{(by Proposition \ref{7.1.11}(g))} \\
                          & = p.c. \int_{[\mathbf{P}]} (c f)           & \text{(by Definition \ref{11.2.9})}     \\
                          & = p.c. \int_I (c f).                       & \text{(by Definition \ref{11.2.14})}
    \end{align*}
\end{proof}

\begin{proof}{(c)}
    We have
    \begin{align*}
        p.c. \int_I f - p.c. \int_I g & = p.c. \int_I f + (-1) p.c. \int_I g                                        \\
                                      & = p.c. \int_I f + p.c. \int_I (-g)   & \text{(by Theorem \ref{11.2.16}(b))} \\
                                      & = p.c. \int_I (f + (-g))             & \text{(by Theorem \ref{11.2.16}(a))} \\
                                      & = p.c. \int_I (f - g).               & \text{(by Definition \ref{9.2.1})}
    \end{align*}
\end{proof}

\begin{proof}{(d)}
    By Definition \ref{11.2.3}, \(\exists\ \mathbf{P}\) such that \(\mathbf{P}\) is a partition of \(I\) and \(f\) is piecewise constant with respect to \(\mathbf{P}\).
    Let \(J \in \mathbf{P}\) and let \(f_J \in \mathbf{R}\) be the constant value of \(f\) on \(J\).
    Since \(\forall\ x \in I\), \(f(x) \geq 0\), we then have \(f_J \geq 0\) and \(f_J \abs*{J} \geq 0\).
    Thus
    \begin{align*}
        p.c. \int_I f & = p.c. \int_{[\mathbf{P}]} f           & \text{(by Definition \ref{11.2.14})}    \\
                      & = \sum_{J \in \mathbf{P}} f_J \abs*{J} & \text{(by Definition \ref{11.2.9})}     \\
                      & \geq \sum_{J \in \mathbf{P}} 0         & \text{(by Proposition \ref{7.1.11}(h))} \\
                      & = 0.
    \end{align*}
\end{proof}

\begin{proof}{(e)}
    Since \(f(x) \geq g(x)\) for all \(x \in I\), we have \(f(x) - g(x) \geq 0\) and
    \begin{align*}
        p.c. \int_I f - p.c. \int_I g & = p.c. \int_I (f - g) & \text{(by Theorem \ref{11.2.16}(c))} \\
                                      & \geq 0.               & \text{(by Theorem \ref{11.2.16}(d))}
    \end{align*}
    Thus
    \[
        p.c. \int_I f \geq p.c. \int_I g.
    \]
\end{proof}

\begin{proof}{(f)}
    Since \(I\) is a partition of \(I\), we have
    \begin{align*}
        p.c. \int_I f & = p.c. \int_{[I]} f         & \text{(by Definition \ref{11.2.14})}    \\
                      & = \sum_{J \in I} c \abs*{J} & \text{(by Definition \ref{11.2.9})}     \\
                      & = c \sum_{J \in I} \abs*{J} & \text{(by Proposition \ref{7.1.11}(g))} \\
                      & = c \abs*{I}.               & \text{(by Theorem \ref{11.1.13})}
    \end{align*}
\end{proof}

\begin{proof}{(g)}
    Let \(I_1, I_2\) be the sets
    \[
        I_1 = \{x \in J, x \leq \inf(I) \land x \notin I\}
    \]
    and
    \[
        I_2 = \{x \in J, x \geq \sup(I) \land x \notin I \cup I_1\}.
    \]
    Then we know that \(I \cap I_1 = I \cap I_2 = I_1 \cap I_2 = \emptyset\).
    Let \(\mathbf{P} = \{I_1, I, I_2\}\).
    We know claim that \(\mathbf{P}\) is a partition of \(J\).
    Since \(J\) is a bounded interval, we know that \(\inf(J) = \inf(I_1)\) and \(\sup(J) = \sup(I_2)\).
    Then we have \(I_1 \subseteq [\inf(J), \inf(I)]\) and \(I_2 \subseteq [\sup(I), \sup(J)]\).
    If \(\inf(J) \in J\), then we know that \(I_1 = [\inf(J), \inf(I)]\) or \(I_1 = [\inf(J), \inf(I))\), which depends on whether \(\inf(I) \in I\).
    Otherwise we have \(I_1 = (\inf(J), \inf(I)]\) or \(I_1 = (\inf(J), \inf(I))\), which again depends on whether \(\inf(I) \in I\).
    Using similar arguments we know thtat \(I_2\) can be one of \((\sup(I), \sup(J))\), \([\sup(I), \sup(J))\), \((\sup(I), \sup(J)]\) or \([\sup(I), \sup(J)]\).
    Thus \(I_1, I_2\) are bounded intervals.
    By the definition of \(I_1, I_2\), we know that \(\bigcup \mathbf{P} \subseteq J\).
    To show that \(\bigcup \mathbf{P} = J\), by Proposition \ref{3.1.18} we need to show that \(J \subseteq \bigcup \mathbf{P}\).
    Let \(x \in J\).
    If \(x \in I\), we have \(x \in \bigcup \mathbf{P}\).
    If \(x \notin I\), we then have \(x \leq \inf(I) \lor x \geq \sup(I)\), thus \(x \in I_1 \lor x \in I_2\), and again \(x \in \bigcup \mathbf{P}\).
    Since \(x\) is arbitrary, we have \(J \subseteq \mathbf{P}\).
    Since \(\bigcup \mathbf{P} = J\) and \(\mathbf{P}\) is finite (\(\abs*{P} = 3\)), by Definition \ref{11.1.10} \(\mathbf{P}\) is a partition of \(J\).

    Now we show that \(F\) is piecewise constant on \(J\).
    Since \(f\) is piecewise constant on \(I\), by Definition \ref{11.2.5} \(\exists\ \mathbf{P}_I\) such that \(\mathbf{P}_I\) be the partition of \(I\) and \(f\) is piecewise constant with respect to \(\mathbf{P}_I\).
    By hypothesis we know that \(\forall\ K \in \mathbf{P}_I\), \(F\) is piecewise constant on \(K\) with constant value \(F(x) = f(x)\) for all \(x \in K\), thus by Definition \ref{11.2.5} \(F\) is piecewise constant on \(I\).
    Since \(\forall\ x \in I_1\), \(x \notin I\), by hypothesis we know that \(F(x) = 0\), thus \(F\) is piecewise constant on \(I_1\).
    Similar arguments show that \(F\) is piecewise constant on \(I_2\).
    Thus \(F\) is piecewise constant on \(\mathbf{P}\), and we have
    \begin{align*}
        p.c. \int_J F & = p.c. \int_{[\mathbf{P}]} F                                                       & \text{(by Definition \ref{11.2.14})}    \\
                      & = \sum_{K \in \mathbf{P}} c_K \abs*{K}                                             & \text{(by Definition \ref{11.2.9})}     \\
                      & = c_{I_1} \abs*{I_1} + \sum_{K \in \mathbf{P}_I} c_K \abs*{K} + c_{I_2} \abs*{I_2} & \text{(by Proposition \ref{7.1.11}(e))} \\
                      & = 0 \abs*{I_1} + \sum_{K \in \mathbf{P}_I} c_K \abs*{K} + 0 \abs*{I_2}             & \text{(by hypothesis)}                  \\
                      & = \sum_{K \in \mathbf{P}_I} c_K \abs*{K}                                                                                     \\
                      & = p.c. \int_{[\mathbf{P}_I]} f                                                     & \text{(by Definition \ref{11.2.9})}     \\
                      & = p.c. \int_I f.                                                                   & \text{(by Definition \ref{11.2.14})}
    \end{align*}
\end{proof}

\begin{proof}{(h)}
    We first show that \(f|_J\) is piecewise constant on \(J\) and \(f|_K\) is piecewise constant on \(K\).
    Since \(f\) is a piecewise constant function on \(I\), by Definition \ref{11.2.5} \(\exists\ \mathbf{P}\) such that \(\mathbf{P}\) is a partition of \(I\) and \(f\) is piecewise constant with respect to \(\mathbf{P}\).
    Let \(\mathbf{P}_J, \mathbf{P}_K\) be the sets
    \[
        \mathbf{P}_J = \{S \cap J : S \in \mathbf{P}\}
    \]
    and
    \[
        \mathbf{P}_K = \{S \cap K : S \in \mathbf{P}\}.
    \]
    Since \(\forall\ S_J \in \mathbf{P}_J\), \(S_J \in \mathbf{P}\), by Definition \ref{11.2.3} we know that \(f\) is constant on \(S_J\),
    By Definition \ref{11.1.10} \(S_J\) is a bounded interval, and \(S_{J'} \in \mathbf{P}_J\) and \(S_{J'} \neq S_J \implies S_{J'} \cap S_J = \emptyset\).
    By the definition of \(\mathbf{P}_J\) we know that \(\bigcup \mathbf{P}_J \subseteq J\).
    Let \(x \in J\).
    Since \(x \in J\), \(x \in I\), by Definition \ref{11.1.10} \(\exists!\ S_J \in \mathbf{P}_J\) such that \(x \in S_J\).
    Then we have \(x \in J \cap S_J\) and \(x \in \bigcup \mathbf{P}_J\), thus \(J \subseteq \bigcup \mathbf{P}_J\) and by Proposition \ref{3.1.18} \(J = \bigcup \mathbf{P}_J\).
    Since \(\mathbf{P}_J \subseteq \mathbf{P}\) and \(\mathbf{P}\) is finite, we know that \(\mathbf{P}_J\) is finite.
    Thus by Definition \ref{11.1.10} \(\mathbf{P}_J\) is a partition of \(J\).
    Using similar arguments we can show that \(\mathbf{P}_K\) is a partition of \(K\).
    Since \(\forall\ S_J \in \mathbf{P}_J\), \(S_J \in \mathbf{P}\) and \(f\) is piecewise constant on \(S_J\), by Definition \ref{11.2.3} \(f|_J\) is piecewise constant with respect to \(\mathbf{P}_J\).
    Using similar arguments we know that \(f|_K\) is piecewise constant with respect to \(\mathbf{P}_K\).
    Thus by Definition \ref{11.2.5} \(f|_J\) is piecewise constant on \(J\) and \(f|_K\) is piecewise constant on \(K\).

    Now we show that \(\mathbf{P} = \mathbf{P}_J \cup \mathbf{P}_K\).
    By the definition of \(\mathbf{P}_J\) and \(\mathbf{P}_K\) we know that \(\mathbf{P}_J \cup \mathbf{P}_J \subseteq \mathbf{P}\).
    Let \(S \in \mathbf{P}\).
    If \(S = \emptyset\), then \(S \subseteq \mathbf{P}_J \cup \mathbf{P}_K\).
    If \(S \neq \emptyset\), since \(S \subseteq I\) and \(\{J, K\}\) is a partition of \(I\), we know that \(S \cap (J \cup K) \neq \emptyset\).
    Thus we have \(S \in \mathbf{P}_J\) or \(S \in \mathbf{P}_K\), which means \(\mathbf{P} \subseteq \mathbf{P}_J \cup \mathbf{P}_K\).
    By Proposition \ref{3.1.18} we have \(\mathbf{P} = \mathbf{P}_J \cup \mathbf{P}_K\).
    Thus we have
    \begin{align*}
        p.c. \int_J f|_J + p.c. \int_K f|_K & = p.c. \int_{[\mathbf{P}_J]} f|_J + p.c. \int_{[\mathbf{P}_K]} f|_K               & \text{(by Definition \ref{11.2.14})}    \\
                                            & = \sum_{S \in \mathbf{P}_J} c_S \abs*{S} + \sum_{S \in \mathbf{P}_K} c_S \abs*{S} & \text{(by Proposition \ref{7.1.11}(e))} \\
                                            & = \sum_{S \in \mathbf{P}_J \cup \mathbf{P}_K} c_S \abs*{S}                        & \text{(by Definition \ref{11.2.9})}     \\
                                            & = \sum_{S \in \mathbf{P}} c_S \abs*{S}                                                                                      \\
                                            & = p.c. \int_{[\mathbf{P}]} f                                                      & \text{(by Definition \ref{11.2.9})}     \\
                                            & = p.c. \int_I f.                                                                  & \text{(by Definition \ref{11.2.14})}
    \end{align*}
\end{proof}
\section{Upper and lower Riemann integrals}\label{sec 11.3}

\begin{definition}[Majorization of functions]\label{11.3.1}
    Let \(f : I \to \mathbf{R}\) and \(g : I \to \mathbf{R}\).
    We say that \(g\) \emph{majorizes} \(f\) on \(I\) if we have \(g(x) \geq f(x)\) for all \(x \in I\), and that \(g\) \emph{minorizes} \(f\) on \(I\) if \(g(x) \leq f(x)\) for all \(x \in I\).
\end{definition}

\begin{definition}[Upper and lower Riemann integrals]\label{11.3.2}
    Let \(f : I \to \mathbf{R}\) be a bounded function defined on a bounded interval \(I\).
    We define the \emph{upper Riemann integral} \(\overline{\int}_I f\) by the formula
    \[
        \overline{\int}_I f \coloneqq \inf\{p.c. \int_I g : g \text{ is a piecewise constant function on \(I\) which majorizes } f\}
    \]
    and the \emph{lower Riemann integral} \(\underline{\int}_I f\) by the formula
    \[
        \underline{\int}_I f \coloneqq \sup\{p.c. \int_I g : g \text{ is a piecewise constant function on \(I\) which minorizes } f\}.
    \]
\end{definition}

\begin{lemma}\label{11.3.3}
    Let \(f : I \to \mathbf{R}\) be a function on a bounded interval \(I\) which is bounded by some real number \(M\), i.e., \(-M \leq f(x) \leq M\) for all \(x \in I\).
    Then we have
    \[
        -M \abs*{I} \leq \underline{\int}_I f \leq \overline{\int}_I f \leq M \abs*{I}.
    \]
    in particular, both the lower and upper Riemann integrals are real numbers (i.e., they are not infinite).
\end{lemma}

\begin{proof}
    The function \(g : I \to \mathbf{R}\) defined by \(g(x) = M\) is constant, hence piecewise constant, and majorizes \(f\);
    thus \(\overline{\int}_I f \leq p.c. \int_I g = M \abs*{I}\) by definition of the upper Riemann integral.
    A similar argument gives \(-M \abs*{I} \leq \underline{\int}_I f\).
    Finally, we have to show that \(\underline{\int}_I f \leq \overline{\int}_I f\).
    Let \(g\) be any piecewise constant function majorizing \(f\), and let \(h\) be any piecewise constant function minorizing \(f\).
    Then \(g\) majorizes \(h\), and hence \(p.c. \int_I h \leq p.c. \int_I g\).
    Taking suprema in \(h\), we obtain that \(\underline{\int}_I f \leq p.c. \int_I g\).
    Taking infima in \(g\), we thus obtain \(\underline{\int}_I f \leq \overline{\int}_I f\), as desired.
\end{proof}

\begin{definition}[Riemann integral]\label{11.3.4}
    Let \(f : I \to \mathbf{R}\) be a bounded function on a bounded interval \(I\).
    If \(\underline{\int}_I f = \overline{\int}_I f\), then we say that \(f\) is \emph{Riemann integrable on \(I\)} and define
    \[
        \int_I f \coloneqq \underline{\int}_I f = \overline{\int}_I f.
    \]
    If the upper and lower Riemann integrals are unequal, we say that \(f\) is not Riemann integrable.
\end{definition}

\begin{remark}\label{11.3.5}
    Compare this definition to the relationship between the \(\limsup\), \(liminf\), and limit of a sequence \(a_n\) that was established in Proposition \ref{6.4.12}(f);
    the \(\limsup\) is always greater than or equal to the \(\liminf\), but they are only equal when the sequence converges, and in this case they are both equal to the limit of the sequence.
    The definition given above may differ from the definition you may have encountered in your calculus courses, based on Riemann sums.
    However, the two definitions turn out to be equivalent.
\end{remark}

\begin{remark}\label{11.3.6}
    Note that we do not consider unbounded functions to be Riemann integrable;
    an integral involving such functions is known as an \emph{improper integral}.
    It is possible to still evaluate such integrals using more sophisticated integration methods (such as the Lebesgue integral).
\end{remark}

\begin{lemma}\label{11.3.7}
    Let \(f : I \to \mathbf{R}\) be a piecewise constant function on a bounded interval \(I\).
    Then \(f\) is Riemann integrable, and \(\int_I f = p.c. \int_I f\).
\end{lemma}

\begin{proof}
    Since \(f(x) \leq f(x)\), by Definition \ref{11.3.2} we have
    \[
        \overline{\int}_I f \leq p.c. \int_I f
    \]
    and
    \[
        p.c. \int_I f \leq \underline{\int}_I f.
    \]
    By Lemma \ref{11.3.3} we know that
    \[
        p.c. \int_I f \leq \underline{\int}_I f \leq \overline{\int}_I f \leq p.c. \int_I f.
    \]
    Thus by Definition \ref{11.3.4} we have
    \[
        \int_I f = \underline{\int}_I f = \overline{\int}_I f = p.c. \int_I f.
    \]
\end{proof}

\begin{remark}\label{11.3.8}
    Because of this lemma, we will not refer to the piecewise constant integral \(p.c. \int_I\) again, and just use the Riemann integral \(\int_I\) throughout
    (until this integral is itself superceded by the Lebesgue integral).
    We observe one special case of Lemma \ref{11.3.7}:
    if \(I\) is a point or the empty set, then \(\int_I f = 0\) for all functions \(f : I \to \mathbf{R}\).
    (Note that all such functions are automatically constant.)
\end{remark}

\begin{definition}[Riemann sums]\label{11.3.9}
    Let \(f : I \to \mathbf{R}\) be a bounded function on a bounded interval \(I\), and let \(\mathbf{P}\) be a partition of \(I\).
    We define the \emph{upper Riemann sum} \(U(f, \mathbf{P})\) and the \emph{lower Riemann sum} \(L(f, \mathbf{P})\) by
    \[
        U(f, \mathbf{P}) \coloneqq \sum_{J \in \mathbf{P} : J \neq \emptyset} (\sup_{x \in J} f(x)) \abs*{J}
    \]
    and
    \[
        L(f, \mathbf{P}) \coloneqq \sum_{J \in \mathbf{P} : J \neq \emptyset} (\inf_{x \in J} f(x)) \abs*{J}
    \]
\end{definition}

\begin{remark}\label{11.3.10}
    The restriction \(J \neq \emptyset\) is required because the quantities \(\inf_{x \in J} f(x)\) and \(\sup_{x \in J} f(x)\) are infinite (or negative infinite) if \(J\) is empty.
\end{remark}
\section{Basic properties of the Riemann integral}\label{sec 11.4}

\begin{theorem}[Laws of Riemann integration]\label{11.4.1}
    Let \(I\) be a bounded interval, and let \(f : I \to \mathbf{R}\) and \(g : I \to \mathbf{R}\) be Riemann integrable functions on \(I\).
    \begin{enumerate}
        \item The function \(f + g\) is Riemann integrable, and we have \(\int_I (f + g) = \int_I f + \int_I g\).
        \item For any real number \(c\), the function \(cf\) is Riemann integrable, and we have \(\int_I (cf) = c(\int_I f)\).
        \item The function \(f - g\) is Riemann integrable, and we have \(\int_I (f - g) = \int_I f - \int_I g\).
        \item If \(f(x) \geq 0\) for all \(x \in I\), then \(\int_I f \geq 0\).
        \item If \(f(x) \geq g(x)\) for all \(x \in I\), then \(\int_I f \geq \int_I g\).
        \item If \(f\) is the constant function \(f(x) = c\) for all \(x \in I\), then \(\int_I f = c \abs*{I}\).
        \item Let \(J\) be a bounded interval containing \(I\) (i.e., \(I \subseteq J\)), and let \(F : J \to \mathbf{R}\) be the function
              \[
                  F(x) \coloneqq \begin{cases}
                      f(x) & \text{if } x \in I    \\
                      0    & \text{if } x \notin I \\
                  \end{cases}
              \]
              Then \(F\) is Riemann integrable on \(J\), and \(\int_J F = \int_I f\).
        \item Suppose that \(\{J, K\}\) is a partition of \(I\) into two intervals \(J\) and \(K\).
              Then the functions \(f|_J : J \to \mathbf{R}\) and \(f|_K : K \to \mathbf{R}\) are Riemann integrable on \(J\) and \(K\) respectively, and we have
              \[
                  \int_I f = \int_J f|_J + \int_K f|_K.
              \]
    \end{enumerate}
\end{theorem}

\begin{proof}{(a)}
    Since \(f, g\) are Riemann integrable on \(I\), by Definition \ref{11.3.4} we have
    \[
        \int_I f = \overline{\int}_I f = \underline{\int}_I f
    \]
    and
    \[
        \int_I g = \overline{\int}_I g = \underline{\int}_I g.
    \]
    Let \(f_U : I \to \mathbf{R}\) and \(g_U : I \to \mathbf{R}\) be piecewise constant functions on \(I\) which majorizes \(f\) and \(g\), respectively.
    Let \(f_L : I \to \mathbf{R}\) and \(g_L : I \to \mathbf{R}\) be piecewise constant functions on \(I\) which minorizes \(f\) and \(g\), respectively.
    \(f_U, g_U, f_L, g_L\) are well-defined since by Definition \ref{11.3.4} \(f, g\) are bounded functions on a bounded interval \(I\).
    By Definition \ref{11.3.2} we have
    \[
        p.c. \int_I f_L \leq \underline{\int}_I f = \int_I f = \overline{\int}_I f \leq p.c. \int_I f_U
    \]
    and
    \[
        p.c. \int_I g_L \leq \underline{\int}_I g = \int_I g = \overline{\int}_I g \leq p.c. \int_I g_U.
    \]
    By Definition \ref{11.3.4} both \(f, g\) are bounded functions, so \(f + g\) is bounded function, and \(\underline{\int}_I (f + g), \overline{\int}_I (f + g)\) are well-defined (by Definition \ref{11.3.2}).
    By Exercise \ref{ex 11.3.2} we know that \(f_U + g_U\) majorizes \(f + g_U\) and \(f + g_U\) majorizes \(f + g\), thus \(f_U + g_U\) majorizes \(f + g\).
    Similarly \(f_L + g_L\) minorizes \(f + g\).
    Then we have
    \begin{align*}
                 & \overline{\int}_I (f + g) \leq p.c. \int_I (f_U + g_U)                   & \text{(by Definition \ref{11.3.2})}     \\
        \implies & \overline{\int}_I (f + g) \leq p.c. \int_I f_U + p.c. \int_I g_U         & \text{(by Theorem \ref{11.2.16}(a))}    \\
        \implies & \overline{\int}_I (f + g) - p.c. \int_I g_U \leq p.c. \int_I f_U         & \text{(note that \(f_U\) is arbitrary)} \\
        \implies & \overline{\int}_I (f + g) - p.c. \int_I g_U \leq \overline{\int}_I f     & \text{(by Definition \ref{11.3.2})}     \\
        \implies & \overline{\int}_I (f + g) - \overline{\int}_I f \leq p.c. \int_I g_U     & \text{(note that \(g_U\) is arbitrary)} \\
        \implies & \overline{\int}_I (f + g) - \overline{\int}_I f \leq \overline{\int}_I g & \text{(by Definition \ref{11.3.2})}     \\
        \implies & \overline{\int}_I (f + g) \leq \overline{\int}_I f + \overline{\int}_I g &                                         \\
        \implies & \overline{\int}_I (f + g) \leq \int_I f + \int_I g                       & \text{(by Definition \ref{11.3.4})}
    \end{align*}
    and
    \begin{align*}
                 & \underline{\int}_I (f + g) \geq p.c. \int_I (f_L + g_L)                     & \text{(by Definition \ref{11.3.2})}     \\
        \implies & \underline{\int}_I (f + g) \geq p.c. \int_I f_L + p.c. \int_I g_L           & \text{(by Theorem \ref{11.2.16}(a))}    \\
        \implies & \underline{\int}_I (f + g) - p.c. \int_I g_L \geq p.c. \int_I f_L           & \text{(note that \(f_L\) is arbitrary)} \\
        \implies & \underline{\int}_I (f + g) - p.c. \int_I g_L \geq \underline{\int}_I f      & \text{(by Definition \ref{11.3.2})}     \\
        \implies & \underline{\int}_I (f + g) - \underline{\int}_I f \geq p.c. \int_I g_L      & \text{(note that \(g_L\) is arbitrary)} \\
        \implies & \underline{\int}_I (f + g) - \underline{\int}_I f \geq \underline{\int}_I g & \text{(by Definition \ref{11.3.2})}     \\
        \implies & \underline{\int}_I (f + g) \geq \underline{\int}_I f + \underline{\int}_I g &                                         \\
        \implies & \underline{\int}_I (f + g) \geq \int_I f + \int_I g.                        & \text{(by Definition \ref{11.3.4})}
    \end{align*}
    By Lemma \ref{11.3.3} we have
    \[
        \int_I f + \int_I g \leq \underline{\int}_I (f + g) \leq \overline{\int}_I (f + g) \leq \int_I f + \int_I g
    \]
    and thus by Definition \ref{11.3.4} we have
    \[
        \int_I (f + g) = \underline{\int}_I (f + g) = \overline{\int}_I (f + g) = \int_I f + \int_I g.
    \]
\end{proof}

\begin{proof}{(b)}
    Since \(f\) is Riemann integrable on \(I\), by Definition \ref{11.3.4} we have
    \[
        \int_I f = \overline{\int}_I f = \underline{\int}_I f.
    \]
    First suppose that \(c = 0\).
    Then we have \((cf)(x) = 0\) for all \(x \in 0\), thus we have
    \begin{align*}
        \int_I (cf) & = p.c. \int_I (cf) & \text{(by Lemma \ref{11.3.7})} \\
                    & = 0                                                 \\
                    & = c \int_I f.
    \end{align*}

    Next suppose that \(c > 0\).
    Let \(f_U : I \to \mathbf{R}\) be a piecewise constant function on \(I\) which majorizes \(f\).
    Let \(f_L : I \to \mathbf{R}\) be a piecewise constant function on \(I\) which minorizes \(f\).
    \(f_U, f_L\) are well-defined since by Definition \ref{11.3.4} \(f\) is a bounded function on a bounded interval \(I\).
    Then by Definition \ref{11.3.2} we have
    \[
        p.c. \int_I f_L \leq \underline{\int}_I f = \int_I f = \overline{\int}_I f \leq p.c. \int_I f_U.
    \]
    Since \(f\) is a bounded function, \(cf\) is also a bounded function, by Definition \ref{11.3.2} both \(\overline{\int}_I (cf), \underline{\int}_I (cf)\) are well-defined.
    Since \(c > 0\), by Definition \ref{11.3.1} we know that \(c f_U\) majorizes \(c f\) and \(c f_L\) minorizes \(c f\).
    Then we have
    \begin{align*}
                 & \overline{\int}_I (cf) \leq p.c. \int_I (c f_U)                         & \text{(by Definition \ref{11.3.2})}     \\
        \implies & \overline{\int}_I (cf) \leq c \bigg(p.c. \int_I f_U\bigg)               & \text{(by Theorem \ref{11.2.16}(b))}    \\
        \implies & \frac{1}{c} \bigg(\overline{\int}_I (cf)\bigg) \leq p.c. \int_I f_U     & \text{(note that \(f_U\) is arbitrary)} \\
        \implies & \frac{1}{c} \bigg(\overline{\int}_I (cf)\bigg) \leq \overline{\int}_I f & \text{(by Definition \ref{11.3.2})}     \\
        \implies & \overline{\int}_I (cf) \leq c\bigg(\overline{\int}_I f\bigg)                                                      \\
        \implies & \overline{\int}_I (cf) \leq c\bigg(\int_I f\bigg)                       & \text{(by Definition \ref{11.3.4})}
    \end{align*}
    and
    \begin{align*}
                 & \underline{\int}_I (cf) \geq p.c. \int_I (c f_L)                          & \text{(by Definition \ref{11.3.2})}     \\
        \implies & \underline{\int}_I (cf) \geq c \bigg(p.c. \int_I f_L\bigg)                & \text{(by Theorem \ref{11.2.16}(b))}    \\
        \implies & \frac{1}{c} \bigg(\underline{\int}_I (cf)\bigg) \geq p.c. \int_I f_L      & \text{(note that \(f_L\) is arbitrary)} \\
        \implies & \frac{1}{c} \bigg(\underline{\int}_I (cf)\bigg) \geq \underline{\int}_I f & \text{(by Definition \ref{11.3.2})}     \\
        \implies & \underline{\int}_I (cf) \geq c\bigg(\underline{\int}_I f\bigg)                                                      \\
        \implies & \underline{\int}_I (cf) \geq c\bigg(\int_I f\bigg).                       & \text{(by Definition \ref{11.3.4})}
    \end{align*}
    By Lemma \ref{11.3.3} we have
    \[
        c\bigg(\int_I f\bigg) \leq \underline{\int}_I (cf) \leq \overline{\int}_I (cf) \leq c\bigg(\int_I f\bigg)
    \]
    and thus by Definition \ref{11.3.4} we have
    \[
        \int_I (cf) = \underline{\int}_I (cf) = \overline{\int}_I (cf) = c\bigg(\int_I f\bigg).
    \]

    Finally suppose that \(c < 0\).
    Using the same definition of \(f_U, f_L\) we have
    \begin{align*}
                 & \overline{\int}_I (cf) \leq p.c. \int_I (c f_U)                                              & \text{(by Definition \ref{11.3.2})}  \\
        \implies & \overline{\int}_I (cf) \leq c \bigg(p.c. \int_I f_U\bigg)                                    & \text{(by Theorem \ref{11.2.16}(b))} \\
        \implies & \frac{1}{c} \bigg(\overline{\int}_I (cf)\bigg) \geq p.c. \int_I f_U                                                                 \\
        \implies & \frac{1}{c} \bigg(\overline{\int}_I (cf)\bigg) \geq p.c. \int_I f_U \geq \overline{\int}_I f & \text{(by Definition \ref{11.3.2})}  \\
        \implies & \overline{\int}_I (cf) \leq c\bigg(\overline{\int}_I f\bigg)                                                                        \\
        \implies & \overline{\int}_I (cf) \leq c\bigg(\int_I f\bigg)                                            & \text{(by Definition \ref{11.3.4})}
    \end{align*}
    and
    \begin{align*}
                 & \underline{\int}_I (cf) \geq p.c. \int_I (c f_L)                                               & \text{(by Definition \ref{11.3.2})}  \\
        \implies & \underline{\int}_I (cf) \geq c \bigg(p.c. \int_I f_L\bigg)                                     & \text{(by Theorem \ref{11.2.16}(b))} \\
        \implies & \frac{1}{c} \bigg(\underline{\int}_I (cf)\bigg) \leq p.c. \int_I f_L                                                                  \\
        \implies & \frac{1}{c} \bigg(\underline{\int}_I (cf)\bigg) \leq p.c. \int_I f_L \leq \underline{\int}_I f & \text{(by Definition \ref{11.3.2})}  \\
        \implies & \underline{\int}_I (cf) \geq c\bigg(\underline{\int}_I f\bigg)                                                                        \\
        \implies & \underline{\int}_I (cf) \geq c\bigg(\int_I f\bigg).                                            & \text{(by Definition \ref{11.3.4})}
    \end{align*}
    By Lemma \ref{11.3.3} we have
    \[
        c\bigg(\int_I f\bigg) \leq \underline{\int}_I (cf) \leq \overline{\int}_I (cf) \leq c\bigg(\int_I f\bigg)
    \]
    and thus by Definition \ref{11.3.4} we have
    \[
        \int_I (cf) = \underline{\int}_I (cf) = \overline{\int}_I (cf) = c\bigg(\int_I f\bigg).
    \]
    We conclude that \(\forall c \in \mathbf{R}\), \(\int_I (cf) = c (\int_I f)\).
\end{proof}

\begin{proof}{(c)}
    We have
    \begin{align*}
        \int_I f - \int_I g & = \int_I f + \int_I (-g)    & \text{(by Theorem \ref{11.4.1}(b))} \\
                            & = \int_I \big(f + (-g)\big) & \text{(by Theorem \ref{11.4.1}(a))} \\
                            & = \int_I (f - g).           & \text{(by Definition \ref{9.2.1})}
    \end{align*}
\end{proof}

\begin{proof}{(d)}
    Let \(f_U : I \to \mathbf{R}\) be a piecewise constant function on \(I\) which majorizes \(f\).
    \(f_U\) is well-defined since by Definition \ref{11.3.4} \(f\) is a bounded function on a bounded interval \(I\).
    Since \(0 \leq f(x) \leq f_U(x)\) for every \(x \in I\), we have
    \begin{align*}
                 & 0 \leq p.c. \int_I f_U     & \text{(by Theorem \ref{11.2.16}(d))} \\
        \implies & 0 \leq \overline{\int}_I f & \text{(by Definition \ref{11.3.2})}  \\
        \implies & 0 \leq \int_I f.           & \text{(by Definition \ref{11.3.4})}
    \end{align*}
\end{proof}

\begin{proof}{(e)}
    We have \(f(x) - g(x) \geq 0\) for every \(x \in I\) and by Theorem \ref{11.4.1}(c) \(f - g\) is Riemann integrable on \(I\).
    Thus
    \begin{align*}
                 & \int_I (f - g) \geq 0      & \text{(by Theorem \ref{11.4.1}(d))} \\
        \implies & \int_I f - \int_I g \geq 0 & \text{(by Theorem \ref{11.4.1}(c))} \\
        \implies & \int_I f \geq \int_I g.
    \end{align*}
\end{proof}

\begin{proof}{(f)}
    We have
    \begin{align*}
        \int_I f & = p.c. \int_I f & \text{(by Lemma \ref{11.3.7})}       \\
                 & = c \abs*{I}.   & \text{(by Theorem \ref{11.2.16}(f))}
    \end{align*}
\end{proof}

\begin{proof}{(g)}
    Let \(f_U : I \to \mathbf{R}\) be a piecewise constant function on \(I\) which majorizes \(f\).
    Let \(f_L : I \to \mathbf{R}\) be a piecewise constant function on \(I\) which minorizes \(f\).
    \(f_U, f_L\) are well-defined since by Definition \ref{11.3.4} \(f\) is a bounded function on a bounded interval \(I\).
    Then by Definition \ref{11.3.2} we have
    \[
        p.c. \int_I f_L \leq \underline{\int}_I f = \int_I f = \overline{\int}_I f \leq p.c. \int_I f_U.
    \]
    Let \(F_U : J \to \mathbf{R}\) be the function
    \[
        F_U(x) = \begin{cases}
            f_U(x) & \text{if } x \in I    \\
            0      & \text{if } x \notin I
        \end{cases}
    \]
    and let \(F_L : J \to \mathbf{R}\) be the function
    \[
        F_L(x) = \begin{cases}
            f_L(x) & \text{if } x \in I     \\
            0      & \text{if } x \notin I.
        \end{cases}
    \]
    We know that \(F_U\) majorizes \(F\) and \(F_L\) minorizes \(F\), and by Theorem \ref{11.2.16}(g) we have \(p.c. \int_J F_U = p.c. \int_I f_U\) and \(p.c. \int_J F_L = p.c. \int_I f_L\).
    Thus \(F\) is a bounded function on a bounded interval \(I\), and we have
    \begin{align*}
                 & \overline{\int}_J F \leq p.c. \int_J F_U     & \text{(by Definition \ref{11.3.2})}                          \\
        \implies & \overline{\int}_J F \leq p.c. \int_I f_U     & \text{(by Theorem \ref{11.2.16}(g))}                         \\
        \implies & \overline{\int}_J F \leq \overline{\int}_I f & \text{(by Definition \ref{11.3.2} and \(f_U\) is arbitrary)} \\
        \implies & \overline{\int}_J F \leq \int_I f            & \text{(by Definition \ref{11.3.4})}
    \end{align*}
    and
    \begin{align*}
                 & \underline{\int}_J F \geq p.c. \int_J F_L      & \text{(by Definition \ref{11.3.2})}                          \\
        \implies & \underline{\int}_J F \geq p.c. \int_I f_L      & \text{(by Theorem \ref{11.2.16}(g))}                         \\
        \implies & \underline{\int}_J F \geq \underline{\int}_I f & \text{(by Definition \ref{11.3.2} and \(f_L\) is arbitrary)} \\
        \implies & \underline{\int}_J F \geq \int_I f.            & \text{(by Definition \ref{11.3.4})}
    \end{align*}
    By Lemma \ref{11.3.3} we have
    \[
        \int_I f \leq \underline{\int}_J F \leq \overline{\int}_J F \leq \int_I f
    \]
    and thus by Definition \ref{11.3.4} we have
    \[
        \int_J F = \underline{\int}_J F = \overline{\int}_J F = \int_I f.
    \]
\end{proof}

\begin{proof}{(h)}
    Let \(f_U : I \to \mathbf{R}\) be a piecewise constant function on \(I\) which majorizes \(f\).
    Let \(f_L : I \to \mathbf{R}\) be a piecewise constant function on \(I\) which minorizes \(f\).
    \(f_U, f_L\) are well-defined since by Definition \ref{11.3.4} \(f\) is a bounded function on a bounded interval \(I\).
    Then by Definition \ref{11.3.2} we have
    \[
        p.c. \int_I f_L \leq \underline{\int}_I f = \int_I f = \overline{\int}_I f \leq p.c. \int_I f_U.
    \]
    By Theorem \ref{11.2.16}(h) we know that \(f_U|_J : J \to \mathbf{R}\), \(f_L|_J : J \to \mathbf{R}\) are piecewise constant function on \(J\) and \(f_U|_K : K \to \mathbf{R}\), \(f_L|_K : K \to \mathbf{R}\) are piecewise constant functions on \(K\).
    By Definition \ref{11.3.1} we know that \(f_U|_J\) majorizes \(f|_J\) and \(f_L|_J\) minorizes \(f|_J\), similarly \(f_U|_K\) majorizes \(f|_K\) and \(f_L|_K\) minorizes \(f|_K\).
    Thus \(f|_J\), \(f|_K\) are bounded functions on bounded intervals \(J, K\), respectively.
    So \(\overline{\int}_J f|_J\), \(\overline{\int}_K f|_K\), \(\underline{\int}_J f|_J\), \(\underline{\int}_K f|_K\) are well-defined.
    Then we have
    \begin{align*}
                 & \overline{\int}_J f|_J + \overline{\int}_K f|_K \leq p.c. \int_J f_U|_J + p.c. \int_K f_U|_K & \text{(by Definition \ref{11.3.2})}  \\
        \implies & \overline{\int}_J f|_J + \overline{\int}_K f|_K \leq p.c. \int_I f_U                         & \text{(by Theorem \ref{11.2.16}(h))} \\
        \implies & \overline{\int}_J f|_J + \overline{\int}_K f|_K \leq \overline{\int}_I f                     & \text{(by Definition \ref{11.3.2})}  \\
        \implies & \overline{\int}_J f|_J + \overline{\int}_K f|_K \leq \int_I f                                & \text{(by Definition \ref{11.3.4})}
    \end{align*}
    and
    \begin{align*}
                 & \underline{\int}_J f|_J + \underline{\int}_K f|_K \geq p.c. \int_J f_L|_J + p.c. \int_K f_L|_K & \text{(by Definition \ref{11.3.2})}  \\
        \implies & \underline{\int}_J f|_J + \underline{\int}_K f|_K \geq p.c. \int_I f_L                         & \text{(by Theorem \ref{11.2.16}(h))} \\
        \implies & \underline{\int}_J f|_J + \underline{\int}_K f|_K \geq \underline{\int}_I f                    & \text{(by Definition \ref{11.3.2})}  \\
        \implies & \underline{\int}_J f|_J + \underline{\int}_K f|_K \geq \int_I f.                               & \text{(by Definition \ref{11.3.4})}
    \end{align*}
    By Lemma \ref{11.3.3} we have
    \[
        \int_I f \leq \underline{\int}_J f|_J + \underline{\int}_K f|_K \leq \overline{\int}_J f|_J + \overline{\int}_K f|_K \leq \int_I f
    \]
    and thus we have
    \[
        \underline{\int}_J f|_J + \underline{\int}_K f|_K = \overline{\int}_J f|_J + \overline{\int}_J f|_K = \int_I f.
    \]
    Since
    \begin{align*}
                 & \underline{\int}_J f|_J + \underline{\int}_K f|_K = \overline{\int}_J f|_J + \overline{\int}_J f|_K                                                \\
        \implies & 0 \geq \underline{\int}_J f|_J - \overline{\int}_J f|_J = \overline{\int}_J f|_K - \underline{\int}_K f|_K \geq 0 & \text{(by Lemma \ref{11.3.3})} \\
        \implies & \underline{\int}_J f|_J - \overline{\int}_J f|_J = \overline{\int}_J f|_K - \underline{\int}_K f|_K = 0,
    \end{align*}
    by Definition \ref{11.3.4} we have
    \begin{align*}
         & \int_J f|_J = \underline{\int}_J f|_J = \overline{\int}_J f|_J, \\
         & \int_K f|_K = \underline{\int}_K f|_K = \overline{\int}_K f|_K, \\
         & \int_J f|_J + \int_K f|_K = \int_I f.
    \end{align*}
\end{proof}

\begin{remark}\label{11.4.2}
    We often abbreviate \(\int_J f|_J\) as \(\int_J f\) even though \(f\) is really defined on a larger domain than just \(J\).
    We also observe from Theorem \ref{11.4.1}(h) and Remark \ref{11.3.8} that if \(f : [a, b] \to \mathbf{R}\) is Riemann integrable on a closed interval \([a, b]\), then \(\int_{[a, b]} f = \int_{(a, b]} f = \int_{[a, b)} f = \int_{(a, b)} f\).
\end{remark}

\begin{theorem}[Max and min preserve integrability]\label{11.4.3}
    Let \(I\) be a bounded interval, and let \(f : I \to \mathbf{R}\) and \(g : I \to \mathbf{R}\) be a Riemann integrable function.
    Then the functions \(\max(f, g) : I \to \mathbf{R}\) and \(\min(f, g) : I \to \mathbf{R}\) defined by \(\max(f, g)(x) \coloneqq \max\big(f(x), g(x)\big)\) and \(\min(f, g)(x) \coloneqq \min\big(f(x), g(x)\big)\) are also Riemann integrable.
\end{theorem}

\begin{proof}
    We shall just prove the claim for \(\max(f, g)\), the case of \(\min(f, g)\) being similar.
    First note that since \(f\) and \(g\) are bounded, then \(\max(f, g)\) is also bounded.

    Let \(\varepsilon > 0\).
    Since \(\int_I f = \underline{\int}_I f\), there exists a piecewise constant function \(\underline{f} : I \to \mathbf{R}\) which minorizes \(f\) on \(I\) such that
    \[
        \int_I \underline{f} \geq \int_I f - \varepsilon.
    \]
    Similarly we can find a piecewise constant \(g : I \to \mathbf{R}\) which minorizes \(g\) on \(I\) such that
    \[
        \int_I \underline{g} \geq \int_I g - \varepsilon,
    \]
    and we can find piecewise functions \(\overline{f}, \overline{g}\) which majorize \(f, g\) respectively on \(I\) such that
    \[
        \int_I \overline{f} \leq \int_I f + \varepsilon
    \]
    and
    \[
        \int_I \overline{g} \leq \int_I g + \varepsilon.
    \]
    In particular, if \(h : I \to \mathbf{R}\) denotes the function
    \[
        h \coloneqq (\overline{f} - \underline{f}) + (\overline{g} - \underline{g})
    \]
    we have
    \[
        \int_I h \leq 4 \varepsilon.
    \]
    On the other hand, \(\max(\underline{f}, \underline{g})\) is a piecewise constant function on \(I\) which minorizes \(\max(f, g)\), while \(\max(\overline{f}, \overline{g})\) is similarly a piecewise constant function on \(I\) which majorizes \(\max(f, g)\).
    Thus
    \[
        \int_I \max(\underline{f}, \underline{g}) \leq \underline{\int}_I \max(f, g) \leq \overline{\int}_I \max(f, g) \leq \int_I \max(\overline{f}, \overline{g}),
    \]
    and so
    \[
        0 \leq \overline{\int}_I \max(f, g) - \underline{\int}_I \max(f, g) \leq \int_I \max(\overline{f}, \overline{g}) - \max(\underline{f}, \underline{g}).
    \]
    But we have
    \[
        \overline{f}(x) = \underline{f}(x) + (\overline{f} - \underline{f})(x) \leq \underline{f}(x) + h(x)
    \]
    and similarly
    \[
        \overline{g}(x) = \underline{g}(x) + (\overline{g} - \underline{g})(x) \leq \underline{g}(x) + h(x)
    \]
    and thus
    \[
        \max\big(\overline{f}(x), \overline{g}(x)\big) \leq \max\big(\underline{f}(x), \underline{g}(x)\big) + h(x).
    \]
    Inserting this into the previous inequality, we obtain
    \[
        0 \leq \overline{\int}_I \max(f, g) - \underline{\int}_I \max(f, g) \leq \int_I h \leq 4 \varepsilon.
    \]
    To summarize, we have shown that
    \[
        0 \leq \overline{\int}_I \max(f, g) - \underline{\int}_I \max(f, g) \leq 4 \varepsilon
    \]
    for every \(\varepsilon\).
    Since \(\overline{\int}_I \max(f, g) - \underline{\int}_I \max(f, g)\) does not depend on \(\varepsilon\), we thus see that
    \[
        \overline{\int}_I \max(f, g) - \underline{\int}_I \max(f, g) = 0
    \]
    and hence that \(\max(f, g)\) is Riemann integrable.
\end{proof}

\begin{corollary}[Absolute values preserve Riemann integrability]\label{11.4.4}
    \quad
    Let \(I\) be a bounded interval.
    If \(f : I \to \mathbf{R}\) is a Riemann integrable function, then the positive part \(f_+ \coloneqq \max(f, 0)\) and the negative part \(f_- \coloneqq \min(f, 0)\) are also Riemann integrable on \(I\).
    Also, the absolute value \(\abs*{f}\), defined by \(\abs*{f}(x) = \abs*{f(x)}\) is also Riemann integrable on \(I\).
    (observe that \(\abs*{f} = f_+ - f_-\))
\end{corollary}

\begin{proof}
    By Theorem \ref{11.4.3} we know that \(f_+, f_-\) are Riemann integrable.
    Since \(\abs*{f} = f_+ - f_-\), by Theorem \ref{11.4.1}(a) we know that \(\abs*{f}\) is Riemann integrable.
\end{proof}

\begin{theorem}[Products preserve Riemann integrability]\label{11.4.5}
    Let \(I\) be a bounded interval.
    If \(f : I \to \mathbf{R}\) and \(g : I \to \mathbf{R}\) are Riemann integrable, then \(fg : I \to \mathbf{R}\) is also Riemann integrable.
\end{theorem}

\begin{proof}
    We split \(f = f_+ + f_-\) and \(g = g_+ + g_-\) into positive and negative parts;
    by Corollary \ref{11.4.4}, the functions \(f_+\), \(f_-\), \(g_+\), \(g_-\) are Riemann integrable.
    Since
    \[
        fg = f_+ g_+ + f_+ g_- + f_- g_+ + f_- g_-
    \]
    then it suffices to show that the functions \(f_+ g_+\), \(f_+ g_-\), \(f_- g_+\), \(f_- g_-\) are individually Riemann integrable.
    We will just show this for \(f_+ g_+\);
    the other three are similar.

    Since \(f_+\) and \(g_+\) are bounded and positive, there are \(M_1, M_2 > 0\) such that
    \[
        0 \leq f_+(x) \leq M_1 \text{ and } 0 \leq g_+(x) \leq M_2
    \]
    for all \(x \in I\).
    Now let \(\varepsilon > 0\) be arbitrary.
    Then, as in the proof of Theorem \ref{11.4.3}, we can find a piecewise constant function \(\underline{f_+}\) minorizing \(f_+\) on \(I\), and a piecewise constant function \(\overline{f_+}\) majorizing \(f_+\) on \(I\), such that
    \[
        \int_I \overline{f_+} \leq \int_I f_+ + \varepsilon
    \]
    and
    \[
        \int_I \underline{f_+} \geq \int_I f_+ - \varepsilon.
    \]
    Note that \(\underline{f_+}\) may be negative at places, but we can fix this by replacing \(\underline{f_+}\) by \(\max(\underline{f_+}, 0)\), since this still minorizes \(f_+\) and still has integral greater than or equal to \(\int_I f_+ - \varepsilon\).
    So without loss of generality we may assume that \(\underline{f_+}(x) \geq 0\) for all \(x \in I\).
    Similarly we may assume that \(\overline{f_+}(x) \leq M_1\) for all \(x \in I\);
    thus
    \[
        0 \leq \underline{f_+}(x) \leq f_+(x) \leq \overline{f_+}(x) \leq M_1
    \]
    for all \(x \in I\).

    Similar reasoning allows us to find piecewise constant \(\underline{g_+}\) minorizing \(g_+\), and \(\overline{g_+}\) majorizing \(g_+\), such that
    \[
        \int_I \overline{g_+} \leq \int_I g_+ + \varepsilon
    \]
    and
    \[
        \int_I \underline{g_+} \geq \int_I g_+ - \varepsilon,
    \]
    and
    \[
        0 \leq \underline{g_+}(x) \leq g_+(x) \leq \overline{g_+}(x) \leq M_2
    \]
    for all \(x \in I\).

    Notice that \(\underline{f_+} \underline{g_+}\) is piecewise constant and minorizes \(f_+ g_+\), while \(\overline{f_+} \overline{g_+}\) is piecewise constant and majorizes \(f_+ g_+\).
    Thus
    \[
        0 \leq \overline{\int}_I f_+ g_+ - \underline{\int}_I f_+ g_+ \leq \int_I \overline{f}_+ \overline{g_+} - \underline{f_+} \underline{g_+}.
    \]
    However, we have
    \begin{align*}
        \overline{f_+}(x) \overline{g_+}(x) - \underline{f_+}(x) \underline{g_+}(x) & = \overline{f_+}(x) (\overline{g_+} - \underline{g_+})(x) + \underline{g_+}(x) (\overline{f_+} - \underline{f_+})(x) \\
                                                                                    & \leq M_1 (\overline{g_+} - \underline{g_+})(x) + M_2 (\overline{f_+} - \underline{f_+})(x)
    \end{align*}
    for all \(x \in I\), and thus
    \begin{align*}
        0 \leq \overline{\int}_I f_+ g_+ - \underline{\int}_I f_+ g_+ & \leq M_1 \int_I (\overline{g_+} - \underline{g_+}) + M_2 \int_I (\overline{f_+} - \underline{f_+}) \\
                                                                      & \leq M_1 (2\varepsilon) + M_2 (2\varepsilon).
    \end{align*}
    Again, since \(\varepsilon\) was arbitrary, we can conclude that \(f_+ g_+\) is Riemann integrable, as before.
    Similar argument show that \(f_+ g_-\), \(f_- g_+\), \(f_- g_-\) are Riemann integrable;
    combining them we obtain that \(fg\) is Riemann integrable.
\end{proof}

\exercisesection

\begin{exercise}\label{ex 11.4.1}
    Prove Theorem \ref{11.4.1}.
\end{exercise}

\begin{proof}
    See Theorem \ref{11.4.1}.
\end{proof}

\begin{exercise}\label{ex 11.4.2}
    Let \(a < b\) be real numbers, and let \(f : [a, b] \to \mathbf{R}\) be a continuous, non-negative function
    (so \(f(x) \geq 0\) for all \(x \in [a, b]\)).
    Suppose that \(\int_{[a, b]} f = 0\).
    Show that \(f(x) = 0\) for all \(x \in [a, b]\).
\end{exercise}

\begin{proof}
    Suppose for sake of contradiction that \(\exists\ x_0 \in [a, b]\) such that \(f(x_0) > 0\).
    Since \(f\) is continuous, by Proposition \ref{9.4.7} we have
    \[
        \forall \varepsilon \in \mathbf{R}^+, \exists\ \delta \in \mathbf{R}^+ : \big(\forall x \in [a, b], \abs*{x - x_0} < \delta \implies \abs*{f(x) - f(x_0)} \leq \varepsilon\big),
    \]
    or equivalently
    \[
        \forall \varepsilon \in \mathbf{R}^+, \exists\ \delta \in \mathbf{R}^+ : \big(\forall x \in [a, b] \cap (x_0 - \delta, x_0 + \delta) \implies \abs*{f(x) - f(x_0)} \leq \varepsilon\big).
    \]
    In particular, we have
    \[
        \exists\ \delta \in \mathbf{R}^+ : \bigg(\forall x \in [a, b] \cap (x_0 - \delta, x_0 + \delta) \implies \abs*{f(x) - f(x_0)} \leq \frac{f(x_0)}{2}\bigg),
    \]
    or equivalently
    \[
        \exists\ \delta \in \mathbf{R}^+ : \bigg(\forall x \in [a, b] \cap (x_0 - \delta, x_0 + \delta) \implies \frac{f(x_0)}{2} \leq f(x) \leq \frac{3 f(x_0)}{2}\bigg).
    \]
    Since \(\delta \neq 0\), we know that \([a, b] \cap (x_0 - \delta, x_0 + \delta) \neq \emptyset\).
    Since \(a \neq b\), we know that
    \[
        \sup\big([a, b] \cap (x_0 - \delta, x_0 + \delta)\big) \neq \inf\big([a, b] \cap (x_0 - \delta, x_0 + \delta)\big).
    \]
    Thus by Definition \ref{11.1.8} we have \(\abs*{[a, b] \cap (x_0 - \delta, x_0 + \delta)} > 0\).
    By Corollary \ref{11.1.6} we know that \([a, b] \cap (x_0 - \delta, x_0 + \delta)\) is a bounded interval.
    Let \(f_L : [a, b] \to \mathbf{R}\) be the function
    \[
        f_L(x) = \begin{cases}
            \frac{f(x_0)}{2} & \text{if } x \in [a, b] \cap (x_0 - \delta, x_0 + \delta)    \\
            0                & \text{if } x \notin [a, b] \cap (x_0 - \delta, x_0 + \delta)
        \end{cases}
    \]
    Since \(f(x) \geq 0\) for all \(x \in [a, b]\), we know that \(f_L\) minorizes \(f\).
    By Theorem \ref{11.2.16}(g) we know that \(f_L\) is a piecewise constant function.
    By Lemma \ref{11.3.7} we have
    \[
        \int_{[a, b]} f_L = p.c. \int_{[a, b]} f_L = \frac{f(x_0)}{2}\abs*{[a, b] \cap (x_0 - \delta, x_0 + \delta)} > 0.
    \]
    But by Definition \ref{11.3.2} and Definition \ref{11.3.4} we have
    \[
        0 < \int_{[a, b]} f_L \leq \underline{\int}_{[a, b]} f = \int_{[a, b]} f = 0,
    \]
    a contradiction.
    Thus we must have \(f(x) = 0\) for all \(x \in [a, b]\).
\end{proof}

\begin{exercise}\label{ex 11.4.3}
    Let \(I\) be a bounded interval, let \(f : I \to \mathbf{R}\) be a Riemann integrable function, and let \(\mathbf{P}\) be a partition of \(I\).
    Show that
    \[
        \int_I f = \sum_{J \in \mathbf{P}} \int_J f|_J.
    \]
\end{exercise}

\begin{proof}
    Let \(P(n)\) be the statement ``\(\#(\mathbf{P}) = n\) and \(\int_I f = \sum_{J \in \mathbf{P}} \int_J f|_J\)''.
    We use induction on \(n\) to show that \(P(n)\) is true \(\forall n \in \mathbf{N}\).
    For \(n = 0\), we have \(\mathbf{P} = \emptyset\) and \(I = \emptyset\).
    Thus
    \begin{align*}
        p.c. \int_{[\emptyset]} f & = \sum_{J \in \emptyset} c_J \abs*{J} & \text{(by Definition \ref{11.2.9})}     \\
                                  & = 0                                   & \text{(by Proposition \ref{7.1.11}(a))} \\
                                  & = p.c. \int_{\emptyset} f             & \text{(by Definition \ref{11.2.14})}    \\
                                  & = \int_{\emptyset} f                  & \text{(by Lemma \ref{11.3.7})}          \\
                                  & = \sum_{J \in \emptyset} \int_J f|_J  & \text{(by Proposition \ref{7.1.11}(a))}
    \end{align*}
    and the base case holds.
    Suppose inductively that \(P(n)\) is true for some \(n \geq 0\).
    Then we need to show that \(P(n + 1)\) is true.
    Let \(K \in \mathbf{P}\) such that \(x < y\) for every \(x \in K\) and \(y \in I \setminus K\).
    Then \(\big\{K, \bigcup (\mathbf{P} \setminus \{K\})\big\}\) is a partition of \(I\), and
    \begin{align*}
        \int_I f & = \int_K f|_K + \int_{\bigcup (\mathbf{P} \setminus \{K\})} f|_{\bigcup (\mathbf{P} \setminus \{K\})} & \text{(by Theorem \ref{11.4.1}(h))}     \\
                 & = \int_K f|_K + \sum_{J \in \mathbf{P} \setminus \{K\}} \int_J f|_J                                   & \text{(by induction hypothesis)}        \\
                 & = \sum_{J \in \mathbf{P}} \int_J f|_J.                                                                & \text{(by Proposition \ref{7.1.11}(e))}
    \end{align*}
    This closes the induction.
\end{proof}

\begin{exercise}\label{ex 11.4.4}
    Without repeating all the computations in the above proofs, give a short explanation as to why the remaining cases of Theorem \ref{11.4.3} and Theorem \ref{11.4.5} follow automatically from the cases presented in the text.
\end{exercise}

\begin{proof}
    We first show that the remaining case of Theorem \ref{11.4.3} is true.
    By Theorem \ref{11.4.1}(b) \(-f\) and \(-g\) are Riemann integrable on \(I\).
    Since \(\max(-f, -g)\) is Riemann integrable and \(\min(f, g) = -\max(-f, -g)\), by Theorem \ref{11.4.1}(b) we know that \(\min(f, g)\) is Riemann integrable.

    Now we show that the remaining cases of Theorem \ref{11.4.5} are true.
    By Corollary \ref{11.4.4} \((-f)_+\) and \((-g)_+\) are Riemann integrable on \(I\).
    Since for any Riemann integrable functions \(p\) and \(q\), \(p_+ q_+\) are Riemann integrable (which is showed in the proof of Theorem \ref{11.4.5}), we have
    \begin{align*}
        f_+ g_- & = f_+ \cdot \big(\min(g, 0)\big)                      & \text{(by Corollary \ref{11.4.4})} \\
                & = f_+ \cdot \big(-\max(-g, 0)\big)                                                         \\
                & = f_+ \cdot \big(-(-g)_+\big)                         & \text{(by Corollary \ref{11.4.4})} \\
                & = -\big(f_+ \cdot (-g)_+\big)                         & \text{(by Definition \ref{9.2.1})} \\
        f_- g_+ & = \big(\min(f, 0)\big) \cdot g_+                      & \text{(by Corollary \ref{11.4.4})} \\
                & = \big(-\max(-f, 0)\big) \cdot g_+                                                         \\
                & = \big(-(-f)_+\big) \cdot g_+                         & \text{(by Corollary \ref{11.4.4})} \\
                & = -\big((-f)_+ \cdot g_+\big)                         & \text{(by Definition \ref{9.2.1})} \\
        f_- g_- & = \big(\min(f, 0)\big) \cdot \big(\min(g, 0)\big)     & \text{(by Corollary \ref{11.4.4})} \\
                & = \big(-\max(-f, 0)\big) \cdot \big(-\max(-g, 0)\big)                                      \\
                & = \big(-(-f)_+\big) \big(-(-g)_+\big)                 & \text{(by Corollary \ref{11.4.4})} \\
                & = (-f)_+ \cdot (-g)_+                                 & \text{(by Definition \ref{9.2.1})}
    \end{align*}
    and thus \(f_+ g_-\), \(f_- g_+\), \(f_- g_-\) are Riemann integrable.
\end{proof}
\section{Riemann integrability of continuous functions}\label{sec 11.5}

\begin{theorem}\label{11.5.1}
  Let \(I\) be a bounded interval, and let \(f\) be a function which is uniformly continuous on \(I\).
  Then \(f\) is Riemann integrable.
\end{theorem}

\begin{proof}
  From \cref{9.9.15} we see that \(f\) is bounded.
  Now we have to show that \(\underline{\int}_I f = \overline{\int}_I f\).

  If \(I\) is a point or the empty set then the theorem is trivial, so let us assume that \(I\) is one of the four intervals \([a, b]\), \((a, b)\), \((a, b]\), or \([a, b)\) for some real numbers \(a < b\).

      Let \(\varepsilon > 0\) be arbitrary.
      By uniform continuity, there exists a \(\delta > 0\) such that \(\abs{f(x) - f(y)} < \varepsilon\) whenever \(x, y \in I\) are such that \(\abs{x - y} < \delta\).
      By the Archimedean principle, there exists an integer \(N > 0\) such that \((b - a) / N < \delta\).

      Note that we can partition \(I\) into \(N\) intervals \(J_1, \dots, J_N\), each of length \((b - a) / N\).
      (How? One has to treat each of the cases \([a, b]\), \((a, b)\), \((a, b]\), \([a, b)\) slightly differently.)
  By \cref{11.3.12}, we thus have
  \[
    \overline{\int}_I f \leq \sum_{k = 1}^N \big(\sup_{x \in J_k} f(x)\big) \abs{J_k}
  \]
  and
  \[
    \underline{\int}_I f \geq \sum_{k = 1}^N \big(\inf_{x \in J_k} f(x)\big) \abs{J_k}
  \]
  so in particular
  \[
    \overline{\int}_I f - \underline{\int}_I f \leq \sum_{k = 1}^N \big(\sup_{x \in J_k} f(x) - \inf_{x \in J_k} f(x)\big) \abs{J_k}.
  \]
  However, we have \(\abs{f(x) - f(y)} < \varepsilon\) for all \(x, y \in J_k\), since \(\abs{J_k} = (b - a) / N < \delta\).
  In particular we have
  \[
    f(x) < f(y) + \varepsilon \text{ for all } x, y \in J_k.
  \]
  Taking suprema in \(x\), we obtain
  \[
    \sup_{x \in J_k} f(x) \leq f(y) + \varepsilon \text{ for all } y \in J_k,
  \]
  and then taking infima in \(y\) we obtain
  \[
    \sup_{x \in J_k} f(x) \leq \inf_{y \in J_k} f(y) + \varepsilon.
  \]
  Inserting this bound into our previous inequality, we obtain
  \[
    \overline{\int}_I f - \underline{\int}_I f \leq \sum_{k = 1}^N \varepsilon \abs{J_k},
  \]
  but by \cref{11.1.13} we thus have
  \[
    \overline{\int}_I f - \underline{\int}_I f \leq \varepsilon (b - a).
  \]
  But \(\varepsilon > 0\) was arbitrary, while \((b - a)\) is fixed.
  Thus \(\overline{\int}_I f - \underline{\int}_I f\) cannot be positive.
  By \cref{11.3.3} and the definition of Riemann integrability we thus have that \(f\) is Riemann integrable.
\end{proof}

\begin{corollary}\label{11.5.2}
  Let \([a, b]\) be a closed interval, and let \(f : [a, b] \to \R\) be continuous.
  Then \(f\) is Riemann integrable.
\end{corollary}

\begin{proof}
  Combining \cref{11.5.1} with \cref{9.9.16} we are done.
\end{proof}

\begin{note}
  Note that \cref{11.5.2} is not true if \([a, b]\) is replaced by any other sort of interval, since it is not even guaranteed then that continuous functions are bounded.
  For instance, the function \(f : (0, 1) \to \R\) defined by \(f(x) \coloneqq 1 / x\) is continuous but not Riemann integrable.
  However, if we assume that a function is both continuous \emph{and} bounded, we can recover Riemann integrability (see \cref{11.5.3}).
\end{note}

\begin{proposition}\label{11.5.3}
  Let \(I\) be a bounded interval, and let \(f : I \to \R\) be both continuous and bounded.
  Then \(f\) is Riemann integrable on \(I\).
\end{proposition}

\begin{proof}
  If \(I\) is a point or an empty set then the claim is trivial;
  if \(I\) is a closed interval the claim follows from \cref{11.5.2}.
  So let us assume that \(I\) is of the form \((a, b]\), \((a, b)\), or \([a, b)\) for some \(a < b\).

  We have a bound \(M\) for \(f\), so that \(-M \leq f(x) \leq M\) for all \(x \in I\).
  Now let \(0 < \varepsilon < (b - a) / 2\) be a small number.
  The function \(f\) when restricted to the interval \([a + \varepsilon, b - \varepsilon]\) is continuous, and hence Riemann integrable by \cref{11.5.2}.
  In particular, we can find a piecewise constant function \(h : [a + \varepsilon, b - \varepsilon] \to \R\) which majorizes \(f\) on \([a + \varepsilon, b - \varepsilon]\) such that
  \[
    \int_{[a + \varepsilon, b - \varepsilon]} h \leq \int_{[a + \varepsilon, b - \varepsilon]} f + \varepsilon.
  \]
  Define \(\tilde{h} : I \to \R\) by
  \[
    \tilde{h}(x) \coloneqq \begin{cases}
      h(x) & \text{if } x \in [a + \varepsilon, b - \varepsilon]             \\
      M    & \text{if } x \in I \setminus [a + \varepsilon, b - \varepsilon]
    \end{cases}
  \]
  Clearly \(\tilde{h}\) is piecewise constant on \(I\) and majorizes \(f\);
  by \cref{11.2.16} we have
  \[
    \int_I \tilde{h} = \varepsilon M + \int_{[a + \varepsilon, b - \varepsilon]} h + \varepsilon M \leq \int_{[a + \varepsilon, b - \varepsilon]} f + (2M + 1) \varepsilon.
  \]
  In particular we have
  \[
    \overline{\int}_I f \leq \int_{[a + \varepsilon, b - \varepsilon]} f + (2M + 1) \varepsilon.
  \]
  This is true since \(\tilde{h}\) majorize \(f\).
  A similar argument gives
  \[
    \underline{\int}_I f \geq \int_{[a + \varepsilon, b - \varepsilon]} f - (2M + 1) \varepsilon.
  \]
  and hence
  \[
    \overline{\int}_I f - \underline{\int}_I f \leq (4M + 2) \varepsilon.
  \]
  But \(\varepsilon\) is arbitrary, and so we can argue as in the proof of \cref{11.5.1} to conclude Riemann integrability.
\end{proof}

\begin{note}
  From \cref{11.5.1}, \cref{11.5.2} and \cref{11.5.3} we see that if we can show a function \(f\) being \emph{uniformly continuous} (not just continuous) on some bounded interval \(I\), then \(f\) is Riemann integrable on \(I\).
\end{note}

\begin{definition}\label{11.5.4}
  Let \(I\) be a bounded interval, and let \(f : I \to \R\).
  We say that \(f\) is \emph{piecewise continuous on \(I\)} iff there exists a partition \(\mathbf{P}\) of \(I\) such that \(f|_J\) is continuous on \(J\) for all \(J \in \mathbf{P}\).
\end{definition}

\setcounter{theorem}{5}
\begin{proposition}\label{11.5.6}
  Let \(I\) be a bounded interval, and let \(f : I \to \R\) be both piecewise continuous and bounded.
  Then \(f\) is Riemann integrable.
\end{proposition}

\begin{proof}
  Since \(f\) is piecewise continuous on \(I\), by \cref{11.5.4} \(\exists\ \mathbf{P}\) such that \(\mathbf{P}\) is a partition of \(I\) and \(f|_J\) is continuous on \(J\) for all \(J \in \mathbf{P}\).
  Since \(f\) is bounded, we know that \(f|_J\) is bounded for all \(J \in \mathbf{P}\).
  Thus by \cref{11.5.3} \(f|_J\) is Riemann integrable on \(J\) for all \(J \in \mathbf{P}\).
  For each \(J \in \mathbf{P}\), we define \(F_J : I \to \R\) to be the function
  \[
    F_J(x) = \begin{cases}
      f|_J(x) & \text{if } x \in J    \\
      0       & \text{if } x \notin J
    \end{cases}
  \]
  Then by \cref{11.4.1}(g) \(F|_J\) is Riemann integrable for all \(J \in \mathbf{P}\) and
  \begin{align*}
    \sum_{J \in \mathbf{P}} \int_I F_J & = \sum_{J \in \mathbf{P}} \int_J f|_J & \text{(by \cref{11.4.1}(g))} \\
                                       & = \int_I f.                           & \text{(by \cref{ex 11.4.3})}
  \end{align*}
  Thus \(f\) is Riemann integrable on \(I\).
\end{proof}

\exercisesection

\begin{exercise}\label{ex 11.5.1}
  Prove \cref{11.5.6}.
\end{exercise}

\begin{proof}
  See \cref{11.5.6}.
\end{proof}

%------------------------------------------------------------------------------
% Back matters.
%------------------------------------------------------------------------------

\backmatter

\end{document}
