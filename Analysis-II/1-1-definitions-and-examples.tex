\section{Definitions and examples}\label{sec 1.1}

\begin{lemma}\label{1.1.1}
    Let \((x_n)_{n = m}^\infty\) be a sequence of real numbers, and let \(x\) be another real number.
    Then \((x_n)_{n = m}^\infty\) converges to \(x\) if and only if \(\lim_{n \to \infty} d(x_n, x) = 0\).
\end{lemma}

\begin{proof}
    We have \(\lim_{n \to \infty} x_n = x\) iff \(\forall\ \varepsilon > 0\), \(\exists\ N \geq m\) such that
    \[
        \forall\ n \geq N, \abs*{x_n - x} \leq \varepsilon.
    \]
    Since
    \[
        \abs*{x_n - x} \leq \varepsilon \iff \abs*{\abs*{x_n - x} - 0} \leq \varepsilon
    \]
    we have \(\lim_{n \to \infty} \abs*{x_n - x} = 0\).
    Thus \(\lim_{n \to \infty} x_n = x\) iff \(\lim_{n \to \infty} d(x_n, x) = 0\).
\end{proof}

\begin{note}
    One would now like to generalize this notion of convergence, so that one can take limits not just of sequences of real numbers, but also sequences of complex numbers, or sequences of vectors, or sequences of matrices, or sequences of functions, even sequences of sequences.
    One way to do this is to redefine the notion of convergence each time we deal with a new type of object.
    A more efficient way is to work \emph{abstractly}, defining a very general class of spaces - which includes such standard spaces as the real numbers, complex numbers, vectors, etc. - and define the notion of convergence on this entire class of spaces at once.
    (A \emph{space} is just the set of all objects of a certain type.
    Mathematically, there is not much distinction between a space and a set, except that spaces tend to have much more structure than what a random set would have.)
\end{note}

\begin{note}
    It turns out that there are two very useful classes of spaces which do the job.
    The first class is that of \emph{metric spaces}.
    There is a more general class of spaces, called \emph{topological spaces}.
\end{note}

\begin{definition}[Metric spaces]\label{1.1.2}
    A metric space \((X, d)\) is a space \(X\) of objects (called \emph{points}), together with a \emph{distance function} or \emph{metric} \(d : X \times X \to [0, +\infty)\), which associates to each pair \(x, y\) of points in \(X\) a non-negative real number \(d(x, y) \geq 0\).
    Furthermore, the metric must satisfy the following four axioms:
    \begin{enumerate}
        \item For any \(x \in X\), we have \(d(x, x) = 0\).
        \item (Positivity) For any distinct \(x, y \in X\), we have \(d(x, y) > 0\).
        \item (Symmetry) For any \(x, y \in X\), we have \(d(x, y) = d(y, x)\).
        \item (Triangle inequality) For any \(x, y, z \in X\), we have \(d(x, z) \leq d(x, y) + d(y, z)\).
    \end{enumerate}
\end{definition}

\begin{note}
    In many cases it will be clear what the metric \(d\) is, and we shall abbreviate \((X, d)\) as just \(X\).
\end{note}

\begin{remark}\label{1.1.3}
    The conditions (a) and (b) of Definition \ref{1.1.1} can be rephrased as follows:
    for any \(x, y \in X\) we have \(d(x, y) = 0\) if and only if \(x = y\).
\end{remark}

\begin{example}[The real line]\label{1.1.4}
    Let \(\mathbf{R}\) be the real numbers, and let \(d : \mathbf{R} \times \mathbf{R} \to [0, \infty)\) be the metric \(d(x, y) \coloneqq \abs*{x - y}\) mentioned earlier.
    Then \((\mathbf{R}, d)\) is a metric space.
    We refer to \(d\) as the \emph{standard metric} on \(\mathbf{R}\), and if we refer to \(\mathbf{R}\) as a metric space, we assume that the metric is given by the standard metric \(d\) unless otherwise specified.
\end{example}

\begin{example}[Induced metric spaces]\label{1.1.5}
    Let \((X, d)\) be any metric space, and let \(Y\) be a subset of \(X\).
    Then we can restrict the metric function \(d : X \times X \to [0, +\infty)\) to the subset \(Y \times Y\) of \(X \times X\) to create a restricted metric function \(d|_{Y \times Y} : Y \times Y \to [0, +\infty)\) of \(Y\);
    this is known as the metric on \(Y\) \emph{induced} by the metric \(d\) on \(X\).
    The pair \((Y, d|_{Y \times Y})\) is a metric space and is known the \emph{subspace} of \((X, d)\) induced by \(Y\).
    Thus for instance the metric on the real line in the Example \ref{1.1.4} induces a metric space structure on any subset of the reals, such as the integers \(Z\), or an interval \([a, b]\), etc.
\end{example}

\begin{example}[Euclidean spaces]\label{1.1.6}
    Let \(n \geq 1\) be a natural number, and let \(\mathbf{R}^n\) be the space of \(n\)-tuples of real numbers:
    \[
        \mathbf{R}^n = \{(x_1, x_2, \dots, x_n) : x_1, \dots, x_n \in \mathbf{R}\}.
    \]
    We define the \emph{Euclidean metric} (also called the \emph{\(l^2\) metric}) \(d_{l^2} : \mathbf{R}^n \times\mathbf{R}^n \to \mathbf{R}\) by
    \begin{align*}
        d_{l^2}((x_1, \dots, x_n), (y_1, \dots, y_n)) & \coloneqq \sqrt{(x_1 - y_1)^2 + \dots + (x_n - y_n)^2} \\
                                                      & = \bigg(\sum_{i = 1}^n (x_i - y_i)^2\bigg)^{1 / 2}.
    \end{align*}
\end{example}

\begin{note}
    Euclidean metric corresponds to the geometric distance between the two points \((x_1, x_2, \dots, x_n)\), \((y_1, y_2, \dots, y_n)\) as given by Pythagoras' theorem.
    While geometry does give some very important examples of metric spaces, it is possible to have metric spaces which have no obvious geometry whatsoever.
    The verification that \((\mathbf{R}^n, d)\) is indeed a metric space can be seen geometrically (for instance, the triangle inequality now asserts that the length of one side of a triangle is always less than or equal to the sum of the lengths of the other two sides), but can also be proven algebraically.
    We refer to \((\mathbf{R}^n , d_{l^2})\) as the \emph{Euclidean space} of \emph{dimension \(n\)}.
    Extending the convention from Example \ref{1.1.4}, if we refer to \(\mathbf{R}^n\) as a metric space, we assume that the metric is given by the Euclidean metric unless otherwise specified.
\end{note}

\begin{example}[Taxi-cab metric]\label{1.1.7}
    Again let \(n \geq 1\), and let \(\mathbf{R}^n\) be as before.
    But now we use a different metric \(d_{l^1}\), the so-called \emph{taxicab metric} (or \emph{\(l^1\) metric}), defined by
    \begin{align*}
        d_{l^1}((x_1, \dots, x_n), (y_1, \dots, y_n)) & \coloneqq \abs*{x_1 - y_1} + \dots + \abs*{x_n - y_n} \\
                                                      & = \sum_{i = 1}^n \abs*{x_i - y_i}.
    \end{align*}
\end{example}

\begin{note}
    This metric is called the taxi-cab metric, because it models the distance a taxi-cab would have to traverse to get from one point to another if the cab was only allowed to move in cardinal directions (north, south, east, west) and not diagonally.
    As such it is always at least as large as the Euclidean metric, which measures distance ``as the crow flies'', as it were.
    We claim that the space \((\mathbf{R}^n, d_{l^1})\) is also a metric space.
    The metrics are not quite the same, but we do have the inequalities
    \[
        d_{l^2}(x, y) \leq d_{l^1}(x, y) \leq \sqrt{n} d_{l^2}(x, y)
    \]
    for all \(x, y\).
\end{note}

\begin{remark}\label{1.1.8}
    The taxi-cab metric is useful in several places, for instance in the theory of error correcting codes.
    A string of \(n\) binary digits can be thought of as an element of \(\mathbf{R}^n\).
    The taxi-cab distance between two binary strings is then the number of bits in the two strings which do not match.
    The goal of error-correcting codes is to encode each piece of information (e.g., a letter of the alphabet) as a binary string in such a way that all the binary strings are as far away in the taxicab metric from each other as possible;
    this minimizes the chance that any distortion of the bits due to random noise can accidentally change one of the coded binary strings to another, and also maximizes the chance that any such distortion can be detected and correctly repaired.
\end{remark}

\begin{example}[Sup norm metric]\label{1.1.9}
    Again let \(n \geq 1\), and let \(\mathbf{R}^n\) be as before.
    But now we use a different metric \(d_{l^\infty}\), the so-called \emph{sup norm metric} (or \emph{\(l^\infty\) metric}), defined by
    \[
        d_{l^\infty} ((x_1, \dots, x_n), (y_1, \dots, y_n)) \coloneqq \sup\{\abs*{x_i - y_i} : 1 \leq i \leq n\}.
    \]
\end{example}

\begin{note}
    The space \((\mathbf{R}^n, d_{l^\infty})\) is also a metric space, and is related to the \(l^2\) metric by the inequalities
    \[
        \frac{1}{\sqrt{n}} d_{l^2}(x, y) \leq d_{l^\infty}(x, y) \leq d_{l^2}(x, y)
    \]
    for all \(x, y\)
\end{note}

\begin{remark}\label{1.1.10}
    The \(l^1\), \(l^2\), and \(l^\infty\) metrics are special cases of the more general \emph{\(l^p\) metrics}, where \(p \in [1, +\infty)\).
\end{remark}