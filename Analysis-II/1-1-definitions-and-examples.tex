\section{Definitions and examples}\label{sec 1.1}

\begin{lemma}\label{1.1.1}
    Let \((x_n)_{n = m}^\infty\) be a sequence of real numbers, and let \(x\) be another real number.
    Then \((x_n)_{n = m}^\infty\) converges to \(x\) if and only if \(\lim_{n \to \infty} d(x_n, x) = 0\).
\end{lemma}

\begin{proof}
    We have \(\lim_{n \to \infty} x_n = x\) iff \(\forall\ \varepsilon > 0\), \(\exists\ N \geq m\) such that
    \[
        \forall\ n \geq N, \abs*{x_n - x} \leq \varepsilon.
    \]
    Since
    \[
        \abs*{x_n - x} \leq \varepsilon \iff \abs*{\abs*{x_n - x} - 0} \leq \varepsilon
    \]
    we have \(\lim_{n \to \infty} \abs*{x_n - x} = 0\).
    Thus \(\lim_{n \to \infty} x_n = x\) iff \(\lim_{n \to \infty} d(x_n, x) = 0\).
\end{proof}

\begin{note}
    One would now like to generalize this notion of convergence, so that one can take limits not just of sequences of real numbers, but also sequences of complex numbers, or sequences of vectors, or sequences of matrices, or sequences of functions, even sequences of sequences.
    One way to do this is to redefine the notion of convergence each time we deal with a new type of object.
    A more efficient way is to work \emph{abstractly}, defining a very general class of spaces - which includes such standard spaces as the real numbers, complex numbers, vectors, etc. - and define the notion of convergence on this entire class of spaces at once.
    (A \emph{space} is just the set of all objects of a certain type.
    Mathematically, there is not much distinction between a space and a set, except that spaces tend to have much more structure than what a random set would have.)
\end{note}

\begin{note}
    It turns out that there are two very useful classes of spaces which do the job.
    The first class is that of \emph{metric spaces}.
    There is a more general class of spaces, called \emph{topological spaces}.
\end{note}

\begin{definition}[Metric spaces]\label{1.1.2}
    A metric space \((X, d)\) is a space \(X\) of objects (called \emph{points}), together with a \emph{distance function} or \emph{metric} \(d : X \times X \to [0, +\infty)\), which associates to each pair \(x, y\) of points in \(X\) a non-negative real number \(d(x, y) \geq 0\).
    Furthermore, the metric must satisfy the following four axioms:
    \begin{enumerate}
        \item For any \(x \in X\), we have \(d(x, x) = 0\).
        \item (Positivity) For any distinct \(x, y \in X\), we have \(d(x, y) > 0\).
        \item (Symmetry) For any \(x, y \in X\), we have \(d(x, y) = d(y, x)\).
        \item (Triangle inequality) For any \(x, y, z \in X\), we have \(d(x, z) \leq d(x, y) + d(y, z)\).
    \end{enumerate}
\end{definition}

\begin{note}
    In many cases it will be clear what the metric \(d\) is, and we shall abbreviate \((X, d)\) as just \(X\).
\end{note}

\begin{remark}\label{1.1.3}
    The conditions (a) and (b) of Definition \ref{1.1.1} can be rephrased as follows:
    for any \(x, y \in X\) we have \(d(x, y) = 0\) if and only if \(x = y\).
\end{remark}

\begin{example}[The real line]\label{1.1.4}
    Let \(\mathbf{R}\) be the real numbers, and let \(d : \mathbf{R} \times \mathbf{R} \to [0, \infty)\) be the metric \(d(x, y) \coloneqq \abs*{x - y}\) mentioned earlier.
    Then \((\mathbf{R}, d)\) is a metric space.
    We refer to \(d\) as the \emph{standard metric} on \(\mathbf{R}\), and if we refer to \(\mathbf{R}\) as a metric space, we assume that the metric is given by the standard metric \(d\) unless otherwise specified.
\end{example}