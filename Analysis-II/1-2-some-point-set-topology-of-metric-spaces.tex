\section{Some point-set topology of metric spaces}\label{sec 1.2}

\begin{note}
    Having defined the operation of convergence on metric spaces, we now define a couple other related notions, including that of open set, closed set, interior, exterior, boundary, and adherent point.
    The study of such notions is known as \emph{point-set topology}.
\end{note}

\begin{definition}[Balls]\label{1.2.1}
    Let \((X, d)\) be a metric space, let \(x_0\) be a point in \(X\), and let \(r > 0\).
    We define the \emph{ball} \(B_{(X, d)}(x_0, r)\) in \(X\), centered at \(x_0\), and with radius \(r\), in the metric \(d\), to be the set
    \[
        B_{(X, d)}(x_0, r) \coloneqq \{x \in X : d(x, x_0) < r\}.
    \]
    When it is clear what the metric space \((X, d)\) is, we shall abbreviate \(B_{(X, d)}(x_0, r)\) as just \(B(x_0, r)\).
\end{definition}

\setcounter{theorem}{3}
\begin{remark}\label{1.2.4}
    Note that the smaller the radius \(r\), the smaller the ball \(B(x_0 , r)\).
    However, \(B(x_0 , r)\) always contains at least one point, namely the center \(x_0\), as long as \(r\) stays positive, thanks to Definition \ref{1.1.2}(a).
    (We don't consider balls of zero radius or negative radius since they are rather boring, being just the empty set.)
\end{remark}