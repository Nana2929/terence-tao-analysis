\section{Some point-set topology of metric spaces}\label{sec 1.2}

\begin{note}
    Having defined the operation of convergence on metric spaces, we now define a couple other related notions, including that of open set, closed set, interior, exterior, boundary, and adherent point.
    The study of such notions is known as \emph{point-set topology}.
\end{note}

\begin{definition}[Balls]\label{1.2.1}
    Let \((X, d)\) be a metric space, let \(x_0\) be a point in \(X\), and let \(r > 0\).
    We define the \emph{ball} \(B_{(X, d)}(x_0, r)\) in \(X\), centered at \(x_0\), and with radius \(r\), in the metric \(d\), to be the set
    \[
        B_{(X, d)}(x_0, r) \coloneqq \{x \in X : d(x, x_0) < r\}.
    \]
    When it is clear what the metric space \((X, d)\) is, we shall abbreviate \(B_{(X, d)}(x_0, r)\) as just \(B(x_0, r)\).
\end{definition}

\setcounter{theorem}{3}
\begin{remark}\label{1.2.4}
    Note that the smaller the radius \(r\), the smaller the ball \(B(x_0 , r)\).
    However, \(B(x_0 , r)\) always contains at least one point, namely the center \(x_0\), as long as \(r\) stays positive, thanks to Definition \ref{1.1.2}(a).
    (We don't consider balls of zero radius or negative radius since they are rather boring, being just the empty set.)
\end{remark}

\begin{definition}[Interior, exterior, boundary]\label{1.2.5}
    Let \((X, d)\) be a metric space, let \(E\) be a subset of \(X\), and let \(x_0\) be a point in \(X\).
    We say that \(x_0\) is an \emph{interior point of} \(E\) if there exists a radius \(r > 0\) such that \(B(x_0, r) \subseteq E\).
    We say that \(x_0\) is an \emph{exterior point of} \(E\) if there exists a radius \(r > 0\) such that \(B(x_0, r) \cap E = \emptyset\).
    We say that \(x_0\) is a \emph{boundary point of} \(E\) if it is neither an interior point nor an exterior point of \(E\).
\end{definition}

\begin{note}
    The set of all interior points of \(E\) is called the interior of \(E\) and is sometimes denoted \(\text{int}(E)\).
    The set of exterior points of \(E\) is called the exterior of \(E\) and is sometimes denoted \(\text{ext}(E)\).
    The set of boundary points of \(E\) is called the boundary of \(E\) and is sometimes denoted \(\partial E\).
\end{note}

\begin{remark}\label{1.2.6}
    If \(x_0\) is an interior point of \(E\), then \(x_0\) must actually be an element of \(E\), since balls \(B(x_0, r)\) always contain their center \(x_0\).
    Conversely, if \(x_0\) is an exterior point of \(E\), then \(x_0\) cannot be an element of \(E\).
    In particular it is not possible for \(x_0\) to simultaneously be an interior and an exterior point of \(E\).
    If \(x_0\) is a boundary point of \(E\), then it could be an element of \(E\), but it could also not lie in \(E\).
\end{remark}

\setcounter{theorem}{7}
\begin{example}\label{1.2.8}
    When we give a set \(X\) the discrete metric \(d_{\text{disc}}\), and \(E\) is any subset of \(X\), then every element of \(E\) is an interior point of \(E\), every point not contained in \(E\) is an exterior point of \(E\), and there are no boundary points.
\end{example}

\begin{proof}
    \(\forall\ x_0 \in E\), we have \(d_{\text{disc}}(x_0, x_0) = 0\) and \(B_{(X, d_{\text{disc}})}(x_0, 1) = \{x_0\} \subseteq E\), thus by Definition \ref{1.2.5} \(x_0 \in \text{int}(E)\) and \(E \subseteq \text{int}(E)\).
    By Remark \ref{1.2.6}, we also have \(\text{int}(E) \subseteq E\), thus \(\text{int}(E) = E\).

    Since \(\text{int}(E) = E\), \(\forall\ x_0 \in X \setminus E\), we have \(x_0 \notin \text{int}(E)\).
    By Definition \ref{1.2.5} this means \(\forall\ r \in \mathbf{R}^+\), we have \(B_{(X, d_{\text{disc}})}(x_0, r) \not\subseteq E\).
    In particular, we have \(B_{(X, d_{\text{disc}})}(x_0, 1) = \{x_0\} \not\subseteq E\).
    Since \(x_0\) is the only element of \(\{x_0\}\), we have \(x_0 \notin E\) and \(\{x_0\} \cap E = \emptyset\).
    Thus by Definition \ref{1.2.5} we have \(x_0 \in \text{ext}(E)\) and \(X \setminus E \subseteq \text{ext}(E)\).
    By Remark \ref{1.2.6} we also have \(\text{ext}(E) \subseteq X \setminus E\), thus \(\text{ext}(E) = X \setminus E\).

    Since \(X = (X \setminus E) \cup E = \text{ext}(E) \cup \text{int}(E)\), every point in \(X\) is either a exterior point or interior point, thus there are no boundary points in \(X\) when given metric \(d_{\text{disc}}\).
\end{proof}
