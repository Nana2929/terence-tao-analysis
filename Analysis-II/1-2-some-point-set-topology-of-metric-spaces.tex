\section{Some point-set topology of metric spaces}\label{sec 1.2}

\begin{note}
    Having defined the operation of convergence on metric spaces, we now define a couple other related notions, including that of open set, closed set, interior, exterior, boundary, and adherent point.
    The study of such notions is known as \emph{point-set topology}.
\end{note}

\begin{definition}[Balls]\label{1.2.1}
    Let \((X, d)\) be a metric space, let \(x_0\) be a point in \(X\), and let \(r > 0\).
    We define the \emph{ball} \(B_{(X, d)}(x_0, r)\) in \(X\), centered at \(x_0\), and with radius \(r\), in the metric \(d\), to be the set
    \[
        B_{(X, d)}(x_0, r) \coloneqq \{x \in X : d(x, x_0) < r\}.
    \]
    When it is clear what the metric space \((X, d)\) is, we shall abbreviate \(B_{(X, d)}(x_0, r)\) as just \(B(x_0, r)\).
\end{definition}

\setcounter{theorem}{3}
\begin{remark}\label{1.2.4}
    Note that the smaller the radius \(r\), the smaller the ball \(B(x_0 , r)\).
    However, \(B(x_0 , r)\) always contains at least one point, namely the center \(x_0\), as long as \(r\) stays positive, thanks to Definition \ref{1.1.2}(a).
    (We don't consider balls of zero radius or negative radius since they are rather boring, being just the empty set.)
\end{remark}

\begin{definition}[Interior, exterior, boundary]\label{1.2.5}
    Let \((X, d)\) be a metric space, let \(E\) be a subset of \(X\), and let \(x_0\) be a point in \(X\).
    We say that \(x_0\) is an \emph{interior point of} \(E\) if there exists a radius \(r > 0\) such that \(B(x_0, r) \subseteq E\).
    We say that \(x_0\) is an \emph{exterior point of} \(E\) if there exists a radius \(r > 0\) such that \(B(x_0, r) \cap E = \emptyset\).
    We say that \(x_0\) is a \emph{boundary point of} \(E\) if it is neither an interior point nor an exterior point of \(E\).
\end{definition}

\begin{note}
    The set of all interior points of \(E\) is called the interior of \(E\) and is sometimes denoted \(\text{int}(E)\).
    The set of exterior points of \(E\) is called the exterior of \(E\) and is sometimes denoted \(\text{ext}(E)\).
    The set of boundary points of \(E\) is called the boundary of \(E\) and is sometimes denoted \(\partial E\).
\end{note}

\begin{note}
    We use the same notation of metric balls for the set of interior points \(\text{int}_{(X, d)}(E)\), the set of exterior points \(\text{ext}_{(X, d)}(E)\) and the set of boundary points \(\partial_{(X, d)}(E)\).
\end{note}

\begin{remark}\label{1.2.6}
    If \(x_0\) is an interior point of \(E\), then \(x_0\) must actually be an element of \(E\), since balls \(B(x_0, r)\) always contain their center \(x_0\).
    Conversely, if \(x_0\) is an exterior point of \(E\), then \(x_0\) cannot be an element of \(E\).
    In particular it is not possible for \(x_0\) to simultaneously be an interior and an exterior point of \(E\).
    If \(x_0\) is a boundary point of \(E\), then it could be an element of \(E\), but it could also not lie in \(E\).
\end{remark}

\begin{note}
    By Remark \ref{1.2.6} we thus have
    \begin{align*}
         & \text{int}_{(X, d)}(E) \subseteq E             \\
         & \text{ext}_{(X, d)}(E) \subseteq X \setminus E
    \end{align*}
    for any metric space \((X, d)\) and any subset \(E\) of \(X\).
\end{note}

\setcounter{theorem}{7}
\begin{example}\label{1.2.8}
    When we give a set \(X\) the discrete metric \(d_{\text{disc}}\), and \(E\) is any subset of \(X\), then every element of \(E\) is an interior point of \(E\), every point not contained in \(E\) is an exterior point of \(E\), and there are no boundary points.
\end{example}

\begin{proof}
    We have
    \begin{align*}
                 & \forall x_0 \in E, d_{\text{disc}}(x_0, x_0) = 0                          & \text{(by Example \ref{1.1.11})}   \\
        \implies & \forall x_0 \in E, B_{(X, d_{\text{disc}})}(x_0, 1) = \{x_0\} \subseteq E & \text{(by Definition \ref{1.2.1})} \\
        \implies & \forall x_0 \in E, x_0 \in \text{int}_{(X, d_{\text{disc}})}(E)           & \text{(by Definition \ref{1.2.5})} \\
        \implies & E \subseteq \text{int}_{(X, d_{\text{disc}})}(E)                                                               \\
        \implies & E = \text{int}_{(X, d_{\text{disc}})}(E)                                  & \text{(by Remark \ref{1.2.6})}
    \end{align*}
    and
    \begin{align*}
                 & E = \text{int}_{(X, d_{\text{disc}})}(E)                                                                                                            \\
        \implies & \forall x_0 \in X \setminus E, x_0 \notin \text{int}_{(X, d_{\text{disc}})}(E)                                                                      \\
        \implies & \forall x_0 \in X \setminus E, \forall r \in \mathbf{R}^+, B_{(X, d_{\text{disc}})}(x_0, r) \not\subseteq E    & \text{(by Definition \ref{1.2.5})} \\
        \implies & \forall x_0 \in X \setminus E, \forall r \in \mathbf{R}^+, B_{(X, d_{\text{disc}})}(x_0, r) \cap E = \emptyset                                      \\
        \implies & \forall x_0 \in X \setminus E, x_0 \in \text{ext}_{(X, d_{\text{disc}})}(E)                                                                         \\
        \implies & X \setminus E \subseteq \text{ext}_{(X, d_{\text{disc}})}(E)                                                                                        \\
        \implies & X \setminus E = \text{ext}_{(X, d_{\text{disc}})}(E).                                                          & \text{(by Remark \ref{1.2.6})}
    \end{align*}
    Since \(X = (X \setminus E) \cup E = \text{ext}(E) \cup \text{int}(E)\), every point in \((X, d_{\text{disc}})\) is either a exterior point or interior point, thus by Definition \ref{1.2.5} there are no boundary points in \((X, d_{\text{disc}})\).
\end{proof}

\begin{definition}[Closure]\label{1.2.9}
    Let \((X, d)\) be a metric space, let \(E\) be a subset of \(X\), and let \(x_0\) be a point in \(X\).
    We say that \(x_0\) is an \emph{adherent point} of \(E\) if for every radius \(r > 0\), the ball \(B(x_0, r)\) has a non-empty intersection with \(E\).
    The set of all adherent points of \(E\) is called the \emph{closure} of \(E\) and is denoted \(\overline{E}\).
\end{definition}

\begin{note}
    Notions in Definition \ref{1.2.9} are consistent with the corresponding notions on the real line.
\end{note}

\begin{note}
    Since the closure of a set \(E\) depends on metric \((X, d)\), we denote the closure of \(E\) with \(\overline{E}_{(X, d)}\).
\end{note}

\begin{proposition}\label{1.2.10}
    Let \((X, d)\) be a metric space, let \(E\) be a subset of \(X\), and let \(x_0\) be a point in \(X\).
    Then the following statements are logically equivalent.
    \begin{enumerate}
        \item \(x_0\) is an adherent point of \(E\).
        \item \(x_0\) is either an interior point or a boundary point of \(E\).
        \item There exists a sequence \((x_n)_{n = 1}^\infty\) in \(E\) which converges to \(x_0\) with respect to the metric \(d\).
    \end{enumerate}
\end{proposition}

\begin{proof}
    We first show that statement (a) implies statement (b).
    \begin{align*}
                 & \forall x_0 \in \overline{E}_{(X, d)}, \forall r \in \mathbf{R}^+, B_{(X, d)}(x_0, r) \cap E \neq \emptyset & \text{(by Definition \ref{1.2.9})} \\
        \implies & \forall x_0 \in \overline{E}_{(X, d)}, x_0 \notin \text{ext}_{(X, d)}(E)                                    & \text{(by Definition \ref{1.2.5})} \\
        \implies & \forall x_0 \in \overline{E}_{(X, d)}, x_0 \in \big(\text{int}_{(X, d)}(E) \cup \partial_{(X, d)}(E)\big)   & \text{(by Definition \ref{1.2.5})} \\
        \implies & \overline{E}_{(X, d)} \subseteq \big(\text{int}_{(X, d)}(E) \cup \partial_{(X, d)}(E)\big)
    \end{align*}

    Next we show that statement (b) implies statement (c).
    \begin{itemize}
        \item Suppose that \(x_0 \in \text{int}_{(X, d)}(E)\).
              Then by setting \(x_n = x_0\) for all \(n \in \mathbf{Z}^+\) we have \(\lim_{n \to \infty} d(x_n, x_0) = \lim_{n \to \infty} 0 = 0\).
        \item Suppose that \(x_0 \in \partial_{(X, d)}(E)\).
              By Definition \ref{1.2.5} for all \(r \in \mathbf{R}^+\), we have \(B_{(X, d)}(x_0, r) \cap E \neq \emptyset\).
              In particular, we have \(B_{(X, d)}(x_0, \frac{1}{n}) \cap E \neq \emptyset\) for all \(n \in \mathbf{Z}^+\).
              Let \(X_n = B_{(X, d)}(x_0, \frac{1}{n}) \cap E\).
              Since \(X_n \neq \emptyset\), by axiom of choice we can choose \((x_n)_{n = 1}^\infty \in \prod_{n \in \mathbf{Z}^+} X_n\).
              Thus \(d(x_n, x_0) < \frac{1}{n}\) for all \(n \in \mathbf{Z}^+\) and by squeeze test we have \(\lim_{n \to \infty} d(x_n, x_0) = 0\).
    \end{itemize}
    From all cases above we conclude that there exists a sequence \((x_n)_{n = 1}^\infty\) in \(E\) which converges to \(x_0\) with respect to \(d\).

    Finally we show that statement (c) implies statement (a).
    Suppose that there exists a sequence \((x_n)_{n = 1}^\infty\) in \(E\) which converges to \(x_0\) with respect to \(d\).
    By Definition \ref{1.1.14} we have
    \[
        \forall r \in \mathbf{R}^+, \exists\ N \in \mathbf{Z}^+ : \forall n \in \mathbf{N}, n \geq N \implies d(x_n, x_0) \leq r.
    \]
    Since \(x_N \in E\) and \(E \subseteq X\), we know that the set \(B_{(X, d)}(x_0, r) \cap E \neq \emptyset\).
    Since \(r\) is arbitrary, by Definition \ref{1.2.9} we know that \(x_0\) is an adherent point of \(E\).
\end{proof}

\begin{corollary}\label{1.2.11}
    Let \((X, d)\) be a metric space, and let \(E\) be a subset of \(X\).
    Then \(\overline{E} = \text{int}(E) \cup \partial E = X \setminus \text{ext}(E)\).
\end{corollary}

\begin{proof}
    By Proposition \ref{1.2.10}(a)(b) we are done.
\end{proof}

\begin{definition}[Open and closed sets]\label{1.2.12}
    Let \((X, d)\) be a metric space, and let \(E\) be a subset of \(X\).
    We say that \(E\) is \emph{closed} if it contains all of its boundary points, i.e., \(\partial E \subseteq E\).
    We say that \(E\) is \emph{open} if it contains none of its boundary points, i.e., \(\partial E \cap E = \emptyset\).
    If \(E\) contains some of its boundary points but not others, then it is neither open nor closed.
\end{definition}

\setcounter{theorem}{13}
\begin{remark}\label{1.2.14}
    It is possible for a set to be simultaneously open and closed, if it has no boundary.
    For instance, in a metric space \((X, d)\), the whole space \(X\) has no boundary (every point in \(X\) is an interior point), and so \(X\) is both open and closed.
    The empty set \(\emptyset\) also has no boundary (every point in \(X\) is an exterior point), and so is both open and closed.
    In many cases these are the only sets that are simultaneously open and closed, but there are exceptions.
    For instance, using the discrete metric \(d_{\text{disc}}\), \emph{every} set is both open and closed! (See Example \ref{1.2.8})
\end{remark}

\begin{note}
    From Remark \ref{1.2.14} we see that the notions of being open and being closed are \emph{not} negations of each other;
    there are sets that are both open and closed, and there are sets which are neither open nor closed.
    Thus, if one knew for instance that \(E\) was not an open set, it would be erroneous to conclude from this that \(E\) was a closed set, and similarly with the roles of open and closed reversed.
\end{note}

\begin{proposition}[Basic properties of open and closed sets]\label{1.2.15}
    Let \((X, d)\) be a metric space.
    \begin{enumerate}
        \item Let \(E\) be a subset of \(X\).
              Then \(E\) is open if and only if \(E = \text{int}(E)\).
              In other words, \(E\) is open if and only if for every \(x \in E\), there exists an \(r > 0\) such that \(B(x, r) \subseteq E\).
        \item Let \(E\) be a subset of \(X\).
              Then \(E\) is closed if and only if \(E\) contains all its adherent points.
              In other words, \(E\) is closed if and only if for every convergent sequence \((x_n)_{n = m}^\infty\) in \(E\), the limit \(\lim_{n \to \infty} x_n\) of that sequence also lies in \(E\).
        \item For any \(x_0 \in X\) and \(r > 0\), then the ball \(B(x_0, r)\) is an open set.
              The set \(\{x \in X : d(x, x_0) \leq r\}\) is a closed set.
              (This set is sometimes called the \emph{closed ball} of radius \(r\) centered at \(x_0\).)
        \item Any singleton set \(\{x_0\}\), where \(x_0 \in X\), is automatically closed.
        \item If \(E\) is a subset of \(X\), then \(E\) is open if and only if the complement \(X \setminus E \coloneqq \{x \in X : x \notin E\}\) is closed.
        \item If \(E_1, \dots, E_n\) are a finite collection of open sets in \(X\), then \(E_1 \cap E_2 \cap \dots \cap E_n\) is also open.
              If \(F_1, \dots, F_n\) is a finite collection of closed sets in \(X\), then \(F_1 \cup F_2 \cup \dots \cup F_n\) is also closed.
        \item If \(\{E_\alpha\}_{\alpha \in I}\) is a collection of open sets in \(X\) (where the index set \(I\) could be finite, countable, or uncountable), then the union \(\bigcup_{\alpha \in I} E_\alpha \coloneqq \{x \in X : x \in E_\alpha \text{ for some } \alpha \in I\}\) is also open.
              If \(\{F_\alpha\}_{\alpha \in I}\) is a collection of closed sets in \(X\), then the intersection \(\bigcap_{\alpha \in I} F_\alpha \coloneqq \{x \in X : x \in F_\alpha \text{ for all } \alpha \in I\}\) is also closed.
        \item If \(E\) is any subset of \(X\), then \(\text{int}(E)\) is the largest open set which is contained in \(E\);
              in other words, \(\text{int}(E)\) is open, and given any other open set \(V \subseteq E\), we have \(V \subseteq \text{int}(E)\).
              Similarly \(\overline{E}\) is the smallest closed set which contains \(E\);
              in other words, \(\overline{E}\) is closed, and given any other closed set \(K \supseteq E\), \(K \supseteq \overline{E}\).
    \end{enumerate}
\end{proposition}

\begin{proof}{(a)}
    \begin{align*}
             & E \text{ is open in } (X, d)                                                                                                                   \\
        \iff & \partial_{(X, d)}(E) \cap E = \emptyset                                                                  & \text{(by Definition \ref{1.2.12})} \\
        \iff & \forall x \in E, \big(x \notin \partial_{(X, d)}(E)\big) \land \big(x \notin \text{ext}_{(X, d)}(E)\big) & \text{(by Remark \ref{1.2.6})}      \\
        \iff & \forall x \in E, x \in \text{int}_{(X, d)}(E)                                                            & \text{(by Definition \ref{1.2.5})}  \\
        \iff & E \subseteq \text{int}_{(X, d)}(E)                                                                                                             \\
        \iff & \text{int}(E) = E.                                                                                       & \text{(by Remark \ref{1.2.6})}
    \end{align*}
\end{proof}

\begin{proof}{(b)}
    \begin{align*}
             & E \text{ is closed in } (X, d)                                                                    \\
        \iff & \partial_{(X, d)}(E) \subseteq E                     & \text{(by Definition \ref{1.2.12})}        \\
        \iff & E = \text{int}_{(X, d)}(E) \cup \partial_{(X, d)}(E) & \text{(by Remark \ref{1.2.6})}             \\
        \iff & E = \overline{E}_{(X, d)}.                           & \text{(by Proposition \ref{1.2.10}(a)(b))}
    \end{align*}
\end{proof}

\begin{proof}{(c)}
    We first show that \(B_{(X, d)}(x_0, r)\) is open in \((X, d)\).
    Since \(x_0 \in B_{(X, d)}(x_0, r)\), we know that \(B_{(X, d)}(x_0, r) \neq \emptyset\).
    Let \(x \in B_{(X, d)}(x_0, r)\) and let \(r' = r - d(x, x_0)\).
    By Definition \ref{1.2.1} we have \(d(x, x_0) < r\), so \(r' > 0\).
    Then we have
    \begin{align*}
                 & \forall y \in B_{(X, d)}(x, r')                                                \\
        \implies & d(y, x) < r'                           & \text{(by Definition \ref{1.2.1})}    \\
        \implies & d(y, x) < r - d(x, x_0)                                                        \\
        \implies & d(y, x) + d(x, x_0) < r                                                        \\
        \implies & d(y, x_0) \leq d(y, x) + d(x, x_0) < r & \text{(by Definition \ref{1.1.2}(d))} \\
        \implies & y \in B_{(X, d)}(x_0, r)               & \text{(by Definition \ref{1.2.1})}
    \end{align*}
    and thus \(B_{(X, d)}(x, r') \subseteq B_{(X, d)}(x_0, r)\).
    Since \(x\) is arbitrary, we have
    \begin{align*}
                 & B_{(X, d)}(x_0, r) \subseteq \text{int}_{(X, d)}\big(B_{(X, d)}(x_0, r)\big) & \text{(by Definition \ref{1.2.5})}      \\
        \implies & B_{(X, d)}(x_0, r) = \text{int}_{(X, d)}\big(B_{(X, d)}(x_0, r)\big)         & \text{(by Remark \ref{1.2.6})}          \\
        \implies & B_{(X, d)}(x_0, r) \text{ is open in } (X, d).                               & \text{(by Proposition \ref{1.2.15}(a))}
    \end{align*}

    Let \(E = \{x \in X : d(x, x_0) \leq r\}\).
    Now we show that \(E\) is closed in \((X, d)\).
    By Proposition \ref{1.2.15}(b) we know that \(E\) is closed in \((X, d)\) iff \(E = \overline{E}_{(X, d)}\).
    By Proposition \ref{1.2.10}(c) we know that \(E \subseteq \overline{E}_{(X, d)}\).
    So we only need to show that \(\overline{E}_{(X, d)} \subseteq E\), or equivalently \(\overline{E}_{(X, d)} \setminus E = \emptyset\).
    Suppose for sake of contradiction that \(\overline{E}_{(X, d)} \setminus E \neq \emptyset\).
    Let \(y \in \overline{E}_{(X, d)} \setminus E\).
    By Proposition \ref{1.2.10}(c), \(\exists\ (y_n)_{n = 1}^\infty\) such that \(y_n \in E\) for all \(n \in \mathbf{Z}^+\) and \(\lim_{n \to \infty} d(y_n, y) = 0\).
    Since \(y \notin E\), we have \(d(y, x_0) > r\).
    Then \(d(y, x_0) - r > 0\) and we have
    \begin{align*}
                 & \exists\ N \in \mathbf{Z}^+ : \forall n \geq N,                                                             \\
                 & d(y_n, y) \leq \frac{d(y, x_0) - r}{2} < d(y, x_0) - r                & \text{(by Definition \ref{1.1.14})} \\
        \implies & \exists\ N \in \mathbf{Z}^+ : \forall n \geq N,                                                             \\
                 & r < d(y, x_0) - d(y_n, y) \leq d(y, x_0) + d(y_n, y) \leq d(y_n, x_0) & \text{(by Definition \ref{1.1.2})}  \\
        \implies & \exists\ N \in \mathbf{Z}^+ : \forall n \geq N, r < d(y_n, x_0).
    \end{align*}
    But \(d(y_n, x_0) > r\) means \(y_n \notin E\), a contradiction.
    Thus we must have \(\overline{E}_{(X, d)} \setminus E = \emptyset\), as desired.
\end{proof}

\begin{proof}{(d)}
    By Proposition \ref{1.2.15}(b) we know that \(\{x_0\}\) is closed in \((X, d)\) iff \(\{x_0\} = \overline{\{x_0\}}_{(X, d)}\).
    By Proposition \ref{1.2.10}(c) we know that \(\{x_0\} \subseteq \overline{\{x_0\}}_{(X, d)}\).
    So we only need to show that \(\overline{\{x_0\}}_{(X, d)} \subseteq \{x_0\}\).
    Let \(y \in \overline{\{x_0\}}_{(X, d)}\).
    By Proposition \ref{1.2.10}(c) we know that \(\exists\ (y_n)_{n = 1}^\infty\) such that \(y_n \in \{x_0\}\) for all \(n \in \mathbf{Z}^+\) and \(\lim_{n \to \infty} d(y_n, y) = 0\).
    But \(y_n \in \{x_0\}\) implies \(y_n = x_0\) for all \(n \in \mathbf{Z}^+\), thus
    \[
        \lim_{n \to \infty} d(y_n, y) = \lim_{n \to \infty} d(x_0, y) = d(x_0, y) = 0.
    \]
    By Definition \ref{1.1.2}(a) we have \(x_0 = y\).
    This means \(\overline{\{x_0\}}_{(X, d)} \subseteq \{x_0\}\), as desired.
\end{proof}

\begin{proof}{(e)}
    Since
    \begin{align*}
             & x_0 \in \text{int}_{(X, d)}(E)                                                                                         \\
        \iff & \exists\ r \in \mathbf{R}^+ : B_{(X, d)}(x_0, r) \subseteq E                      & \text{(by Definition \ref{1.2.5})} \\
        \iff & \exists\ r \in \mathbf{R}^+ : B_{(X, d)}(x_0, r) \cap (X \setminus E) = \emptyset                                      \\
        \iff & x_0 \in \text{ext}_{(X, d)}(X \setminus E),                                       & \text{(by Definition \ref{1.2.5})}
    \end{align*}
    we know that \(\text{int}_{(X, d)}(E) = \text{ext}_{(X, d)}(X \setminus E)\) for any subset \(E\) of \(X\).
    Then we have
    \begin{align*}
        \partial_{(X, d)}(E) & = X \setminus \big(\text{int}_{(X, d)}(E) \cup \text{ext}_{(X, d)}(E)\big)                         & \text{(by Definition \ref{1.2.5})} \\
                             & = X \setminus \big(\text{ext}_{(X, d)}(X \setminus E) \cup \text{int}_{(X, d)}(X \setminus E)\big)                                      \\
                             & = \partial_{(X, d)}(X \setminus E)                                                                 & \text{(by Definition \ref{1.2.5})}
    \end{align*}
    and
    \begin{align*}
             & E \text{ is open in } (X, d)                                                                     \\
        \iff & \partial_{(X, d)}(E) \cap E = \emptyset                    & \text{(by Definition \ref{1.2.12})} \\
        \iff & \partial_{(X, d)}(E) \subseteq (X \setminus E)             & \text{(by Definition \ref{1.2.5})}  \\
        \iff & \partial_{(X, d)}(X \setminus E) \subseteq (X \setminus E)                                       \\
        \iff & X \setminus E \text{ is closed in } (X, d).                & \text{(by Definition \ref{1.2.12})}
    \end{align*}
\end{proof}

\begin{proof}{(f)}
    Let \(I_n = \{i \in \mathbf{N} : 1 \leq i \leq n\}\).
    First suppose that \(E_i\) is open in \((X, d)\) for every \(i \in I_n\).
    Let \(x_0 \in \bigcap_{i \in I_n} E_i\).
    Then we have
    \begin{align*}
                 & x_0 \in \bigcap_{i \in I_n} E_i                                                                                                 \\
        \implies & \forall i \in I_n, x_0 \in E_i                                                                                                  \\
        \implies & \forall i \in I_n, x_0 \in \text{int}_{(X, d)}(E_i)                                   & \text{(by Proposition \ref{1.2.15}(a))} \\
        \implies & \forall i \in I_n, \exists\ r_i \in \mathbf{R}^+ : B_{(X, d)}(x_0, r_i) \subseteq E_i & \text{(by Definition \ref{1.2.5})}      \\
        \implies & \forall i \in I_n, B_{(X, d)}\big(x_0, \min_{j \in I_n}(r_j)\big) \subseteq E_i       & \text{(\(I_n\) is finite)}              \\
        \implies & B_{(X, d)}\big(x_0, \min_{j \in I_n}(r_j)\big) \subseteq \bigcap_{i \in I_n} E_i                                                \\
        \implies & x_0 \in \text{int}_{(X, d)}\bigg(\bigcap_{i \in I_n} E_i\bigg).                       & \text{(by Definition \ref{1.2.5})}
    \end{align*}
    Since \(x_0\) is arbitrary, we have
    \begin{align*}
                 & \bigcap_{i \in I_n} E_i \subseteq \text{int}_{(X, d)}\bigg(\bigcap_{i \in I_n} E_i\bigg)                                           \\
        \implies & \bigcap_{i \in I_n} E_i = \text{int}_{(X, d)}\bigg(\bigcap_{i \in I_n} E_i\bigg)         & \text{(by Remark \ref{1.2.6})}          \\
        \implies & \bigcap_{i \in I_n} E_i \text{ is open in } (X, d).                                      & \text{(by Proposition \ref{1.2.15}(a))}
    \end{align*}

    Now suppose that \(F_i\) is closed in \((X, d)\) for every \(i \in I_n\).
    Then we have
    \begin{align*}
                 & \forall i \in I_n, F_i \text{ is closed in } (X, d)                                                                                  \\
        \implies & \forall i \in I_n, X \setminus F_i \text{ is open in } (X, d)                              & \text{(by Proposition \ref{1.2.15}(e))} \\
        \implies & \bigcap_{i \in I_n} (X \setminus F_i) \text{ is open in } (X, d)                           & \text{(from prove above)}               \\
        \implies & X \setminus \bigg(\bigcap_{i \in I_n} (X \setminus F_i)\bigg) \text{ is closed in } (X, d) & \text{(by Proposition \ref{1.2.15}(e))} \\
        \implies & \bigcup_{i \in I_n} F_i \text{ is closed in } (X, d).
    \end{align*}
\end{proof}

\begin{proof}{(g)}
    First that \(E_\alpha\) is open in \((X, d)\) for every \(\alpha \in I\).
    Let \(x_0 \in \bigcup_{\alpha \in I} E_\alpha\).
    Then we have
    \begin{align*}
                 & x_0 \in \bigcup_{\alpha \in I} E_\alpha                                                                                              \\
        \implies & \exists\ \beta \in I : x_0 \in E_\beta                                                                                               \\
        \implies & \exists\ \beta \in I : x_0 \in \text{int}_{(X, d)}(E_\beta)                                & \text{(by Proposition \ref{1.2.15}(a))} \\
        \implies & \exists\ \beta \in I : \exists\ r \in \mathbf{R}^+ : B_{(X, d)}(x_0, r) \subseteq E_\beta  & \text{(by Definition \ref{1.2.5})}      \\
        \implies & \exists\ r \in \mathbf{R}^+ : B_{(X, d)}(x_0, r) \subseteq \bigcup_{\alpha \in I} E_\alpha                                           \\
        \implies & x_0 \in \text{int}_{(X, d)}\bigg(\bigcup_{\alpha \in I} E_\alpha\bigg).                    & \text{(by Definition \ref{1.2.5})}
    \end{align*}
    Since \(x_0\) is arbitrary, we have
    \begin{align*}
                 & \bigcup_{\alpha \in I} E_\alpha \subseteq \text{int}_{(X, d)}\bigg(\bigcup_{\alpha \in I} E_\alpha\bigg)                                           \\
        \implies & \bigcup_{\alpha \in I} E_\alpha = \text{int}_{(X, d)}\bigg(\bigcup_{\alpha \in I} E_\alpha\bigg)         & \text{(by Remark \ref{1.2.6})}          \\
        \implies & \bigcup_{\alpha \in I} E_\alpha \text{ is open in } (X, d).                                              & \text{(by Proposition \ref{1.2.15}(a))}
    \end{align*}

    Now suppose that \(F_\alpha\) is closed in \((X, d)\) for every \(\alpha \in I\).
    Then we have
    \begin{align*}
                 & \forall \alpha \in I, F_\alpha \text{ is closed in } (X, d)                                                                                  \\
        \implies & \forall \alpha \in I, X \setminus F_\alpha \text{ is open in } (X, d)                              & \text{(by Proposition \ref{1.2.15}(e))} \\
        \implies & \bigcup_{\alpha \in I} (X \setminus F_\alpha) \text{ is open in } (X, d)                                                                     \\
        \implies & X \setminus \bigg(\bigcup_{\alpha \in I} (X \setminus F_\alpha)\bigg) \text{ is closed in } (X, d) & \text{(by Proposition \ref{1.2.15}(e))} \\
        \implies & \bigcap_{\alpha \in I} F_\alpha \text{ is closed in } (X, d).
    \end{align*}
\end{proof}

\begin{proof}{(h)}
    We first show that \(\text{int}_{(X, d)}(E)\) is open in \((X, d)\).
    Let \(x_0 \in \text{int}_{(X, d)}(E)\).
    Then we have
    \begin{align*}
                 & x_0 \in \text{int}_{(X, d)}(E)                                                                                                           \\
        \implies & \exists\ r \in \mathbf{R}^+ : B_{(X, d)}(x_0, r) \subseteq E                                   & \text{(by Definition \ref{1.2.5})}      \\
        \implies & \exists\ r \in \mathbf{R}^+ : \forall y \in B_{(X, d)}(x_0, r), \exists\ r' \in \mathbf{R}^+ :                                           \\
                 & B_{(X, d)}(y, r') \subseteq B_{(X, d)}(x_0, r) \subseteq E                                     & \text{(by Proposition \ref{1.2.15}(c))} \\
        \implies & \exists\ r \in \mathbf{R}^+ : \forall y \in B_{(X, d)}(x_0, r), y \in \text{int}_{(X, d)}(E)   & \text{(by Definition \ref{1.2.5})}      \\
        \implies & \exists\ r \in \mathbf{R}^+ : B_{(X, d)}(x_0, r) \subseteq \text{int}_{(X, d)}(E)                                                        \\
        \implies & x_0 \in \text{int}_{(X, d)}\big(\text{int}_{(X, d)}(E)\big).                                   & \text{(by Definition \ref{1.2.5})}
    \end{align*}
    Since \(x_0\) is arbitrary, we have
    \begin{align*}
                 & \text{int}_{(X, d)}(E) \subseteq \text{int}_{(X, d)}\big(\text{int}_{(X, d)}(E)\big)                                           \\
        \implies & \text{int}_{(X, d)}(E) = \text{int}_{(X, d)}\big(\text{int}_{(X, d)}(E)\big)         & \text{(by Remark \ref{1.2.6})}          \\
        \implies & \text{int}_{(X, d)}(E) \text{ is open in } (X, d).                                   & \text{(by Proposition \ref{1.2.15}(a))}
    \end{align*}

    Next we show that if \(V \subseteq E\) and \((V, d)\) is open in \(X\), then \(V \subseteq \text{int}_{(X, d)}(E)\).
    \begin{align*}
                 & (V \subseteq E) \land \big(V \text{ is open in } (X, d)\big)                                           \\
        \implies & V = \text{int}_{(X, d)}(V) \subseteq E                       & \text{(by Proposition \ref{1.2.15}(a))} \\
        \implies & \forall x_0 \in V, \exists\ r \in \mathbf{R}^+ :                                                       \\
                 & B_{(X, d)}(x_0, r) \subseteq V \subseteq E                   & \text{(by Definition \ref{1.2.5})}      \\
        \implies & \forall x_0 \in V, x_0 \in \text{int}_{(X, d)}(E)            & \text{(by Definition \ref{1.2.5})}      \\
        \implies & V \subseteq \text{int}_{(X, d)}(E).
    \end{align*}

    Next we show that \(\overline{E}_{(X, d)}\) is closed in \((X, d)\).
    Let \(x_0 \in X \setminus \overline{E}_{(X, d)}\).
    Then we have
    \begin{align*}
                 & x_0 \in X \setminus \overline{E}_{(X, d)}                                                                                                    \\
        \implies & \exists\ r \in \mathbf{R}^+ : B_{(X, d)}(x_0, r) \cap E = \emptyset                                     & \text{(by Definition \ref{1.2.9})} \\
        \implies & \exists\ r \in \mathbf{R}^+ : \forall y \in B_{(X, d)}(x_0, r), \exists\ r' \in \mathbf{R}^+ :                                               \\
                 & \begin{cases}
                       B_{(X, d)}(y, r') \subseteq B_{(X, d)}(x_0, r) \\
                       B_{(X, d)}(y, r') \cap E = \emptyset
                   \end{cases}                                                       & \text{(by Proposition \ref{1.2.15}(c))}                                  \\
        \implies & \exists\ r \in \mathbf{R}^+ : \forall y \in B_{(X, d)}(x_0, r), y \in X \setminus \overline{E}_{(X, d)} & \text{(by Definition \ref{1.2.9})} \\
        \implies & \exists\ r \in \mathbf{R}^+ : B_{(X, d)}(x_0, r) \subseteq X \setminus \overline{E}_{(X, d)}                                                 \\
        \implies & x_0 \in \text{int}_{(X, d)}(X \setminus \overline{E}_{(X, d)}).                                         & \text{(by Definition \ref{1.2.5})}
    \end{align*}
    Since \(x_0\) is arbitrary, we have
    \begin{align*}
                 & X \setminus \overline{E}_{(X, d)} \subseteq \text{int}_{(X, d)}(X \setminus \overline{E}_{(X, d)})                                           \\
        \implies & X \setminus \overline{E}_{(X, d)} = \text{int}_{(X, d)}(X \setminus \overline{E}_{(X, d)})         & \text{(by Remark \ref{1.2.6})}          \\
        \implies & X \setminus \overline{E}_{(X, d)} \text{ is open in } (X, d)                                       & \text{(by Proposition \ref{1.2.15}(a))} \\
        \implies & \overline{E}_{(X, d)} \text{ is closed in } (X, d).                                                & \text{(by Proposition \ref{1.2.15}(e))}
    \end{align*}

    Finally we show that if \(E \subseteq K \subseteq X\) and \(K\) is closed in \((X, d)\), then \(\overline{E}_{(X, d)} \subseteq K\).
    \begin{align*}
                 & (E \subseteq K) \land \big(K \text{ is closed in } (X, d)\big)                                                                                        \\
        \implies & \forall x_0 \in \overline{E}_{(X, d)}, \forall r \in \mathbf{R}^+, B_{(X, d)}(x_0, r) \cap E \neq \emptyset & \text{(by Definition \ref{1.2.9})}      \\
        \implies & \forall x_0 \in \overline{E}_{(X, d)}, \forall r \in \mathbf{R}^+, B_{(X, d)}(x_0, r) \cap K \neq \emptyset & (E \subseteq K)                         \\
        \implies & \forall x_0 \in \overline{E}_{(X, d)}, x_0 \in \overline{K}_{(X, d)}                                        & \text{(by Definition \ref{1.2.9})}      \\
        \implies & \overline{E}_{(X, d)} \subseteq \overline{K}_{(X, d)}                                                                                                 \\
        \implies & \overline{E}_{(X, d)} \subseteq K.                                                                          & \text{(by Proposition \ref{1.2.15}(b))}
    \end{align*}
\end{proof}

\exercisesection

\begin{exercise}\label{ex 1.2.1}
    Verify the claims in Example \ref{1.2.8}.
\end{exercise}

\begin{proof}
    See Example \ref{1.2.8}.
\end{proof}

\begin{exercise}\label{ex 1.2.2}
    Prove Proposition \ref{1.2.10}.
\end{exercise}

\begin{proof}
    See Proposition \ref{1.2.10}.
\end{proof}

\begin{exercise}\label{ex 1.2.3}
    Prove Proposition \ref{1.2.15}.
\end{exercise}

\begin{proof}
    See Proposition \ref{1.2.15}.
\end{proof}

\begin{exercise}\label{ex 1.2.4}
    Let \((X, d)\) be a metric space, \(x_0\) be a point in \(X\), and \(r > 0\).
    Let \(B\) be the open ball \(B \coloneqq B(x_0, r) = \{x \in X : d(x, x_0) < r\}\), and let \(C\) be the closed ball \(C \coloneqq \{x \in X : d(x, x_0) \leq r\}\).
    \begin{enumerate}
        \item Show that \(\overline{B} \subseteq C\).
        \item Give an example of a metric space \((X, d)\), a point \(x_0\), and a radius \(r > 0\) such that \(\overline{B}\) is \emph{not} equal to \(C\).
    \end{enumerate}
\end{exercise}

\begin{proof}{(a)}
    Since \(B \subseteq C\) and \(C\) is closed in \((X, d)\), by Proposition \ref{1.2.15}(h) we have \(\overline{B}_{(X, d)} \subseteq C\).
\end{proof}

\begin{proof}{(b)}
    Let \(X = \mathbf{R}\) and let \(d = d_{\text{disc}}\) be the metric function.
    By Exercise \ref{ex 1.1.11} we know that \((\mathbf{R}, d_{\text{disc}})\) is a metric space.
    Let \(B = B_{(\mathbf{R}, d_{\text{disc}})}(0, 1)\) and let \(C = \{x \in \mathbf{R} : d_{\text{disc}}(x, 0) \leq 1\}\).
    Then we know that \(B = \{0\}\) and \(C = \mathbf{R}\).
    But by Example \ref{1.2.8} we have \(\overline{B} = B \neq C\).
\end{proof}