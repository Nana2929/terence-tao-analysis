\section{Cauchy sequences and complete metric spaces}\label{sec 1.4}

\begin{definition}[Subsequences]\label{1.4.1}
    Suppose that \((x^{(n)})_{n = m}^\infty\) is a sequence of points in a metric space \((X, d)\).
    Suppose that \(n_1, n_2, n_3, \dots\) is an increasing sequence of integers which are at least as large as \(m\), thus
    \[
        m \leq n_1 < n_2 < n_3 < \dots.
    \]
    Then we call the sequence \((x^{(n_j)})_{j = 1}^\infty\) a \emph{subsequence} of the original sequence \((x^{(n)})_{n = m}^\infty\).
\end{definition}

\setcounter{theorem}{2}
\begin{lemma}\label{1.4.3}
    Let \((x^{(n)})_{n = m}^\infty\) be a sequence in \((X, d)\) which converges to some limit \(x_0\).
    Then every subsequence \((x^{(n_j)})_{j = 1}^\infty\) of that sequence also converges to \(x_0\).
\end{lemma}

\begin{proof}
    By Definition \ref{1.1.14} we have \(\lim_{n \to \infty} d(x^{(n)}, x_0) = 0\), which means \(\forall\ \varepsilon \in \mathbf{R}^+\), \(\exists\ N \in \mathbf{N}\) and \(N \geq m\) such that
    \[
        \forall\ n \geq N, d(x^{(n)}, x_0) \leq \varepsilon.
    \]
    Let \((x^{(n_j)})_{j = 1}^\infty\) be any subsequence of \((x^{(n)})_{n = m}^\infty\).
    Since \(\forall\ j \in \mathbf{N}\), we have \(n_j \geq j\), we must have \(n_N \geq N\).
    This means \(\forall\ n_j \geq n_N \geq N\), we have \(d(x^{(n_j)}, x_0) \leq \varepsilon\).
    Thus by Definition \ref{1.1.14} \(\lim_{j \to \infty} d(x^{(n_j)}, x_0) = 0\).
\end{proof}

\begin{definition}[Limit points]\label{1.4.4}
    Suppose that \((x^{(n)})_{n = m}^\infty\) is a sequence of points in a metric space \((X, d)\), and let \(L \in X\).
    We say that \(L\) is a \emph{limit point} of \((x^{(n)})_{n = m}^\infty\) iff for every \(N \geq m\) and \(\varepsilon > 0\) there exists an \(n \geq N\) such that \(d(x^{(n)}, L) \leq \varepsilon\).
\end{definition}

\begin{proposition}\label{1.4.5}
    Let \((x^{(n)})_{n = m}^\infty\) be a sequence of points in a metric space \((X, d)\), and let \(L \in X\).
    Then the following are equivalent:
    \begin{itemize}
        \item \(L\) is a limit point of \((x^{(n)})_{n = m}^\infty\).
        \item There exists a subsequence \((x^{(n_j)})_{j = 1}^\infty\) of the original sequence \((x^{(n)})_{n = m}^\infty\) which converges to \(L\).
    \end{itemize}
\end{proposition}

\begin{proof}
    We first show that if \(L\) is a limit point of \((x^{(n)})_{n = 0}^\infty\), then there exists a subsequence of \((x^{(n)})_{n = 0}^\infty\) which converges to \(L\).
    Let \(f : \mathbf{N} \to \mathbf{N}\) be a function where \(f(0) = 0\) and \(f(n_j) = \min\{n > n_{j - 1} : d(x^{(n)}, L) \leq 1 / j\}\).
    Since \(L\) is a limit point of \((x^{(n)})_{n = 0}^\infty\), by Definition \ref{1.4.4} \(\forall\ \varepsilon \in \mathbf{R}^+\), \(\forall\ n_j \geq m\), \(\exists\ n \geq n_j\) such that \(d(x^{(n)}, L) \leq \varepsilon\).
    In particular, \(d(x^{(n)}, L) \leq 1 / j\).
    Thus such \(f\) exists and \((x^{(n_j)})_{j = 1}^\infty\) is a subsequence of \((x^{(n)})_{n = m}^\infty\).
    Since \(\forall\ j \geq 1\) we have \(d(x^{(n_j)}, L) \leq 1 / j\), by squeeze test we have
    \[
        0 = \lim_{j \to \infty} 0 \leq \lim_{j \to \infty} d(x^{(n_j)}, L) \leq \lim_{j \to \infty} 1 / j = 0
    \]
    and thus by Definition \ref{1.1.14} the sequence \((x^{(n_j)})_{n = 1}^\infty\) converges to \(L\).

    Now we show that if a subsequence of \((x^{(n)})_{n = 0}^\infty\) converges to \(L\), then \(L\) is a limit point of \((x^{(n)})_{n = 0}^\infty\).
    Let \((a_n)_{n = 1}^\infty\) be a subsequence of \((x^{(n)})_{n = 0}^\infty\) and \(\lim_{n \to \infty} d(a_n, L) = 0\).
    Since \(\lim_{n \to \infty} d(a_n, L) = 0\), by Definition \ref{1.1.14} \(\forall\ \varepsilon \in \mathbf{R}^+\), \(\exists\ n \geq 0\) such that \(d(a_n, L) \leq \varepsilon\).
    This means \(\forall\ N \in \mathbf{N}\) and \(N \geq 1\), \(\exists\ n \geq N\) such that \(d(a_n, L) = d(x^{(n_j)}, L) \leq \varepsilon\).
    Thus by Definition \ref{1.4.4} \(L\) is a limit point of \((x^{(n)})_{n = 0}^\infty\), and we are done.
\end{proof}

\begin{definition}[Cauchy sequences]\label{1.4.6}
    Let \((x^{(n)})_{n = m}^\infty\) be a sequence of points in a metric space \((X, d)\).
    We say that this sequence is a \emph{Cauchy sequence} iff for every \(\varepsilon > 0\), there exists an \(N \geq m\) such that \(d(x^{(j)}, x^{(k)}) < \varepsilon\) for all \(j, k \geq N\).
\end{definition}

\begin{lemma}[Convergent sequences are Cauchy sequences]\label{1.4.7}
    Let \((x^{(n)})_{n = m}^\infty\) be a sequence in \((X, d)\) which converges to some limit \(x_0\).
    Then \((x^{(n)})_{n = m}^\infty\) is also a Cauchy sequence.
\end{lemma}

\begin{proof}
    Since \(\lim_{n \to \infty} d(x^{(n)}, x_0) = 0\), by Definition \ref{1.1.14} we know that \(\forall\ \varepsilon \in \mathbf{R}^+\), \(\exists\ N \in \mathbf{N}\) and \(N \geq m\) such that \(\forall\ n \geq N\), we have \(d(x^{(n)}, x_0) \leq \varepsilon / 2\).
    Let \(k \in \mathbf{N}\) and \(k \geq N\).
    Then we have
    \begin{align*}
        d(x^{(n)}, x^{(k)}) & \leq d(x^{(n)}, x_0) + d(x_0, x^{(k)}) & \text{(by Definition \ref{1.1.2}(d))} \\
                            & = d(x^{(n)}, x_0) + d(x^{(k)}, x_0)    & \text{(by Definition \ref{1.1.2}(c))} \\
                            & \leq \varepsilon / 2 + \varepsilon / 2                                         \\
                            & = \varepsilon
    \end{align*}
    and by Definition \ref{1.4.6} \((x^{(n)})_{n = m}^\infty\) is a Cauchy sequence.
\end{proof}