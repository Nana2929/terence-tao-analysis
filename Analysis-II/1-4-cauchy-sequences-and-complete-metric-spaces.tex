\section{Cauchy sequences and complete metric spaces}\label{sec 1.4}

\begin{definition}[Subsequences]\label{1.4.1}
    Suppose that \((x^{(n)})_{n = m}^\infty\) is a sequence of points in a metric space \((X, d)\).
    Suppose that \(n_1, n_2, n_3, \dots\) is an increasing sequence of integers which are at least as large as \(m\), thus
    \[
        m \leq n_1 < n_2 < n_3 < \dots.
    \]
    Then we call the sequence \((x^{(n_j)})_{j = 1}^\infty\) a \emph{subsequence} of the original sequence \((x^{(n)})_{n = m}^\infty\).
\end{definition}

\setcounter{theorem}{2}
\begin{lemma}\label{1.4.3}
    Let \((x^{(n)})_{n = m}^\infty\) be a sequence in \((X, d)\) which converges to some limit \(x_0\).
    Then every subsequence \((x^{(n_j)})_{j = 1}^\infty\) of that sequence also converges to \(x_0\).
\end{lemma}

\begin{proof}
    By Definition \ref{1.1.14} we have \(\lim_{n \to \infty} d(x^{(n)}, x_0) = 0\), which means \(\forall\ \varepsilon \in \mathbf{R}^+\), \(\exists\ N \in \mathbf{N}\) and \(N \geq m\) such that
    \[
        \forall\ n \geq N, d(x^{(n)}, x_0) \leq \varepsilon.
    \]
    Let \((x^{(n_j)})_{j = 1}^\infty\) be any subsequence of \((x^{(n)})_{n = m}^\infty\).
    Since \(\forall\ j \in \mathbf{N}\), we have \(n_j \geq j\), we must have \(n_N \geq N\).
    This means \(\forall\ n_j \geq n_N \geq N\), we have \(d(x^{(n_j)}, x_0) \leq \varepsilon\).
    Thus by Definition \ref{1.1.14} \(\lim_{j \to \infty} d(x^{(n_j)}, x_0) = 0\).
\end{proof}

\begin{definition}[Limit points]\label{1.4.4}
    Suppose that \((x^{(n)})_{n = m}^\infty\) is a sequence of points in a metric space \((X, d)\), and let \(L \in X\).
    We say that \(L\) is a \emph{limit point} of \((x^{(n)})_{n = m}^\infty\) iff for every \(N \geq m\) and \(\varepsilon > 0\) there exists an \(n \geq N\) such that \(d(x^{(n)}, L) \leq \varepsilon\).
\end{definition}

\begin{proposition}\label{1.4.5}
    Let \((x^{(n)})_{n = m}^\infty\) be a sequence of points in a metric space \((X, d)\), and let \(L \in X\).
    Then the following are equivalent:
    \begin{itemize}
        \item \(L\) is a limit point of \((x^{(n)})_{n = m}^\infty\).
        \item There exists a subsequence \((x^{(n_j)})_{j = 1}^\infty\) of the original sequence \((x^{(n)})_{n = m}^\infty\) which converges to \(L\).
    \end{itemize}
\end{proposition}

\begin{proof}
    We first show that if \(L\) is a limit point of \((x^{(n)})_{n = 0}^\infty\), then there exists a subsequence of \((x^{(n)})_{n = 0}^\infty\) which converges to \(L\).
    Let \(f : \mathbf{N} \to \mathbf{N}\) be a function where \(f(0) = 0\) and \(f(n_j) = \min\{n > n_{j - 1} : d(x^{(n)}, L) \leq 1 / j\}\).
    Since \(L\) is a limit point of \((x^{(n)})_{n = 0}^\infty\), by Definition \ref{1.4.4} \(\forall\ \varepsilon \in \mathbf{R}^+\), \(\forall\ n_j \geq m\), \(\exists\ n \geq n_j\) such that \(d(x^{(n)}, L) \leq \varepsilon\).
    In particular, \(d(x^{(n)}, L) \leq 1 / j\).
    Thus such \(f\) exists and \((x^{(n_j)})_{j = 1}^\infty\) is a subsequence of \((x^{(n)})_{n = m}^\infty\).
    Since \(\forall\ j \geq 1\) we have \(d(x^{(n_j)}, L) \leq 1 / j\), by squeeze test we have
    \[
        0 = \lim_{j \to \infty} 0 \leq \lim_{j \to \infty} d(x^{(n_j)}, L) \leq \lim_{j \to \infty} 1 / j = 0
    \]
    and thus by Definition \ref{1.1.14} the sequence \((x^{(n_j)})_{n = 1}^\infty\) converges to \(L\).

    Now we show that if a subsequence of \((x^{(n)})_{n = 0}^\infty\) converges to \(L\), then \(L\) is a limit point of \((x^{(n)})_{n = 0}^\infty\).
    Let \((a_n)_{n = 1}^\infty\) be a subsequence of \((x^{(n)})_{n = 0}^\infty\) and \(\lim_{n \to \infty} d(a_n, L) = 0\).
    Since \(\lim_{n \to \infty} d(a_n, L) = 0\), by Definition \ref{1.1.14} \(\forall\ \varepsilon \in \mathbf{R}^+\), \(\exists\ n \geq 0\) such that \(d(a_n, L) \leq \varepsilon\).
    This means \(\forall\ N \in \mathbf{N}\) and \(N \geq 1\), \(\exists\ n \geq N\) such that \(d(a_n, L) = d(x^{(n_j)}, L) \leq \varepsilon\).
    Thus by Definition \ref{1.4.4} \(L\) is a limit point of \((x^{(n)})_{n = 0}^\infty\), and we are done.
\end{proof}

\begin{definition}[Cauchy sequences]\label{1.4.6}
    Let \((x^{(n)})_{n = m}^\infty\) be a sequence of points in a metric space \((X, d)\).
    We say that this sequence is a \emph{Cauchy sequence} iff for every \(\varepsilon > 0\), there exists an \(N \geq m\) such that \(d(x^{(j)}, x^{(k)}) \leq \varepsilon\) for all \(j, k \geq N\).
\end{definition}

\begin{lemma}[Convergent sequences are Cauchy sequences]\label{1.4.7}
    Let \((x^{(n)})_{n = m}^\infty\) be a sequence in \((X, d)\) which converges to some limit \(x_0\).
    Then \((x^{(n)})_{n = m}^\infty\) is also a Cauchy sequence.
\end{lemma}

\begin{proof}
    Since \(\lim_{n \to \infty} d(x^{(n)}, x_0) = 0\), by Definition \ref{1.1.14} we know that \(\forall\ \varepsilon \in \mathbf{R}^+\), \(\exists\ N \in \mathbf{N}\) and \(N \geq m\) such that \(\forall\ n \geq N\), we have \(d(x^{(n)}, x_0) \leq \varepsilon / 2\).
    Let \(k \in \mathbf{N}\) and \(k \geq N\).
    Then we have
    \begin{align*}
        d(x^{(n)}, x^{(k)}) & \leq d(x^{(n)}, x_0) + d(x_0, x^{(k)}) & \text{(by Definition \ref{1.1.2}(d))} \\
                            & = d(x^{(n)}, x_0) + d(x^{(k)}, x_0)    & \text{(by Definition \ref{1.1.2}(c))} \\
                            & \leq \varepsilon / 2 + \varepsilon / 2                                         \\
                            & = \varepsilon
    \end{align*}
    and by Definition \ref{1.4.6} \((x^{(n)})_{n = m}^\infty\) is a Cauchy sequence.
\end{proof}

\setcounter{theorem}{8}
\begin{lemma}\label{1.4.9}
    Let \((x^{(n)})_{n = m}^\infty\) be a Cauchy sequence in \((X, d)\).
    Suppose that there is some subsequence \((x^{(n_j)})_{j = 1}^\infty\) of this sequence which converges to a limit \(x_0\) in \(X\).
    Then the original sequence \((x^{(n)})_{n = m}^\infty\) also converges to \(x_0\).
\end{lemma}

\begin{proof}
    Since \(\lim_{j \to \infty} d(x^{(n_j)}, x_0) = 0\), by Definition \ref{1.1.14} we know that \(\forall\ \varepsilon \in \mathbf{R}^+\), \(\exists\ N_1 \in \mathbf{N}\) and \(N_1 \geq 1\) such that \(\forall\ j \geq N_1\), we have \(d(x^{(n_j)}, x_0) \leq \varepsilon / 2\).
    Since \((x^{(n)})_{n = m}^\infty\) is a Cauchy sequence, by Definition \ref{1.4.6} we know that \(\exists\ N_2 \in \mathbf{N}\) and \(N_2 \geq m\) such that \(\forall\ i, k \in \mathbf{N}\) and \(i, k \geq N\) we have \(d(x^{(i)}, x^{(k)}) \leq \varepsilon / 2\).
    Let \(N = \max(N_1, N_2)\).
    Then \(\forall\ i, j \geq N\), we have
    \begin{align*}
        d(x^{(i)}, x_0) & \leq d(x^{(i)}, x^{(n_j)}) + d(x^{(n_j)}, x_0) & \text{(by Definition \ref{1.1.2}(d))} \\
                        & \leq \varepsilon / 2 + \varepsilon / 2         & (n_j \geq j \geq N)                   \\
                        & = \varepsilon
    \end{align*}
    and thus by Definition \ref{1.1.14} \(\lim_{n \to \infty} d(x^{(n)}, x_0) = 0\).
\end{proof}

\begin{definition}[Complete metric spaces]\label{1.4.10}
    A metric space \((X, d)\) is said to be \emph{complete} iff every Cauchy sequence in \((X, d)\) is in fact convergent in \((X, d)\).
\end{definition}

\setcounter{theorem}{11}
\begin{proposition}\label{1.4.12}
    \begin{enumerate}
        \item Let \((X, d)\) be a metric space, and let \((Y, d|_{Y \times Y})\) be a subspace of \((X, d)\).
              If \((Y, d|_{Y \times Y})\) is complete, then \(Y\) must be closed in \(X\).
        \item Conversely, suppose that \((X, d)\) is a complete metric space, and \(Y\) is a closed subset of \(X\).
              Then the subspace \((Y, d|_{Y \times Y})\) is also complete.
    \end{enumerate}
\end{proposition}

\begin{proof}
    We first show that the statement (a) is true.
    Suppose that \((X, d)\) is a metric space and \((Y, d|_{Y \times Y})\) is a complete subspace of \((X, d)\).
    Let \(x_0 \in \partial Y\).
    By Proposition \ref{1.2.10} \(\exists\ (x^{(n)})_{n = m}^\infty\) in \(Y\) such that \(\lim_{n \to \infty} d(x^{(n)}, x_0) = 0\).
    Since \(Y\) is complete, by Definition \ref{1.4.10} we know that \(x_0 \in Y\).
    Thus we have \(\partial Y \subseteq Y\) and by Definition \ref{1.2.12} \(Y\) is closed.

    Now we show that the statement (b) is true.
    Suppose that \((X, d)\) is a complete metric space and \(Y\) is a closed subset of \(X\).
    Let \((x^{(n)})_{n = m}^\infty\) be a Cauchy sequence in \(Y\).
    Since \(Y \subseteq X\), we know that \((x^{(n)})_{n = m}^\infty\) is also in \(X\).
    Since \(X\) is complete, we know that \(\lim_{n \to \infty} d(x^{(n)}, x_0) = 0\) for some \(x_0 \in X\).
    Since \(\lim_{n \to \infty} d(x^{(n)}, x_0) = 0\), we know that \(x_0\) is an adherent point of \(Y\).
    But \(Y\) is closed, thus by Proposition \ref{1.2.15}(b) we know that \(x_0 \in Y\), and by Definition \ref{1.4.10} \((Y, d|_{Y \times Y})\) is complete.
\end{proof}

\exercisesection

\begin{exercise}\label{ex 1.4.1}
    Prove Lemma \ref{1.4.3}.
\end{exercise}

\begin{proof}
    See Lemma \ref{1.4.3}.
\end{proof}

\begin{exercise}\label{ex 1.4.2}
    Prove Proposition 1.4.5.
\end{exercise}

\begin{proof}
    See Proposition \ref{1.4.5}.
\end{proof}

\begin{exercise}\label{ex 1.4.3}
    Prove Lemma \ref{1.4.7}.
\end{exercise}

\begin{proof}
    See Lemma \ref{1.4.7}.
\end{proof}

\begin{exercise}\label{ex 1.4.4}
    Prove Lemma \ref{1.4.9}.
\end{exercise}

\begin{proof}
    See Lemma \ref{1.4.9}.
\end{proof}

\begin{exercise}\label{ex 1.4.5}
    Let \((x^{(n)})_{n = m}^\infty\) be a sequence of points in a metric space \((X, d)\), and let \(L \in X\).
    Show that if \(L\) is a limit point of the sequence \((x^{(n)})_{n = m}^\infty\), then \(L\) is an adherent point of the set \(\{x^{(n)} : n \geq m\}\).
    Is the converse true?
\end{exercise}

\begin{proof}
    We first show that if \(L\) is a limit point of \((x^{(n)})_{n = m}^\infty\), then \(L\) is an adherent point of \(\{x^{(n)} : n \geq m\}\).
    Suppose that \(L\) is a limit point of \((x^{(n)})_{n = m}^\infty\).
    Let \(E = \{x^{(n)} : n \geq m\}\).
    By Proposition \ref{1.4.5} we know that \(\exists\ (x^{(n_j)})_{j = 1}^\infty\) in \(E\) such that \(\lim_{j \to \infty} d(x^{(n_j)}, L) = 0\).
    Thus by Proposition \ref{1.2.10} \(L\) is an adherent point of \(E\).

    Now we show that if \(L\) is an adherent point of \(\{x^{(n)} : n \geq m\}\), then \(L\) may not be a limit point of \((x^{(n)})_{n = m}^\infty\).
    Let \(x^{(n)} = 1 / n\) and let \(E = \{x^{(n)} : n \geq m\}\).
    Then by Definition \ref{1.2.9} we know that \(1\) is an adherent point of \(E\).
    But by Definition \ref{1.4.4} we know that \(1\) is not an limit point of \(E\).
    Thus if \(L\) is an adherent point of \(\{x^{(n)} : n \geq m\}\), then \(L\) may not be a limit point of \((x^{(n)})_{n = m}^\infty\).
\end{proof}

\begin{exercise}\label{ex 1.4.6}
    Show that every Cauchy sequence can have at most one limit point.
\end{exercise}

\begin{proof}
    Suppose for sake of contradiction that there exists a Cauchy sequence \((x^{(n)})_{n = m}^\infty\) in some metric space \((X, d)\) which has two limit points \(L\) and \(L'\).
    Then by Lemma \ref{1.4.5} \(\exists\ (x^{(n_i)})_{i = 1}^\infty, (x^{(n_j)})_{j = 1}^\infty,\) which converges to \(L\) and \(L'\) respectively.
    Since \((x^{(n)})_{n = m}^\infty\) is a Cauchy sequence, by Lemma \ref{1.4.9} we know that \((x^{(n)})_{n = m}^\infty\) converges to \(L\) and \(L'\), which contradict to Proposition \ref{1.1.20}.
    Thus every Cauchy sequence can have at most one limit point.
\end{proof}

\begin{exercise}\label{ex 1.4.7}
    Prove Proposition \ref{1.4.12}.
\end{exercise}

\begin{proof}
    See Proposition \ref{1.4.12}.
\end{proof}

\begin{exercise}\label{ex 1.4.8}
    The following construction generalizes the construction of the reals from the rationals in Chapter 5, allowing one to view any metric space as a subspace of a complete metric space.
    In what follows we let \((X, d)\) be a metric space.
    \begin{enumerate}
        \item Given any Cauchy sequence \((x^{(n)})_{n = m}^\infty\) in \(X\), we introduce the \emph{formal limit} \\
              \(\text{LIM}_{n \to \infty} x_n\).
              We say that two formal limits \(\text{LIM}_{n \to \infty} x_n\) and \(\text{LIM}_{n \to \infty} y_n\) are equal if \(\text{lim}_{n \to \infty} d(x_n, y_n)\) is equal to zero.
              Show that this equality relation obeys the reflexive, symmetry, and transitive axioms.
        \item Let \(\overline{X}\) be the space of all formal limits of Cauchy sequences in \(X\), with the above equality relation.
              Define a metric \(d_{\overline{X}} : \overline{X} \times \overline{X} \to [0, \infty)\) by setting
              \[
                  d_{\overline{X}}(\text{LIM}_{n \to \infty} x_n, \text{LIM}_{n \to \infty} y_n) \coloneqq \lim_{n \to \infty} d(x_n, y_n).
              \]
              Show that this function is well-defined (this means not only that the limit \\
              \(\lim_{n \to \infty} d(x_n, y_n)\) exists, but also that the axiom of substitution is obeyed;
              cf. Lemma 5.3.7), and gives \(\overline{X}\) the structure of a metric space.
        \item Show that the metric space \((\overline{X}, d_{\overline{X}})\) is complete.
        \item We identify an element \(x \in X\) with the corresponding formal limit \(\text{LIM}_{n \to \infty} x\) in \(X\);
              show that this is legitimate by verifying that \(x = y \iff \text{LIM}_{n \to \infty} x = \text{LIM}_{n \to \infty} y\).
              With this identification, show that \(d(x, y) = d_{\overline{X}}(x, y)\), and thus \((X, d)\) can now be thought of as a subspace of \((\overline{X}, d_{\overline{X}})\).
        \item Show that the closure of \(X\) in \(\overline{X}\) is \(\overline{X}\) (which explains the choice of notation \(\overline{X}\)).
        \item Show that the formal limit agrees with the actual limit, thus if \((x_n)_{n = 1}^\infty\) is any Cauchy sequence in \(X\), then we have \(\lim_{n \to \infty} x_n = \text{LIM}_{n \to \infty} x_n\) in \(\overline{X}\).
    \end{enumerate}
\end{exercise}

\begin{proof}{(a)}
    Let \((x^{(n)})_{n = m}^\infty\), \((y^{(n)})_{n = m}^\infty\), \((z^{(n)})_{n = m}^\infty\) be Cauchy sequences in \(X\).

    For reflexive:
    By Definition \ref{1.1.2}(a) we have \(\lim_{n \to \infty} d(x_n, x_n) = \lim_{n \to \infty} 0 = 0\), thus \(\text{LIM}_{n \to \infty} x_n = \text{LIM}_{n \to \infty} x_n\).

    For symmetry:
    By Definition \ref{1.1.2}(c) we have \(\lim_{n \to \infty} d(x_n, y_n) = \lim_{n \to \infty} d(y_n, x_n)\), thus \(\text{LIM}_{n \to \infty} x_n = \text{LIM}_{n \to \infty} y_n\) iff \(\text{LIM}_{n \to \infty} y_n = \text{LIM}_{n \to \infty} x_n\).

    For transitive:
    We have
    \begin{align*}
                 & (\text{LIM}_{n \to \infty} x_n = \text{LIM}_{n \to \infty} y_n) \land (\text{LIM}_{n \to \infty} y_n = \text{LIM}_{n \to \infty} z_n)                                      \\
        \implies & (\lim_{n \to \infty} d(x_n, y_n) = 0) \land (\lim_{n \to \infty} d(y_n, z_n) = 0)                                                                                          \\
        \implies & \lim_{n \to \infty} d(x_n, y_n) + d(y_n, z_n) = 0                                                                                                                          \\
        \implies & 0 \leq \lim_{n \to \infty} d(x_n, z_n) \leq \lim_{n \to \infty} d(x_n, y_n) + d(y_n, z_n) = 0                                         & \text{(by Definition \ref{1.1.2})} \\
        \implies & \lim_{n \to \infty} d(x_n, z_n) = 0                                                                                                   & \text{(by squeeze test)}           \\
        \implies & \text{LIM}_{n \to \infty} x_n = \text{LIM}_{n \to \infty} z_n.
    \end{align*}
\end{proof}

\begin{proof}{(b)}
    Let \((x^{(n)})_{n = m}^\infty\), \((y^{(n)})_{n = m}^\infty\) be Cauchy sequences in \(X\) with formal limits in \(\overline{X}\).
    We first show that the limit
    \[
        d_{\overline{X}}(\text{LIM}_{n \to \infty} x_n, \text{LIM}_{n \to \infty} y_n) \coloneqq \lim_{n \to \infty} d(x_n, y_n)
    \]
    exists.
    Let \((a_n)_{n = m}^\infty\) be the sequence \(a_n = d(x_n, y_n)\).
    To show that the above limit exists, it will suffice to show that \((a_n)_{n = m}^\infty\) converges in \(\mathbf{R}\).
    Let \(\varepsilon \in \mathbf{R}^+\).
    Since \((x^{(n)})_{n = m}^\infty\) is a Cauchy sequence, by Definition \ref{1.4.6} we know that \(\exists\ N_1 \in \mathbf{N}\) and \(N_1 \geq m\) such that \(\forall\ j, k \in \mathbf{N}\) and \(j, k \geq N_1\) we have \(d(x_j, x_k) \leq \varepsilon / 2\).
    Similarly, \(\exists\ N_2 \in \mathbf{N}\) and \(N_2 \geq m\) such that \(\forall\ j, k \in \mathbf{N}\) and \(j, k \geq N_2\) we have \(d(y_j, y_k) \leq \varepsilon / 2\).
    Let \(N = \max(N_1, N_2)\).
    Then \(\forall\ j, k \geq N\), we have
    \begin{align*}
        \abs*{a_j - a_k} & = \abs*{d(x_j, y_j) - d(x_k, y_k)}                                                                               \\
                         & = \abs*{d(x_j, y_j) + d(y_j, x_k) - d(y_j, x_k) - d(x_k, y_k)}                                                   \\
                         & \leq \abs*{d(x_j, y_j) + d(y_j, x_k)} + \abs*{d(y_j, x_k) + d(x_k, y_k)}                                         \\
                         & = d(x_j, y_j) + d(y_j, x_k) + d(y_j, x_k) + d(x_k, y_k)                                                          \\
                         & \leq d(x_j, x_k) + d(y_j, y_k)                                           & \text{(by Definition \ref{1.1.2}(d))} \\
                         & \leq \varepsilon / 2 + \varepsilon / 2                                                                           \\
                         & = \varepsilon
    \end{align*}
    and thus \((a_n)_{n = m}^\infty\) is a Cauchy sequence in \(\mathbf{R}\) and we know that \((a_n)_{n = m}^\infty\) converges in \(\mathbf{R}\).

    Next we show that \(d_{\overline{X}}\) obeys the axiom of substitution.
    Let \((z^{(n)})_{n = m}^\infty\) be a Cauchy sequence in \(X\) such that \(\text{LIM}_{n \to \infty} x_n = \text{LIM}_{n \to \infty} z_n\).
    Then we have
    \begin{align*}
                 & \begin{cases}
            d(x_n, y_n) \leq d(x_n, z_n) + d(z_n, y_n) \\
            d(z_n, y_n) \leq d(x_n, y_n) + d(x_n, z_n)
        \end{cases}                                                                                           & \text{(by Definition \ref{1.1.2}(c)(d))} \\
        \implies & \begin{cases}
            d(x_n, y_n) - d(z_n, y_n) \leq d(x_n, z_n) \\
            d(z_n, y_n) - d(x_n, y_n) \leq d(x_n, z_n)
        \end{cases}                                                                                                                                      \\
        \implies & 0 \leq \abs*{d(x_n, y_n) - d(z_n, y_n)} \leq d(x_n, z_n)                                                                                                        \\
        \implies & \lim_{n \to \infty} 0 \leq \lim_{n \to \infty} \abs*{d(x_n, y_n) - d(z_n, y_n)} \leq \lim_{n \to \infty} d(x_n, z_n)                                            \\
        \implies & 0 \leq \lim_{n \to \infty} \abs*{d(x_n, y_n) - d(z_n, y_n)} \leq 0                                                   & \text{(by Exercise \ref{ex 1.4.8}(a))}   \\
        \implies & \lim_{n \to \infty} \abs*{d(x_n, y_n) - d(z_n, y_n)} = 0                                                             & \text{(by squeeze test)}                 \\
        \implies & \lim_{n \to \infty} d(x_n, y_n) - d(z_n, y_n) = 0                                                                                                               \\
        \implies & \lim_{n \to \infty} d(x_n, y_n) = \lim_{n \to \infty} d(z_n, y_n)
    \end{align*}
    and thus \(d_{\overline{X}}(\text{LIM}_{n \to \infty} x_n, \text{LIM}_{n \to \infty} y_n) = d_{\overline{X}}(\text{LIM}_{n \to \infty} z_n, \text{LIM}_{n \to \infty} y_n)\).

    Now we show that \((\overline{X}, d_{\overline{X}})\) is a metric space.
    Let \((x^{(n)})_{n = m}^\infty\), \((y^{(n)})_{n = m}^\infty\), \((z^{(n)})_{n = m}^\infty\) be Cauchy sequences in \(X\) with formal limits \(a, b, c\) in \(\overline{X}\) respectively.
    For identify:
    We have
    \[
        d_{\overline{X}}(a, a) = \lim_{n \to \infty} d(x_n, x_n) = 0.
    \]
    For positivity:
    If \(a \neq b\), then by Exercise \ref{ex 1.4.8}(a) we have
    \[
        d_{\overline{X}}(a, b) = \lim_{n \to \infty} d(x_n, y_n) \neq 0.
    \]
    For symmetry:
    By Definition \ref{1.1.2}(c) we have
    \[
        d_{\overline{X}}(a, b) = \lim_{n \to \infty} d(x_n, y_n) = \lim_{n \to \infty} d(y_n, x_n) = d_{\overline{X}}(b, a).
    \]
    For transitive:
    We have
    \begin{align*}
        d_{\overline{X}}(a, c) & = \lim_{n \to \infty} d(x_n, z_n)                                   \\
                               & \leq \lim_{n \to \infty} d(x_n, y_n) + d(y_n, z_n)                  \\
                               & = \lim_{n \to \infty} d(x_n, y_n) + \lim_{n \to \infty} d(y_n, z_n) \\
                               & = d_{\overline{X}}(a, b) + d_{\overline{X}}(b, c).
    \end{align*}
    Thus by Definition \ref{1.1.2} \((\overline{X}, d_{\overline{X}})\) is a metric space.
\end{proof}

\begin{proof}{(c)}
    To show that \((\overline{X}, d_{\overline{X}})\) is complete, by Definition \ref{1.4.10} we need to show that for any Cauchy sequence \((a^{(n)})_{n = 1}^\infty\) in \(\overline{X}\), \(\exists\ x \in \overline{X}\) such that \(\lim_{n \to \infty} d_{\overline{X}}(a^{(n)}, x) = 0\).
    Since \(a^{(n)} \in \overline{X}\), by the definition of \(\overline{X}\) we know that \(\forall\ n \geq 1\), there exists a Cauchy sequence \((a_k^{(n)})_{k = 1}^\infty\) in \(X\) such that \(\text{LIM}_{k \to \infty} a_k^{(n)} = a^{(n)}\).
    Similarly since \(x \in \overline{X}\), there exists a Cauchy sequence \((b_k)_{k = 1}^{\infty}\) in \(X\) such that \(\text{LIM}_{n \to \infty} b_k = x\).
    Thus by Exercise \ref{ex 1.4.8}(b) it will suffice to show that
    \[
        \lim_{n \to \infty} d_{\overline{X}}(a^{(n)}, x) = \lim_{n \to \infty} d_{\overline{X}}(\text{LIM}_{k \to \infty} a_k^{(n)}, \text{LIM}_{k \to \infty} b_k) = \lim_{n \to \infty} (\lim_{k \to \infty} d(a_k^{(n)}, b_k)) = 0
    \]
    for some Cauchy sequence \((b_k)_{k = 1}^{\infty}\) in \(X\).

    Let \(j, K, k_1, k_2, L, N, n_1, n_2 \in \mathbf{Z}^+\).
    Since \((a_k^{(n)})_{k = 1}^\infty\) is a Cauchy sequence, by Definition \ref{1.4.6} we know that \(\forall\ \varepsilon \in \mathbf{R}^+\), \(\exists\ K \geq 1\) such that \(\forall\ k_1, k_2 \geq K\),
    \[
        d(a_{k_1}^{(n)}, a_{k_2}^{(n)}) \leq \varepsilon.
    \]
    In particular, \(\exists\ K \geq 1\) such that \(\forall\ k_1, k_2 \geq K\),
    \[
        d(a_{k_1}^{(n)}, a_{k_2}^{(n)}) \leq \frac{1}{n}.
    \]
    Since the existence of \(K\) depends on \(n\), we denote such \(K\) as \(K_n\).
    Note that we can always choose such \(K_n \geq n\)
    (If \(K_n < n\), then we can simply set \(K_n = n\)).
    Since \((a^{(n)})_{n = 1}^\infty\) is a Cauchy sequence, by Definition \ref{1.4.6} \(\exists\ N \geq 1\) such that \(\forall\ n_1, n_2 \geq N\),
    \begin{align*}
                 & d_{\overline{X}}(a^{(n_1)}, a^{(n_2)}) \leq \varepsilon / 2 < \varepsilon                                                                             \\
        \implies & d_{\overline{X}}(\text{LIM}_{k \to \infty} a_k^{(n_1)}, \text{LIM}_{k \to \infty} a_k^{(n_2)}) < \varepsilon & \text{(by Exercise \ref{ex 1.4.8}(a))} \\
        \implies & \lim_{k \to \infty} d(a_k^{(n_1)}, a_k^{(n_2)}) < \varepsilon.                                               & \text{(by Exercise \ref{ex 1.4.8}(b))}
    \end{align*}
    Since \(\lim_{k \to \infty} d(a_k^{(n_1)}, a_k^{(n_2)}) \in \mathbf{R}\), we know that \(\exists\ L \geq 1\) such that \(\forall\ k \geq L\),
    \[
        d(a_k^{(n_1)}, a_k^{(n_2)}) < \varepsilon
    \]
    (otherwise by squeeze test we would have \(\lim_{k \to \infty} d(a_k^{(n_1)}, a_k^{(n_2)}) \geq \varepsilon\)).
    Let \(M = \max(L, N)\).
    Then we can rewrite the statements above as:
    \(\exists\ M \geq 1\) such that \(\forall\ n_1, n_2, k \geq M\), we have
    \[
        d(a_k^{(n_1)}, a_k^{(n_2)}) < \varepsilon.
    \]
    In particular, \(\exists\ M \geq 1\) such that \(\forall\ n_1, n_2, k \geq M\), we have
    \[
        d(a_k^{(n_1)}, a_k^{(n_2)}) < \frac{1}{j}
    \]
    for all \(j \geq 1\).
    Since the existence of \(M\) depends on \(j\), we denote such \(M\) as \(M_j\).
    Note that we can always choose such \(M_j \geq j\)
    (If \(M_j < j\), then we can simply set \(M_j = j\)).

    Let \(P(j, n) = \max(M_j, K_n)\).
    This means \(P(j, n) \geq \max(j, n)\) (since \(K_n \geq n\) and \(M_j \geq j\)) and \(\forall\ k, k_1, k_2, n, n_1, n_2 \geq P(j, n)\), we have
    \begin{align*}
        d(a_{k_1}^{(n)}, a_{k_2}^{(n)}) & < \frac{1}{P(j, n)} \leq \frac{1}{n}     \\
        d(a_k^{(n_1)}, a_k^{(n_2)})     & \leq \frac{1}{P(j, n)} \leq \frac{1}{j}.
    \end{align*}
    By axiom of choice, we can choose \(a_{P(j, j)}^{(P(j, j))}\) for every \(j \geq 1\) and let \((b_j)_{j = 1}^\infty\) be a sequence in \(X\) where \(b_j = a_{P(j, j)}^{(P(j, j))}\).
    Then we have
    \begin{align*}
        d(b_{k_1}, b_{k_2}) & = d(a_{P(k_1, k_1)}^{(P(k_1, k_1))}, a_{P(k_2, k_2)}^{(P(k_2, k_2))})                                                                                                                  \\
                            & \leq d(a_{P(k_1, k_1)}^{(P(k_1, k_1))}, a_{P(k_2, k_2)}^{(P(k_1, k_1))}) + d(a_{P(k_2, k_2)}^{(P(k_1, k_1))}, a_{P(k_2, k_2)}^{(P(k_2, k_2))}) & \text{(by Definition \ref{1.1.2}(d))} \\
                            & \leq \frac{1}{k_1} + \frac{1}{k_2}                                                                                                                                                     \\
                            & \leq \frac{2}{\min(k_1, k_2)}.
    \end{align*}
    Since \(\varepsilon \in \mathbf{R}^+\), we can always find a \(K \in \mathbf{Z}^+\) such that
    \[
        0 < \frac{2}{K} \leq \varepsilon.
    \]
    Combine the results above we have \(\forall\ \varepsilon \in \mathbf{R}^+\), \(\exists\ K \geq 1\) such that \(\forall\ k_1, k_2 \geq K\),
    \[
        d(b_{k_1}, b_{k_2}) \leq \frac{2}{\min(k_1, k_2)} \leq \frac{2}{K} \leq \varepsilon.
    \]
    By Definition \ref{1.4.6} this means \((b_j)_{j = 1}^\infty\) is a Cauchy sequence.
    Thus by Exercise \ref{ex 1.4.8}(b) we have \(\text{LIM}_{j \to \infty} b_j \in \overline{X}\).

    Now we show that \(\lim_{n \to \infty} (\lim_{k \to \infty} d(a_k^{(n)}, b_k)) = 0\).
    Since \(\forall\ k \geq P(n, n)\), we have
    \begin{align*}
        d(a_k^{(n)}, b_k) & = d(a_k^{(n)}, a_{P(k, k)}^{(P(k, k))})                                                                                      \\
                          & \leq d(a_k^{(n)}, a_{P(k, k)}^{(n)}) + d(a_{P(k, k)}^{(n)}, a_{P(k, k)}^{(P(k, k))}) & \text{(by Definition \ref{1.1.2}(d))} \\
                          & \leq \frac{1}{n} + \frac{1}{k},
    \end{align*}
    we then have
    \begin{align*}
                 & 0 \leq d(a_k^{(n)}, b_k) \leq \frac{1}{n} + \frac{1}{k}                                                      \\
        \implies & 0 \leq \lim_{k \to \infty} d(a_k^{(n)}, b_k) \leq \frac{1}{n}             & \text{(by comparison principle)} \\
        \implies & 0 \leq \lim_{n \to \infty} (\lim_{k \to \infty} d(a_k^{(n)}, b_k)) \leq 0 & \text{(by comparison principle)} \\
        \implies & \lim_{n \to \infty} (\lim_{k \to \infty} d(a_k^{(n)}, b_k)) = 0.          & \text{(by squeeze test)}
    \end{align*}
    Thus \((\overline{X}, d_{\overline{X}})\) is complete.
\end{proof}

\begin{proof}{(d)}
    Since
    \begin{align*}
             & x = y                                                                                               \\
        \iff & d(x, y) = 0                                                & \text{(by Definition \ref{1.1.2}(a))}  \\
        \iff & \lim_{n \to \infty} d(x, y) = 0                                                                     \\
        \iff & \text{LIM}_{n \to \infty} x = \text{LIM}_{n \to \infty} y, & \text{(by Exercise \ref{ex 1.4.8}(a))}
    \end{align*}
    we know that \(x = \lim_{n \to \infty} x = \text{LIM}_{n \to \infty} x \in \overline{X}\).
    Thus
    \begin{align*}
        d_{\overline{X}}(x, y) & = d_{\overline{X}}(\text{LIM}_{n \to \infty} x, \text{LIM}_{n \to \infty} y)                                          \\
                               & = \lim_{n \to \infty} d(x, y)                                                & \text{(by Exercise \ref{ex 1.4.8}(b))} \\
                               & = d(x, y).
    \end{align*}
\end{proof}