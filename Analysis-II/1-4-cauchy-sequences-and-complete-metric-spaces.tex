\section{Cauchy sequences and complete metric spaces}\label{sec 1.4}

\begin{definition}[Subsequences]\label{1.4.1}
    Suppose that \((x^{(n)})_{n = m}^\infty\) is a sequence of points in a metric space \((X, d)\).
    Suppose that \(n_1, n_2, n_3, \dots\) is an increasing sequence of integers which are at least as large as \(m\), thus
    \[
        m \leq n_1 < n_2 < n_3 < \dots.
    \]
    Then we call the sequence \((x^{(n_j)})_{j = 1}^\infty\) a subsequence of the original sequence \((x^{(n)})_{n = m}^\infty\).
\end{definition}

\begin{lemma}\label{1.4.3}
    Let \((x^{(n)})_{n = m}^\infty\) be a sequence in \((X, d)\) which converges to some limit \(x_0\).
    Then every subsequence \((x^{(n_j)})_{j = 1}^\infty\) of that sequence also converges to \(x_0\).
\end{lemma}

\begin{proof}
    By Definition \ref{1.1.14} we have \(\lim_{n \to \infty} d(x^{(n)}, x_0) = 0\), which means \(\forall\ \varepsilon \in \mathbf{R}^+\), \(\exists\ N \in \mathbf{N}\) and \(N \geq m\) such that
    \[
        \forall\ n \geq N, d(x^{(n)}, x_0) \leq \varepsilon.
    \]
    Let \((x^{(n_j)})_{j = 1}^\infty\) be any subsequence of \((x^{(n)})_{n = m}^\infty\).
    Since \(\forall\ j \in \mathbf{N}\), we have \(n_j \geq j\), we must have \(n_N \geq N\).
    This means \(\forall\ n_j \geq n_N \geq N\), we have \(d(x^{(n_j)}, x_0) \leq \varepsilon\).
    Thus by Definition \ref{1.1.14} \(\lim_{j \to \infty} d(x^{(n_j)}, x_0) = 0\).
\end{proof}