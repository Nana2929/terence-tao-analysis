\section{Compact metric spaces}\label{sec 1.5}

\begin{definition}[Compactness]\label{1.5.1}
    A metric space \((X, d)\) is said to be \emph{compact} iff every sequence in \((X, d)\) has at least one convergent subsequence.
    A subset \(Y\) of a metric space \(X\) is said to be \emph{compact} if the subspace \((Y, d|_{Y \times Y})\) is compact.
\end{definition}

\begin{remark}\label{1.5.2}
    The notion of a set \(Y\) being compact is \emph{intrinsic}, in the sense that it only depends on the metric function \(d|_{Y \times Y}\) restricted to \(Y\), and not on the choice of the ambient space \(X\).
    The notions of completeness in Definition \ref{1.4.10}, and of boundedness in Definition \ref{1.5.3}, are also intrinsic, but the notions of open and closed are not
    (see the discussion in Section \ref{sec 1.3}).
\end{remark}

\begin{note}
    The notion of a set \(Y\) being compact only depends on the metric function \(d|_{Y \times Y}\) but not ambient space \(X\) since the elements of a sequence in \(Y\) stays the same no matter which spaces \(Y\) is subset to.
    But the notion of a set being open or closed depends on the definition of a metric ball, which may be different when given different ambient spaces.
\end{note}

\begin{note}
    Heine-Borel theorem shows that in the real line \(\mathbf{R}\) with the usual metric, every closed and bounded set is compact, and conversely every compact set is closed and bounded.
\end{note}

\begin{definition}[Bounded sets]\label{1.5.3}
    Let \((X, d)\) be a metric space, and let \(Y\) be a subset of \(X\).
    We say that \(Y\) is \emph{bounded} iff for every \(x \in X\) there exists a ball \(B(x, r)\) in \(X\) which contains \(Y\).
    We call \((X, d)\) bounded if \(X\) is bounded.
\end{definition}

\begin{remark}\label{1.5.4}
    Definition \ref{1.5.3} is compatible with the definition of a bounded set in \((\mathbf{R}, d_{l^1}|_{\mathbf{R} \times \mathbf{R}})\).
\end{remark}

\begin{proposition}\label{1.5.5}
    Let \((X, d)\) be a compact metric space.
    Then \((X, d)\) is both complete and bounded.
\end{proposition}

\begin{proof}
    We first show that \((X, d)\) is complete.
    Let \((a_n)_{n = 1}^\infty\) be a Cauchy sequence in \((X, d)\).
    Since \((X, d)\) is compact, by Definition \ref{1.5.1} we know that there exists a subsequence of \((a_n)_{n = 1}^\infty\) which converges to some \(x_0 \in X\) with respect to \(d\).
    Since \((a_n)_{n = 1}^\infty\) is a Cauchy sequence in \((X, d)\), by Lemma \ref{1.4.9} we know that \((a_n)_{n = 1}^\infty\) must converge to \(x_0\) with respect to \(d\).
    Since \((a_n)_{n = 1}^\infty\) is arbitrary, by Definition \ref{1.4.10} we know that \((X, d)\) is complete.

    Now we show that \((X, d)\) is bounded by contradiction.
    Suppose for sake of contradiction that \((X, d)\) is not bounded.
    Then by Definition \ref{1.5.3} we have
    \begin{align*}
                 & \lnot\big(\forall\ x_0 \in X, \exists\ r \in \mathbf{R}^+ : X \subseteq B_{(X, d)}(x_0, r)\big)            \\
        \implies & \exists\ x_0 \in X : \forall\ r \in \mathbf{R}^+, X \not\subseteq B_{(X, d)}(x_0, r)                       \\
        \implies & \exists\ x_0 \in X : \forall\ r \in \mathbf{R}^+, \big(X \setminus B_{(X, d)}(x_0, r)\big) \neq \emptyset  \\
        \implies & \exists\ x_0 \in X : \forall\ n \in \mathbf{Z}^+, \big(X \setminus B_{(X, d)}(x_0, n)\big) \neq \emptyset.
    \end{align*}
    If \(X = \emptyset\), then \((\emptyset, d)\) is bound since \(x_0 \notin \emptyset\).
    So we only considered the cases \(X \neq \emptyset\).
    Let \((a_n)_{n = 1}^\infty\) be the sequence where \(a_n \in X \setminus B_{(X, d)}(x_0, n)\) for all \(n \in \mathbf{Z}^+\).
    Note that such sequence is well-defined by axiom of choice.
    Since \(a_n \in X \setminus B_{(X, d)}(x_0, n)\), by Definition \ref{1.2.1} we know that \(d(a_n, x_0) \geq n\) for all \(n \in \mathbf{Z}^+\).
    Since \((X, d)\) is compact, by Definition \ref{1.5.1} there exists a subsequence \((a_{n_j})_{j = 1}^\infty\) of \((a_n)_{n = 1}^\infty\) such that \((a_{n_j})_{j = 1}^\infty\) converges in \(X\).
    Let \(\lim_{j \to \infty} d(a_{n_j}, L) = 0\) for some \(L \in X\).
    By Definition \ref{1.1.14} we know that
    \[
        \forall\ \varepsilon \in \mathbf{R}^+, \exists\ J \in \mathbf{Z}^+ : \forall\ j \geq J, d(a_{n_j}, L) \leq \varepsilon.
    \]
    In particular,
    \[
        \exists\ J \in \mathbf{Z}^+ : \forall\ j \geq J, d(a_{n_j}, L) \leq 1.
    \]
    Now we fix such \(J\) and let \(i = \max\big(J + 1, \ceil{d(L, x_0)} + 2\big)\).
    Then we have
    \begin{align*}
        d(a_{n_i}, x_0) & \leq d(a_{n_i}, L) + d(L, x_0) & \text{(by Definition \ref{1.1.2}(d))} \\
                        & \leq 1 + d(L, x_0)             & (i > J)
    \end{align*}
    and
    \begin{align*}
        d(a_{n_i}, x_0) & \geq n_i            & \text{(by the definition of \(a_{n_i}\))} \\
                        & \geq i                                                          \\
                        & \geq d(L, x_0) + 2. & \text{(by the definition of \(i\))}
    \end{align*}
    But this means \(d(L, x_0) + 2 \leq d(L, x_0) + 1\), a contradiction.
    Thus \((X, d)\) is bounded.
\end{proof}

\begin{corollary}[Compact sets are closed and bounded]\label{1.5.6}
    Let \((X, d)\) be a metric space, and let \(Y\) be a compact subset of \(X\).
    Then \(Y\) is closed and bounded.
\end{corollary}

\begin{proof}
    Since \((Y, d|_{Y \times Y})\) is compact, by Proposition \ref{1.5.5} we know that \((Y, d)\) is complete and bounded.
    Thus by Proposition \ref{1.4.12}(a) we know that \(Y\) is closed in \((X, d)\).
\end{proof}

\begin{theorem}[Heine-Borel theorem]\label{1.5.7}
    Let \((\mathbf{R}^n, d)\) be a Euclidean space with either the Euclidean metric, the taxicab metric, or the supnorm metric.
    Let \(E\) be a subset of \(\mathbf{R}^n\).
    Then \(E\) is compact if and only if it is closed and bounded.
\end{theorem}

\begin{proof}
    We first show that for any \(E \subseteq \mathbf{R}^n\), \(E\) is closed and bounded in \((\mathbf{R}^n, d_{l^1}|_{\mathbf{R}^n \times \mathbf{R}^n})\) iff \((E, d_{l^1}|_{E \times E})\) is compact.
    By Exercise \ref{ex 1.1.7} we know that \((\mathbf{R}^n, d_{l^1}|_{\mathbf{R}^n \times \mathbf{R}^n})\) is a metric space, and by Corollary \ref{1.5.6} we know that if \((E, d_{l^1}|_{E \times E})\) is compact, then \(E\) is closed and bounded in \((\mathbf{R}^n, d_{l^1}|_{\mathbf{R}^n \times \mathbf{R}^n})\).
    So we only need to show that if \(E\) is closed and bounded in \((\mathbf{R}^n, d_{l^1}|_{\mathbf{R}^n \times \mathbf{R}^n})\), then \((E, d_{l^1}|_{E \times E})\) is compact.
    Suppose that \(E \subseteq \mathbf{R}^n\), \(E\) is closed and bounded in \((\mathbf{R}^n, d_{l^1}|_{\mathbf{R}^n \times \mathbf{R}^n})\).
    Since \(E \subseteq \mathbf{R}^n\), we know that for every \(x \in E\), \(x\) is in the form \(x = (x_1, \dots, x_n) = (x_i)_{i = 1}^n \in \mathbf{R}^n\).
    Let \(I_n = \{i \in \mathbf{N} : 1 \leq i \leq n\}\).
    For each \(i \in I_n\), let \(E_i\) be the set
    \[
        E_i = \{y \in \mathbf{R} | \exists\ x \in E : x_i = y\},
    \]
    i.e., \(E_i\) is the collection of \(i^{\text{th}}\) coordinate of all element \(x \in E\).
    We claim that for every \(i \in I_n\), \(E_i\) is a subset of some closed interval and thus \(E_i\) is bounded in \((\mathbf{R}, d_{l^1}|_{\mathbf{R} \times \mathbf{R}})\).
    This is true since
    \begin{align*}
                 & E \text{ is bounded in } (\mathbf{R}^n, d_{l^1}|_{\mathbf{R}^n \times \mathbf{R}^n})                                                                                                          \\
        \implies & \forall\ y \in \mathbf{R}^n, \exists\ r \in \mathbf{R}^+ : E \subseteq B_{(\mathbf{R}^n, d_{l^1}|_{\mathbf{R}^n \times \mathbf{R}^n})}(y, r)             & \text{(by Definition \ref{1.5.3})} \\
        \implies & \forall\ y \in \mathbf{R}^n, \exists\ r \in \mathbf{R}^+ : \forall\ x \in E, x \in B_{(\mathbf{R}^n, d_{l^1}|_{\mathbf{R}^n \times \mathbf{R}^n})}(y, r)                                      \\
        \implies & \forall\ y \in \mathbf{R}^n, \exists\ r \in \mathbf{R}^+ : \forall\ x \in E,                                                                                                                  \\
                 & d_{l^1}|_{\mathbf{R}^n \times \mathbf{R}^n}(x, y) = \sum_{i = 1}^n \abs*{x_i - y_i} < r                                                                  & \text{(by Definition \ref{1.2.1})} \\
        \implies & \forall\ y \in \mathbf{R}^n, \exists\ r \in \mathbf{R}^+ : \forall\ x \in E, \forall\ i \in I_n,                                                                                              \\
                 & d_{l^1}|_{\mathbf{R} \times \mathbf{R}}(x_i, y_i) = \abs*{x_i - y_i} < r                                                                                                                      \\
        \implies & \forall\ y \in \mathbf{R}^n, \exists\ r \in \mathbf{R}^+ : \forall\ x \in E, \forall\ i \in I_n,                                                                                              \\
                 & x_i \in (y_i - r, y_i + r)                                                                                                                                                                    \\
        \implies & \forall\ y \in \mathbf{R}^n, \exists\ r \in \mathbf{R}^+ : \forall\ i \in I_n,                                                                                                                \\
                 & E_i \subseteq (y_i - r, y_i + r) \subseteq [y_i - r, y_i + r]                                                                                                                                 \\
        \implies & \forall\ y \in \mathbf{R}^n, \exists\ r \in \mathbf{R}^+ : \forall\ i \in I_n,                                                                                                                \\
                 & E_i \subseteq B_{(\mathbf{R}, d_{l^1}|_{\mathbf{R} \times \mathbf{R}})}(y_i, r)                                                                          & \text{(by Definition \ref{1.2.1})} \\
        \implies & \forall\ i \in I_n, E_i \text{ is bounded in } (\mathbf{R}, d_{l^1}|_{\mathbf{R} \times \mathbf{R}}).                                                    & \text{(by Definition \ref{1.5.3})}
    \end{align*}

    Let \(P(n)\) be the statement ``If \(F \subseteq \mathbf{R}^n\) such that for every \(i \in I_n\), \(F_i\) is bounded in \((\mathbf{R}, d_{l^1}|_{\mathbf{R} \times \mathbf{R}})\) and \(F_i \subseteq C_i\) for some closed interval \(C_i \subseteq \mathbf{R}\), then for any sequence in \(F\) there exists a subsequence which converges in \(\mathbf{R}^n\) with respect to \(d_{l^1}|_{\mathbf{R}^n \times \mathbf{R}^n}\)''.
    We use induction on \(n\) to show that \(P(n)\) is true for all \(n \in \mathbf{Z}^+\).

    For \(n = 1\), by hypothesis we have \(F = F_1\) is bounded in \((\mathbf{R}, d_{l^1}|_{\mathbf{R} \times \mathbf{R}})\) and \(F_1 \subseteq C_1\) for some closed interval \(C_1 \subseteq \mathbf{R}\).
    By Heine-Borel theorem on real line (Theorem 9.1.24 in Analysis I) we know that for every sequence \((a^{(k)})_{k = 1}^\infty\) in \(F\), there exists a subsequence \((a^{(k_j)})_{j = 1}^\infty\) which converges in \(C_1 \subseteq \mathbf{R}\) with respect to \(d_{l^1}|_{\mathbf{R} \times \mathbf{R}}\).
    Thus the base case holds.

    Suppose inductively that \(P(n)\) is true for some \(n \geq 1\).
    Then we need to show that \(P(n + 1)\) is true.
    Let \(F \subseteq \mathbf{R}^{n + 1}\) such that for every \(i \in I_{n + 1}\), \(F_i\) is bounded in \((\mathbf{R}, d_{l^1}|_{\mathbf{R} \times \mathbf{R}})\) and \(F_i \subseteq C_i\) for some closed interval \(C_i \in \mathbf{R}\).
    Let \((a^{(k)})_{k = 1}^\infty\) be arbitrary sequence in \(F\).
    We define \((b^{(k)})_{k = 1}^\infty\) by setting \(b^{(k)} = (a_1^{(k)}, \dots, a_n^{(k)})\) for each \(k \geq 1\), i.e., \(b^{(k)}\) is the first \(n\) coordinates of \(a^{(k)}\).
    Since for all \(k \geq 1\), \(b^{(k)} \in \mathbf{R}^n\) and \(b_i^{(k)} \in F_i\) for all \(i \in I_n\), by induction hypothesis there exists a subsequence \((b^{(k_j)})_{j = 1}^\infty\) which converges in \(\mathbf{R}^n\) with respect to \(d_{l^1}|_{\mathbf{R}^n \times \mathbf{R}^n}\).
    Since \((a_{n + 1}^{(k_j)})_{j = 1}^\infty\) is in \(F_{n + 1}\) and \(F_{n + 1} \subseteq C_{n + 1}\) for some closed interval \(C_{n + 1} \subseteq \mathbf{R}\), by Heine-Borel theorem on real line (Theorem 9.1.24 in Analysis I) we know that there exists a subsequence \((a^{(k_{j_p})})_{p = 1}^\infty\) which converges in \(C_{n + 1}\) with respect to \(d_{l^1}|_{\mathbf{R} \times \mathbf{R}}\).
    But by Lemma \ref{1.4.3} we know that every subsequence of \((b^{(k_j)})_{j = 1}^\infty\) also converges in \(\mathbf{R}^n\) with respect to \(d_{l^1}|_{\mathbf{R}^n \times \mathbf{R}^n}\).
    In particular, \((b^{k_{j_p}})_{p = 1}^\infty\) converges in \(\mathbf{R}^n\) with respect to \(d_{l^1}|_{\mathbf{R}^n \times \mathbf{R}^n}\).
    Thus by Proposition \ref{1.1.18}(b)(d) we know that \((a^{(k_{j_p})})_{p = 1}^\infty\) converges in \(\mathbf{R}^{n + 1}\) with respect to \(d_{l^1}|_{\mathbf{R}^{n + 1} \times \mathbf{R}^{n + 1}}\), and this close the induction.

    From the proof above we know that if \(E \subseteq \mathbf{R}^n\) such that \(E\) is closed and bounded in \((\mathbf{R}^n, d_{l^1}|_{\mathbf{R}^n \times \mathbf{R}^n})\), then for every \(i \in I_n\), \(E_i\) is bounded in \((\mathbf{R}, d_{l^1}|_{\mathbf{R} \times \mathbf{R}})\) and \(E_i \subseteq C_i\) for some closed interval \(C_i \subseteq \mathbf{R}\).
    Thus we know that for every sequence in \(E\) there exists a subsequence which converges in \(\mathbf{R}^n\) with respect to \(d_{l^1}|_{\mathbf{R}^n \times \mathbf{R}^n}\).
    Since \(E\) is closed in \((\mathbf{R}^n, d_{l^1}|_{\mathbf{R}^n \times \mathbf{R}^n})\), by Proposition \ref{1.2.15}(b) we know that such subsequence must converges in \(E\) with respect to \(d_{l^1}|_{\mathbf{R}^n \times \mathbf{R}^n}\).
    Thus by Definition \ref{1.5.1} \((E, d_{l^1}|_{E \times E})\) is compact.

    Since every sequence in \(E\) has a subsequence which converges in \(E\) with respect to \(d_{l^1}|_{E \times E}\), by Proposition \ref{1.1.18} we know that such subsequence also converges with respect to \(d_{l^2}|_{E \times E}\) and \(d_{l^\infty}|_{E \times E}\).
    Thus \((E, d_{l^2}|_{E \times E})\) and \((E, d_{l^\infty}|_{E \times E})\) are compact iff \((E, d_{l^1}|_{\mathbf{R}^n \times \mathbf{R}^n})\) is compact.
\end{proof}

\begin{note}
    The Heine-Borel theorem is not true for more general metric spaces.
    However, a version of the Heine-Borel theorem is available if one is willing to replace closedness with the stronger notion of completeness, and boundedness with the stronger notion of \emph{total boundedness}.
\end{note}

\begin{note}
    One can characterize compactness topologically via Theorem \ref{1.5.8}:
    every open cover of a compact set has a finite subcover.
\end{note}

\begin{theorem}\label{1.5.8}
    Let \((X, d)\) be a metric space, and let \(Y\) be a compact subset of \(X\).
    Let \((V_{\alpha})_{\alpha \in A}\) be a collection of open sets in \(X\), and suppose that
    \[
        Y \subseteq \bigcup_{\alpha \in A} V_{\alpha}.
    \]
    (i.e., the collection \((V_{\alpha})_{\alpha \in A}\) \emph{covers} \(Y\)).
    Then there exists a \emph{finite} subset \(F\) of \(A\) such that
    \[
        Y \subseteq \bigcup_{\alpha \in F} V_{\alpha}.
    \]
\end{theorem}

\begin{proof}
    We assume for sake of contradiction that there does not exist any finite subset \(F\) of \(A\) for which \(Y \subseteq \bigcup_{\alpha \in F} V_{\alpha}\).

    Let \(y\) be any element of \(Y\).
    Then \(y\) must lie in at least one of the sets \(V_{\alpha}\).
    Since each \(V_{\alpha}\) is open in \((X, d)\), by Proposition \ref{1.2.15}(a) there must therefore be an \(r > 0\) such that \(B_{(X, d)}(y, r) \subseteq V_{\alpha}\).
    Now let \(r(y)\) denote the quantity
    \[
        r(y) \coloneqq \sup\big\{r \in (0, \infty) : B_{(X, d)}(y, r) \subseteq V_{\alpha} \text{ for some } \alpha \in A\big\}.
    \]
    By the above discussion, we know that \(r(y) > 0\) for all \(y \in Y\).
    Now, let \(r_0\) denote the quantity
    \[
        r_0 \coloneqq \inf\big\{r(y) : y \in Y\big\}.
    \]
    Since \(r(y) > 0\) for all \(y \in Y\), we have \(r_0 \geq 0\).
    There are three cases: \(r_0 = 0\), \(0 < r_0 < \infty\) and \(r_0 = \infty\).
    \begin{itemize}
        \item Case 1:
              \(r_0 = 0\).
              Then for every integer \(n \geq 1\), there is at least one point \(y\) in \(Y\) such that \(r(y) < 1 / n\) (otherwise the infimum cannot be \(0\)).
              We thus choose, for each \(n \geq 1\), a point \(y^{(n)}\) in \(Y\) such that \(r(y^{(n)}) < 1 / n\)
              (we can do this because of the axiom of choice, see Proposition 8.4.7 in Analysis I).
              In particular we have \(\lim_{n \to \infty} r(y^{(n)}) = 0\), by the squeeze test.
              The sequence \((y^{(n)})_{n = 1}^\infty\) is a sequence in \(Y\);
              since \((Y, d|_{Y \times Y})\) is compact, we can thus find a subsequence \((y^{(n_j)})_{j = 1}^\infty\) which converges to a point \(y_0 \in Y\) with respect to \(d|_{Y \times Y}\).

              As before, we know that there exists some \(\alpha \in A\) such that \(y_0 \in V_{\alpha}\), and hence (since \(V_{\alpha}\) is open in \((X, d)\)) there exists some \(\varepsilon > 0\) such that \(B_{(X, d)}(y_0, \varepsilon) \subseteq V_{\alpha}\).
              Since \((y^{(n_j)})_{j = 1}^\infty\) converges to \(y_0\), there must exist an \(N \geq 1\) such that \(y^{(n_j)} \in B_{(X, d)}(y_0, \varepsilon / 2)\) for all \(n_j \geq N\).
              In particular, by the triangle inequality we have \(B_{(X, d)}(y^{(n_j)}, \varepsilon / 2) \subseteq B_{(X, d)}(y_0, \varepsilon)\), and thus \(B_{(X, d)}(y^{(n_j)}, \varepsilon / 2) \subseteq V_{\alpha}\).
              By definition of \(r(y^{(n_j)})\), this implies that \(r(y^{(n_j)}) \geq \varepsilon / 2\) for all \(n_j \geq N\).
              But this contradicts the fact that \(\lim_{n \to \infty} r(y^{(n)}) = 0\).
        \item Case 2:
              \(0 < r_0 < \infty\).
              In this case we now have \(r(y) > r_0 / 2\) for all \(y \in Y\).
              This implies that for every \(y \in Y\) there exists an \(\alpha \in A\) such that \(B_{(X, d)}(y, r_0 / 2) \subseteq V_{\alpha}\) (by the definition of \(r(y)\)).

              We now construct a sequence \(y^{(1)}, y^{(2)}, \dots\) by the following recursive procedure.
              We let \(y^{(1)}\) be any point in \(Y\).
              The ball \(B_{(X, d)}(y^{(1)}, r_0 / 2)\) is contained in one of the \(V_{\alpha}\) and thus cannot cover all of \(Y\), since we would then obtain a finite cover, a contradiction.
              Thus there exists a point \(y^{(2)}\) which does not lie in \(B_{(X, d)}(y^{(1)}, r_0 / 2)\), so in particular \(d(y^{(2)}, y^{(1)}) \geq r_0 / 2\).
              Choose such a point \(y^{(2)}\).
              The set \(B_{(X, d)}(y^{(1)}, r_0 / 2) \cup B_{(X, d)}(y^{(2)}, r_0 / 2)\) cannot cover all of \(Y\), since we would then obtain two sets \(V_{\alpha_1}\) and \(V_{\alpha_2}\) which covered \(Y\), a contradiction again.
              So we can choose a point \(y^{(3)}\) which does not lie in \(B_{(X, d)}(y^{(1)}, r_0 / 2) \cup B_{(X, d)}(y^{(2)}, r_0 / 2)\), so in particular \(d(y^{(3)}, y^{(1)}) \geq r_0 / 2\) and \(d(y^{(3)}, y^{(2)}) \geq r_0 / 2\).
              Continuing in this fashion we obtain a sequence \((y^{(n)})_{n = 1}^\infty\) in \(Y\) with the property that \(d(y^{(k)}, y^{(j)}) \geq r_0 / 2\) for all \(k > j\).
              In particular the sequence \((y^{(n)})_{n = 1}^\infty\) is not a Cauchy sequence, and in fact no subsequence of \((y^{(n)})_{n = 1}^\infty\) can be a Cauchy sequence either.
              But this contradicts the assumption that \((Y, d|_{Y \times Y})\) is compact (by Lemma \ref{1.4.7}).
        \item Case 3:
              \(r_0 = \infty\).
              For this case we argue as in Case 2, but replacing the role of \(r_0 / 2\) by (say) \(1\).
    \end{itemize}
\end{proof}

\begin{note}
    It turns out that Theorem \ref{1.5.8} has a converse:
    if \(Y\) has the property that every open cover has a finite sub-cover, then it is compact.
    In fact, this property is often considered the more fundamental notion of compactness than the sequence-based one.
    (For metric spaces, the two notions, that of compactness and sequential compactness, are equivalent, but for more general \emph{topological spaces}, the two notions are slightly different.)
\end{note}

\begin{corollary}\label{1.5.9}
    Let \((X, d)\) be a metric space, and let \(K_1, K_2, K_3, \dots\) be a sequence of non-empty compact subsets of \(X\) such that
    \[
        K_1 \supseteq K_2 \supseteq K_3 \supseteq \dots.
    \]
    Then the intersection \(\bigcap_{n = 1}^\infty K_n\) is non-empty.
\end{corollary}

\begin{proof}
    Since \((K_1, d|_{K_1 \times K_1})\) is compact and \(K_1 \subseteq X\), by Corollary \ref{1.5.6} we know that \(K_1\) is closed in \((X, d)\).
    Since \(K_1\) is closed in \((X, d)\) and \(K_1 \cap K_n = K_n\) for every \(n \geq 1\), by Proposition \ref{1.3.4}(b) we know that \(K_n\) is relatively closed in \((K_1, d|_{K_1 \times K_1})\).
    Let \(V_n = K_1 \setminus K_n\) for every \(n \geq 1\).
    Then for every \(n \geq 1\), we have \(V_n \subseteq K_1\) and by Proposition \ref{1.2.15}(e) \(V_n\) is open in \((K_1, d|_{K_1 \times K_1})\).

    Suppose for sake of contradiction that \(\bigcap_{n = 1}^\infty K_n = \emptyset\).
    Since
    \[
        \bigcup_{n = 1}^\infty V_n = \bigcup_{n = 1}^\infty (K_1 \setminus K_n) = K_1 \setminus \bigg(\bigcap_{n = 1}^\infty K_n\bigg) = K_1
    \]
    and \((K_1, d)\) is compact, by Theorem \ref{1.5.8} we know that there exists a finite set \(F \subseteq \mathbf{Z}^+\) such that
    \[
        K_1 \subseteq \bigcup_{i \in F} V_i.
    \]
    Since \(F\) is finite subset of \(\mathbf{Z}^+\), we know that \(\min(F)\) is well-defined.
    Then we have
    \begin{align*}
                 & K_1 \subseteq \bigcup_{i \in F} V_i \subseteq \bigcup_{n = 1}^\infty V_i = K_1                                                          \\
        \implies & K_1 = \bigcup_{i \in F} V_i                                                                                                             \\
        \implies & K_1 = \bigcup_{i \in F} (K_1 \setminus K_i)                                                                                             \\
        \implies & K_1 = K_1 \setminus \bigg(\bigcap_{i \in F} K_i\bigg)                                                                                   \\
        \implies & \bigcap_{i \in F} K_i = \emptyset                                              & \text{(since \(\bigcap_{i \in F} K_i \subseteq K_1)\)} \\
        \implies & K_{\min(F)} = \emptyset.                                                       & \text{(since \(K_{\min(F)} = \bigcap_{i \in F} K_i)\)}
    \end{align*}
    But by hypothesis we know that \(K_{\min(F)} \neq \emptyset\), a contradiction.
    Thus \(\bigcap_{n = 1}^\infty K_n \neq \emptyset\).
\end{proof}

\begin{theorem}\label{1.5.10}
    Let \((X, d)\) be a metric space.
    \begin{enumerate}
        \item If \(Y\) is a compact subset of \(X\), and \(Z \subseteq Y\), then \(Z\) is compact if and only if \(Z\) is closed.
        \item If \(Y_1, \dots, Y_n\) are a finite collection of compact subsets of \(X\), then their union \(Y_1 \cup \dots \cup Y_n\) is also compact.
        \item Every finite subset of \(X\) (including the empty set) is compact.
    \end{enumerate}
\end{theorem}

\begin{proof}{(a)}
    By Corollary \ref{1.5.6} we know that if \((Z, d|_{Z \times Z})\) is compact then \(Z\) is closed in \((Y, d|_{Y \times Y})\).
    Now we show that if \(Z\) is closed in \((Y, d|_{Y \times Y})\) then \((Z, d|_{Z \times Z})\) is compact.
    Since \((Y, d|_{Y \times Y})\) is compact and \(Z \subseteq Y\), by Definition \ref{1.5.1} we know that every sequence \((z^{(n)})_{n = 1}^\infty\) in \(Z\) has a convergent subsequence \((z^{(n_j)})_{j = 1}^\infty\) which converges in \(Y\) with respect to \(d|_{Y \times Y}\).
    Since \(Z\) is closed in \((Y, d|_{Y \times Y})\), by Proposition \ref{1.2.15}(b) we know that \((z^{(n_j)})_{j = 1}^\infty\) converges in \(Z\) with respect to \(d|_{Y \times Y}\).
    Since \((z^{(n)})_{n = 1}^\infty\) is arbitrary, by Definition \ref{1.5.1} we know that \((Z, d|_{Z \times Z})\) is compact.
\end{proof}

\begin{proof}{(b)}
    We use induction on \(n\) to show that \((\bigcup_{i = 1}^n Y_i, d)\) is compact for every \(n \in \mathbf{Z}^+\).
    For \(n = 1\), we know that \(\bigcup_{i = 1}^1 Y_i = Y_1\) and by hypothesis \((Y_1, d)\) is compact, thus the base case holds.
    Suppose inductively that \((\bigcup_{i = 1}^n Y_i, d)\) is compact for some \(n \geq 1\).
    Then for \(n + 1\), we need to show that \((\bigcup_{i = 1}^{n + 1} Y_i, d)\) is compact.
    Let \((y^{(k)})_{k = 1}^\infty\) be a sequence in \(\bigcup_{i = 1}^{n + 1} Y_i\).
    We now split into two cases:
    \begin{itemize}
        \item If there exists a subsequence \((y^{(k_j)})_{j = 1}^\infty\) whose elements are in \(\bigcup_{i = 1}^n Y_i\), then by induction hypothesis we know that \((y^{(k_j)})_{j = 1}^\infty\) converges in \(\bigcup_{i = 1}^n Y_i\) with respect to \(d\).
              This means \((y^{(k_j)})_{j = 1}^\infty\) also converges in \(\bigcup_{i = 1}^{n + 1} Y_i\) with respect to \(d\) since \(\bigcup_{i = 1}^n Y_i \subseteq \bigcup_{i = 1}^{n + 1} Y_i\).
        \item If there does not exist a subsequence \((y^{(k_j)})_{j = 1}^\infty\) whose elements are in \(\bigcup_{i = 1}^n Y_i\), then there is only finitely many elements in \((y^{(k)})_{k = 1}^\infty\) which are in \(\bigcup_{i = 1}^n Y_i\).
              This means there exists a subsequence \((y^{(k_j)})_{j = 1}^\infty\) whose elements are in \(Y_{n + 1}\).
              By hypothesis we know that \((Y_{n + 1}, d)\) is compact, thus by Definition \ref{1.5.1} there exists a subsequence \((y^{(k_{j_p})})_{p = 1}^\infty\) of \((y^{(k_j)})_{j = 1}^\infty\) converges in \(Y_{n + 1}\) with respect to \(d\).
              Since \(Y_{n + 1} \subseteq \bigcup_{i = 1}^{n + 1} Y_i\), we know that \((y^{(k_{j_p})})_{p = 1}^\infty\) also converges in \(\bigcup_{i = 1}^{n + 1} Y_i\) with respect to \(d\).
    \end{itemize}
    From all cases above we conclude that there exists a subsequence of \((y^{(k)})_{k = 1}^\infty\) which converges in \(\bigcup_{i = 1}^{n + 1} Y_i\) with respect to \(d\).
    Since \((y^{(k)})_{k = 1}^\infty\) is arbitrary, by Definition \ref{1.5.1} \((\bigcup_{i = 1}^{n + 1} Y_i, d)\) is compact.
    This close the induction.
\end{proof}

\begin{proof}{(c)}
    Let \(Y \subseteq X\) and \(\#(Y) = n\).
    Let \(P(n)\) be the statement ``\(\#(Y) = n\) and for every sequence \((y^{(k)})_{k = 1}^\infty\) in \(Y\), there exists a subsequence of \((y^{(k)})_{k = 1}^\infty\) which converges in \(Y\) with respect to \(d\)''.
    We use induction on \(n\) to show that \(P(n)\) is true for all \(n \in \mathbf{N}\).
    For \(n = 0\), we have \(Y = \emptyset\) and the statement \(P(0)\) is trivially true.
    Thus by Definition \ref{1.5.1} \((\emptyset, d)\) is compact and the base case holds.
    Suppose inductively that \(P(n)\) is true for some \(n \geq 0\).
    Then we need to show that \(P(n + 1)\) is true.
    Let \(Y \subseteq X\) such that \(\#(Y) = n + 1\) and let \(x_0 \in Y\).
    Let \((y^{(k)})_{k = 1}^\infty\) be arbitrary sequence in \(Y\).
    Now we split into two cases:
    \begin{itemize}
        \item If the set \(\{k \in \mathbf{N} : y^{(k)} = x_0\}\) is finite, then we can have a subsequence \((y^{(k_j)})_{j = 1}^\infty\) whose elements are in \(Y \setminus \{x_0\}\).
              Since \((y^{(k_j)})_{j = 1}^\infty\) is in \(Y \setminus \{x_0\}\) and \(\#(Y \setminus \{x_0\}) = n\), by induction hypothesis we know that there exists a subsequence of \((y^{(k_j)})_{j = 1}^\infty\) which converges in \(Y \setminus \{x_0\}\) with respect to \(d\).
              But \((y^{(k_j)})_{j = 1}^\infty\) is also in \(Y\), thus we know that there exists a subsequence of \((y^{(k_j)})_{j = 1}^\infty\) which converges in \(Y\) with respect to \(d\).
        \item If the set \(\{k \in \mathbf{N} : y^{(k)} = x_0\}\) is infinite, then we can have a subsequence \((y^{(k_j)})_{j = 1}^\infty\) whose elements are all \(x_0\) and obviously \((y^{(k_j)})_{j = 1}^\infty\) converges to \(x_0\) with respect to \(d\).
              Since \(x_0 \in Y\), we know that \((y^{(k_j)})_{j = 1}^\infty\) converges in \(Y\) with respect to \(d\).
    \end{itemize}
    From all cases above we conclude that there exists a subsequence of \((y^{(k)})_{k = 1}^\infty\) which converges in \(Y\) with respect to \(d\).
    Since \((y^{(k)})_{k = 1}^\infty\) is arbitrary, we conclude that \(P(n + 1)\) is true and this close the induction.
    Since \(P(n)\) is true for every \(n \in \mathbf{N}\), by Definition \ref{1.5.1} we know that if \(Y\) is a finite subset of \(X\), then \((Y, d|_{Y \times Y})\) is compact.
\end{proof}

\exercisesection

\begin{exercise}\label{ex 1.5.1}
    Show that Definitions 9.1.22 in Analysis I and Definition \ref{1.5.3} match when talking about subsets of the real line with the standard metric.
\end{exercise}

\begin{proof}
    Let \(X \subseteq \mathbf{R}\) and let \(d = d_{l^1}|_{\mathbf{R} \times \mathbf{R}}\).
    By Exercise \ref{ex 1.1.2} we know that \((\mathbf{R}, d)\) is a metric space.
    Then we have
    \begin{align*}
             & X \text{ is bounded in the sense of Definition 9.1.22 in Analysis I}                                          \\
        \iff & \exists\ M \in \mathbf{R}^+ : X \subseteq [-M, M] \subseteq (-M - 1, M + 1)                                   \\
        \iff & \exists\ M \in \mathbf{R}^+ : \forall\ x \in X, \abs*{x - 0} < M + 1                                          \\
        \iff & \forall\ y \in \mathbf{R}, \exists\ M \in \mathbf{R}^+ : \forall\ x \in X,                                    \\
             & \abs*{x - y} \leq \abs*{x - 0} + \abs*{y - 0} < M + 1 + \abs*{y}                                              \\
        \iff & \forall\ y \in \mathbf{R}, \exists\ M \in \mathbf{R}^+ : X \subseteq B_{(\mathbf{R}, d)}(y, M + 1 + \abs*{y}) \\
        \iff & X \text{ is bounded in the sense of Definition \ref{1.5.3}}.
    \end{align*}
\end{proof}

\begin{exercise}\label{ex 1.5.2}
    Prove Proposition \ref{1.5.5}.
\end{exercise}

\begin{proof}
    See Proposition \ref{1.5.5}.
\end{proof}

\begin{exercise}\label{ex 1.5.3}
    Prove Theorem \ref{1.5.7}.
\end{exercise}

\begin{proof}
    See Theorem \ref{1.5.7}.
\end{proof}

\begin{exercise}\label{ex 1.5.4}
    Let \((\mathbf{R}, d)\) be the real line with the standard metric.
    Give an example of a continuous function \(f : \mathbf{R} \to \mathbf{R}\), and an open set \(V \subseteq \mathbf{R}\), such that the image \(f(V) \coloneqq \{f(x) : x \in V\}\) of \(V\) is \emph{not} open.
\end{exercise}

\begin{proof}
    Let \(f(x) = 1\) for all \(x \in \mathbf{R}\).
    Since \(f\) is a constant function, we know that \(f\) is continuous.
    By Definition \ref{1.2.1} and Proposition \ref{1.2.15}(c) we know that
    \[
        B_{(\mathbf{R}, d_{l^1}|_{\mathbf{R} \times \mathbf{R}})}(1, 1) = (0, 2)
    \]
    is open in \((\mathbf{R}, d_{l^1}|_{\mathbf{R} \times \mathbf{R}})\) and by Proposition \ref{1.2.15}(d)
    \[
        f\big((0, 2)\big) = \{1\}
    \]
    is closed in \((\mathbf{R}, d_{l^1}|_{\mathbf{R} \times \mathbf{R}})\).
    By Proposition \ref{1.2.15}(a) we know that \(\{1\}\) is not open in \((\mathbf{R}, d_{l^1}|_{\mathbf{R} \times \mathbf{R}})\) since we cannot find an \(r \in \mathbf{R}^+\) such that \(B_{(\mathbf{R}, d_{l^1})}(1, r) \subseteq \{1\}\).
    Thus \(f\) satisfys the requirements.
\end{proof}

\begin{exercise}\label{ex 1.5.5}
    Let \((\mathbf{R}, d)\) be the real line with the standard metric.
    Give an example of a continuous function \(f : \mathbf{R} \to \mathbf{R}\), and a closed set \(F \subseteq \mathbf{R}\), such that \(f(F)\) is \emph{not} closed.
\end{exercise}

\begin{proof}
    Let \(f(x) = 2^x\) for all \(x \in \mathbf{R}\).
    We know that \(2^x\) is continuous on \((-\infty, \infty)\).
    Let \(F = (-\infty, 0]\).
    If \(x_0 \in \partial_{(\mathbf{R}, d)}(F)\), then by Definition \ref{1.2.5} we must have \(B(x_0, r) \not\subseteq F\) and \(B(x_0, r) \cap F \neq \emptyset\) for every \(r \in \mathbf{R}^+\).
    Thus we must have \(\partial_{(\mathbf{R}, d)}(F) = \{0\}\).
    Since \(0 \in F\), by Definition \ref{1.2.12} we know that \(F\) is closed in \((\mathbf{R}, d)\).
    Since
    \[
        f(F) = f\big((-\infty, 0]\big) = (0, 1]
    \]
    and \(0 \notin (0, 1]\), by Definition \ref{1.2.12} we know that \(f(F)\) is not closed in \((\mathbf{R}, d)\).
    Thus the function \(f\) and the closed set \(F\) satisfy the requirements.
\end{proof}

\begin{exercise}\label{ex 1.5.6}
    Prove Corollary \ref{1.5.9}.
\end{exercise}

\begin{proof}
    See Corollary \ref{1.5.9}.
\end{proof}

\begin{exercise}\label{ex 1.5.7}
    Prove Theorem \ref{1.5.10}.
\end{exercise}

\begin{proof}
    See Theorem \ref{1.5.10}.
\end{proof}

\begin{exercise}\label{ex 1.5.8}
    Let \((X, d_{l^1})\) be the metric space from Exercise \ref{ex 1.1.15}.
    For each natural number \(n\), let \(e^{(n)} = (e_j^{(n)})_{j = 0}^\infty\) be the sequence in \(X\) such that \(e_j^{(n)} \coloneqq 1\) when \(n = j\) and \(e_j^{(n)} \coloneqq 0\) when \(n \neq j\).
    Show that the set \(\{e^{(n)} : n \in \mathbf{N}\}\) is a closed and bounded subset of \(X\), but is not compact.
    (This is despite the fact that \((X, d_{l^1})\) is even a complete metric space
    - a fact which we will not prove here.
    The problem is that not that \(X\) is incomplete, but rather that it is ``infinite-dimensional'', in a sense that we will not discuss here.)
\end{exercise}

\begin{proof}
    Let \(E = \{e^{(n)} : n \in \mathbf{N}\}\).
    We first show that \(E\) is bounded in \((X, d_{l^1})\).
    Since
    \[
        \sum_{j = 0}^\infty \abs*{e_j^{(n)}} = 1
    \]
    for every \(n \in \mathbf{N}\), we know that \(e^{(n)}\) is absolutely convergent and by Exercise \ref{ex 1.1.15} \(e^{(n)} \in X\).
    By Exercise \ref{ex 1.1.15} we know that
    \[
        \forall\ (a_j)_{j = 0}^\infty, (b_j)_{j = 0}^\infty \in X, d_{l^1}\big((a_j)_{j = 0}^\infty, (b_j)_{j = 0}^\infty\big)
    \]
    is well-defined, thus \(d_{l^1}\big((a_j)_{j = 0}^\infty, (e_j^{(n)})_{j = 0}^\infty\big)\) is well-defined for every \(n \in \mathbf{N}\).
    Since
    \begin{align*}
         & \forall\ (a_j)_{j = 0}^\infty \in X, d_{l^1}\big((a_j)_{j = 0}^\infty, (e_j^{(n)})_{j = 0}^\infty\big)                                                              \\
         & = \sum_{j = 0}^\infty \abs*{a_j - e_j^{(n)}}                                                                                                                        \\
         & = \sum_{j = 0 : j \neq n}^\infty \abs*{a_j} + \abs*{a_n - 1}                                                                                                        \\
         & \leq \sum_{j = 0}^\infty \abs*{a_j} + \abs*{a_n - 1}                                                   & \text{(well-defined since \((a_j)_{j = 0}^\infty \in X\))} \\
         & \leq \sum_{j = 0}^\infty \abs*{a_j} + \sup_{n \in \mathbf{N}}\abs*{a_n - 1}                            & \text{(well-defined since \((a_j)_{j = 0}^\infty \in X\))} \\
         & < \sum_{j = 0}^\infty \abs*{a_j} + \sup_{n \in \mathbf{N}}\abs*{a_n - 1} + 1,
    \end{align*}
    by Definition \ref{1.2.1} we know that the ball
    \[
        B_{(X, d_{l^1})}\bigg((a_j)_{j = 0}^\infty, \sum_{j = 0}^\infty \abs*{a_j} + \sup_{n \in \mathbf{N}}\abs*{a_n - 1} + 1\bigg)
    \]
    contains the set \(E\) for every \((a_j)_{j = 0}^\infty \in X\).
    Thus by Definition \ref{1.5.3} we know that \(E\) is bounded in \((X, d_{l^1})\).

    Next we show that \(E\) is closed in \((X, d_{l^1})\).
    Let \(\overline{E}_{(X, d_{l^1})}\) be the closure of \(E\) and let \(x \in \overline{E}_{(X, d_{l^1})}\).
    By Proposition \ref{1.2.10}(c) we know that there exists a sequence \((a^{(k)})_{k = 0}^\infty\) in \(E\) such that \(\lim_{k \to \infty} d_{l^1}\big((a_j^{(k)})_{j = 0}^\infty, x\big) = 0\).
    By Lemma \ref{1.4.7} we know that \((a^{(k)})_{k = 0}^\infty\) is a Cauchy sequence in \((X, d_{l^1})\).
    Let \(k, k' \in \mathbf{N}\).
    By Definition \ref{1.4.6} we have
    \[
        \forall\ \varepsilon \in \mathbf{R}^+, \exists\ N \in \mathbf{N} : \forall\ k, k' \geq N, d_{l^1}(a^{(k)}, a^{(k')}) \leq \varepsilon.
    \]
    In particular,
    \[
        \exists\ N \in \mathbf{N} : \forall\ k, k' \geq N, d_{l^1}(a^{(k)}, a^{(k')}) \leq \frac{1}{2} < 1.
    \]
    Since \(a^{(k)}, a^{(k')} \in E\), we know that \(a^{(k)} = e^{(n)}\) and \(a^{(k')} = e^{(n')}\) for some \(n, n' \in \mathbf{N}\) and
    \[
        d_{l^1}(a^{(k)}, a^{(k')}) = d_{l^1}(e^{(n)}, e^{(n')}) = \begin{cases}
            0 & \text{if } n = n';    \\
            2 & \text{if } n \neq n'.
        \end{cases}
    \]
    This means
    \begin{align*}
                 & \exists\ N \in \mathbf{N} : \forall\ k, k' \geq N, d_{l^1}(a^{(k)}, a^{(k')}) \leq \frac{1}{2} < 1                                        \\
        \implies & \exists\ N \in \mathbf{N} : \forall\ k, k' \geq N, d_{l^1}(a^{(k)}, a^{(k')}) = 0                                                         \\
        \implies & \exists\ N \in \mathbf{N} : \forall\ k \geq N, d_{l^1}(a^{(k)}, a^{(N)}) = 0                                                              \\
        \implies & \exists\ N \in \mathbf{N} : \lim_{k \to \infty} d_{l^1}(a^{(k)}, a^{(N)}) = 0                      & \text{(by Definition \ref{1.1.14})}  \\
        \implies & \exists\ N \in \mathbf{N} : a^{(N)} = x                                                            & \text{(by Proposition \ref{1.1.20})} \\
        \implies & x \in E.
    \end{align*}
    Since \(x\) is arbitrary adherent point of \(E\) in \((X, d_{l^1})\), we have
    \begin{align*}
                 & \overline{E}_{(X, d_{l^1})} \subseteq E                                           \\
        \implies & \overline{E}_{(X, d_{l^1})} = E         & \text{(by Proposition \ref{1.2.10}(c))} \\
        \implies & E \text{ is closed in } (X, d_{l^1}).   & \text{(by Proposition \ref{1.2.15}(b))}
    \end{align*}

    Finally we show that \((E, d_{l^1}|_{E \times E})\) is not compact.
    Let \((e^{(n)})_{n = 0}^\infty\) be a sequence and let \(n, n' \in \mathbf{N}\).
    Since \((e^{(n)})_{n = 0}^\infty\) is a sequence in \(E\), we know that
    \[
        \forall\ N \in \mathbf{N}, \forall\ n, n' \geq N, d_{l^1}(e^{(n)}, e^{(n')}) = 2.
    \]
    Thus by Definition \ref{1.4.6} \((e^{(n)})_{n = 0}^\infty\) is not a Cauchy sequence in \((E, d_{l^1}|_{E \times E})\).
    Similarly any subsequence of \((e^{(n)})_{n = 0}^\infty\) is not a Cauchy sequence in \((E, d_{l^1}|_{E \times E})\).
    By Lemma \ref{1.4.7} this means no subsequence is convergent in \((E, d_{l^1}|_{E \times E})\), and by Definition \ref{1.5.1} \((E, d_{l^1}|_{E \times E})\) is not compact.
\end{proof}

\begin{exercise}\label{ex 1.5.9}
    Show that a metric space \((X, d)\) is compact if and only if every sequence in \(X\) has at least one limit point.
\end{exercise}

\begin{proof}
    \begin{align*}
             & (X, d) \text{ is compact}                                                                               \\
        \iff & \text{every sequence in } X \text{ has a convergent subsequence}                                        \\
             & \text{which converges in } X                                      & \text{(by Definition \ref{1.5.1})}  \\
        \iff & \text{every sequence in } X \text{ has at least one limit point}. & \text{(by Proposition \ref{1.4.5})}
    \end{align*}
\end{proof}

\begin{exercise}\label{ex 1.5.10}
    A metric space \((X, d)\) is called \emph{totally bounded} if for every \(\varepsilon > 0\), there exists a natural number \(n\) and a finite number of balls \(B(x^{(1)}, \varepsilon), \dots, B(x^{(n)}, \varepsilon)\) which cover \(X\) (i.e., \(X = \bigcup_{i = 1}^n B(x^{(i)}, \varepsilon)\)).
    (Note that \(x^{(1)}, \dots, x^{(n)} \in X\))
    \begin{enumerate}
        \item Show that every totally bounded space is bounded.
        \item Show the following stronger version of Proposition \ref{1.5.5}:
              if \((X, d)\) is compact, then complete and totally bounded.
        \item Conversely, show that if \(X\) is complete and totally bounded, then \(X\) is compact.
    \end{enumerate}
\end{exercise}

\begin{proof}{(a)}
    Suppose that \(X\) is totally bounded in \((X, d)\).
    Let \(i, j, n \in \mathbf{N}\).
    Then by definition we have
    \[
        \forall\ \varepsilon \in \mathbf{R}^+, \exists\ n \in \mathbf{N} : X = \bigcup_{i = 1}^n B_{(X, d)}(x^{(i)}, \varepsilon).
    \]
    In particular, we have
    \[
        \exists\ n \in \mathbf{N} : X = \bigcup_{i = 1}^n B_{(X, d)}(x^{(i)}, 1).
    \]
    If \(n = 0\), then we have \(X = \emptyset\) and by Definition \ref{1.5.3} \(\emptyset\) is bounded in \((\emptyset, d)\).
    So suppose that \(n > 0\).
    Now we use axiom of choice to choose one finite collection of \(x^{(1)}, \dots, x^{(n)} \in X\).
    Let \(I_n = \{i \in \mathbf{N} : 1 \leq i \leq n\}\) and let \(r = \max\{d(x^{(i)}, x^{(1)}) : i \in I_n\}\).
    We know that \(r\) is well-defined since \(I_n\) is finite.
    Then have
    \begin{align*}
                 & \forall\ i \in I_n, \forall\ y \in B_{(X, d)}(x^{(i)}, 1), d(y, x^{(i)}) < 1   & \text{(by Definition \ref{1.2.1})}             \\
        \implies & \forall\ i \in I_n, \forall\ y \in B_{(X, d)}(x^{(i)}, 1),                                                                      \\
                 & d(y, x^{(1)}) \leq d(y, x^{(i)}) + d(x^{(i)}, x^{(1)}) < 1 + r                 & \text{(by Definition \ref{1.1.2}(d))}          \\
        \implies & \forall\ y \in \bigcup_{i = 1}^n B_{(X, d)}(x^{(i)}, 1), d(y, x^{(1)}) < 1 + r                                                  \\
        \implies & \forall\ y \in X, d(y, x^{(1)}) < 1 + r                                        & (X = \bigcup_{i = 1}^n B_{(X, d)}(x^{(i)}, 1)) \\
        \implies & \forall\ y \in X, y \in B_{(X, d)}(x^{(1)}, 1 + r)                             & \text{(by Definition \ref{1.2.1})}             \\
        \implies & X \subseteq B_{(X, d)}(x^{(1)}, 1 + r)                                                                                          \\
        \implies & \forall\ y \in X, X \subseteq B_{(X, d)}\big(y, d(y, x^{(1)}) + 1 + r\big)     & \text{(by Definition \ref{1.1.2}(d))}          \\
        \implies & X \text{ is bounded in } (X, d).                                               & \text{(by Definition \ref{1.5.3})}
    \end{align*}
\end{proof}

\begin{proof}{(b)}
    By Proposition \ref{1.5.5} we know that if \((X, d)\) is compact then \((X, d)\) is complete.
    Thus we only need to show that \(X\) is totally bound in \((X, d)\).
    Let \(\varepsilon \in \mathbf{R}^+\).
    Then we have
    \begin{align*}
                 & \forall\ y \in X, y \in B_{(X, d)}(y, \varepsilon)                                                                         & \text{(by Definition \ref{1.2.1})}         \\
        \implies & X \subseteq \bigcup_{y \in X} B_{(X, d)}(y, \varepsilon)                                                                                                                \\
        \implies & \bigcup_{y \in X} B_{(X, d)}(y, \varepsilon) \text{ is an open cover of } X \text{ in } (X, d)                             & \text{(by Proposition \ref{1.2.15}(c)(g))} \\
        \implies & \exists\ F \subseteq X : (F \text{ is finite}) \land \bigg(X \subseteq \bigcup_{y \in F} B_{(X, d)}(y, \varepsilon)\bigg). & \text{(by Theorem \ref{1.5.8})}
    \end{align*}
    Since \(\varepsilon\) is arbitrary, by definition we know that \(X\) is totally bounded in \((X, d)\).
\end{proof}

\begin{proof}{(c)}
    Suppose that \((X, d)\) is complete and \(X\) is totally bounded in \((X, d)\).
    If \(X = \emptyset\), then we know that \((\emptyset, d)\) is compact is trivially true.
    So suppose that \(X \neq \emptyset\).
    To show that \((X, d)\) is compact, by Definition \ref{1.5.1} we need to show that every sequence in \(X\) has a convergent subsequence which converges in \(X\) with respect to \(d\).
    So let \((x^{(n)})_{n = 1}^\infty\) be a sequence in \(X\).
    If we have
    \[
        \exists\ N \in \mathbf{Z}^+ : \big\{n \in \mathbf{Z}^+ : x^{(n)} = x^{(N)}\big\} \text{ is infinite},
    \]
    then there must exist a subsequence of \((x^{(n)})_{n = 1}^\infty\) which converges to \(x^{(N)}\) with respect to \(d\) for some \(N \in \mathbf{Z}^+\).
    So suppose that
    \[
        \forall\ N \in \mathbf{Z}^+, \big\{n \in \mathbf{Z}^+ : x^{(n)} = x^{(N)}\big\} \text{ is finite}.
    \]
    Let \(E = \{x^{(n)} : n \in \mathbf{Z}^+\}\).
    For each \(\varepsilon \in \mathbf{R}^+\), we define \(F_\varepsilon\) to be the set
    \[
        F_\varepsilon = \Bigg\{F \subseteq X : (F \text{ is finite}) \land \bigg(X \subseteq \bigcup_{y \in F} B_{(X, d)}(y, \varepsilon)\bigg)\Bigg\}.
    \]
    Since \(X\) is totally bounded in \((X, d)\), by definition we know that \(F_\varepsilon \neq \emptyset\) for every \(\varepsilon \in \mathbf{R}^+\).
    For arbitrary \(\varepsilon \in \mathbf{R}^+\) and arbitrary \(F \in F_\varepsilon\), we claim that
    \[
        \exists\ y \in F : E \cap B_{(X, d)}(y, \varepsilon) \text{ is infinite}.
    \]
    Suppose for sake of contradiction that the claim is false.
    Then we have
    \begin{align*}
                 & \forall\ y \in F, E \cap B_{(X, d)}(y, \varepsilon) \text{ is finite}                                                   \\
        \implies & \bigcup_{y \in F} \bigg(E \cap B_{(X, d)}(y, \varepsilon)\bigg) \text{ is finite}      & \text{(since \(F\) is finite)} \\
        \implies & E \cap \bigg(\bigcup_{y \in F} B_{(X, d)}(y, \varepsilon)\bigg) \text{ is finite}                                       \\
        \implies & E = E \cap X \subseteq E \cap \bigg(\bigcup_{y \in F} B_{(X, d)}(y, \varepsilon)\bigg)
    \end{align*}
    and thus \(E\) is finite.
    But by the definition of \(E\) we know that \(E\) is infinite, a contradiction.
    Thus the claim is true.

    Using the claim above we can now define the set \(A_\varepsilon\) for each \(\varepsilon \in \mathbf{R}^+\).
    \[
        A_\varepsilon = \big\{y \in F : (F \in F_\varepsilon) \land (E \cap B_{(X, d)}(y, \varepsilon) \text{ is infinite})\big\}
    \]
    From the claim above we know that \(A_\varepsilon \neq \emptyset\) for every \(\varepsilon \in \mathbf{R}^+\).
    Now we claim that
    \[
        \forall\ \delta, \varepsilon \in \mathbf{R}^+, \delta < \varepsilon \implies A_\delta \subseteq A_\varepsilon.
    \]
    Let \(\delta, \varepsilon \in \mathbf{R}^+\) and \(\delta < \varepsilon\).
    Then we have
    \begin{align*}
                 & \forall\ y \in A_\delta, E \cap B_{(X, d)}(y, \delta) \text{ is infinite}                                                              \\
        \implies & \forall\ y \in A_\delta, B_{(X, d)}(y, \delta) \text{ is infinite}                                & \text{(since \(E\) is infinite)}   \\
        \implies & \forall\ y \in A_\delta, B_{(X, d)}(y, \delta) \subseteq B_{(X, d)}(y, \varepsilon)               & \text{(by Definition \ref{1.2.1})} \\
        \implies & \forall\ y \in A_\delta, E \cap B_{(X, d)}(y, \delta) \subseteq E \cap B_{(X, d)}(y, \varepsilon)                                      \\
        \implies & \forall\ y \in A_\delta, E \cap B_{(X, d)}(y, \varepsilon) \text{ is infinite}                                                         \\
        \implies & \forall\ y \in A_\delta, y \in A_\varepsilon                                                                                           \\
        \implies & A_\delta \subseteq A_\varepsilon
    \end{align*}
    and thus the claim is true.

    Now we construct a subsequence of \(x^{(n)}\).
    For each \(j \in \mathbf{Z}^+\) we define \(N_j\) as follow:
    \[
        N_j = \bigg\{n \in \mathbf{Z}^+ : x^{(n)} \in E \cap B_{(X, d)}(y, \frac{1}{j}) \text{ for some } y \in A_{\frac{1}{j}}\bigg\}.
    \]
    From the claim above we know that \(A_{\frac{1}{j}}\) is infinite for every \(j \in \mathbf{Z}^+\), thus \(N_j\) is infinite and we have
    \[
        \forall\ i, j \in \mathbf{Z}^+, i < j \implies \frac{1}{j} < \frac{1}{i} \implies A_{\frac{1}{j}} \subseteq A_{\frac{1}{i}} \implies N_j \subseteq N_i.
    \]
    Now we recursively define \(n_j\) for each \(j \in \mathbf{Z}^+\) as follow:
    \[
        n_j = \begin{cases}
            \min N_1                                & \text{if } j = 1 \\
            \min\{n \in N_{j - 1} : n > n_{j - 1}\} & \text{if } j > 1
        \end{cases}
    \]
    Since \(N_j\) is infinite for each \(j \in \mathbf{Z}^+\), we know that the set \(\{n \in N_{j - 1} : n > n_{j - 1}\}\) is also infinite for every \(j \geq 2\).
    If not, then the maximum element of \(N_{j - 1}\) would be \(n_{j - 1}\) and \(N_{j - 1}\) is finite, a contradiction.
    Since \(\{n \in N_j : n > n_{j - 1}\} \subseteq \mathbf{Z}^+\) for each \(j \in \mathbf{Z}^+\) and \(j \geq 2\), by well-ordering principle we know that \(\min\{n \in N_j : n > n_{j - 1}\}\) is well-defined.
    Thus \(n_j\) is well-defined for each \(j \in \mathbf{Z}^+\) and \((x^{(n_j)})_{j = 1}^\infty\) is a subsequence of \((x^{(n)})_{n = 1}^\infty\).

    Now we claim that the subsequence \((x^{(n_j)})_{j = 1}^\infty\) converges in \(X\) with respect to \(d\).
    We have
    \begin{align*}
                 & \forall\ \varepsilon \in \mathbf{R}^+, \exists\ J \in \mathbf{Z}^+ : \frac{1}{J} < \varepsilon                                    & \text{(by Archimedean property)}                  \\
        \implies & \forall\ \varepsilon \in \mathbf{R}^+, \exists\ J \in \mathbf{Z}^+ :                                                                                                                  \\
                 & (\frac{1}{J} < \varepsilon) \land (\forall\ j \geq J, N_j \subseteq N_J)                                                          & \text{(from the claim above)}                     \\
        \implies & \forall\ \varepsilon \in \mathbf{R}^+, \exists\ J \in \mathbf{Z}^+ :                                                                                                                  \\
                 & (\frac{1}{J} < \varepsilon)                                                                                                                                                           \\
                 & \land \big(\exists\ y \in A_{\frac{1}{j}} : \forall\ j \geq J, d(x^{(n_j)}, y) < \frac{1}{j} \leq \frac{1}{J} < \varepsilon\big). & \text{(by the definition of \(A_{\frac{1}{j}}\))}
    \end{align*}
    If we fix \(J\) for each \(\varepsilon\), then we have
    \begin{align*}
                 & \forall\ i, j \geq 2J, \exists\ y \in A_{\frac{1}{2J}} :                                                                               \\
                 & d(x^{(n_i)}, y) + d(x^{(n_j)}, y) < \frac{1}{2J} + \frac{1}{2J} = \frac{1}{J} < \varepsilon                                            \\
        \implies & \forall\ i, j \geq 2J, \exists\ y \in A_{\frac{1}{2J}} :                                                                               \\
                 & d(x^{(n_i)}, x^{(n_j)}) \leq d(x^{(n_i)}, y) + d(x^{(n_j)}, y) < \varepsilon                & \text{(by Definition \ref{1.1.2}(c)(d))} \\
        \implies & \forall\ i, j \geq 2J, d(x^{(n_i)}, x^{(n_j)}) < \varepsilon.
    \end{align*}
    Thus we conclude that
    \[
        \forall\ \varepsilon \in \mathbf{R}^+, \exists\ J \in \mathbf{Z}^+ : \forall\ i, j \geq J, d(x^{(n_i)}, d^{(n_j)}) < \varepsilon
    \]
    and by Definition \ref{1.4.6} \((x^{(n_j)})_{j = 1}^\infty\) is a Cauchy sequence in \((X, d)\).
    Since \((X, d)\) is compatible, by Definition \ref{1.4.10} we know that \((x^{(n_j)})_{j = 1}^\infty\) converges in \(X\) with respect to \(d\).
    Since \((x^{(n)})_{n = 1}^\infty\) is arbitrary, we know that any sequence in \(X\) has a subsequence converges in \(X\) with respect to \(d\), and by Definition \ref{1.5.1} \((X, d)\) is compact.
\end{proof}

\begin{exercise}\label{ex 1.5.11}
    Let \((X, d)\) have the property that every open cover of \(X\) has a finite subcover.
    Show that \(X\) is compact.
\end{exercise}

\begin{proof}
    Suppose that \((X, d)\) is a metric space such that every open cover of \(X\) has a finite subcover.
    We want to show that \((X, d)\) is compact.
    Suppose for sake of contradiction that \((X, d)\) is not compact.
    Then by Definition \ref{1.5.1} we know that there exists a sequence \((x^{(n)})_{n = 1}^\infty\) in \(X\) which has no convergent subsequence which converges in \(X\) with respect to \(d\).
    We know that
    \[
        \forall\ N \in \mathbf{Z}^+, \big\{n \in \mathbf{Z}^+ : x^{(n)} = x^{(N)}\big\} \text{ is finite}.
    \]
    If not, then we would have a subsequence which converges to \(x^{(N)}\) with respect to \(d\) for some \(N \in \mathbf{Z}^+\), a contradiction.
    Thus we know that \(\big\{x^{(n)} : n \in \mathbf{Z}^+\big\}\) is infinite.
    By Exercise \ref{ex 1.5.9} we know that \((x^{(n)})_{n = 1}^\infty\) cannot have a limit point in \(X\) with respect to \(d\).
    By Definition \ref{1.4.4} this means
    \begin{align*}
                 & \lnot\big(\exists\ y \in X : \forall\ \varepsilon \in \mathbf{R}^+, \forall\ N \in \mathbf{Z}^+, \exists\ n \geq N : d(x^{(n)}, y) \leq \varepsilon\big) \\
        \implies & \forall\ y \in X, \exists\ \varepsilon \in \mathbf{R}^+ : \exists\ N \in \mathbf{Z}^+ : \forall\ n \geq N, d(x^{(n)}, y) > \varepsilon                   \\
        \implies & \forall\ y \in X, \exists\ \varepsilon \in \mathbf{R}^+ : \big\{x^{(n)} : n \in \mathbf{Z}^+\big\} \cap B_{(X, d)}(y, \varepsilon) \text{ is finite}.
    \end{align*}
    Now we fix such \(\varepsilon\).
    Since
    \begin{align*}
                 & \forall\ y \in X, B_{(X, d)}(y, \varepsilon) \text{ is open in } (X, d) & \text{(by Proposition \ref{1.2.15}(c))} \\
        \implies & X = \bigcup_{y \in X} B_{(X, d)}(y, \varepsilon)                        & \text{(by Definition \ref{1.2.1})}
    \end{align*}
    and \(\bigcup_{y \in X} B_{(X, d)}(y, \varepsilon)\) is an open cover of \(X\) in \((X, d)\), we know that
    \[
        \exists\ F \subseteq X : (F \text{ is finite}) \land \bigg(X = \bigcup_{y \in F} B_{(X, d)}(y, \varepsilon)\bigg).
    \]
    Now we fix such \(F\).
    Then we have
    \begin{align*}
        \{x^{(n)} : n \in \mathbf{Z}^+\} & = \{x^{(n)} : n \in \mathbf{Z}^+\} \cap X                                                        \\
                                         & = \{x^{(n)} : n \in \mathbf{Z}^+\} \cap \bigg(\bigcup_{y \in F} B_{(X, d)}(y, \varepsilon)\bigg) \\
                                         & = \bigcup_{y \in F} \big(\{x^{(n)} : n \in \mathbf{Z}^+\} \cap B_{(X, d)}(y, \varepsilon)\big).
    \end{align*}
    But we know that \(\bigcup_{y \in F} \big(\{x^{(n)} : n \in \mathbf{Z}^+\} \cap B_{(X, d)}(y, \varepsilon)\big)\) is finite since \(F\) is finite and \(\{x^{(n)} : n \in \mathbf{Z}^+\} \cap B_{(X, d)}(y, \varepsilon)\) is finite for every \(y \in F\).
    This means \(\{x^{(n)} : n \in \mathbf{Z}^+\}\) is finite, a contradiction.
    Thus \((X, d)\) is compact.
\end{proof}

\begin{exercise}\label{ex 1.5.12}
    Let \((X, d_{\text{disc}})\) be a metric space with the discrete metric \(d_{\text{disc}}\).
    \begin{enumerate}
        \item Show that \(X\) is always complete.
        \item When is \(X\) compact, and when is \(X\) not compact?
              Prove your claim.
    \end{enumerate}
\end{exercise}

\begin{proof}{(a)}
    Let \((x^{(n)})_{n = 1}^\infty\) be a Cauchy sequence in \((X, d_{\text{disc}})\).
    Let \(i, j \in \mathbf{Z}^+\).
    By Definition \ref{1.4.6} we know that
    \[
        \forall\ \varepsilon \in \mathbf{R}^+, \exists\ N \in \mathbf{Z}^+ : \forall\ i, j \geq N, d_{\text{disc}}(x^{(i)}, x^{(j)}) \leq \varepsilon.
    \]
    In particular, we have
    \[
        \exists\ N \in \mathbf{Z}^+ : \forall\ i, j \geq N, d_{\text{disc}}(x^{(i)}, x^{(j)}) \leq \frac{1}{2}.
    \]
    But by Example \ref{1.1.11} we know that
    \[
        d_{\text{disc}}(x^{(i)}, x^{(j)}) \leq \frac{1}{2} \iff x^{(i)} = x^{(j)}.
    \]
    Thus we have
    \[
        \forall\ \varepsilon \in \mathbf{R}^+, \exists\ N \in \mathbf{Z}^+ : \forall\ i \geq N, x^{(i)} = x^{(N)}.
    \]
    and by Definition \ref{1.1.14} we have \(\lim_{n \to \infty} d_{\text{disc}}(x^{(n)}, x^{(N)}) = 0\).
    This means \((x^{(n)})_{n = 1}^\infty\) converges to some \(x^{(N)} \in X\) with respect to \(d_{\text{disc}}\).
    Since \((x^{(n)})_{n = 1}^\infty\) is arbitrary, by Definition \ref{1.4.10} we know that \((X, d_{\text{disc}})\) is complete.
\end{proof}

\begin{proof}{(b)}
    We claim that \((X, d_{\text{disc}})\) is compact iff \(X\) is finite.
    By Theorem \ref{1.5.10}(c) we know that if \(X\) is finite then \((X, d_{\text{disc}})\) is compact.
    So we only need to show that if \((X, d_{\text{disc}})\) is compact then \(X\) is finite.
    Suppose that \((X, d_{\text{disc}})\) is compact.
    By Theorem \ref{1.5.8} we know that every open cover of \(X\) in \((X, d_{\text{disc}})\) has a finite subcover.
    In particular, we know that
    \[
        \exists\ F \subseteq X : (F \text{ is finite}) \land \bigg(X = \bigcup_{y \in F} B_{(X, d_{\text{disc}})}(y, \frac{1}{2})\bigg).
    \]
    Now we fix such \(F\).
    We know that
    \begin{align*}
                 & \forall\ y \in F, B_{(X, d)}(y, \frac{1}{2}) = \{z \in X : d(z, y) < \frac{1}{2}\} & \text{(by Definition \ref{1.2.1})} \\
        \implies & \forall\ y \in F, B_{(X, d)}(y, \frac{1}{2}) = \{y\}                               & \text{(by Example \ref{1.1.11})}   \\
        \implies & X = \bigcup_{y \in F} B_{(X, d)}(y, \frac{1}{2}) = F
    \end{align*}
    Thus \(X\) is finite.
\end{proof}

\begin{exercise}\label{ex 1.5.13}
    Let \(E\) and \(F\) be two compact subsets of \(\mathbf{R}\) (with the standard metric \(d(x, y) = \abs*{x - y}\)).
    Show that the Cartesian product \(E \times F \coloneqq \{(x, y) : x \in E, y \in F\}\) is a compact subset of \(\mathbf{R}^2\) (with the Euclidean metric \(d_{l^2}\)).
\end{exercise}

\begin{proof}
    Since \(E \times F \subseteq \mathbf{R}^2\), we know that every element \(x \in E \times F\) is in the form \(x = (x_1, x_2)\) where \(x_1 \in E\) and \(x_2 \in F\).
    Let \((x^{(n)})_{n = 1}^\infty\) be a sequence in \(E \times F\).
    Then we know that \((x_1^{(n)})_{n = 1}^\infty\) is a sequence in \(E\) and \((x_2^{(n)})_{n = 1}^\infty\) is a sequence in \(F\).
    Since \((E, d_{l^1}|_{\mathbf{R} \times \mathbf{R}})\) is compact, by Definition \ref{1.5.1} we know that there exists a subsequence \((x_1^{(n_j)})_{j = 1}^\infty\) which converges to some \(L_E \in E\) with respect to \(d_{l^1}|_{\mathbf{R} \times \mathbf{R}}\).
    Since \((x_2^{(n_j)})_{j = 1}^\infty\) is a subsequence of \((x_2^{(n)})_{n = 1}^\infty\) and \((F, d_{l^1}|_{\mathbf{R} \times \mathbf{R}})\) is compact, by Definition \ref{1.5.1} we know that there exists a subsequence \((x_2^{(n_{j_p})})_{p = 1}^\infty\) which converges to some \(L_F \in F\) with respect to \(d_{l^1}|_{\mathbf{R} \times \mathbf{R}}\).
    Since \(\lim_{j \to \infty} d_{l^1}|_{\mathbf{R} \times \mathbf{R}}(x_1^{(n_j)}, L_E) = 0\), by Lemma \ref{1.4.9} we know that \(\lim_{p \to \infty} d_{l^1}|_{\mathbf{R} \times \mathbf{R}}(x_1^{(n_{j_p})}, L_E) = 0\).
    Thus by Proposition \ref{1.1.18}(a)(b)(d) we have
    \[
        \lim_{p \to \infty} d_{l^1}|_{\mathbf{R}^2 \times \mathbf{R}^2}\big(x^{(n_{j_p})}, (L_E, L_F)\big) = \lim_{p \to \infty} d_{l^2}|_{\mathbf{R}^2 \times \mathbf{R}^2}\big(x^{(n_{j_p})}, (L_E, L_F)\big) = 0.
    \]
    Since \((x^{(n)})_{n = 1}^\infty\) is arbitrary, by Definition \ref{1.5.1} we know that \((E \times F, d_{l^2}|_{\mathbf{R}^2 \times \mathbf{R}^2})\) is compact.
\end{proof}

\begin{exercise}\label{ex 1.5.14}
    Let \((X, d)\) be a metric space, let \(E\) be a non-empty compact subset of \(X\), and let \(x_0\) be a point in \(X\).
    Show that there exists a point \(x \in E\) such that
    \[
        d(x_0, x) = \inf\{d(x_0, y) : y \in E\},
    \]
    i.e., \(x\) is the closest point in \(E\) to \(x_0\).
\end{exercise}

\begin{exercise}\label{ex 1.5.15}
    Let \((X, d)\) be a compact metric space.
    Suppose that \((K_{\alpha})_{\alpha \in I}\) is a collection of closed sets in \(X\) with the property that any finite subcollection of these sets necessarily has non-empty intersection, thus \(\bigcap_{\alpha \in F} K_{\alpha} \neq \emptyset\) for all finite \(F \subseteq I\).
    (This property is known as the \emph{finite intersection property}.)
    Show that the \emph{entire} collection has non-empty intersection, thus \(\bigcap_{\alpha \in I} K_{\alpha} \neq \emptyset\).
    Show by counterexample that this statement fails if \(X\) is not compact.
\end{exercise}