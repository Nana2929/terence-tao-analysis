\section{Compact metric spaces}\label{sec 1.5}

\begin{definition}[Compactness]\label{1.5.1}
    A metric space \((X, d)\) is said to be \emph{compact} iff every sequence in \((X, d)\) has at least one convergent subsequence.
    A subset \(Y\) of a metric space \(X\) is said to be \emph{compact} if the subspace \((Y, d|_{Y \times Y})\) is compact.
\end{definition}

\begin{remark}\label{1.5.2}
    The notion of a set \(Y\) being compact is \emph{intrinsic}, in the sense that it only depends on the metric function \(d|_{Y \times Y}\) restricted to \(Y\), and not on the choice of the ambient space \(X\).
    The notions of completeness in Definition \ref{1.4.10}, and of boundedness below in Definition 1.5.3, are also intrinsic, but the notions of open and closed are not
    (see the discussion in Section \ref{sec 1.3}).
\end{remark}

\begin{note}
    Heine-Borel theorem shows that in the real line \(\mathbf{R}\) with the usual metric, every closed and bounded set is compact, and conversely every compact set is closed and bounded.
\end{note}

\begin{definition}[Bounded sets]\label{1.5.3}
    Let \((X, d)\) be a metric space, and let \(Y\) be a subset of \(X\).
    We say that \(Y\) is \emph{bounded} iff for every \(x \in X\) there exists a ball \(B(x, r)\) in \(X\) which contains \(Y\).
    We call \((X, d)\) bounded if \(X\) is bounded.
\end{definition}

\begin{remark}\label{1.5.4}
    Definition \ref{1.5.3} is compatible with the definition of a bounded set on \(\mathbf{R}\).
\end{remark}

\begin{proposition}\label{1.5.5}
    Let \((X, d)\) be a compact metric space.
    Then \((X, d)\) is both complete and bounded.
\end{proposition}

\begin{proof}
    If \(X = \emptyset\), then every Cauchy sequence \((a_n)_{n = 1}^\infty\) in \((\emptyset, d)\) converges in \((\emptyset, d)\) trivially, thus by Definition \ref{1.4.10} \((\emptyset, d)\) is complete.
    Also, \(\forall\ x \in \emptyset\), \(\forall\ r \in \mathbf{R}^+\), we have \(\emptyset \subseteq B_{(\emptyset, d)}(x, r)\) (again the statement is trivially true), thus by Definition \ref{1.5.3} \((\emptyset, d)\) is bounded.

    Now suppose that \(X \neq \emptyset\).
    We show that \((X, d)\) is complete by contradiction.
    So suppose for sake of contradiction that \((X, d)\) is not complete.
    Then by Definition \ref{1.4.10} we know that there exists a Cauchy sequence \((a_n)_{n = 1}^\infty\) in \((X, d)\) does not converge to \(X\).
    We know that \((a_n)_{n = 1}^\infty\) must converge by Exercise \ref{ex 1.4.8}.
    Since \((X, d)\) is compact, by Definition \ref{1.5.1} we know that there exists a subsequence of \((a_n)_{n = 1}^\infty\) which converges to some \(x_0 \in X\).
    But since \((a_n)_{n = 1}^\infty\) is a Cauchy sequence, by Lemma \ref{1.4.9} we know that \((a_n)_{n = 1}^\infty\) must also converge to \(x_0 \in X\), a contradiction.
    Thus \((X, d)\) is complete.

    Finally, we show that \((X, d)\) is bounded by contradiction.
    Suppose for sake of contradiction that \((X, d)\) is not bounded.
    Then by Definition \ref{1.5.3} \(\exists\ x_0 \in X\) such that \(\forall\ r \in \mathbf{R}^+\), we have \(X \not\subseteq B_{(X, d)}(x_0, r)\).
    In particular, we know that \(X \not\subseteq B_{(X, d)}(x_0, r)\) for all \(r \in \mathbf{Z}^+\).
    Let \((a_n)_{n = 1}^\infty\) be the sequence where \(a_n \in X \setminus B_{(X, d)}(x_0, r)\).
    Note that such sequence is well-defined by axiom of choice.
    Since \(a_n \in X \setminus B_{(X, d)}(x_0, r)\), by Definition \ref{1.2.1} we know that \(d(a_n, x_0) \geq r\).
    Since \((X, d)\) is compact, by Definition \ref{1.5.1} there exists a subsequence \((a_{n_j})_{j = 1}^\infty\) of \((a_n)_{n = 1}^\infty\) such that \((a_{n_j})_{j = 1}^\infty\) converges in \(X\).
    Let \(\lim_{j \to \infty} a_{n_j} = L\).
    By Definition \ref{1.1.14} we know that \(\forall\ \varepsilon \in \mathbf{R}^+\), \(\exists\ J \in \mathbf{Z}^+\) such that \(\forall\ j \geq J\), we have \(d(a_{n_j}, L) \leq \varepsilon\).
    In particular, \(\exists\ J \in \mathbf{Z}^+\) such that \(\forall\ j \geq J\), we have \(d(a_{n_j}, L) \leq 1\).
    We fix such \(J\) and let \(i = \max(J + 1, \ceil{d(L, x_0)} + 2)\).
    Then we have
    \begin{align*}
        d(a_{n_i}, x_0) & \leq d(a_{n_i}, L) + d(L, x_0) \\
                        & \leq 1 + d(L, x_0)             \\
                        & < 2 + d(L, x_0)
    \end{align*}
    and
    \begin{align*}
        d(a_{n_i}, x_0) & \geq n_i            \\
                        & \geq i              \\
                        & \geq d(L, x_0) + 2.
    \end{align*}
    But this means \(d(L, x_0) + 2 < d(L, x_0) + 2\), a contradiction.
    Thus \((X, d)\) is bounded.
\end{proof}

\begin{corollary}[Compact sets are closed and bounded]\label{1.5.6}
    Let \((X, d)\) be a metric space, and let \(Y\) be a compact subset of \(X\).
    Then \(Y\) is closed and bounded.
\end{corollary}

\begin{proof}
    Since \((Y, d|_{Y \times Y})\) is compact, by Proposition \ref{1.5.5} we know that \((Y, d)\) is complete and bounded.
    Thus by Proposition \ref{1.4.12}(a) we know that \(Y\) is closed in \(X\).
\end{proof}

\begin{theorem}[Heine-Borel theorem]\label{1.5.7}
    Let \((\mathbf{R}^n, d)\) be a Euclidean space with either the Euclidean metric, the taxicab metric, or the supnorm metric.
    Let \(E\) be a subset of \(\mathbf{R}^n\).
    Then \(E\) is compact if and only if it is closed and bounded.
\end{theorem}

\begin{proof}
    By Exercise \ref{ex 1.1.6}, \ref{ex 1.1.7}, \ref{ex 1.1.9} we know that \((\mathbf{R}^n, d_{l^2})\), \((\mathbf{R}^n, d_{l^1})\), \((\mathbf{R}^n, d_{l^\infty})\) are metric spaces.
    By Corollary \ref{1.5.6} we know that if \((E, d_{l^1})\) is compact then \(E\) is a closed and bounded.
    Thus we only need to show that if \((E, d_{l^1})\) is closed and bounded then \(E\) is compact.

    Let \(E \subseteq \mathbf{R}^n\) and let \((E, d_{l^1})\) be closed and bounded.
    Since \(E \subseteq \mathbf{R}^n\), we know that \(\forall\ x \in E\), we have \(x\) in the form \(x = (x_1, \dots, x_n) = (x_i)_{i = 1}^n \in \mathbf{R}^n\).
    Let \(E_i\) be the set
    \[
        E_i = \{y \in \mathbf{R} | \exists\ x \in E : x_i = y\},
    \]
    i.e., \(E_i\) is the collection of \(i^{\text{th}}\) coordinate of all element \(x \in E\).
    Now we show that \(E_i\) is bounded and \(E_i\) is a subset of some closed interval \(C_i\).

    Since \((E, d_{l^1})\) is bounded, by Definition \ref{1.5.3} there exists a ball \(B_{(\mathbf{R}^n, d_{l^1})}(y, r)\) such that \(E \subseteq B_{(\mathbf{R}^n, d_{l^1})}(y, r)\).
    Then we have
    \begin{align*}
                 & E \subseteq B_{(\mathbf{R}^n, d_{l^1})}(y, r)                                                      \\
        \implies & \forall\ x \in E, x \in B_{(\mathbf{R}^n, d_{l^1})}(y, r)                                          \\
        \implies & d_{l^1}(x, y) = \sum_{i = 1}^n \abs*{x_i - y_i} < r           & \text{(by Definition \ref{1.2.1})} \\
        \implies & \abs*{x_i - y_i} < r \text{ for every } i \in \{1, \dots, n\}                                      \\
        \implies & x_i \in (y_i - r, y_i + r)                                                                         \\
        \implies & E_i \subseteq (y_i - r, y_i + r)                                                                   \\
        \implies & E_i \subseteq B_{(\mathbf{R}, d_{l^1})}(y_i, r).              & \text{(by Definition \ref{1.2.1})}
    \end{align*}
    Thus by Definition \ref{1.5.3} \((E_i, d_{l^1})\) is bounded.
    Let \(C_i = [y_i - r, y_i + r]\).
    Then we know that \(C_i\) is closed and \(E_i \subseteq (y_i - r, y_i + r) \subseteq C_i\).

    Let \(P(n)\) be the statement ``If \(F \subseteq \mathbf{R}^n\) satisfying \(F_i\) is bounded and \(F_i\) is a subset of some closed set \(C_i \subseteq \mathbf{R}\) for every \(i \in \{1, \dots, n\}\), then for any sequence in \(F\) there exists a subsequence which converges in \(\mathbf{R}^n\) with respect to \(d_{l^1}\)''.
    We use induction on \(n\) to show that \(P(n)\) is true \(\forall\ n \in \mathbf{Z}^+\).

    For \(n = 1\), since \(F = F_1\) is bounded and \(F\) is a subset of some closed set \(C_1 \subseteq \mathbf{R}\), by Heine-Borel theorem (Theorem 9.1.24 in Analysis I) on real line we know that for every sequence \((a^{(k)})_{k = 1}^\infty\) in \(F\), there exists a subsequence \((a^{(k_j)})_{j = 1}^\infty\) which converges in \(C_1 \subseteq \mathbf{R}\) with respect to \(d_{l^1}\).
    Thus the base case holds.

    Suppose inductively that \(P(n)\) is true for some \(n \geq 1\).
    Then we need to show that \(P(n + 1)\) is true.
    So suppose that \(F \subseteq \mathbf{R}^{n + 1}\) such that \(F_i\) is bounded and \(F_i\) is a subset of some closed set \(C_i \in \mathbf{R}\) for every \(i \in \{1, \dots, n, n + 1\}\).
    Let \((a^{(k)})_{k = 1}^\infty\) be any sequence in \(F\) and let \((b^{(k)})_{k = 1}^\infty\) be the sequence \(b^{(k)} = (a_1^{(k)}, \dots, a_n^{(k)})\), i.e., \(b^{(k)}\) is the first \(n\) coordinates of \(a^{(k)}\).
    Since \(b^{(k)} \in \mathbf{R}^n\) for all \(k \geq 1\) and \(b_i^{(k)} \in F_i\) for all \(i \in \{1, \dots, n\}\), by induction hypothesis there exists a subsequence \((b^{(k_j)})_{j = 1}^\infty\) which converges in \(\mathbf{R}^n\) with respect to \(d_{l^1}\).
    This means the first \(n\) coordinates of the sequence \((a^{(k_j)})_{j = 1}^\infty\) converges with respect to \(d_{l^1}\).
    Let \(L \in \mathbf{R}^n\) such that \(\lim_{j \to \infty} d(b^{(k_j)}, L) = 0\).
    By Lemma \ref{1.4.3} we know that every subsequence of \((b^{(k_j)})_{j = 1}^\infty\) also converges to \(L\).
    Since \((a_{n + 1}^{(k_j)})_{j = 1}^\infty\) is in \(F_{n + 1}\), by Heine-Borel theorem (Theorem 9.1.24 in Analysis I) on real line we know that there exists a subsequence \((a^{(k_{j_p})})_{p = 1}^\infty\) which converges in \(C_{n + 1} \subseteq \mathbf{R}\).
    Since \((b^{k_{j_p}})_{p = 1}^\infty\) converges to \(L\) with respect to \(d_{l^1}\), by Proposition \ref{1.1.18} we know that \((a^{(k_{j_p})})_{p = 1}^\infty\) converges in \(\mathbf{R}^{n + 1}\) with respect to \(d_{l^1}\).
    This close the induction.

    From the proof above we know that \(E \subseteq \mathbf{R}^n\) is closed and bounded implies \(E_i\) is bounded and \(E_i\) is a subset of some closed set \(C_i \subseteq \mathbf{R}\) for every \(i \in \{1, \dots, n\}\).
    Thus we know that for every sequence in \(E\) there exists a subsequence which converges in \(\mathbf{R}^n\).
    Since \(E\) is closed, by Proposition \ref{1.2.15}(b) we know that such subsequence must converges in \(E\).
    Thus by Definition \ref{1.5.1} \((E, d_{l^1})\) is compact.
    Since every sequence in \(E\) has a convergent subsequence with respect to \(d_{l^1}\), by Proposition \ref{1.1.18} we know that such subsequence also converges with respect to \(d_{l^2}\) and \(d_{l^\infty}\).
    Thus \((E, d_{l^2})\) and \((E, d_{l^\infty})\) are also compact when \((E, d_{l^1})\) is closed and bounded.
\end{proof}