\section{Compact metric spaces}\label{sec 1.5}

\begin{definition}[Compactness]\label{1.5.1}
    A metric space \((X, d)\) is said to be \emph{compact} iff every sequence in \((X, d)\) has at least one convergent subsequence.
    A subset \(Y\) of a metric space \(X\) is said to be \emph{compact} if the subspace \((Y, d|_{Y \times Y})\) is compact.
\end{definition}

\begin{remark}\label{1.5.2}
    The notion of a set \(Y\) being compact is \emph{intrinsic}, in the sense that it only depends on the metric function \(d|_{Y \times Y}\) restricted to \(Y\), and not on the choice of the ambient space \(X\).
    The notions of completeness in Definition \ref{1.4.10}, and of boundedness below in Definition 1.5.3, are also intrinsic, but the notions of open and closed are not
    (see the discussion in Section \ref{sec 1.3}).
\end{remark}

\begin{note}
    Heine-Borel theorem shows that in the real line \(\mathbf{R}\) with the usual metric, every closed and bounded set is compact, and conversely every compact set is closed and bounded.
\end{note}

\begin{definition}[Bounded sets]\label{1.5.3}
    Let \((X, d)\) be a metric space, and let \(Y\) be a subset of \(X\).
    We say that \(Y\) is \emph{bounded} iff for every \(x \in X\) there exists a ball \(B(x, r)\) in \(X\) which contains \(Y\).
    We call \((X, d)\) bounded if \(X\) is bounded.
\end{definition}

\begin{remark}\label{1.5.4}
    Definition \ref{1.5.3} is compatible with the definition of a bounded set on \(\mathbf{R}\).
\end{remark}

\begin{proposition}\label{1.5.5}
    Let \((X, d)\) be a compact metric space.
    Then \((X, d)\) is both complete and bounded.
\end{proposition}

\begin{proof}
    If \(X = \emptyset\), then every Cauchy sequence \((a_n)_{n = 1}^\infty\) in \((\emptyset, d)\) converges in \((\emptyset, d)\) trivially, thus by Definition \ref{1.4.10} \((\emptyset, d)\) is complete.
    Also, \(\forall\ x \in \emptyset\), \(\forall\ r \in \mathbf{R}^+\), we have \(\emptyset \subseteq B_{(\emptyset, d)}(x, r)\) (again the statement is trivially true), thus by Definition \ref{1.5.3} \((\emptyset, d)\) is bounded.

    Now suppose that \(X \neq \emptyset\).
    We show that \((X, d)\) is complete by contradiction.
    So suppose for sake of contradiction that \((X, d)\) is not complete.
    Then by Definition \ref{1.4.10} we know that there exists a Cauchy sequence \((a_n)_{n = 1}^\infty\) in \((X, d)\) does not converge to \(X\).
    We know that \((a_n)_{n = 1}^\infty\) must converge by Exercise \ref{ex 1.4.8}.
    Since \((X, d)\) is compact, by Definition \ref{1.5.1} we know that there exists a subsequence of \((a_n)_{n = 1}^\infty\) which converges to some \(x_0 \in X\).
    But since \((a_n)_{n = 1}^\infty\) is a Cauchy sequence, by Lemma \ref{1.4.9} we know that \((a_n)_{n = 1}^\infty\) must also converge to \(x_0 \in X\), a contradiction.
    Thus \((X, d)\) is complete.

    Finally, we show that \((X, d)\) is bounded by contradiction.
    Suppose for sake of contradiction that \((X, d)\) is not bounded.
    Then by Definition \ref{1.5.3} \(\exists\ x_0 \in X\) such that \(\forall\ r \in \mathbf{R}^+\), we have \(X \not\subseteq B_{(X, d)}(x_0, r)\).
    In particular, we know that \(X \not\subseteq B_{(X, d)}(x_0, r)\) for all \(r \in \mathbf{Z}^+\).
    Let \((a_n)_{n = 1}^\infty\) be the sequence where \(a_n \in X \setminus B_{(X, d)}(x_0, r)\).
    Note that such sequence is well-defined by axiom of choice.
    Since \(a_n \in X \setminus B_{(X, d)}(x_0, r)\), by Definition \ref{1.2.1} we know that \(d(a_n, x_0) \geq r\).
    Since \((X, d)\) is compact, by Definition \ref{1.5.1} there exists a subsequence \((a_{n_j})_{j = 1}^\infty\) of \((a_n)_{n = 1}^\infty\) such that \((a_{n_j})_{j = 1}^\infty\) converges in \(X\).
    Let \(\lim_{j \to \infty} a_{n_j} = L\).
    By Definition \ref{1.1.14} we know that \(\forall\ \varepsilon \in \mathbf{R}^+\), \(\exists\ J \in \mathbf{Z}^+\) such that \(\forall\ j \geq J\), we have \(d(a_{n_j}, L) \leq \varepsilon\).
    In particular, \(\exists\ J \in \mathbf{Z}^+\) such that \(\forall\ j \geq J\), we have \(d(a_{n_j}, L) \leq 1\).
    We fix such \(J\) and let \(i = \max(J + 1, \ceil{d(L, x_0)} + 2)\).
    Then we have
    \begin{align*}
        d(a_{n_i}, x_0) & \leq d(a_{n_i}, L) + d(L, x_0) \\
                        & \leq 1 + d(L, x_0)             \\
                        & < 2 + d(L, x_0)
    \end{align*}
    and
    \begin{align*}
        d(a_{n_i}, x_0) & \geq n_i            \\
                        & \geq i              \\
                        & \geq d(L, x_0) + 2.
    \end{align*}
    But this means \(d(L, x_0) + 2 < d(L, x_0) + 2\), a contradiction.
    Thus \((X, d)\) is bounded.
\end{proof}