\section{Continuous functions}\label{sec 2.1}

\begin{definition}[Continuous functions]\label{2.1.1}
    Let \((X, d_X)\) be a metric space, and let \((Y, d_Y)\) be another metric space, and let \(f : X \to Y\) be a function.
    If \(x_0 \in X\), we say that \(f\) is \emph{continuous at \(x_0\)} iff for every \(\varepsilon > 0\), there exists a \(\delta > 0\) such that \(d_Y(f(x), f(x_0 )) < \varepsilon\) whenever \(d_X(x, x_0) < \delta\).
    We say that \(f\) is \emph{continuous} iff it is continuous at every point \(x \in X\).
\end{definition}

\begin{remark}\label{2.1.2}
    Continuous functions are also sometimes called \emph{continuous maps}.
    Mathematically, there is no distinction between the two terminologies.
\end{remark}

\begin{remark}\label{2.1.3}
    If \(f : X \to Y\) is continuous, and \(K\) is any subset of \(X\), then the restriction \(f|_K : K \to Y\) of \(f\) to \(K\) is also continuous.
\end{remark}

\begin{proof}
    Let \(x_0 \in K\).
    Suppose that \(f : X \to Y\) is continuous at \(x_0\) in \((X, d_X)\).
    Then we have
    \begin{align*}
                 & f : X \to Y \text{ is continuous at } x_0 \text{ in } (X, d_X)                                                                         \\
        \implies & \forall\ \varepsilon \in \mathbf{R}^+, \exists\ \delta \in \mathbf{R}^+ :                                                              \\
                 & \Big(\forall\ x \in X, d_X(x, x_0) < \delta \implies d_Y\big(f(x), f(x_0) < \varepsilon\big)\Big) & \text{(by definition \ref{2.1.1})} \\
        \implies & \forall\ \varepsilon \in \mathbf{R}^+, \exists\ \delta \in \mathbf{R}^+ :                                                              \\
                 & \Big(\forall\ x \in K, d_X(x, x_0) < \delta \implies d_Y\big(f(x), f(x_0) < \varepsilon\big)\Big) & (K \subseteq X)                    \\
        \implies & f|_K : K \to Y \text{ is continuous at } x_0 \text{ in } (K, d_X|_{K \times K}).                  & \text{(by definition \ref{2.1.1})}
    \end{align*}

    Now suppose that \(f : X \to Y\) is continuous on \(X\) in \((X, d_X)\).
    Then we have
    \begin{align*}
                 & f : X \to Y \text{ is continuous in } (X, d_X)                                                                              \\
        \implies & \forall\ x_0 \in X, f \text{ is continuous at } x_0 \text{ in } (X, d_X)               & \text{(by Definition \ref{2.1.1})} \\
        \implies & \forall\ x_0 \in K, f \text{ is continuous at } x_0 \text{ in } (K, d_X|_{K \times K}) & (K \subseteq X)                    \\
        \implies & f|_K : K \to Y \text{ is continuous in } (K, d_X|_{K \times K}).                       & \text{(by Definition \ref{2.1.1})}
    \end{align*}
\end{proof}

\begin{theorem}[Continuity preserves convergence]\label{2.1.4}
    Suppose that \((X, d_X)\) and \((Y, d_Y)\) are metric spaces.
    Let \(f : X \to Y\) be a function, and let \(x_0 \in X\) be a point in \(X\).
    Then the following three statements are logically equivalent:
    \begin{enumerate}
        \item \(f\) is continuous at \(x_0\).
        \item Whenever \((x^{(n)})_{n = 1}^\infty\) is a sequence in \(X\) which converges to \(x_0\) with respect to the metric \(d_X\), the sequence \(\big(f(x^{(n)})\big)_{n = 1}^\infty\) converges to \(f(x_0)\) with respect to the metric \(d_Y\).
        \item For every open set \(V \subseteq Y\) that contains \(f(x_0)\), there exists an open set \(U \subseteq X\) containing \(x_0\) such that \(f(U) \subseteq V\).
    \end{enumerate}
\end{theorem}

\begin{proof}
    We first show that statement (a) implies statement (b).
    Suppose that \(f : X \to Y\) is continuous at \(x_0\) in \((X, d_X)\).
    Then by Definition \ref{2.1.1} we have
    \[
        \forall\ \varepsilon \in \mathbf{R}^+, \exists\ d \in \mathbf{R}^+ : \Big(\forall\ x \in X, d_X(x, x_0) < \delta \implies d_Y\big(f(x), f(x_0)\big) < \varepsilon\Big).
    \]
    Now we choose \(\delta\) for each \(\varepsilon \in \mathbf{R}^+\) and denoted it as \(\delta_\varepsilon\).
    Let \((x^{(n)})_{n = 1}^\infty\) be a sequence in \(X\) such that \(\lim_{n \to \infty} d_X(x^{(n)}, x_0) = 0\).
    Then we have
    \begin{align*}
                 & \lim_{n \to \infty} d_X(x^{(n)}, x_0) = 0                                                                                                                                   \\
        \implies & \forall\ \delta \in \mathbf{R}^+, \exists\ N \in \mathbf{Z}^+ : \forall\ n \geq N, d_X(x^{(n)}, x_0) \leq \delta                      & \text{(by Definition \ref{1.1.14})} \\
        \implies & \forall\ \varepsilon \in \mathbf{R}^+, \exists\ \delta_\varepsilon \in \mathbf{R}^+ :                                                                                       \\
                 & \bigg(\exists\ N \in \mathbf{Z}^+ : \forall\ n \geq N, d_X(x^{(n)}, x_0) \leq \frac{\delta_\varepsilon}{2} < \delta_\varepsilon\bigg)                                       \\
        \implies & \forall\ \varepsilon \in \mathbf{R}^+, \exists\ \delta_\varepsilon \in \mathbf{R}^+ :                                                                                       \\
                 & \bigg(\exists\ N \in \mathbf{Z}^+ : \forall\ n \geq N, d_Y\big(f(x^{(n)}), f(x_0)\big) < \varepsilon\bigg)                            & \text{(by hypothesis)}              \\
        \implies & \lim_{n \to \infty} d_Y\big(f(x^{(n)}), f(x_0)\big) = 0.                                                                              & \text{(by Definition \ref{1.1.14})}
    \end{align*}
    Since \((x^{(n)})_{n = 1}^\infty\) is arbitrary, we know that statement (a) implies statement (b).

    Next we show that statement (b) implies statement (c).
    Suppose that
    \[
        \forall\ (x^{(n)})_{n = 1}^\infty \text{ in } X, \lim_{n \to \infty} d_X(x^{(n)}, x_0) = 0 \implies \lim_{n \to \infty} d_Y\big(f(x^{(n)}), f(x_0)\big) = 0.
    \]
    Let \(V\) be an open set in \((Y, d_Y)\) such that \(f(x_0) \in V\).
    Then we have
    \begin{align*}
                 & V \text{ is open in } (Y, d_Y)                                                                                                           \\
        \implies & V = \text{int}_{(Y, d_Y)}(V)                                                                   & \text{(by Proposition \ref{1.2.15}(a))} \\
        \implies & \exists\ \varepsilon \in \mathbf{R}^+ : B_{(Y, d_Y)}\big(f(x_0), \varepsilon\big) \subseteq V. & \text{(by Definition \ref{1.2.5})}
    \end{align*}
    Now we choose one \(\varepsilon\) and define \(V_\varepsilon = B_{(Y, d_Y)}\big(f(x_0), \varepsilon\big)\).
    By Remark \ref{1.2.4} we know that \(f(x_0) \in V_\varepsilon\), thus we have \(x_0 \in f^{-1}(V_\varepsilon)\) and \(f^{-1}(V_\varepsilon) \neq \emptyset\).
    Now we claim that
    \[
        \exists\ \delta \in \mathbf{R}^+ : B_{(X, d_X)}(x_0, \delta) \subseteq f^{-1}(V_\varepsilon).
    \]
    Suppose the claim is false.
    Then we have
    \begin{align*}
                 & \forall\ \delta \in \mathbf{R}^+, B_{(X, d_X)}(x_0, \delta) \not\subseteq f^{-1}(V_\varepsilon)                                                 \\
        \implies & \forall\ \delta \in \mathbf{R}^+, B_{(X, d_X)}(x_0, \delta) \setminus f^{-1}(V_\varepsilon) \neq \emptyset                                      \\
        \implies & \forall\ \delta \in \mathbf{R}^+, \exists\ x \in X :                                                                                            \\
                 & (d_X(x, x_0) < \delta) \land \Big(d_Y\big(f(x), f(x_0)\big) \geq \varepsilon\Big)                          & \text{(by Definition \ref{1.2.1})} \\
        \implies & \forall\ n \in \mathbf{Z}^+, \exists\ x \in X :                                                                                                 \\
                 & (d_X(x, x_0) < \frac{1}{n}) \land \Big(d_Y\big(f(x), f(x_0)\big) \geq \varepsilon\Big).
    \end{align*}
    For each \(n \in \mathbf{Z}^+\), we define \(X_n = B_{(X, d_X)}(x_0, \frac{1}{n}) \setminus f^{-1}(V_\varepsilon)\).
    We choose one sequence \((x^{(n)})_{n = 1}^\infty \in \prod_{n \in \mathbf{Z}^+} X_n\).
    Then we have
    \begin{align*}
                 & \forall\ n \in \mathbf{Z}^+, d_X(x^{(n)}, x_0) < \frac{1}{n}                                                                                       \\
        \implies & \lim_{n \to \infty} d_X(x^{(n)}, x_0) = 0                                                                                                          \\
        \implies & \lim_{n \to \infty} d_Y\big(f(x^{(n)}), f(x_0)\big) = 0                                                                                            \\
        \implies & \exists\ N \in \mathbf{Z}^+ : \forall\ n \geq N, d_Y\big(f(x^{(n)}), f(x_0)\big) \leq \frac{\varepsilon}{2}. & \text{(by Definition \ref{1.1.14})}
    \end{align*}
    But by the definition of \((x^{(n)})_{n = 1}^\infty\) we know that
    \[
        \forall\ n \in \mathbf{Z}^+, d_Y\big(f(x^{(n)}), f(x_0)\big) \geq \varepsilon,
    \]
    a contradiction.
    Thus the claim is true.
    Using the claim we choose one \(\delta\) and define \(U = B_{(X, d_X)}(x_0, \delta)\).
    By Proposition \ref{1.2.15}(c) we know that \(U\) is open in \((X, d_X)\).
    By Remark \ref{1.2.4} we know that \(x_0 \in U\).
    Since \(U \subseteq f^{-1}(V_\varepsilon) \subseteq X\), we know that \(f(U) \subseteq V\).

    Finally we show that statement (c) implies statement (a).
    Suppose that
    \begin{align*}
                 & \forall\ V \subseteq Y, \big(V \text{ is open in } (Y, d_Y)\big) \land \big(f(x_0) \in V\big)                         \\
        \implies & \exists\ U \subseteq X : \big(U \text{ is open in } (X, d_X)\big) \land (x_0 \in U) \land \big(f(U) \subseteq V\big).
    \end{align*}
    Let \(\varepsilon \in \mathbf{R}^+\).
    By Proposition \ref{1.2.15}(c) we know that \(B_{(Y, d_Y)}\big(f(x_0), \varepsilon\big)\) is open in \((Y, d_Y)\).
    By hypothesis we know that
    \[
        \exists\ U \subseteq X : \big(U \text{ is open in } (X, d_X)\big) \land (x_0 \in U) \land \Big(f(U) \subseteq B_{(Y, d_Y)}\big(f(x_0), \varepsilon\big)\Big).
    \]
    Now we choose one such \(U\).
    Since \(U\) is open in \((X, d_X)\) and \(x_0 \in U\), we have
    \begin{align*}
                 & x_0 \in \text{int}_{(X, d_X)}(U)                                                                                            & \text{(by Proposition \ref{1.2.15}(a))} \\
        \implies & \exists\ \delta \in \mathbf{R}^+ : B_{(X, d_X)}(x_0, \delta) \subseteq U                                                    & \text{(by Definition \ref{1.2.5})}      \\
        \implies & \exists\ \delta \in \mathbf{R}^+ : f\big(B_{(X, d_X)}(x_0, \delta)\big) \subseteq f(U)                                                                                \\
        \implies & \exists\ \delta \in \mathbf{R}^+ : f\big(B_{(X, d_X)}(x_0, \delta)\big) \subseteq B_{(Y, d_Y)}\big(f(x_0), \varepsilon\big)                                           \\
        \implies & \exists\ \delta \in \mathbf{R}^+ :                                                                                                                                    \\
                 & \Big(\forall\ x \in X, d_X(x, x_0) < \delta \implies d_Y\big(f(x), f(x_0)\big) < \varepsilon\Big).                          & \text{(by Definition \ref{1.2.1})}
    \end{align*}
    Since \(\varepsilon\) is arbitrary, we have
    \[
        \forall\ \varepsilon \in \mathbf{R}^+, \exists\ \delta \in \mathbf{R}^+ : \Big(\forall\ x \in X, d_X(x, x_0) < \delta \implies d_Y\big(f(x), f(x_0)\big) < \varepsilon\Big).
    \]
    Thus by Definition \ref{2.1.1} \(f\) is continuous at \(x_0\) in \((X, d_X)\).
    We conclude that statements (a)(b)(c) are equivalent.
\end{proof}

\begin{theorem}\label{2.1.5}
    Let \((X, d_X)\) be a metric space, and let \((Y, d_Y)\) be another metric space.
    Let \(f : X \to Y\) be a function.
    Then the following four statements are equivalent:
    \begin{enumerate}
        \item \(f\) is continuous.
        \item Whenever \((x^{(n)})_{n = 1}^\infty\) is a sequence in \(X\) which converges to some point \(x_0 \in X\) with respect to the metric \(d_X\), the sequence \(\big(f(x^{(n)})\big)_{n = 1}^\infty\) converges to \(f(x_0)\) with respect to the metric \(d_Y\).
        \item Whenever \(V\) is an open set in \(Y\), the set \(f^{-1}(V) \coloneqq \{x \in X : f(x) \in V\}\) is an open set in \(X\).
        \item Whenever \(F\) is a closed set in \(Y\), the set \(f^{-1}(F) \coloneqq \{x \in X : f(x) \in F\}\) is a closed set in \(X\).
    \end{enumerate}
\end{theorem}

\begin{proof}
    We first show that statements (a)(b) are equivalent.
    \begin{align*}
             & f \text{ is continuous in } (X, d_X)                                                                             \\
        \iff & \forall\ x_0 \in X, f \text{ is continuous at } x_0 \text{ in } (X, d_X) & \text{(by Definition \ref{2.1.1})}    \\
        \iff & \forall\ x_0 \in X, \Big(\lim_{n \to \infty} d_X(x^{(n)}, x_0) = 0                                               \\
             & \implies \lim_{n \to \infty} d_Y\big(f(x^{(n)}), f(x_0)\big) = 0\Big).   & \text{(by Theorem \ref{2.1.4}(a)(b))}
    \end{align*}

    Next we show that statements (a) implies statement (c).
    Suppose that \(f\) is continuous in \((X, d_X)\).
    Let \(V\) be an open set in \((Y, d_Y)\).
    Then we have
    \begin{align*}
                 & f \text{ is continuous in } (X, d_X)                                                                                                       \\
        \implies & f \text{ is continuous in } (f^{-1}(V), d_X)                                                     & \text{(by Remark \ref{2.1.3})}          \\
        \implies & \forall\ x_0 \in f^{-1}(V), f \text{ is continuous at } x_0 \text{ in } (f^{-1}(V), d_X)         & \text{(by Definition \ref{2.1.1})}      \\
        \implies & \forall\ x_0 \in f^{-1}(V), \exists\ U \subseteq X :                                                                                       \\
                 & \big(U \text{ is open in } (X, d_X)\big) \land (x_0 \in U) \land \big(f(U) \subseteq V\big)      & \text{(by Theorem \ref{2.1.4}(a)(c))}   \\
        \implies & \forall\ x_0 \in f^{-1}(V), \exists\ U \subseteq X :                                                                                       \\
                 & \big(U \text{ is open in } (X, d_X)\big) \land (x_0 \in U) \land \big(U \subseteq f^{-1}(V)\big)                                           \\
        \implies & \forall\ x_0 \in f^{-1}(V), \exists\ U \subseteq X :                                                                                       \\
                 & \big(\exists\ r \in \mathbf{R}^+ : B_{(X, d_X)}(x_0, r) \subseteq U \subseteq f^{-1}(V)\big)     & \text{(by Proposition \ref{1.2.15}(a))} \\
        \implies & f^{(-1)}(V) \text{ is open in } (X, d_X).                                                        & \text{(by Proposition \ref{1.2.15}(a))}
    \end{align*}
    Since \(V\) is arbitrary, we know that statement (a) implies statement (c).

    Next we show that statements (c) implies statement (a).
    Suppose that
    \[
        \forall\ V \subseteq Y, V \text{ is open in } (Y, d_Y) \implies f^{-1}(V) \text{ is open in } (X, d_X).
    \]
    Let \(x_0 \in X\).
    Then we have
    \begin{align*}
                 & \forall\ V \subseteq Y, \big(V \text{ is open in } (Y, d_Y)\big) \land \big(f(x_0) \in V\big)                          \\
        \implies & \big(f^{-1}(V) \text{ is open in } (X, d_X)\big) \land \big(x_0 \in f^{-1}(V)\big)            & \text{(by hypothesis)}
    \end{align*}
    and by Theorem \ref{2.1.4}(a)(c) we know that \(f\) is continuous at \(x_0\) in \((X, d_X)\).
    Since \(x_0\) is arbitrary, we know that \(f\) is continuous in \((X, d_X)\).
    Thus statements (c) implies statement (a) and from the proof above we conclude that statements (a)(c) are equivalent.

    Next we show that statements (c) implies statement (d).
    Suppose that
    \[
        \forall\ V \subseteq Y, V \text{ is open in } (Y, d_Y) \implies f^{-1}(V) \text{ is open in } (X, d_X).
    \]
    Let \(F\) be an closed set in \((Y, d_Y)\).
    Then we have
    \begin{align*}
                 & F \text{ is closed in } (Y, d_Y)                                                                                          \\
        \implies & Y \setminus F \text{ is open in } (Y, d_Y)                                      & \text{(by Proposition \ref{1.2.15}(e))} \\
        \implies & f^{-1}(Y \setminus F) \text{ is open in } (X, d_X)                              & \text{(by hypothesis)}                  \\
        \implies & X \setminus f^{-1}(Y \setminus F) \text{ is closed in } (X, d_X)                & \text{(by Proposition \ref{1.2.15}(e))} \\
        \implies & X \setminus \{x \in X : f(x) \in Y \setminus F\} \text{ is closed in } (X, d_X)                                           \\
        \implies & \{x \in X : f(x) \in F\} \text{ is closed in } (X, d_X)                                                                   \\
        \implies & f^{-1}(F) \text{ is closed in } (X, d_X).
    \end{align*}
    Since \(F\) is arbitrary, we know that statement (c) implies statement (d).

    Finally we show that statements (d) implies statement (c).
    Suppose that
    \[
        \forall\ F \subseteq Y, F \text{ is closed in } (Y, d_Y) \implies f^{-1}(F) \text{ is closed in } (X, d_X).
    \]
    Let \(V\) be an open set in \((Y, d_Y)\).
    Then we have
    \begin{align*}
                 & V \text{ is open in } (Y, d_Y)                                                                                          \\
        \implies & Y \setminus V \text{ is closed in } (Y, d_Y)                                  & \text{(by Proposition \ref{1.2.15}(e))} \\
        \implies & f^{-1}(Y \setminus V) \text{ is closed in } (X, d_X)                          & \text{(by hypothesis)}                  \\
        \implies & X \setminus f^{-1}(Y \setminus V) \text{ is open in } (X, d_X)                & \text{(by Proposition \ref{1.2.15}(e))} \\
        \implies & X \setminus \{x \in X : f(x) \in Y \setminus V\} \text{ is open in } (X, d_X)                                           \\
        \implies & \{x \in X : f(x) \in V\} \text{ is open in } (X, d_X)                                                                   \\
        \implies & f^{-1}(V) \text{ is open in } (X, d_X).
    \end{align*}
    Since \(V\) is arbitrary, we know that statement (d) implies statement (c).
    We conclude that statements (a)(b)(c)(d) are all equivalent.
\end{proof}

\begin{remark}\label{2.1.6}
    It may seem strange that continuity ensures that the \emph{inverse} image of an open set is open.
    One may guess instead that the reverse should be true, that the \emph{forward} image of an open set is open;
    but this is not true;
    see Exercises \ref{ex 1.5.4}, \ref{ex 1.5.5}.
\end{remark}

\begin{corollary}[Continuity preserved by composition]\label{2.1.7}
    Let \((X, d_X)\), \((Y, d_Y)\), and \((Z, d_Z)\) be metric spaces.
    \begin{enumerate}
        \item If \(f : X \to Y\) is continuous at a point \(x_0 \in X\), and \(g : Y \to Z\) is continuous at \(f(x_0)\), then the composition \(g \circ f : X \to Z\), defined by \(g \circ f(x) \coloneqq g(f(x))\), is continuous at \(x_0\).
        \item If \(f : X \to Y\) is continuous, and \(g : Y \to Z\) is continuous, then \(g \circ f : X \to Z\) is also continuous.
    \end{enumerate}
\end{corollary}

\begin{proof}{(a)}
    Since \(f\) is continuous at \(x_0\) in \((X, d_X)\), by Theorem \ref{2.1.4}(a)(c) we know that
    \begin{align*}
                 & \forall\ V \subseteq Y, \big(V \text{ is open in } (Y, d_Y)\big) \land \big(f(x_0) \in V\big)                        \\
        \implies & \exists\ U \subseteq X : \big(U \text{ is open in } (X, d_X)\big) \land (x_0 \in U) \land \big(f(U) \subseteq V\big)
    \end{align*}
    Now we choose such \(U\) for each open set \(V\) in \((Y, d_Y)\) and denote it as \(U_V\).
    Since \(g\) is continuous at \(f(x_0)\) in \((Y, d_Y)\), by Theorem \ref{2.1.4}(a)(c) we know that
    \begin{align*}
                 & \forall\ W \subseteq Z, \big(W \text{ is open in } (Z, d_Z)\big) \land \Big(g\big(f(x_0)\big) \in W\Big)                                                \\
        \implies & \exists\ V \subseteq Y : \big(V \text{ is open in } (Y, d_Y)\big) \land (y_0 \in V) \land \big(g(V) \subseteq W\big)                                    \\
        \implies & \exists\ U_V \subseteq X : \big(U_V \text{ is open in } (X, d_X)\big) \land (x_0 \in U_V) \land \big(f(U_V) \subseteq V\big)                            \\
        \implies & \exists\ U_V \subseteq X : \big(U_V \text{ is open in } (X, d_X)\big) \land (x_0 \in U_V) \land \Big(g\big(f(U_V)\big) \subseteq g(V) \subseteq W\Big).
    \end{align*}
    Thus by Theorem \ref{2.1.4}(a)(c) we know that \(g \circ f\) is continuous at \(x_0\) in \((X, d_X)\).
\end{proof}

\begin{proof}{(b)}
    Let \(x_0 \in X\).
    Then we have
    \begin{align*}
                 & f \text{ is continuous in } (X, d_X)                                                       \\
        \implies & f \text{ is continuous at } x_0 \text{ in } (X, d_X). & \text{(by Definition \ref{2.1.1})}
    \end{align*}
    Since \(f(x_0) \in Y\), we have
    \begin{align*}
                 & g \text{ is continuous in } (Y, d_Y)                                                                 \\
        \implies & g \text{ is continuous at } f(x_0) \text{ in } (Y, d_Y)       & \text{(by Definition \ref{2.1.1})}   \\
        \implies & g \circ f \text{ is continuous at } x_0 \text{ in } (X, d_X). & \text{(by Corollary \ref{2.1.7}(a))}
    \end{align*}
    Since \(x_0\) is arbitrary, by Definition \ref{2.1.1} we know that \(g \circ f\) is continuous in \((X, d_X)\).
\end{proof}