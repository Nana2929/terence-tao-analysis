\section{Continuity and compactness}\label{sec 2.3}

\begin{theorem}[Continuous maps preserve compactness]\label{2.3.1}
    Let \(f : X \to Y\) be a continuous map from one metric space \((X, d_X)\) to another \((Y, d_Y)\).
    Let \(K \subseteq X\) be any compact subset of \(X\).
    Then the image \(f(K) \coloneqq \{f(x) : x \in K\}\) of \(K\) is also compact.
\end{theorem}

\begin{proof}
    Let \(\bigcup_{\alpha \in I} V_\alpha\) be an open cover of \(f(K)\) in \((Y, d_Y)\), i.e., \(I \subseteq Y\) and for each \(\alpha \in I\), \(V_{\alpha}\) is an open set in \((Y, d_Y)\) such that \(f(K) \subseteq \bigcup_{\alpha \in I} V_\alpha\).
    Since
    \begin{align*}
                 & f \text{ is continuous from } (X, d_X) \text{ to } (Y, d_Y)                                                                                                             \\
        \implies & f|_K \text{ is continuous from } (K, d_X|_{K \times K}) \text{ to } (Y, d_Y)                                                  & \text{(by Remark \ref{2.1.3})}          \\
        \implies & \forall\ \alpha \in I, f|_K^{-1}(V_\alpha) \text{ is open in } (K, d_X|_{K \times K})                                         & \text{(by Theorem \ref{2.1.5}(a)(d))}   \\
        \implies & K = \bigcup_{\alpha \in I} f|_K^{-1}(V_\alpha) \text{ is open in } (K, d_X|_{K \times K})                                     & \text{(by Proposition \ref{1.2.15}(g))} \\
        \implies & \exists\ F \subseteq I : (F \text{ is finite}) \land \bigg(K = \bigcup_{\alpha \in F} f|_K^{-1}(V_\alpha)\bigg)               & \text{(by Theorem \ref{1.5.8})}         \\
        \implies & \exists\ F \subseteq I : (F \text{ is finite}) \land \bigg(f(K) = f\big(\bigcup_{\alpha \in F} f|_K^{-1}(V_\alpha)\big)\bigg)                                           \\
        \implies & \exists\ F \subseteq I : (F \text{ is finite}) \land \bigg(f(K) \subseteq \bigcup_{\alpha \in F} V_\alpha)\bigg),
    \end{align*}
    we know that there exists an finite subcover of \(f(K)\) with respect to \(\bigcup_{\alpha \in I} V_\alpha\) in \((Y, d_Y)\).
    Since \(I\) is arbitrary open cover of \(f(K)\) in \((Y, d_Y)\), by Exercise \ref{ex 1.5.11} we know that \(\big(f(K), d_Y|_{Y \times Y}\big)\) is compact.
\end{proof}

\begin{proposition}[Maximum principle]\label{2.3.2}
    Let \((X, d)\) be a compact metric space, and let \(f : X \to \mathbf{R}\) be a continuous function.
    Then \(f\) is bounded.
    Furthermore, if \(X\) is non-empty, \(f\) attains its maximum at some point \(x_{\max} \in X\), and also attains its minimum at some point \(x_{\min} \in X\).
\end{proposition}

\begin{proof}
    We have
    \begin{align*}
                 & \begin{cases}
            (X, d) \text{ is compact} \\
            f \text{ is continuous from } (X, d) \text{ to } (\mathbf{R}, d_{l^1}|_{\mathbf{R} \times \mathbf{R}})
        \end{cases}                                                                            \\
        \implies & \big(f(X), d_{l^1}|_{f(X) \times f(X)}\big) \text{ is compact} & \text{(by Theorem \ref{2.3.1})}     \\
        \implies & \begin{cases}
            f(X) \text{ is closed in } (\mathbf{R}, d_{l^1}|_{\mathbf{R} \times \mathbf{R}}) \\
            \big(f(X), d_{l^1}|_{f(X) \times f(X)}\big) \text{ is bounded}
        \end{cases}                                      & \text{(by Corollary \ref{1.5.6})}   \\
        \implies & \begin{cases}
            f(X) \text{ is closed in } (\mathbf{R}, d_{l^1}|_{\mathbf{R} \times \mathbf{R}}) \\
            f(X) \text{ is bounded subset of } \mathbf{R}
        \end{cases}                                      & \text{(by Exercise \ref{ex 1.5.1})}
    \end{align*}

    Now we show that if \(X \neq \emptyset\), then
    \[
        \exists\ x_{\min}, x_{\max} \in X : \forall\ x \in X, f(x_{\min}) \leq f(x) \leq f(x_{\max}).
    \]
    Let \(U = \sup\big(f(X)\big)\) and let \(L = \inf\big(f(X)\big)\).
    Since \(f\) is bounded subset of \(\mathbf{R}\), we know that \(U, L \in \mathbf{R}\).
    By the definition of \(U\) and \(L\) we know that
    \[
        \forall\ n \in \mathbf{Z}^+, \exists\ u_n, l_n \in f(X) : \begin{cases}
            U - u_n < \frac{1}{n} \\
            l_n - L < \frac{1}{n}
        \end{cases}
    \]
    Thus we have
    \begin{align*}
                 & \begin{cases}
            0 = \lim_{n \to \infty} U - u_n = \lim_{n \to \infty} \frac{1}{n} = 0 \\
            0 = \lim_{n \to \infty} l_n - L = \lim_{n \to \infty} \frac{1}{n} = 0
        \end{cases} \\
        \implies & \begin{cases}
            \lim_{n \to \infty} u_n = U \\
            \lim_{n \to \infty} l_n = L
        \end{cases}
    \end{align*}
    Since \(f(X)\) is closed in \((\mathbf{R}, d_{l^1}|_{\mathbf{R} \times \mathbf{R}})\) and \((u_n)_{n = 1}^\infty\), \((l_n)_{n = 1}^\infty\) are convergent sequences in \(\mathbf{R}\) with respect to \(d_{l^1}|_{\mathbf{R} \times \mathbf{R}}\), by Proposition \ref{1.2.15}(b) we know that \(U, L \in f(X)\).
    Since \(U, L \in f(X)\), we know that
    \[
        \exists\ x_{\min}, x_{\max} \in X : \big(f(x_{\min}) = L\big) \land \big(f(x_{\max}) = U\big).
    \]
\end{proof}

\begin{remark}\label{2.3.3}
    As was already noted in Exercise 9.6.1 in Analysis I, this principle can fail if \(X\) is not compact.
    Proposition \ref{2.3.2} should be compared with Lemma 9.6.3 in Analysis I and Proposition 9.6.7 in Analysis I.
\end{remark}