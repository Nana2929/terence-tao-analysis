\section{Continuity and connectedness}\label{sec 2.4}

\begin{definition}[Connected spaces]\label{2.4.1}
    Let \((X, d)\) be a metric space.
    We say that \(X\) is \emph{disconnected} iff there exist disjoint non-empty open sets \(V\) and \(W\) in \(X\) such that \(V \cup W = X\).
    (Equivalently, \(X\) is disconnected if and only if \(X\) contains a non-empty proper subset which is simultaneously closed and open, see Proposition \ref{1.2.15}(e).)
    We say that \(X\) is \emph{connected} iff it is non-empty and not disconnected.
\end{definition}

\begin{note}
    We declare the empty set \(\emptyset\) as being special
    - it is neither connected nor disconnected;
    one could think of the empty set as ``unconnected''.
\end{note}

\setcounter{theorem}{2}
\begin{definition}[Connected sets]\label{2.4.3}
    Let \((X, d)\) be a metric space, and let \(Y\) be a subset of \(X\).
    We say that \(Y\) is \emph{connected} iff the metric space \((Y, d|_{Y \times Y})\) is connected, and we say that \(Y\) is \emph{disconnected} iff the metric space \((Y, d|_{Y \times Y})\) is disconnected.
\end{definition}

\begin{remark}\label{2.4.4}
    This definition is intrinsic;
    whether a set \(Y\) is connected or not depends only on what the metric is doing on \(Y\), but not on what ambient space \(X\) one placing \(Y\) in.
\end{remark}