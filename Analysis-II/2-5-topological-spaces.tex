\section{Topological spaces}\label{sec 2.5}

\begin{note}
    The concept of a metric space can be generalized to that of a \emph{topological space}.
    The idea here is not to view the metric \(d\) as the fundamental object;
    indeed, in a general topological space there is no metric at all.
    Instead, it is the collection of \emph{open sets} which is the fundamental concept.
    Thus, whereas in a metric space one introduces the metric \(d\) first, and then uses the metric to define first the concept of an open ball and then the concept of an open set, in a topological space one starts just with the notion of an open set.
    As it turns out, starting from the open sets, one cannot necessarily reconstruct a usable notion of a ball or metric (thus not all topological spaces will be metric spaces), but remarkably one can still define many of the concepts in the preceding sections.
\end{note}

\begin{definition}[Topological spaces]\label{2.5.1}
    A \emph{topological space} is a pair \((X, \mathcal{F})\), where \(X\) is a set, and \(\mathcal{F} \subseteq 2^X\) is a collection of subsets of \(X\), whose elements are referred to as \emph{open sets}.
    Furthermore, the collection \(\mathcal{F}\) must obey the following properties:
    \begin{itemize}
        \item The empty set \(\emptyset\) and the whole set \(X\) are open;
              in other words, \(\emptyset \in \mathcal{F}\) and \(X \in \mathcal{F}\).
        \item Any finite intersection of open sets is open.
              In other words, if \(V_1 , \dots, V_n\) are elements of \(\mathcal{F}\), then \(V_1 \cap \dots \cap V_n\) is also in \(\mathcal{F}\).
        \item Any arbitrary union of open sets is open (including infinite unions).
              In other words, if \((V_\alpha)_{\alpha \in I}\) is a family of sets in \(\mathcal{F}\), then \(\bigcup_{\alpha \in I} V_\alpha\) is also in \(\mathcal{F}\).
    \end{itemize}
\end{definition}

\begin{note}
    In many cases, the collection \(\mathcal{F}\) of open sets can be deduced from context, and we shall refer to the topological space \((X, \mathcal{F})\) simply as \(X\).
\end{note}

\begin{note}
    From Proposition \ref{1.2.15} we see that every metric space \((X, d)\) is automatically also a topological space
    (if we set \(\mathcal{F}\) equal to the collection of sets which are open in \((X, d)\)).
    However, there do exist topological spaces which do not arise from metric spaces.
\end{note}

\begin{definition}[Neighbourhoods]\label{2.5.2}
    Let \((X, \mathcal{F})\) be a topological space, and let \(x \in X\).
    A neighbourhood of \(x\) is defined to be any open set in \(\mathcal{F}\) which contains \(x\).
\end{definition}