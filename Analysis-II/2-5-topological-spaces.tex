\section{Topological spaces}\label{sec 2.5}

\begin{note}
    The concept of a metric space can be generalized to that of a \emph{topological space}.
    The idea here is not to view the metric \(d\) as the fundamental object;
    indeed, in a general topological space there is no metric at all.
    Instead, it is the collection of \emph{open sets} which is the fundamental concept.
    Thus, whereas in a metric space one introduces the metric \(d\) first, and then uses the metric to define first the concept of an open ball and then the concept of an open set, in a topological space one starts just with the notion of an open set.
    As it turns out, starting from the open sets, one cannot necessarily reconstruct a usable notion of a ball or metric (thus not all topological spaces will be metric spaces), but remarkably one can still define many of the concepts in the preceding sections.
\end{note}

\begin{definition}[Topological spaces]\label{2.5.1}
    A \emph{topological space} is a pair \((X, \mathcal{F})\), where \(X\) is a set, and \(\mathcal{F} \subseteq 2^X\) is a collection of subsets of \(X\), whose elements are referred to as \emph{open sets}.
    Furthermore, the collection \(\mathcal{F}\) must obey the following properties:
    \begin{itemize}
        \item The empty set \(\emptyset\) and the whole set \(X\) are open;
              in other words, \(\emptyset \in \mathcal{F}\) and \(X \in \mathcal{F}\).
        \item Any finite intersection of open sets is open.
              In other words, if \(V_1 , \dots, V_n\) are elements of \(\mathcal{F}\), then \(V_1 \cap \dots \cap V_n\) is also in \(\mathcal{F}\).
        \item Any arbitrary union of open sets is open (including infinite unions).
              In other words, if \((V_\alpha)_{\alpha \in I}\) is a family of sets in \(\mathcal{F}\), then \(\bigcup_{\alpha \in I} V_\alpha\) is also in \(\mathcal{F}\).
    \end{itemize}
\end{definition}

\begin{note}
    In many cases, the collection \(\mathcal{F}\) of open sets can be deduced from context, and we shall refer to the topological space \((X, \mathcal{F})\) simply as \(X\).
\end{note}

\begin{note}
    From Proposition \ref{1.2.15} we see that every metric space \((X, d)\) is automatically also a topological space
    (if we set \(\mathcal{F}\) equal to the collection of sets which are open in \((X, d)\)).
    However, there do exist topological spaces which do not arise from metric spaces.
\end{note}

\begin{definition}[Neighbourhoods]\label{2.5.2}
    Let \((X, \mathcal{F})\) be a topological space, and let \(x \in X\).
    A \emph{neighbourhood} of \(x\) is defined to be any open set in \(\mathcal{F}\) which contains \(x\).
\end{definition}

\begin{example}\label{2.5.3}
    If \((X, d)\) is a metric space, \(x \in X\), and \(r > 0\), then \(B_{(X, d)}(x, r)\) is a neighbourhood of \(x\) (see Proposition \ref{1.2.15}(c)).
\end{example}

\begin{definition}[Topological convergence]\label{2.5.4}
    Let m be an integer, \((X, \mathcal{F})\) be a topological space and let \((x^{(n)})_{n = m}^\infty\) be a sequence of points in \(X\).
    Let \(x\) be a point in \(X\).
    We say that \((x^{(n)})_{n = m}^\infty\) \emph{converges to} \(x\) if and only if, for every neighbourhood \(V\) of \(x\), there exists an \(N \geq m\) such that \(x^{(n)} \in V\) for all \(n \geq N\).
\end{definition}

\begin{note}
    Definition \ref{2.5.4} is consistent with that of convergence in metric spaces (Definition \ref{1.1.14}).
    One can then ask whether one has the basic property of uniqueness of limits (Proposition \ref{1.1.20}).
    The answer turns out to usually be yes
    - if the topological space has an additional property known as the Hausdorff property
    - but the answer can be no for other topologies.
\end{note}

\begin{definition}[Interior, exterior, boundary]\label{2.5.5}
    Let \((X, \mathcal{F})\) be a topological space, let \(E\) be a subset of \(X\), and let \(x_0\) be a point in \(X\).
    We say that \(x_0\) is an \emph{interior point of} \(E\) if there exists a neighbourhood \(V\) of \(x_0\) such that \(V \subseteq E\).
    We say that \(x_0\) is an \emph{exterior point of} \(E\) if there exists a neighbourhood \(V\) of \(x_0\) such that \(V \cap E = \emptyset\).
    We say that \(x_0\) is a \emph{boundary point of} \(E\) if it is neither an interior point nor an exterior point of \(E\).
\end{definition}

\begin{note}
    Definition \ref{2.5.5} is consistent with the corresponding notion for metric spaces (Definition \ref{1.2.5}).
\end{note}

\begin{definition}[Closure]\label{2.5.6}
    Let \((X, \mathcal{F})\) be a topological space, let \(E\) be a subset of \(X\), and let \(x_0\) be a point in \(X\).
    We say that \(x_0\) is an adherent point of \(E\) if every neighbourhood \(V\) of \(x_0\) has a non-empty intersection with \(E\).
    The set of all adherent points of \(E\) is called the closure of \(E\) and is denoted \(\overline{E}\).
\end{definition}

\begin{note}
    We define a set \(K\) in a topological space \((X, \mathcal{F})\) to be closed iff its complement \(X \setminus K\) is open;
    this is consistent with the metric space definition, thanks to Proposition \ref{1.2.15}(e).
\end{note}

\begin{definition}[Relative topology]\label{2.5.7}
    Let \((X, \mathcal{F})\) be a topological space, and \(Y\) be a subset of \(X\).
    Then we define \(\mathcal{F}_Y \coloneqq \{V \cap Y : V \in F\}\), and refer this as the topology on \(Y\) \emph{induced} by \((X, \mathcal{F})\).
    We call \((Y, \mathcal{F}_Y)\) a \emph{topological subspace} of \((X, \mathcal{F})\).
\end{definition}

\begin{note}
    From Proposition \ref{1.3.4} we see that Definition \ref{2.5.7} is compatible with the one for metric spaces.
\end{note}

\begin{definition}[Continuous functions]\label{2.5.8}
    Let \((X, \mathcal{F}_X)\) and \((Y, \mathcal{F}_Y)\) be topological spaces, and let \(f : X \to Y\) be a function.
    If \(x_0 \in X\), we say that \(f\) is \emph{continuous at} \(x_0\) iff for every neighbourhood \(V\) of \(f(x_0)\), there exists a neighbourhood \(U\) of \(x_0\) such that \(f(U) \subseteq V\).
    We say that \(f\) is \emph{continuous} iff it is continuous at every point \(x \in X\).
\end{definition}

\begin{note}
    Definition \ref{2.5.8} is consistent with that in Definition \ref{2.1.1}.
    In particular, a function is continuous iff the pre-images of every open set is open.
\end{note}

\begin{note}
    There is unfortunately no notion of a Cauchy sequence, a complete space, or a bounded space, for topological spaces.
    However, there is certainly a notion of a compact space.
\end{note}

\begin{definition}[Compact topological spaces]\label{2.5.9}
    Let \((X, \mathcal{F})\) be a topological space.
    We say that this space is \emph{compact} if every open cover of \(X\) has a finite subcover.
    If \(Y\) is a subset of \(X\), we say that \(Y\) is compact if the topological space on \(Y\) induced by \((X, \mathcal{F})\) is compact.
\end{definition}

\begin{note}
    Many basic facts about compact metric spaces continue to hold true for compact topological spaces, notably Theorem \ref{2.3.1} and Proposition \ref{2.3.2}.
    However, there is no notion of uniform continuity, and so there is no analogue of Theorem \ref{2.3.5}.
\end{note}

\begin{note}
    We can also define the notion of connectedness by repeating Definition \ref{2.4.1} verbatim, and also repeating Definition \ref{2.4.3} (but with Definition \ref{2.5.7} instead of Definition \ref{1.3.3}).
    Many of the results and exercises in Section \ref{sec 2.4} continue to hold for topological spaces
    (with almost no changes to any of the proofs!).
\end{note}

\exercisesection

\begin{exercise}\label{ex 2.5.1}
    Let \(X\) be an arbitrary set, and let \(\mathcal{F} \coloneqq \{\emptyset, X\}\).
    Show that \((X, \mathcal{F})\) is a topology
    (called the \emph{trivial topology} on \(X\)).
    If \(X\) contains more than one element, show that the trivial topology cannot be obtained from by placing a metric \(d\) on \(X\).
    Show that this topological space is both compact and connected.
\end{exercise}

\begin{proof}
    Let \(X\) be a set and let \(\mathcal{F} = \{\emptyset, X\}\).
    First we show that \((X, \mathcal{F})\) is a topology.
    Let \(n \in \mathbf{N}\), let \(S_1, \dots, S_n \in \mathcal{F}\) and let \(i, j \in \mathbf{Z}^+\).
    If there exists some \(1 \leq j \leq n\) such that \(S_j = \emptyset\), then we know that \(\bigcap_{i = 1}^n S_i = \emptyset \in \mathcal{F}\).
    If such \(j\) does not exist, then we have \(S_i = X\) for every \(1 \leq i \leq n\) and \(\bigcap_{i = 1}^n S_i = X \in \mathcal{F}\).
    Since \(n\) is arbitrary, we conclude that for arbitrary finite collection of element in \(\mathcal{F}\) there intersection is still in \(\mathcal{F}\).

    Let \(S \subseteq 2^{\mathcal{F}}\).
    Then we have
    \[
        \forall\ s \in S, (s = \emptyset) \lor (s = X) \implies \bigcup S \in F
    \]
    and we conclude that any union of open sets is open.

    Since \(\emptyset, X \in \mathcal{F}\) and the claims above, by Definition \ref{2.5.1} we know that \((X, \mathcal{F})\) is a topology.

    Next we show that if \(X\) contains more than one element, then \((X, \mathcal{F})\) cannot be obtained from by placing a metric \(d\) on \(X\).
    Let \((X, \mathcal{F})\) be a trivial topology and let \(x, y \in X\) such that \(x \neq y\).
    Given arbitrary metric \(d\), by Definition \ref{1.1.2}(b) we know that \(d(x, y) > \mathbf{R}^+\).
    But by Proposition \ref{1.2.15}(c) we know that \(B_{(X, d)}\big(x, \frac{d(x, y)}{2}\big)\) is open in \((X, d)\), thus by Definition \ref{2.5.1} we must have \(B_{(X, d)}\big(x, \frac{d(x, y)}{2}\big) \in \mathcal{F}\), which means \((X, \mathcal{F})\) is a not trivial topology.

    Finally we show that if \(X\) contains more than one element, then \((X, \mathcal{F})\) is compact and connected.
    Since \(\emptyset, X\) are the only two open sets in \(\mathcal{F}\), we know that an open cover of \(X\) is either \(\{X\}\) or \(\{\emptyset, X\}\), and both are finite.
    Thus by Definition \ref{2.5.9} \((X, \mathcal{F})\) is compact.
    Since \(X\) is the only non-empty open set in \(\mathcal{F}\), by Definition \ref{2.4.3} we know that \((X, \mathcal{F})\) is connected.
\end{proof}

\begin{exercise}\label{ex 2.5.2}
    Let \((X, d)\) be a metric space
    (and hence a topological space).
    Show that the two notions of convergence of sequences in Definition \ref{1.1.14} and Definition \ref{2.5.4} coincide.
\end{exercise}

\begin{proof}
    Let \(\mathcal{F}\) be the set of all open sets in \((X, d)\) and let \(N \in \mathbf{N}\).
    First suppose that \((x^{(n)})_{n = m}^\infty\) converges to \(x\) in the sense of Definition \ref{1.1.14}.
    Let \(V \in \mathcal{F}\) be a neighbourhood of \(x\).
    By Definition \ref{2.5.2} we know that \(V\) is open in \((X, d)\).
    Since \(x \in V\), by Proposition \ref{1.2.15}(a) we know that
    \[
        \exists\ \varepsilon \in \mathbf{R}^+ : B_{(X, d)}(x, \varepsilon) \subseteq V.
    \]
    Now we fix such \(\varepsilon\).
    Then we have
    \begin{align*}
                 & \lim_{n \to \infty} d(x^{(n)}, x) = 0                                                                                    \\
        \implies & \exists\ N \geq m : \forall\ n \geq N, d(x^{(n)}, x) < \varepsilon            & \text{(by Definition \ref{1.1.14})}      \\
        \implies & \exists\ N \geq m : \forall\ n \geq N, x^{(n)} \in B_{(X, d)}(x, \varepsilon) & \text{(by Definition \ref{1.2.1})}       \\
        \implies & \exists\ N \geq m : \forall\ n \geq N, x^{(n)} \in V.                         & (B_{(X, d)}(x, \varepsilon) \subseteq V)
    \end{align*}
    Since \(V\) is arbitrary, we know that \((x^{(n)})_{n = m}^\infty\) converges to \(x\) in the sense of Definition \ref{2.5.4}.

    Now suppose that \((x^{(n)})_{n = m}^\infty\) converges to \(x\) in the sense of Definition \ref{2.5.4}.
    Then we have
    \begin{align*}
                 & \forall\ \varepsilon \in \mathbf{R}^+, B_{(X, d)}(x, \varepsilon) \in \mathcal{F}                                    & \text{(by Proposition \ref{1.2.15}(c))} \\
        \implies & \forall\ \varepsilon \in \mathbf{R}^+, \exists\ N \geq m : \forall\ n \geq N, x^{(n)} \in B_{(X, d)}(x, \varepsilon) & \text{(by Definition \ref{2.5.4})}      \\
        \implies & \forall\ \varepsilon \in \mathbf{R}^+, \exists\ N \geq m : \forall\ n \geq N, d(x^{(n)}, x) < \varepsilon            & \text{(by Definition \ref{1.2.1})}      \\
        \implies & \lim_{n \to \infty} d(x^{(n)}, x) = 0.                                                                               & \text{(by Definition \ref{1.1.14})}
    \end{align*}
    Thus Definition \ref{1.1.14} and Definition \ref{2.5.4} coincide.
\end{proof}

\begin{exercise}\label{ex 2.5.3}
    Let \((X, d)\) be a metric space (and hence a topological space).
    Show that the two notions of interior, exterior, and boundary in Definition
    \ref{1.2.5} and Definition \ref{2.5.5} coincide.
\end{exercise}

\begin{proof}
    Let \(\mathcal{F}\) be the set of all open sets in \((X, d)\), let \(E \subseteq X\) and let \(x_0 \in X\).
    First suppose that \(x_0\) is an interior point of \(E\) in the sense of Definition \ref{1.2.5}.
    Then we have
    \begin{align*}
                 & x_0 \in \text{int}_{(X, d)}(E)                                                                         \\
        \implies & \exists\ r \in \mathbf{R}^+ : B_{(X, d)}(x_0, r) \subseteq E & \text{(by Definition \ref{1.2.5})}      \\
        \implies & \exists\ r \in \mathbf{R}^+ : \begin{cases}
            B_{(X, d)}(x_0, r) \in \mathcal{F} \\
            B_{(X, d)}(x_0, r) \subseteq E
        \end{cases}     & \text{(by Proposition \ref{1.2.15}(c))} \\
        \implies & x_0 \in \text{int}_{(X, \mathcal{F})}(E).                    & \text{(by Definition \ref{2.5.5})}
    \end{align*}

    Next suppose that \(x_0\) is an interior point of \(E\) in the sense of Definition \ref{2.5.5}.
    Then we have
    \begin{align*}
                 & x_0 \in \text{int}_{(X, \mathcal{F})}(E)                                                                           \\
        \implies & \exists\ V \in \mathcal{F} : (x_0 \in V) \land (V \subseteq E)           & \text{(by Definition \ref{2.5.5})}      \\
        \implies & \exists\ r \in \mathbf{R}^+ : B_{(X, d)}(x_0, r) \subseteq V \subseteq E & \text{(by Proposition \ref{1.2.15}(a))} \\
        \implies & x_0 \in \text{int}_{(X, d)}(E).                                          & \text{(by Definition \ref{1.2.5})}
    \end{align*}

    Next suppose that \(x_0\) is an exterior point of \(E\) in the sense of Definition \ref{1.2.5}.
    Then we have
    \begin{align*}
                 & x_0 \in \text{ext}_{(X, d)}(E)                                                                                \\
        \implies & \exists\ r \in \mathbf{R}^+ : B_{(X, d)}(x_0, r) \cap E = \emptyset & \text{(by Definition \ref{1.2.5})}      \\
        \implies & \exists\ r \in \mathbf{R}^+ : \begin{cases}
            B_{(X, d)}(x_0, r) \in \mathcal{F} \\
            B_{(X, d)}(x_0, r) \cap E = \emptyset
        \end{cases}            & \text{(by Proposition \ref{1.2.15}(c))} \\
        \implies & x_0 \in \text{ext}_{(X, \mathcal{F})}(E).                           & \text{(by Definition \ref{2.5.5})}
    \end{align*}

    Next suppose that \(x_0\) is an exterior point of \(E\) in the sense of Definition \ref{2.5.5}.
    Then we have
    \begin{align*}
                 & x_0 \in \text{ext}_{(X, \mathcal{F})}(E)                                                                        \\
        \implies & \exists\ V \in \mathcal{F} : (x_0 \in V) \land (V \cap E = \emptyset) & \text{(by Definition \ref{2.5.5})}      \\
        \implies & \exists\ r \in \mathbf{R}^+ : \begin{cases}
            B_{(X, d)}(x_0, r) \subseteq V \\
            B_{(X, d)}(x_0, r) \cap E = \emptyset
        \end{cases}              & \text{(by Proposition \ref{1.2.15}(a))} \\
        \implies & x_0 \in \text{ext}_{(X, d)}(E).                                       & \text{(by Definition \ref{1.2.5})}
    \end{align*}

    Next suppose that \(x_0\) is an boundary point of \(E\) in the sense of Definition \ref{1.2.5}.
    Then we have
    \begin{align*}
                 & x_0 \in \partial_{(X, d)}(E)                                                                      \\
        \implies & \forall\ r \in \mathbf{R}^+, \begin{cases}
            B_{(X, d)}(x_0, r) \not\subseteq E \\
            B_{(X, d)}(x_0, r) \cap E \neq \emptyset
        \end{cases} & \text{(by Definition \ref{1.2.5})}      \\
        \implies & \forall\ V \in \mathcal{F}, x_0 \in V \text{ implies}                                             \\
                 & \begin{cases}
            \exists\ r \in \mathbf{R}^+ : B_{(X, d)}(x_0, r) \subseteq V \\
            V \not\subseteq E                                            \\
            V \cap E \neq \emptyset
        \end{cases}                              & \text{(by Proposition \ref{1.2.15}(a))} \\
        \implies & x_0 \in \partial_{(X, \mathcal{F})}(E).                 & \text{(by Definition \ref{2.5.5})}
    \end{align*}

    Finally suppose that \(x_0\) is an boundary point of \(E\) in the sense of Definition \ref{2.5.5}.
    Then we have
    \begin{align*}
                 & x_0 \in \partial_{(X, \mathcal{F})}(E)                                                                                     \\
        \implies & \forall\ V \in \mathcal{F}, x_0 \in V \text{ implies} \begin{cases}
            V \not\subseteq E \\
            V \cap E \neq \emptyset
        \end{cases} & \text{(by Definition \ref{2.5.5})}      \\
        \implies & \forall\ r \in \mathbf{R}^+, \begin{cases}
            B_{(X, d)}(x_0, r) \in \mathcal{F} \\
            B_{(X, d)}(x_0, r) \not\subseteq E \\
            B_{(X, d)}(x_0, r) \cap E \neq \emptyset
        \end{cases}                          & \text{(by Proposition \ref{1.2.15}(c))} \\
        \implies & x_0 \in \partial_{(X, d)}(E).                                                    & \text{(by Definition \ref{1.2.5})}
    \end{align*}
    Thus Definition \ref{1.2.5} and Definition \ref{2.5.5} coincide.
\end{proof}

\begin{exercise}\label{ex 2.5.4}
    A topological space \((X, \mathcal{F})\) is said to be \emph{Hausdorff} if given any two distinct points \(x, y \in X\), there exists a neighbourhood \(V\) of \(x\) and a neighbourhood \(W\) of \(y\) such that \(V \cap W = \emptyset\).
    Show that any topological space coming from a metric space is Hausdorff, and show that the trivial topology is not Hausdorff if the space contains at least two elements.
    Show that the analogue of Proposition \ref{1.1.20} holds for Hausdorff topological spaces, but give an example of a non-Hausdorff topological space in which Proposition \ref{1.1.20} fails.
    (In practice, most topological spaces one works with are Hausdorff;
    non-Hausdorff topological spaces tend to be so pathological that it is not very profitable to work with them.)
\end{exercise}

\begin{proof}
    We first show that every topological space coming from a metric space is Hausdorff.
    Let \((X, d)\) be a metric space and let \(\mathcal{F}\) be the set of all open sets in \((X, d)\).
    Let \(x, y \in X\).
    Then we have
    \begin{align*}
                 & x \neq y                                                                                                                                                             \\
        \implies & d(x, y) \in \mathbf{R}^+                                                                                                   & \text{(by Definition \ref{1.1.2}(b))}   \\
        \implies & \begin{cases}
            B_{(X, d)}\big(x, \frac{d(x, y)}{2}\big) \in \mathcal{F} \\
            B_{(X, d)}\big(y, \frac{d(x, y)}{2}\big) \in \mathcal{F}
        \end{cases}                                                                                                 & \text{(by Proposition \ref{1.2.15}(c))} \\
        \implies & \bigg(B_{(X, d)}\big(x, \frac{d(x, y)}{2}\big)\bigg) \cap \bigg(B_{(X, d)}\big(y, \frac{d(x, y)}{2}\big)\bigg) = \emptyset & \text{(by Definition \ref{1.1.2}(d))}   \\
        \implies & (X, \mathcal{F}) \text{ is a Hausdorff space}.                                                                             & \text{(by definition)}
    \end{align*}

    Next we show that if a trivial topological space contains at least two elements, then it is not Hausdorff.
    Let \(X\) be a set such that \(x, y \in X\) and \(x \neq y\).
    Let \(\mathcal{F} = \{\emptyset, X\}\).
    By Definition \ref{2.5.2} the only neighbourhood of \(x\) in \(\mathcal{F}\) is \(X\), similarly the only neighbourhood of \(y\) in \(\mathcal{F}\) is \(X\).
    But \(X \cap X \neq \emptyset\) implies \((X, \mathcal{F})\) is not Hausdorff.

    Next we show that every convergent sequence in a Hausdorff space has only one limit.
    Let \((X, \mathcal{F})\) be a Hausdorff space, let \((x^{(n)})_{n = 1}^\infty\) be a sequence in \(X\) and let \(x, x' \in X\) such that \((x^{(n)})_{n = 1}^\infty\) converges to \(x, x'\), respectively.
    Suppose for sake of contradiction that \(x \neq x'\).
    Since \(x \neq x'\) and \((X, \mathcal{F})\) is Hausdorff, by definition we know that
    \[
        \exists\ V, V' \in \mathcal{F} : \begin{cases}
            x \in V   \\
            x' \in V' \\
            V \cap V' = \emptyset
        \end{cases}
    \]
    But then we have
    \begin{align*}
                 & \begin{cases}
            (x^{(n)})_{n = 1}^\infty \text{ converges to } x \\
            (x^{(n)})_{n = 1}^\infty \text{ converges to } x'
        \end{cases}                                                                                                \\
        \implies & \begin{cases}
            \exists\ N \in \mathbf{Z}^+ : \forall\ n \geq N, x^{(n)} \in V \\
            \exists\ N' \in \mathbf{Z}^+ : \forall\ n \geq N', x^{(n)} \in V'
        \end{cases}                                                           & \text{(by Definition \ref{2.5.4})} \\
        \implies & \exists\ N, N' \in \mathbf{Z}^+ : \forall\ n \geq \max(N, N'), x^{(n)} \in V \cap V'                                      \\
        \implies & V \cap V' \neq \emptyset,
    \end{align*}
    a contradiction.
    Thus we must have \(x = x'\).

    Finally we give an non-Hausdorff topology space in which Proposition \ref{1.1.20} fails.
    Let \(X = \{0, 1\}\) and let \(\mathcal{F} = \{\emptyset, X\}\).
    From the proof above we know that \((X, \mathcal{F})\) is not Hausdorff.
    Let \((x^{(n)})_{n = 1}^\infty\) be a sequence in \(X\).
    By Definition \ref{2.5.2} we know that the only neighbourhood of \(0\) in \(\mathcal{F}\) is \(X\).
    Similarly the only neighbourhood of \(1\) in \(\mathcal{F}\) is \(X\).
    Thus we have
    \begin{align*}
                 & \forall\ n \in \mathbf{Z}^+, x^{(n)} \in X                                      \\
        \implies & \begin{cases}
            (x^{(n)})_{n = 1}^\infty \text{ converges to } 0 \\
            (x^{(n)})_{n = 1}^\infty \text{ converges to } 1
        \end{cases}                 & \text{(by Definition \ref{2.5.4})}
    \end{align*}
    but \(0 \neq 1\).
\end{proof}

\begin{exercise}\label{ex 2.5.5}
    Given any totally ordered set \(X\) with order relation \(\leq\), declare a set \(V \subseteq X\) to be \emph{open} if for every \(x \in V\) there exists a set \(I\) which is an interval \(\{y \in X : a < y < b\}\) for some \(a, b \in X\), a ray \(\{y \in X : a < y\}\) for some \(a \in X\), the ray \(\{y \in X : y < b\}\) for some \(b \in X\), or the whole space \(X\), which contains \(x\) and is contained in \(V\).
    Let \(\mathcal{F}\) be the set of all open subsets of \(X\).
    Show that \((X, \mathcal{F})\) is a topology (this is the \emph{order topology} on the totally ordered set \((X, \leq)\)) which is Hausdorff in the sense of Exercise \ref{ex 2.5.4}.
    Show that on the real line \(\mathbf{R}\) (with the standard ordering \(\leq\)), the order topology matches the standard topology (i.e., the topology arising from the standard metric).
    If instead one applies this to the extended real line \(\mathbf{R}^*\), show that \(\mathbf{R}\) is an open set with boundary \(\{-\infty, +\infty\}\).
    If \((x_n)_{n = 1}^\infty\) is a sequence of numbers in \(\mathbf{R}\) (and hence in \(\mathbf{R}^*\)), show that \(x_n\) converges to \(+\infty\) if and only if \(\liminf_{n \to \infty} x_n = +\infty\), and \(x_n\) converges to \(-\infty\) if and only if \(\limsup_{n \to \infty} x_n = -\infty\).
\end{exercise}

\begin{proof}
    We first show that \((X, \mathcal{F})\) is a topology space.
    By definition we know that \(X\) is open and \(\emptyset\) is open trivially, thus \(X, \emptyset \in \mathcal{F}\).
    Let \(n \in \mathbf{N}\) and let \(S_n \subseteq \mathcal{F}\) such that \(\#(S_n) = n\).
    We use induction on \(n\) to show that \(\bigcap S_n \in \mathcal{F}\) for every \(n \in \mathbf{N}\).
    For \(n = 0\), we have \(S_0 = \emptyset\) and \(\bigcap S_0 = \emptyset\).
    From the proof above we know that \(\emptyset \in \mathcal{F}\), thus the base case holds.
    Suppose inductively that \(\bigcap S_n \in \mathcal{F}\) for some \(n \geq 0\).
    Let \(S_{n + 1} \subseteq \mathcal{F}\) such that \(\#(S_{n + 1}) = n + 1\).
    Then we have \(S_{n + 1} = \{V_1, \dots, V_{n + 1} : \forall\ i \in \mathbf{Z}^+, V_i \in \mathcal{F}\}\) and \(\bigcap S_{n + 1} = \bigcap_{i = 1}^{n + 1} V_i\).
    If \(\bigcap S_{n + 1} = \emptyset\), then from the proof above we know that \(\emptyset \in \mathcal{F}\).
    So suppose that \(\bigcap S_{n + 1} \neq \emptyset\).
    Let \(x \in \bigcap S_{n + 1}\).
    Since \(x \in \bigcap_{i = 1}^n V_i\) and \(\#(\{V_1, \dots, V_n\}) = n\), by induction hypothesis we know that there exists a set \(I\) in one of the following forms
    \[
        I = \begin{cases}
            \{y \in X : a < y < b\} \text{ for some } a, b \in X \\
            \{y \in X : a < y\} \text{ for some } a \in X        \\
            \{y \in X : y < b\} \text{ for some } b \in X        \\
            X
        \end{cases}
    \]
    such that \(x \in I\) and \(I \subseteq \bigcap_{i = 1}^n V_i\).
    Since \(x \in V_{n + 1}\) and \(V_{n + 1} \in \mathcal{F}\), we know that there exists a set \(I'\) in one of the following forms
    \[
        I' = \begin{cases}
            \{y \in X : a' < y < b'\} \text{ for some } a', b' \in X \\
            \{y \in X : a' < y\} \text{ for some } a' \in X          \\
            \{y \in X : y < b'\} \text{ for some } b' \in X          \\
            X
        \end{cases}
    \]
    such that \(x \in I'\) and \(I' \subseteq V_{n + 1}\).
    Then we have \(x \in I \cap I'\) and \(I \cap I' \subseteq \bigcap_{i = 1}^{n + 1} V_i\).
    Since \((X, \leq)\) is totally ordered, we know that \(I \cap I'\) is in one of the following forms
    \[
        I \cap I' = \begin{cases}
            \{y \in X : \max_{(X, \leq)}(a, a') < y < \min_{(X, \leq)}(b, b')\} \\
            \{y \in X : \max_{(X, \leq)}(a, a') < y < b\}                       \\
            \{y \in X : a < y < \min_{(X, \leq)}(b, b')\}                       \\
            \{y \in X : a < y < b\}                                             \\
            \{y \in X : \max_{(X, \leq)}(a, a') < y < b'\}                      \\
            \{y \in X : \max_{(X, \leq)}(a, a') < y\}                           \\
            \{y \in X : a < y < b'\}                                            \\
            \{y \in X : a < y\}                                                 \\
            \{y \in X : a' < y < \min_{(X, \leq)}(b, b')\}                      \\
            \{y \in X : a' < y < b\}                                            \\
            \{y \in X : y < \min_{(X, \leq)}(b, b')\}                           \\
            \{y \in X : y < b\}                                                 \\
            \{y \in X : a' < y < b'\}                                           \\
            \{y \in X : a' < y\}                                                \\
            \{y \in X : y < b'\}                                                \\
            X
        \end{cases}
    \]
    Thus by definition \(I \cap I'\) is an interval.
    Since \(x\) is arbitrary, we know that \(\bigcap S_{n + 1}\) is open in \((X, \mathcal{F})\), and this closes the induction.
    We conclude that for any finite collection of open sets, their intersection is again open in \((X, \mathcal{F})\).

    Let \(S \subseteq \mathcal{F}\).
    If \(\bigcup S = \emptyset\), then from proof above we know that \(\emptyset \in \mathcal{F}\).
    So suppose that \(\bigcup S \neq \emptyset\).
    Let \(x \in \bigcup S\).
    We know that there exists an \(V \in S\) such that \(x \in V\).
    Since \(V \in S\), we know that \(V\) is open in \((X, \mathcal{F})\) and by definition there exists an interval \(I\) such that \(x \in I\) and \(I \subseteq V\).
    Then we have \(I \subseteq V \subseteq \bigcup S\).
    Since \(x\) is arbitrary, we know that \(\bigcup S\) is open in \((X, \mathcal{F})\).
    Combine all the results above we know that \((X, \mathcal{F})\) is a topological space by Definition \ref{2.5.1}.

    Next we show that \((X, \mathcal{F})\) is Hausdorff.
    Let \(x_1, x_2 \in X\) such that \(x_1 \neq x_2\).
    Since \((X, \leq)\) is totally ordered, we have either \(x_1 < x_2\) or \(x_2 < x_1\).
    Without the loss of generality suppose that \(x_1 < x_2\).
    Let \(I_1 = \{y \in X : y < x_2\}\) and let \(I_2 = \{y \in X : x_1 < y\}\).
    Then we have \(x_1 \in I_1\) and \(x_2 \in I_2\).
    By definition we know that \(I_1, I_2 \in \mathcal{F}\).
    If \(I_1 \cap I_2 = \emptyset\), then we are done.
    So suppose that \(I_1 \cap I_2 \neq \emptyset\).
    Let \(x \in I_1 \cap I_2\), let \(J_1 = \{y \in X : y < x\}\) and let \(J_2 = \{y \in X : x < y\}\).
    Since \(I_1 \cap I_2 = \{y \in X : x_1 < y < x_2\}\), we know that \(x \neq x_1\) and \(x \neq x_2\).
    Since \(x_1 < x\), we have \(x_1 \in J_1\).
    Similarly we have \(x_2 \in J_2\).
    By definition we know that \(J_1, J_2 \in \mathcal{F}\).
    Since \((X, \leq)\) is totally ordered, we know that \(J_1 \cap J_2 = \emptyset\).
    Since \(x_1, x_2\) are arbitrary, by Exercise \ref{ex 2.5.4} we know that \((X, \mathcal{F})\) is Hausdorff.

    Next we show that the order topology in \(\mathbf{R}\) with order relation \(\leq\) matches standard topology.
    Let \(\mathcal{F}_o\) be the order topology in \(\mathbf{R}\) and let \(\mathcal{F}_s\) be the standard topology in \(\mathbf{R}\).
    We want to show that \(\mathcal{F}_o = \mathcal{F}_s\).

    Let \(V \in \mathcal{F}_o\) and let \(x \in V\).
    Then we have
    \[
        \exists\ I \subseteq \mathbf{R} : \begin{cases}
            I \text{ is an open interval in } \mathbf{R} \\
            x \in I                                      \\
            I \subseteq V
        \end{cases}
    \]
    Now we split into four cases:
    \begin{itemize}
        \item If \(I = (a, b)\) for some \(a, b \in \mathbf{R}\), then we have
              \begin{align*}
                           & x \in (a, b)                                                                                                                      \\
                  \implies & r = \min(\abs*{x - a}, \abs*{x - b}) = \min(x - a, b - x) > 0                                                                     \\
                  \implies & (x - r, x + r) \subseteq (a, b) \subseteq V                                                                                       \\
                  \implies & B_{(\mathbf{R}, d_{l^1}|_{\mathbf{R} \times \mathbf{R}})}(x, r) \subseteq (a, b) \subseteq V & \text{(by Definition \ref{1.2.1})} \\
                  \implies & x \in \text{int}_{(\mathbf{R}, d_{l^1}|_{\mathbf{R} \times \mathbf{R}})}(V).                 & \text{(by Definition \ref{1.2.5})}
              \end{align*}
        \item If \(I = (a, \infty)\) for some \(a \in \mathbf{R}\), then we have
              \begin{align*}
                           & x \in (a, \infty)                                                                                                                      \\
                  \implies & r = \abs*{x - a} = x - a > 0                                                                                                           \\
                  \implies & (x - r, x + r) \subseteq (a, \infty) \subseteq V                                                                                       \\
                  \implies & B_{(\mathbf{R}, d_{l^1}|_{\mathbf{R} \times \mathbf{R}})}(x, r) \subseteq (a, \infty) \subseteq V & \text{(by Definition \ref{1.2.1})} \\
                  \implies & x \in \text{int}_{(\mathbf{R}, d_{l^1}|_{\mathbf{R} \times \mathbf{R}})}(V).                      & \text{(by Definition \ref{1.2.5})}
              \end{align*}
        \item If \(I = (-\infty, b)\) for some \(b \in \mathbf{R}\), then we have
              \begin{align*}
                           & x \in (-\infty, b)                                                                                                                      \\
                  \implies & r = \abs*{x - b} = b - x > 0                                                                                                            \\
                  \implies & (x - r, x + r) \subseteq (-\infty, b) \subseteq V                                                                                       \\
                  \implies & B_{(\mathbf{R}, d_{l^1}|_{\mathbf{R} \times \mathbf{R}})}(x, r) \subseteq (-\infty, b) \subseteq V & \text{(by Definition \ref{1.2.1})} \\
                  \implies & x \in \text{int}_{(\mathbf{R}, d_{l^1}|_{\mathbf{R} \times \mathbf{R}})}(V).                       & \text{(by Definition \ref{1.2.5})}
              \end{align*}
        \item If \(I = \mathbf{R}\), then we have \(V = \mathbf{R}\) and \(x \in \text{int}_{(\mathbf{R}, d_{l^1}|_{\mathbf{R} \times \mathbf{R}})}(V) = \mathbf{R}\).
    \end{itemize}
    From all cases above we conclude that \(x \in \text{int}_{(\mathbf{R}, d_{l^1}|_{\mathbf{R} \times \mathbf{R}})}(V)\).
    Since \(x\) is arbitrary, by Proposition \ref{1.2.15}(a) we know that \(V\) is open in \((X, d_{l^1}|_{\mathbf{R} \times \mathbf{R}})\) and thus \(V \in \mathcal{F}_s\).
    Since \(V\) is arbitrary, we have \(\mathcal{F}_o \subseteq \mathcal{F}_s\).

    Let \(W \in \mathcal{F}_s\).
    Since \(W\) is open in \((\mathbf{R}, d_{l^1}|_{\mathbf{R} \times \mathbf{R}})\), we have
    \begin{align*}
                 & \forall\ x \in W, \exists\ r \in \mathbf{R}^+ : B_{(\mathbf{R}, d_{l^1}|_{\mathbf{R} \times \mathbf{R}})}(x, r) \subseteq W & \text{(by Proposition \ref{1.2.15}(a))} \\
        \implies & \forall\ x \in W, \exists\ r \in \mathbf{R}^+ : (x - r, x + r) \subseteq W                                                  & \text{(by Definition \ref{1.2.1})}      \\
        \implies & W \in \mathcal{F}_o.                                                                                                        & \text{(by definition)}
    \end{align*}
    Since \(W\) is arbitrary, we have \(\mathcal{F}_s \subseteq \mathcal{F}_o\).
    From the proof above we thus have \(\mathcal{F}_o = \mathcal{F}_s\).

    Next we show that if \((\mathbf{R}^*, \mathcal{F})\) is an order topology with order relation \(\leq\), then \(\mathbf{R}\) is open in \((\mathbf{R}^*, \mathcal{F})\).
    Since \((\mathbf{R}^*, \leq)\) is totally ordered, we know that \((\mathbf{R}^*, \mathcal{F})\) is an order topology.
    Since
    \[
        \forall\ x \in \mathbf{R}, (x - 1, x + 1) \subseteq \mathbf{R},
    \]
    by definition we know that \(\mathbf{R}\) is open in \((\mathbf{R}^*, \mathcal{F})\).

    Next we show that if \((\mathbf{R}^*, \mathcal{F})\) is an order topology with order relation \(\leq\), then the boundary points of \(\mathbf{R}\) in \((\mathbf{R}^*, \mathcal{F})\) are \(-\infty\) and \(\infty\).
    Since \(\infty \notin \mathbf{R}\) and \(\mathbf{R}\) is open in \((\mathbf{R}^*, \mathcal{F})\), by Definition \ref{2.5.5} we know that \(\infty\) is either an exterior point or a boundary point of \(\mathbf{R}\) in \((\mathbf{R}^*, \mathcal{F})\).
    Suppose for sake of contradiction that \(\infty\) is an exterior point of \(\mathbf{R}\) in \((\mathbf{R}^*, \mathcal{F})\).
    Then by Definition \ref{2.5.5} we have
    \[
        \exists\ V \in \mathcal{F} : (\infty \in V) \land (\mathbf{R} \cap V = \emptyset).
    \]
    Since \(V \cap \mathbf{R} = \emptyset\), we know that the only possible choices of \(V\) are \(\{\infty\}\) or \(\{-\infty, \infty\}\).
    But in either cases we cannot find an open interval \(I \subseteq \mathbf{R}^*\) such that \(I \subseteq V\), a contradiction.
    Thus \(\infty\) is a boundary point of \(\mathbf{R}\) in \((\mathbf{R}^*, \mathcal{F})\).
    Using similar arguments as above we can show that \(-\infty\) is a boundary point of \(\mathbf{R}\) in \((\mathbf{R}^*, \mathcal{F})\).

    Finally we show that if  \((\mathbf{R}^*, \mathcal{F})\) is an order topology with order relation \(\leq\) and \((x^{(n)})_{n = 1}^\infty\), \((y^{(n)})_{n = 1}^\infty\) are sequence in \(\mathbf{R}\), then we have
    \[
        \begin{cases}
            x_n \text{ converges to } \infty \text{ in } (\mathbf{R}^*, \mathcal{F})  \\
            y_n \text{ converges to } -\infty \text{ in } (\mathbf{R}^*, \mathcal{F}) \\
        \end{cases} \iff \begin{cases}
            \liminf_{n \to \infty} x_n = \infty \\
            \limsup_{n \to \infty} y_n = -\infty
        \end{cases}
    \]
    This is true since
    \begin{align*}
             & \begin{cases}
            \liminf_{n \to \infty} x^{(n)} = \infty \\
            \limsup_{n \to \infty} y^{(n)} = -\infty
        \end{cases} \\
        \iff & \begin{cases}
            \sup\big\{\inf\{x^{(n)} : n \geq N\} : N \geq 1\big\} = \infty \\
            \inf\big\{\sup\{y^{(n)} : n \geq N\} : N \geq 1\big\} = -\infty
        \end{cases} \\
        \iff & \begin{cases}
            \sup\{x^{(n)} : n \geq 1\} = \infty \\
            \inf\{y^{(n)} : n \geq 1\} = -\infty
        \end{cases} \\
        \iff & \begin{cases}
            \forall\ \varepsilon \in \mathbf{R}^+, \exists\ N \geq 1 : \forall\ n \geq N, x^{(n)} > \varepsilon \\
            \forall\ \varepsilon \in \mathbf{R}^+, \exists\ N \geq 1 : \forall\ n \geq N, y^{(n)} < -\varepsilon
        \end{cases} \\
        \iff & \begin{cases}
            \forall\ \varepsilon \in \mathbf{R}^+, \exists\ N \geq 1 : \forall\ n \geq N, x^{(n)} \in (\varepsilon, \infty] \\
            \forall\ \varepsilon \in \mathbf{R}^+, \exists\ N \geq 1 : \forall\ n \geq N, y^{(n)} \in [-\infty, -\varepsilon)
        \end{cases} \\
        \iff & \begin{cases}
            \forall\ V \in \mathcal{F}, \infty \in V \implies \exists\ \varepsilon \in \mathbf{R}^+ : \begin{cases}
                (\varepsilon, \infty] \subseteq V \\
                \exists\ N \geq 1 : \forall\ n \geq N, x^{(n)} \in V
            \end{cases} \\
            \forall\ V \in \mathcal{F}, -\infty \in V \implies \exists\ \varepsilon \in \mathbf{R}^+ : \begin{cases}
                [-\infty, -\varepsilon) \subseteq V \\
                \exists\ N \geq 1 : \forall\ n \geq N, y^{(n)} \in V
            \end{cases}
        \end{cases} \\
        \iff & \begin{cases}
            x_n \text{ converges to } \infty \text{ in } (\mathbf{R}^*, \mathcal{F})  \\
            y_n \text{ converges to } -\infty \text{ in } (\mathbf{R}^*, \mathcal{F}) \\
        \end{cases}
    \end{align*}
\end{proof}

\begin{exercise}\label{ex 2.5.6}
    Let \(X\) be an uncountable set, and let \(\mathcal{F}\) be the collection of all subsets \(E\) in \(X\) which are either empty or co-finite (which means that \(X \setminus E\) is finite).
    Show that \((X, \mathcal{F})\) is a topology (this is called the \emph{cofinite topology} on \(X\)) which is not Hausdorff in the sense of Exercise \ref{ex 2.5.4}, and is compact and connected.
    Also, show that if \(x \in X\) and \((V_n)_{n = 1}^\infty\) is any countable collection of open sets containing \(x\), then \(\bigcap_{n = 1}^\infty V_n \neq \{x\}\).
    Use this to show that the cofinite topology cannot be obtained by placing a metric \(d\) on \(X\).
\end{exercise}

\begin{exercise}\label{ex 2.5.7}
    Let \(X\) be an uncountable set, and let \(\mathcal{F}\) be the collection of all subsets \(E\) in \(X\) which are either empty or co-countable
    (which means that \(X \setminus E\) is at most countable).
    Show that \((X, \mathcal{F})\) is a topology (this is called the \emph{cocountable topology} on \(X\)) which is not Hausdorff in the sense of Exercise \ref{ex 2.5.4}, and connected, but cannot arise from a metric space and is not compact.
\end{exercise}

\setcounter{exercise}{8}
\begin{exercise}\label{ex 2.5.9}
    Let \((X, \mathcal{F})\) be a compact topological space.
    Assume that this space is \emph{first countable}, which means that for every \(x \in X\) there exists a countable collection \(V_1 , V_2 , \dots\) of neighbourhoods of \(x\), such that every neighbourhood of \(x\) contains one of the \(V_n\).
    Show that every sequence in \(X\) has a convergent subsequence, by modifying Exercise \ref{ex 1.5.11}.
\end{exercise}

\begin{exercise}\label{ex 2.5.10}
    Prove the following partial analogue of Proposition \ref{1.2.10} for topological spaces:
    (c) implies both (a) and (b), which are equivalent to each other.
    Show that in the co-countable topology in Exercise \ref{ex 2.5.7}, it is possible for (a) and (b) to hold without (c) holding.
\end{exercise}

\begin{exercise}\label{ex 2.5.11}
    Let \(E\) be a subset of a topological space \((X, \mathcal{F})\)
    Show that \(E\) is open if and only if every element of \(E\) is an interior point, and show that \(E\) is closed if and only if \(E\) contains all of its adherent points.
    Prove analogues of Proposition \ref{1.2.15}(e)-(h) (some of these are automatic by definition).
    If we assume in addition that \(X\) is Hausdorff, prove an analogue of Proposition \ref{1.2.15}(d) also, but give an example to show that (d) can fail when \(X\) is not Hausdorff.
\end{exercise}

\begin{exercise}\label{ex 2.5.12}
    Show that the pair \((Y, \mathcal{F}_Y)\) defined in Definition \ref{2.5.7} is indeed a topological space.
\end{exercise}

\begin{exercise}\label{ex 2.5.13}
    Generalize Corollary \ref{1.5.9} to compact sets in a Hausdorff topological space.
\end{exercise}

\begin{exercise}\label{ex 2.5.14}
    Generalize Theorem \ref{1.5.10} to compact sets in a Hausdorff topological space.
\end{exercise}

\begin{exercise}\label{ex 2.5.15}
    Let \((X, d_X)\) and \((Y, d_Y)\) be metric spaces (and hence a topological space).
    Show that the two notions continuity (both at a point, and on the whole domain) of a function \(f : X \to Y\) in Definition \ref{2.1.1} and Definition \ref{2.5.8} coincide.
\end{exercise}

\begin{exercise}\label{ex 2.5.16}
    Show that when Theorem \ref{2.1.4} is extended to topological spaces, that (a) implies (b).
    (The converse is false, but constructing an example is diffcult.)
    Show that when Theorem \ref{2.1.5} is extended to topological spaces, that (a), (c), (d) are all equivalent to each other, and imply (b).
    (Again, the converse implications are false, but diffcult to prove.)
\end{exercise}

\begin{exercise}\label{ex 2.5.17}
    Generalize both Theorem \ref{2.3.1} and Proposition \ref{2.3.2} to compact sets in a topological space.
\end{exercise}