\section{Topological spaces}\label{sec 2.5}

\begin{note}
    The concept of a metric space can be generalized to that of a \emph{topological space}.
    The idea here is not to view the metric \(d\) as the fundamental object;
    indeed, in a general topological space there is no metric at all.
    Instead, it is the collection of \emph{open sets} which is the fundamental concept.
    Thus, whereas in a metric space one introduces the metric \(d\) first, and then uses the metric to define first the concept of an open ball and then the concept of an open set, in a topological space one starts just with the notion of an open set.
    As it turns out, starting from the open sets, one cannot necessarily reconstruct a usable notion of a ball or metric (thus not all topological spaces will be metric spaces), but remarkably one can still define many of the concepts in the preceding sections.
\end{note}

\begin{definition}[Topological spaces]\label{2.5.1}
    A \emph{topological space} is a pair \((X, \mathcal{F})\), where \(X\) is a set, and \(\mathcal{F} \subseteq 2^X\) is a collection of subsets of \(X\), whose elements are referred to as \emph{open sets}.
    Furthermore, the collection \(\mathcal{F}\) must obey the following properties:
    \begin{itemize}
        \item The empty set \(\emptyset\) and the whole set \(X\) are open;
              in other words, \(\emptyset \in \mathcal{F}\) and \(X \in \mathcal{F}\).
        \item Any finite intersection of open sets is open.
              In other words, if \(V_1 , \dots, V_n\) are elements of \(\mathcal{F}\), then \(V_1 \cap \dots \cap V_n\) is also in \(\mathcal{F}\).
        \item Any arbitrary union of open sets is open (including infinite unions).
              In other words, if \((V_\alpha)_{\alpha \in I}\) is a family of sets in \(\mathcal{F}\), then \(\bigcup_{\alpha \in I} V_\alpha\) is also in \(\mathcal{F}\).
    \end{itemize}
\end{definition}

\begin{note}
    In many cases, the collection \(\mathcal{F}\) of open sets can be deduced from context, and we shall refer to the topological space \((X, \mathcal{F})\) simply as \(X\).
\end{note}

\begin{note}
    From Proposition \ref{1.2.15} we see that every metric space \((X, d)\) is automatically also a topological space
    (if we set \(\mathcal{F}\) equal to the collection of sets which are open in \((X, d)\)).
    However, there do exist topological spaces which do not arise from metric spaces.
\end{note}

\begin{definition}[Neighbourhoods]\label{2.5.2}
    Let \((X, \mathcal{F})\) be a topological space, and let \(x \in X\).
    A \emph{neighbourhood} of \(x\) is defined to be any open set in \(\mathcal{F}\) which contains \(x\).
\end{definition}

\begin{example}\label{2.5.3}
    If \((X, d)\) is a metric space, \(x \in X\), and \(r > 0\), then \(B_{(X, d)}(x, r)\) is a neighbourhood of \(x\) (see Proposition \ref{1.2.15}(c)).
\end{example}

\begin{definition}[Topological convergence]\label{2.5.4}
    Let m be an integer, \((X, \mathcal{F})\) be a topological space and let \((x^{(n)})_{n = m}^\infty\) be a sequence of points in \(X\).
    Let \(x\) be a point in \(X\).
    We say that \((x^{(n)})_{n = m}^\infty\) \emph{converges to} \(x\) if and only if, for every neighbourhood \(V\) of \(x\), there exists an \(N \geq m\) such that \(x^{(n)} \in V\) for all \(n \geq N\).
\end{definition}

\begin{note}
    Definition \ref{2.5.4} is consistent with that of convergence in metric spaces (Definition \ref{1.1.14}).
    One can then ask whether one has the basic property of uniqueness of limits (Proposition \ref{1.1.20}).
    The answer turns out to usually be yes
    - if the topological space has an additional property known as the Hausdorff property
    - but the answer can be no for other topologies.
\end{note}

\begin{definition}[Interior, exterior, boundary]\label{2.5.5}
    Let \((X, \mathcal{F})\) be a topological space, let \(E\) be a subset of \(X\), and let \(x_0\) be a point in \(X\).
    We say that \(x_0\) is an \emph{interior point of} \(E\) if there exists a neighbourhood \(V\) of \(x_0\) such that \(V \subseteq E\).
    We say that \(x_0\) is an \emph{exterior point of} \(E\) if there exists a neighbourhood \(V\) of \(x_0\) such that \(V \cap E = \emptyset\).
    We say that \(x_0\) is a \emph{boundary point of} \(E\) if it is neither an interior point nor an exterior point of \(E\).
\end{definition}

\begin{note}
    Definition \ref{2.5.5} is consistent with the corresponding notion for metric spaces (Definition \ref{1.2.5}).
\end{note}

\begin{definition}[Closure]\label{2.5.6}
    Let \((X, \mathcal{F})\) be a topological space, let \(E\) be a subset of \(X\), and let \(x_0\) be a point in \(X\).
    We say that \(x_0\) is an adherent point of \(E\) if every neighbourhood \(V\) of \(x_0\) has a non-empty intersection with \(E\).
    The set of all adherent points of \(E\) is called the closure of \(E\) and is denoted \(\overline{E}\).
\end{definition}

\begin{note}
    We define a set \(K\) in a topological space \((X, \mathcal{F})\) to be closed iff its complement \(X \setminus K\) is open;
    this is consistent with the metric space definition, thanks to Proposition \ref{1.2.15}(e).
\end{note}

\begin{definition}[Relative topology]\label{2.5.7}
    Let \((X, \mathcal{F})\) be a topological space, and \(Y\) be a subset of \(X\).
    Then we define \(\mathcal{F}_Y \coloneqq \{V \cap Y : V \in F\}\), and refer this as the topology on \(Y\) \emph{induced} by \((X, \mathcal{F})\).
    We call \((Y, \mathcal{F}_Y)\) a \emph{topological subspace} of \((X, \mathcal{F})\).
\end{definition}

\begin{note}
    From Proposition \ref{1.3.4} we see that Definition \ref{2.5.7} is compatible with the one for metric spaces.
\end{note}

\begin{definition}[Continuous functions]\label{2.5.8}
    Let \((X, \mathcal{F}_X)\) and \((Y, \mathcal{F}_Y)\) be topological spaces, and let \(f : X \to Y\) be a function.
    If \(x_0 \in X\), we say that \(f\) is \emph{continuous at} \(x_0\) iff for every neighbourhood \(V\) of \(f(x_0)\), there exists a neighbourhood \(U\) of \(x_0\) such that \(f(U) \subseteq V\).
    We say that \(f\) is \emph{continuous} iff it is continuous at every point \(x \in X\).
\end{definition}

\begin{note}
    Definition \ref{2.5.8} is consistent with that in Definition \ref{2.1.1}.
    In particular, a function is continuous iff the pre-images of every open set is open.
\end{note}

\begin{note}
    There is unfortunately no notion of a Cauchy sequence, a complete space, or a bounded space, for topological spaces.
    However, there is certainly a notion of a compact space.
\end{note}

\begin{definition}[Compact topological spaces]\label{2.5.9}
    Let \((X, \mathcal{F})\) be a topological space.
    We say that this space is \emph{compact} if every open cover of \(X\) has a finite subcover.
    If \(Y\) is a subset of \(X\), we say that \(Y\) is compact if the topological space on \(Y\) induced by \((X, \mathcal{F})\) is compact.
\end{definition}

\begin{note}
    Many basic facts about compact metric spaces continue to hold true for compact topological spaces, notably Theorem \ref{2.3.1} and Proposition \ref{2.3.2}.
    However, there is no notion of uniform continuity, and so there is no analogue of Theorem \ref{2.3.5}.
\end{note}

\begin{note}
    We can also define the notion of connectedness by repeating Definition \ref{2.4.1} verbatim, and also repeating Definition \ref{2.4.3} (but with Definition \ref{2.5.7} instead of Definition \ref{1.3.3}).
    Many of the results and exercises in Section \ref{sec 2.4} continue to hold for topological spaces
    (with almost no changes to any of the proofs!).
\end{note}

\exercisesection

\begin{exercise}\label{ex 2.5.1}
    Let \(X\) be an arbitrary set, and let \(\mathcal{F} \coloneqq \{\emptyset, X\}\).
    Show that \((X, \mathcal{F})\) is a topology
    (called the \emph{trivial topology} on \(X\)).
    If \(X\) contains more than one element, show that the trivial topology cannot be obtained from by placing a metric \(d\) on \(X\).
    Show that this topological space is both compact and connected.
\end{exercise}

\begin{proof}
    Let \(X\) be a set and let \(\mathcal{F} = \{\emptyset, X\}\).
    First we show that \((X, \mathcal{F})\) is a topology.
    Let \(n \in \N\), let \(S_1, \dots, S_n \in \mathcal{F}\) and let \(i, j \in \Z^+\).
    If there exists some \(1 \leq j \leq n\) such that \(S_j = \emptyset\), then we know that \(\bigcap_{i = 1}^n S_i = \emptyset \in \mathcal{F}\).
    If such \(j\) does not exist, then we have \(S_i = X\) for every \(1 \leq i \leq n\) and \(\bigcap_{i = 1}^n S_i = X \in \mathcal{F}\).
    Since \(n\) is arbitrary, we conclude that for arbitrary finite collection of element in \(\mathcal{F}\) there intersection is still in \(\mathcal{F}\).

    Let \(S \subseteq 2^{\mathcal{F}}\).
    Then we have
    \[
        \forall s \in S, (s = \emptyset) \lor (s = X) \implies \bigcup S \in F
    \]
    and we conclude that any union of open sets is open.

    Since \(\emptyset, X \in \mathcal{F}\) and the claims above, by Definition \ref{2.5.1} we know that \((X, \mathcal{F})\) is a topology.

    Next we show that if \(X\) contains more than one element, then \((X, \mathcal{F})\) cannot be obtained from by placing a metric \(d\) on \(X\).
    Let \((X, \mathcal{F})\) be a trivial topology and let \(x, y \in X\) such that \(x \neq y\).
    Given arbitrary metric \(d\), by Definition \ref{1.1.2}(b) we know that \(d(x, y) > \R^+\).
    But by Proposition \ref{1.2.15}(c) we know that \(B_{(X, d)}\big(x, \frac{d(x, y)}{2}\big)\) is open in \((X, d)\), thus by Definition \ref{2.5.1} we must have \(B_{(X, d)}\big(x, \frac{d(x, y)}{2}\big) \in \mathcal{F}\), which means \((X, \mathcal{F})\) is a not trivial topology.

    Finally we show that if \(X\) contains more than one element, then \((X, \mathcal{F})\) is compact and connected.
    Since \(\emptyset, X\) are the only two open sets in \(\mathcal{F}\), we know that an open cover of \(X\) is either \(\{X\}\) or \(\{\emptyset, X\}\), and both are finite.
    Thus by Definition \ref{2.5.9} \((X, \mathcal{F})\) is compact.
    Since \(X\) is the only non-empty open set in \(\mathcal{F}\), by Definition \ref{2.4.3} we know that \((X, \mathcal{F})\) is connected.
\end{proof}

\begin{exercise}\label{ex 2.5.2}
    Let \((X, d)\) be a metric space
    (and hence a topological space).
    Show that the two notions of convergence of sequences in Definition \ref{1.1.14} and Definition \ref{2.5.4} coincide.
\end{exercise}

\begin{proof}
    Let \(\mathcal{F}\) be the set of all open sets in \((X, d)\) and let \(N \in \N\).
    First suppose that \((x^{(n)})_{n = m}^\infty\) converges to \(x\) in the sense of Definition \ref{1.1.14}.
    Let \(V \in \mathcal{F}\) be a neighbourhood of \(x\).
    By Definition \ref{2.5.2} we know that \(V\) is open in \((X, d)\).
    Since \(x \in V\), by Proposition \ref{1.2.15}(a) we know that
    \[
        \exists\ \varepsilon \in \R^+ : B_{(X, d)}(x, \varepsilon) \subseteq V.
    \]
    Now we fix such \(\varepsilon\).
    Then we have
    \begin{align*}
                 & \lim_{n \to \infty} d(x^{(n)}, x) = 0                                                                                   \\
        \implies & \exists\ N \geq m : \forall n \geq N, d(x^{(n)}, x) < \varepsilon            & \text{(by Definition \ref{1.1.14})}      \\
        \implies & \exists\ N \geq m : \forall n \geq N, x^{(n)} \in B_{(X, d)}(x, \varepsilon) & \text{(by Definition \ref{1.2.1})}       \\
        \implies & \exists\ N \geq m : \forall n \geq N, x^{(n)} \in V.                         & (B_{(X, d)}(x, \varepsilon) \subseteq V)
    \end{align*}
    Since \(V\) is arbitrary, we know that \((x^{(n)})_{n = m}^\infty\) converges to \(x\) in the sense of Definition \ref{2.5.4}.

    Now suppose that \((x^{(n)})_{n = m}^\infty\) converges to \(x\) in the sense of Definition \ref{2.5.4}.
    Then we have
    \begin{align*}
                 & \forall \varepsilon \in \R^+, B_{(X, d)}(x, \varepsilon) \in \mathcal{F}                                   & \text{(by Proposition \ref{1.2.15}(c))} \\
        \implies & \forall \varepsilon \in \R^+, \exists\ N \geq m : \forall n \geq N, x^{(n)} \in B_{(X, d)}(x, \varepsilon) & \text{(by Definition \ref{2.5.4})}      \\
        \implies & \forall \varepsilon \in \R^+, \exists\ N \geq m : \forall n \geq N, d(x^{(n)}, x) < \varepsilon            & \text{(by Definition \ref{1.2.1})}      \\
        \implies & \lim_{n \to \infty} d(x^{(n)}, x) = 0.                                                                     & \text{(by Definition \ref{1.1.14})}
    \end{align*}
    Thus Definition \ref{1.1.14} and Definition \ref{2.5.4} coincide.
\end{proof}

\begin{exercise}\label{ex 2.5.3}
    Let \((X, d)\) be a metric space (and hence a topological space).
    Show that the two notions of interior, exterior, and boundary in Definition
    \ref{1.2.5} and Definition \ref{2.5.5} coincide.
\end{exercise}

\begin{proof}
    Let \(\mathcal{F}\) be the set of all open sets in \((X, d)\), let \(E \subseteq X\) and let \(x_0 \in X\).
    First suppose that \(x_0\) is an interior point of \(E\) in the sense of Definition \ref{1.2.5}.
    Then we have
    \begin{align*}
                 & x_0 \in \text{int}_{(X, d)}(E)                                                             \\
        \implies & \exists\ r \in \R^+ : B_{(X, d)}(x_0, r) \subseteq E & \text{(by Definition \ref{1.2.5})}  \\
        \implies & \exists\ r \in \R^+ : \begin{cases}
                                             B_{(X, d)}(x_0, r) \in \mathcal{F} \\
                                             B_{(X, d)}(x_0, r) \subseteq E
                                         \end{cases}                & \text{(by Proposition \ref{1.2.15}(c))} \\
        \implies & x_0 \in \text{int}_{(X, \mathcal{F})}(E).            & \text{(by Definition \ref{2.5.5})}
    \end{align*}

    Next suppose that \(x_0\) is an interior point of \(E\) in the sense of Definition \ref{2.5.5}.
    Then we have
    \begin{align*}
                 & x_0 \in \text{int}_{(X, \mathcal{F})}(E)                                                                   \\
        \implies & \exists\ V \in \mathcal{F} : (x_0 \in V) \land (V \subseteq E)   & \text{(by Definition \ref{2.5.5})}      \\
        \implies & \exists\ r \in \R^+ : B_{(X, d)}(x_0, r) \subseteq V \subseteq E & \text{(by Proposition \ref{1.2.15}(a))} \\
        \implies & x_0 \in \text{int}_{(X, d)}(E).                                  & \text{(by Definition \ref{1.2.5})}
    \end{align*}

    Next suppose that \(x_0\) is an exterior point of \(E\) in the sense of Definition \ref{1.2.5}.
    Then we have
    \begin{align*}
                 & x_0 \in \text{ext}_{(X, d)}(E)                                                                    \\
        \implies & \exists\ r \in \R^+ : B_{(X, d)}(x_0, r) \cap E = \emptyset & \text{(by Definition \ref{1.2.5})}  \\
        \implies & \exists\ r \in \R^+ : \begin{cases}
                                             B_{(X, d)}(x_0, r) \in \mathcal{F} \\
                                             B_{(X, d)}(x_0, r) \cap E = \emptyset
                                         \end{cases}                       & \text{(by Proposition \ref{1.2.15}(c))} \\
        \implies & x_0 \in \text{ext}_{(X, \mathcal{F})}(E).                   & \text{(by Definition \ref{2.5.5})}
    \end{align*}

    Next suppose that \(x_0\) is an exterior point of \(E\) in the sense of Definition \ref{2.5.5}.
    Then we have
    \begin{align*}
                 & x_0 \in \text{ext}_{(X, \mathcal{F})}(E)                                                                    \\
        \implies & \exists\ V \in \mathcal{F} : (x_0 \in V) \land (V \cap E = \emptyset) & \text{(by Definition \ref{2.5.5})}  \\
        \implies & \exists\ r \in \R^+ : \begin{cases}
                                             B_{(X, d)}(x_0, r) \subseteq V \\
                                             B_{(X, d)}(x_0, r) \cap E = \emptyset
                                         \end{cases}                                 & \text{(by Proposition \ref{1.2.15}(a))} \\
        \implies & x_0 \in \text{ext}_{(X, d)}(E).                                       & \text{(by Definition \ref{1.2.5})}
    \end{align*}

    Next suppose that \(x_0\) is an boundary point of \(E\) in the sense of Definition \ref{1.2.5}.
    Then we have
    \begin{align*}
                 & x_0 \in \partial_{(X, d)}(E)                                                                 \\
        \implies & \forall r \in \R^+, \begin{cases}
                                           B_{(X, d)}(x_0, r) \not\subseteq E \\
                                           B_{(X, d)}(x_0, r) \cap E \neq \emptyset
                                       \end{cases}                & \text{(by Definition \ref{1.2.5})}          \\
        \implies & \forall V \in \mathcal{F}, x_0 \in V \text{ implies}                                         \\
                 & \begin{cases}
                       \exists\ r \in \R^+ : B_{(X, d)}(x_0, r) \subseteq V \\
                       V \not\subseteq E                                    \\
                       V \cap E \neq \emptyset
                   \end{cases} & \text{(by Proposition \ref{1.2.15}(a))}                                        \\
        \implies & x_0 \in \partial_{(X, \mathcal{F})}(E).                 & \text{(by Definition \ref{2.5.5})}
    \end{align*}

    Finally suppose that \(x_0\) is an boundary point of \(E\) in the sense of Definition \ref{2.5.5}.
    Then we have
    \begin{align*}
                 & x_0 \in \partial_{(X, \mathcal{F})}(E)                                                                  \\
        \implies & \forall V \in \mathcal{F}, x_0 \in V \text{ implies} \begin{cases}
                                                                            V \not\subseteq E \\
                                                                            V \cap E \neq \emptyset
                                                                        \end{cases} & \text{(by Definition \ref{2.5.5})}   \\
        \implies & \forall r \in \R^+, \begin{cases}
                                           B_{(X, d)}(x_0, r) \in \mathcal{F} \\
                                           B_{(X, d)}(x_0, r) \not\subseteq E \\
                                           B_{(X, d)}(x_0, r) \cap E \neq \emptyset
                                       \end{cases}                           & \text{(by Proposition \ref{1.2.15}(c))}     \\
        \implies & x_0 \in \partial_{(X, d)}(E).                                      & \text{(by Definition \ref{1.2.5})}
    \end{align*}
    Thus Definition \ref{1.2.5} and Definition \ref{2.5.5} coincide.
\end{proof}

\begin{exercise}\label{ex 2.5.4}
    A topological space \((X, \mathcal{F})\) is said to be \emph{Hausdorff} if given any two distinct points \(x, y \in X\), there exists a neighbourhood \(V\) of \(x\) and a neighbourhood \(W\) of \(y\) such that \(V \cap W = \emptyset\).
    Show that any topological space coming from a metric space is Hausdorff, and show that the trivial topology is not Hausdorff if the space contains at least two elements.
    Show that the analogue of Proposition \ref{1.1.20} holds for Hausdorff topological spaces, but give an example of a non-Hausdorff topological space in which Proposition \ref{1.1.20} fails.
    (In practice, most topological spaces one works with are Hausdorff;
    non-Hausdorff topological spaces tend to be so pathological that it is not very profitable to work with them.)
\end{exercise}

\begin{proof}
    We first show that every topological space coming from a metric space is Hausdorff.
    Let \((X, d)\) be a metric space and let \(\mathcal{F}\) be the set of all open sets in \((X, d)\).
    Let \(x, y \in X\).
    Then we have
    \begin{align*}
                 & x \neq y                                                                                                                                                           \\
        \implies & d(x, y) \in \R^+                                                                                                           & \text{(by Definition \ref{1.1.2}(b))} \\
        \implies & \begin{cases}
                       B_{(X, d)}\big(x, \frac{d(x, y)}{2}\big) \in \mathcal{F} \\
                       B_{(X, d)}\big(y, \frac{d(x, y)}{2}\big) \in \mathcal{F}
                   \end{cases}                                                          & \text{(by Proposition \ref{1.2.15}(c))}                                                     \\
        \implies & \bigg(B_{(X, d)}\big(x, \frac{d(x, y)}{2}\big)\bigg) \cap \bigg(B_{(X, d)}\big(y, \frac{d(x, y)}{2}\big)\bigg) = \emptyset & \text{(by Definition \ref{1.1.2}(d))} \\
        \implies & (X, \mathcal{F}) \text{ is a Hausdorff space}.                                                                             & \text{(by definition)}
    \end{align*}

    Next we show that if a trivial topological space contains at least two elements, then it is not Hausdorff.
    Let \(X\) be a set such that \(x, y \in X\) and \(x \neq y\).
    Let \(\mathcal{F} = \{\emptyset, X\}\).
    By Definition \ref{2.5.2} the only neighbourhood of \(x\) in \(\mathcal{F}\) is \(X\), similarly the only neighbourhood of \(y\) in \(\mathcal{F}\) is \(X\).
    But \(X \cap X \neq \emptyset\) implies \((X, \mathcal{F})\) is not Hausdorff.

    Next we show that every convergent sequence in a Hausdorff space has only one limit.
    Let \((X, \mathcal{F})\) be a Hausdorff space, let \((x^{(n)})_{n = 1}^\infty\) be a sequence in \(X\) and let \(x, x' \in X\) such that \((x^{(n)})_{n = 1}^\infty\) converges to \(x, x'\), respectively.
    Suppose for sake of contradiction that \(x \neq x'\).
    Since \(x \neq x'\) and \((X, \mathcal{F})\) is Hausdorff, by definition we know that
    \[
        \exists\ V, V' \in \mathcal{F} : \begin{cases}
            x \in V   \\
            x' \in V' \\
            V \cap V' = \emptyset
        \end{cases}
    \]
    But then we have
    \begin{align*}
                 & \begin{cases}
                       (x^{(n)})_{n = 1}^\infty \text{ converges to } x \\
                       (x^{(n)})_{n = 1}^\infty \text{ converges to } x'
                   \end{cases}                            \\
        \implies & \begin{cases}
                       \exists\ N \in \Z^+ : \forall n \geq N, x^{(n)} \in V \\
                       \exists\ N' \in \Z^+ : \forall n \geq N', x^{(n)} \in V'
                   \end{cases}                    & \text{(by Definition \ref{2.5.4})}         \\
        \implies & \exists\ N, N' \in \Z^+ : \forall n \geq \max(N, N'), x^{(n)} \in V \cap V' \\
        \implies & V \cap V' \neq \emptyset,
    \end{align*}
    a contradiction.
    Thus we must have \(x = x'\).

    Finally we give an non-Hausdorff topology space in which Proposition \ref{1.1.20} fails.
    Let \(X = \{0, 1\}\) and let \(\mathcal{F} = \{\emptyset, X\}\).
    From the proof above we know that \((X, \mathcal{F})\) is not Hausdorff.
    Let \((x^{(n)})_{n = 1}^\infty\) be a sequence in \(X\).
    By Definition \ref{2.5.2} we know that the only neighbourhood of \(0\) in \(\mathcal{F}\) is \(X\).
    Similarly the only neighbourhood of \(1\) in \(\mathcal{F}\) is \(X\).
    Thus we have
    \begin{align*}
                 & \forall n \in \Z^+, x^{(n)} \in X                \\
        \implies & \begin{cases}
                       (x^{(n)})_{n = 1}^\infty \text{ converges to } 0 \\
                       (x^{(n)})_{n = 1}^\infty \text{ converges to } 1
                   \end{cases} & \text{(by Definition \ref{2.5.4})}
    \end{align*}
    but \(0 \neq 1\).
\end{proof}

\begin{exercise}\label{ex 2.5.5}
    Given any totally ordered set \(X\) with order relation \(\leq\), declare a set \(V \subseteq X\) to be \emph{open} if for every \(x \in V\) there exists a set \(I\) which is an interval \(\{y \in X : a < y < b\}\) for some \(a, b \in X\), a ray \(\{y \in X : a < y\}\) for some \(a \in X\), the ray \(\{y \in X : y < b\}\) for some \(b \in X\), or the whole space \(X\), which contains \(x\) and is contained in \(V\).
    Let \(\mathcal{F}\) be the set of all open subsets of \(X\).
    Show that \((X, \mathcal{F})\) is a topology (this is the \emph{order topology} on the totally ordered set \((X, \leq)\)) which is Hausdorff in the sense of Exercise \ref{ex 2.5.4}.
    Show that on the real line \(\R\) (with the standard ordering \(\leq\)), the order topology matches the standard topology (i.e., the topology arising from the standard metric).
    If instead one applies this to the extended real line \(\R^*\), show that \(\R\) is an open set with boundary \(\{-\infty, +\infty\}\).
    If \((x_n)_{n = 1}^\infty\) is a sequence of numbers in \(\R\) (and hence in \(\R^*\)), show that \(x_n\) converges to \(+\infty\) if and only if \(\liminf_{n \to \infty} x_n = +\infty\), and \(x_n\) converges to \(-\infty\) if and only if \(\limsup_{n \to \infty} x_n = -\infty\).
\end{exercise}

\begin{proof}
    We first show that \((X, \mathcal{F})\) is a topology space.
    By definition we know that \(X\) is open and \(\emptyset\) is open trivially, thus \(X, \emptyset \in \mathcal{F}\).
    Let \(n \in \N\) and let \(S_n \subseteq \mathcal{F}\) such that \(\#(S_n) = n\).
    We use induction on \(n\) to show that \(\bigcap S_n \in \mathcal{F}\) for every \(n \in \N\).
    For \(n = 0\), we have \(S_0 = \emptyset\) and \(\bigcap S_0 = \emptyset\).
    From the proof above we know that \(\emptyset \in \mathcal{F}\), thus the base case holds.
    Suppose inductively that \(\bigcap S_n \in \mathcal{F}\) for some \(n \geq 0\).
    Let \(S_{n + 1} \subseteq \mathcal{F}\) such that \(\#(S_{n + 1}) = n + 1\).
    Then we have \(S_{n + 1} = \{V_1, \dots, V_{n + 1} : \forall i \in \Z^+, V_i \in \mathcal{F}\}\) and \(\bigcap S_{n + 1} = \bigcap_{i = 1}^{n + 1} V_i\).
    If \(\bigcap S_{n + 1} = \emptyset\), then from the proof above we know that \(\emptyset \in \mathcal{F}\).
    So suppose that \(\bigcap S_{n + 1} \neq \emptyset\).
    Let \(x \in \bigcap S_{n + 1}\).
    Since \(x \in \bigcap_{i = 1}^n V_i\) and \(\#(\{V_1, \dots, V_n\}) = n\), by induction hypothesis we know that there exists a set \(I\) in one of the following forms
    \[
        I = \begin{cases}
            \{y \in X : a < y < b\} \text{ for some } a, b \in X \\
            \{y \in X : a < y\} \text{ for some } a \in X        \\
            \{y \in X : y < b\} \text{ for some } b \in X        \\
            X
        \end{cases}
    \]
    such that \(x \in I\) and \(I \subseteq \bigcap_{i = 1}^n V_i\).
    Since \(x \in V_{n + 1}\) and \(V_{n + 1} \in \mathcal{F}\), we know that there exists a set \(I'\) in one of the following forms
    \[
        I' = \begin{cases}
            \{y \in X : a' < y < b'\} \text{ for some } a', b' \in X \\
            \{y \in X : a' < y\} \text{ for some } a' \in X          \\
            \{y \in X : y < b'\} \text{ for some } b' \in X          \\
            X
        \end{cases}
    \]
    such that \(x \in I'\) and \(I' \subseteq V_{n + 1}\).
    Then we have \(x \in I \cap I'\) and \(I \cap I' \subseteq \bigcap_{i = 1}^{n + 1} V_i\).
    Since \((X, \leq)\) is totally ordered, we know that \(I \cap I'\) is in one of the following forms
    \[
        I \cap I' = \begin{cases}
            \{y \in X : \max_{(X, \leq)}(a, a') < y < \min_{(X, \leq)}(b, b')\} \\
            \{y \in X : \max_{(X, \leq)}(a, a') < y < b\}                       \\
            \{y \in X : a < y < \min_{(X, \leq)}(b, b')\}                       \\
            \{y \in X : a < y < b\}                                             \\
            \{y \in X : \max_{(X, \leq)}(a, a') < y < b'\}                      \\
            \{y \in X : \max_{(X, \leq)}(a, a') < y\}                           \\
            \{y \in X : a < y < b'\}                                            \\
            \{y \in X : a < y\}                                                 \\
            \{y \in X : a' < y < \min_{(X, \leq)}(b, b')\}                      \\
            \{y \in X : a' < y < b\}                                            \\
            \{y \in X : y < \min_{(X, \leq)}(b, b')\}                           \\
            \{y \in X : y < b\}                                                 \\
            \{y \in X : a' < y < b'\}                                           \\
            \{y \in X : a' < y\}                                                \\
            \{y \in X : y < b'\}                                                \\
            X
        \end{cases}
    \]
    Thus by definition \(I \cap I'\) is an interval.
    Since \(x\) is arbitrary, we know that \(\bigcap S_{n + 1}\) is open in \((X, \mathcal{F})\), and this closes the induction.
    We conclude that for any finite collection of open sets, their intersection is again open in \((X, \mathcal{F})\).

    Let \(S \subseteq \mathcal{F}\).
    If \(\bigcup S = \emptyset\), then from proof above we know that \(\emptyset \in \mathcal{F}\).
    So suppose that \(\bigcup S \neq \emptyset\).
    Let \(x \in \bigcup S\).
    We know that there exists an \(V \in S\) such that \(x \in V\).
    Since \(V \in S\), we know that \(V\) is open in \((X, \mathcal{F})\) and by definition there exists an interval \(I\) such that \(x \in I\) and \(I \subseteq V\).
    Then we have \(I \subseteq V \subseteq \bigcup S\).
    Since \(x\) is arbitrary, we know that \(\bigcup S\) is open in \((X, \mathcal{F})\).
    Combine all the results above we know that \((X, \mathcal{F})\) is a topological space by Definition \ref{2.5.1}.

    Next we show that \((X, \mathcal{F})\) is Hausdorff.
    Let \(x_1, x_2 \in X\) such that \(x_1 \neq x_2\).
    Since \((X, \leq)\) is totally ordered, we have either \(x_1 < x_2\) or \(x_2 < x_1\).
    Without the loss of generality suppose that \(x_1 < x_2\).
    Let \(I_1 = \{y \in X : y < x_2\}\) and let \(I_2 = \{y \in X : x_1 < y\}\).
    Then we have \(x_1 \in I_1\) and \(x_2 \in I_2\).
    By definition we know that \(I_1, I_2 \in \mathcal{F}\).
    If \(I_1 \cap I_2 = \emptyset\), then we are done.
    So suppose that \(I_1 \cap I_2 \neq \emptyset\).
    Let \(x \in I_1 \cap I_2\), let \(J_1 = \{y \in X : y < x\}\) and let \(J_2 = \{y \in X : x < y\}\).
    Since \(I_1 \cap I_2 = \{y \in X : x_1 < y < x_2\}\), we know that \(x \neq x_1\) and \(x \neq x_2\).
    Since \(x_1 < x\), we have \(x_1 \in J_1\).
    Similarly we have \(x_2 \in J_2\).
    By definition we know that \(J_1, J_2 \in \mathcal{F}\).
    Since \((X, \leq)\) is totally ordered, we know that \(J_1 \cap J_2 = \emptyset\).
    Since \(x_1, x_2\) are arbitrary, by Exercise \ref{ex 2.5.4} we know that \((X, \mathcal{F})\) is Hausdorff.

    Next we show that the order topology in \(\R\) with order relation \(\leq\) matches standard topology.
    Let \(\mathcal{F}_o\) be the order topology in \(\R\) and let \(\mathcal{F}_s\) be the standard topology in \(\R\).
    We want to show that \(\mathcal{F}_o = \mathcal{F}_s\).

    Let \(V \in \mathcal{F}_o\) and let \(x \in V\).
    Then we have
    \[
        \exists\ I \subseteq \R : \begin{cases}
            I \text{ is an open interval in } \R \\
            x \in I                              \\
            I \subseteq V
        \end{cases}
    \]
    Now we split into four cases:
    \begin{itemize}
        \item If \(I = (a, b)\) for some \(a, b \in \R\), then we have
              \begin{align*}
                           & x \in (a, b)                                                                                              \\
                  \implies & r = \min(\abs{x - a}, \abs{x - b}) = \min(x - a, b - x) > 0                                               \\
                  \implies & (x - r, x + r) \subseteq (a, b) \subseteq V                                                               \\
                  \implies & B_{(\R, d_{l^1}|_{\R \times \R})}(x, r) \subseteq (a, b) \subseteq V & \text{(by Definition \ref{1.2.1})} \\
                  \implies & x \in \text{int}_{(\R, d_{l^1}|_{\R \times \R})}(V).                 & \text{(by Definition \ref{1.2.5})}
              \end{align*}
        \item If \(I = (a, \infty)\) for some \(a \in \R\), then we have
              \begin{align*}
                           & x \in (a, \infty)                                                                                              \\
                  \implies & r = \abs{x - a} = x - a > 0                                                                                    \\
                  \implies & (x - r, x + r) \subseteq (a, \infty) \subseteq V                                                               \\
                  \implies & B_{(\R, d_{l^1}|_{\R \times \R})}(x, r) \subseteq (a, \infty) \subseteq V & \text{(by Definition \ref{1.2.1})} \\
                  \implies & x \in \text{int}_{(\R, d_{l^1}|_{\R \times \R})}(V).                      & \text{(by Definition \ref{1.2.5})}
              \end{align*}
        \item If \(I = (-\infty, b)\) for some \(b \in \R\), then we have
              \begin{align*}
                           & x \in (-\infty, b)                                                                                              \\
                  \implies & r = \abs{x - b} = b - x > 0                                                                                     \\
                  \implies & (x - r, x + r) \subseteq (-\infty, b) \subseteq V                                                               \\
                  \implies & B_{(\R, d_{l^1}|_{\R \times \R})}(x, r) \subseteq (-\infty, b) \subseteq V & \text{(by Definition \ref{1.2.1})} \\
                  \implies & x \in \text{int}_{(\R, d_{l^1}|_{\R \times \R})}(V).                       & \text{(by Definition \ref{1.2.5})}
              \end{align*}
        \item If \(I = \R\), then we have \(V = \R\) and \(x \in \text{int}_{(\R, d_{l^1}|_{\R \times \R})}(V) = \R\).
    \end{itemize}
    From all cases above we conclude that \(x \in \text{int}_{(\R, d_{l^1}|_{\R \times \R})}(V)\).
    Since \(x\) is arbitrary, by Proposition \ref{1.2.15}(a) we know that \(V\) is open in \((X, d_{l^1}|_{\R \times \R})\) and thus \(V \in \mathcal{F}_s\).
    Since \(V\) is arbitrary, we have \(\mathcal{F}_o \subseteq \mathcal{F}_s\).

    Let \(W \in \mathcal{F}_s\).
    Since \(W\) is open in \((\R, d_{l^1}|_{\R \times \R})\), we have
    \begin{align*}
                 & \forall x \in W, \exists\ r \in \R^+ : B_{(\R, d_{l^1}|_{\R \times \R})}(x, r) \subseteq W & \text{(by Proposition \ref{1.2.15}(a))} \\
        \implies & \forall x \in W, \exists\ r \in \R^+ : (x - r, x + r) \subseteq W                          & \text{(by Definition \ref{1.2.1})}      \\
        \implies & W \in \mathcal{F}_o.                                                                       & \text{(by definition)}
    \end{align*}
    Since \(W\) is arbitrary, we have \(\mathcal{F}_s \subseteq \mathcal{F}_o\).
    From the proof above we thus have \(\mathcal{F}_o = \mathcal{F}_s\).

    Next we show that if \((\R^*, \mathcal{F})\) is an order topology with order relation \(\leq\), then \(\R\) is open in \((\R^*, \mathcal{F})\).
    Since \((\R^*, \leq)\) is totally ordered, we know that \((\R^*, \mathcal{F})\) is an order topology.
    Since
    \[
        \forall x \in \R, (x - 1, x + 1) \subseteq \R,
    \]
    by definition we know that \(\R\) is open in \((\R^*, \mathcal{F})\).

    Next we show that if \((\R^*, \mathcal{F})\) is an order topology with order relation \(\leq\), then the boundary points of \(\R\) in \((\R^*, \mathcal{F})\) are \(-\infty\) and \(\infty\).
    Since \(\infty \notin \R\) and \(\R\) is open in \((\R^*, \mathcal{F})\), by Definition \ref{2.5.5} we know that \(\infty\) is either an exterior point or a boundary point of \(\R\) in \((\R^*, \mathcal{F})\).
    Suppose for sake of contradiction that \(\infty\) is an exterior point of \(\R\) in \((\R^*, \mathcal{F})\).
    Then by Definition \ref{2.5.5} we have
    \[
        \exists\ V \in \mathcal{F} : (\infty \in V) \land (\R \cap V = \emptyset).
    \]
    Since \(V \cap \R = \emptyset\), we know that the only possible choices of \(V\) are \(\{\infty\}\) or \(\{-\infty, \infty\}\).
    But in either cases we cannot find an open interval \(I \subseteq \R^*\) such that \(I \subseteq V\), a contradiction.
    Thus \(\infty\) is a boundary point of \(\R\) in \((\R^*, \mathcal{F})\).
    Using similar arguments as above we can show that \(-\infty\) is a boundary point of \(\R\) in \((\R^*, \mathcal{F})\).

    Finally we show that if  \((\R^*, \mathcal{F})\) is an order topology with order relation \(\leq\) and \((x^{(n)})_{n = 1}^\infty\), \((y^{(n)})_{n = 1}^\infty\) are sequence in \(\R\), then we have
    \[
        \begin{cases}
            x_n \text{ converges to } \infty \text{ in } (\R^*, \mathcal{F})  \\
            y_n \text{ converges to } -\infty \text{ in } (\R^*, \mathcal{F}) \\
        \end{cases} \iff \begin{cases}
            \liminf_{n \to \infty} x_n = \infty \\
            \limsup_{n \to \infty} y_n = -\infty
        \end{cases}
    \]
    This is true since
    \begin{align*}
             & \begin{cases}
                   \liminf_{n \to \infty} x^{(n)} = \infty \\
                   \limsup_{n \to \infty} y^{(n)} = -\infty
               \end{cases}                                                                                       \\
        \iff & \begin{cases}
                   \sup\big\{\inf\{x^{(n)} : n \geq N\} : N \geq 1\big\} = \infty \\
                   \inf\big\{\sup\{y^{(n)} : n \geq N\} : N \geq 1\big\} = -\infty
               \end{cases}                                                                \\
        \iff & \begin{cases}
                   \sup\{x^{(n)} : n \geq 1\} = \infty \\
                   \inf\{y^{(n)} : n \geq 1\} = -\infty
               \end{cases}                                                                                           \\
        \iff & \begin{cases}
                   \forall \varepsilon \in \R^+, \exists\ N \geq 1 : \forall n \geq N, x^{(n)} > \varepsilon \\
                   \forall \varepsilon \in \R^+, \exists\ N \geq 1 : \forall n \geq N, y^{(n)} < -\varepsilon
               \end{cases}                                     \\
        \iff & \begin{cases}
                   \forall \varepsilon \in \R^+, \exists\ N \geq 1 : \forall n \geq N, x^{(n)} \in (\varepsilon, \infty] \\
                   \forall \varepsilon \in \R^+, \exists\ N \geq 1 : \forall n \geq N, y^{(n)} \in [-\infty, -\varepsilon)
               \end{cases} \\
        \iff & \begin{cases}
                   \forall V \in \mathcal{F}, \infty \in V \implies \exists\ \varepsilon \in \R^+ : \begin{cases}
                                                                                                 (\varepsilon, \infty] \subseteq V \\
                                                                                                 \exists\ N \geq 1 : \forall n \geq N, x^{(n)} \in V
                                                                                             \end{cases} \\
                   \forall V \in \mathcal{F}, -\infty \in V \implies \exists\ \varepsilon \in \R^+ : \begin{cases}
                                                                                                  [-\infty, -\varepsilon) \subseteq V \\
                                                                                                  \exists\ N \geq 1 : \forall n \geq N, y^{(n)} \in V
                                                                                              \end{cases}
               \end{cases}                   \\
        \iff & \begin{cases}
                   x_n \text{ converges to } \infty \text{ in } (\R^*, \mathcal{F})  \\
                   y_n \text{ converges to } -\infty \text{ in } (\R^*, \mathcal{F}) \\
               \end{cases}
    \end{align*}
\end{proof}

\begin{exercise}\label{ex 2.5.6}
    Let \(X\) be an uncountable set, and let \(\mathcal{F}\) be the collection of all subsets \(E\) in \(X\) which are either empty or co-finite (which means that \(X \setminus E\) is finite).
    Show that \((X, \mathcal{F})\) is a topology (this is called the \emph{co-finite topology} on \(X\)) which is not Hausdorff in the sense of Exercise \ref{ex 2.5.4}, and is compact and connected.
    Also, show that if \(x \in X\) and \((V_n)_{n = 1}^\infty\) is any countable collection of open sets containing \(x\), then \(\bigcap_{n = 1}^\infty V_n \neq \{x\}\).
    Use this to show that the co-finite topology cannot be obtained by placing a metric \(d\) on \(X\).
\end{exercise}

\begin{proof}
    We first show that \((X, \mathcal{F})\) is a topological space.
    By definition we have \(\emptyset \in \mathcal{F}\).
    Since \(X \setminus X = \emptyset\) is finite, we know that \(X\) is co-finite and \(X \in \mathcal{F}\).
    Let \(n \in \N\), let \(I_n = \{i \in \N : 1 \leq i \leq n\}\) and let \(A = \{V_i \in \mathcal{F} : i \in I_n\}\) be a finite collection of open sets in \((X, \mathcal{F})\).
    If \(\bigcap A = \emptyset\), then from the proof above we know that \(\emptyset \in \mathcal{F}\).
    So suppose that \(A \neq \emptyset\).
    Then we have
    \begin{align*}
                 & \forall i \in I_n, V_i \text{ is co-finite}                     \\
        \implies & \forall i \in I_n, X \setminus V_i \text{ is finite}            \\
        \implies & \bigcup_{i = 1}^n (X \setminus V_i) \text{ is finite}           \\
        \implies & X \setminus \bigg(\bigcap_{i = 1}^n V_i\bigg) \text{ is finite} \\
        \implies & X \setminus \bigg(\bigcap A\bigg) \text{ is finite}             \\
        \implies & \bigcap A \text{ is co-finite}
    \end{align*}
    and thus \(\bigcap A \in \mathcal{F}\).
    Since \(n\) is arbitrary, we conclude that the intersection of any finite collection of open sets in \((X, \mathcal{F})\) is open in \((X, \mathcal{F})\).

    Let \(S \subseteq \mathcal{F}\).
    Then we have
    \begin{align*}
                 & \forall V \in S, V \text{ is co-finite}             \\
        \implies & \forall V \in S, X \setminus V \text{ is finite}    \\
        \implies & \bigcap_{V \in S} (X \setminus V) \text{ is finite} \\
        \implies & X \setminus \bigg(\bigcup S\bigg) \text{ is finite} \\
        \implies & \bigcup S \text{ is co-finite}
    \end{align*}
    and thus \(\bigcup S \in \mathcal{F}\).
    Since \(S\) is arbitrary, we conclude that the union of arbitrary open sets in \((X, \mathcal{F})\) is open in \((X, \mathcal{F})\).
    Combine all the proofs above we conclude that \((X, \mathcal{F})\) is a topological space by Definition \ref{2.5.1}.

    Next we show that \((X, \mathcal{F})\) is not Hausdorff.
    Suppose for sake of contradiction that \((X, \mathcal{F})\) is Hausdorff.
    Let \(x_1, x_2 \in X\) such that \(x_1 \neq x_2\).
    By Exercise \ref{ex 2.5.4} we know that
    \[
        \exists\ V_1, V_2 \in \mathcal{F} : \begin{cases}
            x_1 \in V_1 \\
            x_2 \in V_2 \\
            V_1 \cap V_2 = \emptyset
        \end{cases}
    \]
    But then we have
    \begin{align*}
                 & V_1, V_2 \text{ are co-finite}                                                          \\
        \implies & X \setminus V_1, X \setminus V_2 \text{ are finite}                                     \\
        \implies & (X \setminus V_1) \cup (X \setminus V_2) \text{ is finite}                              \\
        \implies & X \setminus (V_1 \cap V_2) \text{ is finite}                                            \\
        \implies & X \text{ is finite},                                       & (V_1 \cap V_2 = \emptyset)
    \end{align*}
    a contradiction.
    Thus \((X, \mathcal{F})\) is not Hausdorff.

    Next we show that \((X, \mathcal{F})\) is compact.
    Let \(S\) be an open cover of \(X\) in \((X, \mathcal{F})\).
    Let \(V_0 \in S\).
    Since \(V_0\) is co-finite, we know that \(X \setminus V_0\) is finite.
    Let \(n = \#(X \setminus V_0)\), let \(I_n = \{i \in \N : 1 \leq i \leq n\}\) and let \(X \setminus V_0 = \{x_i \in X : i \in I_n\}\).
    Then we have
    \begin{align*}
                 & \forall i \in I_n, x_i \in X                        \\
        \implies & \forall i \in I_n, \exists\ V_i \in S : x_i \in V_i \\
        \implies & X = V_0 \cup \bigg(\bigcup_{i \in I_n} V_i\bigg).
    \end{align*}
    Since \(S\) is arbitrary, by Definition \ref{2.5.9} we know that \((X, \mathcal{F})\) is compact.

    Next we show that \((X, \mathcal{F})\) is connected.
    Suppose for sake of contradiction that \((X, \mathcal{F})\) is disconnected.
    Then by Definition \ref{2.4.1} we have
    \[
        \exists\ V, W \in \mathcal{F} : \begin{cases}
            V \neq \emptyset \neq W \\
            V \cup W = X            \\
            V \cap W = \emptyset
        \end{cases}
    \]
    But then we have
    \begin{align*}
                 & V, W \text{ are co-finite}                                                      \\
        \implies & X \setminus V, X \setminus W \text{ are finite}                                 \\
        \implies & (X \setminus V) \cup (X \setminus W) \text{ is finite}                          \\
        \implies & X \setminus (V \cap W) \text{ is finite}                                        \\
        \implies & X \text{ is finite},                                   & (V \cap W = \emptyset)
    \end{align*}
    a contradiction.
    Thus \((X, \mathcal{F})\) is connected.

    Next we show that if \(x \in X\) and \((V_n)_{n = 1}^\infty\) is any countable collection of open sets in \((X, \mathcal{F})\) such that \(x \in V_n\) for all \(n \in \Z^+\), then \(\bigcap_{n = 1}^\infty V_n \neq \{x\}\).
    This is true since
    \begin{align*}
                 & \forall n \in \Z^+, V_n \text{ is co-finite}                                                                                                 \\
        \implies & \forall n \in \Z^+, X \setminus V_n \text{ is finite}                                                                                        \\
        \implies & \bigcup_{n = 1}^\infty (X \setminus V_n) \text{ is at most countable}                             & \text{(by Exercise 8.1.9 in Analysis I)} \\
        \implies & X \setminus \bigg(\bigcap_{n = 1}^\infty V_n\bigg) \text{ is at most countable}                                                              \\
        \implies & X \setminus \Bigg(X \setminus \bigg(\bigcap_{n = 1}^\infty V_n\bigg)\Bigg) \text{ is uncountable} & \text{(\(X\) is uncountable)}            \\
        \implies & \bigcap_{n = 1}^\infty V_n \text{ is uncountable}                                                                                            \\
        \implies & \bigcap_{n = 1}^\infty V_n \neq \{x\}.
    \end{align*}

    Finally we show that \((X, \mathcal{F})\) cannot be obtained by placing a metric \(d\) on \(X\).
    Suppose for sake of contradiction that there exists some metric \(d\) such that \(\mathcal{F} = \{V \subseteq X : V \text{ is open in } (X, d)\}\).
    Let \(x \in X\) and for each \(n \in \Z^+\) let \(V_n = B_{(X, d)}(x, \frac{1}{n})\).
    By Proposition \ref{1.2.15}(c) we know that \(V_n\) is open in \((X, d)\) for each \(n \in \Z^+\).
    From the proof above we must have \(\bigcap_{n = 1}^\infty V_n \neq \{x\}\).
    So let \(y \in \bigcap_{n = 1}^\infty V_n\) such that \(y \neq x\).
    By Definition \ref{1.1.2}(b) we know that \(d(y, x) \in \R^+\).
    But then we have
    \begin{align*}
                 & \exists\ n \in \Z^+ : d(y, x) > \frac{1}{n}               & \text{(by Archimedean property)}   \\
        \implies & \exists\ n \in \Z^+ : y \notin B_{(X, d)}(x, \frac{1}{n}) & \text{(by Definition \ref{1.2.1})} \\
        \implies & \exists\ n \in \Z^+ : y \notin V_n                                                             \\
        \implies & y \notin \bigcap_{n = 1}^\infty V_n,
    \end{align*}
    a contradiction.
    Thus \((X, \mathcal{F})\) cannot be obtained by placing a metric \(d\) on \(X\).
\end{proof}

\begin{exercise}\label{ex 2.5.7}
    Let \(X\) be an uncountable set, and let \(\mathcal{F}\) be the collection of all subsets \(E\) in \(X\) which are either empty or co-countable
    (which means that \(X \setminus E\) is at most countable).
    Show that \((X, \mathcal{F})\) is a topology (this is called the \emph{co-countable topology} on \(X\)) which is not Hausdorff in the sense of Exercise \ref{ex 2.5.4}, and connected, but cannot arise from a metric space and is not compact.
\end{exercise}

\begin{proof}
    We first show that \((X, \mathcal{F})\) is a topological space.
    By definition we have \(\emptyset \in \mathcal{F}\).
    Since \(X \setminus X = \emptyset\) is finite, we know that \(X\) is co-countable and \(X \in \mathcal{F}\).
    Let \(n \in \N\), let \(I_n = \{i \in \N : 1 \leq i \leq n\}\) and let \(A = \{V_i \in \mathcal{F} : i \in I_n\}\) be a finite collection of open sets in \((X, \mathcal{F})\).
    If \(\bigcap A = \emptyset\), then from the proof above we know that \(\emptyset \in \mathcal{F}\).
    So suppose that \(A \neq \emptyset\).
    Then we have
    \begin{align*}
                 & \forall i \in I_n, V_i \text{ is co-countable}                                                                        \\
        \implies & \forall i \in I_n, X \setminus V_i \text{ is at most countable}                                                       \\
        \implies & \bigcup_{i = 1}^n (X \setminus V_i) \text{ is at most countable}           & \text{(by Exercise 8.1.9 in Analysis I)} \\
        \implies & X \setminus \bigg(\bigcap_{i = 1}^n V_i\bigg) \text{ is at most countable}                                            \\
        \implies & X \setminus \bigg(\bigcap A\bigg) \text{ is at most countable}                                                        \\
        \implies & \bigcap A \text{ is co-countable}
    \end{align*}
    and thus \(\bigcap A \in \mathcal{F}\).
    Since \(n\) is arbitrary, we conclude that the intersection of any finite collection of open sets in \((X, \mathcal{F})\) is open in \((X, \mathcal{F})\).

    Let \(S \subseteq \mathcal{F}\).
    Then we have
    \begin{align*}
                 & \forall V \in S, V \text{ is co-countable}                     \\
        \implies & \forall V \in S, X \setminus V \text{ is at most countable}    \\
        \implies & \bigcap_{V \in S} (X \setminus V) \text{ is at most countable} \\
        \implies & X \setminus \bigg(\bigcup S\bigg) \text{ is at most countable} \\
        \implies & \bigcup S \text{ is co-countable}
    \end{align*}
    and thus \(\bigcup S \in \mathcal{F}\).
    Since \(S\) is arbitrary, we conclude that the union of arbitrary open sets in \((X, \mathcal{F})\) is open in \((X, \mathcal{F})\).
    Combine all the proofs above we conclude that \((X, \mathcal{F})\) is a topological space by Definition \ref{2.5.1}.

    Next we show that \((X, \mathcal{F})\) is not Hausdorff.
    Suppose for sake of contradiction that \((X, \mathcal{F})\) is Hausdorff.
    Let \(x_1, x_2 \in X\) such that \(x_1 \neq x_2\).
    By Exercise \ref{ex 2.5.4} we know that
    \[
        \exists\ V_1, V_2 \in \mathcal{F} : \begin{cases}
            x_1 \in V_1 \\
            x_2 \in V_2 \\
            V_1 \cap V_2 = \emptyset
        \end{cases}
    \]
    But then we have
    \begin{align*}
                 & V_1, V_2 \text{ are co-countable}                                                                  \\
        \implies & X \setminus V_1, X \setminus V_2 \text{ are at most countable}                                     \\
        \implies & (X \setminus V_1) \cup (X \setminus V_2) \text{ is at most countable}                              \\
        \implies & X \setminus (V_1 \cap V_2) \text{ is at most countable}                                            \\
        \implies & X \text{ is at most countable},                                       & (V_1 \cap V_2 = \emptyset)
    \end{align*}
    a contradiction.
    Thus \((X, \mathcal{F})\) is not Hausdorff.

    Next we show that \((X, \mathcal{F})\) is connected.
    Suppose for sake of contradiction that \((X, \mathcal{F})\) is disconnected.
    Then by Definition \ref{2.4.1} we have
    \[
        \exists\ V, W \in \mathcal{F} : \begin{cases}
            V \neq \emptyset \neq W \\
            V \cup W = X            \\
            V \cap W = \emptyset
        \end{cases}
    \]
    But then we have
    \begin{align*}
                 & V, W \text{ are co-countable}                                                              \\
        \implies & X \setminus V, X \setminus W \text{ are at most countable}                                 \\
        \implies & (X \setminus V) \cup (X \setminus W) \text{ is at most countable}                          \\
        \implies & X \setminus (V \cap W) \text{ is at most countable}                                        \\
        \implies & X \text{ is at most countable},                                   & (V \cap W = \emptyset)
    \end{align*}
    a contradiction.
    Thus \((X, \mathcal{F})\) is connected.

    Next we show that if \(x \in X\) and \((V_n)_{n = 1}^\infty\) is any countable collection of open sets in \((X, \mathcal{F})\) such that \(x \in V_n\) for all \(n \in \Z^+\), then \(\bigcap_{n = 1}^\infty V_n \neq \{x\}\).
    This is true since
    \begin{align*}
                 & \forall n \in \Z^+, V_n \text{ is co-countable}                                                                                              \\
        \implies & \forall n \in \Z^+, X \setminus V_n \text{ is at most countable}                                                                             \\
        \implies & \bigcup_{n = 1}^\infty (X \setminus V_n) \text{ is at most countable}                             & \text{(by Exercise 8.1.9 in Analysis I)} \\
        \implies & X \setminus \bigg(\bigcap_{n = 1}^\infty V_n\bigg) \text{ is at most countable}                                                              \\
        \implies & X \setminus \Bigg(X \setminus \bigg(\bigcap_{n = 1}^\infty V_n\bigg)\Bigg) \text{ is uncountable} & \text{(\(X\) is uncountable)}            \\
        \implies & \bigcap_{n = 1}^\infty V_n \text{ is uncountable}                                                                                            \\
        \implies & \bigcap_{n = 1}^\infty V_n \neq \{x\}.
    \end{align*}

    Next we show that \((X, \mathcal{F})\) cannot be obtained by placing a metric \(d\) on \(X\).
    Suppose for sake of contradiction that there exists some metric \(d\) such that \(\mathcal{F} = \{V \subseteq X : V \text{ is open in } (X, d)\}\).
    Let \(x \in X\) and for each \(n \in \Z^+\) let \(V_n = B_{(X, d)}(x, \frac{1}{n})\).
    By Proposition \ref{1.2.15}(c) we know that \(V_n\) is open in \((X, d)\) for each \(n \in \Z^+\).
    From the proof above we must have \(\bigcap_{n = 1}^\infty V_n \neq \{x\}\).
    So let \(y \in \bigcap_{n = 1}^\infty V_n\) such that \(y \neq x\).
    By Definition \ref{1.1.2}(b) we know that \(d(y, x) \in \R^+\).
    But then we have
    \begin{align*}
                 & \exists\ n \in \Z^+ : d(y, x) > \frac{1}{n}               & \text{(by Archimedean property)}   \\
        \implies & \exists\ n \in \Z^+ : y \notin B_{(X, d)}(x, \frac{1}{n}) & \text{(by Definition \ref{1.2.1})} \\
        \implies & \exists\ n \in \Z^+ : y \notin V_n                                                             \\
        \implies & y \notin \bigcap_{n = 1}^\infty V_n,
    \end{align*}
    a contradiction.
    Thus \((X, \mathcal{F})\) cannot be obtained by placing a metric \(d\) on \(X\).

    Finally we show that \((X, \mathcal{F})\) is not compact.
    Let \((x^{(n)})_{n = 1}^\infty\) be a countable collection of elements in \(X\).
    For each \(n \in \Z^+\), we define \(E_n = (X \setminus \{x^{(i)} : i \in \Z^+\}) \cup \{x^{(n)}\}\).
    Since
    \begin{align*}
                 & \forall n \in \Z^+, X \setminus E_n = \{x^{(i)} : i \in \N\} \setminus \{x^{(n)}\} \text{ is countable} \\
        \implies & \forall n \in \Z^+, E_n \in \mathcal{F}
    \end{align*}
    and
    \begin{align*}
        \bigcup_{n = 1}^\infty E_n & = \bigcup_{n = 1}^\infty \Big(\big(X \setminus \{x^{(i)} : i \in \Z^+\}\big) \cup \{x^{(n)}\}\Big)                                        \\
                                   & = \bigg(\bigcup_{n = 1}^\infty \big(X \setminus \{x^{(i)} : i \in \Z^+\}\big)\bigg) \cup \bigg(\bigcup_{n = 1}^\infty \{x^{(n)}\}\bigg)   \\
                                   & = \Bigg(X \setminus \bigg(\bigcap_{n = 1}^\infty \{x^{(i)} : i \in \Z^+\}\bigg)\Bigg) \cup \bigg(\bigcup_{n = 1}^\infty \{x^{(n)}\}\bigg) \\
                                   & = \big(X \setminus \{x^{(i)} : i \in \Z^+\}\big) \cup \bigg(\bigcup_{n = 1}^\infty \{x^{(n)}\}\bigg)                                      \\
                                   & = X,
    \end{align*}
    we know that \(\bigcup_{n = 1}^\infty E_n\) is an open cover of \(X\) in \((X, \mathcal{F})\).
    Let \((E_{n_i})_{i = 1}^k\) be a finite subset of \((E_n)_{n = 1}^\infty\).
    Then we have
    \begin{align*}
                 & \forall 1 \leq i \leq k, x_{n_i} \in \bigcup_{j = 1}^k E_{n_j}                             \\
        \implies & \forall 1 \leq i \leq k, x_{n_i} \notin X \setminus \bigg(\bigcup_{i = 1}^k E_{n_j}\bigg).
    \end{align*}
    Since \((E_{n_i})_{i = 1}^k\) is arbitrary, we know that every finite subset of \((E_n)_{n = 1}^\infty\) cannot cover \(X\) in \((X, \mathcal{F})\).
    Thus by Definition \ref{2.5.9} \((X, \mathcal{F})\) is not compact.
\end{proof}

\setcounter{exercise}{8}
\begin{exercise}\label{ex 2.5.9}
    Let \((X, \mathcal{F})\) be a compact topological space.
    Assume that this space is \emph{first countable}, which means that for every \(x \in X\) there exists a countable collection \(V_1 , V_2 , \dots\) of neighbourhoods of \(x\), such that every neighbourhood of \(x\) contains one of the \(V_n\).
    Show that every sequence in \(X\) has a convergent subsequence, by modifying Exercise \ref{ex 1.5.11}.
\end{exercise}

\begin{proof}
    If \(X = \emptyset\), then the statement is trivial.
    So suppose that \(X \neq \emptyset\).
    Let \((x^{(n)})_{n = 1}^\infty\) be a sequence in \(X\).
    If the set \(E = \{x^{(n)} : n \in \Z^+\}\) is finite, then there exists a subsequence \((x^{(n_j)})_{j = 1}^\infty\) such that \(x^{(n_j)} = x^{(n_1)}\) for every \(j \in \Z^+\).
    By Definition \ref{2.5.4} we know that \((x^{(n_j)})_{j = 1}^\infty\) converges to \(x\).

    Now suppose that \(E\) is infinite.
    We claim that
    \[
        \exists\ y \in X : \forall W \in \mathcal{F}, y \in W \implies W \cap E \text{ is infinite}.
    \]
    Suppose for sake of contradiction that the claim is false.
    Then we have
    \[
        \forall y \in X, \exists\ W \in \mathcal{F} : \begin{cases}
            y \in W; \\
            W \cap E \text{ is finite}.
        \end{cases}
    \]
    We choose one such \(W\) for each \(y \in X\) and denote it as \(W_y\).
    But then we have
    \begin{align*}
                 & X = \bigcup_{y \in X} W_y                                                               \\
        \implies & \exists\ Y \subseteq X : \begin{cases}
                                                Y \text{ is finite} \\
                                                X = \bigcup_{y \in Y} W_y
                                            \end{cases}               & \text{(by Definition \ref{2.5.9})} \\
        \implies & \exists\ Y \subseteq X : \begin{cases}
                                                Y \text{ is finite} \\
                                                E = \bigcup_{y \in Y} (W_y \cap E) \text{ is finite}
                                            \end{cases}
    \end{align*}
    which contradict to the fact that \(E\) is infinite.
    Thus the claim is true and we can choose one \(y \in X\) such that every neighbourhood of \(y\) contains infinitely many elements of \((x^{(n)})_{n = 1}^\infty\).
    Since \((X, \mathcal{F})\) is first countable, we know that there exists a countable collection \((V_j)_{j = 1}^\infty\) of neighbourhoods of \(y\) such that
    \[
        \forall W \in \mathcal{F}, y \in W \implies \exists\ j \in \Z^+ : V_j \subseteq W.
    \]
    We fix such \((V_j)_{j = 1}^\infty\) and define \(U_j = \bigcap_{i = 1}^j V_j\) for each \(j \in \Z^+\).
    Since \(y \in V_j\) for every \(j \in \Z^+\), we know that \(y \in \bigcap_{j = 1}^\infty V_j\).
    Thus we have \(y \in U_j\) and \(U_j \neq \emptyset\) for every \(j \in \Z^+\).
    Observe that
    \[
        \forall p_1, p_2 \in \Z^+, p_1 < p_2 \implies \bigcap_{i = 1}^{p_2} V_i \subseteq \bigcap_{i = 1}^{p_1} V_i \implies U_{p_2} \subseteq U_{p_1}.
    \]
    Now we construct a subsequence \((x^{(n_j)})_{j = 1}^\infty\) which converges to \(y\).
    Let
    \[
        A_1 = \{n \in \Z^+ : x^{(n)} \in U_1\}.
    \]
    Since \(U_1 = V_1\) is a neighbourhood of \(y\), by the definition of \(y\) we know that \(A_1\) is infinite.
    Since \(A_1 \subseteq \Z^+\), by well-ordering principle we know that \(\min(A_1)\) is well-defined.
    Let \(n_1 = \min(A_1)\).
    Suppose that \(n_j\) is already defined for some \(j \geq 1\).
    Then we define \(n_{j + 1}\) as follow:
    \begin{align*}
         & A_{j + 1} = \{n \in \Z^+ : (n > n_j) \land (x^{(n)} \in U_{j + 1})\} \\
         & n_{j + 1} = \min(A_{j + 1})
    \end{align*}
    Since \(U_{j + 1} = \bigcap_{i = 1}^{j + 1} V_i\), by Definition \ref{2.5.1} we know that \(U_{j + 1}\) is a neighbourhood of \(y\).
    Thus by the definition of \(y\) we know that \(A_{j + 1}\) is infinite.
    Since \(A_{j + 1} \subseteq \Z^+\), by well-ordering principle we know that \(n_{j + 1}\) is well-defined.
    Thus we have construct a subsequence \((x^{(n_j)})_{j = 1}^\infty\).
    Let \(W\) be a neighbourhood of \(y\).
    Since \((X, \mathcal{F})\) is first countable, we know that
    \begin{align*}
                 & \exists\ N \in \Z^+ : V_N \subseteq W                                                                               \\
        \implies & \exists\ N \in \Z^+ : U_N \subseteq V_N \subseteq W                   & \text{(by the definition of \(U_N\))}       \\
        \implies & \exists\ N \in \Z^+ : \forall j \geq N, U_j \subseteq U_N \subseteq W & \text{(by the definition of \(U_N\))}       \\
        \implies & \exists\ N \in \Z^+ : \forall j \geq N, x^{(n_j)} \in W.              & \text{(by the definition of \(x^{(n_j)}\))}
    \end{align*}
    Since \(W\) is arbitrary, by Definition \ref{2.5.4} we know that \((x^{(n_j)})_{j = 1}^\infty\) converges to \(y\) in \((X, \mathcal{F})\).
    Thus we have found a subsequence of \((x^{(n)})_{n = 1}^\infty\) which converges in \((X, \mathcal{F})\).
    Since \((x^{(n)})_{n = 1}^\infty\) is arbitrary, we conclude that \((X, \mathcal{F})\) is sequentially compact, i.e., every sequence in \(X\) has a convergent subsequence.
\end{proof}

\begin{exercise}\label{ex 2.5.10}
    Prove the following partial analogue of Proposition \ref{1.2.10} for topological spaces:
    (c) implies both (a) and (b), which are equivalent to each other.
    Show that in the co-countable topology in Exercise \ref{ex 2.5.7}, it is possible for (a) and (b) to hold without (c) holding.
\end{exercise}

\begin{proof}
    Let \((X, \mathcal{F})\) be a topological space, let \(E \subseteq X\), let \(x_0 \in X\).
    We first show that if there exists a sequence \((x^{(n)})_{n = 1}^\infty\) in \(E\) which converges to \(x_0\) in \((X, \mathcal{F})\), then \(x_0\) is an adherent point of \(E\) in \((X, \mathcal{F})\).
    This is true since
    \begin{align*}
                 & (x^{(n)})_{n = 1}^\infty \text{ converges to } x_0 \text{ in } (X, \mathcal{F})                                                                    \\
        \implies & \big(\forall V \in \mathcal{F}, x_0 \in V \implies \exists\ N \in \Z^+ : \forall n \geq N, x^{(n)} \in V\big) & \text{(by Definition \ref{2.5.4})} \\
        \implies & \big(\forall V \in \mathcal{F}, x_0 \in V \implies \exists\ N \in \Z^+ : x^{(N)} \in V \cap E\big)                                                 \\
        \implies & \big(\forall V \in \mathcal{F}, x_0 \in V \implies V \cap E \neq \emptyset\big)                                                                    \\
        \implies & x_0 \in \overline{E}_{(X, \mathcal{F})}.                                                                      & \text{(by Definition \ref{2.5.6})}
    \end{align*}

    Next we show that \(x_0\) is an adherent point of \(E\) in \((X, \mathcal{F})\) iff \(x_0\) is an interior point or a boundary point of \(E\) in \((X, \mathcal{F})\).
    This is true since
    \begin{align*}
             & x_0 \in \overline{E}_{(X, \mathcal{F})}                                                                                                        \\
        \iff & \forall V \in \mathcal{F}, x_0 \in V \implies V \cap E \neq \emptyset                                     & \text{(by Definition \ref{2.5.6})} \\
        \iff & x_0 \notin \text{ext}_{(X, \mathcal{F})}(E)                                                               & \text{(by Definition \ref{2.5.5})} \\
        \iff & \big(x_0 \in \text{int}_{(X, \mathcal{F})}(E)\big) \lor \big(x_0 \in \partial_{(X, \mathcal{F})}(E)\big). & \text{(by Definition \ref{2.5.5})}
    \end{align*}

    Finally we show that if \(X\) is uncountable, \((X, \mathcal{F})\) is a co-countable topology and \(x_0 \in E\) for some \(E \in \mathcal{F}\), then there may not exist a sequence in \(E\) which coverges to \(x_0\) in \((X, \mathcal{F})\).
    Let \(C \subseteq X\) such that \(C\) is countable and let \(E = X \setminus C\).
    Then by Exercise \ref{ex 2.5.7} we know that \(E \in \mathcal{F}\) and \(E \neq \emptyset\).

    Let \(x_0 \in C\).
    Since \(x_0 \in C\), we know that \(x_0 \notin E\) and by Definition \ref{2.5.5} we know that \(x_0 \notin \text{int}_{(X, \mathcal{F})}(E)\).
    This means \(x_0 \in \text{ext}_{(X, \mathcal{F})}(E)\) or \(x_0 \in \partial_{(X, \mathcal{F})}(E)\).
    Now we claim that \(x_0 \in \partial_{(X, \mathcal{F})}(E)\).
    Suppose for sake of contradiction that the claim is false.
    Then we have \(x_0 \in \text{ext}_{(X, \mathcal{F})}(E)\), i.e.,
    \[
        \exists\ V \in \mathcal{F} : x_0 \in V \implies V \cap E = \emptyset.
    \]
    Fix this \(V\).
    Since \((X, \mathcal{F})\) is a co-countable, by Exercise \ref{ex 2.5.7} we know that \(X \setminus V\) is at most countable.
    But then we have
    \begin{align*}
                 & (X \setminus V) \cup C \text{ is countable}                                                   \\
        \implies & X \setminus \big((X \setminus V) \cup C\big) \text{ is co-countable}                          \\
        \implies & V \cap (X \setminus C) \text{ is co-countable}                                                \\
        \implies & V \cap E \text{ is co-countable}                                                              \\
        \implies & X \setminus (V \cap E) \text{ is at most countable}                                           \\
        \implies & X \text{ is at most countable},                                      & (V \cap E = \emptyset)
    \end{align*}
    which contradict to the hypothesis that \(X\) is uncountable.
    Thus the claim is true.
    From the proof above we know that \(x_0 \in \partial_{(X, \mathcal{F})}(E)\) implies \(x_0 \in \overline{E}_{(X, \mathcal{F})}\).

    Suppose for sake of contradiction that there exists a sequence \((x^{(n)})_{n = 1}^\infty\) in \(E\) which converges to \(x_0\) in \((X, \mathcal{F})\).
    Then we have
    \begin{align*}
                 & \{x^{(n)} : n \in \Z^+\} \text{ is countable}                \\
        \implies & X \setminus \{x^{(n)} : n \in \Z^+\} \text{ is co-countable} \\
        \implies & X \setminus \{x^{(n)} : n \in \Z^+\} \in \mathcal{F}.
    \end{align*}
    Since \(x_0 \notin E\), we have
    \begin{align*}
                 & x_0 \notin \{x^{(n)} : n \in \Z^+\}           \\
        \implies & x_0 \in X \setminus \{x^{(n)} : n \in \Z^+\}.
    \end{align*}
    But \(X \setminus \{x^{(n)} : n \in \Z^+\} \in \mathcal{F}\) means we have found one neighbourhood of \(x_0\) which does not contain any elements of \((x^{(n)})_{n = 1}^\infty\).
    By Definition \ref{2.5.4} this means \((x^{(n)})_{n = 1}^\infty\) does not coverges to \(x_0\) in \((X, \mathcal{F})\), a contradiction.
    Thus we conclude that there does not exist a sequence in \(E\) which converges to \(x_0\) when \(x_0\) is an adherent point of \(E\) in \((X, \mathcal{F})\).
\end{proof}

\begin{exercise}\label{ex 2.5.11}
    Let \(E\) be a subset of a topological space \((X, \mathcal{F})\).
    Show that \(E\) is open if and only if every element of \(E\) is an interior point, and show that \(E\) is closed if and only if \(E\) contains all of its adherent points.
    Prove analogues of Proposition \ref{1.2.15}(e)-(h) (some of these are automatic by definition).
    If we assume in addition that \(X\) is Hausdorff, prove an analogue of Proposition \ref{1.2.15}(d) also, but give an example to show that (d) can fail when \(X\) is not Hausdorff.
\end{exercise}

\begin{proof}
    We first show that \(E\) is open in \((X, \mathcal{F})\) iff \(E = \text{int}_{(X, \mathcal{F})}(E)\).
    \begin{align*}
             & E \text{ is open in } (X, \mathcal{F})                                                             \\
        \iff & E \in \mathcal{F}                                             & \text{(by Definition \ref{2.5.1})} \\
        \iff & \forall x \in E, \exists\ V_x \in \mathcal{F} : \begin{cases}
                                                                   x \in V_x \\
                                                                   V_x \subseteq E
                                                               \end{cases} & (\bigcup_{x \in E} V_x = E)          \\
        \iff & \forall x \in E, x \in \text{int}_{(X, \mathcal{F})}          & \text{(by Definition \ref{2.5.5})} \\
        \iff & E = \text{int}_{(X, \mathcal{F})}.                            & \text{(by Definition \ref{2.5.5})}
    \end{align*}

    Next we show that \(E\) is closed in \((X, \mathcal{F})\) iff \(E = \overline{E}_{(X, \mathcal{F})}\).
    \begin{align*}
             & E \text{ is closed in } (X, \mathcal{F})                                                                                            \\
        \iff & X \setminus E \text{ is open in } (X, \mathcal{F})                                                                                  \\
        \iff & X \setminus E = \text{int}_{(X, \mathcal{F})}(X \setminus E)                                 & \text{(from the proof above)}        \\
        \iff & \forall x \in X \setminus E, \exists\ V \in \mathcal{F} : \begin{cases}
                                                                             x \in V \\
                                                                             V \subseteq X \setminus E
                                                                         \end{cases}                      & \text{(by Definition \ref{2.5.5})}     \\
        \iff & \forall x \in X \setminus E, \exists\ V \in \mathcal{F} : \begin{cases}
                                                                             x \in V \\
                                                                             V \cap E = \emptyset
                                                                         \end{cases}                                                       \\
        \iff & \forall x \in X \setminus E, x \in \text{ext}_{(X, \mathcal{F})}(E)                          & \text{(by Definition \ref{2.5.5})}   \\
        \iff & X \setminus E = \text{ext}_{(X, \mathcal{F})}(E)                                             & \text{(by Definition \ref{2.5.5})}   \\
        \iff & E = \big(\text{int}_{(X, \mathcal{F})}(E)\big) \cup \big(\partial_{(X, \mathcal{F})}(E)\big) & \text{(by Definition \ref{2.5.5})}   \\
        \iff & E = \overline{E}_{(X, \mathcal{F})}.                                                         & \text{(by Exercise \ref{ex 2.5.10})}
    \end{align*}

    Next we show that \(E\) is open in \((X, \mathcal{F})\) iff \(X \setminus E\) is closed in \((X, \mathcal{F})\).
    This is true by definition.

    Next we show that if \(\{E_1, \dots, E_n\}\) is a finite collection of open sets in \((X, \mathcal{F})\), then \(\bigcap_{i = 1}^n E_i\) is open in \((X, \mathcal{F})\).
    This is true by Definition \ref{2.5.1}.

    Next we show that if \(\{E_1, \dots, E_n\}\) is a finite collection of closed sets in \((X, \mathcal{F})\), then \(\bigcup_{i = 1}^n E_i\) is closed in \((X, \mathcal{F})\).
    \begin{align*}
                 & \forall 1 \leq i \leq n, E_i \text{ is closed in } (X, \mathcal{F})                                                     \\
        \implies & \forall 1 \leq i \leq n, X \setminus E_i \text{ is open in } (X, \mathcal{F})                                           \\
        \implies & \bigcap_{i = 1}^n (X \setminus E_i) \text{ is open in } (X, \mathcal{F})           & \text{(by Definition \ref{2.5.1})} \\
        \implies & X \setminus \bigg(\bigcup_{i = 1}^n E_i\bigg) \text{ is open in } (X, \mathcal{F})                                      \\
        \implies & \bigcup_{i = 1}^n E_i \text{ is closed in } (X, \mathcal{F}).
    \end{align*}

    Next we show that if \((E_\alpha)_{\alpha \in I}\) is a collection of open sets in \((X, \mathcal{F})\) where \(I\) is some index set, then \(\bigcup_{\alpha \in I} E_\alpha\) is open in \((X, \mathcal{F})\).
    This is true by Definition \ref{2.5.1}.

    Next we show that if \((E_\alpha)_{\alpha \in I}\) is a collection of closed sets in \((X, \mathcal{F})\) where \(I\) is some index set, then \(\bigcap_{\alpha \in I} E_\alpha\) is closed in \((X, \mathcal{F})\).
    \begin{align*}
                 & \forall \alpha \in I, E_\alpha \text{ is closed in } (X, \mathcal{F})                                                             \\
        \implies & \forall \alpha \in I, X \setminus E_\alpha \text{ is open in } (X, \mathcal{F})                                                   \\
        \implies & \bigcup_{\alpha \in I} (X \setminus E_\alpha) \text{ is open in } (X, \mathcal{F})           & \text{(by Definition \ref{2.5.1})} \\
        \implies & X \setminus \bigg(\bigcap_{\alpha \in I} E_\alpha\bigg) \text{ is open in } (X, \mathcal{F})                                      \\
        \implies & \bigcap_{\alpha \in I} E_\alpha \text{ is closed in } (X, \mathcal{F}).
    \end{align*}

    Next we show that if \(E \subseteq X\), then \(\text{int}_{(X, \mathcal{F})}(E)\) is open in \((X, \mathcal{F})\).
    Suppose for sake of contradiction that \(\text{int}_{(X, \mathcal{F})}(E) \notin \mathcal{F}\).
    Then by Definition \ref{2.5.5} we know that
    \[
        \exists\ x \in \text{int}_{(X, \mathcal{F})}(E) : \forall V \in \mathcal{F}, x \in V \implies V \not\subseteq \text{int}_{(X, \mathcal{F})}(E).
    \]
    Fix this \(x\).
    By Definition \ref{2.5.5} we know that \(\text{int}_{(X, \mathcal{F})}(E) \subseteq E\).
    But then we have
    \[
        \forall V \in \mathcal{F}, x \in V \implies V \not\subseteq \text{int}_{(X, \mathcal{F})}(E) \subseteq E,
    \]
    which means \(x \notin \text{int}_{(X, \mathcal{F})}(E)\) by Definition \ref{2.5.5}, a contradiction.
    Thus we must have
    \[
        \forall x \in \text{int}_{(X, \mathcal{F})}(E), \exists\ V \in \mathcal{F} : \begin{cases}
            x \in V \\
            V \subseteq \text{int}_{(X, \mathcal{F})}(E)
        \end{cases}
    \]
    and by Definition \ref{2.5.5} we have \(\text{int}_{(X, \mathcal{F})}\big(\text{int}_{(X, \mathcal{F})}(E)\big) = \text{int}_{(X, \mathcal{F})}(E)\).
    From the proof above we conclude that \(\text{int}_{(X, \mathcal{F})}(E)\) is open in \((X, \mathcal{F})\).

    Next we show that if \(E \subseteq X\), then \(\text{int}_{(X, \mathcal{F})}(E)\) is the largest open set in \((X, \mathcal{F})\) which is contained in \(E\).
    Let \(V \in \mathcal{F}\) such that \(V \subseteq E\).
    Then we have
    \begin{align*}
                 & \forall x \in V, x \in E                                                                     \\
        \implies & \forall x \in V, x \in \text{int}_{(X, \mathcal{F})}(E) & \text{(by Definition \ref{2.5.5})} \\
        \implies & V \subseteq \text{int}_{(X, \mathcal{F})}(E).
    \end{align*}
    Since \(V\) is arbitrary, we conclude that \(\text{int}_{(X, \mathcal{F})}(E)\) is the largest open set in \((X, \mathcal{F})\) which is contained in \(E\).

    Next we show that if \(E \subseteq X\), then \(\partial_{(X, \mathcal{F})}(E) = \partial_{(X, \mathcal{F})}(X \setminus E)\).
    We have
    \begin{align*}
             & x \in \partial_{(X, \mathcal{F})}(E)                                                           \\
        \iff & \forall V \in \mathcal{F}, x \in V \implies \begin{cases}
                                                               V \cap E \neq \emptyset \\
                                                               V \not\subseteq E
                                                           \end{cases} & \text{(by Definition \ref{2.5.5})}   \\
        \iff & \forall V \in \mathcal{F}, x \in V \implies \begin{cases}
                                                               V \cap (X \setminus E) \neq \emptyset \\
                                                               V \not\subseteq (X \setminus E)
                                                           \end{cases}              \\
        \iff & x \in \partial_{(X, \mathcal{F})}(X \setminus E).         & \text{(by Definition \ref{2.5.5})}
    \end{align*}
    Thus \(\partial_{(X, \mathcal{F})}(E) = \partial_{(X, \mathcal{F})}(X \setminus E)\).

    Next we show that if \(E \subseteq X\), then \(\text{int}_{(X, \mathcal{F})}(E) = \text{ext}_{(X, \mathcal{F})}(X \setminus E)\).
    We have
    \begin{align*}
             & x \in \text{int}_{(X, \mathcal{F})}(E)                                                   \\
        \iff & \exists\ V \in \mathcal{F} : \begin{cases}
                                                x \in V \\
                                                V \subseteq E
                                            \end{cases}          & \text{(by Definition \ref{2.5.5})}   \\
        \iff & \exists\ V \in \mathcal{F} : \begin{cases}
                                                x \in V \\
                                                V \cap (X \setminus E) = \emptyset
                                            \end{cases}                           \\
        \iff & x \in \text{ext}_{(X, \mathcal{F})}(X \setminus E). & \text{(by Definition \ref{2.5.5})}
    \end{align*}
    Thus \(\text{int}_{(X, \mathcal{F})}(E) = \text{ext}_{(X, \mathcal{F})}(X \setminus E)\).

    Next we show that if \(E \subseteq X\), then \(\overline{E}_{(X, \mathcal{F})}\) is closed in \((X, \mathcal{F})\).
    \begin{align*}
             & \text{int}_{(X, \mathcal{F})}(X \setminus E) \text{ is open in } (X, \mathcal{F})                                                                       & \text{(from the proof above)}        \\
        \iff & X \setminus \text{int}_{(X, \mathcal{F})}(X \setminus E) \text{ is closed in } (X, \mathcal{F})                                                                                                \\
        \iff & \big(\text{ext}_{(X, \mathcal{F})}(X \setminus E)\big) \cup \big(\partial_{(X, \mathcal{F})}(X \setminus E)\big) \text{ is closed in } (X, \mathcal{F}) & \text{(by Definition \ref{2.5.5})}   \\
        \iff & \big(\text{int}_{(X, \mathcal{F})}(E)\big) \cup \big(\partial_{(X, \mathcal{F})}(E)\big) \text{ is closed in } (X, \mathcal{F})                         & \text{(from the proof above)}        \\
        \iff & \overline{E}_{(X, \mathcal{F})}(E) \text{ is closed in } (X, \mathcal{F}).                                                                              & \text{(by Exercise \ref{ex 2.5.10})}
    \end{align*}

    Next we show that if \(E \subseteq X\), then \(\overline{E}_{(X, \mathcal{F})}\) is the smallest closed set in \((X, \mathcal{F})\) which contains \(E\).
    Let \(V \in \mathcal{F}\) such that \(E \subseteq X \setminus V\).
    Then we have
    \begin{align*}
                 & E \subseteq (X \setminus V)                                                                     \\
        \implies & E \cap V = \emptyset                                                                            \\
        \implies & \overline{E}_{(X, \mathcal{F})} \cap V = \emptyset         & \text{(by Definition \ref{2.5.6})} \\
        \implies & \overline{E}_{(X, \mathcal{F})} \subseteq (X \setminus V).
    \end{align*}
    Since \(X \setminus V\) is arbitrary, we know that \(\overline{E}_{(X, \mathcal{F})}\) is the smallest closed set in \((X, \mathcal{F})\) which contains \(E\).

    Next we show that if \((X, \mathcal{F})\) is a Hausdorff space, then \(\{x_0\}\) is closed in \((X, \mathcal{F})\) for any \(x_0 \in X\).
    Let \(x_0 \in X\).
    We have
    \begin{align*}
                 & \forall y \in X \setminus \{x_0\}, y \neq x_0                                                                                    \\
        \implies & \forall y \in X \setminus \{x_0\}, \exists\ V, W \in \mathcal{F} : \begin{cases}
                                                                                          V \neq \emptyset \neq W \\
                                                                                          x_0 \in V               \\
                                                                                          y \in W                 \\
                                                                                          V \cap W = \emptyset
                                                                                      \end{cases}            & \text{(by Exercise \ref{ex 2.5.4})}  \\
        \implies & \forall y \in X \setminus \{x_0\}, \exists\ W \in \mathcal{F} : \begin{cases}
                                                                                       y \in W \\
                                                                                       W \subseteq (X \setminus \{x_0\})
                                                                                   \end{cases}                                 \\
        \implies & \forall y \in X \setminus \{x_0\}, y \in \text{int}_{(X, \mathcal{F})}(X \setminus \{x_0\}) & \text{(by Definition \ref{2.5.5})} \\
        \implies & X \setminus \{x_0\} = \text{int}_{(X, \mathcal{F})}(X \setminus \{x_0\})                    & \text{(by Definition \ref{2.5.5})} \\
        \implies & X \setminus \{x_0\} \text{ is open in } (X, \mathcal{F})                                    & \text{(from the proof above)}      \\
        \implies & \{x_0\} \text{ is closed in } (X, \mathcal{F}).
    \end{align*}
    Since \(x_0\) is arbitrary, we conclude that \(\{x_0\}\) is closed in \((X, \mathcal{F})\) for any \(x_0 \in X\).

    Finally we give an counterexample of Proposition \ref{1.2.15}(d) when \((X, \mathcal{F})\) is not Hausdorff.
    Let \(X \neq \emptyset\) and let \((X, \mathcal{F})\) be a trivial topology.
    Then by Exercise \ref{ex 2.5.1} we know that \(\mathcal{F} = \{\emptyset, X\}\) and by Exercise \ref{ex 2.5.4} \((X, \mathcal{F})\) is not Hausdorff.
    For any \(x_0 \in X\), we have \(X \setminus \{x_0\} \notin \mathcal{F}\), thus \(X \setminus \{x_0\}\) is not open in \((X, \mathcal{F})\) and \(\{x_0\}\) is not closed in \((X, \mathcal{F})\).
\end{proof}

\begin{exercise}\label{ex 2.5.12}
    Show that the pair \((Y, \mathcal{F}_Y)\) defined in Definition \ref{2.5.7} is indeed a topological space.
\end{exercise}

\begin{proof}
    We have
    \begin{align*}
                 & \begin{cases}
                       X \in \mathcal{F} \\
                       \emptyset \in \mathcal{F}
                   \end{cases}                      & \text{(by Definition \ref{2.5.1})} \\
        \implies & \begin{cases}
                       Y \cap X = Y \in \mathcal{F}_Y \\
                       Y \cap \emptyset = \emptyset \in \mathcal{F}_Y
                   \end{cases}
    \end{align*}
    Let \(n \in \N\) and let \((V_Y^{(i)})_{i = 1}^n\) be a finite collection of open sets in \(\mathcal{F}_Y\).
    Then we have
    \begin{align*}
                 & \forall 1 \leq i \leq n, \exists\ V_X^{(i)} \in \mathcal{F} : V_X^{(i)} \cap Y = V_Y^{(i)} & \text{(by Definition \ref{2.5.7})} \\
        \implies & \bigcap_{i = 1}^n V_X^{(i)} \in \mathcal{F}                                                & \text{(by Definition \ref{2.5.1})} \\
        \implies & Y \cap \bigg(\bigcap_{i = 1}^n V_X^{(i)}\bigg) \in \mathcal{F}_Y                           & \text{(by Definition \ref{2.5.7})} \\
        \implies & \bigcap_{i = 1}^n (V_X^{(i)} \cap Y) = \bigcap_{i = 1}^n V_Y^{(i)} \in \mathcal{F}_Y.
    \end{align*}
    Since \(n\) is arbitrary, we conclude that the intersection of any finite collection of open sets in \((Y, \mathcal{F}_Y)\) is open in \((Y, \mathcal{F}_Y)\).
    Let \(S \subseteq \mathcal{F}_Y\).
    Then we have
    \begin{align*}
                 & \forall V_Y \in S, \exists\ V_X \in \mathcal{F} : V_X \cap Y = V_Y                                         & \text{(by Definition \ref{2.5.7})} \\
        \implies & \bigcup_{V_X \in \mathcal{F} : V_X \cap Y \in S} V_X \in \mathcal{F}                                       & \text{(by Definition \ref{2.5.1})} \\
        \implies & Y \cap \bigg(\bigcup_{V_X \in \mathcal{F} : V_X \cap Y \in S} V_X\bigg) \in \mathcal{F}_Y                  & \text{(by Definition \ref{2.5.7})} \\
        \implies & \bigcup_{V_X \in \mathcal{F} : V_X \cap Y \in S} (V_X \cap Y) = \bigcup_{V_Y \in S} V_Y \in \mathcal{F}_Y.
    \end{align*}
    Since \(S\) is arbitrary, we conclude that the union of arbitrary many open sets in \((Y, \mathcal{F}_Y)\) is open in \((Y, \mathcal{F}_Y)\).
    Combine all the proofs above we conclude by Definition \ref{2.5.1} that \((Y, \mathcal{F}_Y)\) is a topological space.
\end{proof}

\begin{exercise}\label{ex 2.5.13}
    Generalize Corollary \ref{1.5.9} to compact sets in a Hausdorff topological space.
\end{exercise}

\begin{proof}
    Let \((X, \mathcal{F})\) be a Hausdorff space, and let \((K_n)_{n = 1}^\infty\) be a countable collection of non-empty compact topological subspaces of \(X\) such that
    \[
        K_1 \supseteq K_2 \supseteq K_3 \supseteq \dots.
    \]
    We want to show that the intersection \(\bigcap_{n = 1}^\infty K_n\) is non-empty.

    Since \(K_n \subseteq K_1\) for each \(n \in \Z^+\), by Definition \ref{2.5.7} we know that
    \begin{align*}
        \forall n \in \Z^+, \mathcal{F}_{K_n} & = \{K_n \cap V : V \in \mathcal{F}\}                                  \\
                                              & = \{(K_1 \cap K_n) \cap V : V \in \mathcal{F}\} & (K_n \subseteq K_1) \\
                                              & = \{K_n \cap (K_1 \cap V) : V \in \mathcal{F}\}                       \\
                                              & = \{K_n \cap V : V \in \mathcal{F}_{K_1}\}.
    \end{align*}
    Thus by Definition \ref{2.5.7} we know that for every \(n \in \Z^+\), \((K_n, \mathcal{F}_{K_n})\) induced by \((X, \mathcal{F})\) can also induced by \((K_1, \mathcal{F}_{K_1})\).
    By Definition \ref{2.5.9} we have
    \begin{align*}
                 & \forall n \in \Z^+, (K_n, \mathcal{F}_{K_n}) \text{ is compact topological subspace of } (X, \mathcal{F})                                                                \\
        \implies & \forall n \in \Z^+, \exists\ S_n \subseteq \mathcal{F}_{K_n} : \begin{cases}
                                                                                      S_n \text{ is finite} \\
                                                                                      K_n = \bigcup \mathcal{F}_{K_n} = \bigcup S_n
                                                                                  \end{cases}                                                              \\
        \implies & \forall n \in \Z^+, \exists\ S_n \subseteq \mathcal{F}_{K_n} : \begin{cases}
                                                                                      S_n' = \{V \cap W : (V, W) \in S_1 \times S_n\} \text{ is finite}  \\
                                                                                      S_n' \subseteq \mathcal{F}_{K_n} \text{ by Definition \ref{2.5.7}} \\
                                                                                      \forall S \subseteq \mathcal{F}, K_n \subseteq K_1 \subseteq \bigcup S \implies K_n \subseteq \bigcup S_n'
                                                                                  \end{cases} \\
        \implies & \forall n \in \Z^+, (K_n, \mathcal{F}_{K_n}) \text{ is compact topological subspace of } (K_1, \mathcal{F}_{K_1}).
    \end{align*}

    Now we fix one \(n \in \Z^+\).
    Since \((X, \mathcal{F})\) is Hausdorff, we know that
    \begin{align*}
                 & \forall x, y \in K_n, x \neq y                                                                     \\
        \implies & \exists\ V, W \in \mathcal{F} : \begin{cases}
                                                       V \neq \emptyset \neq W \\
                                                       x \in V                 \\
                                                       y \in W                 \\
                                                       V \cap W = \emptyset
                                                   \end{cases}                   & (K_n \subseteq X)                  \\
        \implies & \exists\ V_{K_n}, W_{K_n} \in \mathcal{F}_{K_n} : \begin{cases}
                                                                         V_{K_n} = V \cap K_n                \\
                                                                         W_{K_n} = W \cap K_n                \\
                                                                         x \in V_{K_n}                       \\
                                                                         y \in W_{K_n}                       \\
                                                                         V_{K_n} \neq \emptyset \neq W_{K_n} \\
                                                                         V_{K_n} \cap W_{K_n} = \emptyset
                                                                     \end{cases} & \text{(by Definition \ref{2.5.7})}
    \end{align*}
    Thus \((K_n, \mathcal{F}_{K_n})\) is Hausdorff.
    Since \(n\) is arbitrary, we know that for each \(n \in \Z^+\), \((K_n, \mathcal{F}_{K_n})\) is Hausdorff.

    We claim that for each \(n \in \Z^+\), \((K_n, \mathcal{F}_{K_n})\) is closed in \((K_1, \mathcal{F}_{K_1})\).
    Suppose for sake of contradiction that there exists some \(n \in \Z^+\) such that \((K_n, \mathcal{F}_{K_n})\) is not closed in \((K_1, \mathcal{F}_{K_1})\).
    Then by Exercise \ref{ex 2.5.11} we know that \(\overline{K_n}_{(K_1, \mathcal{F}_{K_1})} \setminus K_n = \emptyset\).
    Let \(y \in \overline{K_n}_{(K_1, \mathcal{F}_{K_1})} \setminus K_n\).
    Since \((K_1, \mathcal{F}_{K_1})\) is Hausdorff, we know that
    \[
        \forall x \in K_n, \exists\ V_x, W_x \in \mathcal{F}_{K_1} : \begin{cases}
            V_x \neq \emptyset \neq W_x \\
            x \in V_x                   \\
            y \in W_x                   \\
            V_x \cap W_x = \emptyset
        \end{cases}
    \]
    Since \((K_n, \mathcal{F}_{K_n})\) is a compact topological subspace of \((K_1, \mathcal{F}_{K_1})\), by Definition \ref{2.5.9} we have
    \[
        K_n \subseteq \bigcup_{x \in K_n} V_x \implies \exists\ S \subseteq K_n : \begin{cases}
            S \text{ is finite} \\
            K_n \subseteq \bigcup_{x \in S} V_x
        \end{cases}
    \]
    By Definition \ref{2.5.1} we have
    \begin{align*}
                 & \begin{cases}
                       \forall x \in S, V_x \in \mathcal{F}_{K_1} \\
                       y \in \bigcap_{x \in S} W_x \in \mathcal{F}_{K_1}
                   \end{cases}                                         \\
        \implies & \forall x \in S, V_x \cap \bigg(\bigcap_{x' \in S} W_{x'}\bigg) = \emptyset              \\
        \implies & \bigcup_{x \in S} \Bigg(V_x \cap \bigg(\bigcap_{x' \in S} W_{x'}\bigg)\Bigg) = \emptyset \\
        \implies & \bigg(\bigcup_{x \in S} V_x\bigg) \cap \bigg(\bigcap_{x' \in S} W_{x'}\bigg) = \emptyset \\
        \implies & K_n \cap \bigg(\bigcap_{x' \in S} W_{x'}\bigg) = \emptyset
    \end{align*}
    But this contradict to the fact that \(y \in \overline{K_n}_{(K_1, \mathcal{F}_{K_1})}\).
    Thus \((K_n, \mathcal{F}_{K_n})\) is closed in \((K_1, \mathcal{F}_{K_1})\) for each \(n \in \Z^+\).

    Let \(V_n = K_1 \setminus K_n\) for every \(n \geq 1\).
    Then for every \(n \geq 1\), we have \(V_n \subseteq K_1\) and \(V_n\) is open in \((K_1, \mathcal{F}_{K_1})\).
    Suppose for sake of contradiction that \(\bigcap_{n = 1}^\infty K_n = \emptyset\).
    Since
    \[
        \bigcup_{n = 1}^\infty V_n = \bigcup_{n = 1}^\infty (K_1 \setminus K_n) = K_1 \setminus \bigg(\bigcap_{n = 1}^\infty K_n\bigg) = K_1
    \]
    and \((K_1, \mathcal{F}_{K_1})\) is compact, by Theorem \ref{1.5.8} we know that there exists a finite set \(F \subseteq \Z^+\) such that
    \[
        K_1 \subseteq \bigcup_{i \in F} V_i.
    \]
    Since \(F\) is finite subset of \(\Z^+\), we know that \(\min(F)\) is well-defined.
    Then we have
    \begin{align*}
                 & K_1 \subseteq \bigcup_{i \in F} V_i \subseteq \bigcup_{n = 1}^\infty V_i = K_1                                                          \\
        \implies & K_1 = \bigcup_{i \in F} V_i                                                                                                             \\
        \implies & K_1 = \bigcup_{i \in F} (K_1 \setminus K_i)                                                                                             \\
        \implies & K_1 = K_1 \setminus \bigg(\bigcap_{i \in F} K_i\bigg)                                                                                   \\
        \implies & \bigcap_{i \in F} K_i = \emptyset                                              & \text{(since \(\bigcap_{i \in F} K_i \subseteq K_1)\)} \\
        \implies & K_{\min(F)} = \emptyset.                                                       & \text{(since \(K_{\min(F)} = \bigcap_{i \in F} K_i)\)}
    \end{align*}
    But by hypothesis we know that \(K_{\min(F)} \neq \emptyset\), a contradiction.
    Thus \(\bigcap_{n = 1}^\infty K_n \neq \emptyset\).
\end{proof}

\begin{exercise}\label{ex 2.5.14}
    Generalize Theorem \ref{1.5.10} to compact sets in a Hausdorff topological space.
\end{exercise}

\begin{proof}
    Let \((X, \mathcal{F})\) be a Hausdorff topological space.
    We first show that if \(Z \subseteq Y \subseteq X\) such that \((Y, \mathcal{F}_Y)\) is compact, then \((Z, \mathcal{F}_Z)\) is compact iff \(Z\) is closed in \((Y, \mathcal{F}_Y)\).
    By Exercise \ref{ex 2.5.13} we know that if \((Z, \mathcal{F}_Z)\) is compact then \(Z\) is closed in \((Y, \mathcal{F}_Y)\).
    So we only need to show that if \(Z\) is closed in \((Y, \mathcal{F}_Y)\) then \((Z, \mathcal{F}_Z)\) is compact.
    Let \(S_Z \subseteq \mathcal{F}_Z\) be an open cover of \(Z\).
    Then we have
    \begin{align*}
                 & Z \text{ is closed in } (Y, \mathcal{F}_Y)           \\
        \implies & Y \setminus Z \text{ is open in } (Y, \mathcal{F}_Y)
    \end{align*}
    and
    \begin{align*}
                 & Z = \bigcup S_Z                                                                                                                   \\
        \implies & Z \subseteq \bigcup \{V \in \mathcal{F}_Y : V \cap Z \in S_Z\}                               & \text{(by Definition \ref{2.5.7})} \\
        \implies & Y = \bigg(\bigcup \{V \in \mathcal{F}_Y : V \cap Z \in S_Z\}\bigg) \cup (Y \setminus Z)                                           \\
        \implies & \exists\ S_Y \subseteq \mathcal{F}_Y : \begin{cases}
                                                              S_Y \text{ is finite}                                    \\
                                                              Y = \bigg(\bigcup \{V \in S_Y : V \cap Z \in S_Z\}\bigg) \\
                                                              \quad \cup (Y \setminus Z)
                                                          \end{cases}                                  & \text{(by Definition \ref{2.5.9})}          \\
        \implies & \exists\ S_Y \subseteq \mathcal{F}_Y : \begin{cases}
                                                              S_Y \text{ is finite} \\
                                                              Z = \bigcup \{V \cap Z : V \in S_Y\}
                                                          \end{cases}                                                        \\
        \implies & \{V \cap Z : V \in S_Y\} \text{ is an finite subcover of } Z \text{ in } (Z, \mathcal{F}_Z).
    \end{align*}
    Since \(S_Z\) is arbitrary, by Definition \ref{2.5.9} we know that \((Z, \mathcal{F}_Z)\) is compact.

    Next we show that if \((Y_i)_{i = 1}^n\) is a finite collection of compact topological subspaces of \((X, \mathcal{F})\), then \(\big(\bigcup_{i = 1}^n Y_i, \mathcal{F}_{\bigcup_{i = 1}^n Y_i}\big)\) is also a compact topological subspace of \((X, \mathcal{F})\).
    By Definition \ref{2.5.7} we have
    \[
        \mathcal{F}_{\bigcup_{i = 1}^n Y_i} = \bigg\{V \cap \bigg(\bigcup_{i = 1}^n Y_i\bigg) : V \in \mathcal{F}\bigg\}.
    \]
    Observe that
    \begin{align*}
                 & \begin{cases}
                       X \in \mathcal{F} \\
                       \emptyset \in \mathcal{F}
                   \end{cases}                                                                                 \\
        \implies & \begin{cases}
                       X \cap \bigg(\bigcup_{i = 1}^n Y_i\bigg) = \bigcup_{i = 1}^n Y_i \in \mathcal{F}_{\bigcup_{i = 1}^n Y_i} \\
                       \emptyset \cap \bigg(\bigcup_{i = 1}^n Y_i\bigg) = \emptyset \in \mathcal{F}_{\bigcup_{i = 1}^n Y_i}
                   \end{cases}
    \end{align*}
    Let \((V^{(j)})_{j = 1}^m\) be a finite collection of elements in \(\mathcal{F}_{\bigcup_{i = 1}^n Y_i}\).
    Then we have
    \begin{align*}
                 & \forall 1 \leq j \leq m, \exists\ V_X^{(j)} \in \mathcal{F} : V_X^{(j)} \cap \bigg(\bigcup_{i = 1}^n Y_i\bigg) = V^{(j)}                            & \text{(by Definition \ref{2.5.7})} \\
        \implies & \bigcap_{j = 1}^m V_X^{(j)} \in \mathcal{F}                                                                                                         & \text{(by Definition \ref{2.5.1})} \\
        \implies & \bigg(\bigcap_{j = 1}^m V_X^{(j)}\bigg) \cap \bigg(\bigcup_{i = 1}^n Y_i\bigg) \in \mathcal{F}_{\bigcup_{i = 1}^n Y_i}                              & \text{(by Definition \ref{2.5.7})} \\
        \implies & \bigcap_{j = 1}^m \Bigg(V_X^{(j)} \cap \bigg(\bigcup_{i = 1}^n Y_i\bigg)\Bigg) = \bigcap_{j = 1}^m V^{(j)} \in \mathcal{F}_{\bigcup_{i = 1}^n Y_i}.
    \end{align*}
    Since \((V^{(j)})_{j = 1}^m\) is arbitrary, we conclude that the intersection of any finite collection of open sets in \(\big(\bigcup_{i = 1}^n Y_i, \mathcal{F}_{\bigcup_{i = 1}^n Y_i}\big)\) is open in \(\big(\bigcup_{i = 1}^n Y_i, \mathcal{F}_{\bigcup_{i = 1}^n Y_i}\big)\).
    Let \(S \subseteq \mathcal{F}_{\bigcup_{i = 1}^n Y_i}\).
    Then we have
    \begin{align*}
                 & \forall V \in S, \exists\ V_X \in \mathcal{F} : V_X \cap \bigg(\bigcup_{i = 1}^n Y_i\bigg) = V                                                                                                          & \text{(by Definition \ref{2.5.7})} \\
        \implies & \bigcup_{V_x \in \mathcal{F} : V_X \cap (\bigcup_{i = 1}^n Y_i) \in \mathcal{F}_{\bigcup_{i = 1}^n Y_i}} V_x \in \mathcal{F}                                                                            & \text{(by Definition \ref{2.5.1})} \\
        \implies & \bigg(\bigcup_{V_x \in \mathcal{F} : V_X \cap (\bigcup_{i = 1}^n Y_i) \in \mathcal{F}_{\bigcup_{i = 1}^n Y_i}} V_x\bigg) \cap \bigg(\bigcup_{i = 1}^n Y_i\bigg) \in \mathcal{F}_{\bigcup_{i = 1}^n Y_i} & \text{(by Definition \ref{2.5.7})} \\
        \implies & \bigcup_{V \in S} V \in \mathcal{F}_{\bigcup_{i = 1}^n Y_i}.
    \end{align*}
    Since \(S\) is arbitrary, we conclude that the union of arbitrary many open sets in \(\big(\bigcup_{i = 1}^n Y_i, \mathcal{F}_{\bigcup_{i = 1}^n Y_i}\big)\) is open in \(\big(\bigcup_{i = 1}^n Y_i, \mathcal{F}_{\bigcup_{i = 1}^n Y_i}\big)\).
    By Definition \ref{2.5.1} and all the proofs above we conclude that \(\big(\bigcup_{i = 1}^n Y_i, \mathcal{F}_{\bigcup_{i = 1}^n Y_i}\big)\) is a topological subspace of \((X, \mathcal{F})\).

    Let \(S \subseteq \mathcal{F}_{\bigcup_{i = 1}^n Y_i}\) be an open cover of \(\bigcup_{i = 1}^n Y_i\) in \(\big(\bigcup_{i = 1}^n Y_i, \mathcal{F}_{\bigcup_{i = 1}^n Y_i}\big)\).
    Then we have
    \begin{align*}
                 & \forall 1 \leq i \leq n, Y_i \subseteq \bigcup_{j = 1}^n Y_i = \bigcup S        \\
        \implies & \forall 1 \leq i \leq n, \exists\ S_i \in \mathcal{F}_{\bigcup_{j = 1}^n Y_j} : \\
                 & \begin{cases}
                       S_i \text{ is finite} \\
                       Y_i \subseteq \bigcup S_i \subseteq \bigcup_{i = 1}^n Y_i = \bigcup S
                   \end{cases}           & \text{(by Definition \ref{2.5.9})}            \\
        \implies & \bigcup_{i = 1}^n Y_i = \bigcup \bigg(\bigcup_{i = 1}^n S_i\bigg).
    \end{align*}
    Since \(S_i\) is finite for each \(1 \leq i \leq n\), we know that \(\bigcup_{i = 1}^n S_i\) is a finite subcover of \(\bigcup_{i = 1}^n Y_i\) in \(\big(\bigcup_{i = 1}^n Y_i, \mathcal{F}_{\bigcup_{i = 1}^n Y_i}\big)\).
    Since \(S\) is arbitrary, by Definition \ref{2.5.9} we know that \(\big(\bigcup_{i = 1}^n Y_i, \mathcal{F}_{\bigcup_{i = 1}^n Y_i}\big)\) is a compact topological subspace of \((X, \mathcal{F})\).

    Finally we show that every finite subset of \(X\) is compact.
    Let \(x_0 \in X\) and let \(\mathcal{F}_{\{x_0\}} = \big\{V \cap \{x_0\} : V \in \mathcal{F}\big\}\).
    We have
    \begin{align*}
                 & \begin{cases}
                       X \in \mathcal{F} \\
                       \emptyset \in \mathcal{F}
                   \end{cases}                                    & \text{(by Definition \ref{2.5.1})} \\
        \implies & \begin{cases}
                       X \cap \{x_0\} = \{x_0\} \in \mathcal{F}_{\{x_0\}} \\
                       \emptyset \cap \{x_0\} = \emptyset \in \mathcal{F}_{\{x_0\}}
                   \end{cases}
    \end{align*}
    Since \(\mathcal{F}_{\{x_0\}} = \big\{\emptyset, \{x_0\}\big\}\), by Exercise \ref{ex 2.5.1} we know that \((\{x_0\}, \mathcal{F}_{\{x_0\}})\) is a topological space.
    Let \(S \subseteq \mathcal{F}_{\{x_0\}}\) such that \(\{x_0\} \subseteq \bigcup S\).
    Then we have
    \begin{align*}
                 & \{x_0\} \subseteq \bigcup S  \\
        \implies & \exists\ V \in S : x_0 \in V \\
        \implies & \{x_0\} \subseteq V.
    \end{align*}
    Since \(S\) is arbitrary, we conclude that every open cover of \(\{x_0\}\) has a finite subcover in \((\{x_0\}, \mathcal{F}_{\{x_0\}})\), and by Definition \ref{2.5.9} we know that \((\{x_0\}, \mathcal{F}_{\{x_0\}})\) is compact.
    Since \(x_0\) is arbitrary, we conclude that every singleton subset of \(X\) is compact.
    And from the proof above we conclude that every finite subset of \(X\) is compact.
\end{proof}

\begin{exercise}\label{ex 2.5.15}
    Let \((X, d_X)\) and \((Y, d_Y)\) be metric spaces (and hence a topological space).
    Show that the two notions continuity (both at a point, and on the whole domain) of a function \(f : X \to Y\) in Definition \ref{2.1.1} and Definition \ref{2.5.8} coincide.
\end{exercise}

\begin{proof}
    Let
    \begin{align*}
         & \mathcal{F}_X = \{V \subseteq X : V \text{ is open in } (X, d_X)\}; \\
         & \mathcal{F}_Y = \{V \subseteq Y : V \text{ is open in } (Y, d_Y)\}.
    \end{align*}
    Since \((X, d_X)\) and \((Y, d_Y)\) are metric spaces, we know that \((X, \mathcal{F}_X)\) and \((Y, \mathcal{F}_Y)\) are topological spaces.

    Let \(x_0 \in X\).
    First suppose that \(f\) is continuous at \(x_0\) in the sense of Definition \ref{2.1.1}.
    Then we have
    \begin{align*}
                 & \forall \varepsilon \in \R^+, \exists\ \delta \in \R^+ :                                                               \\
                 & \Big(\forall x \in X, d_X(x, x_0) < \delta \implies d_Y\big(f(x), f(x_0)\big) < \varepsilon\Big)                       \\
        \implies & \forall \varepsilon \in \R^+, \exists\ \delta \in \R^+ :                                                               \\
                 & \Big(\forall x \in X, x \in B_{(X, d_X)}(x_0, \delta) \implies f(x) \in B_{(Y, d_Y)}\big(f(x_0), \varepsilon\big)\Big) \\
        \implies & \forall \varepsilon \in \R^+, \exists\ \delta \in \R^+ :                                                               \\
                 & f\big(B_{(X, d_X)}(x_0, \delta)\big) \subseteq B_{(Y, d_Y)}\big(f(x_0), \varepsilon\big)
    \end{align*}
    Let \(V \in \mathcal{F}_Y\) such that \(f(x_0) \in V\).
    Then we have
    \begin{align*}
                 & f(x_0) \in V                                                                                                                                                               \\
        \implies & \exists\ \varepsilon \in \R^+ : B_{(Y, d_Y)}\big(f(x_0), \varepsilon\big) \subseteq V                                            & \text{(by Proposition \ref{1.2.15}(a))} \\
        \implies & \exists\ \delta \in \R^+ : f\big(B_{(X, d_X)}(x_0, \delta)\big) \subseteq B_{(Y, d_Y)}\big(f(x_0), \varepsilon\big) \subseteq V.
    \end{align*}
    By Proposition \ref{1.2.15}(c) we know that \(B_{(X, d_X)}(x_0, \delta) \in \mathcal{F}_X\).
    Since \(V\) is arbitrary, by Definition \ref{2.5.8} we know that \(f\) is continuous at \(x_0\) from \((X, \mathcal{F}_X)\) to \((Y, \mathcal{F}_Y)\).

    Now suppose that \(f\) is continuous at \(x_0\) in the sense of Definition \ref{2.5.8}.
    Then we have
    \[
        \forall V \in \mathcal{F}_Y, f(x_0) \in V \implies \exists\ U \in \mathcal{F}_X : \begin{cases}
            x_0 \in U \\
            f(U) \subseteq V
        \end{cases}
    \]
    Let \(\varepsilon \in \R^+\).
    Then we have
    \begin{align*}
                 & B_{(Y, d_Y)}\big(f(x_0), \varepsilon\big) \in \mathcal{F}_Y                                       & \text{(by Proposition \ref{1.2.15}(c))} \\
        \implies & \exists\ U \in \mathcal{F}_X : \begin{cases}
                                                      x_0 \in U \\
                                                      f(U) \subseteq B_{(Y, d_Y)}\big(f(x_0), \varepsilon\big)
                                                  \end{cases}                                                      \\
        \implies & \exists\ \delta \in \R^+ : \begin{cases}
                                                  B_{(X, d_X)}(x_0, \delta) \subseteq U \\
                                                  f(U) \subseteq B_{(Y, d_Y)}\big(f(x_0), \varepsilon\big)
                                              \end{cases}                                          & \text{(by Proposition \ref{1.2.15}(a))}                   \\
        \implies & \exists\ \delta \in \R^+ :                                                                                                                  \\
                 & f\big(B_{(X, d_X)}(x_0, \delta)\big) \subseteq B_{(Y, d_Y)}\big(f(x_0), \varepsilon\big)                                                    \\
        \implies & \exists\ \delta \in \R^+ :                                                                                                                  \\
                 & \Big(\forall x \in X, d_X(x, x_0) < \delta \implies d_Y\big(f(x), f(x_0)\big) < \varepsilon\Big).
    \end{align*}
    Since \(\varepsilon\) is arbitrary, by Definition \ref{2.1.1} we know that \(f\) is continuous at \(x_0\) from \((X, d_X)\) to \((Y, d_Y)\).

    Since \(x_0\) is arbitrary, we conclude that \(f\) is continuous on \(X\) from \((X, d_X)\) to \((Y, d_Y)\) iff \(f\) is continuous on \(X\) from \((X, \mathcal{F}_X)\) to \((Y, \mathcal{F}_Y)\).
\end{proof}

\begin{exercise}\label{ex 2.5.16}
    Show that when Theorem \ref{2.1.4} is extended to topological spaces, that (a) implies (b).
    (The converse is false, but constructing an example is diffcult.)
    Show that when Theorem \ref{2.1.5} is extended to topological spaces, that (a), (c), (d) are all equivalent to each other, and imply (b).
    (Again, the converse implications are false, but diffcult to prove.)
\end{exercise}

\begin{proof}
    Let \((X, \mathcal{F}_X)\) and \((Y, \mathcal{F}_Y)\) be two topological spaces.
    Let \(f : X \to Y\) be a function.
    Let \(x_0 \in X\).
    We first show that if \(f\) is continuous at \(x_0\) from \((X, \mathcal{F}_X)\) to \((Y, \mathcal{F}_Y)\) and \((x^{(n)})_{n = 1}^\infty\) is a sequence in \(X\) which converges to \(x_0\) in \((X, \mathcal{F}_X)\), then \(\big(f(x)\big)_{n = 1}^\infty\) converges to \(f(x_0)\) in \((Y, \mathcal{F}_Y)\).
    Since \(f\) is continuous at \(x_0\) from \((X, \mathcal{F}_X)\) to \((Y, \mathcal{F}_Y)\), by Definition \ref{2.5.8} we know that
    \[
        \forall V \in \mathcal{F}_Y, f(x_0) \in V \implies \exists\ U \in \mathcal{F}_X : \begin{cases}
            x_0 \in U \\
            f(U) \subseteq V
        \end{cases}
    \]
    Since \((x^{(n)})_{n = 1}^\infty\) converges to \(x_0\) in \((X, \mathcal{F}_X)\), by Definition \ref{2.5.4} we have
    \[
        \forall U \in \mathcal{F}_X, x_0 \in U \implies \exists\ N \in \Z^+ : \forall n \geq N, x^{(n)} \in U
    \]
    This means
    \[
        \forall V \in \mathcal{F}_Y, f(x_0) \in V \implies \exists\ U \in \mathcal{F}_X : \begin{cases}
            x_0 \in U        \\
            f(U) \subseteq V \\
            \exists\ N \in \Z^+ : \forall n \geq N, f(x^{(n)}) \in f(U) \subseteq V
        \end{cases}
    \]
    and we have
    \[
        \forall V \in \mathcal{F}_Y, f(x_0) \in V \implies \exists\ N \in \Z^+ : \forall n \geq N, f(x^{(n)}) \in V.
    \]
    By Definition \ref{2.5.4} we know that \(\big(f(x^{(n)})\big)_{n = 1}^\infty\) converges to \(f(x_0)\) in \((Y, \mathcal{F}_Y)\).
    Since \(x_0\) is arbitrary, we conclude that if \(f\) is continuous on \(X\) from \((X, \mathcal{F}_X)\) to \((Y, \mathcal{F}_Y)\), then whenever \((x^{(n)})_{n = 1}^\infty\) is a sequence in \(X\) which converges to some \(x_0 \in X\) in \((X, \mathcal{F})\), the sequence \(\big(f(x^{(n)})\big)_{n = 1}^\infty\) converges to \(f(x_0)\) in \((Y, \mathcal{F}_Y)\).

    Next we show that if \(f\) is continuous on \(X\) from \((X, \mathcal{F}_X)\) to \((Y, \mathcal{F}_Y)\), then whenever \(V \in \mathcal{F}_Y\), the set \(f^{-1}(V) \in \mathcal{F}_X\).
    Let \(V \in \mathcal{F}_Y\) and let \(x_0 \in f^{-1}(V)\).
    Since \(f\) is continuous at \(x_0\) from \((X, \mathcal{F}_X)\) to \((Y, \mathcal{F}_Y)\), we know that
    \[
        \exists\ U \in \mathcal{F}_X : \begin{cases}
            x_0 \in U \\
            f(U) \subseteq V
        \end{cases}
    \]
    Since \(f(U) \subseteq V\), we know that \(U \subseteq f^{-1}(V)\).
    Since \(U \in \mathcal{F}_X\), by Definition \ref{2.5.5} we know that \(x_0 \in \text{int}_{(X, \mathcal{F}_X)}\big(f^{-1}(V)\big)\).
    Since \(x_0\) is arbitrary, by Exercise \ref{ex 2.5.11} we know that \(f^{-1}(V) \in \mathcal{F}_X\).

    Next we show that if \(V \in \mathcal{F}_Y\) implies \(f^{-1}(V) \in \mathcal{F}_X\), then \(U\) is closed in \((Y, \mathcal{F}_Y)\) implies \(f^{-1}(U)\) is closed in \((X, \mathcal{F}_X)\).
    Let \(U \subseteq Y\) such that \(U\) is closed in \((Y, \mathcal{F}_Y)\).
    Then we have
    \begin{align*}
                 & U \text{ is closed in } (Y, \mathcal{F}_Y)                                                                 \\
        \implies & Y \setminus U \text{ is open in } (Y, \mathcal{F}_Y)                                                       \\
        \implies & f^{-1}(Y \setminus U) \text{ is open in } (X, \mathcal{F}_X)               & \text{(from the proof above)} \\
        \implies & X \setminus f^{-1}(Y \setminus U) \text{ is closed in } (X, \mathcal{F}_X)                                 \\
        \implies & f^{-1}(U) \text{ is closed in } (X, \mathcal{F}_X).
    \end{align*}

    Next we show that if \(U\) is closed in \((Y, \mathcal{F}_Y)\) implies \(f^{-1}(U)\) is closed in \((X, \mathcal{F}_X)\), then \(f\) is continuous on \(X\) from \((X, \mathcal{F}_X)\) to \((Y, \mathcal{F}_Y)\).
    Let \(x_0 \in X\) and let \(V \in \mathcal{F}_Y\) such that \(f(x_0) \in V\).
    We have
    \begin{align*}
                 & V \in \mathcal{F}_Y                                                                               \\
        \implies & V \text{ is open in } (Y, \mathcal{F}_Y)                                                          \\
        \implies & Y \setminus V \text{ is closed in } (Y, \mathcal{F}_Y)                                            \\
        \implies & f^{-1}(Y \setminus V) \text{ is closed in } (Y, \mathcal{F}_Y)           & \text{(by hypothesis)} \\
        \implies & X \setminus f^{-1}(Y \setminus V) \text{ is open in } (Y, \mathcal{F}_Y)                          \\
        \implies & f^{-1}(V) \text{ is open in } (Y, \mathcal{F}_Y)                                                  \\
        \implies & f^{-1}(V) \in \mathcal{F}_X.
    \end{align*}
    Since \(V\) is arbitrary, we know that
    \[
        \forall V \in \mathcal{F}_Y, f(x_0) \in V \implies \exists\ U \in \mathcal{F}_X : \begin{cases}
            x_0 \in U \\
            f(U) \subseteq V
        \end{cases}
    \]
    and by Definition \ref{2.5.8} \(f\) is continuous at \(x_0\) from \((X, \mathcal{F}_X)\) to \((Y, \mathcal{F}_Y)\).
    Since \(x_0\) is arbitrary, we conclude that \(f\) is continuous on \(X\) from \((X, \mathcal{F}_X)\) to \((Y, \mathcal{F}_Y)\).

    From all proofs above we conclude that Theorem \ref{2.1.5}(a)(c)(d) are equivalent in the topological version.
\end{proof}

\begin{exercise}\label{ex 2.5.17}
    Generalize both Theorem \ref{2.3.1} and Proposition \ref{2.3.2} to compact sets in a topological space.
\end{exercise}

\begin{proof}
    Let \((X, \mathcal{F}_X)\) and \((Y, \mathcal{F}_Y)\) be two topological spaces.
    Let \(f : X \to Y\) be continuous function from \((X, \mathcal{F}_X)\) to \((Y, \mathcal{F}_Y)\), and let \(K \subseteq X\) such that \((K, \mathcal{F}_K)\) is compact.
    We want to show that \(\big(f(K), \mathcal{F}_{f(K)}\big)\) is also compact.
    Let \(S \subseteq \mathcal{F}_{f(K)}\) be an open cover of \(f(K)\), i.e., \(f(K) = \bigcup_{V_K \in S} V_K\).
    By Definition \ref{2.5.7} we know that the set \(S_Y = \{V \in \mathcal{F}_Y : V \cap f(K) \in S\}\) is non-empty.
    Then we have
    \begin{align*}
                 & \forall V \in S_Y, f^{-1}(V) \in \mathcal{F}_X                                                     & \text{(by Exercise \ref{ex 2.5.16})} \\
        \implies & \forall V \in S_Y, f^{-1}(V) \cap K \in \mathcal{F}_K                                              & \text{(by Definition \ref{2.5.7})}   \\
        \implies & \forall V \in S_Y, f\big(f^{-1}(V) \cap K\big) = V \cap f(K)                                                                              \\
        \implies & f(K) = \bigcup_{V \in S_Y} f\big(f^{-1}(V) \cap K\big) = \bigcup_{V \in S_Y} \big(V \cap f(K)\big)                                        \\
        \implies & K \subseteq \bigcup_{V \in S_Y} \big(f^{-1}(V) \cap K\big)                                                                                \\
        \implies & \exists\ S_Y' \subseteq S_Y : \begin{cases}
                                                     S_Y' \text{ is finite}                                      \\
                                                     K \subseteq \bigcup_{V \in S_Y'} \big(f^{-1}(V) \cap K\big) \\
                                                     f(K) = \bigcup_{V \in S_Y'} \big(V \cap f(K)\big)
                                                 \end{cases}                                     & \text{(by Definition \ref{2.5.9})}                        \\
    \end{align*}
    Since \(S\) is arbitrary, by Definition \ref{2.5.9} we know that \(\big(f(K), \mathcal{F}_{f(K)}\big)\) is compact.

    Now let \((X, \mathcal{F})\) be a compact topological space.
    Define
    \[
        \mathcal{F}_{\R} = \{V \subseteq \R : V \text{ is open in } (\R, d_{l^1}|_{\R \times \R})\}.
    \]
    Let \(f : X \to \R\) be a continuous function from \((X, \mathcal{F})\) to \((\R, \mathcal{F}_{\R})\).
    We want to show that \(f\) is bounded.
    Furthermore, if \(X \neq \emptyset\), then there exists some \(x_{\min}, x_{\max} \in X\) such that \(f(x_{\min}) \leq f(x) \leq f(x_{\max})\) for each \(x \in X\).
    We have
    \begin{align*}
                 & \begin{cases}
                       (X, \mathcal{F}) \text{ is compact} \\
                       f \text{ is continuous from } (X, \mathcal{F}) \text{ to } (\R, \mathcal{F}_{\R})
                   \end{cases}                                  \\
        \implies & \big(f(X), \mathcal{F}_{f(X)}\big) \text{ is compact}                             & \text{(from the proof above)} \\
        \implies & \begin{cases}
                       f(X) \text{ is closed in } (\R, \mathcal{F}_{\R}) \\
                       \big(f(X), d_{l^1}|_{f(X) \times f(X)}\big) \text{ is bounded}
                   \end{cases}                    & \text{(by Theorem \ref{1.5.7})}                                                  \\
        \implies & \begin{cases}
                       f(X) \text{ is closed in } (\R, \mathcal{F}_{\R}) \\
                       f(X) \text{ is bounded subset of } \R
                   \end{cases}                              & \text{(by Exercise \ref{ex 1.5.1})}
    \end{align*}
    The rest follows as in Proposition \ref{2.3.2}.
\end{proof}