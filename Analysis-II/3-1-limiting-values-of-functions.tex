\section{Limiting values of functions}\label{sec 3.1}

\begin{definition}[Limiting value of a function]\label{3.1.1}
    Let \((X, d_X)\) and \((Y, d_Y)\) be metric spaces, let \(E\) be a subset of \(X\), and let \(f : X \to Y\) be a function.
    If \(x_0 \in X\) is an adherent point of \(E\), and \(L \in Y\), we say that \(f(x)\) converges to \(L\) in \(Y\) as \(x\) converges to \(x_0\) in \(E\), or write \(\lim_{x \to x_0 ; x \in E} f(x) = L\), if for every \(\varepsilon > 0\) there exists a \(\delta > 0\) such that \(d_Y\big(f(x), L\big) < \varepsilon\) for all \(x \in E\) such that \(d_X(x, x_0) < \delta\).
\end{definition}