\section{Uniform convergence and continuity}\label{sec 3.3}

\begin{theorem}[Uniform limits preserve continuity I]\label{3.3.1}
    Suppose \((f^{(n)})_{n = 1}^\infty\) is a sequence of functions from one metric space \((X, d_X)\) to another \((Y, d_Y)\), and suppose that this sequence converges uniformly to another function \(f : X \to Y\).
    Let \(x_0\) be a point in \(X\).
    If the functions \(f^{(n)}\) are continuous at \(x_0\) for each \(n\), then the limiting function \(f\) is also continuous at \(x_0\).
\end{theorem}

\begin{proof}
    We have
    \begin{align*}
                 & (f^{(n)})_{n = 1}^\infty \text{ converges uniformly to } f \text{ on } X                                                       \\
                 & \text{with respect to } d_Y                                                                                                    \\
        \implies & \forall \varepsilon \in \R^+, \exists\ N \in \Z^+ :                                                                            \\
                 & \forall n \geq N, \forall x \in X, d_Y\big(f^{(n)}(x), f(x)\big) < \frac{\varepsilon}{3}. & \text{(by Definition \ref{3.2.7})}
    \end{align*}
    We choose one pair of \(\varepsilon\) and \(N\).
    For each \(n \in \Z^+\), since \(f^{(n)}\) is continuous at \(x_0\) from \((X, d_X)\) to \((Y, d_Y)\), by Definition \ref{2.1.1} we have
    \begin{align*}
                 & \forall n \geq N, f^{(n)} \text{ is continuous at } x_0 \text{ from } (X, d_X) \text{ to } (Y, d_Y)                                                                                         \\
        \implies & \forall n \geq N, \exists\ \delta \in \R^+ :                                                                                                                                                \\
                 & \Big(\forall x \in X, d_X(x, x_0) < \delta \implies d_Y\big(f^{(n)}(x), f^{(n)}(x_0)\big) < \frac{\varepsilon}{3}\Big)                                                                      \\
        \implies & \forall n \geq N, \exists\ \delta \in \R^+ :                                                                                                                                                \\
                 & \Big(\forall x \in X, d_X(x, x_0) < \delta \implies d_Y\big(f(x), f(x_0)\big)                                                                                                               \\
                 & \leq d_Y\big(f(x), f^{(n)}(x)\big) + d_Y\big(f^{(n)}(x), f^{(n)}(x_0)\big) + d_Y\big(f^{(n)}(x_0), f(x_0)\big) < \frac{\varepsilon}{3} + \frac{\varepsilon}{3} + \frac{\varepsilon}{3}\Big) \\
        \implies & \forall n \geq N, \exists\ \delta \in \R^+ :                                                                                                                                                \\
                 & \Big(\forall x \in X, d_X(x, x_0) < \delta \implies d_Y\big(f(x), f(x_0)\big) < \varepsilon.
    \end{align*}
    Since \(\varepsilon\) is arbitrary, by Definition \ref{2.1.1} we know that \(f\) is continuous at \(x_0\) from \((X, d_X)\) to \((Y, d_Y)\).
\end{proof}

\begin{corollary}[Uniform limits preserve continuity II]\label{3.3.2}
    Let \((f^{(n)})_{n = 1}^\infty\) be a sequence of functions from one metric space \((X, d_X)\) to another \((Y, d_Y)\), and suppose that this sequence converges uniformly to another function \(f : X \to Y\).
    If the functions \(f^{(n)}\) are continuous on \(X\) for each \(n\), then the limiting function \(f\) is also continuous on \(X\).
\end{corollary}

\begin{proof}
    By applying Theorem \ref{3.3.1} to each \(x \in X\) we conclude that \(f\) is continuous on \(X\) from \((X, d_X)\) to \((Y, d_Y)\).
\end{proof}

\begin{proposition}[Interchange of limits and uniform limits]\label{3.3.3}
    Let
    \((X, d_X)\) and \((Y, d_Y)\) be metric spaces, with \(Y\) complete, and let \(E\) be a subset of \(X\).
    Let \((f^{(n)})_{n = 1}^\infty\) be a sequence of functions from \(E\) to \(Y\), and suppose that this sequence converges uniformly in \(E\) to some function \(f : E \to Y\).
    Let \(x_0 \in X\) be an adherent point of \(E\), and suppose that for each \(n\) the limit \(\lim_{x \to x_0 ; x \in E} f^{(n)}(x)\) exists.
    Then the limit \(\lim_{x \to x_0 ; x \in E} f(x)\) also exists, and is equal to the limit of the sequence \(\big(\lim_{x \to x_0 ; x \in E} f^{(n)}(x)\big)_{n = 1}^\infty\);
    in other words we have the interchange of limits
    \[
        \lim_{n \to \infty} \lim_{x \to x_0 ; x \in E} f^{(n)}(x) = \lim_{x \to x_0 ; x \in E} \lim_{n \to \infty} f^{(n)}(x).
    \]
\end{proposition}

\begin{proof}
    For each \(n \in \Z^+\), we define \(d_Y - \lim_{x \to x_0 ; x \in E} f^{(n)}(x) = L^{(n)}\).
    We claim that the sequence \((L^{(n)})_{n = 1}^\infty\) converges in \(Y\) with respect to \(d_Y\).
    Since \((Y, d_Y)\) is complete, by Definition \ref{1.4.10} it suffices to show that \((L^{(n)})_{n = 1}^\infty\) is a Cauchy sequence in \((Y, d_Y)\).
    Let \(n_1, n_2 \in \Z^+\).
    Then by Definition \ref{3.2.7} we have
    \begin{align*}
                 & (f^{(n)})_{n = 1}^\infty \text{ converges uniformly to } f \text{ on } X \text{ with respect to } d_Y                                        \\
        \implies & \forall \varepsilon \in \R^+, \exists\ N \in \Z^+ : \forall n \geq N, \forall x \in X, d_Y\big(f^{(n)}(x), f(x)\big) < \frac{\varepsilon}{4}
    \end{align*}
    Now fix one pair of \(\varepsilon\) and \(N\).
    Since \(L^{(n)}\) exists for each \(n \in \N\), by Definition \ref{3.1.1} we have
    \begin{align*}
                 & \forall n \geq N, d_Y - \lim_{x \to x_0 ; x \in E} f^{(n)}(x) = L^{(n)}                                                                                        \\
        \implies & \forall n \geq N, \exists\ \delta \in \R^+ : \Big(\forall x \in X, d_X(x, x_0) < \delta \implies d_Y\big(f^{(n)}(x), L^{(n)}\big) < \frac{\varepsilon}{4}\Big) \\
        \implies & \forall n_1, n_2 \geq N, \exists\ \delta \in \R^+ :                                                                                                            \\
                 & \Big(\forall x \in X, d_X(x, x_0) < \delta \implies d_Y\big(L^{(n_1)}, L^{(n_2)}\big)                                                                          \\
                 & \leq d_Y\big(L^{(n_1)}, f^{(n_1)}(x)\big) + d_Y\big(f^{(n_1)}(x), f(x)\big)                                                                                    \\
                 & \quad + d_Y\big(f(x), f^{(n_2)}(x)\big) + d_Y\big(f^{(n_2)}(x), L^{(n_2)}\big)                                                                                 \\
                 & < \frac{\varepsilon}{4} + \frac{\varepsilon}{4} + \frac{\varepsilon}{4} + \frac{\varepsilon}{4}\Big)                                                           \\
        \implies & \forall n_1, n_2 \geq N, d_Y\big(L^{(n_1)}, L^{(n_2)}\big) < \varepsilon
    \end{align*}
    Since \(\varepsilon\) is arbitrary, we have
    \[
        \forall \varepsilon \in \R^+, \exists\ N \in \Z^+ : \forall n_1, n_2 \geq N, d_Y\big(L^{(n_1)}, L^{(n_2)}\big) < \varepsilon
    \]
    and by Definition \ref{1.4.6} \((L^{(n)})_{n = 1}^\infty\) is a Cauchy sequence in \((Y, d_Y)\).

    Let \(L \in Y\) such that \(d_Y - \lim_{n \to \infty} L^{(n)} = L\).
    Again by Definition \ref{3.2.7} we have
    \begin{align*}
                 & (f^{(n)})_{n = 1}^\infty \text{ converges uniformly to } f \text{ on } X \text{with respect to } d_Y                                              \\
        \implies & \forall \varepsilon \in \R^+, \exists\ N_1 \in \Z^+ : \forall n \geq N_1, \forall x \in X, d_Y\big(f^{(n)}(x), f(x)\big) < \frac{\varepsilon}{3}.
    \end{align*}
    Again we choose one pair of \(\varepsilon\) and \(N_1\).
    Since \(L\) exists, by Definition \ref{3.1.1} we have
    \begin{align*}
                 & \lim_{n \to \infty} d_Y\big(L^{(n)}, L\big) = 0                                                                                                                              \\
        \implies & \exists\ N_2 \in \Z^+ : \forall n \geq N_2, d_Y(L^{(n)}, L) < \frac{\varepsilon}{3}                                                                                          \\
        \implies & \exists\ N = \max(N_1, N_2) : \forall n \geq N,                                                                                                                              \\
                 & \begin{cases}
                       \exists\ \delta \in \R^+ : \forall x \in X, d_X(x, x_0) < \delta \implies d_Y\big(f^{(n)}(x), L^{(n)}\big) < \frac{\varepsilon}{3} \\
                       d_Y(L^{(n)}, L) < \frac{\varepsilon}{3}                                                                                            \\
                       \forall x \in X, d_Y\big(f^{(n)}(x), f(x)\big) < \frac{\varepsilon}{3}
                   \end{cases}                                           \\
        \implies & \exists\ N = \max(N_1, N_2) : \forall n \geq N, \exists\ \delta \in \R^+ :                                                                                                   \\
                 & \Big(\forall x \in X, d_X(x, x_0) < \delta \implies d_Y\big(f(x), L\big)                                                                                                     \\
                 & \leq d_Y\big(f(x), f^{(n)}(x)\big) + d_Y\big(f^{(n)}(x), L^{(n)}\big) + d_Y\big(L^{(n)}, L\big) < \frac{\varepsilon}{3} + \frac{\varepsilon}{3} + \frac{\varepsilon}{3}\Big) \\
        \implies & \exists\ \delta \in \R^+ : \Big(\forall x \in X, d_X(x, x_0) < \delta \implies d_Y\big(f(x), L\big) < \varepsilon\Big).
    \end{align*}
    Since \(\varepsilon\) is arbitrary, by Definition \ref{3.1.1} we know that \(d_Y - \lim_{x \to x_0 ; x \in E} f(x) = L\).
\end{proof}

\begin{proposition}\label{3.3.4}
    Let \((f^{(n)})_{n = 1}^\infty\) be a sequence of continuous functions from one metric space \((X, d_X)\) to another \((Y, d_Y)\), and suppose that this sequence converges uniformly to another function \(f : X \to Y\).
    Let \(x^{(n)}\) be a sequence of points in \(X\) which converge to some limit \(x\).
    Then \(f^{(n)}(x^{(n)})\) converges (in \(Y\)) to \(f(x)\).
\end{proposition}

\begin{proof}
    Let \(x_0 \in X\).
    Suppose that \((x^{(n)})_{n = 1}^\infty\) is a sequence in \(X\) such that
    \[
        \lim_{n \to \infty} d_X(x^{(n)}, x_0) = 0.
    \]
    By Theorem \ref{3.3.1} we know that \(f\) is continuous at \(x_0\) from \((X, d_X)\) to \((Y, d_Y)\).
    Thus by Theorem \ref{2.1.4}(a)(b) we have
    \[
        \lim_{n \to \infty} d_Y\big(f(x^{(n)}), f(x_0)\big) = 0
    \]
    and by Definition \ref{1.1.14} we have
    \[
        \forall \varepsilon \in \R^+, \exists\ N_1 \in \Z^+ : \forall n \geq N_1, d_Y\big(f(x^{(n)}), f(x_0)\big) < \frac{\varepsilon}{2}.
    \]
    Now we choose one pair of \(\varepsilon\) and \(N_1\).
    Since \((f^{(n)})_{n = 1}^\infty\) converges uniformly to \(f\) on \(X\) with respect to \(d_Y\), by Definition \ref{3.2.7} we have
    \begin{align*}
                 & \exists\ N_2 \in \Z^+ : \forall n \geq N_2, \forall x \in X, d_Y\big(f^{(n)}(x), f(x)\big) < \frac{\varepsilon}{2}                                                     \\
        \implies & \exists\ N_2 \in \Z^+ : \forall n \geq N_2, d_Y\big(f^{(n)}(x^{(n)}), f(x^{(n)})\big) < \frac{\varepsilon}{2}                                                          \\
        \implies & \exists\ N = \max(N_1, N_2) : \forall n \geq N,                                                                                                                        \\
                 & d_Y\big(f^{(n)}(x^{(n)}), f(x_0)\big) \leq d_Y\big(f^{(n)}(x^{(n)}), f(x^{(n)})\big) + d_Y\big(f(x^{(n)}), f(x_0)\big) < \frac{\varepsilon}{2} + \frac{\varepsilon}{2} \\
        \implies & \exists\ N = \max(N_1, N_2) : \forall n \geq N, d_Y\big(f^{(n)}(x^{(n)}), f(x_0)\big) < \varepsilon.
    \end{align*}
    Since \(\varepsilon\) is arbitrary, by Definition \ref{1.1.14} we know that
    \[
        \lim_{n \to \infty} d_Y\big(f^{(n)}(x^{(n)}), f(x_0)\big) = 0.
    \]
    Since \(x_0\) is arbitrary, we conclude that for any \(x_0 \in X\), if \((x^{(n)})_{n = 1}^\infty\) is a sequence in \(X\) such that
    \[
        \lim_{n \to \infty} d_X(x^{(n)}, x_0) = 0,
    \]
    then we have
    \[
        \lim_{n \to \infty} d_Y\big(f^{(n)}(x^{(n)}), f(x_0)\big) = 0.
    \]
\end{proof}

\begin{definition}[Bounded functions]\label{3.3.5}
    A function \(f : X \to Y\) from one metric space \((X, d_X)\) to another \((Y, d_Y)\) is \emph{bounded} if \(f(X)\) is a bounded set, i.e., there exists a ball \(B_{(Y, d_Y)}(y_0, R)\) in \(Y\) such that \(f(x) \in B_{(Y, d_Y)}(y_0, R)\) for all \(x \in X\).
\end{definition}

\begin{proposition}[Uniform limits preserve boundedness]\label{3.3.6}
    Let \((f^{(n)})_{n = 1}^\infty\) be a sequence of functions from one metric space \((X, d_X)\) to another \((Y, d_Y)\), and suppose that this sequence converges uniformly to another function \(f : X \to Y\).
    If the functions \(f^{(n)}\) are bounded on \(X\) for each \(n\), then the limiting function \(f\) is also bounded on \(X\).
\end{proposition}

\begin{proof}
    Since \(f^{(n)}\) is bounded in \((Y, d_Y)\) for each \(n \in \Z^+\), by Definition \ref{3.3.5} and Definition \ref{1.5.3} we have
    \begin{align*}
                 & \forall n \in \Z^+, \forall y \in Y, \exists\ \varepsilon \in \R^+ : f^{(n)}(X) \subseteq B_{(Y, d_Y)}(y, \varepsilon)          \\
        \implies & \forall n \in \Z^+, \forall y \in Y, \exists\ \varepsilon \in \R^+ : \forall x \in X, d_Y\big(f^{(n)}(x), y\big) < \varepsilon.
    \end{align*}
    Now we choose \(y\) and \(\varepsilon\) for each \(n \in \Z^+\) and we denote them as \(y^{(n)}\) and \(\varepsilon^{(n)}\).
    Since \((f^{(n)})_{n = 1}^\infty\) converges uniformly to \(f\) on \(X\) with respect to \(d_Y\), by Definition \ref{3.2.7} we have
    \begin{align*}
                 & \exists\ N \in \Z^+ : \forall n \geq N, \forall x \in X, d_Y\big(f^{(n)}(x), f(x)\big) < 1                               \\
                 & \exists\ N \in \Z^+ : \forall x \in X, d_Y\big(f^{(N)}(x), f(x)\big) < 1                                                 \\
        \implies & \exists\ N \in \Z^+ : \forall x \in X,                                                                                   \\
                 & d_Y\big(f(x), y^{(N)}\big) \leq d_Y\big(f(x), f^{(N)}(x)\big) + d_Y\big(f^{(N)}(x), y^{(N)}\big) < \varepsilon^{(N)} + 1 \\
        \implies & \exists\ N \in \Z^+ : \forall x \in X, d_Y\big(f(x), y^{(N)}\big) < \varepsilon^{(N)} + 1                                \\
        \implies & \exists\ N \in \Z^+ : f(X) \subseteq B_{(Y, d_Y)}(y^{(N)}, \varepsilon^{(N)} + 1).
    \end{align*}
    Since \(y^{(N)}\) is arbitrary, we have
    \[
        \forall y \in Y, \exists\ \varepsilon \in \R^+ : f(X) \subseteq B_{(Y, d_Y)}(y, \varepsilon)
    \]
    and by Definition \ref{1.5.3} and Definition \ref{3.3.5} \(f\) is bounded in \((Y, d_Y)\).
\end{proof}

\begin{remark}\label{3.3.7}
    The above propositions sound very reasonable, but one should caution that it only works if one assumes uniform convergence;
    pointwise convergence is not enough.
\end{remark}

\exercisesection

\begin{exercise}\label{ex 3.3.1}
    Prove Theorem \ref{3.3.1}.
    Explain briefly why your proof requires uniform convergence, and why pointwise convergence would not suffice.
\end{exercise}

\begin{proof}
    See Theorem \ref{3.3.1}.
    Without uniform convergence we cannot make \(f^{(n)}(x)\) and \(f(x)\) arbitrary close.
\end{proof}

\begin{exercise}\label{ex 3.3.2}
    Prove Proposition \ref{3.3.3}.
\end{exercise}

\begin{proof}
    See Proposition \ref{3.3.3}.
\end{proof}

\begin{exercise}\label{ex 3.3.3}
    Compare Proposition \ref{3.3.3} with Example 1.2.8 in Analysis I.
    Can you now explain why the interchange of limits in Example 1.2.8 in Analysis I led to a false statement, whereas the interchange of limits in Proposition \ref{3.3.3} is justified?
\end{exercise}

\begin{proof}
    By Exercise \ref{ex 3.2.2}(b) we know that \(f^{(n)}(x) = x^{(n)}\) does not converge uniformly to any function \(f : (-1, 1) \to \R\), thus the interchange of limits in Example 1.2.8 in Analysis I failed.
\end{proof}

\begin{exercise}\label{ex 3.3.4}
    Prove Proposition \ref{3.3.4}.
\end{exercise}

\begin{proof}
    See Proposition \ref{3.3.4}.
\end{proof}

\begin{exercise}\label{ex 3.3.5}
    Give an example to show that Proposition \ref{3.3.4} fails if the phrase ``converges uniformly'' is replaced by ``converges pointwise''.
\end{exercise}

\begin{proof}
    For each \(n \in \Z^+\), let \(f^{(n)} : [0, 1] \to \R\) be the function where \(f^{(n)}(x) = x^n\) for each \(x \in [0, 1]\).
    Let \(f : [0, 1] \to \R\) be the function where
    \[
        \forall x \in [0, 1], f(x) = \begin{cases}
            1 & \text{if } x = 1        \\
            0 & \text{if } x \in [0, 1)
        \end{cases}
    \]
    By Example 3.2.4 in Analysis II we know that \((f^{(n)})_{n = 1}^\infty\) converges pointwise to \(f\) on \(X\) with respect to \(d_{l^1}|_{\R \times \R}\).
    By Exercise \ref{ex 3.2.2}(b) we know that \(f^{(n)}(x) = x^{(n)}\) does not converge uniformly to \(f\) on \(X\) with respect to \(d_{l^1}|_{\R \times \R}\).
    Let \((x^{(n)})_{n = 1}^\infty\) be the sequence where \(x^{(n)} = (\frac{1}{2})^{\frac{1}{n}}\) for each \(n \in \Z^+\).
    Then we have
    \[
        \lim_{n \to \infty} x^{(n)} = 1 = f(1).
    \]
    But
    \[
        \lim_{n \to \infty} f^{(n)}(x^{(n)}) = \lim_{n \to \infty} \big((\frac{1}{2})^{\frac{1}{n}}\big)^n = \lim_{n \to \infty} \frac{1}{2} = \frac{1}{2} \neq 1.
    \]
    Thus Proposition \ref{3.3.4} fails when the phrase ``converges uniformly'' is replaced by ``converges pointwise''.
\end{proof}

\begin{exercise}\label{ex 3.3.6}
    Prove Proposition \ref{3.3.6}.
\end{exercise}

\begin{proof}
    See Proposition \ref{3.3.6}.
\end{proof}

\begin{exercise}\label{ex 3.3.7}
    Give an example to show that Proposition \ref{3.3.6} fails if the phrase ``converges uniformly'' is replaced by ``converges pointwise''.
\end{exercise}

\begin{proof}
    By Exercise \ref{ex 3.2.2}(c) we know that \(g\) is unbounded since
    \[
        \lim_{x \to -1 ; x \in (-1, 1)} g(x) = \lim_{x \to -1 ; x \in (-1, 1)} \frac{x}{1 - x} = \lim_{x \to -1 ; x \in (-1, 1)} \frac{1}{1 - x} - 1 = +\infty.
    \]
\end{proof}

\begin{exercise}\label{ex 3.3.8}
    Let \((X, d)\) be a metric space, and for every positive integer \(n\), let \(f_n : X \to \R\) and \(g_n : X \to \R\) be functions.
    Suppose that \((f_n)_{n = 1}^\infty\) converges uniformly to another function \(f : X \to \R\), and that \((g_n)_{n = 1}^\infty\) converges uniformly to another function \(g : X \to \R\).
    Suppose also that the functions \((f_n)_{n = 1}^\infty\) and \((g_n)_{n = 1}^\infty\) are uniformly bounded, i.e., there exists an \(M > 0\) such that \(\abs{f_n(x)} \leq M\) and \(\abs{g_n(x)} \leq M\) for all \(n \geq 1\) and \(x \in X\).
    Prove that the functions \(f_n g_n : X \to \R\) converge uniformly to \(fg : X \to \R\).
\end{exercise}

\begin{proof}
    Let \(d_1 = d_{l^1}|_{\R \times \R}\).
    Since \((f_n)_{n = 1}^\infty\) and \((g_n)_{n = 1}^\infty\) are uniformly bounded, by Definition \ref{3.3.5} we know that \(f_n\) and \(g_n\) are bounded in \((\R, d_1)\) for each \(n \in \Z^+\).
    By Proposition \ref{3.3.6} we know that \(f\) and \(g\) are bounded in \((\R, d_1)\), i.e.,
    \[
        \exists\ U \in \R^+ : \forall x \in X, \begin{cases}
            \abs{f(x)} < U \\
            \abs{g(x)} < U
        \end{cases}
    \]
    Since \((f_n)_{n = 1}^\infty\) and \((g_n)_{n = 1}^\infty\) converge uniformly to \(f\) on \(X\) with respect to \(d_1\), by Definition \ref{3.2.7} we have
    \begin{align*}
         & \forall \varepsilon \in \R^+, \exists\ N_1 \in \Z^+ : \forall n \geq N_1, \forall x \in X, \abs{f_n(x) - f(x)} < \frac{\varepsilon}{2M}; \\
         & \forall \varepsilon \in \R^+, \exists\ N_2 \in \Z^+ : \forall n \geq N_2, \forall x \in X, \abs{g_n(x) - g(x)} < \frac{\varepsilon}{2U}.
    \end{align*}
    Now we fix one \(\varepsilon\) and its corresponding \(N_1, N_2\).
    Let \(N = \max(N_1, N_2)\).
    Then we have
    \begin{align*}
        \forall n \geq N, \forall x \in X, & \abs{f_n(x) g_n(x) - f(x) g(x)}                                        \\
                                           & = \abs{f_n(x) g_n(x) - f(x) g_n(x) + f(x) g_n(x) - f(x) g(x)}          \\
                                           & \leq \abs{f_n(x) g_n(x) - f(x) g_n(x)} + \abs{f(x) g_n(x) - f(x) g(x)} \\
                                           & = \abs{f_n(x) - f(x)} \abs{g_n(x)} + \abs{f(x)} \abs{g_n(x) - g(x)}    \\
                                           & < \frac{\varepsilon}{2M} M + U \frac{\varepsilon}{2U}                  \\
                                           & = \varepsilon.
    \end{align*}
    Since \(\varepsilon\) is arbitrary, we have
    \[
        \forall \varepsilon \in \R^+, \exists\ N \in \Z^+ : \forall n \geq N, \forall x \in X, \abs{f_n(x) g_n(x) - f(x) g(x)} < \varepsilon
    \]
    and by Definition \ref{3.2.7} \((f_n g_n)_{n = 1}^\infty\) converges uniformly to \(fg\) on \(X\) with respect to \(d_1\).
\end{proof}