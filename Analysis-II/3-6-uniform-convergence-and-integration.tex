\section{Uniform convergence and integration}\label{sec 3.6}

\begin{theorem}\label{3.6.1}
    Let \([a, b]\) be an interval, and for each integer \(n \geq 1\), let \(f^{(n)} : [a, b] \to \mathbf{R}\) be a Riemann-integrable function.
    Suppose \(f^{(n)}\) converges uniformly on \([a, b]\) to a function \(f : [a, b] \to \mathbf{R}\).
    Then \(f\) is also Riemann integrable, and
    \[
        \lim_{n \to \infty} \int_{[a, b]} f^{(n)} = \int_{[a, b]} f.
    \]
\end{theorem}

\begin{proof}
    We first show that \(f\) is Riemann integrable on \([a, b]\).
    This is the same as showing that the upper and lower Riemann integrals of \(f\) match:
    \(\underline{\int}_{[a, b]} f = \overline{\int}_{[a, b]} f\).

    Let \(\varepsilon > 0\).
    Since \(f^{(n)}\) converges uniformly to \(f\), we see that there exists an \(N > 0\) such that \(\abs*{f^{(n)}(x) - f(x)} < \varepsilon\) for all \(n > N\) and \(x \in [a, b]\).
    In particular we have
    \[
        f^{(n)}(x) - \varepsilon < f(x) < f^{(n)}(x) + \varepsilon
    \]
    for all \(x \in [a, b]\).
    Integrating this on \([a, b]\) we obtain
    \[
        \underline{\int}_{[a, b]} (f^{(n)} - \varepsilon) \leq \underline{\int}_{[a, b]} f \leq \overline{\int}_{[a, b]} f \leq \overline{\int}_{[a, b]} (f^{(n)} + \varepsilon).
    \]
    Since \(f^{(n)}\) is assumed to be Riemann integrable, we thus see
    \[
        \Bigg(\int_{[a, b]} f^{(n)}\Bigg) - \varepsilon (b - a) \leq \underline{\int}_{[a, b]} f \leq \overline{\int}_{[a, b]} f \leq \Bigg(\int_{[a, b]} f^{(n)}\Bigg) + \varepsilon (b - a).
    \]
    In particular, we see that
    \[
        0 \leq \overline{\int}_{[a, b]} f - \underline{\int}_{[a, b]} f \leq 2 \varepsilon (b - a).
    \]
    Since this is true for every \(\varepsilon > 0\), we obtain \(\underline{\int}_{[a, b]} f = \overline{\int}_{[a, b]} f\) as desired.

    The above argument also shows that for every \(\varepsilon > 0\) there exists an \(N > 0\) such that
    \[
        \abs*{\int_{[a, b]} f^{(n)} - \int_{[a, b]} f} \leq \varepsilon (b - a)
    \]
    for all \(n \geq N\).
    Since \(\varepsilon\) is arbitrary, we see that \(\int_{[a, b]} f^{(n)}\) converges to \(\int_{[a, b]} f\) as desired.
\end{proof}

\begin{note}
    To rephrase Theorem \ref{3.6.1}:
    we can rearrange limits and integrals (on compact intervals \([a, b]\)),
    \[
        \lim_{n \to \infty} \int_{[a, b]} f^{(n)} = \int_{[a, b]} \lim_{n \to \infty} f^{(n)},
    \]
    \emph{provided that} the convergence is uniform.
\end{note}

\begin{corollary}\label{3.6.2}
    Let \([a, b]\) be an interval, and let \((f^{(n)})_{n = 1}^\infty\) be a sequence of Riemann integrable functions on \([a, b]\) such that the series \(\sum_{n = 1}^\infty f^{(n)}\) is uniformly convergent.
    Then we have
    \[
        \sum_{n = 1}^\infty \int_{[a, b]} f^{(n)} = \int_{[a, b]} \sum_{n = 1}^\infty f^{(n)}.
    \]
\end{corollary}

\begin{proof}
    By Theorem 11.4.1(a) in Analysis I we know that
    \[
        \forall\ N \in \mathbf{Z}^+, \int_{[a, b]} \sum_{n = 1}^N f^{(n)} = \sum_{n = 1}^N \int_{[a, b]} f^{(n)}.
    \]
    Let \(f : [a, b] \to \mathbf{R}\) be the function such that \(\sum_{n = 1}^\infty f^{(n)}\) converges uniformly to \(f\) on \([a, b]\) with respect to \(d_{l^1}|_{\mathbf{R} \times \mathbf{R}}\).
    By Theorem \ref{3.6.1} we have
    \[
        \sum_{n = 1}^\infty \int_{[a, b]} f^{(n)} = \lim_{N \to \infty} \sum_{n = 1}^N \int_{[a, b]} f^{(n)} = \lim_{N \to \infty} \int_{[a, b]} \sum_{n = 1}^N f^{(n)} = \int_{[a, b]} f = \int_{[a, b]} \sum_{n = 1}^\infty f^{(n)}.
    \]
\end{proof}

\begin{note}
    Corollary \ref{3.6.2} works particularly well in conjunction with the Weierstrass \(M\)-test
    (Theorem \ref{3.5.7}).
\end{note}

\exercisesection

\begin{exercise}\label{ex 3.6.1}
    Use Theorem \ref{3.6.1} to prove Corollary \ref{3.6.2}.
\end{exercise}

\begin{proof}
    See Corollary \ref{3.6.2}.
\end{proof}