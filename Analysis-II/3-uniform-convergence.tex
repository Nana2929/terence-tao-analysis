\chapter{Uniform convergence}\label{ch 3}

\begin{note}
    It turns out that there are several different concepts of convergence of functions;
    here we describe the two most important ones, \emph{pointwise convergence} and \emph{uniform convergence}.
    (There are other types of convergence for functions, such as \(L^1\) convergence, \(L^2\) convergence, convergence in measure, almost everywhere convergence, and so forth, but these are beyond the scope of this text.)
    The two notions are related, but not identical;
    the relationship between the two is somewhat analogous to the relationship between continuity and uniform continuity.
\end{note}

\section{Limiting values of functions}\label{sec 3.1}

\begin{definition}[Limiting value of a function]\label{3.1.1}
    Let \((X, d_X)\) and \((Y, d_Y)\) be metric spaces, let \(E\) be a subset of \(X\), and let \(f : E \to Y\) be a function.
    If \(x_0 \in X\) is an adherent point of \(E\), and \(L \in Y\), we say that \emph{\(f(x)\) converges to \(L\) in \(Y\) as \(x\) converges to \(x_0\) in \(E\)}, or write \(\lim_{x \to x_0 ; x \in E} f(x) = L\), if for every \(\varepsilon > 0\) there exists a \(\delta > 0\) such that \(d_Y\big(f(x), L\big) < \varepsilon\) for all \(x \in E\) such that \(d_X(x, x_0) < \delta\).
\end{definition}

\begin{remark}\label{3.1.2}
    Some authors exclude the case \(x = x_0\) from the above definition, thus requiring \(0 < d_X(x, x_0) < \delta\).
    In our current notation, this would correspond to removing \(x_0\) from \(E\), thus one would consider
    \[
        \lim_{x \to x_0 ; x \in E \setminus \{x_0\}} f(x)
    \]
    instead of
    \[
        \lim_{x \to x_0 ; x \in E} f(x).
    \]
\end{remark}

\begin{note}
    Comparing this with Definition \ref{2.1.1}, we see that \(f\) is continuous at \(x_0\) if and only if
    \[
        \lim_{x \to x_0 ; x \in X} f(x) = f(x_0).
    \]
    Thus \(f\) is continuous on \(X\) iff we have
    \[
        \lim_{x \to x_0 ; x \in X} f(x) = f(x_0) \text{ for all } x_0 \in X.
    \]
\end{note}

\setcounter{theorem}{3}
\begin{remark}\label{3.1.4}
    Often we shall omit the condition \(x \in X\), and abbreviate
    \[
        \lim_{x \to x_0 ; x \in X} f(x)
    \]
    as simply
    \[
        \lim_{x \to x_0} f(x)
    \]
    when it is clear what space \(x\) will range in.
\end{remark}

\begin{proposition}\label{3.1.5}
    Let \((X, d_X)\) and \((Y, d_Y)\) be metric spaces, let \(E\) be a subset of \(X\), and let \(f : E \to Y\) be a function.
    Let \(x_0 \in X\) be an adherent point of \(E\) and \(L \in Y\).
    Then the following four statements are logically equivalent:
    \begin{enumerate}
        \item \(\lim_{x \to x_0 ; x \in E} f(x) = L\).
        \item For every sequence \((x^{(n)})_{n = 1}^\infty\) in \(E\) which converges to \(x_0\) with respect to the metric \(d_X\), the sequence \(\big(f(x^{(n)})\big)_{n = 1}^\infty\) converges to \(L\) with respect to the metric \(d_Y\).
        \item For every open set \(V \subseteq Y\) which contains \(L\), there exists an open set \(U \subseteq X\) containing \(x_0\) such that \(f(U \cap E) \subseteq V\).
        \item If one defines the function \(g : E \cup \{x_0\} \to Y\) by defining \(g(x_0) \coloneqq L\), and \(g(x) \coloneqq f(x)\) for \(x \in E \setminus \{x_0\}\), then \(g\) is continuous at \(x_0\).
              Furthermore, if \(x_0 \in E\), then \(f(x_0) = L\).
    \end{enumerate}
\end{proposition}

\begin{proof}
    We first show that statement (a) implies statement (b).
    Suppose that
    \[
        d_Y - \lim_{x \to x_0 ; x \in E} f(x) = L.
    \]
    By Definition \ref{3.1.1} we have
    \[
        \forall\ \varepsilon \in \mathbf{R}^+, \exists\ \delta \in \mathbf{R}^+ : \Big(\forall\ x \in E, d_X(x, x_0) < \delta \implies d_Y\big(f(x), L\big) < \varepsilon\Big).
    \]
    Let \((x^{(n)})_{n = 1}^\infty\) be a sequence in \(E\) such that \(\lim_{n \to \infty} d_X(x^{(n)}, x_0) = 0\).
    By Definition \ref{1.1.14} we have
    \[
        \forall\ \delta \in \mathbf{R}^+, \exists\ N \in \mathbf{Z}^+ : \forall\ n \geq N, d_X(x^{(n)}, x_0) < \delta.
    \]
    Since \((x^{(n)})_{n = 1}^\infty\) is in \(E\), we have
    \[
        \forall\ \varepsilon \in \mathbf{R}^+, \exists\ \delta \in \mathbf{R}^+ : \begin{cases}
            \exists\ N \in \mathbf{Z}^+ : \forall\ n \geq N, d_X(x^{(n)}, x_0) < \delta \\
            d_X(x^{(n)}, x_0) < \delta \implies d_Y\big(f(x^{(n)}), L\big) < \varepsilon
        \end{cases}
    \]
    and
    \[
        \forall\ \varepsilon \in \mathbf{R}^+, \exists\ N \in \mathbf{Z}^+ : \forall\ n \geq N, d_Y\big(f(x^{(n)}, L)\big) < \varepsilon.
    \]
    By Definition \ref{1.1.14} we have \(\lim_{n \to \infty} d_Y\big(f(x^{(n)}), L\big) = 0\).
    Since \((x^{(n)})_{n = 1}^\infty\) is arbitrary, we conclude that (a) implies (b).

    Next we show that statement (b) implies statement (a).
    Suppose that if \((x^{(n)})_{n = 1}^\infty\) is a sequence in \(X\) such that \(\lim_{n \to \infty} d_X(x^{(n)}, x_0) = 0\), then \(\lim_{n \to \infty} d_Y\big(f(x), L\big) = 0\).
    Suppose for sake of contradiction that
    \[
        d_Y - \lim_{x \to x_0 ; x \in X} f(x) \neq L.
    \]
    Then by Definition \ref{3.1.1} we have
    \[
        \exists\ \varepsilon \in \mathbf{R}^+ : \forall\ \delta \in \mathbf{R}^+, \exists\ x \in X : \begin{cases}
            d_X(x, x_0) < \delta \\
            d_Y\big(f(x), L\big) \geq \varepsilon
        \end{cases}
    \]
    Thus we can choose one sequence \((x^{(n)})_{n = 1}^\infty\) which satsifies
    \[
        \forall\ n \in \mathbf{Z}^+, \begin{cases}
            d_X(x^{(n)}, x_0) < \frac{1}{n} \\
            d_Y\big(f(x^{(n)}), L\big) \geq \varepsilon
        \end{cases}
    \]
    By squeeze test we have \(\lim_{n \to \infty} d_X(x^{(n)}, x_0) = 0\).
    But by hypothesis we know that \(\lim_{n \to \infty} d_Y\big(f(x^{(n)}), L\big) = 0\), which means
    \[
        \exists\ N \in \mathbf{Z}^+ : \forall\ n \geq N, d_Y\big(f(x^{(n)}), L\big) < \varepsilon,
    \]
    a contradiction.
    Thus we have
    \[
        d_Y - \lim_{x \to x_0 ; x \in X} f(x) = L
    \]
    and we conclude that statements (a)(b) are equivalent.

    Next we show that statement (a) implies statement (c).
    Suppose that
    \[
        d_Y - \lim_{x \to x_0 ; x \in E} f(x) = L.
    \]
    By Definition \ref{3.1.1} we have
    \begin{align*}
                 & \forall\ \varepsilon \in \mathbf{R}^+, \exists\ \delta \in \mathbf{R}^+ : \Big(\forall\ x \in E, d_X(x, x_0) < \delta \implies d_Y\big(f(x), L\big) < \varepsilon\Big)  \\
        \implies & \forall\ \varepsilon \in \mathbf{R}^+, \exists\ \delta \in \mathbf{R}^+ : \Big(x \in B_{(X, d_X)}(x_0, \delta) \cap E \implies d_Y\big(f(x), L\big) < \varepsilon\Big).
    \end{align*}
    Let \(V\) be an open set in \((Y, d_Y)\) such that \(L \in V\).
    Then we have
    \begin{align*}
                 & V = \text{int}_{(Y, d_Y)}(V)                                                     & \text{(by Proposition \ref{1.2.15}(a))} \\
        \implies & \exists\ \varepsilon \in \mathbf{R}^+ : B_{(Y, d_Y)}(L, \varepsilon) \subseteq V & \text{(by Definition \ref{1.2.5})}      \\
        \implies & \exists\ \delta \in \mathbf{R}^+ :                                                                                         \\
                 & \begin{cases}
            x \in B_{(X, d_X)}(x_0, \delta) \cap E \implies d_Y\big(f(x), L\big) < \varepsilon \\
            f\big(B_{(X, d_X)}(x_0, \delta) \cap E\big) \subseteq B_{(Y, d_Y)}\big(L, \varepsilon\big) \subseteq V
        \end{cases}
    \end{align*}
    and by Proposition \ref{1.2.15}(c) we know that \(B_{(X, d_X)}(x_0, \delta)\) is open in \((X, d_X)\).
    Since \(V\) is arbitrary, we conclude that statement (a) implies statement (c).

    Next we show that statement (c) implies statement (a).
    Suppose that
    \[
        \forall\ V \subseteq Y, \begin{cases}
            L \in V \\
            V \text{ is open in } (Y, d_Y)
        \end{cases} \implies \exists\ U \subseteq X : \begin{cases}
            x_0 \in U                      \\
            U \text{ is open in } (X, d_X) \\
            f(U \cap E) \subseteq V
        \end{cases}
    \]
    Let \(\varepsilon \in \mathbf{R}^+\).
    By Proposition \ref{1.2.15}(c) we know that \(B_{(Y, d_Y)}(L, \varepsilon)\) is open in \((Y, d_Y)\).
    By hypothesis we know that there exists some \(U \subseteq X\) such that
    \[
        \begin{cases}
            x_0 \in U                      \\
            U \text{ is open in } (X, d_X) \\
            f(U \cap E) \subseteq B_{(Y, d_Y)}(L, \varepsilon)
        \end{cases}
    \]
    Then we have
    \begin{align*}
                 & \begin{cases}
            x_0 \in U \\
            U = \text{int}_{(X, d_X)}(U)
        \end{cases}                                                                               & \text{(by Proposition \ref{1.2.15}(a))} \\
        \implies & \exists\ \delta \in \mathbf{R}^+ : B_{(X, d_X)}(x_0, \delta) \subseteq U                                 & \text{(by Definition \ref{1.2.5})}      \\
        \implies & \exists\ \delta \in \mathbf{R}^+ : B_{(X, d_X)}(x_0, \delta) \cap E \subseteq U \cap E                                                             \\
        \implies & \exists\ \delta \in \mathbf{R}^+ :                                                                                                                 \\
                 & f\big(B_{(X, d_X)}(x_0, \delta) \cap E\big) \subseteq f(U \cap E) \subseteq B_{(Y, d_Y)}(L, \varepsilon)                                           \\
        \implies & \exists\ \delta \in \mathbf{R}^+ :                                                                                                                 \\
                 & \Big(\forall\ x \in E, d_X(x, x_0) < \delta \implies d_Y\big(f(x), L\big) < \varepsilon\Big).
    \end{align*}
    Since \(\varepsilon\) is arbitrary, by Definition \ref{3.1.1} we have
    \[
        d_Y - \lim_{x \to x_0 ; x \in E} f(x) = L
    \]
    and we conclude that statements (a)(c) are equivalent.

    Next we show that statement (a) implies statement (d).
    Suppose that
    \[
        d_Y - \lim_{x \to x_0 ; x \in E} f(x) = L.
    \]
    Then by Definition \ref{3.1.1} we have
    \[
        \forall\ \varepsilon \in \mathbf{R}^+, \exists\ \delta \in \mathbf{R}^+ : \Big(\forall\ x \in E, d_X(x, x_0) < \delta \implies d_Y\big(f(x), L\big) < \varepsilon\Big).
    \]
    Let \(g : E \cup \{x_0\} \to Y\) be a function where
    \[
        \forall\ x \in E \cup \{x_0\}, g(x) = \begin{cases}
            L    & \text{if } x = x_0    \\
            f(x) & \text{if } x \neq x_0
        \end{cases}
    \]
    Then we have
    \begin{align*}
                 & \forall\ \varepsilon \in \mathbf{R}^+, \exists\ \delta \in \mathbf{R}^+ :                                       \\
                 & \Big(\forall\ x \in E, d_X(x, x_0) < \delta \implies d_Y\big(f(x), L\big) < \varepsilon\Big)                    \\
        \implies & \forall\ \varepsilon \in \mathbf{R}^+, \exists\ \delta \in \mathbf{R}^+ :                                       \\
                 & \Big(\forall\ x \in E \cup \{x_0\}, d_X(x, x_0) < \delta \implies d_Y\big(g(x), g(x_0)\big) < \varepsilon\Big).
    \end{align*}
    Thus by Definition \ref{2.1.1} \(g\) is continuous at \(x_0\) from \((E \cup \{x_0\}, d_X|_{(E \cup \{x_0\}) \times (E \cup \{x_0\})})\) to \((Y, d_Y)\).

    Now suppose that \(x_0 \in E\).
    We claim that \(f(x_0) = L\).
    Suppose for sake of contradiction that \(f(x_0) \neq L\).
    Then by Definition \ref{1.1.2}(b) we have \(d_Y\big(f(x_0), L\big) > 0\).
    Let \(r = d_Y\big(f(x_0), L\big)\).
    By Definition \ref{3.1.1} we have
    \[
        \exists\ \delta \in \mathbf{R}^+ : \forall\ x \in E, d_X(x, x_0) < \delta \implies d_Y\big(f(x), L\big) < r.
    \]
    Since \(x_0 \in E\), we have \(d_X(x_0, x_0) = 0 < \delta\).
    But then we have \(d_Y\big(f(x_0), L\big) < r = d_Y\big(f(x_0), L\big)\), a contradiction.
    Thus we have \(f(x_0) = L\).

    Finally we show that statement (d) implies statement (a).
    Suppose that \(g : E \cup \{x_0\} \to Y\) is a function where
    \[
        \forall\ x \in E \cup \{x_0\}, g(x) = \begin{cases}
            L    & \text{if } x = x_0    \\
            f(x) & \text{if } x \neq x_0
        \end{cases}
    \]
    and \(g\) is continuous from \((E \cup \{x_0\}, d_X|_{(E \cup \{x_0\}) \times (E \cup \{x_0\})})\) to \((Y, d_Y)\).
    Then by Definition \ref{2.1.1} we have
    \begin{align*}
                 & \forall\ \varepsilon \in \mathbf{R}^+, \exists\ \delta \in \mathbf{R}^+ :                                      \\
                 & \Big(\forall\ x \in E \cup \{x_0\}, d_X(x, x_0) < \delta \implies d_Y\big(g(x), g(x_0)\big) < \varepsilon\Big) \\
        \implies & \forall\ \varepsilon \in \mathbf{R}^+, \exists\ \delta \in \mathbf{R}^+ :                                      \\
                 & \Big(\forall\ x \in E, d_X(x, x_0) < \delta \implies d_Y\big(f(x), L\big) < \varepsilon\Big).
    \end{align*}
    By Definition \ref{3.1.1} this means
    \[
        d_Y - \lim_{x \to x_0 ; x \in E} f(x) = L.
    \]
    We conclude that statements (a)(b)(c)(d) are all equivalent.
\end{proof}

\begin{remark}\label{3.1.6}
    Observe from Proposition \ref{3.1.5}(b) and Proposition \ref{1.1.20} that a function \(f(x)\) can converge to at most one limit \(L\) as \(x\) converges to \(x_0\).
    In other words, if the limit
    \[
        \lim_{x \to x_0 ; x \in E} f(x)
    \]
    exists at all, then it can only take at most one value.
\end{remark}

\begin{remark}\label{3.1.7}
    The requirement that \(x_0\) be an adherent point of \(E\) is necessary for the concept of limiting value to be useful, otherwise \(x_0\) will lie in the exterior of \(E\), the notion that \(f(x)\) converges to \(L\) as \(x\) converges to \(x_0\) in \(E\) is vacuous
    (for \(\delta\) sufficiently small, there are no points \(x \in E\) so that \(d(x, x_0) < \delta\)).
\end{remark}

\begin{remark}\label{3.1.8}
    Strictly speaking, we should write
    \[
        d_Y - \lim_{x \to x_0 ; x \in E} f(x) \text{ instead of } \lim_{x \to x_0 ; x \in E} f(x),
    \]
    since the convergence depends on the metric \(d_Y\).
    However in practice it will be obvious what the metric \(d_Y\) is and so we will omit the \(d_Y -\) prefix from the notation.
\end{remark}

\exercisesection

\begin{exercise}\label{ex 3.1.1}
    Let \((X, d_X)\) and \((Y, d_Y)\) be metric spaces, let \(E\) be a subset of \(X\), let \(f : E \to Y\) be a function, and let \(x_0\) be an element of \(E\).
    Assume that \(x_0\) is an adherent point of \(E \setminus \{x_0\}\)
    (or equivalently, that \(x_0\) is not an \emph{isolated point} of \(E\)).
    Show that the limit \(\lim_{x \to x_0 ; x \in E} f(x)\) exists if and only if the limit \(\lim_{x \to x_0 ; x \in E \setminus \{x_0\}} f(x)\) exists and is equal to \(f(x_0)\).
    Also, show that if the limit \(\lim_{x \to x_0 ; x \in E} f(x)\) exists at all, then it must equal \(f(x_0)\).
\end{exercise}

\begin{proof}
    Let \(L \in Y\).
    By Definition \ref{1.1.2}(a) we know that
    \[
        \forall\ \varepsilon \in \mathbf{R}^+, d_Y\big(f(x_0), L\big) < \varepsilon \iff L = f(x_0).
    \]
    Thus we have
    \begin{align*}
             & d_Y - \lim_{x \to x_0 ; x \in E \setminus \{x_0\}} f(x) = f(x_0)                                                                                         \\
        \iff & \forall\ \varepsilon \in \mathbf{R}^+, \exists\ \delta \in \mathbf{R}^+ :                                                                                \\
             & \Big(\forall\ x \in E \setminus \{x_0\}, d_X(x, x_0) < \delta \implies d_Y\big(f(x), f(x_0)\big) < \varepsilon\Big) & \text{(by Definition \ref{3.1.1})} \\
        \iff & \forall\ \varepsilon \in \mathbf{R}^+, \exists\ \delta \in \mathbf{R}^+ :                                                                                \\
             & \Big(\forall\ x \in E, d_X(x, x_0) < \delta \implies d_Y\big(f(x), f(x_0)\big) < \varepsilon\Big)                   & (E \setminus \{x_0\} \subseteq E)  \\
        \iff & d_Y - \lim_{x \to x_0 ; x \in E} f(x) = f(x_0).                                                                     & \text{(by Definition \ref{3.1.1})}
    \end{align*}
\end{proof}

\begin{exercise}\label{ex 3.1.2}
    Prove Proposition \ref{3.1.5}.
\end{exercise}

\begin{proof}
    See Proposition \ref{3.1.5}.
\end{proof}

\begin{exercise}\label{ex 3.1.3}
    Use Proposition \ref{3.1.5}(c) to define a notion of a limiting value of a function \(f : E \to Y\) from one topological space \((X, \mathcal{F}_X)\) to another \((Y, \mathcal{F}_Y)\) where \(E \subseteq X\).
    If \(X\) is a topological space and \(Y\) is a Hausdorff topological space (see Exercise \ref{ex 2.5.4}), prove the equivalence of Proposition \ref{3.1.5}(c) and \ref{3.1.5}(d) in this setting, as well as an analogue of Remark \ref{3.1.6}.
    What happens to these statements of \(Y\) is not Hausdorff?
\end{exercise}

\begin{proof}
    Let \((X, \mathcal{F}_X)\), \((Y, \mathcal{F}_Y)\) be topological spaces, let \(E \subseteq X\), let \(f : E \to Y\) be a function, let \(x_0 \in \overline{E}_{(X, \mathcal{F}_X)}\), and let \(L \in Y\).
    We say that \(f(x)\) converges to \(L\) in \(Y\) as \(x\) converges to \(x_0\) in \(E\) iff
    \[
        \forall\ V \in \mathcal{F}_Y, L \in V \implies \exists\ U \in \mathcal{F}_X : \begin{cases}
            x_0 \in U \\
            f(U \cap E) \subseteq V
        \end{cases}
    \]
    We want to show that if \((Y, \mathcal{F}_Y)\) is Hausdorff, then the definition above is equivalent to the follow:
    If \(g : E \cup \{x_0\} \to Y\) is a function such that
    \[
        \forall\ x \in E \cup \{x_0\}, g(x) = \begin{cases}
            L    & \text{if } x = x_0    \\
            f(x) & \text{if } x \neq x_0
        \end{cases}
    \]
    and \((E \cup \{x_0\}, \mathcal{F}_{E \cup \{x_0\}})\) is a topological subspace induced by \((X, \mathcal{F}_X)\), then \(g\) is continuous at \(x_0\) from \((E \cup \{x_0\}, \mathcal{F}_{E \cup \{x_0\}})\) to \((Y, \mathcal{F}_Y)\).

    First suppose that \(f(x)\) converges to \(L\) in \(Y\) as \(x\) converges to \(x_0\) in \(E\).
    Let \(g\) be the function in the definition and let \(V \in \mathcal{F}_Y\) such that \(L \in V\).
    By hypothesis we know that
    \[
        \exists\ U \in \mathcal{F}_X : \begin{cases}
            x_0 \in U \\
            f(U \cap E) \subseteq V
        \end{cases}
    \]
    Then by Definition \ref{2.5.7} we have \(U \cap (E \cup \{x_0\}) \in \mathcal{F}_{E \cup \{x_0\}}\) and
    \begin{align*}
        g\big(U \cap (E \cup \{x_0\})\big) & = g\big((U \cap E) \cup \{x_0\}\big)                      \\
                                           & = g\big((U \cap E) \setminus \{x_0\}\big) \cup g(\{x_0\}) \\
                                           & = f\big((U \cap E) \setminus \{x_0\}\big) \cup \{L\}      \\
                                           & \subseteq f(U \cap E) \cup \{L\}                          \\
                                           & \subseteq V.
    \end{align*}
    Since \(V\) is arbitrary, by Definition \ref{2.5.8} we know that \(g\) is continuous at \(x_0\) from \((E \cup \{x_0\}, \mathcal{F}_{E \cup \{x_0\}})\) to \((Y, \mathcal{F}_Y)\).

    Next suppose that \(x_0 \in E\) and \(f(x)\) converges to \(L\) in \(Y\) as \(x\) converges to \(x_0\) in \(E\).
    We want to show that \(f(x_0) = L\).
    Suppose for sake of contradiction that \(f(x_0) \neq L\).
    Since \((Y, \mathcal{F}_Y)\) is Hausdorff, by Exercise \ref{ex 2.5.4} we know that
    \[
        \exists\ V, W \in \mathcal{F}_Y : \begin{cases}
            L \in V      \\
            f(x_0) \in W \\
            V \cap W = \emptyset
        \end{cases}
    \]
    By definition we have
    \[
        \exists\ U_V, U_W \in \mathcal{F}_X : \begin{cases}
            x_0 \in U_V               \\
            x_0 \in U_W               \\
            f(U_V \cap E) \subseteq V \\
            f(U_W \cap E) \subseteq W
        \end{cases}
    \]
    By Definition \ref{2.5.1} we know that \(U_V \cap U_W \in \mathcal{F}_X\).
    But then we have
    \[
        \begin{cases}
            x_0 \in U_V \cap U_W                                       \\
            f(U_V \cap U_W \cap E) \subseteq f(U_V \cap E) \subseteq V \\
            f(U_V \cap U_W \cap E) \subseteq f(U_W \cap E) \subseteq W
        \end{cases}
    \]
    which means \(V \cap W \neq \emptyset\), a contradiction.
    Thus we have \(f(x_0) = L\).

    Now suppose that \(g\) is the function in the definition such that \(g\) is continuous at \(x_0\) from \((E \cup \{x_0\}, \mathcal{F}_{E \cup \{x_0\}})\) to \((Y, \mathcal{F}_Y)\).
    Also suppose that if \(x_0 \in E\), then \(f(x_0) = L\).
    Let \(V \in \mathcal{F}_Y\) such that \(g(x_0) = L \in V\).
    By Definition \ref{2.5.8} we know that
    \[
        \exists\ U \in \mathcal{F}_{E \cup \{x_0\}} : \begin{cases}
            x_0 \in U \\
            g(U) \subseteq V
        \end{cases}
    \]
    By Definition \ref{2.5.7} we know that
    \[
        \exists\ U_X \in \mathcal{F}_X : U_X \cap (E \cup \{x_0\}) = U.
    \]
    Since \(x_0 \in U\), we know that \(x_0 \in U_X\).
    Thus we have
    \begin{align*}
        f(U_X \cap E) & = f\big((U_X \cap E) \setminus \{x_0\}\big) \cup f(E \cap \{x_0\})  & (f(E \cap \{x_0\}) = \emptyset \iff x_0 \notin E) \\
                      & \subseteq g\big((U_X \cap E) \setminus \{x_0\}\big) \cup g(\{x_0\}) & (x_0 \in E \iff f(x_0) = L = g(x_0))              \\
                      & = g\big(U_X \cap (E \cup \{x_0\})\big)                                                                                  \\
                      & = g(U)                                                                                                                  \\
                      & \subseteq V.
    \end{align*}
    Since \(V\) is arbitrary, we conclude that \(f(x)\) converges to \(L\) in \(Y\) as \(x\) converges to \(x_0\) in \(E\).

    If \((Y, \mathcal{F}_Y)\) is not Hausdorff, then \(x_0 \in E\) may not implies \(f(x_0) = L\).
\end{proof}

\begin{exercise}\label{ex 3.1.4}
    Recall from Exercise \ref{ex 2.5.5} that the extended real line \(\mathbf{R}^*\) comes with a standard topology (the order topology).
    We view the natural numbers \(\mathbf{N}\) as a subspace of this topological space, and \(+\infty\) as an adherent point of \(\mathbf{N}\) in \(\mathbf{R}^*\).
    Let \((a_n)_{n = 0}^\infty\) be a sequence taking values in a topological space \((Y, \mathcal{F}_Y)\), and let \(L \in Y\).
    Show that \(\lim_{n \to +\infty ; n \in \mathbf{N}} a_n = L\) (in the sense of Exercise \ref{ex 3.1.3}) if and only if \(\lim_{n \to \infty} a_n = L\) (in the sense of Definition \ref{2.5.4}).
    This shows that the notions of limiting values of a sequence, and limiting values of a function, are compatible.
\end{exercise}

\begin{exercise}\label{ex 3.1.5}
    Let \((X, d_X)\), \((Y, d_Y)\), \((Z, d Z)\) be metric spaces, and let \(x_0 \in X\), \(y_0 \in Y\), \(z_0 \in Z\).
    Let \(f : E \to Y\) and \(g : Y \to Z\) be functions, and let \(E\) be a set.
    If we have \(\lim_{x \to x_0 ; x \in E} f(x) = y_0\) and \(\lim_{y \to y_0 ; y \in f(E)} g(y) = z_0\), conclude that \(\lim_{x \to x_0 ; x \in E} g \circ f(x) = z_0\).
\end{exercise}

\begin{exercise}\label{ex 3.1.6}
    State and prove an analogue of the limit laws in Proposition 9.3.14 in Analysis I when \(X\) is now a metric space rather than a subset of \(\mathbf{R}\).
\end{exercise}
\section{Pointwise and uniform convergence}\label{sec 3.2}

\begin{definition}[Pointwise convergence]\label{3.2.1}
    Let \((f^{(n)})_{n = 1}^\infty\) be a sequence of functions from one metric space \((X, d_X)\) to another \((Y, d_Y)\), and let \(f : X \to Y\) be another function.
    We say that \emph{\((f^{(n)})_{n = 1}^\infty\) converges pointwise to \(f\) on \(X\)} if we have
    \[
        \lim_{n \to \infty} f^{(n)}(x) = f(x)
    \]
    for all \(x \in X\), i.e.,
    \[
        \lim_{n \to \infty} d_Y\big(f^{(n)}(x), f(x)\big) = 0.
    \]
\end{definition}
\section{Uniform convergence and continuity}\label{sec 3.3}

\begin{theorem}[Uniform limits preserve continuity I]\label{3.3.1}
    Suppose \((f^{(n)})_{n = 1}^\infty\) is a sequence of functions from one metric space \((X, d_X)\) to another \((Y, d_Y)\), and suppose that this sequence converges uniformly to another function \(f : X \to Y\).
    Let \(x_0\) be a point in \(X\).
    If the functions \(f^{(n)}\) are continuous at \(x_0\) for each \(n\), then the limiting function \(f\) is also continuous at \(x_0\).
\end{theorem}

\begin{proof}
    We have
    \begin{align*}
                 & (f^{(n)})_{n = 1}^\infty \text{ converges uniformly to } f \text{ on } X                                                         \\
                 & \text{with respect to } d_Y                                                                                                      \\
        \implies & \forall\ \varepsilon \in \mathbf{R}^+, \exists\ N \in \mathbf{Z}^+ :                                                             \\
                 & \forall\ n \geq N, \forall\ x \in X, d_Y\big(f^{(n)}(x), f(x)\big) < \frac{\varepsilon}{3}. & \text{(by Definition \ref{3.2.7})}
    \end{align*}
    We choose one pair of \(\varepsilon\) and \(N\).
    For each \(n \in \mathbf{Z}^+\), since \(f^{(n)}\) is continuous at \(x_0\) from \((X, d_X)\) to \((Y, d_Y)\), by Definition \ref{2.1.1} we have
    \begin{align*}
                 & \forall\ n \geq N, f^{(n)} \text{ is continuous at } x_0 \text{ from } (X, d_X) \text{ to } (Y, d_Y)                                                                                        \\
        \implies & \forall\ n \geq N, \exists\ \delta \in \mathbf{R}^+ :                                                                                                                                       \\
                 & \Big(\forall\ x \in X, d_X(x, x_0) < \delta \implies d_Y\big(f^{(n)}(x), f^{(n)}(x_0)\big) < \frac{\varepsilon}{3}\Big)                                                                     \\
        \implies & \forall\ n \geq N, \exists\ \delta \in \mathbf{R}^+ :                                                                                                                                       \\
                 & \Big(\forall\ x \in X, d_X(x, x_0) < \delta \implies d_Y\big(f(x), f(x_0)\big)                                                                                                              \\
                 & \leq d_Y\big(f(x), f^{(n)}(x)\big) + d_Y\big(f^{(n)}(x), f^{(n)}(x_0)\big) + d_Y\big(f^{(n)}(x_0), f(x_0)\big) < \frac{\varepsilon}{3} + \frac{\varepsilon}{3} + \frac{\varepsilon}{3}\Big) \\
        \implies & \forall\ n \geq N, \exists\ \delta \in \mathbf{R}^+ :                                                                                                                                       \\
                 & \Big(\forall\ x \in X, d_X(x, x_0) < \delta \implies d_Y\big(f(x), f(x_0)\big) < \varepsilon.
    \end{align*}
    Since \(\varepsilon\) is arbitrary, by Definition \ref{2.1.1} we know that \(f\) is continuous at \(x_0\) from \((X, d_X)\) to \((Y, d_Y)\).
\end{proof}

\begin{corollary}[Uniform limits preserve continuity II]\label{3.3.2}
    Let \((f^{(n)})_{n = 1}^\infty\) be a sequence of functions from one metric space \((X, d_X)\) to another \((Y, d_Y)\), and suppose that this sequence converges uniformly to another function \(f : X \to Y\).
    If the functions \(f^{(n)}\) are continuous on \(X\) for each \(n\), then the limiting function \(f\) is also continuous on \(X\).
\end{corollary}

\begin{proof}
    By applying Theorem \ref{3.3.1} to each \(x \in X\) we conclude that \(f\) is continuous on \(X\) from \((X, d_X)\) to \((Y, d_Y)\).
\end{proof}

\begin{proposition}[Interchange of limits and uniform limits]\label{3.3.3}
    Let
    \((X, d_X)\) and \((Y, d_Y)\) be metric spaces, with \(Y\) complete, and let \(E\) be a subset of \(X\).
    Let \((f^{(n)})_{n = 1}^\infty\) be a sequence of functions from \(E\) to \(Y\), and suppose that this sequence converges uniformly in \(E\) to some function \(f : E \to Y\).
    Let \(x_0 \in X\) be an adherent point of \(E\), and suppose that for each \(n\) the limit \(\lim_{x \to x_0 ; x \in E} f^{(n)}(x)\) exists.
    Then the limit \(\lim_{x \to x_0 ; x \in E} f(x)\) also exists, and is equal to the limit of the sequence \(\big(\lim_{x \to x_0 ; x \in E} f^{(n)}(x)\big)_{n = 1}^\infty\);
    in other words we have the interchange of limits
    \[
        \lim_{n \to \infty} \lim_{x \to x_0 ; x \in E} f^{(n)}(x) = \lim_{x \to x_0 ; x \in E} \lim_{n \to \infty} f^{(n)}(x).
    \]
\end{proposition}

\begin{proof}
    For each \(n \in \mathbf{Z}^+\), we define \(d_Y - \lim_{x \to x_0 ; x \in E} f^{(n)}(x) = L^{(n)}\).
    We claim that the sequence \((L^{(n)})_{n = 1}^\infty\) converges in \(Y\) with respect to \(d_Y\).
    Since \((Y, d_Y)\) is complete, by Definition \ref{1.4.10} it suffices to show that \((L^{(n)})_{n = 1}^\infty\) is a Cauchy sequence in \((Y, d_Y)\).
    Let \(n_1, n_2 \in \mathbf{Z}^+\).
    Then by Definition \ref{3.2.7} we have
    \begin{align*}
                 & (f^{(n)})_{n = 1}^\infty \text{ converges uniformly to } f \text{ on } X \text{ with respect to } d_Y                                                           \\
        \implies & \forall\ \varepsilon \in \mathbf{R}^+, \exists\ N \in \mathbf{Z}^+ : \forall\ n \geq N, \forall\ x \in X, d_Y\big(f^{(n)}(x), f(x)\big) < \frac{\varepsilon}{4}
    \end{align*}
    Now fix one pair of \(\varepsilon\) and \(N\).
    Since \(L^{(n)}\) exists for each \(n \in \mathbf{N}\), by Definition \ref{3.1.1} we have
    \begin{align*}
                 & \forall\ n \geq N, d_Y - \lim_{x \to x_0 ; x \in E} f^{(n)}(x) = L^{(n)}                                                                                                 \\
        \implies & \forall\ n \geq N, \exists\ \delta \in \mathbf{R}^+ : \Big(\forall\ x \in X, d_X(x, x_0) < \delta \implies d_Y\big(f^{(n)}(x), L^{(n)}\big) < \frac{\varepsilon}{4}\Big) \\
        \implies & \forall\ n_1, n_2 \geq N, \exists\ \delta \in \mathbf{R}^+ :                                                                                                             \\
                 & \Big(\forall\ x \in X, d_X(x, x_0) < \delta \implies d_Y\big(L^{(n_1)}, L^{(n_2)}\big)                                                                                   \\
                 & \leq d_Y\big(L^{(n_1)}, f^{(n_1)}(x)\big) + d_Y\big(f^{(n_1)}(x), f(x)\big)                                                                                              \\
                 & \quad + d_Y\big(f(x), f^{(n_2)}(x)\big) + d_Y\big(f^{(n_2)}(x), L^{(n_2)}\big)                                                                                           \\
                 & < \frac{\varepsilon}{4} + \frac{\varepsilon}{4} + \frac{\varepsilon}{4} + \frac{\varepsilon}{4}\Big)                                                                     \\
        \implies & \forall\ n_1, n_2 \geq N, d_Y\big(L^{(n_1)}, L^{(n_2)}\big) < \varepsilon
    \end{align*}
    Since \(\varepsilon\) is arbitrary, we have
    \[
        \forall\ \varepsilon \in \mathbf{R}^+, \exists\ N \in \mathbf{Z}^+ : \forall\ n_1, n_2 \geq N, d_Y\big(L^{(n_1)}, L^{(n_2)}\big) < \varepsilon
    \]
    and by Definition \ref{1.4.6} \((L^{(n)})_{n = 1}^\infty\) is a Cauchy sequence in \((Y, d_Y)\).

    Let \(L \in Y\) such that \(d_Y - \lim_{n \to \infty} L^{(n)} = L\).
    Again by Definition \ref{3.2.7} we have
    \begin{align*}
                 & (f^{(n)})_{n = 1}^\infty \text{ converges uniformly to } f \text{ on } X \text{with respect to } d_Y                                                                 \\
        \implies & \forall\ \varepsilon \in \mathbf{R}^+, \exists\ N_1 \in \mathbf{Z}^+ : \forall\ n \geq N_1, \forall\ x \in X, d_Y\big(f^{(n)}(x), f(x)\big) < \frac{\varepsilon}{3}.
    \end{align*}
    Again we choose one pair of \(\varepsilon\) and \(N_1\).
    Since \(L\) exists, by Definition \ref{3.1.1} we have
    \begin{align*}
                 & \lim_{n \to \infty} d_Y\big(L^{(n)}, L\big) = 0                                                                                                                              \\
        \implies & \exists\ N_2 \in \mathbf{Z}^+ : \forall\ n \geq N_2, d_Y(L^{(n)}, L) < \frac{\varepsilon}{3}                                                                                 \\
        \implies & \exists\ N = \max(N_1, N_2) : \forall\ n \geq N,                                                                                                                             \\
                 & \begin{cases}
            \exists\ \delta \in \mathbf{R}^+ : \forall\ x \in X, d_X(x, x_0) < \delta \implies d_Y\big(f^{(n)}(x), L^{(n)}\big) < \frac{\varepsilon}{3} \\
            d_Y(L^{(n)}, L) < \frac{\varepsilon}{3}                                                                                                     \\
            \forall\ x \in X, d_Y\big(f^{(n)}(x), f(x)\big) < \frac{\varepsilon}{3}
        \end{cases}                                                                                                                                                   \\
        \implies & \exists\ N = \max(N_1, N_2) : \forall\ n \geq N, \exists\ \delta \in \mathbf{R}^+ :                                                                                          \\
                 & \Big(\forall\ x \in X, d_X(x, x_0) < \delta \implies d_Y\big(f(x), L\big)                                                                                                    \\
                 & \leq d_Y\big(f(x), f^{(n)}(x)\big) + d_Y\big(f^{(n)}(x), L^{(n)}\big) + d_Y\big(L^{(n)}, L\big) < \frac{\varepsilon}{3} + \frac{\varepsilon}{3} + \frac{\varepsilon}{3}\Big) \\
        \implies & \exists\ \delta \in \mathbf{R}^+ : \Big(\forall\ x \in X, d_X(x, x_0) < \delta \implies d_Y\big(f(x), L\big) < \varepsilon\Big).
    \end{align*}
    Since \(\varepsilon\) is arbitrary, by Definition \ref{3.1.1} we know that \(d_Y - \lim_{x \to x_0 ; x \in E} f(x) = L\).
\end{proof}

\begin{proposition}\label{3.3.4}
    Let \((f^{(n)})_{n = 1}^\infty\) be a sequence of continuous functions from one metric space \((X, d_X)\) to another \((Y, d_Y)\), and suppose that this sequence converges uniformly to another function \(f : X \to Y\).
    Let \(x^{(n)}\) be a sequence of points in \(X\) which converge to some limit \(x\).
    Then \(f^{(n)}(x^{(n)})\) converges (in \(Y\)) to \(f(x)\).
\end{proposition}

\begin{proof}
    Let \(x_0 \in X\).
    Suppose that \((x^{(n)})_{n = 1}^\infty\) is a sequence in \(X\) such that
    \[
        \lim_{n \to \infty} d_X(x^{(n)}, x_0) = 0.
    \]
    By Theorem \ref{3.3.1} we know that \(f\) is continuous at \(x_0\) from \((X, d_X)\) to \((Y, d_Y)\).
    Thus by Theorem \ref{2.1.4}(a)(b) we have
    \[
        \lim_{n \to \infty} d_Y\big(f(x^{(n)}), f(x_0)\big) = 0
    \]
    and by Definition \ref{1.1.14} we have
    \[
        \forall\ \varepsilon \in \mathbf{R}^+, \exists\ N_1 \in \mathbf{Z}^+ : \forall\ n \geq N_1, d_Y\big(f(x^{(n)}), f(x_0)\big) < \frac{\varepsilon}{2}.
    \]
    Now we choose one pair of \(\varepsilon\) and \(N_1\).
    Since \((f^{(n)})_{n = 1}^\infty\) converges uniformly to \(f\) on \(X\) with respect to \(d_Y\), by Definition \ref{3.2.7} we have
    \begin{align*}
                 & \exists\ N_2 \in \mathbf{Z}^+ : \forall\ n \geq N_2, \forall\ x \in X, d_Y\big(f^{(n)}(x), f(x)\big) < \frac{\varepsilon}{2}                                           \\
        \implies & \exists\ N_2 \in \mathbf{Z}^+ : \forall\ n \geq N_2, d_Y\big(f^{(n)}(x^{(n)}), f(x^{(n)})\big) < \frac{\varepsilon}{2}                                                 \\
        \implies & \exists\ N = \max(N_1, N_2) : \forall\ n \geq N,                                                                                                                       \\
                 & d_Y\big(f^{(n)}(x^{(n)}), f(x_0)\big) \leq d_Y\big(f^{(n)}(x^{(n)}), f(x^{(n)})\big) + d_Y\big(f(x^{(n)}), f(x_0)\big) < \frac{\varepsilon}{2} + \frac{\varepsilon}{2} \\
        \implies & \exists\ N = \max(N_1, N_2) : \forall\ n \geq N, d_Y\big(f^{(n)}(x^{(n)}), f(x_0)\big) < \varepsilon.
    \end{align*}
    Since \(\varepsilon\) is arbitrary, by Definition \ref{1.1.14} we know that
    \[
        \lim_{n \to \infty} d_Y\big(f^{(n)}(x^{(n)}), f(x_0)\big) = 0.
    \]
    Since \(x_0\) is arbitrary, we conclude that for any \(x_0 \in X\), if \((x^{(n)})_{n = 1}^\infty\) is a sequence in \(X\) such that
    \[
        \lim_{n \to \infty} d_X(x^{(n)}, x_0) = 0,
    \]
    then we have
    \[
        \lim_{n \to \infty} d_Y\big(f^{(n)}(x^{(n)}), f(x_0)\big) = 0.
    \]
\end{proof}

\begin{definition}[Bounded functions]\label{3.3.5}
    A function \(f : X \to Y\) from one metric space \((X, d_X)\) to another \((Y, d_Y)\) is \emph{bounded} if \(f(X)\) is a bounded set, i.e., there exists a ball \(B_{(Y, d_Y)}(y_0, R)\) in \(Y\) such that \(f(x) \in B_{(Y, d_Y)}(y_0, R)\) for all \(x \in X\).
\end{definition}

\begin{proposition}[Uniform limits preserve boundedness]\label{3.3.6}
    Let \((f^{(n)})_{n = 1}^\infty\) be a sequence of functions from one metric space \((X, d_X)\) to another \((Y, d_Y)\), and suppose that this sequence converges uniformly to another function \(f : X \to Y\).
    If the functions \(f^{(n)}\) are bounded on \(X\) for each \(n\), then the limiting function \(f\) is also bounded on \(X\).
\end{proposition}

\begin{proof}
    Since \(f^{(n)}\) is bounded in \((Y, d_Y)\) for each \(n \in \mathbf{Z}^+\), by Definition \ref{3.3.5} and Definition \ref{1.5.3} we have
    \begin{align*}
                 & \forall\ n \in \mathbf{Z}^+, \forall\ y \in Y, \exists\ \varepsilon \in \mathbf{R}^+ : f^{(n)}(X) \subseteq B_{(Y, d_Y)}(y, \varepsilon)           \\
        \implies & \forall\ n \in \mathbf{Z}^+, \forall\ y \in Y, \exists\ \varepsilon \in \mathbf{R}^+ : \forall\ x \in X, d_Y\big(f^{(n)}(x), y\big) < \varepsilon.
    \end{align*}
    Now we choose \(y\) and \(\varepsilon\) for each \(n \in \mathbf{Z}^+\) and we denote them as \(y^{(n)}\) and \(\varepsilon^{(n)}\).
    Since \((f^{(n)})_{n = 1}^\infty\) converges uniformly to \(f\) on \(X\) with respect to \(d_Y\), by Definition \ref{3.2.7} we have
    \begin{align*}
                 & \exists\ N \in \mathbf{Z}^+ : \forall\ n \geq N, \forall\ x \in X, d_Y\big(f^{(n)}(x), f(x)\big) < 1                     \\
                 & \exists\ N \in \mathbf{Z}^+ : \forall\ x \in X, d_Y\big(f^{(N)}(x), f(x)\big) < 1                                        \\
        \implies & \exists\ N \in \mathbf{Z}^+ : \forall\ x \in X,                                                                          \\
                 & d_Y\big(f(x), y^{(N)}\big) \leq d_Y\big(f(x), f^{(N)}(x)\big) + d_Y\big(f^{(N)}(x), y^{(N)}\big) < \varepsilon^{(N)} + 1 \\
        \implies & \exists\ N \in \mathbf{Z}^+ : \forall\ x \in X, d_Y\big(f(x), y^{(N)}\big) < \varepsilon^{(N)} + 1                       \\
        \implies & \exists\ N \in \mathbf{Z}^+ : f(X) \subseteq B_{(Y, d_Y)}(y^{(N)}, \varepsilon^{(N)} + 1).
    \end{align*}
    Since \(y^{(N)}\) is arbitrary, we have
    \[
        \forall\ y \in Y, \exists\ \varepsilon \in \mathbf{R}^+ : f(X) \subseteq B_{(Y, d_Y)}(y, \varepsilon)
    \]
    and by Definition \ref{1.5.3} and Definition \ref{3.3.5} \(f\) is bounded in \((Y, d_Y)\).
\end{proof}

\begin{remark}\label{3.3.7}
    The above propositions sound very reasonable, but one should caution that it only works if one assumes uniform convergence;
    pointwise convergence is not enough.
\end{remark}

\exercisesection

\begin{exercise}\label{ex 3.3.1}
    Prove Theorem \ref{3.3.1}.
    Explain briefly why your proof requires uniform convergence, and why pointwise convergence would not suffice.
\end{exercise}

\begin{proof}
    See Theorem \ref{3.3.1}.
    Without uniform convergence we cannot make \(f^{(n)}(x)\) and \(f(x)\) arbitrary close.
\end{proof}

\begin{exercise}\label{ex 3.3.2}
    Prove Proposition \ref{3.3.3}.
\end{exercise}

\begin{proof}
    See Proposition \ref{3.3.3}.
\end{proof}

\begin{exercise}\label{ex 3.3.3}
    Compare Proposition \ref{3.3.3} with Example 1.2.8 in Analysis I.
    Can you now explain why the interchange of limits in Example 1.2.8 in Analysis I led to a false statement, whereas the interchange of limits in Proposition \ref{3.3.3} is justified?
\end{exercise}

\begin{exercise}\label{ex 3.3.4}
    Prove Proposition \ref{3.3.4}.
\end{exercise}

\begin{proof}
    See Proposition \ref{3.3.4}.
\end{proof}

\begin{exercise}\label{ex 3.3.5}
    Give an example to show that Proposition \ref{3.3.4} fails if the phrase ``converges uniformly'' is replaced by ``converges pointwise''.
\end{exercise}

\begin{exercise}\label{ex 3.3.6}
    Prove Proposition \ref{3.3.6}.
\end{exercise}

\begin{proof}
    See Proposition \ref{3.3.6}.
\end{proof}

\begin{exercise}\label{ex 3.3.7}
    Give an example to show that Proposition \ref{3.3.6} fails if the phrase ``converges uniformly'' is replaced by ``converges pointwise''.
\end{exercise}

\begin{exercise}\label{ex 3.3.8}
    Let \((X, d)\) be a metric space, and for every positive integer \(n\), let \(f_n : X \to \mathbf{R}\) and \(g_n : X \to \mathbf{R}\) be functions.
    Suppose that \((f_n)_{n = 1}^\infty\) converges uniformly to another function \(f : X \to \mathbf{R}\), and that \((g_n)_{n = 1}^\infty\) converges uniformly to another function \(g : X \to \mathbf{R}\).
    Suppose also that the functions \((f_n)_{n = 1}^\infty\) and \((g_n)_{n = 1}^\infty\) are uniformly bounded, i.e., there exists an \(M > 0\) such that \(\abs*{f_n(x)} \leq M\) and \(\abs*{g_n(x)} \leq M\) for all \(n \geq 1\) and \(x \in X\).
    Prove that the functions \(f_n g_n : X \to \mathbf{R}\) converge uniformly to \(fg : X \to \mathbf{R}\).
\end{exercise}
\section{The metric of uniform convergence}\label{sec 3.4}

\begin{note}
    We have now developed at least four, apparently separate, notions of limit in this text:
    \begin{enumerate}
        \item limits \(\lim_{n \to \infty} x^{(n)}\) of sequences of points in a metric space
              (Definition \ref{1.1.14};
              see also Definition \ref{2.5.4});
        \item limiting values \(\lim_{x \to x_0 ; x \in E} f(x)\) of functions at a point
              (Definition \ref{3.1.1});
        \item pointwise limits \(f\) of functions \(f^{(n)}\)
              (Definition \ref{3.2.1});
              and
        \item uniform limits \(f\) of functions \(f^{(n)}\)
              (Definition \ref{3.2.7}).
    \end{enumerate}

    This proliferation of limits may seem rather complicated.
    However, we can reduce the complexity slightly by observing that (d) can be viewed as a special case of (a), though in doing so it should be cautioned that because we are now dealing with functions instead of points, the convergence is not in \(X\) or in \(Y\), but rather in a new space, the space of functions from \(X\) to \(Y\).
\end{note}

\begin{remark}\label{3.4.1}
    If one is willing to work in topological spaces instead of metric spaces, we can also view (a) as a special case of (b), see Exercise \ref{ex 3.1.4}, and (c) is also a special case of (a), see Exercise \ref{ex 3.4.4}.
    Thus the notion of convergence in a topological space can be used to unify all the notions of limits we have encountered so far.
\end{remark}

\begin{definition}[Metric space of bounded functions]\label{3.4.2}
    Suppose \((X, d_X)\) and \((Y, d_Y)\) are metric spaces.
    We let \(B(X \to Y)\) denote the space of bounded functions from \(X\) to \(Y\) :
    \[
        B(X \to Y) \coloneqq \{f | f : X \to Y \text{ is a bounded function}\}.
    \]
    We define a metric \(d_\infty : B(X \to Y) \times B(X \to Y) \to [0, +\infty)\) by defining
    \[
        d_\infty(f, g) \coloneqq \sup_{x \in X} d_Y\big(f(x), g(x)\big) = \sup\Big\{d_Y\big(f(x), g(x)\big) : x \in X\Big\}
    \]
    for all \(f, g \in B(X \to Y)\).
    This metric is sometimes known as the \emph{uniform metric}, or \emph{sup norm metric}, or the \emph{\(L^\infty\) metric}.
    We will also use \(d_{B(X \to Y)}\) as a synonym for \(d_\infty\).
    We restrict the definition of \(d_\infty\) to the case when \(X \neq \emptyset\).
    If \(X = \emptyset\), then we instead define \(d_\infty(f, g) = 0\).
\end{definition}

\begin{note}
    \(B(X \to Y)\) is a set, thanks to the power set axiom (Axiom 3.10 in Analysis I) and the axiom of specification (Axiom 3.5 in Analysis I).
\end{note}

\begin{note}
    The distance \(d_\infty(f, g)\) is always finite because \(f\) and \(g\) are assumed to be bounded on \(X\).
\end{note}

\setcounter{theorem}{3}
\begin{proposition}\label{3.4.4}
    Let \((X, d_X)\) and \((Y, d_Y)\) be metric spaces.
    Let \((f^{(n)})_{n = 1}^\infty\) be a sequence of functions in \(B(X \to Y)\), and let \(f\) be another function in \(B(X \to Y)\).
    Then \((f^{(n)})_{n = 1}^\infty\) converges to \(f\) in the metric \(d_{B(X \to Y)}\) if and only if \((f^{(n)})_{n = 1}^\infty\) converges uniformly to \(f\).
\end{proposition}

\begin{proof}
    We have
    \begin{align*}
             & d_{B(X \to Y)} - \lim_{n \to \infty} f^{(n)} = f                                                                                                \\
        \iff & \forall \varepsilon \in \mathbf{R}^+, \exists\ N \in \mathbf{Z}^+ :                                                                             \\
             & \forall n \geq N, d_{B(X \to Y)}(f^{(n)}, f) \leq \frac{\varepsilon}{2} < \varepsilon                     & \text{(by Definition \ref{1.1.14})} \\
        \iff & \forall \varepsilon \in \mathbf{R}^+, \exists\ N \in \mathbf{Z}^+ :                                                                             \\
             & \forall n \geq N, \sup_{x \in X} d_Y\big(f^{(n)}(x), f(x)\big) \leq \frac{\varepsilon}{2} < \varepsilon   & \text{(by Definition \ref{3.4.2})}  \\
        \iff & \forall \varepsilon \in \mathbf{R}^+, \exists\ N \in \mathbf{Z}^+ :                                                                             \\
             & \forall n \geq N, \forall x \in X, d_Y\big(f^{(n)}(x), f(x)\big) \leq \frac{\varepsilon}{2} < \varepsilon                                       \\
        \iff & (f^{(n)})_{n = 1}^\infty \text{ converges uniformly to } f \text{ on } X                                                                        \\
             & \text{with respect to } d_Y.                                                                              & \text{(by Definition \ref{3.2.7})}
    \end{align*}
\end{proof}

\begin{note}
    Now let \(C(X \to Y)\) be the space of bounded continuous functions from \(X\) to \(Y\) :
    \[
        C(X \to Y) \coloneqq \{f \in B(X \to Y) | f \text{ is continuous}\}.
    \]
    This set \(C(X \to Y)\) is clearly a subset of \(B(X \to Y)\).
    Corollary \ref{3.3.2} asserts that this space \(C(X \to Y)\) is closed in \(\big(B(X \to Y), d_{B(X \to Y)}\big)\).
\end{note}

\begin{theorem}[The space of continuous functions is complete]\label{3.4.5}
    Let \((X, d_X)\) be a metric space, and let \((Y, d_Y)\) be a complete metric space.
    The space \(\big(C(X \to Y), d_{B(X \to Y)}|_{C(X \to Y) \times C(X \to Y)}\big)\) is a complete subspace of \(\big(B(X \to Y), d_{B(X \to Y)}\big)\).
    In other words, every Cauchy sequence of functions in \(C(X \to Y)\) converges to a function in \(C(X \to Y)\).
\end{theorem}

\begin{proof}
    Let \(d_{C(X \to Y)} = d_{B(X \to Y)}|_{C(X \to Y) \times C(X \to Y)}\) and let \(n_1, n_2 \in \mathbf{Z}^+\).
    Let \((f_n)_{n = 1}^\infty\) be a Cauchy sequence in \(\big(C(X \to Y), d_{C(X \to Y)}\big)\).
    Observe that
    \begin{align*}
                 & \forall \varepsilon \in \mathbf{R}^+, \exists\ N \in \mathbf{Z}^+ : \forall n_1, n_2 \geq N,                                                           \\
                 & d_{C(X \to Y)}\big(f^{(n_1)}, f^{(n_2)}\big) < \varepsilon                                                        & \text{(by Definition \ref{1.4.6})} \\
        \implies & \forall \varepsilon \in \mathbf{R}^+, \exists\ N \in \mathbf{Z}^+ : \forall n_1, n_2 \geq N,                                                           \\
                 & \sup_{x \in X} d_Y\big(f^{(n_1)}(x), f^{(n_2)}(x)\big) < \varepsilon                                              & \text{(by Definition \ref{3.4.2})} \\
        \implies & \forall x \in X, \forall \varepsilon \in \mathbf{R}^+, \exists\ N \in \mathbf{Z}^+ : \forall n_1, n_2 \geq N,                                          \\
                 & d_Y\big(f^{(n_1)}(x), f^{(n_2)}(x)\big) \leq \sup_{x \in X} d_Y\big(f^{(n_1)}(x), f^{(n_2)}(x)\big) < \varepsilon                                      \\
        \implies & \forall x \in X, \big(f_n(x)\big)_{n = 1}^\infty \text{ is a Cauchy sequence in } (Y, d_Y).                       & \text{(by Definition \ref{1.4.6})}
    \end{align*}
    By hypothesis we know that \((Y, d_Y)\) is complete, thus by Definition \ref{1.4.10} we have
    \[
        \forall x \in X, d_Y - \lim_{n \to \infty} f_n(x) \in Y
    \]
    and we can define a function \(f : X \to Y\) such that
    \[
        \forall x \in X, f(x) = d_Y - \lim_{n \to \infty} f_n(x).
    \]
    By Definition \ref{1.1.14} we have
    \[
        \forall x \in X, \forall \varepsilon \in \mathbf{R}^+, \exists\ N \in \mathbf{Z}^+ : \forall n \geq N, d_Y\big(f_n(x), f(x)\big) < \frac{\varepsilon}{3}.
    \]
    We choose one \(N\) for each pairs of \(x\) and \(\varepsilon\) and denote it as \(N_{x, \varepsilon}\).
    Since \(f_n \in C(X \to Y)\) for all \(n \in \mathbf{Z}^+\), by Definition \ref{2.1.1} we know that
    \[
        \forall x_0 \in X, \forall \varepsilon \in \mathbf{R}^+, \exists\ \delta \in \mathbf{R}^+ : \forall x \in X, d_X(x, x_0) < \delta \implies d_Y\big(f_n(x), f_n(x_0)\big) < \frac{\varepsilon}{3}.
    \]
    If we denote \(M_{x, x_0, \varepsilon} = \max(N_{x, \varepsilon}, N_{x_0, \varepsilon})\), then by Definition \ref{1.1.2}(d) we have
    \begin{align*}
                 & \forall x_0 \in X, \forall \varepsilon \in \mathbf{R}^+, \exists\ \delta \in \mathbf{R}^+ : \forall x \in X, d_X(x, x_0) < \delta \\
        \implies & \begin{cases}
                       \forall n \geq M_{x, x_0, \varepsilon}, d_Y\big(f_n(x), f(x)\big) < \frac{\varepsilon}{3}     \\
                       \forall n \geq M_{x, x_0, \varepsilon}, d_Y\big(f_n(x_0), f(x_0)\big) < \frac{\varepsilon}{3} \\
                       d_Y\big(f_n(x), f_n(x_0)\big) < \frac{\varepsilon}{3}
                   \end{cases}                                     \\
        \implies & \forall n \geq M_{x, x_0, \varepsilon},                                                                                           \\
                 & d_Y\big(f(x), f(x_0)\big) \leq d_Y\big(f_n(x), f(x)\big) + d_Y\big(f_n(x), f_n(x_0)\big) + d_Y\big(f_n(x_0), f(x_0)\big)          \\
                 & < \frac{\varepsilon}{3} + \frac{\varepsilon}{3} + \frac{\varepsilon}{3} = \varepsilon                                             \\
        \implies & d_Y\big(f(x), f(x_0)\big) < \varepsilon.
    \end{align*}
    By Definition \ref{2.1.1} this means \(f \in C(X \to Y)\).
    Since \((f_n)_{n = 1}^\infty\) is arbitrary, by Definition \ref{1.4.10} \(\big(C(X \to Y), d_{C(X \to Y)}\big)\) is complete.
\end{proof}

\exercisesection

\begin{exercise}\label{ex 3.4.1}
    Let \((X, d_X)\) and \((Y, d_Y)\) be metric spaces.
    Show that the space \(B(X \to Y)\) defined in Definition \ref{3.4.2}, with the metric \(d_{B(X \to Y)}\), is indeed a metric space.
\end{exercise}

\begin{proof}
    If \(X = \emptyset\), then by Definition \ref{3.4.2} we have
    \begin{itemize}
        \item If \(f \in B(\emptyset \to Y)\), then \(d_{B(X \to Y)}(f, f) = 0\).
        \item If \(f, g \in B(\emptyset \to Y)\), then \(d_{B(X \to Y)}(f, g) = 0 = d_{B(X \to Y)}(g, f)\).
        \item If \(f, g, h \in B(\emptyset \to Y)\), then \(d_{B(X \to Y)}(f, h) = 0 = d_{B(X \to Y)}(f, g) + d_{B(X \to Y)}(g, h)\).
    \end{itemize}
    Thus by Definition \ref{1.1.2} \(\big(B(\emptyset \to Y), d_{B(X \to Y)}\big)\) is a metric space.
    Now suppose that \(X \neq \emptyset\).
    Since
    \begin{align*}
        \forall f \in B(X \to Y), d_{B(X \to Y)}(f, f) & = \sup_{x \in X} d_Y\big(f(x), f(x)\big) & \text{(by Definition \ref{3.4.2})}    \\
                                                       & = \sup \{0\}                             & \text{(by Definition \ref{1.1.2}(a))} \\
                                                       & = 0,
    \end{align*}
    by Definition \ref{1.1.2}(a) we know that \(\big(B(X \to Y), d_{B(X \to Y)}\big)\) is reflexive.
    Since
    \begin{align*}
                 & \forall f, g \in B(X \to Y), f \neq g                                                                      \\
        \implies & \exists\ x \in X : f(x) \neq g(x)                                                                          \\
        \implies & \exists\ x \in X : d_Y\big(f(x), g(x)\big) > 0                     & \text{(by Definition \ref{1.1.2}(b))} \\
        \implies & d_{B(X \to Y)}(f, g) = \sup_{x \in X} d_Y\big(f(x), g(x)\big) > 0, & \text{(by Definition \ref{3.4.2})}
    \end{align*}
    by Definition \ref{1.1.2}(b) we know that \(\big(B(X \to Y), d_{B(X \to Y)}\big)\) is positive.
    Since
    \begin{align*}
        \forall f, g \in B(X \to Y), d_{B(X \to Y)}(f, g) & = \sup_{x \in X} d_Y\big(f(x), g(x)\big) & \text{(by Definition \ref{3.4.2})}    \\
                                                          & = \sup_{x \in X} d_Y\big(g(x), f(x)\big) & \text{(by Definition \ref{1.1.2}(c))} \\
                                                          & = d_{B(X \to Y)}(g, f),                  & \text{(by Definition \ref{3.4.2})}
    \end{align*}
    by Definition \ref{1.1.2}(c) we know that \(\big(B(X \to Y), d_{B(X \to Y)}\big)\) is symmetry.
    Since
    \begin{align*}
         & \forall f, g, h \in B(X \to Y),                                                                                           \\
         & d_{B(X \to Y)}(f, g) + d_{B(X \to Y)}(g, h)                                                                               \\
         & = \sup_{x \in X} d_Y\big(f(x), g(x)\big) + \sup_{x \in X} d_Y\big(g(x), h(x)\big) & \text{(by Definition \ref{3.4.2})}    \\
         & \geq \sup_{x \in X} \Big(d_Y\big(f(x), g(x)\big) + d_Y\big(g(x), h(x)\big)\Big)                                           \\
         & \geq \sup_{x \in X} d_Y\big(f(x), h(x)\big)                                       & \text{(by Definition \ref{1.1.2}(d))} \\
         & = d_{B(X \to Y)}(f, h),                                                           & \text{(by Definition \ref{3.4.2})}
    \end{align*}
    by Definition \ref{1.1.2}(d) we know that \(\big(B(X \to Y), d_{B(X \to Y)}\big)\) is transitive.
    Combine all the proofs above we conclude by Definition \ref{1.1.2} that \(\big(B(X \to Y), d_{B(X \to Y)}\big)\) is a metric space.
\end{proof}

\begin{exercise}\label{ex 3.4.2}
    Prove Proposition \ref{3.4.4}.
\end{exercise}

\begin{proof}
    See Proposition \ref{3.4.4}.
\end{proof}

\begin{exercise}\label{ex 3.4.3}
    Prove Theorem \ref{3.4.5}.
\end{exercise}

\begin{proof}
    See Theorem \ref{3.4.5}.
\end{proof}

\begin{exercise}\label{ex 3.4.4}
    Let \((X, d_X)\) and \((Y, d_Y)\) be metric spaces, and let \(Y^X \coloneqq \{f | f : X \to Y \}\) be the space of all functions from \(X\) to \(Y\)
    (cf. Axiom 3.10 in Analysis I).
    If \(x_0 \in X\) and \(V\) is an open set in \(Y\), let \(V^{(x_0)} \subseteq Y^X\) be the set
    \[
        V^{(x_0)} \coloneqq \big\{f \in Y^X : f(x_0) \in V\big\}.
    \]
    If \(E\) is a subset of \(Y^X\), we say that \(E\) is \emph{open} if for every \(f \in E\), there exists a finite number of points \(x_1, \dots, x_n \in X\) and open sets \(V_1, \dots, V_n \subseteq Y\) such that
    \[
        f \in V_1^{(x_1)} \cap \dots \cap V_n^{(x_n)} \subseteq E.
    \]
    \begin{itemize}
        \item Show that if \(\mathcal{F}\) is the collection of open sets in \(Y^X\), then \((Y^X , \mathcal{F})\) is a topological space.
        \item For each natural number \(n\), let \(f^{(n)} : X \to Y\) be a function from \(X\) to \(Y\), and let \(f : X \to Y\) be another function from \(X\) to \(Y\).
              Show that \(f^{(n)}\) converges to \(f\) in the topology \(\mathcal{F}\) (in the sense of Definition \ref{2.5.4}) if and only if \(f^{(n)}\) converges to \(f\) pointwise (in the sense of Definition \ref{3.2.1}).
    \end{itemize}
    The topology \(\mathcal{F}\) is known as the \emph{topology of pointwise convergence}, for obvious reasons;
    it is also known as the \emph{product topology}.
    It shows that the concept of pointwise convergence can be viewed as a special case of the more general concept of convergence in a topological space.
\end{exercise}

\begin{proof}
    We know that \(\emptyset \in \mathcal{F}\) trivially.
    First we show that \(Y^X \in \mathcal{F}\).
    Let \(f \in Y^X\) and let \(x_0 \in X\).
    By Proposition \ref{1.2.15}(c) we know that \(B_{(Y, d_Y)}\big(f(x_0), 1\big)\) is open in \((Y, d_Y)\).
    Then we have
    \[
        f \in \Big(B_{(Y, d_Y)}\big(f(x_0), 1\big)\Big)^{(x_0)} \subseteq Y^X.
    \]
    Since \(f\) is arbitrary, by definition we know that \(Y^X \in \mathcal{F}\).

    Next we show that the intersection of any finite collection of open sets in \((Y^X, \mathcal{F})\) is open in \((Y^X, \mathcal{F})\).
    Let \(n \in \mathbf{N}\) and let \((U_i)_{i = 1}^n\) be a finite collection of open sets in \((Y^X, \mathcal{F})\).
    If \(\bigcap_{i = 1}^n U_i = \emptyset\), then from the proof above we know that \(\emptyset \in \mathcal{F}\).
    So suppose that \(\bigcap_{i = 1}^n U_i \neq \emptyset\).
    Let \(f \in \bigcap_{i = 1}^n U_i\).
    Since
    \begin{align*}
                 & \forall 1 \leq i \leq n, f \in U_i                                                                                          \\
        \implies & \forall 1 \leq i \leq n, \exists\ m_i \in \mathbf{Z}^+ : \begin{cases}
                                                                                x_{(i, 1)}, \dots, x_{(i, m_i)} \in X                              \\
                                                                                V_{(i, 1)}, \dots, V_{(i, m_i)} \text{ are open sets in } (Y, d_Y) \\
                                                                                f \in \bigcap_{j = 1}^{m_i} V_{(i, j)}^{(x_{(i, j)})} \subseteq U_i
                                                                            \end{cases} \\
        \implies & f \in \bigcap_{i = 1}^n \bigg(\bigcap_{j = 1}^{m_i} V_{(i, j)}^{(x_{(i, j)})}\bigg) \subseteq \bigcap_{i = 1}^n U_i
    \end{align*}
    and \(f\) is arbitrary, we know that \(\bigcap_{i = 1}^n U_i \in \mathcal{F}\).
    Since \(n\) is arbitrary, we know that the intersection of any finite collection of open sets in \((Y^X, \mathcal{F})\) is open in \((Y^X, \mathcal{F})\).

    Next we show that the union of arbitrary open sets in \((Y^X, \mathcal{F})\) is open in \((Y^X, \mathcal{F})\).
    Let \(S \subseteq \mathcal{F}\) and let \(f \in \bigcup S\).
    Since
    \begin{align*}
                 & f \in \bigcup S                                                                       \\
        \implies & \exists\ U \in S : f \in U                                                            \\
        \implies & \exists\ U \in S : \begin{cases}
                                          x_1, \dots, x_n \in X                              \\
                                          V_1, \dots, V_n \text{ are open sets in } (Y, d_Y) \\
                                          f \in \bigcap_{i = 1}^n V_i^{(x_i)} \subseteq U \subseteq \bigcup S
                                      \end{cases}
    \end{align*}
    and \(f\) is arbitrary, we know that \(\bigcup S \in \mathcal{F}\).
    Since \(S\) is arbitrary, we know that the union of arbitrary open sets in \((Y^X, \mathcal{F})\) is open in \((Y^X, \mathcal{F})\).
    Combine all the proofs above we conclude by Definition \ref{2.5.1} that \((Y^X, \mathcal{F})\) is a topological space.

    Next suppose that \((f^{(n)})_{n = 1}^\infty\) is a sequence in \(Y^X\) and \(f \in Y^X\).
    Suppose also that \((f^{(n)})_{n = 1}^\infty\) converges to \(f\) in \((Y^X, \mathcal{F})\).
    Then by Definition \ref{2.5.4} we have
    \[
        \forall E \in \mathcal{F}, f \in E \implies \exists\ N \in \mathbf{Z}^+ : \forall n \geq N, f^{(n)} \in E.
    \]
    Let \(x_0 \in X\).
    Then we have
    \begin{align*}
                 & \forall \varepsilon \in \mathbf{R}^+, B_{(Y, d_Y)}\big(f(x_0), \varepsilon\big) \text{ is open in } (Y, d_Y)            & \text{(by Proposition \ref{1.2.15}(c))} \\
        \implies & \forall \varepsilon \in \mathbf{R}^+, f \in \Big(B_{(Y, d_Y)}\big(f(x_0), \varepsilon\big)\Big)^{(x_0)} \in \mathcal{F} & \text{(by definition)}                  \\
        \implies & \forall \varepsilon \in \mathbf{R}^+, \exists\ N \in \mathbf{Z}^+ : \forall n \geq N,                                                                             \\
                 & f^{(n)} \in \Big(B_{(Y, d_Y)}\big(f(x_0), \varepsilon\big)\Big)^{(x_0)}                                                 & \text{(by Definition \ref{2.5.4})}      \\
        \implies & \forall \varepsilon \in \mathbf{R}^+, \exists\ N \in \mathbf{Z}^+ : \forall n \geq N,                                                                             \\
                 & d_Y\big(f^{(n)}(x_0), f(x_0)\big) < \varepsilon                                                                         & \text{(by definition)}                  \\
        \implies & \lim_{n \to \infty} d_Y\big(f^{(n)}(x_0), f(x_0)\big).                                                                  & \text{(by Definition \ref{1.1.14})}
    \end{align*}
    Since \(x_0\) is arbitrary, by Definition \ref{3.2.1} \((f^{(n)})_{n = 1}^\infty\) converges pointwise to \(f\) on \(X\) with respect to \(d_Y\).

    Finally suppose that \((f^{(n)})_{n = 1}^\infty\) is a sequence in \(Y^X\) and \(f \in Y^X\).
    Suppose also that \((f^{(n)})_{n = 1}^\infty\) converges pointwise to \(f\) on \(X\) with respect to \(d_Y\).
    Then we have
    \begin{align*}
                 & \forall x \in X, \lim_{n \to \infty} d_Y\big(f^{(n)}(x), f(x)\big)                                     & \text{(by Definition \ref{3.2.1})}  \\
        \implies & \forall x \in X, \forall \varepsilon \in \mathbf{R}^+, \exists\ N \in \mathbf{Z}^+ : \forall n \geq N,                                       \\
                 & d_Y\big(f^{(n)}(x), f(x)\big) < \varepsilon.                                                           & \text{(by Definition \ref{1.1.14})}
    \end{align*}
    We choose one \(N\) for each pair of \((x, \varepsilon)\) and denote it as \(N_{(x, \varepsilon)}\).
    Let \(E \in \mathcal{F}\) such that \(f \in E\).
    By definition we know that
    \[
        \exists\ m \in \mathbf{Z}^+ : \begin{cases}
            x_1, \dots, x_m \in X                              \\
            V_1, \dots, V_m \text{ are open sets in } (Y, d_Y) \\
            f \in \bigcap_{i = 1}^m V_i^{(x_i)} \subseteq E
        \end{cases}
    \]
    Then we have
    \begin{align*}
                 & \forall 1 \leq i \leq m, f \in V_i^{(x_i)}                                                                                                                                        \\
        \implies & \forall 1 \leq i \leq m, \begin{cases}
                                                f(x_i) \in V_i \\
                                                V_i \text{ is open in } (Y, d_Y)
                                            \end{cases}                                                                                                                          \\
        \implies & \forall 1 \leq i \leq m, \exists\ \varepsilon_i \in \mathbf{R}^+ : B_{(Y, d_Y)}\big(f(x_i), \varepsilon_i\big) \subseteq V_i            & \text{(by Proposition \ref{1.2.15}(a))} \\
        \implies & \forall 1 \leq i \leq m, \exists\ \varepsilon_i \in \mathbf{R}^+ : \exists\ N_{(x_i, \varepsilon_i)} \in \mathbf{Z}^+ :                                                           \\
                 & \forall n \geq N_{(x_i, \varepsilon_i)}, f^{(n)}(x_i) \in B_{(Y, d_Y)}\big(f(x_i), \varepsilon_i\big) \subseteq V_i                                                               \\
        \implies & \exists\ N = \max_{1 \leq i \leq m} N_{(x_i, \varepsilon_i)} : \forall n \geq N, f^{(n)} \in \bigcap_{i = 1}^m V_i^{(x_i)} \subseteq E.
    \end{align*}
    Since \(E\) is arbitrary, by Definition \ref{2.5.4} we know that \((f^{(n)})_{n = 1}^\infty\) converges to \(f\) in \((Y^X, \mathcal{F})\).
\end{proof}
\section{Series of functions; the Weierstrass \emph{M}-test}\label{sec 3.5}

\begin{note}
    Functions whose range is \(\mathbf{R}\) are sometimes called \emph{real-valued} functions.
\end{note}

\begin{note}
    given any finite collection \(f^{(1)}, \dots, f^{(N)}\) of functions from \(X\) to \(\mathbf{R}\), we can define the finite sum \(\sum_{i = 1}^N f^{(i)} : X \to \mathbf{R}\) by
    \[
        \bigg(\sum_{i = 1}^N f^{(i)}\bigg)(x) \coloneqq \sum_{i = 1}^N f^{(i)}(x).
    \]
\end{note}

\setcounter{theorem}{1}
\begin{definition}[Infinite series]\label{3.5.2}
    Let \((X, d_X)\) be a metric space.
    Let \((f^{(n)})_{n = 1}^\infty\) be a sequence of functions from \(X\) to \(\mathbf{R}\), and let \(f\) be another function from \(X\) to \(\mathbf{R}\).
    If the partial sums \(\sum_{n = 1}^N f^{(n)}\) converges pointwise to \(f\) on \(X\) as \(N \to \infty\), we say that the infinite series \(\sum_{n = 1}^\infty f^{(n)}\) \emph{converges pointwise} to \(f\), and write \(f = \sum_{n = 1}^\infty f^{(n)}\).
    If the partial sums \(\sum_{n = 1}^N f^{(n)}\) converge uniformly to \(f\) on \(X\) as \(N \to \infty\), we say that the infinite series \(\sum_{n = 1}^\infty f^{(n)}\) \emph{converges uniformly} to \(f\), and again write \(f = \sum_{n = 1}^\infty f^{(n)}\).
    (Thus when one sees an expression such as \(f = \sum_{n = 1}^\infty f^{(n)}\), one should look at the context to see in what sense this infinite series converges.)
\end{definition}

\begin{remark}\label{3.5.3}
    A series \(\sum_{n = 1}^\infty f^{(n)}\) converges pointwise to \(f\) on \(X\) if and only if \(\sum_{n = 1}^\infty f^{(n)}(x)\) converges to \(f(x)\) for \emph{every} \(x \in X\).
    (Thus if \(\sum_{n = 1}^\infty f^{(n)}\) does not converge pointwise to \(f\), this does not mean that it diverges pointwise;
    it may just be that it converges for some points \(x\) but diverges at other points \(x\).)
\end{remark}

\begin{note}
    If a series \(\sum_{n = 1}^\infty f^{(n)}\) converges uniformly to \(f\), then it also converges pointwise to \(f\);
    but not vice versa.
\end{note}

\setcounter{theorem}{4}
\begin{definition}[Sup norm]\label{3.5.5}
    If \(f : X \to \mathbf{R}\) is a bounded real-valued function, we define the \emph{sup norm} \(\norm*{f}_\infty\) of \(f\) to be the number
    \[
        \norm*{f}_\infty \coloneqq \sup\big\{\abs*{f(x)} : x \in X\big\}.
    \]
    In other words, \(\norm*{f}_\infty = d_\infty(f, 0)\), where \(0 : X \to \mathbf{R}\) is the zero function \(0(x) \coloneqq 0\), and \(d_\infty\) was defined in Definition \ref{3.4.2}.
    We restrict the definition of \(\norm*{f}_\infty\) to the case when \(X \neq \emptyset\).
    If \(X = \emptyset\), then we instead define \(\norm*{f}_\infty = 0\).
\end{definition}

\begin{note}
    When \(f\) is bounded then \(\norm*{f}_\infty\) will always be a non-negative real number.
\end{note}

\setcounter{theorem}{6}
\begin{theorem}[Weierstrass \(M\)-test]\label{3.5.7}
    Let \((X, d)\) be a metric space, and let \((f^{(n)})_{n = 1}^\infty\) be a sequence of bounded real-valued continuous functions on \(X\) such that the series \(\sum_{n = 1}^\infty \norm*{f^{(n)}}_\infty\) is convergent.
    (Note that this is a series of plain old real numbers, not of functions.)
    Then the series \(\sum_{n = 1}^\infty f^{(n)}\) converges uniformly to some function \(f\) on \(X\), and that function \(f\) is also continuous.
\end{theorem}

\begin{proof}
    Let \(N_1, N_2 \in \mathbf{Z}^+\).
    Let \(d_{C(X \to \mathbf{R})} = d_{B(X \to \mathbf{R})}|_{C(X \to \mathbf{R}) \times C(X \to \mathbf{R})}\).
    We have
    \begin{align*}
                 & \sum_{n = 1}^\infty \norm*{f^{(n)}}_\infty = \lim_{N \to \infty} \sum_{n = 1}^N \norm*{f^{(n)}}_\infty                                                    \\
        \implies & \forall\ \varepsilon \in \mathbf{R}^+, \exists\ M \in \mathbf{Z}^+ : \forall\ N \geq M,                                                                   \\
                 & \abs*{\sum_{n = 1}^\infty \norm*{f^{(n)}}_\infty - \sum_{n = 1}^N \norm*{f^{(n)}}_\infty} < \varepsilon & \text{(by Definition \ref{1.1.14})}             \\
        \implies & \forall\ \varepsilon \in \mathbf{R}^+, \exists\ M \in \mathbf{Z}^+ : \forall\ N \geq M,                                                                   \\
                 & \abs*{\sum_{n = N + 1}^\infty \norm*{f^{(n)}}_\infty} < \varepsilon                                     & \text{(by Proposition 7.2.14(c) in Analysis I)} \\
        \implies & \forall\ \varepsilon \in \mathbf{R}^+, \exists\ M \in \mathbf{Z}^+ : \forall\ N \geq M,                                                                   \\
                 & \sum_{n = N + 1}^\infty \norm*{f^{(n)}}_\infty < \varepsilon                                            & (\norm*{f^{(n)}}_\infty \geq 0)                 \\
        \implies & \forall\ \varepsilon \in \mathbf{R}^+, \exists\ M \in \mathbf{Z}^+ : \forall\ N \geq M,                                                                   \\
                 & \sum_{n = N + 1}^\infty \sup_{x \in X} \abs*{f^{(n)}(x)} < \varepsilon.                                 & \text{(by Definition \ref{3.5.5})}
    \end{align*}
    Fix one \(\varepsilon\) and \(M\).
    Since \(f^{(n)} \in C(X \to \mathbf{R})\), by Exercise \ref{ex 3.5.1} we know that \(\sum_{n = 1}^N f^{(n)} \in C(X \to \mathbf{R})\) for each \(N \in \mathbf{Z}^+\).
    Thus \(d_{C(X \to \mathbf{R})}\bigg(\sum_{n = 1}^{N_1} f^{(n)}, \sum_{n = 1}^{N_2} f^{(n)}\bigg)\) is well defined for each \(N_1, N_2 \geq M\) and
    \begin{align*}
        \forall\ N_1, N_2 \geq M, & d_{C(X \to \mathbf{R})}\bigg(\sum_{n = 1}^{N_1} f^{(n)}, \sum_{n = 1}^{N_2} f^{(n)}\bigg)                                             \\
                                  & = \sup_{x \in X} \abs*{\sum_{n = 1}^{N_1} f^{(n)}(x) - \sum_{n = 1}^{N_2} f^{(n)}(x)}            & \text{(by Definition \ref{3.4.2})} \\
                                  & = \sup_{x \in X} \abs*{\sum_{n = \min(N_1, N_2) + 1}^{\max(N_1, N_2)} f^{(n)}(x)}                                                     \\
                                  & \leq \sup_{x \in X} \bigg(\sum_{n = \min(N_1, N_2) + 1}^{\max(N_1, N_2)} \abs*{f^{(n)}(x)}\bigg)                                      \\
                                  & \leq \sup_{x \in X} \bigg(\sum_{n = M + 1}^\infty \abs*{f^{(n)}(x)}\bigg)                                                             \\
                                  & \leq \sum_{n = M + 1}^\infty \sup_{x \in X} \abs*{f^{(n)}(x)} < \varepsilon.
    \end{align*}
    Since \(\varepsilon\) is arbitrary, we have
    \[
        \forall\ \varepsilon \in \mathbf{R}^+, \exists\ M \in \mathbf{Z}^+ : \forall\ N_1, N_2 \geq M, d_{C(X \to \mathbf{R})}\bigg(\sum_{n = 1}^{N_1} f^{(n)}, \sum_{n = 1}^{N_2} f^{(n)}\bigg) < \varepsilon.
    \]
    By Definition \ref{1.4.6} \(\bigg(\sum_{n = 1}^N f^{(n)}\bigg)_{N = 1}^\infty\) is a Cauchy sequence in \(\big(C(X \to \mathbf{R}), d_{C(X \to \mathbf{R})}\big)\).
    Since \((\mathbf{R}, d_{l^1}|_{\mathbf{R} \times \mathbf{R}})\) is complete, by Theorem \ref{3.4.5} we know that \(\bigg(\sum_{n = 1}^N f^{(n)}\bigg)_{N = 1}^\infty\) converges uniformly to some \(f \in C(X \to \mathbf{R})\) on \(X\) with respect to \(d_{l^1}|_{\mathbf{R} \times \mathbf{R}}\).
\end{proof}

\begin{note}
    To put the Weierstrass \(M\)-test succinctly:
    absolute convergence of sup norms implies uniform convergence of functions.
\end{note}

\begin{example}\label{3.5.8}
    Let \(0 < r < 1\) be a real number, and let \(f^{(n)} : [-r, r] \to \mathbf{R}\) be the series of functions \(f^{(n)}(x) \coloneqq x^n\).
    Then each \(f^{(n)}\) is continuous and bounded, and \(\norm*{f^{(n)}}_\infty = r^n\).
    Since the series \(\sum_{n = 1}^\infty r^n\) is absolutely convergent (e.g., by the root test, Theorem 7.5.1 in Analysis I), we thus see that \(f^{(n)}\) converges uniformly in \([-r, r]\) to some continuous function;
    in Exercise \ref{ex 3.2.2}(c) we see that this function must in fact be the function \(f : [-r, r] \to \mathbf{R}\) defined by \(f(x) \coloneqq x / (1 - x)\).
    In other words, the series \(\sum_{n = 1}^\infty x^n\) is pointwise convergent, but not uniformly convergent, on \((-1, 1)\), but is uniformly convergent on the smaller interval \([-r, r]\) for any \(0 < r < 1\).
\end{example}

\begin{note}
    The Weierstrass \(M\)-test is especially useful in relation to power series.
\end{note}

\exercisesection

\begin{exercise}\label{ex 3.5.1}
    Let \(f^{(1)}, \dots, f^{(N)}\) be a finite sequence of bounded functions from a metric space \((X, d_X)\) to \(\mathbf{R}\).
    Show that \(\sum_{i = 1}^N f^{(i)}\) is also bounded.
    Prove a similar claim when ``bounded'' is replaced by ``continuous''.
    What if ``continuous'' was replaced by ``uniformly continuous''?
\end{exercise}

\begin{proof}
    Let \(d_1 = d_{l^1}|_{\mathbf{R} \times \mathbf{R}}\).
    We first show that \(\sum_{n = 1}^N f^{(n)}\) is bounded on \(X\) with respect to \(d_1\) for each \(N \in \mathbf{Z}^+\).
    Suppose that \(f^{(n)}\) is bounded on \(X\) with respect to \(d_1\) for each \(n \in \mathbf{Z}^+\).
    We use induction on \(N\).
    For \(N = 1\), by hypothesis we know that \(\sum_{n = 1}^1 f^{(n)} = f^{(1)}\) is bounded on \(X\).
    Thus the base case holds.
    Suppose inductively that \(\sum_{n = 1}^N f^{(n)}\) is bounded on \(X\) with respect to \(d_1\) for some \(N \geq 1\).
    By induction hypothesis we have
    \[
        \exists\ M \in \mathbf{R}^+ : \bigg(\sum_{n = 1}^N f^{(n)}\bigg)(X) \subseteq [-M, M].
    \]
    By hypothesis we know that \(f^{(N + 1)}\) is bounded on \(X\) with respect to \(d_1\), thus we have
    \[
        \exists\ M' \in \mathbf{R}^+ : f^{(N + 1)}(X) \subseteq [-M', M'].
    \]
    Then we have
    \begin{align*}
        \bigg(\sum_{n = 1}^{N + 1} f^{(n)}\bigg)(X) & = \bigg\{\sum_{n = 1}^{N + 1} f^{(n)}(x) : x \in X\bigg\}            \\
                                                    & = \bigg\{\sum_{n = 1}^N f^{(n)}(x) + f^{(N + 1)}(x) : x \in X\bigg\} \\
                                                    & \subseteq [-(M + M'), M + M'].
    \end{align*}
    This closes the induction.

    Next we show that \(\sum_{n = 1}^N f^{(n)}\) is continuous from \((X, d_X)\) to \((\mathbf{R}, d_1)\) for each \(N \in \mathbf{Z}^+\).
    Suppose that \(f^{(n)}\) is continuous from \((X, d_X)\) to \((\mathbf{R}, d_1)\) for each \(n \in \mathbf{Z}^+\).
    We use induction on \(N\).
    For \(N = 1\), by hypothesis we know that \(\sum_{n = 1}^1 f^{(n)} = f^{(1)}\) is continuous from \((X, d_X)\) to \((\mathbf{R}, d_1)\).
    Thus the base case holds.
    Suppose inductively that \(\sum_{n = 1}^N f^{(n)}\) is continuous from \((X, d_X)\) to \((\mathbf{R}, d_1)\) for some \(N \geq 1\).
    Then by Additional Corollary \ref{ac 2.2.1}
    \[
        \sum_{n = 1}^{N + 1} f^{(n)} = \bigg(\sum_{n = 1}^N f^{(n)}\bigg) \oplus f^{(N + 1)}
    \]
    is continuous from \((X, d_X)\) to \((\mathbf{R}, d_1)\).
    This closes the induction.

    Finally we show that \(\sum_{n = 1}^N f^{(n)}\) is uniformly continuous from \((X, d_X)\) to \((\mathbf{R}, d_1)\) for each \(N \in \mathbf{Z}^+\).
    Suppose that \(f^{(n)}\) is uniformly continuous from \((X, d_X)\) to \((\mathbf{R}, d_1)\) for each \(n \in \mathbf{Z}^+\).
    We use induction on \(N\).
    For \(N = 1\), by hypothesis we know that \(\sum_{n = 1}^1 f^{(n)} = f^{(1)}\) is uniformly continuous from \((X, d_X)\) to \((\mathbf{R}, d_1)\).
    Thus the base case holds.
    Suppose inductively that \(\sum_{n = 1}^N f^{(n)}\) is uniformly continuous from \((X, d_X)\) to \((\mathbf{R}, d_1)\) for some \(N \geq 1\).
    Then by Exercise \ref{ex 2.3.5}
    \[
        \sum_{n = 1}^{N + 1} f^{(n)} = \bigg(\sum_{n = 1}^N f^{(n)}\bigg) \oplus f^{(N + 1)}
    \]
    is uniformly continuous from \((X, d_X)\) to \((\mathbf{R}, d_1)\).
    This closes the induction.
\end{proof}

\begin{exercise}\label{ex 3.5.2}
    Prove Theorem \ref{3.5.7}.
\end{exercise}

\begin{proof}
    See Theorem \ref{3.5.7}.
\end{proof}
\section{Uniform convergence and integration}\label{sec 3.6}

\begin{theorem}\label{3.6.1}
    Let \([a, b]\) be an interval, and for each integer \(n \geq 1\), let \(f^{(n)} : [a, b] \to \mathbf{R}\) be a Riemann-integrable function.
    Suppose \(f^{(n)}\) converges uniformly on \([a, b]\) to a function \(f : [a, b] \to \mathbf{R}\).
    Then \(f\) is also Riemann integrable, and
    \[
        \lim_{n \to \infty} \int_{[a, b]} f^{(n)} = \int_{[a, b]} f.
    \]
\end{theorem}

\begin{proof}
    We first show that \(f\) is Riemann integrable on \([a, b]\).
    This is the same as showing that the upper and lower Riemann integrals of \(f\) match:
    \(\underline{\int}_{[a, b]} f = \overline{\int}_{[a, b]} f\).

    Let \(\varepsilon > 0\).
    Since \(f^{(n)}\) converges uniformly to \(f\), we see that there exists an \(N > 0\) such that \(\abs*{f^{(n)}(x) - f(x)} < \varepsilon\) for all \(n > N\) and \(x \in [a, b]\).
    In particular we have
    \[
        f^{(n)}(x) - \varepsilon < f(x) < f^{(n)}(x) + \varepsilon
    \]
    for all \(x \in [a, b]\).
    Integrating this on \([a, b]\) we obtain
    \[
        \underline{\int}_{[a, b]} (f^{(n)} - \varepsilon) \leq \underline{\int}_{[a, b]} f \leq \overline{\int}_{[a, b]} f \leq \overline{\int}_{[a, b]} (f^{(n)} + \varepsilon).
    \]
    Since \(f^{(n)}\) is assumed to be Riemann integrable, we thus see
    \[
        \Bigg(\int_{[a, b]} f^{(n)}\Bigg) - \varepsilon (b - a) \leq \underline{\int}_{[a, b]} f \leq \overline{\int}_{[a, b]} f \leq \Bigg(\int_{[a, b]} f^{(n)}\Bigg) + \varepsilon (b - a).
    \]
    In particular, we see that
    \[
        0 \leq \overline{\int}_{[a, b]} f - \underline{\int}_{[a, b]} f \leq 2 \varepsilon (b - a).
    \]
    Since this is true for every \(\varepsilon > 0\), we obtain \(\underline{\int}_{[a, b]} f = \overline{\int}_{[a, b]} f\) as desired.

    The above argument also shows that for every \(\varepsilon > 0\) there exists an \(N > 0\) such that
    \[
        \abs*{\int_{[a, b]} f^{(n)} - \int_{[a, b]} f} \leq \varepsilon (b - a)
    \]
    for all \(n \geq N\).
    Since \(\varepsilon\) is arbitrary, we see that \(\int_{[a, b]} f^{(n)}\) converges to \(\int_{[a, b]} f\) as desired.
\end{proof}

\begin{note}
    To rephrase Theorem \ref{3.6.1}:
    we can rearrange limits and integrals (on compact intervals \([a, b]\)),
    \[
        \lim_{n \to \infty} \int_{[a, b]} f^{(n)} = \int_{[a, b]} \lim_{n \to \infty} f^{(n)},
    \]
    \emph{provided that} the convergence is uniform.
\end{note}

\begin{corollary}\label{3.6.2}
    Let \([a, b]\) be an interval, and let \((f^{(n)})_{n = 1}^\infty\) be a sequence of Riemann integrable functions on \([a, b]\) such that the series \(\sum_{n = 1}^\infty f^{(n)}\) is uniformly convergent.
    Then we have
    \[
        \sum_{n = 1}^\infty \int_{[a, b]} f^{(n)} = \int_{[a, b]} \sum_{n = 1}^\infty f^{(n)}.
    \]
\end{corollary}

\begin{proof}
    By Theorem 11.4.1(a) in Analysis I we know that
    \[
        \forall N \in \mathbf{Z}^+, \int_{[a, b]} \sum_{n = 1}^N f^{(n)} = \sum_{n = 1}^N \int_{[a, b]} f^{(n)}.
    \]
    Let \(f : [a, b] \to \mathbf{R}\) be the function such that \(\sum_{n = 1}^\infty f^{(n)}\) converges uniformly to \(f\) on \([a, b]\) with respect to \(d_{l^1}|_{\mathbf{R} \times \mathbf{R}}\).
    By Theorem \ref{3.6.1} we have
    \[
        \sum_{n = 1}^\infty \int_{[a, b]} f^{(n)} = \lim_{N \to \infty} \sum_{n = 1}^N \int_{[a, b]} f^{(n)} = \lim_{N \to \infty} \int_{[a, b]} \sum_{n = 1}^N f^{(n)} = \int_{[a, b]} f = \int_{[a, b]} \sum_{n = 1}^\infty f^{(n)}.
    \]
\end{proof}

\begin{note}
    Corollary \ref{3.6.2} works particularly well in conjunction with the Weierstrass \(M\)-test
    (Theorem \ref{3.5.7}).
\end{note}

\exercisesection

\begin{exercise}\label{ex 3.6.1}
    Use Theorem \ref{3.6.1} to prove Corollary \ref{3.6.2}.
\end{exercise}

\begin{proof}
    See Corollary \ref{3.6.2}.
\end{proof}
\section{Uniform convergence and derivatives}\label{sec 3.7}

\begin{note}
    In particular we have
    \[
        \frac{d}{dx} \lim_{n \to \infty} f_n(x) \neq \lim_{n \to \infty} \frac{d}{dx} f_n(x)
    \]
    So, in summary, uniform convergence of the functions \(f_n\) says nothing about the convergence of the derivatives \(f_n'\).
\end{note}

\begin{theorem}\label{3.7.1}
    Let \([a, b]\) be an interval, and for every integer \(n \geq 1\), let \(f_n : [a, b] \to \mathbf{R}\) be a differentiable function whose derivative \(f_n' : [a, b] \to \mathbf{R}\) is continuous.
    Suppose that the derivatives \(f_n'\) converge uniformly to a function \(g : [a, b] \to \mathbf{R}\).
    Suppose also that there exists a point \(x_0\) such that the limit \(\lim_{n \to \infty} f_n(x_0)\) exists.
    Then the functions \(f_n\) converge uniformly to a differentiable function \(f\), and the derivative of \(f\) equals \(g\).
\end{theorem}

\begin{proof}
    Since \(f_n'\) is continuous, by Corollary 11.5.2 in Analysis I we know that \(f_n'\) is Riemann integrable.
    We see from the fundamental theorem of calculus (Theorem 11.9.4 in Analysis I) that
    \[
        f_n(x) - f_n(x_0) = \int_{[x_0, x]} f_n'
    \]
    when \(x \in [x_0, b]\), and
    \[
        f_n(x) - f_n(x_0) = -\int_{[x, x_0]} f_n'
    \]
    when \(x \in [a, x_0]\).
    Let \(L\) be the limit of \(f_n(x_0)\) as \(n \to \infty\):
    \[
        L \coloneqq \lim_{n \to \infty} f_n(x_0).
    \]
    By hypothesis, \(L\) exists.
    Now, since each \(f_n'\) is continuous on \([a, b]\), and \(f_n'\) converges uniformly to \(g\), we see by Corollary \ref{3.3.2} that \(g\) is also continuous.
    By Theorem \ref{3.6.1} we have
    \[
        \forall\ x \in [a, b], \lim_{n \to \infty} \big(f_n(x) - f_n(a)\big) = \lim_{n \to \infty} \int_{[a, x]} f_n' = \int_{[a, x]} \big(\lim_{n \to \infty} f_n'\big) = \int_{[a, x]} g.
    \]
    Now define the function \(f : [a, b] \to \mathbf{R}\) by setting
    \[
        f(x) \coloneqq L - \int_{[a, x_0]} g + \int_{[a, x]} g
    \]
    for all \(x \in [a, b]\).
    To finish the proof, we have to show that \(f_n\) converges uniformly to \(f\), and that \(f\) is differentiable with derivative \(g\).

    We know that \(a \neq b\) since if \(a = b\), then we have \(x_0 = a = b\) and
    \[
        \forall\ n \in \mathbf{Z}^+, \lim_{x \to x_0; x \in \{x_0\} \setminus \{x_0\}} \frac{f_n(x) - f_n(x_0)}{x - x_0} \text{ is undefined}
    \]
    which contradict to the hypothesis that \(f_n\) is differentiable on \([a, b]\).
    Observe that
    \begin{align*}
                 & L = \lim_{n \to \infty} f_n(x_0)                                                                                                                \\
        \implies & \forall\ \varepsilon \in \mathbf{R}^+, \exists\ N_1 \in \mathbf{Z}^+ : \forall\ n \geq N_1, \abs*{f_n(x_0) - L} < \frac{\varepsilon}{3(b - a)}.
    \end{align*}
    Now we fix one pair of \(\varepsilon\) and \(N_1\).
    Since \((f_n')_{n = 1}^\infty\) converges uniformly to \(g\) on \([a, b]\) with respect to \(d_{l^1}|_{\mathbf{R} \times \mathbf{R}}\), by Definition \ref{3.2.7} we have
    \begin{align*}
                 & \exists\ N_2 \in \mathbf{Z}^+ : \forall\ n \geq N_2, \forall\ x \in [a, b],                                                     \\
                 & \abs*{f_n'(x) - g(x)} < \frac{\varepsilon}{3(b - a)}                                                                            \\
        \implies & \exists\ N_2 \in \mathbf{Z}^+ : \forall\ n \geq N_2, \forall\ x \in [a, b],                                                     \\
                 & \frac{-\varepsilon}{3(b - a)} < f_n'(x) - g(x) < \frac{\varepsilon}{3(b - a)}                                                   \\
        \implies & \exists\ N_2 \in \mathbf{Z}^+ : \forall\ n \geq N_2, \forall\ x \in [a, b],                                                     \\
                 & \frac{-\varepsilon (x - a)}{3(b - a)} \leq \int_{[a, x]} f_n'(x) - \int_{[a, x]} g(x) \leq \frac{\varepsilon (x - a)}{3(b - a)} \\
        \implies & \exists\ N_2 \in \mathbf{Z}^+ : \forall\ n \geq N_2, \forall\ x \in [a, b],                                                     \\
                 & \frac{-\varepsilon (x - a)}{3(b - a)} \leq f_n(x) - f_n(a) - \int_{[a, x]} g(x) \leq \frac{\varepsilon (x - a)}{3(b - a)}       \\
        \implies & \exists\ N_2 \in \mathbf{Z}^+ : \forall\ n \geq N_2, \forall\ x \in [a, b],                                                     \\
                 & \abs*{f_n(x) - f_n(a) - \int_{[a, x]} g(x)} \leq \frac{\varepsilon \abs*{x - a}}{3(b - a)}.
    \end{align*}
    Let \(N = \max(N_1, N_2)\).
    Then we have
    \begin{align*}
         & \forall\ n \geq N, \forall\ x \in [a, b], \abs*{f_n(x) - f(x)}                                                           \\
         & = \abs*{f_n(x) - f_n(x_0) + f_n(x_0) - f_n(a) + f_n(a) - L + \int_{[a, x_0]} g - \int_{[a, x]} g}                        \\
         & \leq \abs*{f_n(x) - f_n(a) - \int_{[a, x]} g} + \abs*{f_n(x_0) - L} + \abs*{f_n(x_0) - f_n(a) - \int_{[a, x_0]} g}       \\
         & < \frac{\varepsilon \abs*{x - a}}{3(b - a)} + \frac{\varepsilon \abs*{x_0 - a}}{3(b - a)} + \frac{\varepsilon}{3(b - a)} \\
         & < \frac{\varepsilon}{3} + \frac{\varepsilon}{3} + \frac{\varepsilon}{3} = \varepsilon.
    \end{align*}
    Since \(\varepsilon\) is arbitrary, we have
    \[
        \forall\ \varepsilon \in \mathbf{R}^+, \exists\ N \in \mathbf{Z}^+ : \forall\ n \geq N, \forall\ x \in [a, b], \abs*{f_n(x) - f(x)} < \varepsilon
    \]
    and by Definition \ref{3.2.7} \((f_n)_{n = 1}^\infty\) converges uniformly to \(f\) with respect to \(d_{l^1}|_{\mathbf{R} \times \mathbf{R}}\).

    Since \(f_n\) is continuous on \([a, b]\) for each \(n \in \mathbf{Z}^+\), by Corollary \ref{3.3.2} we know that \(f\) is also continuous on \([a, b]\).
    Since \(g\) is continuous on \([a, b]\), by fundamental theorem of calculus (Theorem 11.9.1 in Analysis) we know that
    \[
        \forall\ x \in X, G(x) = \int_{[a, x]} g \text{ is differentiable at } x.
    \]
    Since \(L + \int_{[a, x_0]} g\) is constant, we know that
    \[
        \forall\ x \in X, f(x) = L + \int_{[a, x_0]} g + G(x) = L + \int_{[a, x_0]} g + \int_{[a, x]} g \text{ is differentiable at } x
    \]
    and by fundamental theorem of calculus (Theorem 11.9.1 in Analysis) we have
    \[
        \forall\ x \in X, f'(x) = \bigg(\int_{[a, x]} g\bigg)' = g(x).
    \]
\end{proof}

\begin{note}
    Informally, Theorem \ref{3.7.1} says that if \(f_n'\) converges uniformly, and \(f_n(x_0)\) converges for some \(x_0\), then \(f_n\) also converges uniformly, and
    \[
        \frac{d}{dx} \lim_{n \to \infty} f_n(x) = \lim_{n \to \infty} \frac{d}{dx} f_n(x)
    \]
\end{note}

\begin{remark}\label{3.7.2}
    It turns out that Theorem \ref{3.7.1} is still true when the functions \(f_n'\) are not assumed to be continuous, but the proof is more difficult;
    see Exercise \ref{ex 3.7.2}.
\end{remark}

\begin{corollary}\label{3.7.3}
    Let \([a, b]\) be an interval, and for every integer \(n \geq 1\), let \(f_n : [a, b] \to \mathbf{R}\) be a differentiable function whose derivative \(f_n' : [a, b] \to \mathbf{R}\) is continuous.
    Suppose that the series \(\sum_{n = 1}^\infty \norm*{f_n'}_\infty\) is absolutely convergent, where
    \[
        \norm*{f_n'}_\infty \coloneqq \sup_{x \in [a, b]} \abs*{f_n'(x)}
    \]
    is the sup norm of \(f_n'\), as defined in Definition \ref{3.5.5}.
    Suppose also that the series \(\sum_{n = 1}^\infty f_n(x_0)\) is convergent for some \(x_0 \in [a, b]\).
    Then the series \(\sum_{n = 1}^\infty f_n\) converges uniformly on \([a, b]\) to a differentiable function, and in fact
    \[
        \frac{d}{dx} \sum_{n = 1}^\infty f_n(x) = \sum_{n = 1}^\infty \frac{d}{dx} f_n(x)
    \]
    for all \(x \in [a, b]\).
\end{corollary}

\begin{proof}
    Let \(F_N = \sum_{n = 1}^N f_n\) for each \(N \in \mathbf{Z}^+\).
    Then by Theorem 10.1.13(c) in Analysis I we have
    \[
        \forall\ N \in \mathbf{Z}^+, F_N' = \bigg(\sum_{n = 1}^N f_n\bigg)' = \sum_{n = 1}^N f_n'.
    \]
    Since \(f_n'\) is continuous on \([a, b]\) for each \(n \in \mathbf{Z}^+\), by Proposition 9.6.7 in Analysis I we know that \(f_n' \in B\big([a, b] \to \mathbf{R}\big)\) and thus \(f_n' \in C\big([a, b] \to \mathbf{R}\big)\).
    By Exercise \ref{ex 3.5.1} we know that
    \[
        \forall\ N \in \mathbf{Z}^+, F_N' = \sum_{n = 1}^N f_n' \in C([a, b] \to \mathbf{R}).
    \]
    Since \(\sum_{n = 1}^\infty \norm*{f_n'}_\infty\) converges and \(f_n' \in C\big([a, b] \to \mathbf{R}\big)\) for each \(n \in \mathbf{Z}^+\), by Theorem \ref{3.5.7} we know that there exists some \(G : [a, b] \to \mathbf{R}\) such that \(\big(\sum_{n = 1}^N f_n'\big)_{N = 1}^\infty\) converges uniformly to \(G\) on \([a, b]\) with respect to \(d_{l^1}|_{\mathbf{R} \times \mathbf{R}}\).
    Equivalently, \((F_N')_{N = 1}^\infty\) converges uniformly to \(G\) on \([a, b]\) with respect to \(d_{l^1}|_{\mathbf{R} \times \mathbf{R}}\).
    Since
    \[
        \sum_{n = 1}^\infty f_n(x_0) = \lim_{N \to \infty} \sum_{n = 1}^N f_n(x_0) = \lim_{N \to \infty} F_N(x_0),
    \]
    by Theorem \ref{3.7.1} we know that there exists some \(F : [a, b] \to \mathbf{R}\) such that \((F_N)_{N = 1}^\infty\) converges uniformly to \(F\) on \([a, b]\) with respect to \(d_{l^1}|_{\mathbf{R} \times \mathbf{R}}\) and \(F' = G\).
    Then we have
    \begin{align*}
                 & \forall\ x \in [a, b], \begin{cases}
            F(x) = \lim_{N \to \infty} F_N(x) = \lim_{N \to \infty} \sum_{n = 1}^N f_n(x) = \sum_{n = 1}^\infty f_n(x)    \\
            G(x) = \lim_{N \to \infty} F_N'(x) = \lim_{N \to \infty} \sum_{n = 1}^N f_n'(x) = \sum_{n = 1}^\infty f_n'(x) \\
            F'(x) = G(x)
        \end{cases}                                             \\
        \implies & \forall\ x \in [a, b], \bigg(\sum_{n = 1}^\infty f_n\bigg)'(x) = \sum_{n = 1}^\infty f_n'(x) \\
        \implies & \bigg(\sum_{n = 1}^\infty f_n\bigg)' = \sum_{n = 1}^\infty f_n'.
    \end{align*}
\end{proof}

\begin{note}
    Example \ref{3.7.4} was discovered by Weierstrass.
\end{note}

\begin{example}\label{3.7.4}
    Let \(f : \mathbf{R} \to \mathbf{R}\) be the function
    \[
        f(x) \coloneqq \sum_{n = 1}^\infty 4^{-n} \cos(32^n \pi x).
    \]
    Note that this series is uniformly convergent, thanks to the Weierstrass \(M\)-test, and since each individual function \(4^{-n} \cos(32^n \pi x)\) is continuous, the function \(f\) is also continuous.
    However, it is not differentiable;
    in fact it is a \emph{nowhere differentiable function}, one which is not differentiable at any point, despite being continuous everywhere!
\end{example}

\exercisesection

\begin{exercise}\label{ex 3.7.1}
    Complete the proof of Theorem \ref{3.7.1}.
    Compare this theorem with Example 1.2.10 in Analysis I, and explain why this example does not contradict the theorem.
\end{exercise}

\begin{exercise}\label{ex 3.7.2}
    Prove Theorem \ref{3.7.1} without assuming that \(f_n'\) is continuous.
    (This means that you cannot use the fundamental theorem of calculus.
    However, the mean value theorem (Corollary 10.2.9 in Analysis I) is still available.
    Use this to show that if \(d_\infty(f_n', f_m') \leq \varepsilon\), then \(\abs*{\big(f_n(x) - f_m(x)\big) - \big(f_n(x_0) - f_m(x_0)\big)} \leq \varepsilon \abs*{x - x_0}\) for all \(x \in [a, b]\), and then use this to complete the proof of Theorem \ref{3.7.1}.)
\end{exercise}

\begin{exercise}\label{ex 3.7.3}
    Prove Corollary \ref{3.7.3}.
\end{exercise}

\begin{proof}
    See Corollary \ref{3.7.3}.
\end{proof}