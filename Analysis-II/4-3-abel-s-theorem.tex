\section{Abel's theorem}\label{sec 4.3}

\begin{note}
  Let \(f(x) = \sum_{n = 0}^\infty c_n (x - a)^n\) be a power series centered at \(a\) with a radius of convergence \(0 < R < \infty\) strictly between \(0\) and infinity.
  From \cref{4.1.6} we know that the power series converges absolutely whenever \(\abs{x - a} < R\), and diverges when \(\abs{x - a} > R\).
  However, at the boundary \(\abs{x - a} = R\) the situation is more complicated;
  the series may either converge or diverge (see \cref{ex 4.1.2}).
  However, if the series does converge at the boundary point, then it is reasonably well behaved;
  in particular, it is continuous at that boundary point.
\end{note}

\begin{theorem}[Abel's theorem]\label{4.3.1}
  Let \(f(x) = \sum_{n = 0}^\infty c_n (x - a)^n\) be a power series centered at \(a\) with radius of convergence \(0 < R < \infty\).
  If the power series converges at \(a + R\), then \(f\) is continuous at \(a + R\), i.e.
  \[
    \lim_{x \to a + R ; x \in (a - R, a + R)} \sum_{n = 0}^\infty c_n (x - a)^n = \sum_{n = 0}^\infty c_n R^n.
  \]
  Similarly, if the power series converges at \(a - R\), then \(f\) is continuous at \(a - R\), i.e.
  \[
    \lim_{x \to a - R ; x \in (a - R, a + R)} \sum_{n = 0}^\infty c_n (x - a)^n = \sum_{n = 0}^\infty c_n (-R)^n.
  \]
\end{theorem}

\begin{proof}
  It will suffice to prove the first claim, i.e., that
  \[
    \lim_{x \to a + R ; x \in (a - R, a + R)} \sum_{n = 0}^\infty c_n (x - a)^n = \sum_{n = 0}^\infty c_n R^n.
  \]
  whenever the sum \(\sum_{n = 0}^\infty c_n R^n\) converges;
  the second claim will then follow by replacing \(c_n\) by \((-1)^n c_n\) in the above claim.
  If we make the substitutions \(d_n \coloneqq c_n R^n\) and \(y \coloneqq \frac{x - a}{R}\), then the above claim can be rewritten as
  \[
    \lim_{y \to 1 ; y \in (-1, 1)} \sum_{n = 0}^\infty d_n y^n = \sum_{n = 0}^\infty d_n
  \]
  whenever the sum \(\sum_{n = 0}^\infty d_n\) converges.

  Write \(D \coloneqq \sum_{n = 0}^\infty d_n\), and for every \(N \geq 0\) write
  \[
    S_N \coloneqq \bigg(\sum_{n = 0}^{N - 1} d_n\bigg) - D
  \]
  so in particular \(S_0 = -D\).
  Then observe that \(\lim_{N \to \infty} S_N = 0\), and that \(d_n = S_{n + 1} - S_n\).
  Thus for any \(y \in (-1, 1)\) we have
  \[
    \sum_{n = 0}^\infty d_n y^n = \sum_{n = 0}^\infty (S_{n + 1} - S_n) y^n.
  \]
  Applying the summation by parts formula (\cref{4.3.2}), and noting that \(\lim_{n \to \infty} y^n = 0\), we obtain
  \[
    \sum_{n = 0}^\infty d_n y^n = - S_0 y^0 - \sum_{n = 0}^\infty S_{n + 1} (y^{n + 1} - y^n).
  \]
  Observe that \(- S_0 y^0 = +D\).
  Thus to finish the proof of Abel's theorem,
  it will suffice to show that
  \[
    \lim_{y \to 1 ; y \in (-1, 1)} \sum_{n = 0}^\infty S_{n + 1} (y^{n + 1} - y^n) = 0.
  \]
  Since \(y\) converges to \(1\), we may as well restrict \(y\) to \([0, 1)\) instead of \((-1, 1)\);
  in particular we may take \(y\) to be positive.

  From the triangle inequality for series (Proposition 7.2.9 in Analysis I), we have
  \begin{align*}
    \abs{\sum_{n = 0}^\infty S_{n + 1} (y^{n + 1} - y^n)} & \leq \sum_{n = 0}^\infty \abs{S_{n + 1} (y^{n + 1} - y^n)} \\
                                                          & = \sum_{n = 0}^\infty \abs{S_{n + 1}} (y^n - y^{n + 1}),
  \end{align*}
  so by the squeeze test (Corollary 6.4.14 in Analysis I) it suffices to show that
  \[
    \lim_{y \to 1 ; y \in [0, 1)} \sum_{n = 0}^\infty \abs{S_{n + 1}} (y^n - y^{n + 1}) = 0.
  \]
  The expression \(\sum_{n = 0}^\infty \abs{S_{n + 1}} (y^n - y^{n + 1})\) is clearly non-negative, so it will suffice to show that
  \[
    \limsup_{y \to 1 ; y \in [0, 1)} \sum_{n = 0}^\infty \abs{S_{n + 1}} (y^n - y^{n + 1}) = 0.
  \]
  Let \(\varepsilon > 0\).
  Since \(S_n\) converges to \(0\), there exists an \(N\) such that \(\abs{S_n} \leq \varepsilon\) for all \(n > N\).
  Thus we have
  \[
    \sum_{n = 0}^\infty \abs{S_{n + 1}} (y^n - y^{n + 1}) \leq \sum_{n = 0}^N \abs{S_{n + 1}} (y^n - y^{n + 1}) + \sum_{n = N + 1}^\infty \varepsilon (y^n - y^{n + 1}).
  \]
  The last summation is a telescoping series, which sums to \(\varepsilon y^{N + 1}\) (See Lemma 7.2.15 in Analysis I, recalling from Lemma 6.5.2 in Analysis I that \(y^n \to 0\) as \(n \to \infty\)), and thus
  \[
    \sum_{n = 0}^\infty \abs{S_{n + 1}} (y^n - y^{n + 1}) \leq \sum_{n = 0}^N \abs{S_{n + 1}} (y^n - y^{n + 1}) + \varepsilon y^{N + 1}.
  \]
  Now take limits as \(y \to 1\).
  Observe that \(y^n - y^{n + 1} \to 0\) as \(y \to 1\) for every \(n \in 0, 1, \dots, N\).
  Since we can interchange limits and \emph{finite} sums (Exercise 7.1.5 in Analysis I), we thus have
  \[
    \limsup_{y \to 1 ; y \in [0, 1)} \sum_{n = 0}^\infty \abs{S_{n + 1}} (y^n - y^{n + 1}) \leq \varepsilon.
  \]
  But \(\varepsilon > 0\) was arbitrary, and thus we must have
  \[
    \limsup_{y \to 1 ; y \in [0, 1)} \sum_{n = 0}^\infty \abs{S_{n + 1}} (y^n - y^{n + 1}) = 0
  \]
  since the left-hand side must be non-negative.
  The claim follows.
\end{proof}

\begin{lemma}[Summation by parts formula]\label{4.3.2}
  Let \((a_n)_{n = 0}^\infty\) and \((b_n)_{n = 0}^\infty\) be sequences of real numbers which converge to limits \(A\) and \(B\) respectively, i.e., \(\lim_{n \to \infty} a_n = A\) and \(\lim_{n \to \infty} b_n = B\).
  Suppose that the sum \(\sum_{n = 0}^\infty (a_{n + 1} - a_n) b_n\) is convergent.
  Then the sum \(\sum_{n = 0}^\infty a_{n + 1} (b_{n + 1} - b_n)\) is also convergent, and
  \[
    \sum_{n = 0}^\infty (a_{n + 1} - a_n) b_n = AB - a_0 b_0 - \sum_{n = 0}^\infty a_{n + 1} (b_{n + 1} - b_n).
  \]
\end{lemma}

\begin{proof}
  Since
  \begin{align*}
    \forall N \in \N, & \sum_{n = 0}^N (a_{n + 1} - a_n) b_n + \sum_{n = 0}^N a_{n + 1} (b_{n + 1} - b_n) \\
                      & = \sum_{n = 0}^N a_{n + 1} (b_n - a_n b_n + a_{n + 1} b_{n + 1} - a_{n + 1} b_n)  \\
                      & = \sum_{n = 0}^N a_{n + 1} (b_{n + 1} - a_n b_n)                                  \\
                      & = a_{N + 1} b_{N + 1} - a_0 b_0
  \end{align*}
  and
  \begin{align*}
             & \begin{cases}
                 \lim_{n \to \infty} a_n = A \\
                 \lim_{n \to \infty} b_n = B
               \end{cases}       \\
    \implies & \lim_{n \to \infty} a_n b_n = AB,
  \end{align*}
  we have
  \begin{align*}
     & \lim_{N \to \infty} \bigg(\sum_{n = 0}^N (a_{n + 1} - a_n) b_n + \sum_{n = 0}^N a_{n + 1} (b_{n + 1} - b_n)\bigg) \\
     & = \lim_{N \to \infty} (a_{N + 1} b_{N + 1} - a_0 b_0)                                                             \\
     & = \lim_{N \to \infty} (a_{N + 1} b_{N + 1}) - a_0 b_0                                                             \\
     & = AB - a_0 b_0.
  \end{align*}
  Thus
  \begin{align*}
     & \bigg(\sum_{n = 0}^\infty (a_{n + 1} - a_n) b_n\bigg) - AB + a_0 b_0                                                                                                                       \\
     & = \lim_{N \to \infty} \bigg(\sum_{n = 0}^N (a_{n + 1} - a_n) b_n\bigg) - \lim_{N \to \infty} \bigg(\sum_{n = 0}^N (a_{n + 1} - a_n) b_n + \sum_{n = 0}^N a_{n + 1} (b_{n + 1} - b_n)\bigg) \\
     & = \lim_{N \to \infty} - \bigg(\sum_{n = 0}^N a_{n + 1} (b_{n + 1} - b_n)\bigg)                                                                                                             \\
     & = - \lim_{N \to \infty} \bigg(\sum_{n = 0}^N a_{n + 1} (b_{n + 1} - b_n)\bigg)                                                                                                             \\
     & = - \sum_{n = 0}^\infty a_{n + 1} (b_{n + 1} - b_n)
  \end{align*}
  and
  \[
    \sum_{n = 0}^\infty (a_{n + 1} - a_n) b_n = AB - a_0 b_0 - \sum_{n = 0}^\infty a_{n + 1} (b_{n + 1} - b_n).
  \]
\end{proof}

\begin{remark}\label{4.3.3}
  One should compare this formula with the more well-known \emph{integration by parts formula}
  \[
    \int_0^\infty f'(x) g(x) \; dx = f(x) g(x) |_0^\infty - \int_0^\infty f(x) g'(x) \; dx,
  \]
  see Proposition 11.10.1 in Analysis I.
\end{remark}

\exercisesection

\begin{exercise}\label{ex 4.3.1}
  Prove \cref{4.3.2}.
\end{exercise}

\begin{proof}
  See \cref{4.3.2}.
\end{proof}