\section{The exponential and logarithm functions}\label{sec 4.5}

\begin{definition}[Exponential function]\label{4.5.1}
    For every real number \(x\), we define the \emph{exponential function} \(\exp(x)\) to be the real number
    \[
        \exp(x) \coloneqq \sum_{n = 0}^\infty \frac{x^n}{n!}.
    \]
\end{definition}

\begin{theorem}[Basic properties of exponential]\label{4.5.2}
    \quad
    \begin{enumerate}
        \item For every real number \(x\), the series \(\sum_{n = 0}^\infty \frac{x^n}{n!}\) is absolutely convergent.
              In particular, \(\exp(x)\) exists and is real for every \(x \in \mathbf{R}\), the power series \(\sum_{n = 0}^\infty \frac{x^n}{n!}\) has an infinite radius of convergence, and \(\exp\) is a real analytic function on \((-\infty, \infty)\).
        \item \(\exp\) is differentiable on \(R\), and for every \(x \in \mathbf{R}\), \(\exp'(x) = \exp(x)\).
        \item \(\exp\) is continuous on \(R\), and for every interval \([a, b]\), we have \(\int_{[a, b]} \exp(x) \; dx = \exp(b) - \exp(a)\).
        \item For every \(x, y \in \mathbf{R}\), we have \(\exp(x + y) = \exp(x) \exp(y)\).
        \item We have \(\exp(0) = 1\).
              Also, for every \(x \in \mathbf{R}\), \(\exp(x)\) is positive, and \(\exp(-x) = 1 / \exp(x)\).
        \item \(\exp\) is strictly monotone increasing:
              in other words, if \(x, y\) are real numbers, then we have \(\exp(y) > \exp(x)\) if and only if \(y > x\).
    \end{enumerate}
\end{theorem}

\begin{proof}{(a)}
    If \(x = 0\), then we have
    \[
        1 = 1 + 0 = \frac{0^0}{0!} + \sum_{n = 1}^\infty \frac{0^n}{n!} = \sum_{n = 0}^\infty \frac{0^n}{n!} = \exp(0).
    \]
    So suppose that \(x \in \mathbf{R} \setminus \{0\}\).
    Since
    \begin{align*}
                 & \limsup_{n \to \infty} \abs*{\frac{x^{n + 1}}{(n + 1)!} \frac{n!}{x^n}} = \limsup_{n \to \infty} \frac{\abs*{x}}{n + 1} = 0 < 1                          \\
        \implies & \sum_{n = 0}^\infty \frac{x^n}{n!} \text{ is absolutely convergent},                                                            & \text{(by ratio test)}
    \end{align*}
    we know that \(\exp(x)\) exists for all \(x \in \mathbf{R} \setminus \{0\}\).
    Combine all proofs above we have
    \[
        \forall\ x \in \mathbf{R}, \begin{cases}
            \sum_{n = 0}^\infty \frac{x^n}{n!} \text{ is absolutely convergent} \\
            \exp(x) \in \mathbf{R}
        \end{cases}
    \]
    and by Definition \ref{4.2.1} \(\exp\) is real analytic on \((-\infty, \infty)\).
\end{proof}

\begin{proof}{(b)}
    By Theorem \ref{4.5.2}(a) we know that \(\exp\) is real analytic on \(\mathbf{R}\), thus by Theorem \ref{4.1.6}(d) we know that \(\exp\) is differentiable on \(\mathbf{R}\) and
    \[
        \forall\ x \in \mathbf{R}, \exp'(x) = \sum_{n = 1}^\infty n \frac{x^{n - 1}}{n!} = \sum_{n = 1}^\infty \frac{x^{n - 1}}{(n - 1)!} = \sum_{n = 0}^\infty \frac{x^n}{n!} = \exp(x).
    \]
\end{proof}

\begin{proof}{(c)}
    By Theorem \ref{4.5.2}(a) we know that \(\exp\) is real analytic on \(\mathbf{R}\), thus by Theorem \ref{4.1.6}(c) we know that \(\exp\) is continuous on \(\mathbf{R}\).
    Let \(a, b \in \mathbf{R}\) such that \(a \leq b\).
    Then by Theorem \ref{4.5.2}(a) we know that \(\exp(a)\) and \(\exp(b)\) are well-defined.
    Since \([a, b] \subseteq \mathbf{R}\), by Theorem \ref{4.1.6}(e) we know that \(\exp\) is Riemann integrable on \([a, b]\) and
    \begin{align*}
        \int_{[a, b]} \exp & = \int_a^b \exp(x) \; dx                                                                                                \\
                           & = \sum_{n = 0}^\infty \frac{1}{n!} \frac{(b - 0)^{n + 1} - (a - 0)^{n + 1}}{n + 1} & \text{(by Theorem \ref{4.1.6}(e))} \\
                           & = \sum_{n = 0}^\infty \frac{b^{n + 1} - a^{n + 1}}{(n + 1)!}                                                            \\
                           & = \sum_{n = 1}^\infty \frac{b^n - a^n}{n!}                                                                              \\
                           & = \frac{b^0 - a^0}{0!} + \sum_{n = 1}^\infty \frac{b^n - a^n}{n!}                                                       \\
                           & = \sum_{n = 0}^\infty \frac{b^n - a^n}{n!}                                                                              \\
                           & = \sum_{n = 0}^\infty \frac{b^n}{n!} - \sum_{n = 0}^\infty \frac{a^n}{n!}                                               \\
                           & = \exp(b) - \exp(a).                                                               & \text{(by Definition \ref{4.5.1})}
    \end{align*}
    Since \(a, b\) is arbitrary, we conclude that
    \[
        \forall\ [a, b] \subseteq \mathbf{R}, \int_{[a, b]} \exp = \exp(b) - \exp(a).
    \]
\end{proof}

\begin{proof}{(d)}
    If \(x = 0\), then we have
    \[
        \exp(0) = \sum_{n = 0}^\infty \frac{0^n}{n!} = \frac{0^0}{0!} + \sum_{n = 1}^\infty \frac{0^n}{n!} = 1
    \]
    and thus
    \[
        \forall\ y \in \mathbf{R}, \exp(0 + y) = \exp(y) = \exp(0) \exp(y).
    \]
    Since addition and multiplication of real numbers are commutative, we have
    \[
        \forall\ x \in \mathbf{R}, \exp(x + 0) = \exp(0 + x) = \exp(0) \exp(x) = \exp(x) \exp(0).
    \]
    So suppose that \(x, y \in \mathbf{R} \setminus \{0\}\).
    By Theorem \ref{4.5.2}(a) we know that \(\sum_{n = 0}^\infty \frac{x^n}{n!}\) and \(\sum_{n = 0}^\infty \frac{y^n}{n!}\) are absolutely convergent.
    Thus we know that
    \begin{align*}
        X & = \sum_{n = 0}^\infty \abs*{\frac{x^n}{n!}} \\
        Y & = \sum_{n = 0}^\infty \abs*{\frac{y^n}{n!}}
    \end{align*}
    are well-defined.
    Since
    \begin{align*}
        \forall\ N \in \mathbf{N}, & \sum_{n = 0}^N \sum_{m = 0}^\infty \abs*{\frac{x^n y^m}{n! m!}}                              \\
                                   & = \sum_{n = 0}^N \bigg(\abs*{\frac{x^n}{n!}} \sum_{m = 0}^\infty \abs*{\frac{y^m}{m!}}\bigg) \\
                                   & = \sum_{n = 0}^N \bigg(\abs*{\frac{x^n}{n!}} Y\bigg)                                         \\
                                   & = Y \sum_{n = 0}^N \abs*{\frac{x^n}{n!}}                                                     \\
                                   & \leq Y X,
    \end{align*}
    and \((\sum_{n = 0}^N \sum_{m = 0}^\infty \abs*{\frac{x^n y^m}{n! m!}})_{N = 0}^\infty\) is monotone increasing, we know that \(\sum_{n = 0}^\infty \sum_{m = 0}^\infty \frac{x^n y^m}{n! m!}\) is absolutely convergent.
    Now we define
    \[
        \forall\ n \in \mathbf{Z}, c_n = \begin{cases}
            \frac{1}{n!} & \text{if } n \geq 0; \\
            0            & \text{if } n < 0.
        \end{cases}
    \]
    Then we have
    \begin{align*}
         & \exp(x) \exp(y)                                                                                                                            \\
         & = \sum_{n = 0}^\infty \sum_{m = 0}^\infty \frac{x^n y^m}{n! m!}                                                                            \\
         & = \sum_{n = 0}^\infty \sum_{m = 0}^\infty c_n c_m x^n y^m                                                                                  \\
         & = \sum_{n = 0}^\infty \sum_{m = n}^\infty c_n c_{m - n} x^n y^{m - n}                                                                      \\
         & = \sum_{n = 0}^\infty \sum_{m = 0}^\infty c_n c_{m - n} x^n y^{m - n}                                & (c_{m - n} = 0 \text{ if } m < n)   \\
         & = \sum_{m = 0}^\infty \sum_{n = 0}^\infty c_n c_{m - n} x^n y^{m - n}                                & \text{(by Fubini's theorem)}        \\
         & = \sum_{m = 0}^\infty \sum_{n = 0}^m c_n c_{m - n} x^n y^{m - n}                                     & (c_{m - n} = 0 \text{ if } n > m)   \\
         & = \sum_{m = 0}^\infty \sum_{n = 0}^m \frac{x^n y^{m - n}}{n! (m - n)!}                                                                     \\
         & = \sum_{m = 0}^\infty \bigg(\frac{1}{m!} \sum_{n = 0}^m \frac{m!}{n! (m - n)!} (x^n y^{m - n})\bigg)                                       \\
         & = \sum_{m = 0}^\infty \frac{(x + y)^m}{m!}                                                           & \text{(by Exercise \ref{ex 4.2.5})} \\
         & = \exp(x + y).                                                                                       & \text{(by Definition \ref{4.5.1})}
    \end{align*}
    Combine all proofs above we conclude that
    \[
        \forall\ x, y \in \mathbf{R}, \exp(x + y) = \exp(x) \exp(y).
    \]
\end{proof}

\begin{proof}{(e)}
    We have
    \[
        \exp(0) = \sum_{n = 0}^\infty \frac{0^n}{n!} = \frac{0^0}{0!} + \sum_{n = 1}^\infty \frac{0^n}{n!} = 1
    \]
    and
    \[
        \forall\ x \in \mathbf{R}^+, \exp(x) = \sum_{n = 0}^\infty \frac{x^n}{n!} = \frac{x^0}{0!} + \sum_{n = 1}^\infty \frac{x^n}{n!} \geq 1.
    \]
    Since
    \begin{align*}
        \forall\ x \in \mathbf{R}, \exp(0) & = \exp(x - x)                                           \\
                                           & = \exp(x) \exp(-x) & \text{(by Theorem \ref{4.5.2}(d))} \\
                                           & = 1,
    \end{align*}
    we know that
    \[
        \forall\ x \in \mathbf{R}, \exp(-x) = \frac{1}{\exp(x)}.
    \]
    Thus
    \[
        \forall\ x \in \mathbf{R}^-, \exp(-x) \geq 1 \implies \exp(x) = \frac{1}{\exp(-x)} > 0.
    \]
    Combine all proofs above we conclude that
    \[
        \forall\ x \in \mathbf{R}, \exp(x) > 0.
    \]
\end{proof}

\begin{proof}{(f)}
    By Theorem \ref{4.5.2}(b)(e) we know that
    \[
        \forall\ x \in \mathbf{R}, \exp'(x) = \exp(x) > 0.
    \]
    Thus by Proposition 10.3.3 in Analysis I we know that \(\exp\) is strictly monotone increasing.
\end{proof}

\begin{note}
    One can write the exponential function in a more compact form, introducing famous \emph{Euler's number} \(e = 2.71828183 \dots\), also known as the \emph{base of the natural logarithm}.
\end{note}

\begin{definition}[Euler's number]\label{4.5.3}
    The number \(e\) is defined to be
    \[
        e = \exp(1) = \sum_{n = 0}^\infty \frac{1}{n!} = \frac{1}{0!} + \frac{1}{1!} + \frac{1}{2!} + \frac{1}{3!} + \dots.
    \]
\end{definition}

\begin{proposition}\label{4.5.4}
    For every real number \(x\), we have \(\exp(x) = e^x\).
\end{proposition}

\begin{proof}
    First we use induction on \(x\) to show that \(e^x = \exp(x)\) for all \(x \in \mathbf{N}\).
    For \(x = 0\), we have
    \begin{align*}
        e^0 & = \big(\exp(1)\big)^0 & \text{(by Definition \ref{4.5.3})} \\
            & = 1                   & \text{(by Theorem \ref{4.5.2}(a))} \\
            & = \exp(0)             & \text{(by Theorem \ref{4.5.2}(e))}
    \end{align*}
    and the base case holds.
    Suppose inductively that \(e^x = \exp(x)\) for some \(x \geq 0\).
    Then for \(x + 1\), we have
    \begin{align*}
        e^{x + 1} & = \exp(1)^{x + 1}   & \text{(by Definition \ref{4.5.3})} \\
                  & = \exp(1) \exp(1)^x & \text{(by Theorem \ref{4.5.2}(a))} \\
                  & = \exp(1) e^x       & \text{(by Definition \ref{4.5.3})} \\
                  & = \exp(1) \exp(x)   & \text{(by induction hypothesis)}   \\
                  & = \exp(x + 1)       & \text{(by Theorem \ref{4.5.2}(d))}
    \end{align*}
    and this closes the induction.

    Next we show that \(e^x = \exp(x)\) for all \(x \in \mathbf{Z}\).
    Let \(x \in \mathbf{Z}^-\).
    Since \(-x \in \mathbf{N}\), we know that
    \begin{align*}
        e^{-x} & = \exp(-x)               & \text{(from the proof above)}      \\
               & = \frac{1}{\exp(x)}      & \text{(by Theorem \ref{4.5.2}(e))} \\
               & = \big(\exp(x)\big)^{-1}
    \end{align*}
    and thus \(e^x = \exp(x)\).
    Since \(x\) is arbitrary, combine the proofs above we conclude that \(e^x = \exp(x)\) for all \(x \in \mathbf{Z}\).

    Next we show that \(e^x = \exp(x)\) for all \(x \in \mathbf{Q}\).
    Let \(x \in \mathbf{Q}\).
    Since \(x = \frac{a}{b}\) for some \(a \in \mathbf{Z}\) and \(b \in \mathbf{Z}^+\), we know that
    \begin{align*}
        e^a & = \exp(a)                           & \text{(from the proof above)}      \\
            & = \exp(\frac{ab}{b})                                                     \\
            & = \exp(\sum_{i = 1}^b \frac{a}{b})                                       \\
            & = \prod_{i = 1}^b \exp(\frac{a}{b}) & \text{(by Theorem \ref{4.5.2}(d))} \\
            & = \exp(\frac{a}{b})^b                                                    \\
            & = \exp(x)^b
    \end{align*}
    and thus
    \begin{align*}
        e^x & = e^{\frac{a}{b}}                                                             \\
            & = \big(e^a\big)^{\frac{1}{b}}                                                 \\
            & = \Big(\big(\exp(x)\big)^b\Big)^{\frac{1}{b}} & \text{(from the proof above)} \\
            & = \exp(x).
    \end{align*}
    Since \(x\) is arbitrary, we conclude that \(e^x = \exp(x)\) for all \(x \in \mathbf{Q}\).

    Finally we show that \(e^x = \exp(x)\) for all \(x \in \mathbf{R}\).
    Let \(x \in \mathbf{R}\).
    Then we know that there exists a Cauchy sequence \((q_n)_{n = 1}^\infty\) in \(Q\) such that \(\lim_{n \to \infty} q_n = x\).
    Thus
    \begin{align*}
        e^x & = \lim_{n \to \infty} e^{q_n}                                        \\
            & = \lim_{n \to \infty} \exp(q_n) & \text{(from the proof above)}      \\
            & = \exp(x).                      & \text{(by Theorem \ref{4.5.2}(c))}
    \end{align*}
    Since \(x\) is arbitrary, we conclude that \(e^x = \exp(x)\) for all \(x \in \mathbf{R}\).
\end{proof}

\begin{note}
    In light of Proposition \ref{4.5.3} we can and will use \(e^x\) and \(\exp(x)\) interchangeably.
\end{note}

\begin{note}
    Since \(e > 1\), we see that \(e^x \to +\infty\) as \(x \to +\infty\), and \(e^x \to 0\) as \(x \to -\infty\).
    From this and the intermediate value theorem (Theorem 9.7.1 in Analysis I) we see that the range of the function \(\exp\) is \((0, \infty)\).
    Since \(\exp\) is increasing, it is injective, and hence \(\exp\) is a bijection from \(\mathbf{R}\) to \((0, \infty)\), and thus has an inverse from \((0, \infty) \to \mathbf{R}\).
\end{note}

\begin{definition}[Logarithm]\label{4.5.5}
    We define the \emph{natural logarithm function}
    \[
        \log : (0, \infty) \to \mathbf{R}
    \]
    (also called \(\ln\)) to be the inverse of the exponential function.
    Thus \(\exp\big(\log(x)\big) = x\) and \(\log\big(\exp(x)\big) = x\).
\end{definition}

\begin{theorem}[Logarithm properties]\label{4.5.6}
    \quad
    \begin{enumerate}
        \item For every \(x \in (0, \infty)\), we have \(\ln'(x) = \frac{1}{x}\).
              In particular, by the fundamental theorem of calculus, we have \(\int_{[a, b]} \frac{1}{x} \; dx = \ln(b) - \ln(a)\) for any interval \([a, b]\) in \((0, \infty)\).
        \item We have \(\ln(xy) = \ln(x) + \ln(y)\) for all \(x, y \in (0, \infty)\).
        \item We have \(\ln(1) = 0\) and \(\ln(1 / x) = -\ln(x)\) for all \(x \in (0, \infty)\).
        \item For any \(x \in (0, \infty)\) and \(y \in R\), we have \(\ln(x^y) = y \ln(x)\).
        \item For any \(x \in (-1, 1)\), we have
              \[
                  \ln(1 - x) = - \sum_{n = 1}^\infty \frac{x^n}{n}.
              \]
              In particular, \(\ln\) is analytic at \(1\), with the power series expansion
              \[
                  \ln(x) = \sum_{n = 1}^\infty \frac{(-1)^{n + 1}}{n} (x - 1)^n
              \]
              for \(x \in (0, 2)\), with radius of convergence \(1\).
    \end{enumerate}
\end{theorem}

\begin{proof}{(a)}
    Since \(\exp\) is continuous and strictly monotone increasing, we see that \(\ln\) is also continuous and strictly monotone increasing (see Proposition 9.8.3 in Analysis I).
    Since \(\exp\) is also differentiable, and the derivative is never zero, we see from the inverse function theorem (Theorem 10.4.2 in Analysis I) that \(\ln\) is also differentiable.
    Thus we have
    \begin{align*}
                 & \forall\ x \in (0, \infty), \exp\big(\ln(x)\big) = x                  & \text{(by Definition \ref{4.5.5})}    \\
        \implies & \forall\ x \in (0, \infty), \ln'(x) = \frac{1}{\exp'\big(\ln(x)\big)} & \text{(Theorem 10.4.2 in Analysis I)} \\
        \implies & \forall\ x \in (0, \infty), \ln'(x) = \frac{1}{\exp\big(\ln(x)\big)}  & \text{(by Theorem \ref{4.5.2}(b))}    \\
        \implies & \forall\ x \in (0, \infty), \ln'(x) = \frac{1}{x}.                    & \text{(by Definition \ref{4.5.5})}
    \end{align*}
    Since \(\frac{1}{x}\) is continuous on arbitrary interval \([a, b] \subseteq (0, \infty)\), by Corollary 11.5.2 in Analysis I we know that \(\frac{1}{x}\) is Riemann integrable on \([a, b]\).
    By the fundamental theorem of calculus (Theorem 11.9.4) we thus have
    \[
        \int_a^b \frac{1}{x} \; dx = \ln(b) - \ln(a).
    \]
\end{proof}

\begin{proof}{(b)}
    We have
    \begin{align*}
        \forall\ x, y \in (0, \infty), \ln(xy) & = \ln(e^{\ln(x)} e^{\ln(y)}) & \text{(by Definition \ref{4.5.5})} \\
                                               & = \ln(e^{\ln(x) + \ln(y)})   & \text{(by Theorem \ref{4.5.2}(d))} \\
                                               & = \ln(x) + \ln(y).           & \text{(by Definition \ref{4.5.5})}
    \end{align*}
\end{proof}

\begin{proof}{(c)}
    We have
    \begin{align*}
        \ln(1) & = \ln(e^0) & \text{(by Theorem \ref{4.5.2}(e))} \\
               & = 0        & \text{(by Definition \ref{4.5.5})}
    \end{align*}
    and
    \begin{align*}
                 & \forall\ x \in (0, \infty), \ln(\frac{x}{x}) = 0 \text{(from the proof above)}                                      \\
        \implies & \forall\ x \in (0, \infty), \ln(x) + \ln(\frac{1}{x}) = 0                      & \text{(by Theorem \ref{4.5.6}(b))} \\
        \implies & \forall\ x \in (0, \infty), \ln(\frac{1}{x}) = -\ln(x).
    \end{align*}
\end{proof}

\begin{proof}{(d)}
    Let \(x \in (0, \infty)\).
    We know that \(x^y \in (0, \infty)\) for all \(y \in \mathbf{R}\), thus by Definition \ref{4.5.5} \(\ln(x^y)\) is well-defined and we have
    \begin{align*}
        y \ln(x) & = \ln(e^{y \ln(x)})             & \text{(by Definition \ref{4.5.5})} \\
                 & = \ln\big((e^{\ln(x)})^{y}\big)                                      \\
                 & = \ln(x^y).                     & \text{(by Definition \ref{4.5.5})}
    \end{align*}
    Since \(x\) is arbitrary, we conclude that
    \[
        \forall\ x \in (0, \infty), \forall\ y \in \mathbf{R}, y \ln(x) = \ln(x^y).
    \]
\end{proof}

\begin{proof}{(e)}
    Since
    \[
        x \in (-1, 1) \iff 1 - x \in (0, 2),
    \]
    by Definition \ref{4.5.5} we know that \(\ln(1 - x)\) is well-defined.
    Observe that
    \begin{align*}
        \forall\ x \in (-1, 1), & \ln'(1 - x)                                                                                                  \\
                                & = \big(\ln'(y)|_{y = 1 - x}\big) \times \big((y \mapsto 1 - y)'(x)\big) & \text{(by chain rule)}             \\
                                & = \frac{1}{1 - x} \times (-1)                                           & \text{(by Theorem \ref{4.5.6}(a))} \\
                                & = \frac{-1}{1 - x}
    \end{align*}
    and
    \begin{align*}
                 & x \in (-1, 1)                                                                        \\
        \implies & \forall\ n \in \mathbf{Z}^+, x^n \in (-1, 1)                                         \\
        \implies & \sum_{n = 0}^\infty x^n = \frac{1}{1 - x}.   & \text{(by Lemma 7.3.3 in Analysis I)}
    \end{align*}

    First suppose that \(x = 0\).
    Then we have
    \begin{align*}
        \ln(1 - 0) & = \ln(1)                                                                    \\
                   & = 0                                    & \text{(by Theorem \ref{4.5.6}(c))} \\
                   & = - \sum_{n = 1}^\infty \frac{0^n}{n}.
    \end{align*}

    Now suppose that \(x \in (0, 1)\).
    Then we have
    \begin{align*}
                 & - \sum_{n = 0}^\infty x^n = \ln'(1 - x)                                                                                              \\
        \implies & \int_0^x \bigg(- \sum_{n = 0}^\infty y^n\bigg) \; dy = \int_0^x \ln'(1 - y) \; dy      & \text{(by Theorem \ref{4.5.6}(a))}          \\
        \implies & - \sum_{n = 0}^\infty \frac{x^{n + 1} - 0^{n + 1}}{n + 1} = \int_0^x \ln'(1 - y) \; dy & \text{(by Theorem \ref{4.1.6}(e))}          \\
        \implies & - \sum_{n = 0}^\infty \frac{x^{n + 1}}{n + 1} = \ln(1 - x) - \ln(1 - 0)                & \text{(by fundamental theorem of calculus)} \\
        \implies & - \sum_{n = 0}^\infty \frac{x^{n + 1}}{n + 1} = \ln(1 - x)                             & \text{(by Theorem \ref{4.5.6}(c))}          \\
        \implies & - \sum_{n = 1}^\infty \frac{x^n}{n} = \ln(1 - x).
    \end{align*}

    Now suppose that \(x \in (-1, 0)\).
    Then we have
    \begin{align*}
                 & - \sum_{n = 0}^\infty x^n = \ln'(1 - x)                                                                                              \\
        \implies & \int_x^0 \bigg(- \sum_{n = 0}^\infty y^n\bigg) \; dy = \int_x^0 \ln'(1 - y) \; dy      & \text{(by Theorem \ref{4.5.6}(a))}          \\
        \implies & - \sum_{n = 0}^\infty \frac{0^{n + 1} - x^{n + 1}}{n + 1} = \int_x^0 \ln'(1 - y) \; dy & \text{(by Theorem \ref{4.1.6}(e))}          \\
        \implies & \sum_{n = 0}^\infty \frac{x^{n + 1}}{n + 1} = \ln(1 - 0) - \ln(1 - x)                  & \text{(by fundamental theorem of calculus)} \\
        \implies & \sum_{n = 0}^\infty \frac{x^{n + 1}}{n + 1} = -\ln(1 - x)                              & \text{(by Theorem \ref{4.5.6}(c))}          \\
        \implies & - \sum_{n = 1}^\infty \frac{x^n}{n} = \ln(1 - x).
    \end{align*}

    Combine all proofs above we conclude that
    \begin{align*}
                 & \forall\ x \in (-1, 1), \ln(1 - x) = - \sum_{n = 1}^\infty \frac{x^n}{n}                                                  \\
        \implies & \forall\ -x \in (-1, 1), \ln\big(1 - (-x)\big) = - \sum_{n = 1}^\infty \frac{(-x)^n}{n}                                   \\
        \implies & \forall\ -(x - 1) \in (-1, 1), \ln\Big(1 - \big(-(x - 1)\big)\Big) = - \sum_{n = 1}^\infty \frac{\big(-(x - 1)\big)^n}{n} \\
        \implies & \forall\ x \in (0, 2), \ln(x) = - \sum_{n = 1}^\infty \frac{(-1)^n (x - 1)^n}{n}                                          \\
        \implies & \forall\ x \in (0, 2), \ln(x) = \sum_{n = 1}^\infty \frac{(-1)^{n + 1} (x - 1)^n}{n}.
    \end{align*}
    By Definition \ref{4.2.1} \(\ln\) is real analytic at \(1\) with radius of convergence \(1\).
\end{proof}

\begin{example}\label{4.5.7}
    We now give a modest application of Abel's theorem (Theorem \ref{4.3.1}):
    from the alternating series test we see that \(\sum_{n = 1}^\infty \frac{(-1)^{n + 1}}{n}\) is convergent.
    By Abel's theorem we thus see that
    \[
        \sum_{n = 1}^\infty \frac{(-1)^{n + 1}}{n} = \lim_{x \to 2 ; x \in (0, 2)} \sum_{n = 1}^\infty \frac{(-1)^{n + 1}}{n} (x - 1)^n = \lim_{x \to 2 ; x \in (0, 2)} \ln(x) = \ln(2),
    \]
    thus we have the formula
    \[
        \ln(2) = 1 - \frac{1}{2} + \frac{1}{3} - \frac{1}{4} +\frac{1}{5} - \dots.
    \]
\end{example}

\exercisesection

\begin{exercise}\label{ex 4.5.1}
    Prove Theorem \ref{4.5.2}.
\end{exercise}

\begin{proof}
    See Theorem \ref{4.5.2}.
\end{proof}

\begin{exercise}\label{ex 4.5.2}
    Show that for every integer \(n \geq 3\), we have
    \[
        0 < \frac{1}{(n + 1)!} + \frac{1}{(n + 2)!} + \dots < \frac{1}{n!}.
    \]
    Conclude that \(n! e\) is not an integer for every \(n \geq 3\).
    Deduce from this that \(e\) is irrational.
\end{exercise}

\begin{proof}
    We first show that \((n + k)! > 2^k n!\) for all \(n \geq 3\) and \(k \in \mathbf{Z}^+\).
    We use induction on \(k\).
    For \(k = 1\), we have
    \[
        \forall\ n \geq 3, (n + 1)! = (n + 1) (n!) \geq 4 (n!) > 2^1 (n!)
    \]
    and the base case holds.
    Suppose inductively that \((n + k)! > 2^k n!\) for some \(k \geq 1\).
    Then for \(k + 1\), we have
    \begin{align*}
        \forall\ n \geq 3, (n + k + 1)! & = (n + k + 1) (n + k)!                                    \\
                                        & \geq (4 + k)(n + k)!                                      \\
                                        & > 2 (n + k)!                                              \\
                                        & > 2 (2^k) (n!)         & \text{(by induction hypothesis)} \\
                                        & = 2^{k + 1} (n!)
    \end{align*}
    and this closes the induction.

    Next we show that
    \[
        \forall\ n \geq 3, 0 < \frac{1}{(n + 1)!} + \frac{1}{(n + 2)!} + \dots < \frac{1}{n!}.
    \]
    Since
    \begin{align*}
                 & \forall\ n \geq 3, \forall\ k \geq 1, \frac{1}{(n + k)!} < \frac{1}{2^k (n!)}                                                                                                                  & \text{(from the proof above)} \\
        \implies & \forall\ n \geq 3,                                                                                                                                                                                                             \\
                 & \sum_{k = 1}^\infty \frac{1}{(n + k)!} \leq \sum_{k = 1}^\infty \frac{1}{2^k (n!)} = \frac{1}{n!} \sum_{k = 1}^\infty \frac{1}{2^k} = \frac{1}{2 (n!)} \sum_{k = 1}^\infty \frac{1}{2^{k - 1}} & \text{(geometric series)}     \\
                 & = \frac{1}{2 (n!)} \sum_{k = 0}^\infty \frac{1}{2^k} = \frac{2}{2 (n!)} = \frac{1}{n!}                                                                                                                                         \\
        \implies & \forall\ n \geq 3, \sum_{k = 1}^\infty \frac{1}{(n + k)!} \leq \frac{1}{n!},
    \end{align*}
    we only need to show that
    \[
        \forall\ n \geq 3, \sum_{k = 1}^\infty \frac{1}{(n + k)!} \neq \frac{1}{n!}.
    \]
    So suppose for sake of contradiction that there exists some \(n \geq 3\) such that the identity above holds.
    Then we have
    \begin{align*}
                 & \sum_{k = 1}^\infty \frac{1}{(n + k)!} = \frac{1}{n!} = \sum_{k = 1}^\infty \frac{1}{2^k (n!)} \\
        \implies & \sum_{k = 1}^\infty \bigg(\frac{1}{2^k (n!)} - \frac{1}{(n + k)!}\bigg) = 0.
    \end{align*}
    But we know that
    \[
        \forall\ k \geq 1, \frac{1}{2^k (n!)} - \frac{1}{(n + k)!} > 0 \implies \sum_{k = 1}^\infty \bigg(\frac{1}{2^k (n!)} - \frac{1}{(n + k)!}\bigg) > 0,
    \]
    a contradiction.
    Thus we have
    \[
        \forall\ n \geq 3, \sum_{k = 1}^\infty \frac{1}{(n + k)!} < \frac{1}{n!}.
    \]

    Now we show that \(n! e\) is not an integer for all \(n \geq 3\).
    Since
    \begin{align*}
                 & \forall\ n \geq 3, \sum_{m = n + 1}^\infty \frac{1}{m!} < \frac{1}{n!} \\
        \implies & \forall\ n \geq 3, \sum_{m = n + 1}^\infty \frac{n!}{m!} < 1
    \end{align*}
    and
    \[
        \forall\ n \geq 3, \sum_{m = 0}^n \frac{n!}{m!} = \sum_{m = 0}^n (n - m)! \in \mathbf{N},
    \]
    we have
    \begin{align*}
        \forall\ n \geq 3, n! e & = n! \sum_{m = 0}^\infty \frac{1}{m!}                                                         & \text{(by Definition \ref{4.5.3})} \\
                                & = \sum_{m = 0}^\infty \frac{n!}{m!}                                                                                                \\
                                & = \sum_{m = 0}^n \frac{n!}{m!} + \sum_{m = n + 1}^\infty \frac{n!}{m!}                                                             \\
                                & = \sum_{m = 0}^n (n - m)! + \sum_{m = n + 1}^\infty \frac{n!}{m!}                                                                  \\
                                & \in \Bigg(\bigg(\sum_{m = 0}^n (n - m)!\bigg), \bigg(\sum_{m = 0}^n (n - m)! + 1\bigg)\Bigg).
    \end{align*}
    Thus \(n! e\) is not an integer for all \(n \geq 3\).

    Finally we show that \(e\) is irrational.
    Suppose for sake of contradiction that \(e \in \mathbf{Q}\).
    Then we know that \(e = \frac{a}{b}\) for some \(a \in \mathbf{Z}\) and \(b \in \mathbf{Z}^+\).
    But then we have
    \begin{align*}
                 & e = \frac{a}{b} = \frac{3a}{3b}                                  \\
        \implies & (3b)! e = \frac{(3a) (3b)!}{3b} = (3a) (3b - 1)! \in \mathbf{N},
    \end{align*}
    a contradiction.
    Thus \(e \in \mathbf{R} \setminus \mathbf{Q}\).
\end{proof}

\begin{exercise}\label{ex 4.5.3}
    Prove Proposition \ref{4.5.4}.
\end{exercise}

\begin{proof}
    See Proposition \ref{4.5.4}.
\end{proof}

\begin{exercise}\label{ex 4.5.4}
    Let \(f : \mathbf{R} \to \mathbf{R}\) be the function defined by setting \(f(x) \coloneqq \exp(-1 / x)\) when \(x > 0\), and \(f(x) \coloneqq 0\) when \(x \leq 0\).
    Prove that \(f\) is infinitely differentiable, and \(f^{(k)}(0) = 0\) for every integer \(k \geq 0\), but that \(f\) is not real analytic at \(0\).
\end{exercise}

\begin{proof}
    First we show that
    \[
        \forall\ k \in \mathbf{N}, \forall\ x \in \mathbf{R}^+, \bigg(y \mapsto \frac{-1}{y}\bigg)^{(k)}(x) = (-1)^{k + 1} (k!) x^{-(k + 1)}.
    \]
    We use induction on \(k\).
    For \(k = 0\), we have
    \[
        \forall\ x \in \mathbf{R}^+, \bigg(y \mapsto \frac{-1}{y}\bigg)^{(0)}(x) = \frac{-1}{x} = (-1)^{0 + 1} (0!) x^{-(0 + 1)}
    \]
    and the base case holds.
    Suppose inductively that
    \[
        \forall\ x \in \mathbf{R}^+, \bigg(y \mapsto \frac{-1}{y}\bigg)^{(k)}(x) = (-1)^{k + 1} (k!) x^{-(k + 1)}.
    \]
    for some \(k \geq 0\).
    Then for \(k + 1\), we have
    \begin{align*}
        \forall\ x \in \mathbf{R}^+, & \bigg(y \mapsto \frac{-1}{y}\bigg)^{(k + 1)}(x)                                                 \\
                                     & = \bigg((y \mapsto \frac{-1}{y})^{(k)}\bigg)'(x)           & \text{(by Definition \ref{4.2.4})} \\
                                     & = \bigg(y \mapsto (-1)^{k + 1} (k!) y^{-(k + 1)}\bigg)'(x) & \text{(by induction hypothesis)}   \\
                                     & = (-1)^{k + 2} \big((k + 1)!\big) x^{-(k + 2)}
    \end{align*}
    and this closes the induction.
    By Definition \ref{4.2.4} this means \(y \mapsto \frac{-1}{y}\) is infinitely differentiable on \(\mathbf{R}^+\).

    By Theorem \ref{4.5.2}(a) we know that \(\exp\) is real analytic on \(\mathbf{R}\), thus by Corollary \ref{4.2.7} we know that \(\exp\) is infinitely differentiable on \(\mathbf{R}\).
    Since \(x \mapsto \frac{-1}{x}\) is infinitely differentiable on \(\mathbf{R}^+\) and \(\exp\) is infinitely differentiable on \(\mathbf{R}\), by chain rule we know that \(f(x) = \exp(\frac{-1}{x})\) is infinitely differentiable on \(\mathbf{R}^+\).
    Since \(f(x) = 0\) for all \(x \in \mathbf{R}^-\), we know that \(f\) infinitely differentiable on \(\in \mathbf{R}^-\).
    So we only left to show that \(f\) is infinitely differentiable at \(0\).
    Since
    \begin{align*}
         & \lim_{x \to 0^+ ; x \in \mathbf{R}} \frac{f(x) - f(0)}{x - 0}                                 \\
         & = \lim_{x \to 0^+ ; x \in \mathbf{R}} \frac{e^{\frac{-1}{x}}}{x}                              \\
         & = \lim_{x \to 0^+ ; x \in \mathbf{R}} \frac{1}{e^{\frac{1}{x}} x}                             \\
         & = \lim_{x \to 0^+ ; x \in (0, 1)} \frac{1}{e^{\frac{1}{x}} x}                                 \\
         & \leq \lim_{x \to 0^+ ; x \in (0, 1)} \frac{1}{e^{\frac{1}{x}}}    & (x < 1 < e^{\frac{1}{x}})
    \end{align*}
\end{proof}

\begin{exercise}\label{ex 4.5.5}
    Prove Theorem \ref{4.5.6}.
\end{exercise}

\begin{proof}
    See Theorem \ref{4.5.6}.
\end{proof}

\begin{exercise}\label{ex 4.5.6}
    Prove that the natural logarithm function is real analytic on \((0, +\infty)\).
\end{exercise}

\begin{exercise}\label{ex 4.5.7}
    Let \(f : \mathbf{R} \to (0, \infty)\) be a positive, real analytic function such that \(f'(x) = f(x)\) for all \(x \in \mathbf{R}\).
    Show that \(f(x) = C e^x\) for some positive constant \(C\);
    justify your reasoning.
\end{exercise}

\begin{exercise}\label{ex 4.5.8}
    Let \(m > 0\) be an integer.
    Show that
    \[
        \lim_{x \to +\infty} \frac{e^x}{x^m} +\infty.
    \]
\end{exercise}

\begin{exercise}\label{ex 4.5.9}
    Let \(P(x)\) be a polynomial, and let \(c > 0\).
    Show that there exists a real number \(N > 0\) such that \(e^{cx} > \abs*{P(x)}\) for all \(x > N\);
    thus an exponentially growing function, no matter how small the growth rate \(c\), will eventually overtake any given polynomial \(P(x)\), no matter how large.
\end{exercise}

\begin{exercise}\label{ex 4.5.10}
    Let \(f : (0, +\infty) \times \mathbf{R} \to \mathbf{R}\) be the exponential function \(f(x, y) \coloneqq x^y\).
    Show that \(f\) is continuous.
\end{exercise}