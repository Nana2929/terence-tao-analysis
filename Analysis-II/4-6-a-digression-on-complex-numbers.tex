\section{A digression on complex numbers}\label{sec 4.6}

\setcounter{theorem}{1}
\begin{definition}[Formal definition of complex numbers]\label{4.6.2}
    A \emph{complex number} is any pair of the form \((a, b)\), where \(a, b\) are real numbers.
    Two complex numbers \((a, b)\), \((c, d)\) are said to be equal iff \(a = c\) and \(b = d\).
    The set of all complex numbers is denoted \(\mathbf{C}\).
\end{definition}

\begin{additional corollary}\label{ac 4.6.1}
Definition \ref{4.6.2} is reflexive, symmetry and transitive.
\end{additional corollary}

\begin{proof}
    Let \((a_1, b_1), (a_2, b_2), (a_3, b_3)\) be complex numbers.
    By Definition \ref{4.6.2} we know that \(a_1, a_2, a_3, b_1, b_2, b_3 \in \mathbf{R}\).
    Since
    \begin{align*}
                 & (a_1 = a_1) \land (b_1 = b_1) & (a_1, b_1 \in \mathbf{R})          \\
        \implies & (a_1, b_1) = (a_1, b_1),      & \text{(by Definition \ref{4.6.2})}
    \end{align*}
    we know that Definition \ref{4.6.2} is reflexive.
    Since
    \begin{align*}
             & (a_1, b_1) = (a_2, b_2)                                             \\
        \iff & (a_1 = a_2) \land (b_1 = b_2) & \text{(by Definition \ref{4.6.2})}  \\
        \iff & (a_2 = a_1) \land (b_2 = b_1) & (a_1, a_2, b_1, b_2 \in \mathbf{R}) \\
        \iff & (a_2, b_2) = (a_1, b_1),      & \text{(by Definition \ref{4.6.2})}
    \end{align*}
    we know that Definition \ref{4.6.2} is symmetry.
    Since
    \begin{align*}
             & \big((a_1, b_1) = (a_2, b_2)\big) \land \big((a_2, b_2) = (a_3, b_3)\big)                                                 \\
        \iff & (a_1 = a_2) \land (b_1 = b_2) \land (a_2 = a_3) \land (b_2 = b_3)         & \text{(by Definition \ref{4.6.2})}            \\
        \iff & (a_1 = a_3) \land (b_1 = b_3)                                             & (a_1, a_2, a_3, b_1, b_2, b_3 \in \mathbf{R}) \\
        \iff & (a_1, b_1) = (a_3, b_3),                                                  & \text{(by Definition \ref{4.6.2})}
    \end{align*}
    we know that Definition \ref{4.6.2} is transitive.
\end{proof}

\begin{note}
    At this stage the complex numbers \(\mathbf{C}\) are indistinguishable from the Cartesian product \(\mathbf{R}^2 = \mathbf{R} \times \mathbf{R}\)
    (also known as the \emph{Cartesian plane}).
    However, we will introduce a number of operations on the complex numbers, notably that of \emph{complex multiplication}, which are not normally placed on the Cartesian plane \(\mathbf{R}^2\).
    Thus one can think of the complex number system \(\mathbf{C}\) as the Cartesian plane \(\mathbf{R}^2\) equipped with a number of additional structures.
\end{note}

\begin{definition}[Complex addition, negation, and zero]\label{4.6.3}
    If \(z = (a, b)\) and \(w = (c, d)\) are two complex numbers, we define their \emph{sum} \(z + w\) to be the complex number \(z + w \coloneqq (a + c, b + d)\).
    We also define the \emph{negation} \(-z\) of \(z\) to be the complex number \(-z \coloneqq (-a, -b)\).
    We also define the \emph{complex zero} \(0_{\mathbf{C}}\) to be the complex number \(0_{\mathbf{C}} = (0, 0)\).
\end{definition}

\begin{additional corollary}\label{ac 4.6.2}
If \(w, w', z, z' \in \mathbf{C}\) and \(w = w'\) and \(z = z'\), then \(w + z = w' + z'\) and \(-w = -w'\).
\end{additional corollary}

\begin{proof}
    Let \(w = (a, b), w' = (a', b'), z = (c, d), z' = (c', d')\).
    By Definition \ref{4.6.2} we know that \(a, a', b, b', c, c', d, d' \in \mathbf{R}\).
    By Additional Corollary \ref{ac 4.6.1} we know that
    \begin{align*}
        a & = a'; \\
        b & = b'; \\
        c & = c'; \\
        d & = d'.
    \end{align*}
    Then we have
    \begin{align*}
        w + z & = (a + c, b + d)     & \text{(by Definition \ref{4.6.3})}          \\
              & = (a' + c', b' + d') & (a, a', b, b', c, c', d, d' \in \mathbf{R}) \\
              & = w' + z'            & \text{(by Definition \ref{4.6.3})}
    \end{align*}
    and
    \begin{align*}
        -w & = (-a, -b)   & \text{(by Definition \ref{4.6.3})} \\
           & = (-a', -b') & (a, a', b, b' \in \mathbf{R})      \\
           & = -w'.       & \text{(by Definition \ref{4.6.3})}
    \end{align*}
\end{proof}

\begin{lemma}[The complex numbers are an additive group]\label{4.6.4}
    If \(z_1, z_2, z_3\) are complex numbers, then we have the commutative property \(z_1 + z_2 = z_2 + z_1\), the associative property \((z_1 + z_2) + z_3 = z_1 + (z_2 + z_3)\), the identity property \(z_1 + 0_{\mathbf{C}} = 0_{\mathbf{C}} + z_1 = z_1\), and the inverse property \(z_1 + (-z_1) = (-z_1) + z_1 = 0_{\mathbf{C}}\).
\end{lemma}

\begin{proof}
    Let \(z_1 = (a, b), z_2 = (c, d), z_3 = (e, f)\).
    By Definition \ref{4.6.2} we know that \(a, b, c, d, e, f \in \mathbf{R}\).
    Since
    \begin{align*}
        z_1 + z_2 & = (a + c, b + d) & \text{(by Definition \ref{4.6.3})} \\
                  & = (c + a, d + b) & (a, b, c, d \in \mathbf{R})        \\
                  & = z_2 + z_1,     & \text{(by Definition \ref{4.6.3})}
    \end{align*}
    we know that the addition operation in Definition \ref{4.6.3} is commutative.
    Since
    \begin{align*}
        (z_1 + z_2) + z_3 & = (a + c, b + d) + z_3               & \text{(by Definition \ref{4.6.3})} \\
                          & = \big((a + c) + e, (b + d) + f\big) & \text{(by Definition \ref{4.6.3})} \\
                          & = \big(a + (c + e), b + (d + f)\big) & (a, b, c, d, e, f \in \mathbf{R})  \\
                          & = z_1 + (c + e, d + f)               & \text{(by Definition \ref{4.6.3})} \\
                          & = z_1 + (z_2 + z_3),                 & \text{(by Definition \ref{4.6.3})}
    \end{align*}
    we know that the addition operation in Definition \ref{4.6.3} is associative.
    Since
    \begin{align*}
        0_{\mathbf{C}} + z_1 & = z_1 + 0_{\mathbf{C}} & \text{(from the proof above)}      \\
                             & = (a + 0, b + 0)       & \text{(by Definition \ref{4.6.3})} \\
                             & = (a, b)               & (a, b, 0 \in \mathbf{R})           \\
                             & = z_1
    \end{align*}
    and
    \begin{align*}
        (-z_1) + z_1 & = z_1 + (-z_1)                 & \text{(from the proof above)}      \\
                     & = \big(a + (-a), b + (-b)\big) & \text{(by Definition \ref{4.6.3})} \\
                     & = (0, 0)                       & (a, b \in \mathbf{R})              \\
                     & = 0_{\mathbf{C}},              & \text{(by Definition \ref{4.6.3})}
    \end{align*}
    we know that \(0_{\mathbf{C}}\) is the additive identity in \(\mathbf{C}\).
\end{proof}

\begin{definition}[Complex multiplication]\label{4.6.5}
    If \(z = (a, b)\) and \(w = (c, d)\) are complex numbers, then we define their \emph{product} \(zw\) to be the complex number \(zw \coloneqq (ac - bd, ad + bc)\).
    We also introduce the \emph{complex identity} \(1_{\mathbf{C}} \coloneqq (1, 0)\).
\end{definition}

\begin{additional corollary}\label{ac 4.6.3}
If \(w, w', z, z' \in \mathbf{C}\) and \(w = w'\) and \(z = z'\), then \(wz = w'z'\).
\end{additional corollary}

\begin{proof}
    Let \(w = (a, b), w' = (a', b'), z = (c, d), z' = (c', d')\).
    By Definition \ref{4.6.2} we know that \(a, a', b, b', c, c', d, d' \in \mathbf{R}\).
    By Additional Corollary \ref{ac 4.6.1} we know that
    \begin{align*}
        a & = a'; \\
        b & = b'; \\
        c & = c'; \\
        d & = d'.
    \end{align*}
    Then we have
    \begin{align*}
        wz & = (ac - bd, ad + bc)             & \text{(by Definition \ref{4.6.5})}          \\
           & = (a' c' - b' d', a' d' + b' c') & (a, a', b, b', c, c', d, d' \in \mathbf{R}) \\
           & = w' z'.                         & \text{(by Definition \ref{4.6.5})}
    \end{align*}
\end{proof}

\begin{lemma}\label{4.6.6}
    If \(z_1, z_2, z_3\) are complex numbers, then we have the commutative property \(z_1 z_2 = z_2 z_1\), the associative property \((z_1 z_2) z_3 = z_1 (z_2 z_3)\), the identity property \(z_1 1_{\mathbf{C}} = 1_{\mathbf{C}} z_1 = z_1\), and the distributivity properties \(z_1 (z_2 + z_3) = z_1 z_2 + z_1 z_3\) and \((z_2 + z_3) z_1 = z_2 z_1 + z_3 z_1\).
\end{lemma}

\begin{proof}
    Let \(z_1 = (a, b), z_2 = (c, d), z_3 = (e, f)\).
    By Definition \ref{4.6.2} we know that \(a, b, c, d, e, f \in \mathbf{R}\).
    Since
    \begin{align*}
        z_1 z_2 & = (ac - bd, ad + bc) & \text{(by Definition \ref{4.6.5})} \\
                & = (ca - db, da + cb) & (a, b, c, d \in \mathbf{R})        \\
                & = z_2 z_1,           & \text{(by Definition \ref{4.6.5})}
    \end{align*}
    we know that the multiplication operation in Definition \ref{4.6.5} is commutative.
    Since
    \begin{align*}
         & (z_1 z_2) z_3                                                                                         \\
         & = (ac - bd, ad + bc) z_3                                         & \text{(by Definition \ref{4.6.5})} \\
         & = \big((ac - bd) e - (ad + bc) f, (ac - bd) f + (ad + bc) e\big) & \text{(by Definition \ref{4.6.5})} \\
         & = \big(a (ce - df) - b (cf + de), a (cf + de) + b (ce - df)\big) & (a, b, c, d, e, f \in \mathbf{R})  \\
         & = z_1 (ce - df, cf + de)                                         & \text{(by Definition \ref{4.6.5})} \\
         & = z_1 (z_2 z_3),                                                 & \text{(by Definition \ref{4.6.5})}
    \end{align*}
    we know that the multiplication operation in Definition \ref{4.6.5} is associative.
    Since
    \begin{align*}
        1_{\mathbf{C}} z_1 & = z_1 1_{\mathbf{C}}     & \text{(from the proof above)}      \\
                           & = (a 1 - b 0, a 0 + b 1) & \text{(by Definition \ref{4.6.5})} \\
                           & = (a, b)                 & (a, b, 1 \in \mathbf{R})           \\
                           & = z_1,
    \end{align*}
    we know that \(1_{\mathbf{C}}\) is the multiplicative identity in \(\mathbf{C}\).
    Since
    \begin{align*}
        z_1 (z_2 + z_3) & = z_1 (c + e, d + f)                                      & \text{(by Definition \ref{4.6.3})} \\
                        & = \big(a (c + e) - b (d + f), a (d + f) + b (c + e)\big)  & \text{(by Definition \ref{4.6.5})} \\
                        & = \big((ac - bd) + (ae - bf), (ad + bc) + (af + be) \big) & (a, b, c, d, e, f \in \mathbf{R})  \\
                        & = (ac - bd, ad + bc) + (ae - bf, af + be)                 & \text{(by Definition \ref{4.6.3})} \\
                        & = z_1 z_2 + z_1 z_3                                       & \text{(by Definition \ref{4.6.3})}
    \end{align*}
    and
    \begin{align*}
        (z_2 + z_3) z_1 & = z_1 (z_2 + z_3)    & \text{(from the proof above)} \\
                        & = z_1 z_2 + z_1 z_3  & \text{(from the proof above)} \\
                        & = z_2 z_1 + z_3 z_1, & \text{(from the proof above)}
    \end{align*}
    we know that the multiplication operation in Definition \ref{4.6.5} and the addition operation in Definition \ref{4.6.3} are distributive.
\end{proof}

\begin{note}
    Lemma \ref{4.6.6} can also be stated more succinctly, as the assertion that \(\mathbf{C}\) is a commutative ring.
    As is usual, we now write \(z - w\) as shorthand for \(z + (-w)\).
\end{note}

\begin{note}
    We now identify the real numbers \(\mathbf{R}\) with a subset of the complex numbers \(\mathbf{C}\) by identifying any real number \(x\) with the complex number \((x, 0)\), thus \(x \equiv (x, 0)\).
    Note that this identification is consistent with equality (thus \(x = y\) iff \((x, 0) = (y, 0)\)), with addition (\(x_1 + x_2 = x_3\) iff \((x_1, 0) + (x_2, 0) = (x_3, 0)\)), with negation (\(x = -y\) iff \((x, 0) = -(y, 0)\)), and multiplication (\(x_1 x_2 = x_3\) iff \((x_1, 0) (x_2, 0) = (x_3, 0)\)), so we will no longer need to distinguish between ``real addition'' and ``complex addition'', and similarly for equality, negation, and multiplication.
    Note also that \(0 \equiv 0_\mathbf{C}\) and \(1 \equiv 1_\mathbf{C}\), so we can now drop the \(\mathbf{C}\) subscripts from the zero \(0\) and the identity \(1\).
\end{note}

\begin{note}
    We now define \(i\) to be the complex number \(i \coloneqq (0, 1)\).
\end{note}

\begin{lemma}\label{4.6.7}
    Every complex number \(z \in \mathbf{C}\) can be written as \(z = a + bi\) for exactly one pair \(a, b\) of real numbers.
    Also, we have \(i^2 = -1\), and \(-z = (-1)z\).
\end{lemma}

\begin{proof}
    Let \(z = (a, b)\).
    We have
    \begin{align*}
        a + bi & = (a, 0) + (b, 0) \times (0, 1)                                      \\
               & = (a, 0) + (0, b)               & \text{(by Definition \ref{4.6.5})} \\
               & = (a, b)                        & \text{(by Definition \ref{4.6.3})} \\
               & = z.
    \end{align*}
    If \(z' = (a', b')\) such that \(z = z'\), then we have
    \begin{align*}
        z' & = (a', b')                                                                          \\
           & = (a', 0) + (0, b')               & \text{(by Definition \ref{4.6.3})}              \\
           & = (a', 0) + (b', 0) \times (0, 1) & \text{(by Definition \ref{4.6.5})}              \\
           & = a' + b' i                                                                         \\
           & = a + bi.                         & \text{(by Additional Corollary \ref{ac 4.6.1})}
    \end{align*}

    Now we show that \(i \times i = -1\).
    \begin{align*}
        i \times i & = (0, 1) \times (0, 1)                                                      \\
                   & = (0^2 - 1^2, 0 \times 1 + 1 \times 0) & \text{(by Definition \ref{4.6.5})} \\
                   & = (-1, 0)                                                                   \\
                   & = -(1, 0)                              & \text{(by Definition \ref{4.6.3})} \\
                   & = -1.
    \end{align*}

    Finally we show that \(-z = (-1) z\).
    \begin{align*}
        -z & = -(a, b)                                                    \\
           & = (-a, -b)              & \text{(by Definition \ref{4.6.3})} \\
           & = (-1, 0) \times (a, b) & \text{(by Definition \ref{4.6.5})} \\
           & = (-1) z.
    \end{align*}
\end{proof}

\begin{note}
    Because Lemma \ref{4.6.7}, we will now refer to complex numbers in the more usual notation \(a + bi\), and discard the formal notation \((a, b)\) henceforth.
\end{note}

\begin{definition}[Real and imaginary parts]\label{4.6.8}
    If \(z\) is a complex number with the representation \(z = a + bi\) for some real numbers \(a, b\), we shall call \(a\) the \emph{real part} of \(z\) and denote \(\Re(z) \coloneqq a\), and call \(b\) the \emph{imaginary part} of \(z\) and denote \(\Im(z) \coloneqq b\).
    In general \(z = \Re(z) + i \Im(z)\).
    Note that \(z\) is real iff \(\Im(z) = 0\).
    We say that \(z\) is \emph{imaginary} iff \(\Re(z) = 0\).
    \(0\) is both real and imaginary.
    We define the \emph{complex conjugate} \(\overline{z}\) of \(z\) to be the complex number \(\overline{z} \coloneqq \Re(z) - i \Im(z)\).
\end{definition}

\begin{additional corollary}\label{ac 4.6.4}
If \(z, z' \in \mathbf{C}\) such that \(z = z'\), then \(\Re(z) = \Re(z')\), \(\Im(z) = \Im(z')\) and \(\overline{z} = \overline{z'}\).
\end{additional corollary}

\begin{proof}
    Let \(z = a + bi\) and \(z' = a' + b' i\).
    By Lemma \ref{4.6.7} we know that \(a = a'\) and \(b = b'\).
    Thus by Definition \ref{4.6.8} we have
    \begin{align*}
         & \Re(z) = a = a' = \Re(z')                         \\
         & \Im(z) = b = b' = \Im(z')                         \\
         & \overline{z} = a - bi = a' - b' i = \overline{z'}
    \end{align*}
\end{proof}

\begin{lemma}[Complex conjugation is an involution]\label{4.6.9}
    Let \(z, w\) be complex numbers, then \(\overline{z + w} = \overline{z} + \overline{w}\), \(\overline{-z} = -\overline{z}\), and \(\overline{zw} = \overline{z} \; \overline{w}\).
    Also \(\overline{\overline{z}} = z\).
    Finally, we have \(\overline{z} = \overline{w}\) if and only if \(z = w\), and \(\overline{z} = z\) if and only if \(z\) is real.
\end{lemma}

\begin{proof}
    First we show that \(\overline{z + w} = \overline{z} + \overline{w}\).
    \begin{align*}
        \overline{z + w} & = \overline{\Re(z) + i \Im(z) + \Re(w) + i \Im(w)}                     & \text{(by Definition \ref{4.6.8})} \\
                         & = \overline{\big(\Re(z) + \Re(w)\big) + \big(i \Im(z) + i \Im(w)\big)} & \text{(by Lemma \ref{4.6.4})}      \\
                         & = \overline{\big(\Re(z) + \Re(w)\big) + i \big(\Im(z) + \Im(w)\big)}   & \text{(by Lemma \ref{4.6.6})}      \\
                         & = \big(\Re(z) + \Re(w)\big) - i \big(\Im(z) + \Im(w)\big)              & \text{(by Definition \ref{4.6.8})} \\
                         & = \big(\Re(z) + \Re(w)\big) + (-1) i \big(\Im(z) + \Im(w)\big)         & \text{(by Lemma \ref{4.6.7})}      \\
                         & = \big(\Re(z) + \Re(w)\big) + \big((-1) i \Im(z) + (-1) i \Im(w)\big)  & \text{(by Lemma \ref{4.6.6})}      \\
                         & = \big(\Re(z) + (-1) i \Im(z)\big) + \big(\Re(w) + (-1) i \Im(w)\big)  & \text{(by Lemma \ref{4.6.4})}      \\
                         & = \big(\Re(z) - i \Im(z)\big) + \big(\Re(w) - i \Im(w)\big)            & \text{(by Lemma \ref{4.6.7})}      \\
                         & = \overline{\Re(z) + i \Im(z)} + \overline{\Re(w) + i \Im(w)}          & \text{(by Definition \ref{4.6.8})} \\
                         & = \overline{z} + \overline{w}.                                         & \text{(by Definition \ref{4.6.8})}
    \end{align*}

    Next we show that \(\overline{-z} = -\overline{z}\).
    \begin{align*}
        \overline{-z} & = \overline{-\Re(z) - i \Im(z)}          & \text{(by Definition \ref{4.6.8})} \\
                      & = \overline{(-1) \Re(z) + (-1) i \Im(z)} & \text{(by Lemma \ref{4.6.7})}      \\
                      & = (-1) \Re(z) - (-1) i \Im(z)            & \text{(by Definition \ref{4.6.8})} \\
                      & = (-1) \big(\Re(z) - i \Im(z)\big)       & \text{(by Lemma \ref{4.6.6})}      \\
                      & = (-1) \overline{\Re(z) + i \Im(z)}      & \text{(by Definition \ref{4.6.8})} \\
                      & = (-1) \overline{z}                      & \text{(by Definition \ref{4.6.8})} \\
                      & = -\overline{z}.                         & \text{(by Lemma \ref{4.6.7})}
    \end{align*}

    Next we show that \(\overline{zw} = \overline{z} \; \overline{w}\).
    \begin{align*}
        \overline{zw} & = \overline{\big(\Re(z) + i \Im(z)\big) \times \big(\Re(w) + i \Im(w)\big)}            & \text{(by Definition \ref{4.6.8})} \\
                      & = \overline{\Re(z) \Re(w) - \Im(z) \Im(w) + i \big(\Re(z) \Im(w) + \Im(z) \Re(w)\big)} & \text{(by Definition \ref{4.6.5})} \\
                      & = \Re(z) \Re(w) - \Im(z) \Im(w) - i \big(\Re(z) \Im(w) + \Im(z) \Re(w)\big)            & \text{(by Definition \ref{4.6.8})} \\
                      & = \big(\Re(z) - i \Im(z)\big) \times \big(\Re(w) - i \Im(w)\big)                       & \text{(by Definition \ref{4.6.5})} \\
                      & = \overline{\Re(z) + i \Im(z)} \; \overline{\Re(w) + i \Im(w)}                         & \text{(by Definition \ref{4.6.8})} \\
                      & = \overline{z} \; \overline{w}.                                                        & \text{(by Definition \ref{4.6.5})}
    \end{align*}

    Next we show that \(\overline{\overline{z}} = z\).
    \begin{align*}
        \overline{\overline{z}} & = \overline{\overline{\Re(z) + i \Im(z)}} & \text{(by Definition \ref{4.6.8})} \\
                                & = \overline{\Re(z) - i \Im(z)}            & \text{(by Definition \ref{4.6.8})} \\
                                & = \overline{\Re(z) + (-1) i \Im(z)}       & \text{(by Lemma \ref{4.6.7})}      \\
                                & = \overline{\Re(z) + i (-1) \Im(z)}       & \text{(by Lemma \ref{4.6.6})}      \\
                                & = \Re(z) - i (-1) \Im(z)                  & \text{(by Definition \ref{4.6.8})} \\
                                & = \Re(z) + (-1) i (-1) \Im(z)             & \text{(by Lemma \ref{4.6.7})}      \\
                                & = \Re(z) + (-1) (-1) i \Im(z)             & \text{(by Lemma \ref{4.6.7})}      \\
                                & = \Re(z) + i \Im(z)                       & \text{(by Definition \ref{4.6.5})} \\
                                & = z.                                      & \text{(by Definition \ref{4.6.8})}
    \end{align*}

    Next we show that \(\overline{z} = \overline{w} \iff z = w\).
    \begin{align*}
                 & \overline{z} = \overline{w}                                                                         \\
        \implies & \overline{\overline{z}} = \overline{\overline{w}} & \text{(by Additional Corollary \ref{ac 4.6.4})} \\
        \implies & z = w                                             & \text{(from the proof above)}                   \\
        \implies & \overline{z} = \overline{w}.                      & \text{(by Additional Corollary \ref{ac 4.6.4})}
    \end{align*}

    Finally we show that \(\overline{z} = z \iff \Im(z) = 0\).
    \begin{align*}
             & \overline{z} = z                                                                                    \\
        \iff & \overline{\Re(z) + i \Im(z)} = \Re(z) + i \Im(z)               & \text{(by Definition \ref{4.6.8})} \\
        \iff & \Re(z) - i \Im(z) = \Re(z) + i \Im(z)                          & \text{(by Definition \ref{4.6.8})} \\
        \iff & \Re(z) + (-1) i \Im(z) = \Re(z) + i \Im(z)                     & \text{(by Lemma \ref{4.6.7})}      \\
        \iff & \Re(z) + i (-1) \Im(z) = \Re(z) + i \Im(z)                     & \text{(by Lemma \ref{4.6.6})}      \\
        \iff & \big(\Re(z) = \Re(z)\big) \land \big((-1) \Im(z) = \Im(z)\big) & \text{(by Definition \ref{4.6.2})} \\
        \iff & (-1) \Im(z) = \Im(z)                                                                                \\
        \iff & \Im(z) = 0.                                                    & (\Im(z) \in \mathbf{R})
    \end{align*}
\end{proof}

\begin{note}
    We cannot extend the definition of absolute value directly to the complex numbers, as most complex numbers are neither positive nor negative.
    (For instance, we do not classify \(i\) as either a positive or negative number)
\end{note}

\begin{definition}[Complex absolute value]\label{4.6.10}
    If \(z = a + bi\) is a complex number, we define the \emph{absolute value} \(\abs*{z}\) of \(z\) to be the real number \(\abs*{z} \coloneqq \sqrt{a^2 + b^2} = (a^2 + b^2)^{1 / 2}\).
\end{definition}

\begin{note}
    From Exercise 5.6.3 in Analysis I we see that Definition \ref{4.6.10} generalizes the notion of real absolute value.
\end{note}

\begin{lemma}[Properties of complex absolute value]\label{4.6.11}
    Let \(z, w\) be complex numbers.
    Then \(\abs*{z}\) is a non-negative real number, and \(\abs*{z} = 0\) if and only if \(z = 0\).
    Also we have the identity \(z \overline{z} = \abs*{z}^2\), and so \(\abs*{z} = \sqrt{z \overline{z}}\).
    As a consequence we have \(\abs*{zw} = \abs*{z} \abs*{w}\) and \(\abs*{\overline{z}} = \abs*{z}\).
    Finally, we have the inequalities
    \[
        -\abs*{z} \leq \Re(z) \leq \abs*{z}; \quad -\abs*{z} \leq \Im(z) \leq \abs*{z}; \quad \abs*{z} \leq \abs*{\Re(z)} + \abs*{\Im(z)}
    \]
    as well as the triangle inequality \(\abs*{z + w} \leq \abs*{z} + \abs*{w}\).
\end{lemma}

\begin{proof}
    We have
    \begin{align*}
        \abs*{z} & = \abs*{\Re(z) + i \Im(z)}                       & \text{(by Definition \ref{4.6.8})}  \\
                 & = \sqrt{\big(\Re(z)\big)^2 + \big(\Im(z)\big)^2} & \text{(by Definition \ref{4.6.10})} \\
                 & \geq 0                                           & (\Re(z), \Im(z) \in \mathbf{R})
    \end{align*}
    and
    \begin{align*}
             & \abs*{z} = 0                                                                             \\
        \iff & \abs*{\Re(z) + i \Im(z)} = 0                       & \text{(by Definition \ref{4.6.8})}  \\
        \iff & \sqrt{\big(\Re(z)\big)^2 + \big(\Im(z)\big)^2} = 0 & \text{(by Definition \ref{4.6.10})} \\
        \iff & \big(\Re(z) = 0\big) \land \big(\Im(z) = 0\big)    & (\Re(z), \Im(z) \in \mathbf{R})     \\
        \iff & z = 0.                                             & \text{(by Definition \ref{4.6.2})}
    \end{align*}
    Since
    \begin{align*}
        z \overline{z} & = \big(\Re(z) + i \Im(z)\big) \times \big(\overline{\Re(z) + i \Im(z)}\big)                     & \text{(by Definition \ref{4.6.8})}      \\
                       & = \big(\Re(z) + i \Im(z)\big) \times \big(\Re(z) - i \Im(z)\big)                                & \text{(by Definition \ref{4.6.8})}      \\
                       & = \big(\Re(z) + i \Im(z)\big) \times \big(\Re(z) + (-1) i \Im(z)\big)                           & \text{(by Lemma \ref{4.6.7})}           \\
                       & = \big(\Re(z) + i \Im(z)\big) \times \big(\Re(z) + i (-1) \Im(z)\big)                           & \text{(by Lemma \ref{4.6.6})}           \\
                       & = \big(\Re(z)\big)^2 - (-1) \big(\Im(z)\big)^2 + i \big((-1) \Re(z) \Im(z) + \Re(z) \Im(z)\big) & \text{(by Definition \ref{4.6.5})}      \\
                       & = \big(\Re(z)\big)^2 + \big(\Im(z)\big)^2                                                       & (\Re(z), \Im(z) \in \mathbf{R})         \\
                       & = \Big(\sqrt{\big(\Re(z)\big)^2 + \big(\Im(z)\big)^2}\Big)^2                                    & \big(\Re(z)\big)^2 + \big(\Im(z)\big)^2 \\
                       & = \abs*{z}^2,                                                                                   & \text{(by Definition \ref{4.6.10})}
    \end{align*}
    we know that \(\abs*{z} = \sqrt{\abs*{z}^2} = \sqrt{z \overline{z}}\).
    Thus
    \begin{align*}
        \abs*{z} \abs*{w} & = \sqrt{z \overline{z}} \sqrt{w \overline{w}} & \text{(from the proof above)}                   \\
                          & = \sqrt{z \overline{z} w \overline{w}}        & (z \overline{z}, w \overline{w} \in \mathbf{R}) \\
                          & = \sqrt{zw \overline{z} \; \overline{w}}      & \text{(by Lemma \ref{4.6.6})}                   \\
                          & = \sqrt{zw \overline{zw}}                     & \text{(by Lemma \ref{4.6.9})}                   \\
                          & = \abs*{zw}                                   & \text{(from the proof above)}
    \end{align*}
    and
    \begin{align*}
        \abs*{\overline{z}} & = \sqrt{\overline{z} \; \overline{\overline{z}}} & \text{(from the proof above)} \\
                            & = \sqrt{\overline{z} z}                          & \text{(by Lemma \ref{4.6.9})} \\
                            & = \sqrt{z \overline{z}}                          & \text{(by Lemma \ref{4.6.6})} \\
                            & = \abs*{z}.                                      & \text{(from the proof above)}
    \end{align*}
    Since
    \begin{align*}
                 & \begin{cases}
            \abs*{\Re(z)} = \sqrt{\abs*{\Re(z)}^2} \leq \sqrt{\big(\Re(z)\big)^2 + \big(\Im(z)\big)^2} \\
            \abs*{\Im(z)} = \sqrt{\abs*{\Im(z)}^2} \leq \sqrt{\big(\Re(z)\big)^2 + \big(\Im(z)\big)^2} \\
            \big(\abs*{\Re(z)} + \abs*{\Im(z)}\big)^2 \geq \big(\Re(z)\big)^2 + \big(\Im(z)\big)^2
        \end{cases} & (\Re(z), \Im(z) \in \mathbf{R})     \\
        \implies & \begin{cases}
            \abs*{\Re(z)} \leq \abs*{z} \\
            \abs*{\Im(z)} \leq \abs*{z} \\
            \abs*{\Re(z)} + \abs*{\Im(z)} \geq \sqrt{\big(\Re(z)\big)^2 + \big(\Im(z)\big)^2} = \abs*{z}
        \end{cases} & \text{(by Definition \ref{4.6.10})} \\
        \implies & \begin{cases}
            -\abs*{z} \leq \Re(z) \leq \abs*{z} \\
            -\abs*{z} \leq \Im(z) \leq \abs*{z} \\
            \abs*{\Re(z)} + \abs*{\Im(z)} \geq \abs*{z}
        \end{cases}
    \end{align*}
    we know that
    \begin{align*}
        \Re(z \overline{w}) & \leq \abs*{z \overline{w}}     & \text{(from the proof above)} \\
                            & = \abs*{z} \abs*{\overline{w}} & \text{(from the proof above)} \\
                            & = \abs*{z} \abs*{w}.           & \text{(from the proof above)}
    \end{align*}
    Thus
    \begin{align*}
                 & \overline{z \overline{w}} = \overline{z} \; \overline{\overline{w}} = \overline{z} w                         & \text{(by Lemma \ref{4.6.9})}                     \\
        \implies & \Re(z \overline{w}) = \frac{z \overline{w} + \overline{z} w}{2}                                              & \text{(by Exercise \ref{ex 4.6.5})}               \\
        \implies & \frac{z \overline{w} + \overline{z} w}{2} \leq \abs*{z} \abs*{w}                                             & \text{(from the proof above)}                     \\
        \implies & z \overline{w} + \overline{z} w \leq 2 \abs*{z} \abs*{w}                                                                                                         \\
        \implies & \abs*{z}^2 + z \overline{w} + \overline{z} w + \abs*{w}^2 \leq \abs*{z}^2 + 2 \abs*{z} \abs*{w} + \abs*{w}^2                                                     \\
        \implies & \abs*{z}^2 + z \overline{w} + \overline{z} w + \abs*{w}^2 \leq (\abs*{z} + \abs*{w})^2                                                                           \\
        \implies & z \overline{z} + z \overline{w} + \overline{z} w + w \overline{w} \leq (\abs*{z} + \abs*{w})^2               & \text{(from the proof above)}                     \\
        \implies & (z + w) (\overline{z} + \overline{w}) \leq (\abs*{z} + \abs*{w})^2                                           & \text{(by Lemma \ref{4.6.6})}                     \\
        \implies & (z + w) (\overline{z + w}) \leq (\abs*{z} + \abs*{w})^2                                                      & \text{(by Lemma \ref{4.6.9})}                     \\
        \implies & \abs*{z + w}^2 \leq (\abs*{z} + \abs*{w})^2                                                                  & \text{(from the proof above)}                     \\
        \implies & \abs*{z + w} \leq \abs*{z} + \abs*{w}.                                                                       & (\abs*{z + w}, \abs*{z}, \abs*{w} \in \mathbf{R})
    \end{align*}
\end{proof}

\begin{definition}[Complex reciprocal]\label{4.6.12}
    If \(z\) is a non-zero complex number, we define the \emph{reciprocal} \(z^{-1}\) of \(z\) to be the complex number \(z^{-1} \coloneqq \abs*{z}^{-2} \overline{z}\)
    (note that \(\abs*{z}^{-2}\) is well-defined as a positive real number because \(\abs*{z}\) is positive real, thanks to Lemma \ref{4.6.11}).
    If \(z\) is zero, \(z = 0\), we leave the reciprocal \(0^{-1}\) undefined.
\end{definition}

\begin{additional corollary}\label{ac 4.6.5}
If \(z, w \in \mathbf{C}\) such that \(z = w \neq 0\), then \(z^{-1} = w^{-1}\).
\end{additional corollary}

\begin{proof}
    \begin{align*}
                 & z = w                                                                                                     \\
        \implies & \overline{z} = \overline{w}                             & \text{(by Additional Corollary \ref{ac 4.6.4})} \\
        \implies & z \overline{z} = w \overline{w}                         & \text{(by Additional Corollary \ref{ac 4.6.3})} \\
        \implies & \abs*{z} = \abs*{w}                                     & \text{(by Lemma \ref{4.6.11})}                  \\
        \implies & \abs*{z}^{-2} = \abs*{w}^{-2}                           & (z = w \neq 0)                                  \\
        \implies & \abs*{z}^{-2} \overline{z} = \abs*{w}^{-2} \overline{w} & \text{(by Additional Corollary \ref{ac 4.6.3})} \\
        \implies & z^{-1} = w^{-1}.                                        & \text{(by Definition \ref{4.6.12})}
    \end{align*}
\end{proof}

\begin{note}
    From the Definition \ref{4.6.12} and Lemma \ref{4.6.11}, we see that
    \[
        z z^{-1} = z^{-1} z = \abs*{z}^{-2} \overline{z} z = \abs*{z}^{-2} \abs*{z}^2 = 1,
    \]
    and so \(z^{-1}\) is indeed the reciprocal of \(z\).
    We can thus define a notion of quotient \(z / w\) for any two complex numbers \(z, w\) with \(w \neq 0\) in the usual manner by the formula \(z / w \coloneqq z w^{-1}\).
\end{note}

\begin{lemma}\label{4.6.13}
    If we define \(d(z, w) = \abs*{z - w}\), then the complex numbers \(\mathbf{C}\) with the distance \(d\) form a metric space.
    If \((z_n)_{n = 1}^\infty\) is a sequence of complex numbers, and \(z\) is another complex number, then we have \(\lim_{n \to \infty} z_n = z\) in this metric space if and only if \(\lim_{n \to \infty} \Re(z_n) = \Re(z)\) and \(\lim_{n \to \infty} \Im(z_n) = \Im(z)\).
\end{lemma}

\begin{lemma}[Complex limit laws]\label{4.6.14}
    Let \((z_n)_{n = 1}^\infty\) and \((w_n)_{n = 1}^\infty\) be convergent sequences of complex numbers, and let \(c\) be a complex number.
    Then the sequences \((z_n + w_n)_{n = 1}^\infty\), \((z_n - w_n)_{n = 1}^\infty\), \((c z_n)_{n = 1}^\infty\), \((z_n w_n)_{n = 1}^\infty\), and \((\overline{z_n})_{n = 1}^\infty\) are also convergent, with
    \begin{align*}
        \lim_{n \to \infty} z_n + w_n      & = \lim_{n \to \infty} z_n + \lim_{n \to \infty} w_n                       \\
        \lim_{n \to \infty} z_n - w_n      & = \lim_{n \to \infty} z_n - \lim_{n \to \infty} w_n                       \\
        \lim_{n \to \infty} c z_n          & = c \lim_{n \to \infty} z_n                                               \\
        \lim_{n \to \infty} z_n w_n        & = \bigg(\lim_{n \to \infty} z_n\bigg) \bigg(\lim_{n \to \infty} w_n\bigg) \\
        \lim_{n \to \infty} \overline{z_n} & = \overline{\lim_{n \to \infty} z_n}
    \end{align*}
    Also, if the \(w_n\) are all non-zero and \(\lim_{n \to \infty} w_n\) is also non-zero, then \((z_n / w_n)_{n = 1}^\infty\) is also a convergent sequence, with
    \[
        \lim_{n \to \infty} z_n / w_n = \bigg(\lim_{n \to \infty} z_n\bigg) / \bigg(\lim_{n \to \infty} w_n\bigg).
    \]
\end{lemma}

\begin{note}
    Observe that the real and complex number systems are in fact quite similar;
    they both obey similar laws of arithmetic, and they have similar structure as metric spaces.
    Indeed many of the results in this textbook that were proven for real-valued functions, are also valid for complex-valued functions, simply by replacing ``real'' with ``complex'' in the proofs but otherwise leaving all the other details of the proof unchanged.
    Alternatively, one can always split a complex-valued function \(f\) into real and imaginary parts \(\Re(f)\), \(\Im(f)\), thus \(f = \Re(f) + i \Im(f)\), and then deduce results for the complex-valued function \(f\) from the corresponding results for the real-valued functions \(\Re(f)\), \(\Im(f)\).
    For instance, the theory of pointwise and uniform convergence from Chapter \ref{ch 3}, or the theory of power series from this chapter, extends without any difficulty to complex-valued functions.
    In particular, we can define the complex exponential function in exactly the same manner as for real numbers.
\end{note}

\begin{definition}[Complex exponential]\label{4.6.15}
    If \(z\) is a complex number, we define the function \(\exp(z)\) by the formula
    \[
        \exp(z) \coloneqq \sum_{n = 0}^\infty \frac{z^n}{n!}.
    \]
\end{definition}

\begin{note}
    Inspired by Proposition \ref{4.5.4}, we shall use \(\exp(z)\) and \(e^z\) interchangeably.
    It is also possible to define \(a^z\) for complex \(z\) and real \(a > 0\), but we will not need to do so in this text.
\end{note}

\exercisesection

\begin{exercise}\label{ex 4.6.1}
    Prove Lemma \ref{4.6.4}.
\end{exercise}

\begin{proof}
    See Lemma \ref{4.6.4}.
\end{proof}

\begin{exercise}\label{ex 4.6.2}
    Prove Lemma \ref{4.6.6}.
\end{exercise}

\begin{proof}
    See Lemma \ref{4.6.6}.
\end{proof}

\begin{exercise}\label{ex 4.6.3}
    Prove Lemma \ref{4.6.7}.
\end{exercise}

\begin{proof}
    See Lemma \ref{4.6.7}.
\end{proof}

\begin{exercise}\label{ex 4.6.4}
    Prove Lemma \ref{4.6.9}.
\end{exercise}

\begin{proof}
    See Lemma \ref{4.6.9}.
\end{proof}

\begin{exercise}\label{ex 4.6.5}
    If \(z\) is a complex number, show that \(\Re(z) = \frac{z + \overline{z}}{2}\) and \(\Im(z) = \frac{z - \overline{z}}{2i}\).
\end{exercise}

\begin{proof}
    We have
    \begin{align*}
        \frac{z + \overline{z}}{2} & = \frac{\Re(z) + i \Im(z) + \overline{\Re(z) + i \Im(z)}}{2} & \text{(by Definition \ref{4.6.8})} \\
                                   & = \frac{\Re(z) + i \Im(z) + \Re(z) - i \Im(z)}{2}            & \text{(by Definition \ref{4.6.8})} \\
                                   & = \frac{\Re(z) + \Re(z) + i \Im(z) - i \Im(z)}{2}            & \text{(by Lemma \ref{4.6.4})}      \\
                                   & = \frac{2 \Re(z)}{2}                                         & \text{(by Lemma \ref{4.6.4})}      \\
                                   & = \Re(z)                                                     & (\Re(z) \in \mathbf{R})
    \end{align*}
    and
    \begin{align*}
        \frac{z - \overline{z}}{2i} & = \frac{\Re(z) + i \Im(z) - \overline{\Re(z) + i \Im(z)}}{2i}            & \text{(by Definition \ref{4.6.8})}  \\
                                    & = \frac{\Re(z) + i \Im(z) + \overline{-\big(\Re(z) + i \Im(z)\big)}}{2i} & \text{(by Lemma \ref{4.6.9})}       \\
                                    & = \frac{\Re(z) + i \Im(z) + \overline{-\Re(z) - i \Im(z)\big)}}{2i}      & \text{(by Definition \ref{4.6.3})}  \\
                                    & = \frac{\Re(z) + i \Im(z) - \Re(z) + i \Im(z)}{2i}                       & \text{(by Definition \ref{4.6.8})}  \\
                                    & = \frac{\Re(z) - \Re(z) + i \Im(z) + i \Im(z)}{2i}                       & \text{(by Lemma \ref{4.6.4})}       \\
                                    & = \frac{2i \Im(z)}{2i}                                                   & \text{(by Lemma \ref{4.6.4})}       \\
                                    & = \Im(z).                                                                & \text{(by Definition \ref{4.6.12})}
    \end{align*}
\end{proof}

\begin{exercise}\label{ex 4.6.6}
    Prove Lemma \ref{4.6.11}.
\end{exercise}

\begin{proof}
    See Lemma \ref{4.6.11}.
\end{proof}

\begin{exercise}\label{ex 4.6.7}
    Show that if \(z, w\) are complex numbers with \(w \neq 0\), then \(\abs*{z / w} = \abs*{z} / \abs*{w}\).
\end{exercise}

\begin{proof}
    \begin{align*}
        \abs*{\frac{z}{w}} & = \abs*{z w^{-1}}                                                                  \\
                           & = \abs*{z \abs*{w}^{-2} \overline{w}}        & \text{(by Definition \ref{4.6.12})} \\
                           & = \abs*{z} \abs*{w}^{-2} \abs*{\overline{w}} & \text{(by Lemma \ref{4.6.11})}      \\
                           & = \abs*{z} \abs*{w}^{-2} \abs*{w}            & \text{(by Lemma \ref{4.6.11})}      \\
                           & = \abs*{z} \abs*{w}^{-1}                     & (\abs*{z}, \abs*{w} \in \mathbf{R}) \\
                           & = \frac{\abs*{z}}{\abs*{w}}.
    \end{align*}
\end{proof}

\begin{exercise}\label{ex 4.6.8}
    Let \(z, w\) be non-zero complex numbers.
    Show that \(\abs*{z + w} = \abs*{z} + \abs*{w}\) if and only if there exists a positive real number \(c > 0\) such that \(z = cw\).
\end{exercise}

\begin{proof}
    Since
    \begin{align*}
        \abs*{z + w} & = \abs*{\Re(z) + \Re(w) + i \big(\Im(z) + \Im(w)\big)}             & \text{(by Definition \ref{4.6.8})}  \\
                     & = \sqrt{\big(\Re(z) + \Re(w)\big)^2 + \big(\Im(z) + \Im(w)\big)^2} & \text{(by Definition \ref{4.6.10})}
    \end{align*}
    and
    \begin{align*}
        \abs*{z} + \abs*{w} & = \abs*{\Re(z) + i \Im(z)} + \abs*{\Re(w) + i \Im(w)}                                              & \text{(by Definition \ref{4.6.8})}  \\
                            & = \sqrt{\big(\Re(z)\big)^2 + \big(\Im(z)\big)^2} + \sqrt{\big(\Re(w)\big)^2 + \big(\Im(w)\big)^2}, & \text{(by Definition \ref{4.6.10})}
    \end{align*}
    we know that
    \begin{align*}
             & \abs*{z + w} = \abs*{z} + \abs*{w}                                                                              \\
        \iff & \abs*{z + w}^2 = \abs*{z}^2 + 2 \abs*{z} \abs*{w} + \abs*{w}^2                                                  \\
        \iff & \big(\Re(z) + \Re(w)\big)^2 + \big(\Im(z) + \Im(w)\big)^2                                                       \\
             & = \big(\Re(z)\big)^2 + \big(\Im(z)\big)^2 + 2 \abs*{z} \abs*{w} + \big(\Re(w)\big)^2 + \big(\Im(w)\big)^2       \\
        \iff & \Re(z) \Re(w) + \Im(z) \Im(w) = \abs*{z} \abs*{w}                                                               \\
        \iff & \big(\Re(z) \Re(w)\big)^2 + 2 \Re(z) \Re(w) \Im(z) \Im(w) + \big(\Im(z) \Im(w)\big)^2                           \\
             & = \big(\Re(z) \Re(w)\big)^2 + \big(\Re(z) \Im(w)\big)^2 + \big(\Im(z) \Re(w)\big)^2 + \big(\Im(z) \Im(w)\big)^2 \\
        \iff & 2 \Re(z) \Re(w) \Im(z) \Im(w) = \big(\Re(z) \Im(w)\big)^2 + \big(\Im(z) \Re(w)\big)^2                           \\
        \iff & \big(\Re(z) \Im(w)\big)^2 - 2 \Re(z) \Re(w) \Im(z) \Im(w) + \big(\Im(z) \Re(w)\big)^2 = 0                       \\
        \iff & \big(\Re(z) \Im(w) - \Im(z) \Re(w)\big)^2 = 0                                                                   \\
        \iff & \Re(z) \Im(w) - \Im(z) \Re(w) = 0                                                                               \\
        \iff & \Re(z) \Im(w) = \Im(z) \Re(w).
    \end{align*}
    By hypothesis we know that \(w \neq 0 \neq z\).
    Thus we can split into two cases:
    \begin{itemize}
        \item \(\Re(z) \neq 0\) and \(\Re(w) \neq 0\).
              Then we have
              \begin{align*}
                       & \abs*{z + w} = \abs*{z} + \abs*{w}            \\
                  \iff & \Re(z) \Im(w) = \Im(z) \Re(w)                 \\
                  \iff & \frac{\Im(z)}{\Re(z)} = \frac{\Im(w)}{\Re(w)} \\
                  \iff & \exists\ c \in \mathbf{R}^+ : z = cw.
              \end{align*}
        \item \(\Re(z) \neq 0\) and \(\Im(w) \neq 0\).
              Then we have
              \begin{align*}
                       & \abs*{z + w} = \abs*{z} + \abs*{w}            \\
                  \iff & \Re(z) \Im(w) = \Im(z) \Re(w) \neq 0          \\
                  \iff & \frac{\Im(z)}{\Re(z)} = \frac{\Im(w)}{\Re(w)} \\
                  \iff & \exists\ c \in \mathbf{R}^+ : z = cw.
              \end{align*}
        \item \(\Im(z) \neq 0\) and \(\Re(w) \neq 0\).
              Then we have
              \begin{align*}
                       & \abs*{z + w} = \abs*{z} + \abs*{w}            \\
                  \iff & \Re(z) \Im(w) = \Im(z) \Re(w) \neq 0          \\
                  \iff & \frac{\Im(z)}{\Re(z)} = \frac{\Im(w)}{\Re(w)} \\
                  \iff & \exists\ c \in \mathbf{R}^+ : z = cw.
              \end{align*}
        \item \(\Im(z) \neq 0\) and \(\Im(w) \neq 0\).
              Then we have
              \begin{align*}
                       & \abs*{z + w} = \abs*{z} + \abs*{w}            \\
                  \iff & \Re(z) \Im(w) = \Im(z) \Re(w)                 \\
                  \iff & \frac{\Re(z)}{\Im(z)} = \frac{\Re(w)}{\Im(w)} \\
                  \iff & \exists\ c \in \mathbf{R}^+ : z = cw.
              \end{align*}
    \end{itemize}
    From all cases above we conclude that
    \begin{align*}
             & \abs*{z + w} = \abs*{z} + \abs*{w}    \\
        \iff & \Re(z) \Im(w) = \Im(z) \Re(w)         \\
        \iff & \exists\ c \in \mathbf{R}^+ : z = cw.
    \end{align*}
\end{proof}

\begin{exercise}\label{ex 4.6.9}
    Prove Lemma \ref{4.6.13}.
\end{exercise}

\begin{proof}
    See Lemma \ref{4.6.13}.
\end{proof}

\begin{exercise}\label{ex 4.6.10}
    Show that the complex numbers \(\mathbf{C}\) (with the usual metric \(d\)) form a complete metric space.
\end{exercise}

\begin{exercise}\label{ex 4.6.11}
    Let \(f : \mathbf{R}^2 \to \mathbf{C}\) be the map \(f(a, b) \coloneqq a + bi\).
    Show that \(f\) is a bijection, and that \(f\) and \(f^{-1}\) are both continuous maps.
\end{exercise}

\begin{exercise}\label{ex 4.6.12}
    Show that the complex numbers \(\mathbf{C}\) (with the usual metric \(d\)) form a connected metric space.
\end{exercise}

\begin{exercise}\label{ex 4.6.13}
    Let \(E\) be a subset of \(\mathbf{C}\).
    Show that \(E\) is compact if and only if \(E\) is closed and bounded.
    In particular, show that \(\mathbf{C}\) is not compact.
\end{exercise}

\begin{exercise}\label{ex 4.6.14}
    Prove Lemma \ref{4.6.14}.
\end{exercise}

\begin{proof}
    See Lemma \ref{4.6.14}.
\end{proof}

\begin{exercise}\label{ex 4.6.15}
    The purpose of this exercise is to explain why we do not try to organize the complex numbers into positive and negative parts.
    Suppose that there was a notion of a ``positive complex number'' and a ``negative complex number'' which obeyed the following reasonable axioms (cf. Proposition 4.2.9 in Analysis I):
    \begin{itemize}
        \item (Trichotomy)
              For every complex number \(z\), exactly one of the following statements is true:
              \(z\) is positive, \(z\) is negative, \(z\) is zero.
        \item (Negation)
              If \(z\) is a positive complex number, then \(-z\) is negative.
              If \(z\) is a negative complex number, then \(-z\) is positive.
        \item (Additivity)
              If \(z\) and \(w\) are positive complex numbers, then \(z + w\) is also positive.
        \item (Multiplicativity)
              If \(z\) and \(w\) are positive complex numbers, then \(zw\) is also positive.
    \end{itemize}
    Show that these four axioms are inconsistent, i.e., one can use these axioms to deduce a contradiction.
\end{exercise}

\begin{exercise}\label{ex 4.6.16}
    Prove the ratio test for complex series, and use it to show that the series used to define the complex exponential is absolutely convergent.
    Then prove that \(\exp(z + w) = \exp(z) \exp(w)\) for all complex numbers \(z, w\).
\end{exercise}