\section{Periodic functions}\label{sec 5.1}

\begin{definition}\label{5.1.1}
    Let \(L > 0\) be a real number.
    A function \(f : \mathbf{R} \to \mathbf{C}\) is periodic with period \(L\), or \(L\)-periodic, if we have \(f(x + L) = f(x)\) for every real number \(x\).
\end{definition}

\begin{example}\label{5.1.2}
    The real-valued functions \(f(x) = \sin(x)\) and \(f(x) = \cos(x)\) are \(2\pi\)-periodic, as is the complex-valued function \(f(x) = e^{i x}\).
    These functions are also \(4\pi\)-periodic, \(6\pi\)-periodic, etc.
    The function \(f(x) = x\), however, is not periodic.
    The constant function \(f(x) = 1\) is \(L\)-periodic for every \(L\).
\end{example}

\begin{remark}\label{5.1.3}
    If a function \(f\) is \(L\)-periodic, then we have \(f(x + kL) = f(x)\) for every integer \(k\)
    (why? Use induction for the positive \(k\), and then use a substitution to convert the positive \(k\) result to a negative \(k\) result.
    The \(k = 0\) case is of course trivial).
    In particular, if a function \(f\) is \(1\)-periodic, then we have \(f(x + k) = f(x)\) for every \(k \in \mathbf{Z}\).
    Because of this, \(1\)-periodic functions are sometimes also called \(\mathbf{Z}\)-periodic
    (and \(L\)-periodic functions called \(L \mathbf{Z}\)-periodic).
\end{remark}

\begin{example}\label{5.1.4}
    For any integer \(n\), the functions \(x \mapsto \cos(2 \pi n x)\), \(x \mapsto \sin(2 \pi n x)\), and \(x \mapsto e^{2 \pi i n x}\) are all \(\mathbf{Z}\)-periodic.
    Another example of a \(\mathbf{Z}\)-periodic function is the function \(f : \mathbf{R} \to \mathbf{C}\) defined by \(f(x) \coloneqq 1\) when \(x \in [n, n + \frac{1}{2})\) for some integer \(n\), and \(f(x) \coloneqq 0\) when \(x \in [n + \frac{1}{2}, n + 1)\) for some integer \(n\).
    This function is an example of a \emph{square wave}.
\end{example}

\begin{note}
    In order to completely specify a \(\mathbf{Z}\)-periodic function \(f : \mathbf{R} \to \mathbf{C}\), one only needs to specify its values on the interval \([0, 1)\), since this will determine the values of \(f\) everywhere else.
    This is because every real number \(x\) can be written in the form \(x = k + y\) where \(k\) is an integer (called the \emph{integer part} of \(x\), and sometimes denoted \([x]\)) and \(y \in [0, 1)\) (this is called the \emph{fractional part} of \(x\), and sometimes denoted \(\{x\}\)).
    Because of this, sometimes when we wish to describe a \(\mathbf{Z}\)-periodic function \(f\) we just describe what it does on the interval \([0, 1)\), and then say that it is \emph{extended periodically} to all of \(\mathbf{R}\).
    This means that we define \(f(x)\) for any real number \(x\) by setting \(f(x) \coloneqq f(y)\), where we have decomposed \(x = k + y\) as discussed above.
    (One can in fact replace the interval \([0, 1)\) by any other half-open interval of length \(1\), but we will not do so here.)
\end{note}

\begin{note}
    The space of complex-valued continuous \(\mathbf{Z}\)-periodic functions is denoted
    \[
        C(\mathbf{R} / \mathbf{Z} ; \mathbf{C}).
    \]
    (The notation \(\mathbf{R} / \mathbf{Z}\) comes from algebra, and denotes the quotient group of the additive group \(\mathbf{R}\) by the additive group \(\mathbf{Z}\);
    more information in this can be found in any algebra text.)
    By ``continuous'' we mean continuous at all points on \(\mathbf{R}\);
    merely being continuous on an interval such as \([0, 1]\) will not suffice, as there may be a discontinuity between the left and right limits at \(1\) (or at any other integer).
    Thus for instance, the functions \(x \mapsto \sin(2 \pi n x)\), \(x \mapsto \cos(2 \pi n x)\), and \(x \mapsto e^{2 \pi i n x}\) are all elements of \(C(\mathbf{R} / \mathbf{Z} ; \mathbf{C})\), as are the constant functions, however the square wave function in Example \ref{5.1.4} is not in \(C(\mathbf{R} / \mathbf{Z} ; \mathbf{C})\) because it is not continuous at every integer.
    Also the function \(\sin(x)\) would also not qualify to be in \(C(\mathbf{R} / \mathbf{Z} ; \mathbf{C})\) since it is not \(\mathbf{Z}\)-periodic.
\end{note}

\begin{lemma}[Basic properties of \(C(\mathbf{R} / \mathbf{Z} ; \mathbf{C})\)]\label{5.1.5}
    \quad
    \begin{enumerate}
        \item (Boundedness)
              If \(f \in C(\mathbf{R} / \mathbf{Z} ; \mathbf{C})\), then \(f\) is bounded
              (i.e., there
              exists a real number \(M > 0\) such that \(\abs*{f(x)} \leq M\) for all \(x \in \mathbf{R}\)).
        \item (Vector space and algebra properties)
              If \(f, g \in C(\mathbf{R} / \mathbf{Z} ; \mathbf{C})\), then the functions \(f + g\), \(f - g\), and \(f g\) are also in \(C(\mathbf{R} / \mathbf{Z} ; \mathbf{C})\).
              Also, if \(c\) is any complex number, then the function \(cf\) is also in \(C(\mathbf{R} / \mathbf{Z} ; \mathbf{C})\).
        \item (Closure under uniform limits)
              If \((f_n)_{n = 1}^\infty\) is a sequence of functions in \(C(\mathbf{R} / \mathbf{Z} ; \mathbf{C})\) which converges uniformly to another function \(f : \mathbf{R} \to \mathbf{C}\), then \(f\) is also in \(C(\mathbf{R} / \mathbf{Z} ; \mathbf{C})\).
    \end{enumerate}
\end{lemma}

\begin{note}
    One can make \(C(\mathbf{R} / \mathbf{Z} ; \mathbf{C})\) into a metric space by re-introducing the now familiar sup-norm metric
    \[
        d_\infty(f, g) = \sup_{x \in \mathbf{R}} \abs*{f(x) - g(x)} = \sup_{x \in [0, 1)} \abs*{f(x) - g(x)}
    \]
    of uniform convergence.
\end{note}

\exercisesection

\begin{exercise}\label{ex 5.1.1}
    Show that every real number \(x\) can be written in exactly one way in the form \(x = k + y\), where \(k\) is an integer and \(y \in [0, 1)\).
\end{exercise}

\begin{exercise}\label{ex 5.1.2}
    Prove Lemma \ref{5.1.5}.
\end{exercise}

\begin{proof}
    See Lemma \ref{5.1.5}.
\end{proof}

\begin{exercise}\label{ex 5.1.3}
    Show that \(C(\mathbf{R} / \mathbf{Z} ; \mathbf{C})\) with the sup-norm metric \(d_\infty\) is a metric space.
    Furthermore, show that this metric space is complete.
\end{exercise}