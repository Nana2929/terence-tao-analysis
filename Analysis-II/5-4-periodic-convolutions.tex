\section{Periodic convolutions}\label{sec 5.4}

\begin{theorem}\label{5.4.1}
  Let \(f \in C(\R / \Z ; \C)\), and let \(\varepsilon > 0\).
  Then there exists a trigonometric polynomial \(P\) such that \(\norm*{f - P}_{\infty} \leq \varepsilon\).
\end{theorem}

\begin{proof}
  Let \(f\) be any element of \(C(\R / \Z ; \C)\);
  we know that \(f\) is bounded (by \cref{5.1.5}(a)), so that we have some \(M > 0\) such that \(\abs{f(x)} \leq M\) for all \(x \in \R\).

  Let \(\varepsilon > 0\) be arbitrary.
  Since \(f\) is uniformly continuous (by \cref{2.3.5}), there exists a \(\delta > 0\) such that \(\abs{f(x) - f(y)} \leq \varepsilon\) whenever \(\abs{x - y} \leq \delta\).
  Now use \cref{5.4.6} to find a trigonometric polynomial \(P\) which is a \((\varepsilon, \delta)\) approximation to the identity.
  Then \(f * P\) is also a trigonometric polynomial (by \cref{ac 5.4.1}).
  We now estimate \(\norm*{f - f * P}_{\infty}\).

  Let \(x\) be any real number.
  We have
  \begin{align*}
     & \abs{f(x) - f * P(x)}                                                                                      \\
     & = \abs{f(x) - P * f(x)}                                                   & \text{(by \cref{5.4.4}(a)(b))} \\
     & = \abs{f(x) - \int_{[0, 1]} f(x - y) P(y) \; dy}                          & \text{(by \cref{5.4.2})}       \\
     & = \abs{\int_{[0, 1]} f(x) P(y) \; dy - \int_{[0, 1]} f(x - y) P(y) \; dy} & \text{(by \cref{5.4.5}(a))}    \\
     & = \abs{\int_{[0, 1]} \big(f(x) - f(x - y)\big) P(y) \; dy}                                                 \\
     & \leq \int_{[0, 1]} \abs{f(x) - f(x - y)} P(y) \; dy.                      & \text{(by \cref{5.2.2})}
  \end{align*}
  The right-hand side can be split as
  \begin{align*}
    \int_{[0, \delta]} \abs{f(x) - f(x - y)} P(y) \; dy & + \int_{[\delta, 1 - \delta]} \abs{f(x) - f(x - y)} P(y) \; dy \\
                                                        & + \int_{[1 - \delta, 1]} \abs{f(x) - f(x - y)} P(y) \; dy
  \end{align*}
  which we can bound from above by
  \begin{align*}
     & \leq \int_{[0, \delta]} \varepsilon P(y) \; dy + \int_{[\delta, 1 - \delta]} 2 M \varepsilon \; dy + \int_{[1 - \delta, 1]} \abs{f(x - 1) - f(x - y)} P(y) \; dy \\
     & \leq \int_{[0, \delta]} \varepsilon P(y) \; dy + \int_{[\delta, 1 - \delta]} 2 M \varepsilon \; dy + \int_{[1 - \delta, 1]} \varepsilon P(y) \; dy               \\
     & \leq \varepsilon + 2 M \varepsilon + \varepsilon                                                                                                                 \\
     & = (2M + 2) \varepsilon.
  \end{align*}
  Thus we have \(\norm*{f - f * P}_{\infty} \leq (2M + 2) \varepsilon\).
  Since \(M\) is fixed and \(\varepsilon\) is arbitrary, we can thus make \(f * P\) arbitrarily close to \(f\) in sup norm, which proves the periodic Weierstrass approximation theorem.
\end{proof}

\begin{note}
  \cref{5.4.1} asserts that any continuous periodic function can be uniformly approximated by trigonometric polynomials.
  To put it another way, if we let
  \[
    P(\R / \Z ; \C)
  \]
  denote the space of all trigonometric polynomials, then the closure of \(P(\R / \Z ; \C)\) in the \(L^\infty\) metric is \(C(\R / \Z ; \C)\).
\end{note}

\begin{note}
  It is possible to prove this theorem directly from the Weierstrass approximation theorem for polynomials (\cref{3.8.3}), and both theorems are a special case of a much more general theorem known as the \emph{Stone-Weierstrass theorem}, which we will not discuss here.
  However we shall instead prove this theorem from scratch, in order to introduce a couple of interesting notions, notably that of periodic convolution.
  The proof here, though, should strongly remind you of the arguments used to prove \cref{3.8.3}.
\end{note}

\begin{definition}[Periodic convolution]\label{5.4.2}
  Let \(f, g \in C(\R / \Z ; \C)\).
  Then we define the periodic convolution \(f * g : \R \to \C\) of \(f\) and \(g\) by the formula
  \[
    f * g(x) \coloneqq \int_{[0, 1]} f(y) g(x - y) \; dy
  \]
\end{definition}

\begin{remark}\label{5.4.3}
  Note that \cref{5.4.2} is slightly different from the convolution for compactly supported functions defined in \cref{3.8.9}, because we are only integrating over \([0, 1]\) and not on all of \(\R\).
  Thus, in principle we have given the symbol \(f * g\) two conflicting meanings.
  However, in practice there will be no confusion, because it is not possible for a non-zero function to both be periodic and compactly supported.
\end{remark}

\begin{lemma}[Basic properties of periodic convolution]\label{5.4.4}
  Let \(f, g, h \in C(\R / \Z ; \C)\).
  \begin{enumerate}
    \item (Closure)
          The convolution \(f * g\) is continuous and \(\Z\)-periodic.
          In other words, \(f * g \in C(\R / \Z ; \C)\).
    \item (Commutativity)
          We have \(f * g = g * f\).
    \item (Bilinearity)
          We have \(f * (g + h) = f * g + f * h\) and \((f + g) * h = f * h + g * h\).
          For any complex number \(c\), we have \(c(f * g) = (cf) * g = f * (cg)\).
  \end{enumerate}
\end{lemma}

\begin{proof}{(a)}
  By \cref{5.2.2} we know that \(f * g\) is continuous on \(\R\).
  Since
  \begin{align*}
    \forall x \in \R, f * g(x + 1) & = \int_{[0, 1]} f(y) g(x + 1 - y) \; dy & \text{(by \cref{5.4.2})} \\
                                   & = \int_{[0, 1]} f(y) g(x - y) \; dy     & (g \in C(\R / \Z ; \C))  \\
                                   & = f * g(x),                             & \text{(by \cref{5.4.2})}
  \end{align*}
  we know that \(f * g \in C(\R / \Z ; \C)\).
\end{proof}

\begin{proof}{(b)}
  Let \(x \in \R\) and let \(\phi : [x - 1, x] \mapsto [0, 1]\) be the function \(\phi(y) = x - y\).
  Then we have
  \begin{align*}
     & f * g(x)                                                                                                                   \\
     & = \int_{[0, 1]} f(y) g(x - y) \; dy                                           & \text{(by \cref{5.4.2})}                   \\
     & = \int_{\big[\phi(x), \phi(x - 1)\big]} f(y) g(x - y) \; dy                                                                \\
     & = -\int_{[x - 1, x]} f\big(\phi(y)\big) g\big(x - \phi(y)\big) \phi'(y) \; dy & \text{(by Exercise 11.10.4 in Analysis I)} \\
     & = \int_{[x - 1, x]} f(x - y) g(y) \; dy                                                                                    \\
     & = \int_{[x - 1, x]} g(y) f(x - y) \; dy.                                                                                   \\
  \end{align*}
  Let \([x]\) be the integer defined in \cref{ex 5.1.1}.
  Then we have
  \begin{align*}
             & [x] \leq x < [x] + 1     \\
    \implies & [x] - 1 \leq x - 1 < [x]
  \end{align*}
  and
  \begin{align*}
     & f * g(x)                                                                                                                     \\
     & = \int_{[x - 1, x]} g(y) f(x - y) \; dy                                                                                      \\
     & = \int_{\big[x - 1, [x]\big]} g(y) f(x - y) \; dy + \int_{\big[[x], x\big]} g(y) f(x - y) \; dy                              \\
     & = \int_{\big[[x], x\big]} g(y) f(x - y) \; dy + \int_{\big[x - 1, [x]\big]} g(y) f(x - y) \; dy                              \\
     & = \int_{\big[[x], x\big]} g(y) f(x - y) \; dy                                                                                \\
     & \quad + \int_{\big[x - 1 + 1, [x] + 1\big]} g(y - 1) f(x - y - 1) \; dy                                                      \\
     & = \int_{\big[[x], x\big]} g(y) f(x - y) \; dy + \int_{\big[x, [x] + 1\big]} g(y) f(x - y) \; dy & (f, g \in C(\R / \Z ; \C)) \\
     & = \int_{\big[[x], [x] + 1\big]} g(y) f(x - y) \; dy                                                                          \\
     & = \int_{\big[[x] - [x], [x] + 1 - [x]\big]} g(y + [x]) f(x - y + [x]) \; dy                                                  \\
     & = \int_{[0, 1]} g(y) f(x - y) \; dy                                                             & (f, g \in C(\R / \Z ; \C)) \\
     & = g * f(x).                                                                                     & \text{(by \cref{5.4.2})}
  \end{align*}
  Since \(x\) is arbitrary, we conclude that \(f * g = g * f\).
\end{proof}

\begin{proof}{(c)}
  By \cref{5.1.5}(b) we know that \(f + g, g + h, cf, cg \in C(\R / \Z ; \C)\).
  Thus \(f * (g + h), (f + g) * h, (cf) * g, f * (cg)\) are well-defined.
  Let \(x \in \R\).
  Then we have
  \begin{align*}
     & \big(f * (g + h)\big)(x)                                                                                           \\
     & = \int_{[0, 1]} f(y) \cdot (g + h)(x - y) \; dy                         & \text{(by \cref{5.4.2})}                 \\
     & = \int_{[0, 1]} f(y) \cdot \big(g(x - y) + h(x - y)\big) \; dy                                                     \\
     & = \int_{[0, 1]} f(y) g(x - y) + f(y) h(x - y) \; dy                                                                \\
     & = \int_{[0, 1]} f(y) g(x - y) \; dy + \int_{[0, 1]} f(y) h(x - y) \; dy & \text{(cf the proof of \cref{5.2.5}(c))} \\
     & = (f * g)(x) + (f * h)(x)                                               & \text{(by \cref{5.4.2})}                 \\
     & = (f * g + f * h)(x)
  \end{align*}
  and
  \begin{align*}
    \big((cf) * g\big)(x) & = \int_{[0, 1]} (cf)(y) \cdot g(x - y) \; dy & \text{(by \cref{5.4.2})}                 \\
                          & = \int_{[0, 1]} c f(y) g(x - y) \; dy                                                   \\
                          & = c \int_{[0, 1]} f(y) g(x - y) \; dy        & \text{(cf the proof of \cref{5.2.5}(c))} \\
                          & = c (f * g)(x).                              & \text{(by \cref{5.4.2})}
  \end{align*}
  Since \(x\) is arbitrary, we conclude that \(f * (g + h) = f * g + f * h\) and \((cf) * g = c (f * g)\).
  This implies
  \begin{align*}
    (f + g) * h & = h * (f + g)   & \text{(by \cref{5.4.4}(b))}   \\
                & = h * f + h * g & \text{(from the proof above)} \\
                & = f * h + g * h & \text{(by \cref{5.4.4}(b))}
  \end{align*}
  and
  \begin{align*}
    f * (cg) & = (cg) * f  & \text{(by \cref{5.4.4}(b))}   \\
             & = c(g * f)  & \text{(from the proof above)} \\
             & = c(f * g). & \text{(by \cref{5.4.4}(b))}
  \end{align*}
\end{proof}

\begin{additional corollary}\label{ac 5.4.1}
Now we observe an interesting identity:
for any \(f \in C(\R / \Z ; \C)\) and any integer \(n\), we have
\[
  f * e_n = \hat{f}(n) e_n.
\]
To prove this, we compute
\begin{align*}
  f * e_n(x) & = \int_{[0, 1]} f(y) e_n(x - y) \; dy                        & \text{(by \cref{5.4.2})}                 \\
             & = \int_{[0, 1]} f(y) e^{2 \pi i n (x - y)} \; dy             & \text{(by \cref{5.3.1})}                 \\
             & = e^{2 \pi i n x} \int_{[0, 1]} f(y) e^{- 2 \pi i n y} \; dy & \text{(cf the proof of \cref{5.2.5}(c))} \\
             & = \inner*{f, e_n} e^{2 \pi i n x}                            & \text{(by \cref{5.2.1})}                 \\
             & = \hat{f}(n) e^{2 \pi i n x}                                 & \text{(by \cref{5.3.7})}                 \\
             & = \hat{f}(n) e_n                                             & \text{(by \cref{5.3.1})}
\end{align*}
as desired.
More generally, we see from \cref{5.4.4}(c) that for any trigonometric polynomial \(P = \sum_{n = -N}^N c_n e_n\), we have
\[
  f * P = \sum_{n = -N}^N c_n (f * e_n) = \sum_{n = -N}^N \hat{f}(n) c_n e_n.
\]
Thus the periodic convolution of any function in \(C(\R / \Z ; \C)\) with a trigonometric polynomial, is again a trigonometric polynomial.
(Compare with \cref{3.8.13}.)
\end{additional corollary}

\begin{definition}[Periodic approximation to the identity]\label{5.4.5}
  Let \(\varepsilon > 0\) and \(0 < \delta < 1 / 2\).
  A function \(f \in C(\R / \Z ; \C)\) is said to be a \emph{periodic \((\varepsilon, \delta)\) approximation to the identity} if the following properties are true:
  \begin{enumerate}
    \item \(f(x) \geq 0\) for all \(x \in \R\), and \(\int_{[0, 1]} f(x) \; dx = 1\).
    \item We have \(f(x) < \varepsilon\) for all \(\delta \leq \abs{x} \leq 1 - \delta\).
  \end{enumerate}
\end{definition}

\begin{additional corollary}[Fejér kernel]\label{ac 5.4.2}
Let \(N \geq 1\) be an integer.
Then we have
\[
  \sum_{n = -N}^N \bigg(1 - \frac{\abs{n}}{N}\bigg) e_n = \frac{1}{N} \abs{\sum_{n = 0}^{N - 1} e_n}^2.
\]
\end{additional corollary}

\begin{proof}
  First we claim that
  \[
    \sum_{n = 0}^{2N - 2} (N - \abs{n - N + 1}) \cdot e_n = \bigg(\sum_{n = 0}^{N - 1} e_n\bigg)^2.
  \]
  We proof the claim by induction on \(N\) and we start with \(N = 1\).
  For \(N = 1\), we have
  \begin{align*}
    \sum_{n = 0}^0 (1 - \abs{n - 1 + 1}) \cdot e_n & = \sum_{n = 0}^0 (1 - \abs{n}) e_n                                \\
                                                   & = 1 e_0                                                           \\
                                                   & = 1                                 & \text{(by \cref{4.5.2}(e))} \\
                                                   & = e_0^2                             & \text{(by \cref{4.5.2}(e))} \\
                                                   & = \bigg(\sum_{n = 0}^0 e_n\bigg)^2.
  \end{align*}
  Thus the base case holds.
  Suppose inductively that the claim is true for some \(N \geq 1\).
  Then for \(N + 1\), we want to show that
  \[
    \sum_{n = 0}^{2(N + 1) - 2} \big(N + 1 - \abs{n - (N + 1) + 1}\big) \cdot e_n = \sum_{n = 0}^{2N} (N + 1 - \abs{n - N}) \cdot e_n = \bigg(\sum_{n = 0}^N e_n\bigg)^2.
  \]
  This is true since
  \begin{align*}
     & \bigg(\sum_{n = 0}^N e_n\bigg)^2                                                                                                                                  \\
     & = \Bigg(\bigg(\sum_{n = 0}^{N - 1} e_n\bigg) + e_N\Bigg)^2                                                                     & \text{(by \cref{4.6.4})}         \\
     & = \bigg(\sum_{n = 0}^{N - 1} e_n\bigg)^2 + 2 e_N \bigg(\sum_{n = 0}^{N - 1} e_n\bigg) + e_N^2                                  & \text{(by \cref{4.6.6})}         \\
     & = \sum_{n = 0}^{2N - 2} (N - \abs{n - N + 1}) \cdot e_n + 2 e_N \bigg(\sum_{n = 0}^{N - 1} e_n\bigg) + e_N^2                   & \text{(by induction hypothesis)} \\
     & = \sum_{n = 0}^{N - 1} (N - \abs{n - N + 1}) \cdot e_n + \sum_{n = N}^{2N - 2} (N - \abs{n - N + 1}) \cdot e_n                 & \text{(by \cref{4.6.4})}         \\
     & \quad + 2 e_N \bigg(\sum_{n = 0}^{N - 1} e_n\bigg) + e_N^2                                                                                                        \\
     & = \sum_{n = 0}^{N - 1} (n + 1) \cdot e_n + \sum_{n = N}^{2N - 2} (2N - n - 1) \cdot e_n                                                                           \\
     & \quad + 2 e_N \bigg(\sum_{n = 0}^{N - 1} e_n\bigg) + e_N^2                                                                                                        \\
     & = \sum_{n = 0}^{N - 1} n e_n + \sum_{n = 0}^{N - 1} e_n + \sum_{n = N}^{2N - 2} (2N - n) \cdot e_n - \sum_{n = N}^{2N - 2} e_n & \text{(by \cref{4.6.6})}         \\
     & \quad + 2 \bigg(\sum_{n = 0}^{N - 1} e_N e_n\bigg) + e_N^2                                                                     & \text{(by \cref{4.6.6})}         \\
     & = \sum_{n = 0}^{N - 1} n e_n + \sum_{n = N}^{2N - 2} (2N - n) \cdot e_n                                                        & \text{(by \cref{4.6.4})}         \\
     & \quad + \sum_{n = 0}^{N - 1} e_n - \sum_{n = N}^{2N - 2} e_n + 2 \bigg(\sum_{n = 0}^{N - 1} e_N e_n\bigg) + e_N^2                                                 \\
     & = \sum_{n = 0}^{N - 1} (N - \abs{n - N}) \cdot e_n + \sum_{n = N}^{2N - 2} (N - \abs{n - N}) \cdot e_n                                                            \\
     & \quad + \sum_{n = 0}^{N - 1} e_n - \sum_{n = N}^{2N - 2} e_n + 2 \bigg(\sum_{n = 0}^{N - 1} e_N e_n\bigg) + e_N^2
  \end{align*}
  \begin{align*}
     & = \sum_{n = 0}^{2N - 2} (N - \abs{n - N}) \cdot e_n                                                               & \text{(conti. from above)}   \\
     & \quad + \sum_{n = 0}^{N - 1} e_n - \sum_{n = N}^{2N - 2} e_n + 2 \bigg(\sum_{n = 0}^{N - 1} e_N e_n\bigg) + e_N^2                                \\
     & = \sum_{n = 0}^{2N - 2} (N + 1 - \abs{n - N}) \cdot e_n - \sum_{n = 0}^{2N - 2} e_n                               & \text{(by \cref{4.6.6})}     \\
     & \quad + \sum_{n = 0}^{N - 1} e_n - \sum_{n = N}^{2N - 2} e_n + 2 \bigg(\sum_{n = 0}^{N - 1} e_N e_n\bigg) + e_N^2                                \\
     & = \sum_{n = 0}^{2N - 2} (N + 1 - \abs{n - N}) \cdot e_n                                                           & \text{(by \cref{4.6.6})}     \\
     & \quad - 2 \bigg(\sum_{n = N}^{2N - 2} e_n\bigg) + 2 \bigg(\sum_{n = 0}^{N - 1} e_N e_n\bigg) + e_N^2                                             \\
     & = \sum_{n = 0}^{2N} (N + 1 - \abs{n - N}) \cdot e_n - 2 e_{2N - 1} - e_{2N}                                       & \text{(by \cref{4.6.6})}     \\
     & \quad - 2 \bigg(\sum_{n = N}^{2N - 2} e_n\bigg) + 2 \bigg(\sum_{n = 0}^{N - 1} e_N e_n\bigg) + e_N^2                                             \\
     & = \sum_{n = 0}^{2N} (N + 1 - \abs{n - N}) \cdot e_n - 2 e_{2N - 1} - e_{2N}                                                                      \\
     & \quad - 2 \bigg(\sum_{n = N}^{2N - 2} e_n\bigg) + 2 \bigg(\sum_{n = 0}^{N - 1} e_{n + N}\bigg) + e_{2N}           & \text{(by \cref{ex 4.6.16})} \\
     & = \sum_{n = 0}^{2N} (N + 1 - \abs{n - N}) \cdot e_n                                                               & \text{(by \cref{4.6.4})}     \\
     & \quad - 2 \bigg(\sum_{n = N}^{2N - 1} e_n\bigg) + 2 \bigg(\sum_{n = 0}^{N - 1} e_{n + N}\bigg)                                                   \\
     & = \sum_{n = 0}^{2N} (N + 1 - \abs{n - N}) \cdot e_n                                                                                              \\
     & \quad - 2 \bigg(\sum_{n = N}^{2N - 1} e_n\bigg) + 2 \bigg(\sum_{n = N}^{2N - 1} e_n\bigg)                                                        \\
     & = \sum_{n = 0}^{2N} (N + 1 - \abs{n - N}) \cdot e_n.
  \end{align*}
  This closes the induction.

  Using the claim above we have
  \begin{align*}
     & \sum_{n = -N}^N \bigg(1 - \frac{\abs{n}}{N}\bigg) e_n                                                                              \\
     & = \frac{1}{N} \sum_{n = -N}^N (N - \abs{n}) e_n                                                    & \text{(by \cref{4.6.6})}      \\
     & = \frac{1}{N} \sum_{n = -(N - 1)}^{N - 1} (N - \abs{n}) \cdot e_n                                                                  \\
     & = \frac{1}{N} \sum_{n = 0}^{2N - 2} (N - \abs{n - N + 1}) \cdot e_{n - N + 1}                                                      \\
     & = \frac{1}{N} \sum_{n = 0}^{2N - 2} (N - \abs{n - N + 1}) \cdot e_n \cdot e_{-N + 1}               & \text{(by \cref{ex 4.6.16})}  \\
     & = \frac{e_{-N + 1}}{N} \sum_{n = 0}^{2N - 2} (N - \abs{n - N + 1}) \cdot e_n                       & \text{(by \cref{4.6.6})}      \\
     & = \frac{e_{-N + 1}}{N} \bigg(\sum_{n = 0}^{N - 1} e_n\bigg)^2                                      & \text{(from the claim above)} \\
     & = \frac{e_{-N + 1}}{N} \bigg(\sum_{n = 0}^{N - 1} e_n\bigg) \bigg(\sum_{n = 0}^{N - 1} e_n\bigg)                                   \\
     & = \frac{1}{N} \bigg(\sum_{n = 0}^{N - 1} e_n\bigg) \bigg(\sum_{n = 0}^{N - 1} e_{-N + 1} e_n\bigg) & \text{(by \cref{4.6.6})}      \\
     & = \frac{1}{N} \bigg(\sum_{n = 0}^{N - 1} e_n\bigg) \bigg(\sum_{n = 0}^{N - 1} e_{n - N + 1}\bigg)  & \text{(by \cref{ex 4.6.16})}  \\
     & = \frac{1}{N} \bigg(\sum_{n = 0}^{N - 1} e_n\bigg) \bigg(\sum_{n = 0}^{N - 1} e_{-n}\bigg)         & \text{(by \cref{4.6.4})}      \\
     & = \frac{1}{N} \bigg(\sum_{n = 0}^{N - 1} e_n\bigg) \bigg(\sum_{n = 0}^{N - 1} \overline{e_n}\bigg) & \text{(by \cref{4.6.15})}     \\
     & = \frac{1}{N} \bigg(\sum_{n = 0}^{N - 1} e_n\bigg) \bigg(\overline{\sum_{n = 0}^{N - 1} e_n}\bigg) & \text{(by \cref{4.6.9})}      \\
     & = \frac{1}{N} \abs{\sum_{n = 0}^{N - 1} e_n}^2.                                                    & \text{(by \cref{4.6.11})}
  \end{align*}
\end{proof}

\begin{lemma}\label{5.4.6}
  For every \(\varepsilon > 0\) and \(0 < \delta < 1 / 2\), there exists a trigonometric polynomial \(P\) which is an \((\varepsilon, \delta)\) approximation to the identity.
\end{lemma}

\begin{proof}
  Let \(N \geq 1\) be an integer.
  We define the \emph{Fejér kernel} \(F_N\) to be the function
  \[
    F_N = \sum_{n = -N}^N \bigg(1 - \frac{\abs{n}}{N}\bigg) e_n.
  \]
  Clearly \(F_N\) is a trigonometric polynomial.
  We observe the identity
  \[
    F_N = \frac{1}{N} \abs{\sum_{n = 0}^{N - 1} e_n}^2
  \]
  by \cref{ac 5.4.2}.
  But from the geometric series formula (Lemma 7.3.3 in Analysis I) we have
  \begin{align*}
    \sum_{n = 0}^{N - 1} e_n(x) & = \sum_{n = 0}^{N - 1} \big(e_1(x)\big)^n                                                                               & \text{(by \cref{ex 4.6.16})} \\
                                & = \frac{\big(e_1(x)\big)^N - 1}{e_1(x) - 1}                                                                             & \text{(geometric series)}    \\
                                & = \frac{\big(e_1(x)\big)^N - e_0(x)}{e_1(x) - e_0(x)}                                                                   & \text{(by \cref{4.5.2}(e))}  \\
                                & = \frac{e_N(x) - e_0(x)}{e_1(x) - e_0(x)}                                                                               & \text{(by \cref{ex 4.6.16})} \\
                                & = \frac{e^{2 \pi i N x} - e^0}{e^{2 \pi i x} - e^0}                                                                     & \text{(by \cref{5.3.1})}     \\
                                & = \frac{e^{\pi i N x} e^{\pi i N x} - e^{\pi i N x} e^{-\pi i N x}}{e^{\pi i x} e^{\pi i x} - e^{\pi i x} e^{-\pi i x}} & \text{(by \cref{ex 4.6.16})} \\
                                & = \frac{e^{\pi i N x} (e^{\pi i N x} - e^{-\pi i N x})}{e^{\pi i x} (e^{\pi i x} - e^{-\pi i x})}                       & \text{(by \cref{4.6.6})}     \\
                                & = \frac{e^{\pi i (N - 1) x} (e^{\pi i N x} - e^{-\pi i N x})}{e^{\pi i x} - e^{-\pi i x}}                               & \text{(by \cref{4.6.12})}    \\
                                & = \frac{2i e^{\pi i (N - 1) x} \sin(\pi N x)}{2i \sin(\pi x)}                                                           & \text{(by \cref{4.7.1})}     \\
                                & = \frac{e^{\pi i (N - 1) x} \sin(\pi N x)}{\sin(\pi x)}                                                                 & \text{(by \cref{4.6.12})}
  \end{align*}
  when \(x\) is not an integer, and hence we have the formula
  \begin{align*}
    F_N(x) & = \frac{1}{N} \abs{\sum_{n = 0}^{N - 1} e_n(x)}^2                                                       & \text{(by \cref{ac 5.4.2})}   \\
           & = \frac{1}{N} \abs{\frac{e^{\pi i (N - 1) x} \sin(\pi N x)}{\sin(\pi x)}}^2                             & \text{(from the proof above)} \\
           & = \frac{\abs{e^{\pi i (N - 1) x}}^2 \abs{\sin(\pi N x)}^2}{N \abs{\sin(\pi x)}^2}                       & \text{(by \cref{ex 4.6.7})}   \\
           & = \frac{\abs{e^{\pi i (N - 1) x}}^2 \big(\sin(\pi N x)\big)^2}{N \big(\sin(\pi x)\big)^2}               & (x \in \R)                    \\
           & = \frac{e^{\pi i (N - 1) x} e^{- \pi i (N - 1) x} \big(\sin(\pi N x)\big)^2}{N \big(\sin(\pi x)\big)^2} & \text{(by \cref{4.6.11})}     \\
           & = \frac{e^0 \big(\sin(\pi N x)\big)^2}{N \big(\sin(\pi x)\big)^2}                                       & \text{(by \cref{ex 4.6.16})}  \\
           & = \frac{\big(\sin(\pi N x)\big)^2}{N \big(\sin(\pi x)\big)^2}.                                          & \text{(by \cref{4.5.2}(e))}
  \end{align*}
  When \(x\) is an integer, the geometric series formula does not apply, but one has \(F_N(x) = N\) in that case, as one can see by direct computation.
  In either case we see that \(F_N(x) \geq 0\) for any \(x\).
  Also, we have
  \begin{align*}
     & \int_{[0, 1]} F_N(x) \; dx                                                                                                                           \\
     & = \int_{[0, 1]} \sum_{n = -N}^N \bigg(1 - \frac{\abs{n}}{N}\bigg) e_n(x) \; dx                                                                       \\
     & = \sum_{n = -N}^N \Bigg(\bigg(1 - \frac{\abs{n}}{N}\bigg) \int_{[0, 1]} e_n(x) \; dx\Bigg)                            & \text{(by \cref{5.2.2})}     \\
     & = \sum_{n = -N}^N \Bigg(\bigg(1 - \frac{\abs{n}}{N}\bigg) \int_{[0, 1]} e_{n - 1}(x) e_1(x) \; dx\Bigg)               & \text{(by \cref{ex 4.6.16})} \\
     & = \sum_{n = -N}^N \Bigg(\bigg(1 - \frac{\abs{n}}{N}\bigg) \int_{[0, 1]} e_{n - 1}(x) \overline{e_{-1}(x)} \; dx\Bigg) & \text{(by \cref{4.6.15})}    \\
     & = \sum_{n = -N}^N \Bigg(\bigg(1 - \frac{\abs{n}}{N}\bigg) \inner*{e_{n - 1}, e_{-1}}\Bigg)                            & \text{(by \cref{5.2.1})}     \\
     & = \bigg(1 - \frac{\abs{0}}{N}\bigg) 1                                                                                 & \text{(by \cref{5.3.5})}     \\
     & = 1.
  \end{align*}
  Finally, since \(\sin(\pi N x) \leq 1\), we have
  \[
    F_N(x) \leq \frac{1}{N \big(\sin(\pi x)\big)^2} \leq \frac{1}{N \big(\sin(\pi \delta)\big)^2}
  \]
  whenever \(\delta < \abs{x} < 1 - \delta\)
  (this is because \(\sin\) is increasing on \([0, \pi / 2]\) and decreasing on \([\pi / 2, \pi]\)).
  Thus by choosing \(N\) large enough, we can make \(F_N (x) \leq \varepsilon\) for all \(\delta < \abs{x} < 1 - \delta\).
  Note that since
  \begin{align*}
    \big(\sin(\pi \abs{x})\big)^2 & = \begin{cases}
                                        \big(\sin(\pi x)\big)^2  & \text{if } x \geq 0 \\
                                        \big(\sin(\pi -x)\big)^2 & \text{if } x < 0
                                      \end{cases} \\
                                  & = \begin{cases}
                                        \big(\sin(\pi x)\big)^2  & \text{if } x \geq 0 \\
                                        \big(-\sin(\pi x)\big)^2 & \text{if } x < 0
                                      \end{cases} & \text{(by \cref{4.7.2}(c))} \\
                                  & = \big(\sin(\pi x)\big)^2
  \end{align*}
  and
  \begin{align*}
             & \begin{cases}
                 \delta < \abs{x} \leq \frac{1}{2}  & \text{if } \abs{x} \leq \frac{1}{2} \\
                 \frac{1}{2} < \abs{x} < 1 - \delta & \text{if } \abs{x} > \frac{1}{2}    \\
               \end{cases}                               \\
    \implies & \begin{cases}
                 \pi \delta < \pi \abs{x} \leq \frac{\pi}{2}    & \text{if } \abs{x} \leq \frac{1}{2} \\
                 \frac{\pi}{2} < \pi \abs{x} < \pi - \pi \delta & \text{if } \abs{x} > \frac{1}{2}    \\
               \end{cases}                   \\
    \implies & \begin{cases}
                 \sin(\pi \delta) < \sin(\pi \abs{x}) \leq \sin(\frac{\pi}{2})    & \text{if } \abs{x} \leq \frac{1}{2} \\
                 \sin(\frac{\pi}{2}) > \sin(\pi \abs{x}) > \sin(\pi - \pi \delta) & \text{if } \abs{x} > \frac{1}{2}    \\
               \end{cases} \\
    \implies & \begin{cases}
                 \sin(\pi \delta) < \sin(\pi \abs{x}) \leq \sin(\frac{\pi}{2}) & \text{if } \abs{x} \leq \frac{1}{2} \\
                 \sin(\frac{\pi}{2}) > \sin(\pi \abs{x}) > -\sin(-\pi \delta)  & \text{if } \abs{x} > \frac{1}{2}    \\
               \end{cases}    & \text{(by \cref{4.7.5}(a))}    \\
    \implies & \begin{cases}
                 \sin(\pi \delta) < \sin(\pi \abs{x}) \leq \sin(\frac{\pi}{2}) & \text{if } \abs{x} \leq \frac{1}{2} \\
                 \sin(\frac{\pi}{2}) > \sin(\pi \abs{x}) > \sin(\pi \delta)    & \text{if } \abs{x} > \frac{1}{2}    \\
               \end{cases}    & \text{(by \cref{4.7.2}(c))}    \\
    \implies & \sin(\pi \delta) < \sin(\pi \abs{x}),
  \end{align*}
  we have
  \begin{align*}
             & 0 < \big(\sin(\pi \delta)\big)^2 < \big(\sin(\pi \abs{x})\big)^2                & \text{(by \cref{ac 4.7.2}(d))} \\
    \implies & 0 < \big(\sin(\pi \delta)\big)^2 < \big(\sin(\pi x)\big)^2                      & \text{(from the proof above)}  \\
    \implies & 0 < \frac{1}{\big(\sin(\pi x)\big)^2} < \frac{1}{\big(\sin(\pi \delta)\big)^2}.
  \end{align*}
\end{proof}

\exercisesection

\begin{exercise}\label{ex 5.4.1}
  Show that if \(f : \R \to \C\) is both compactly supported and \(\Z\)-periodic, then it is identically zero.
\end{exercise}

\begin{proof}
  Since \(f\) is compactly supported, by \cref{3.8.4} we know that
  \[
    \exists\ L, U \in \R : \forall x \in \R \setminus [L, U], f(x) = 0.
  \]
  Fix such \(L, U\).
  Let \([U - L]\) be the integer defined in \cref{ex 5.1.1}.
  Then we have
  \begin{align*}
             & \forall x \in [L, U], L \leq x                                                     \\
    \implies & L + [U - L] \leq x + [U - L]                                                       \\
    \implies & U = L + U - L < L + [U - L] + 1 \leq x + [U - L] + 1 & \text{(by \cref{ex 5.1.1})} \\
    \implies & f(x) = f(x + [U - L] + 1) = 0.                       & (f \in C(\R / \Z ; \C))
  \end{align*}
  Thus \(f = 0\).
\end{proof}

\begin{exercise}\label{ex 5.4.2}
  Prove \cref{5.4.4}.
\end{exercise}

\begin{proof}
  See \cref{5.4.4}.
\end{proof}

\begin{exercise}\label{ex 5.4.3}
  Fill in the gaps marked in \cref{5.4.6}.
\end{exercise}

\begin{proof}
  See \cref{5.4.6}.
\end{proof}