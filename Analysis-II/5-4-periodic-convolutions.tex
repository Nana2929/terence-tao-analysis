\section{Periodic convolutions}\label{sec 5.4}

\begin{theorem}\label{5.4.1}
    Let \(f \in C(\mathbf{R} / \mathbf{Z} ; \mathbf{C})\), and let \(\varepsilon > 0\).
    Then there exists a trigonometric polynomial \(P\) such that \(\norm*{f - P}_{\infty} \leq \varepsilon\).
\end{theorem}

\begin{proof}
    Let \(f\) be any element of \(C(\mathbf{R} / \mathbf{Z} ; \mathbf{C})\);
    we know that \(f\) is bounded, so that we have some \(M > 0\) such that \(\abs*{f(x)} \leq M\) for all \(x \in \mathbf{R}\).

    Let \(\varepsilon > 0\) be arbitrary.
    Since \(f\) is uniformly continuous, there exists a \(\delta > 0\) such that \(\abs*{f(x) - f(y)} \leq \varepsilon\) whenever \(\abs*{x - y} \leq \delta\).
    Now use Lemma \ref{5.4.6} to find a trigonometric polynomial \(P\) which is a \((\varepsilon, \delta)\) approximation to the identity.
    Then \(f * P\) is also a trigonometric polynomial.
    We now estimate \(\norm*{f - f * P}_{\infty}\).

    Let \(x\) be any real number.
    We have
    \begin{align*}
        \abs*{f(x) - f * P(x)} & = \abs*{f(x) - P * f(x)}                                                   \\
                               & = \abs*{f(x) - \int_{[0, 1]} f(x - y) P(y) \; dy}                          \\
                               & = \abs*{\int_{[0, 1]} f(x) P(y) \; dy - \int_{[0, 1]} f(x - y) P(y) \; dy} \\
                               & = \abs*{\int_{[0, 1]} \big(f(x) - f(x - y)\big) P(y) \; dy}                \\
                               & \leq \int_{[0, 1]} \abs*{f(x) - f(x - y)} P(y) \; dy.
    \end{align*}
    The right-hand side can be split as
    \begin{align*}
        \int_{[0, \delta]} \abs*{f(x) - f(x - y)} P(y) \; dy & + \int_{[\delta, 1 - \delta]} \abs*{f(x) - f(x - y)} P(y) \; dy \\
                                                             & + \int_{[1 - \delta, 1]} \abs*{f(x) - f(x - y)} P(y) \; dy
    \end{align*}
    which we can bound from above by
    \begin{align*}
         & \leq \int_{[0, \delta]} \varepsilon P(y) \; dy + \int_{[\delta, 1 - \delta]} 2 M \varepsilon \; dy                                                 \\
         & \quad + \int_{[1 - \delta, 1]} \abs*{f(x - 1) - f(x - y)} P(y) \; dy                                                                               \\
         & \leq \int_{[0, \delta]} \varepsilon P(y) \; dy + \int_{[\delta, 1 - \delta]} 2 M \varepsilon \; dy + \int_{[1 - \delta, 1]} \varepsilon P(y) \; dy \\
         & \leq \varepsilon + 2 M \varepsilon + \varepsilon                                                                                                   \\
         & = (2M + 2) \varepsilon.
    \end{align*}
    Thus we have \(\norm*{f - f * P}_{\infty} \leq (2M + 2) \varepsilon\).
    Since \(M\) is fixed and \(\varepsilon\) is arbitrary, we can thus make \(f * P\) arbitrarily close to \(f\) in sup norm, which proves the periodic Weierstrass approximation theorem.
\end{proof}

\begin{note}
    Theorem \ref{5.4.1} asserts that any continuous periodic function can be uniformly approximated by trigonometric polynomials.
    To put it another way, if we let
    \[
        P(\mathbf{R} / \mathbf{Z} ; \mathbf{C})
    \]
    denote the space of all trigonometric polynomials, then the closure of \(P(\mathbf{R} / \mathbf{Z} ; \mathbf{C})\) in the \(L^\infty\) metric is \(C(\mathbf{R} / \mathbf{Z} ; \mathbf{C})\).
\end{note}

\begin{note}
    It is possible to prove this theorem directly from the Weierstrass approximation theorem for polynomials (Theorem \ref{3.8.3}), and both theorems are a special case of a much more general theorem known as the \emph{Stone-Weierstrass theorem}, which we will not discuss here.
    However we shall instead prove this theorem from scratch, in order to introduce a couple of interesting notions, notably that of periodic convolution.
    The proof here, though, should strongly remind you of the arguments used to prove Theorem \ref{3.8.3}.
\end{note}

\begin{definition}[Periodic convolution]\label{5.4.2}
    Let \(f, g \in C(\mathbf{R} / \mathbf{Z} ; \mathbf{C})\).
    Then we define the periodic convolution \(f * g : \mathbf{R} \to \mathbf{C}\) of \(f\) and \(g\) by the formula
    \[
        f * g(x) \coloneqq \int_{[0, 1]} f(y) g(x - y) \; dy
    \]
\end{definition}

\begin{remark}\label{5.4.3}
    Note that Definition \ref{5.4.2} is slightly different from the convolution for compactly supported functions defined in Definition \ref{3.8.9}, because we are only integrating over \([0, 1]\) and not on all of \(\mathbf{R}\).
    Thus, in principle we have given the symbol \(f * g\) two conflicting meanings.
    However, in practice there will be no confusion, because it is not possible for a non-zero function to both be periodic and compactly supported.
\end{remark}

\begin{lemma}[Basic properties of periodic convolution]\label{5.4.4}
    Let \(f, g, h \in C(\mathbf{R} / \mathbf{Z} ; \mathbf{C})\).
    \begin{enumerate}
        \item (Closure)
              The convolution \(f * g\) is continuous and \(\mathbf{Z}\)-periodic.
              In other words, \(f * g \in C(\mathbf{R} / \mathbf{Z} ; \mathbf{C})\).
        \item (Commutativity)
              We have \(f * g = g * f\).
        \item (Bilinearity)
              We have \(f * (g + h) = f * g + f * h\) and \((f + g) * h = f * h + g * h\).
              For any complex number \(c\), we have \(c(f * g) = (cf) * g = f * (cg)\).
    \end{enumerate}
\end{lemma}

\begin{proof}{(a)}
    By Remark \ref{5.2.2} we know that \(f * g\) is continuous on \(\mathbf{R}\).
    Since
    \begin{align*}
        \forall\ x \in \mathbf{R}, f * g(x + 1) & = \int_{[0, 1]} f(y) g(x + 1 - y) \; dy & \text{(by Definition \ref{5.4.2})}              \\
                                                & = \int_{[0, 1]} f(y) g(x - y) \; dy     & (g \in C(\mathbf{R} / \mathbf{Z} ; \mathbf{C})) \\
                                                & = f * g(x),                             & \text{(by Definition \ref{5.4.2})}
    \end{align*}
    we know that \(f * g \in C(\mathbf{R} / \mathbf{Z} ; \mathbf{C})\).
\end{proof}

\begin{proof}{(b)}
    Let \(x \in \mathbf{R}\) and let \(\phi : [x - 1, x] \mapsto [0, 1]\) be the function \(\phi(y) = x - y\).
    Then we have
    \begin{align*}
         & f * g(x)                                                                                                                   \\
         & = \int_{[0, 1]} f(y) g(x - y) \; dy                                           & \text{(by Definition \ref{5.4.2})}         \\
         & = \int_{\big[\phi(x), \phi(x - 1)\big]} f(y) g(x - y) \; dy                                                                \\
         & = -\int_{[x - 1, x]} f\big(\phi(y)\big) g\big(x - \phi(y)\big) \phi'(y) \; dy & \text{(by Exercise 11.10.4 in Analysis I)} \\
         & = \int_{[x - 1, x]} f(x - y) g(y) \; dy                                                                                    \\
         & = \int_{[x - 1, x]} g(y) f(x - y) \; dy.                                                                                   \\
    \end{align*}
    Let \([x]\) be the integer defined in Exercise \ref{ex 5.1.1}.
    Then we have
    \begin{align*}
                 & [x] \leq x < [x] + 1     \\
        \implies & [x] - 1 \leq x - 1 < [x]
    \end{align*}
    and
    \begin{align*}
         & f * g(x)                                                                                                                                             \\
         & = \int_{[x - 1, x]} g(y) f(x - y) \; dy                                                                                                              \\
         & = \int_{\big[x - 1, [x]\big]} g(y) f(x - y) \; dy + \int_{\big[[x], x\big]} g(y) f(x - y) \; dy                                                      \\
         & = \int_{\big[[x], x\big]} g(y) f(x - y) \; dy + \int_{\big[x - 1, [x]\big]} g(y) f(x - y) \; dy                                                      \\
         & = \int_{\big[[x], x\big]} g(y) f(x - y) \; dy                                                                                                        \\
         & \quad + \int_{\big[x - 1 + 1, [x] + 1\big]} g(y - 1) f(x - y - 1) \; dy                                                                              \\
         & = \int_{\big[[x], x\big]} g(y) f(x - y) \; dy + \int_{\big[x, [x] + 1\big]} g(y) f(x - y) \; dy & (f, g \in C(\mathbf{R} / \mathbf{Z} ; \mathbf{C})) \\
         & = \int_{\big[[x], [x] + 1\big]} g(y) f(x - y) \; dy                                                                                                  \\
         & = \int_{\big[[x] - [x], [x] + 1 - [x]\big]} g(y + [x]) f(x - y + [x]) \; dy                                                                          \\
         & = \int_{[0, 1]} g(y) f(x - y) \; dy                                                             & (f, g \in C(\mathbf{R} / \mathbf{Z} ; \mathbf{C})) \\
         & = g * f(x).                                                                                     & \text{(by Definition \ref{5.4.2})}
    \end{align*}
    Since \(x\) is arbitrary, we conclude that \(f * g = g * f\).
\end{proof}

\begin{proof}{(c)}
    By Lemma \ref{5.1.5}(b) we know that \(f + g, g + h, cf, cg \in C(\mathbf{R} / \mathbf{Z} ; \mathbf{C})\).
    Thus \(f * (g + h), (f + g) * h, (cf) * g, f * (cg)\) are well-defined.
    Let \(x \in \mathbf{R}\).
    Then we have
    \begin{align*}
         & \big(f * (g + h)\big)(x)                                                                                                \\
         & = \int_{[0, 1]} f(y) \cdot (g + h)(x - y) \; dy                         & \text{(by Definition \ref{5.4.2})}            \\
         & = \int_{[0, 1]} f(y) \cdot \big(g(x - y) + h(x - y)\big) \; dy                                                          \\
         & = \int_{[0, 1]} f(y) g(x - y) + f(y) h(x - y) \; dy                                                                     \\
         & = \int_{[0, 1]} f(y) g(x - y) \; dy + \int_{[0, 1]} f(y) h(x - y) \; dy & \text{(cf the proof of Lemma \ref{5.2.5}(c))} \\
         & = (f * g)(x) + (f * h)(x)                                               & \text{(by Definition \ref{5.4.2})}            \\
         & = (f * g + f * h)(x)
    \end{align*}
    and
    \begin{align*}
        \big((cf) * g\big)(x) & = \int_{[0, 1]} (cf)(y) \cdot g(x - y) \; dy & \text{(by Definition \ref{5.4.2})}            \\
                              & = \int_{[0, 1]} c f(y) g(x - y) \; dy                                                        \\
                              & = c \int_{[0, 1]} f(y) g(x - y) \; dy        & \text{(cf the proof of Lemma \ref{5.2.5}(c))} \\
                              & = c (f * g)(x).                              & \text{(by Definition \ref{5.4.2})}
    \end{align*}
    Since \(x\) is arbitrary, we conclude that \(f * (g + h) = f * g + f * h\) and \((cf) * g = c (f * g)\).
    This implies
    \begin{align*}
        (f + g) * h & = h * (f + g)   & \text{(by Lemma \ref{5.4.4}(b))} \\
                    & = h * f + h * g & \text{(from the proof above)}    \\
                    & = f * h + g * h & \text{(by Lemma \ref{5.4.4}(b))}
    \end{align*}
    and
    \begin{align*}
        f * (cg) & = (cg) * f  & \text{(by Lemma \ref{5.4.4}(b))} \\
                 & = c(g * f)  & \text{(from the proof above)}    \\
                 & = c(f * g). & \text{(by Lemma \ref{5.4.4}(b))}
    \end{align*}
\end{proof}

\begin{note}
    Now we observe an interesting identity:
    for any \(f \in C(\mathbf{R} / \mathbf{Z} ; \mathbf{C})\) and any integer \(n\), we have
    \[
        f * e_n = \hat{f}(n) e_n.
    \]
    To prove this, we compute
    \begin{align*}
        f * e_n(x) & = \int_{[0, 1]} f(y) e^{2 \pi i n (x - y)} \; dy             \\
                   & = e^{2 \pi i n x} \int_{[0, 1]} f(y) e^{- 2 \pi i n y} \; dy \\
                   & = \hat{f}(n) e^{2 \pi i n x} = \hat{f}(n) e_n
    \end{align*}
    as desired.
    More generally, we see from Lemma \ref{5.4.4}(c) that for any trigonometric polynomial \(P = \sum_{n = -N}^N c_n e_n\), we have
    \[
        f * P = \sum_{n = -N}^N c_n (f * e_n) = \sum_{n = -N}^N \hat{f}(n) c_n e_n.
    \]
    Thus the periodic convolution of any function in \(C(\mathbf{R} / \mathbf{Z} ; \mathbf{C})\) with a trigonometric polynomial, is again a trigonometric polynomial.
    (Compare with Lemma \ref{3.8.13}.)
\end{note}

\begin{definition}[Periodic approximation to the identity]\label{5.4.5}
    Let \(\varepsilon > 0\) and \(0 < \delta < 1 / 2\).
    A function \(f \in C(\mathbf{R} / \mathbf{Z} ; \mathbf{C})\) is said to be a \emph{periodic \((\varepsilon, \delta)\) approximation to the identity} if the following properties are true:
    \begin{enumerate}
        \item \(f(x) \geq 0\) for all \(x \in \mathbf{R}\), and \(\int_{[0, 1]} f(x) \; dx = 1\).
        \item We have \(f(x) < \varepsilon\) for all \(\delta \leq \abs*{x} \leq 1 - \delta\).
    \end{enumerate}
\end{definition}

\begin{lemma}\label{5.4.6}
    For every \(\varepsilon > 0\) and \(0 < \delta < 1 / 2\), there exists a trigonometric polynomial \(P\) which is an \((\varepsilon, \delta)\) approximation to the identity.
\end{lemma}

\begin{proof}
    Let \(N \geq 1\) be an integer.
    We define the \emph{Fejér kernel} \(F_N\) to be the function
    \[
        F_N = \sum_{n = -N}^N \bigg(1 - \frac{\abs*{n}}{N}\bigg) e_n.
    \]
    Clearly \(F_N\) is a trigonometric polynomial.
    We observe the identity
    \[
        F_N = \frac{1}{N} \abs*{\sum_{n = 0}^{N - 1} e_n}^2.
    \]
    But from the geometric series formula (Lemma 7.3.3 in Analysis I) we have
    \[
        \sum_{n = 0}^{N - 1} e_n(x) = \frac{e_N - e_0}{e_1 - e_0} = \frac{e^{\pi i (N - 1) x} \sin(\pi N x)}{\sin(\pi x)}
    \]
    when \(x\) is not an integer, and hence we have the formula
    \[
        F_N(x) = \frac{\big(\sin(\pi N x)\big)^2}{N \big(\sin(\pi x)\big)^2}.
    \]
    When \(x\) is an integer, the geometric series formula does not apply, but one has \(F_N(x) = N\) in that case, as one can see by direct computation.
    In either case we see that \(F_N(x) \geq 0\) for any \(x\).
    Also, we have
    \[
        \int_{[0, 1]} F_N(x) \; dx = \sum_{n = -N}^N \bigg(1 - \frac{\abs*{n}}{N}\bigg) \int_{[0, 1] e_n} = \bigg(1 - \frac{\abs*{0}}{N}\bigg) 1 = 1.
    \]
    Finally, since \(\sin(\pi N x) \leq 1\), we have
    \[
        F_N(x) \leq \frac{1}{N \big(\sin(\pi x)\big)^2} \leq \frac{1}{N \big(\sin(\pi \delta)\big)^2}
    \]
    whenever \(\delta < \abs*{x} < 1 - \delta\)
    (this is because \(\sin\) is increasing on \([0, \pi / 2]\) and decreasing on \([\pi / 2, \pi]\)).
    Thus by choosing \(N\) large enough, we can make \(F_N (x) \leq \varepsilon\) for all \(\delta < \abs*{x} < 1 - \delta\).
\end{proof}

\exercisesection

\begin{exercise}\label{ex 5.4.1}
    Show that if \(f : \mathbf{R} \to \mathbf{C}\) is both compactly supported and \(\mathbf{Z}\)-periodic, then it is identically zero.
\end{exercise}

\begin{exercise}\label{ex 5.4.2}
    Prove Lemma \ref{5.4.4}.
\end{exercise}

\begin{proof}
    See Lemma \ref{5.4.4}.
\end{proof}

\begin{exercise}\label{ex 5.4.3}
    Fill in the gaps marked in Lemma \ref{5.4.6}.
\end{exercise}

\begin{proof}
    See Lemma \ref{5.4.6}.
\end{proof}