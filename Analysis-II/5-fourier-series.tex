\chapter{Fourier series}\label{ch 5}

\begin{note}
    Power series are already immensely useful, especially when dealing with special functions such as the exponential and trigonometric functions discussed earlier.
    However, there are some circumstances where power series are not so useful, because one has to deal with functions (e.g., \(\sqrt{x}\)) which are not real analytic, and so do not have power series.
\end{note}

\begin{note}
    Fortunately, there is another type of series expansion, known as \emph{Fourier series}, which is also a very powerful tool in analysis
    (though used for slightly different purposes).
    Instead of analyzing compactly supported functions, it instead analyzes \emph{periodic functions};
    instead of decomposing into polynomials, it decomposes into \emph{trigonometric polynomials}.
    Roughly speaking, the theory of Fourier series asserts that just about every periodic function can be decomposed as an (infinite) sum of sines and cosines.
\end{note}

\begin{remark}\label{5.0.1}
    Jean-Baptiste Fourier (1768 -- 1830) was, among other things, an administrator accompanying Napoleon on his invasion of Egypt, and then a Prefect in France during Napoleons reign.
    After the Napoleonic wars, he returned to mathematics.
    He introduced Fourier series in an important 1807 paper in which he used them to solve what is now known as the heat equation.
    At the time, the claim that every periodic function could be expressed as a sum of sines and cosines was extremely controversial, even such leading mathematicians as Euler declared that it was impossible.
    Nevertheless, Fourier managed to show that this was indeed the case, although the proof was not completely rigorous and was not totally accepted for almost another hundred years.
\end{remark}

\begin{note}
    For instance, the convergence of Fourier series is usually not uniform (i.e., not in the \(L^\infty\) metric), but instead we have convergence in a different metric, the \(L^2\)-metric.
    We will need to use complex numbers heavily in our theory, while they played only a tangential rôle in power series.
\end{note}

\begin{note}
    The theory of Fourier series (and of related topics such as Fourier integrals and the Laplace transform) is vast, and deserves an entire course in itself.
    It has many, many applications, most directly to differential equations, signal processing, electrical engineering, physics, and analysis, but also to algebra and number theory.
\end{note}

\section{Periodic functions}\label{sec 5.1}
\section{Inner products on periodic functions}\label{sec 5.2}
\section{Trigonometric polynomials}\label{sec 5.3}

\begin{note}
    We now define the concept of a \emph{trigonometric polynomial}.
    Just as polynomials are combinations of the functions \(x^n\) (sometimes called \emph{monomials}), trigonometric polynomials are combinations of the functions \(e^{2 \pi i n x}\) (sometimes called \emph{characters}).
\end{note}

\begin{definition}[Characters]\label{5.3.1}
    For every integer \(n\), we let \(e_n \in C(\mathbf{R} / \mathbf{Z} ; \mathbf{C})\) denote the function
    \[
        e_n(x) \coloneqq e^{2 \pi i n x}.
    \]
    This is sometimes referred to as the \emph{character with frequency \(n\)}.
\end{definition}

\begin{definition}[Trigonometric polynomials]\label{5.3.2}
    A function \(f\) in \(C(\mathbf{R} / \mathbf{Z} ; \mathbf{C})\) is said to be a \emph{trigonometric polynomial} if we can write
    \(f = \sum_{n = -N}^N c_n e_n\) for some integer \(N \geq 0\) and some complex numbers \((c_n)_{n = -N}^N\).
\end{definition}

\setcounter{theorem}{3}
\begin{example}\label{5.3.4}
    For any integer \(n\), the function \(\cos(2 \pi n x)\) is a trigonometric polynomial, since
    \[
        \cos(2 \pi n x) = \frac{e^{2 \pi n x} + e^{- 2 \pi n x}}{2} = \frac{1}{2} e_{-n} + \frac{1}{2} e_n.
    \]
    Similarly the function \(\sin(2 \pi n x) = \frac{-1}{2i} e_{-n} + \frac{1}{2i} e_n\) is a trigonometric polynomial.
    In fact, any linear combination of sines and cosines is also a trigonometric polynomial.
\end{example}

\begin{lemma}[Characters are an orthonormal system]\label{5.3.5}
    For any integers \(n\) and \(m\), we have \(\inner*{e_n, e_m} = 1\) when \(n = m\) and \(\inner*{e_n, e_m} = 0\) when \(n \neq m\).
    Also, we have \(\norm*{e_n}_2 = 1\).
\end{lemma}

\begin{proof}
    Let \(n, m \in \mathbf{Z}\).
    Observe that
    \begin{align*}
        \inner*{e_n, e_m} & = \int_{[0, 1]} e_n(x) \overline{e_m(x)} \; dx                   & \text{(by Definition \ref{5.2.1})}    \\
                          & = \int_{[0, 1]} e^{2 \pi i n x} \overline{e^{2 \pi i m x}} \; dx & \text{(by Definition \ref{5.3.1})}    \\
                          & = \int_{[0, 1]} e^{2 \pi i n x} e^{- 2 \pi i m x} \; dx          & \text{(by Theorem \ref{4.7.2}(c)(f))} \\
                          & = \int_{[0, 1]} e^{2 \pi i n x - 2 \pi i m x} \; dx              & \text{(by Exercise \ref{ex 4.6.16})}  \\
                          & = \int_{[0, 1]} e^{2 \pi i (n - m) x} \; dx.
    \end{align*}
    If \(n = m\), then we have
    \begin{align*}
        \inner*{e_n, e_n} & = \int_{[0, 1]} e^{2 \pi i (n - n) x} \; dx                                      \\
                          & = \int_{[0, 1]} e^0 \; dx                                                        \\
                          & = \int_{[0, 1]} 1 \; dx                     & \text{(by Theorem \ref{4.5.2}(e))} \\
                          & = 1
    \end{align*}
    and
    \begin{align*}
        \norm*{e_n}_2 & = \sqrt{\inner*{e_n, e_n}} & \text{(by Additional Corollary \ref{ac 5.2.1})} \\
                      & = \sqrt{1} = 1.
    \end{align*}
    If \(n \neq m\), then we have
    \begin{align*}
         & \inner*{e_n, e_m}                                                                                                                                \\
         & = \int_{[0, 1]} e^{2 \pi i (n - m) x} \; dx                                                                                                      \\
         & = \int_{[0, 1]} \cos\big(2 \pi (n - m) x\big) + i \sin\big(2 \pi (n - m) x\big) \; dx       & \text{(by Theorem \ref{4.7.2}(f))}                 \\
         & = \int_{[0, 1]} \cos\big(2 \pi (n - m) x\big) \; dx                                         & \text{(by Remark \ref{5.2.2})}                     \\
         & \quad + i \int_{[0, 1]} \sin\big(2 \pi (n - m) x\big) \; dx                                                                                      \\
         & = \bigg(\frac{\sin\big(2 \pi (n - m) x\big)}{2 \pi (n - m)}|_{x = 0}^{x = 1}\bigg)          & \text{(by Theorem \ref{4.7.2}(b))}                 \\
         & \quad + i \bigg(\frac{-\cos\big(2 \pi (n - m) x\big)}{2 \pi (n - m)}|_{x = 0}^{x = 1}\bigg)                                                      \\
         & = 0 - 0                                                                                     & \text{(by Additional Corollary \ref{ac 4.7.2}(c))} \\
         & \quad + i \bigg(\frac{- (-1) + (-1)}{2 \pi (n - m)}\bigg)                                   & \text{(by Additional Corollary \ref{ac 4.7.2}(f))} \\
         & = 0.
    \end{align*}
\end{proof}

\begin{corollary}\label{5.3.6}
    Let \(f = \sum_{n = -N}^N c_n e_n\) be a trigonometric polynomial.
    Then we have the formula
    \[
        c_n = \inner*{f, e_n}
    \]
    for all integers \(-N \leq n \leq N\).
    Also, we have \(0 = \inner*{f, e_n}\) whenever \(n > N\) or \(n < -N\).
    Also, we have the identity
    \[
        \norm*{f}_2^2 = \sum_{n = -N}^N \abs*{c_n}^2.
    \]
\end{corollary}

\begin{proof}
    Let \(m \in \mathbf{N}\).
    Then we have
    \begin{align*}
        \inner*{f, e_m} & = \inner*{\sum_{n = -N}^N (c_n e_n), e_m}         & \text{(by hypothesis)}           \\
                        & = \sum_{n = -N}^N \inner*{c_n e_n, e_m}           & \text{(by Lemma \ref{5.2.5}(c))} \\
                        & = \sum_{n = -N}^N \big(c_n \inner*{e_n, e_m}\big) & \text{(by Lemma \ref{5.2.5}(c))} \\
                        & = \begin{cases}
            c_m & \text{if } -N \leq m \leq N         \\
            0   & \text{if } m > N \text{ or } m < -N
        \end{cases}                      & \text{(by Lemma \ref{5.3.5})}
    \end{align*}
    and
    \begin{align*}
        \norm*{f}_2^2 & = \inner*{f, f}                                            & \text{(by Additional Corollary \ref{ac 5.2.1})} \\
                      & = \inner*{f, \sum_{n = -N}^N (c_n e_n)}                    & \text{(by hypothesis)}                          \\
                      & = \sum_{n = -N}^N \inner*{f, c_n e_n}                      & \text{(by Lemma \ref{5.2.5}(d))}                \\
                      & = \sum_{n = -N}^N \big(\overline{c_n} \inner*{f, e_n}\big) & \text{(by Lemma \ref{5.2.5}(d))}                \\
                      & = \sum_{n = -N}^N \big(\overline{c_n} c_n\big)             & \text{(from the proof above)}                   \\
                      & = \sum_{n = -N}^N \abs*{c_n}^2.                            & \text{(by Lemma \ref{4.6.11})}
    \end{align*}
\end{proof}

\begin{definition}[Fourier transform]\label{5.3.7}
    For any function \(f \in C(\mathbf{R} / \mathbf{Z} ; \mathbf{C})\), and any integer \(n \in \mathbf{Z}\), we define the \(n^{\text{th}}\) \emph{Fourier coefficient of} \(f\), denoted \(\hat{f}(n)\), by the formula
    \[
        \hat{f}(n) \coloneqq \inner*{f, e_n} = \int_{[0, 1]} f(x) e^{- 2 \pi i n x} \; dx.
    \]
    The function \(\hat{f} : \mathbf{Z} \to \mathbf{C}\) is called the \emph{Fourier transform} of \(f\).
\end{definition}

\begin{additional corollary}\label{ac 5.3.1}
From Corollary \ref{5.3.6}, we see that whenever
\[
    f = \sum_{n = -N}^N c_n e_n
\]
is a trigonometric polynomial, we have
\[
    f = \sum_{n = -N}^N \inner*{f, e_n} e_n = \sum_{n = -\infty}^\infty \inner*{f, e_n} e_n
\]
and in particular we have the \emph{Fourier inversion formula}
\[
    f = \sum_{n = -\infty}^\infty \hat{f}(n) e_n
\]
or in other words
\[
    f(x) = \sum_{n = -\infty}^\infty \hat{f}(n) e^{2 \pi i n x}.
\]
The right-hand side is referred to as the \emph{Fourier series} of \(f\).
Also, from the second identity of Corollary \ref{5.3.6} we have the \emph{Plancherel formula}
\[
    \norm*{f}_2^2 = \sum_{n = -\infty}^\infty \abs*{\hat{f}(n)}^2.
\]
\end{additional corollary}

\begin{remark}\label{5.3.8}
    We stress that at present we have only proven the Fourier inversion and Plancherel formulae in the case when \(f\) is a trigonometric polynomial.
    Note that in this case that the Fourier coefficients \(\hat{f}(n)\) are mostly zero (indeed, they can only be non-zero when \(-N \leq n \leq N\)), and so this infinite sum is really just a finite sum in disguise.
    In particular there are no issues about what sense the above series converge in;
    they both converge pointwise, uniformly, and in \(L^2\) metric, since they are just finite sums.
\end{remark}

\begin{note}
    In the next few sections we will extend the Fourier inversion and Plancherel formulae to general functions in \(C(\mathbf{R} / \mathbf{Z} ; \mathbf{C})\), not just trigonometric polynomials.
    (It is also possible to extend the formula to discontinuous functions such as the square wave, but we will not do so here.)
    To do this we will need a version of the Weierstrass approximation theorem, this time requiring that a continuous periodic function be approximated uniformly by \emph{trigonometric} polynomials.
    Just as convolutions were used in the proof of the polynomial Weierstrass approximation theorem, we will also need a notion of convolution tailored for periodic functions.
\end{note}

\exercisesection

\begin{exercise}\label{ex 5.3.1}
    Show that the sum or product of any two trigonometric polynomials is again a trigonometric polynomial.
\end{exercise}

\begin{proof}
    Let \(f, g \in C(\mathbf{R} / \mathbf{Z} ; \mathbf{C})\) such that
    \begin{align*}
         & \exists\ N \in \mathbf{N} : \big((c_n)_{n = -N}^N \text{ is in } \mathbf{C}\big) \land \bigg(f = \sum_{n = -N}^N c_n e_n\bigg); \\
         & \exists\ M \in \mathbf{N} : \big((d_n)_{n = -M}^M \text{ is in } \mathbf{C}\big) \land \bigg(g = \sum_{n = -M}^M d_n e_n\bigg).
    \end{align*}
    Without the loss of generality suppose that \(N \leq M\).
    Then we have
    \begin{align*}
        f + g & = \sum_{n = -N}^N (c_n e_n) + \sum_{n = -M}^M (d_n e_n) \\
              & = \sum_{n = -M}^M (a_n e_n)
    \end{align*}
    where
    \[
        a_n = \begin{cases}
            c_n + d_n & \text{if } -N \leq n \leq N                     \\
            d_n       & \text{if } (-M \leq n < -N) \lor (N < n \leq M)
        \end{cases}
    \]
    For \(fg\), we use induction on \(M\) to show that \(fg\) is trigonometric polynomial.
    For \(M = 0\), we have
    \begin{align*}
        fg & = \bigg(\sum_{n = -N}^N (c_n e_n)\bigg) (d_0 e^0)                                      \\
           & = \bigg(\sum_{n = -N}^N (c_n e_n)\bigg) d_0       & \text{(by Theorem \ref{4.5.2}(e))} \\
           & = \sum_{n = -N}^N (c_n d_0 e_n).
    \end{align*}
    Clearly \(fg\) is trigonometric polynomial and thus the base case holds.
    Suppose inductively that \(fg\) is trigonometric polynomial for some \(M \geq 0\).
    Then for \(M + 1\), we have
    \begin{align*}
        f g & = \bigg(\sum_{n = -N}^N (c_n e_n)\bigg) \bigg(\sum_{m = -(M + 1)}^{M + 1} (d_m e_m)\bigg)                                                        \\
            & = \bigg(\sum_{n = -N}^N (c_n e_n)\bigg) \bigg(\sum_{m = -M}^M (d_m e_m) + d_{-M - 1} e_{-M - 1} + d_{M + 1} e_{M + 1}\bigg)                      \\
            & = \bigg(\sum_{n = -N}^N (c_n e_n)\bigg) \bigg(\sum_{m = -M}^M (d_m e_m)\bigg) + \bigg(\sum_{n = -N}^N (c_n e_n)\bigg) (d_{-M - 1} e_{-M - 1})    \\
            & \quad + \bigg(\sum_{n = -N}^N (c_n e_n)\bigg) (d_{M + 1} e_{M + 1})                                                                              \\
            & = \bigg(\sum_{n = -N}^N (c_n e_n)\bigg) \bigg(\sum_{m = -M}^M (d_m e_m)\bigg) + \sum_{n = -N}^N (c_n d_{-M - 1} e_{n - M - 1})                   \\
            & \quad + \sum_{n = -N}^N (c_n d_{M + 1} e_{n + M + 1})                                                                                            \\
            & = \bigg(\sum_{n = -N}^N (c_n e_n)\bigg) \bigg(\sum_{m = -M}^M (d_m e_m)\bigg) + \sum_{n = -N - M - 1}^{N - M - 1} (c_{n + M + 1} d_{-M - 1} e_n) \\
            & \quad + \sum_{n = -N + M + 1}^{N + M + 1} (c_{n - M - 1} d_{M + 1} e_n).
    \end{align*}
    By setting
    \begin{align*}
         & a_n = \begin{cases}
            c_{n + M + 1} d_{-M - 1} & \text{if } -N - M - 1 \leq n \leq N - M - 1 \\
            0                        & \text{if } N - M - 1 < n \leq N + M + 1
        \end{cases} \\
         & b_n = \begin{cases}
            c_{n - M - 1} d_{M + 1} & \text{if } -N + M + 1 \leq n \leq N + M + 1 \\
            0                       & \text{if } -N - M - 1 \leq n < -N + M + 1
        \end{cases}
    \end{align*}
    we have
    \[
        fg = \bigg(\sum_{n = -N}^N (c_n e_n)\bigg) \bigg(\sum_{m = -M}^M (d_m e_m)\bigg) + \sum_{n = -N - M - 1}^{N + M + 1} (a_n e_n) + \sum_{n = -N - M - 1}^{N + M + 1} (b_n e_n).
    \]
    By induction hypothesis we know that \(\bigg(\sum_{n = -N}^N (c_n e_n)\bigg) \bigg(\sum_{m = -M}^M (d_m e_m)\bigg)\) is trigonometric polynomial.
    Thus from the proof above we know that \(fg\) is trigonometric polynomial, and this closes the induction.
\end{proof}

\begin{exercise}\label{ex 5.3.2}
    Prove Lemma \ref{5.3.5}.
\end{exercise}

\begin{proof}
    See Lemma \ref{5.3.5}.
\end{proof}

\begin{exercise}\label{ex 5.3.3}
    Prove Corollary \ref{5.3.6}.
\end{exercise}

\begin{proof}
    See Corollary \ref{5.3.6}.
\end{proof}
\section{Periodic convolutions}\label{sec 5.4}

\begin{theorem}\label{5.4.1}
    Let \(f \in C(\mathbf{R} / \mathbf{Z} ; \mathbf{C})\), and let \(\varepsilon > 0\).
    Then there exists a trigonometric polynomial \(P\) such that \(\norm*{f - P}_{\infty} \leq \varepsilon\).
\end{theorem}

\begin{proof}
    Let \(f\) be any element of \(C(\mathbf{R} / \mathbf{Z} ; \mathbf{C})\);
    we know that \(f\) is bounded (by Lemma \ref{5.1.5}(a)), so that we have some \(M > 0\) such that \(\abs*{f(x)} \leq M\) for all \(x \in \mathbf{R}\).

    Let \(\varepsilon > 0\) be arbitrary.
    Since \(f\) is uniformly continuous (by Theorem \ref{2.3.5}), there exists a \(\delta > 0\) such that \(\abs*{f(x) - f(y)} \leq \varepsilon\) whenever \(\abs*{x - y} \leq \delta\).
    Now use Lemma \ref{5.4.6} to find a trigonometric polynomial \(P\) which is a \((\varepsilon, \delta)\) approximation to the identity.
    Then \(f * P\) is also a trigonometric polynomial (by Additional Corollary \ref{ac 5.4.1}).
    We now estimate \(\norm*{f - f * P}_{\infty}\).

    Let \(x\) be any real number.
    We have
    \begin{align*}
         & \abs*{f(x) - f * P(x)}                                                                                             \\
         & = \abs*{f(x) - P * f(x)}                                                   & \text{(by Lemma \ref{5.4.4}(a)(b))}   \\
         & = \abs*{f(x) - \int_{[0, 1]} f(x - y) P(y) \; dy}                          & \text{(by Definition \ref{5.4.2})}    \\
         & = \abs*{\int_{[0, 1]} f(x) P(y) \; dy - \int_{[0, 1]} f(x - y) P(y) \; dy} & \text{(by Definition \ref{5.4.5}(a))} \\
         & = \abs*{\int_{[0, 1]} \big(f(x) - f(x - y)\big) P(y) \; dy}                                                        \\
         & \leq \int_{[0, 1]} \abs*{f(x) - f(x - y)} P(y) \; dy.                      & \text{(by Remark \ref{5.2.2})}
    \end{align*}
    The right-hand side can be split as
    \begin{align*}
        \int_{[0, \delta]} \abs*{f(x) - f(x - y)} P(y) \; dy & + \int_{[\delta, 1 - \delta]} \abs*{f(x) - f(x - y)} P(y) \; dy \\
                                                             & + \int_{[1 - \delta, 1]} \abs*{f(x) - f(x - y)} P(y) \; dy
    \end{align*}
    which we can bound from above by
    \begin{align*}
         & \leq \int_{[0, \delta]} \varepsilon P(y) \; dy + \int_{[\delta, 1 - \delta]} 2 M \varepsilon \; dy + \int_{[1 - \delta, 1]} \abs*{f(x - 1) - f(x - y)} P(y) \; dy \\
         & \leq \int_{[0, \delta]} \varepsilon P(y) \; dy + \int_{[\delta, 1 - \delta]} 2 M \varepsilon \; dy + \int_{[1 - \delta, 1]} \varepsilon P(y) \; dy                \\
         & \leq \varepsilon + 2 M \varepsilon + \varepsilon                                                                                                                  \\
         & = (2M + 2) \varepsilon.
    \end{align*}
    Thus we have \(\norm*{f - f * P}_{\infty} \leq (2M + 2) \varepsilon\).
    Since \(M\) is fixed and \(\varepsilon\) is arbitrary, we can thus make \(f * P\) arbitrarily close to \(f\) in sup norm, which proves the periodic Weierstrass approximation theorem.
\end{proof}

\begin{note}
    Theorem \ref{5.4.1} asserts that any continuous periodic function can be uniformly approximated by trigonometric polynomials.
    To put it another way, if we let
    \[
        P(\mathbf{R} / \mathbf{Z} ; \mathbf{C})
    \]
    denote the space of all trigonometric polynomials, then the closure of \(P(\mathbf{R} / \mathbf{Z} ; \mathbf{C})\) in the \(L^\infty\) metric is \(C(\mathbf{R} / \mathbf{Z} ; \mathbf{C})\).
\end{note}

\begin{note}
    It is possible to prove this theorem directly from the Weierstrass approximation theorem for polynomials (Theorem \ref{3.8.3}), and both theorems are a special case of a much more general theorem known as the \emph{Stone-Weierstrass theorem}, which we will not discuss here.
    However we shall instead prove this theorem from scratch, in order to introduce a couple of interesting notions, notably that of periodic convolution.
    The proof here, though, should strongly remind you of the arguments used to prove Theorem \ref{3.8.3}.
\end{note}

\begin{definition}[Periodic convolution]\label{5.4.2}
    Let \(f, g \in C(\mathbf{R} / \mathbf{Z} ; \mathbf{C})\).
    Then we define the periodic convolution \(f * g : \mathbf{R} \to \mathbf{C}\) of \(f\) and \(g\) by the formula
    \[
        f * g(x) \coloneqq \int_{[0, 1]} f(y) g(x - y) \; dy
    \]
\end{definition}

\begin{remark}\label{5.4.3}
    Note that Definition \ref{5.4.2} is slightly different from the convolution for compactly supported functions defined in Definition \ref{3.8.9}, because we are only integrating over \([0, 1]\) and not on all of \(\mathbf{R}\).
    Thus, in principle we have given the symbol \(f * g\) two conflicting meanings.
    However, in practice there will be no confusion, because it is not possible for a non-zero function to both be periodic and compactly supported.
\end{remark}

\begin{lemma}[Basic properties of periodic convolution]\label{5.4.4}
    Let \(f, g, h \in C(\mathbf{R} / \mathbf{Z} ; \mathbf{C})\).
    \begin{enumerate}
        \item (Closure)
              The convolution \(f * g\) is continuous and \(\mathbf{Z}\)-periodic.
              In other words, \(f * g \in C(\mathbf{R} / \mathbf{Z} ; \mathbf{C})\).
        \item (Commutativity)
              We have \(f * g = g * f\).
        \item (Bilinearity)
              We have \(f * (g + h) = f * g + f * h\) and \((f + g) * h = f * h + g * h\).
              For any complex number \(c\), we have \(c(f * g) = (cf) * g = f * (cg)\).
    \end{enumerate}
\end{lemma}

\begin{proof}{(a)}
    By Remark \ref{5.2.2} we know that \(f * g\) is continuous on \(\mathbf{R}\).
    Since
    \begin{align*}
        \forall\ x \in \mathbf{R}, f * g(x + 1) & = \int_{[0, 1]} f(y) g(x + 1 - y) \; dy & \text{(by Definition \ref{5.4.2})}              \\
                                                & = \int_{[0, 1]} f(y) g(x - y) \; dy     & (g \in C(\mathbf{R} / \mathbf{Z} ; \mathbf{C})) \\
                                                & = f * g(x),                             & \text{(by Definition \ref{5.4.2})}
    \end{align*}
    we know that \(f * g \in C(\mathbf{R} / \mathbf{Z} ; \mathbf{C})\).
\end{proof}

\begin{proof}{(b)}
    Let \(x \in \mathbf{R}\) and let \(\phi : [x - 1, x] \mapsto [0, 1]\) be the function \(\phi(y) = x - y\).
    Then we have
    \begin{align*}
         & f * g(x)                                                                                                                   \\
         & = \int_{[0, 1]} f(y) g(x - y) \; dy                                           & \text{(by Definition \ref{5.4.2})}         \\
         & = \int_{\big[\phi(x), \phi(x - 1)\big]} f(y) g(x - y) \; dy                                                                \\
         & = -\int_{[x - 1, x]} f\big(\phi(y)\big) g\big(x - \phi(y)\big) \phi'(y) \; dy & \text{(by Exercise 11.10.4 in Analysis I)} \\
         & = \int_{[x - 1, x]} f(x - y) g(y) \; dy                                                                                    \\
         & = \int_{[x - 1, x]} g(y) f(x - y) \; dy.                                                                                   \\
    \end{align*}
    Let \([x]\) be the integer defined in Exercise \ref{ex 5.1.1}.
    Then we have
    \begin{align*}
                 & [x] \leq x < [x] + 1     \\
        \implies & [x] - 1 \leq x - 1 < [x]
    \end{align*}
    and
    \begin{align*}
         & f * g(x)                                                                                                                                             \\
         & = \int_{[x - 1, x]} g(y) f(x - y) \; dy                                                                                                              \\
         & = \int_{\big[x - 1, [x]\big]} g(y) f(x - y) \; dy + \int_{\big[[x], x\big]} g(y) f(x - y) \; dy                                                      \\
         & = \int_{\big[[x], x\big]} g(y) f(x - y) \; dy + \int_{\big[x - 1, [x]\big]} g(y) f(x - y) \; dy                                                      \\
         & = \int_{\big[[x], x\big]} g(y) f(x - y) \; dy                                                                                                        \\
         & \quad + \int_{\big[x - 1 + 1, [x] + 1\big]} g(y - 1) f(x - y - 1) \; dy                                                                              \\
         & = \int_{\big[[x], x\big]} g(y) f(x - y) \; dy + \int_{\big[x, [x] + 1\big]} g(y) f(x - y) \; dy & (f, g \in C(\mathbf{R} / \mathbf{Z} ; \mathbf{C})) \\
         & = \int_{\big[[x], [x] + 1\big]} g(y) f(x - y) \; dy                                                                                                  \\
         & = \int_{\big[[x] - [x], [x] + 1 - [x]\big]} g(y + [x]) f(x - y + [x]) \; dy                                                                          \\
         & = \int_{[0, 1]} g(y) f(x - y) \; dy                                                             & (f, g \in C(\mathbf{R} / \mathbf{Z} ; \mathbf{C})) \\
         & = g * f(x).                                                                                     & \text{(by Definition \ref{5.4.2})}
    \end{align*}
    Since \(x\) is arbitrary, we conclude that \(f * g = g * f\).
\end{proof}

\begin{proof}{(c)}
    By Lemma \ref{5.1.5}(b) we know that \(f + g, g + h, cf, cg \in C(\mathbf{R} / \mathbf{Z} ; \mathbf{C})\).
    Thus \(f * (g + h), (f + g) * h, (cf) * g, f * (cg)\) are well-defined.
    Let \(x \in \mathbf{R}\).
    Then we have
    \begin{align*}
         & \big(f * (g + h)\big)(x)                                                                                                \\
         & = \int_{[0, 1]} f(y) \cdot (g + h)(x - y) \; dy                         & \text{(by Definition \ref{5.4.2})}            \\
         & = \int_{[0, 1]} f(y) \cdot \big(g(x - y) + h(x - y)\big) \; dy                                                          \\
         & = \int_{[0, 1]} f(y) g(x - y) + f(y) h(x - y) \; dy                                                                     \\
         & = \int_{[0, 1]} f(y) g(x - y) \; dy + \int_{[0, 1]} f(y) h(x - y) \; dy & \text{(cf the proof of Lemma \ref{5.2.5}(c))} \\
         & = (f * g)(x) + (f * h)(x)                                               & \text{(by Definition \ref{5.4.2})}            \\
         & = (f * g + f * h)(x)
    \end{align*}
    and
    \begin{align*}
        \big((cf) * g\big)(x) & = \int_{[0, 1]} (cf)(y) \cdot g(x - y) \; dy & \text{(by Definition \ref{5.4.2})}            \\
                              & = \int_{[0, 1]} c f(y) g(x - y) \; dy                                                        \\
                              & = c \int_{[0, 1]} f(y) g(x - y) \; dy        & \text{(cf the proof of Lemma \ref{5.2.5}(c))} \\
                              & = c (f * g)(x).                              & \text{(by Definition \ref{5.4.2})}
    \end{align*}
    Since \(x\) is arbitrary, we conclude that \(f * (g + h) = f * g + f * h\) and \((cf) * g = c (f * g)\).
    This implies
    \begin{align*}
        (f + g) * h & = h * (f + g)   & \text{(by Lemma \ref{5.4.4}(b))} \\
                    & = h * f + h * g & \text{(from the proof above)}    \\
                    & = f * h + g * h & \text{(by Lemma \ref{5.4.4}(b))}
    \end{align*}
    and
    \begin{align*}
        f * (cg) & = (cg) * f  & \text{(by Lemma \ref{5.4.4}(b))} \\
                 & = c(g * f)  & \text{(from the proof above)}    \\
                 & = c(f * g). & \text{(by Lemma \ref{5.4.4}(b))}
    \end{align*}
\end{proof}

\begin{additional corollary}\label{ac 5.4.1}
Now we observe an interesting identity:
for any \(f \in C(\mathbf{R} / \mathbf{Z} ; \mathbf{C})\) and any integer \(n\), we have
\[
    f * e_n = \hat{f}(n) e_n.
\]
To prove this, we compute
\begin{align*}
    f * e_n(x) & = \int_{[0, 1]} f(y) e_n(x - y) \; dy                        & \text{(by Definition \ref{5.4.2})}            \\
               & = \int_{[0, 1]} f(y) e^{2 \pi i n (x - y)} \; dy             & \text{(by Definition \ref{5.3.1})}            \\
               & = e^{2 \pi i n x} \int_{[0, 1]} f(y) e^{- 2 \pi i n y} \; dy & \text{(cf the proof of Lemma \ref{5.2.5}(c))} \\
               & = \inner*{f, e_n} e^{2 \pi i n x}                            & \text{(by Definition \ref{5.2.1})}            \\
               & = \hat{f}(n) e^{2 \pi i n x}                                 & \text{(by Definition \ref{5.3.7})}            \\
               & = \hat{f}(n) e_n                                             & \text{(by Definition \ref{5.3.1})}
\end{align*}
as desired.
More generally, we see from Lemma \ref{5.4.4}(c) that for any trigonometric polynomial \(P = \sum_{n = -N}^N c_n e_n\), we have
\[
    f * P = \sum_{n = -N}^N c_n (f * e_n) = \sum_{n = -N}^N \hat{f}(n) c_n e_n.
\]
Thus the periodic convolution of any function in \(C(\mathbf{R} / \mathbf{Z} ; \mathbf{C})\) with a trigonometric polynomial, is again a trigonometric polynomial.
(Compare with Lemma \ref{3.8.13}.)
\end{additional corollary}

\begin{definition}[Periodic approximation to the identity]\label{5.4.5}
    Let \(\varepsilon > 0\) and \(0 < \delta < 1 / 2\).
    A function \(f \in C(\mathbf{R} / \mathbf{Z} ; \mathbf{C})\) is said to be a \emph{periodic \((\varepsilon, \delta)\) approximation to the identity} if the following properties are true:
    \begin{enumerate}
        \item \(f(x) \geq 0\) for all \(x \in \mathbf{R}\), and \(\int_{[0, 1]} f(x) \; dx = 1\).
        \item We have \(f(x) < \varepsilon\) for all \(\delta \leq \abs*{x} \leq 1 - \delta\).
    \end{enumerate}
\end{definition}

\begin{additional corollary}[Fejér kernel]\label{ac 5.4.2}
Let \(N \geq 1\) be an integer.
Then we have
\[
    \sum_{n = -N}^N \bigg(1 - \frac{\abs*{n}}{N}\bigg) e_n = \frac{1}{N} \abs*{\sum_{n = 0}^{N - 1} e_n}^2.
\]
\end{additional corollary}

\begin{proof}
    First we claim that
    \[
        \sum_{n = 0}^{2N - 2} (N - \abs*{n - N + 1}) \cdot e_n = \bigg(\sum_{n = 0}^{N - 1} e_n\bigg)^2.
    \]
    We proof the claim by induction on \(N\) and we start with \(N = 1\).
    For \(N = 1\), we have
    \begin{align*}
        \sum_{n = 0}^0 (1 - \abs*{n - 1 + 1}) \cdot e_n & = \sum_{n = 0}^0 (1 - \abs*{n}) e_n                                      \\
                                                        & = 1 e_0                                                                  \\
                                                        & = 1                                 & \text{(by Theorem \ref{4.5.2}(e))} \\
                                                        & = e_0^2                             & \text{(by Theorem \ref{4.5.2}(e))} \\
                                                        & = \bigg(\sum_{n = 0}^0 e_n\bigg)^2.
    \end{align*}
    Thus the base case holds.
    Suppose inductively that the claim is true for some \(N \geq 1\).
    Then for \(N + 1\), we want to show that
    \[
        \sum_{n = 0}^{2(N + 1) - 2} \big(N + 1 - \abs*{n - (N + 1) + 1}\big) \cdot e_n = \sum_{n = 0}^{2N} (N + 1 - \abs*{n - N}) \cdot e_n = \bigg(\sum_{n = 0}^N e_n\bigg)^2.
    \]
    This is true since
    \begin{align*}
         & \bigg(\sum_{n = 0}^N e_n\bigg)^2                                                                                                                                  \\
         & = \Bigg(\bigg(\sum_{n = 0}^{N - 1} e_n\bigg) + e_N\Bigg)^2                                                                     & \text{(by Lemma \ref{4.6.4})}    \\
         & = \bigg(\sum_{n = 0}^{N - 1} e_n\bigg)^2 + 2 e_N \bigg(\sum_{n = 0}^{N - 1} e_n\bigg) + e_N^2                                  & \text{(by Lemma \ref{4.6.6})}    \\
         & = \sum_{n = 0}^{2N - 2} (N - \abs*{n - N + 1}) \cdot e_n + 2 e_N \bigg(\sum_{n = 0}^{N - 1} e_n\bigg) + e_N^2                  & \text{(by induction hypothesis)} \\
         & = \sum_{n = 0}^{N - 1} (N - \abs*{n - N + 1}) \cdot e_n + \sum_{n = N}^{2N - 2} (N - \abs*{n - N + 1}) \cdot e_n               & \text{(by Lemma \ref{4.6.4})}    \\
         & \quad + 2 e_N \bigg(\sum_{n = 0}^{N - 1} e_n\bigg) + e_N^2                                                                                                        \\
         & = \sum_{n = 0}^{N - 1} (n + 1) \cdot e_n + \sum_{n = N}^{2N - 2} (2N - n - 1) \cdot e_n                                                                           \\
         & \quad + 2 e_N \bigg(\sum_{n = 0}^{N - 1} e_n\bigg) + e_N^2                                                                                                        \\
         & = \sum_{n = 0}^{N - 1} n e_n + \sum_{n = 0}^{N - 1} e_n + \sum_{n = N}^{2N - 2} (2N - n) \cdot e_n - \sum_{n = N}^{2N - 2} e_n & \text{(by Lemma \ref{4.6.6})}    \\
         & \quad + 2 \bigg(\sum_{n = 0}^{N - 1} e_N e_n\bigg) + e_N^2                                                                     & \text{(by Lemma \ref{4.6.6})}    \\
         & = \sum_{n = 0}^{N - 1} n e_n + \sum_{n = N}^{2N - 2} (2N - n) \cdot e_n                                                        & \text{(by Lemma \ref{4.6.4})}    \\
         & \quad + \sum_{n = 0}^{N - 1} e_n - \sum_{n = N}^{2N - 2} e_n + 2 \bigg(\sum_{n = 0}^{N - 1} e_N e_n\bigg) + e_N^2                                                 \\
         & = \sum_{n = 0}^{N - 1} (N - \abs*{n - N}) \cdot e_n + \sum_{n = N}^{2N - 2} (N - \abs*{n - N}) \cdot e_n                                                          \\
         & \quad + \sum_{n = 0}^{N - 1} e_n - \sum_{n = N}^{2N - 2} e_n + 2 \bigg(\sum_{n = 0}^{N - 1} e_N e_n\bigg) + e_N^2
    \end{align*}
    \begin{align*}
         & = \sum_{n = 0}^{2N - 2} (N - \abs*{n - N}) \cdot e_n                                                              & \text{(conti. from above)}           \\
         & \quad + \sum_{n = 0}^{N - 1} e_n - \sum_{n = N}^{2N - 2} e_n + 2 \bigg(\sum_{n = 0}^{N - 1} e_N e_n\bigg) + e_N^2                                        \\
         & = \sum_{n = 0}^{2N - 2} (N + 1 - \abs*{n - N}) \cdot e_n - \sum_{n = 0}^{2N - 2} e_n                              & \text{(by Lemma \ref{4.6.6})}        \\
         & \quad + \sum_{n = 0}^{N - 1} e_n - \sum_{n = N}^{2N - 2} e_n + 2 \bigg(\sum_{n = 0}^{N - 1} e_N e_n\bigg) + e_N^2                                        \\
         & = \sum_{n = 0}^{2N - 2} (N + 1 - \abs*{n - N}) \cdot e_n                                                          & \text{(by Lemma \ref{4.6.6})}        \\
         & \quad - 2 \bigg(\sum_{n = N}^{2N - 2} e_n\bigg) + 2 \bigg(\sum_{n = 0}^{N - 1} e_N e_n\bigg) + e_N^2                                                     \\
         & = \sum_{n = 0}^{2N} (N + 1 - \abs*{n - N}) \cdot e_n - 2 e_{2N - 1} - e_{2N}                                      & \text{(by Lemma \ref{4.6.6})}        \\
         & \quad - 2 \bigg(\sum_{n = N}^{2N - 2} e_n\bigg) + 2 \bigg(\sum_{n = 0}^{N - 1} e_N e_n\bigg) + e_N^2                                                     \\
         & = \sum_{n = 0}^{2N} (N + 1 - \abs*{n - N}) \cdot e_n - 2 e_{2N - 1} - e_{2N}                                                                             \\
         & \quad - 2 \bigg(\sum_{n = N}^{2N - 2} e_n\bigg) + 2 \bigg(\sum_{n = 0}^{N - 1} e_{n + N}\bigg) + e_{2N}           & \text{(by Exercise \ref{ex 4.6.16})} \\
         & = \sum_{n = 0}^{2N} (N + 1 - \abs*{n - N}) \cdot e_n                                                              & \text{(by Lemma \ref{4.6.4})}        \\
         & \quad - 2 \bigg(\sum_{n = N}^{2N - 1} e_n\bigg) + 2 \bigg(\sum_{n = 0}^{N - 1} e_{n + N}\bigg)                                                           \\
         & = \sum_{n = 0}^{2N} (N + 1 - \abs*{n - N}) \cdot e_n                                                                                                     \\
         & \quad - 2 \bigg(\sum_{n = N}^{2N - 1} e_n\bigg) + 2 \bigg(\sum_{n = N}^{2N - 1} e_n\bigg)                                                                \\
         & = \sum_{n = 0}^{2N} (N + 1 - \abs*{n - N}) \cdot e_n.
    \end{align*}
    This closes the induction.

    Using the claim above we have
    \begin{align*}
         & \sum_{n = -N}^N \bigg(1 - \frac{\abs*{n}}{N}\bigg) e_n                                                                                    \\
         & = \frac{1}{N} \sum_{n = -N}^N (N - \abs*{n}) e_n                                                   & \text{(by Lemma \ref{4.6.6})}        \\
         & = \frac{1}{N} \sum_{n = -(N - 1)}^{N - 1} (N - \abs*{n}) \cdot e_n                                                                        \\
         & = \frac{1}{N} \sum_{n = 0}^{2N - 2} (N - \abs*{n - N + 1}) \cdot e_{n - N + 1}                                                            \\
         & = \frac{1}{N} \sum_{n = 0}^{2N - 2} (N - \abs*{n - N + 1}) \cdot e_n \cdot e_{-N + 1}              & \text{(by Exercise \ref{ex 4.6.16})} \\
         & = \frac{e_{-N + 1}}{N} \sum_{n = 0}^{2N - 2} (N - \abs*{n - N + 1}) \cdot e_n                      & \text{(by Lemma \ref{4.6.6})}        \\
         & = \frac{e_{-N + 1}}{N} \bigg(\sum_{n = 0}^{N - 1} e_n\bigg)^2                                      & \text{(from the claim above)}        \\
         & = \frac{e_{-N + 1}}{N} \bigg(\sum_{n = 0}^{N - 1} e_n\bigg) \bigg(\sum_{n = 0}^{N - 1} e_n\bigg)                                          \\
         & = \frac{1}{N} \bigg(\sum_{n = 0}^{N - 1} e_n\bigg) \bigg(\sum_{n = 0}^{N - 1} e_{-N + 1} e_n\bigg) & \text{(by Lemma \ref{4.6.6})}        \\
         & = \frac{1}{N} \bigg(\sum_{n = 0}^{N - 1} e_n\bigg) \bigg(\sum_{n = 0}^{N - 1} e_{n - N + 1}\bigg)  & \text{(by Exercise \ref{ex 4.6.16})} \\
         & = \frac{1}{N} \bigg(\sum_{n = 0}^{N - 1} e_n\bigg) \bigg(\sum_{n = 0}^{N - 1} e_{-n}\bigg)         & \text{(by Lemma \ref{4.6.4})}        \\
         & = \frac{1}{N} \bigg(\sum_{n = 0}^{N - 1} e_n\bigg) \bigg(\sum_{n = 0}^{N - 1} \overline{e_n}\bigg) & \text{(by Definition \ref{4.6.15})}  \\
         & = \frac{1}{N} \bigg(\sum_{n = 0}^{N - 1} e_n\bigg) \bigg(\overline{\sum_{n = 0}^{N - 1} e_n}\bigg) & \text{(by Lemma \ref{4.6.9})}        \\
         & = \frac{1}{N} \abs*{\sum_{n = 0}^{N - 1} e_n}^2.                                                   & \text{(by Lemma \ref{4.6.11})}
    \end{align*}
\end{proof}

\begin{lemma}\label{5.4.6}
    For every \(\varepsilon > 0\) and \(0 < \delta < 1 / 2\), there exists a trigonometric polynomial \(P\) which is an \((\varepsilon, \delta)\) approximation to the identity.
\end{lemma}

\begin{proof}
    Let \(N \geq 1\) be an integer.
    We define the \emph{Fejér kernel} \(F_N\) to be the function
    \[
        F_N = \sum_{n = -N}^N \bigg(1 - \frac{\abs*{n}}{N}\bigg) e_n.
    \]
    Clearly \(F_N\) is a trigonometric polynomial.
    We observe the identity
    \[
        F_N = \frac{1}{N} \abs*{\sum_{n = 0}^{N - 1} e_n}^2
    \]
    by Additional Corollary \ref{ac 5.4.2}.
    But from the geometric series formula (Lemma 7.3.3 in Analysis I) we have
    \begin{align*}
        \sum_{n = 0}^{N - 1} e_n(x) & = \sum_{n = 0}^{N - 1} \big(e_1(x)\big)^n                                                                               & \text{(by Exercise \ref{ex 4.6.16})} \\
                                    & = \frac{\big(e_1(x)\big)^N - 1}{e_1(x) - 1}                                                                             & \text{(geometric series)}            \\
                                    & = \frac{\big(e_1(x)\big)^N - e_0(x)}{e_1(x) - e_0(x)}                                                                   & \text{(by Theorem \ref{4.5.2}(e))}   \\
                                    & = \frac{e_N(x) - e_0(x)}{e_1(x) - e_0(x)}                                                                               & \text{(by Exercise \ref{ex 4.6.16})} \\
                                    & = \frac{e^{2 \pi i N x} - e^0}{e^{2 \pi i x} - e^0}                                                                     & \text{(by Definition \ref{5.3.1})}   \\
                                    & = \frac{e^{\pi i N x} e^{\pi i N x} - e^{\pi i N x} e^{-\pi i N x}}{e^{\pi i x} e^{\pi i x} - e^{\pi i x} e^{-\pi i x}} & \text{(by Exercise \ref{ex 4.6.16})} \\
                                    & = \frac{e^{\pi i N x} (e^{\pi i N x} - e^{-\pi i N x})}{e^{\pi i x} (e^{\pi i x} - e^{-\pi i x})}                       & \text{(by Lemma \ref{4.6.6})}        \\
                                    & = \frac{e^{\pi i (N - 1) x} (e^{\pi i N x} - e^{-\pi i N x})}{e^{\pi i x} - e^{-\pi i x}}                               & \text{(by Definition \ref{4.6.12})}  \\
                                    & = \frac{2i e^{\pi i (N - 1) x} \sin(\pi N x)}{2i \sin(\pi x)}                                                           & \text{(by Definition \ref{4.7.1})}   \\
                                    & = \frac{e^{\pi i (N - 1) x} \sin(\pi N x)}{\sin(\pi x)}                                                                 & \text{(by Definition \ref{4.6.12})}
    \end{align*}
    when \(x\) is not an integer, and hence we have the formula
    \begin{align*}
        F_N(x) & = \frac{1}{N} \abs*{\sum_{n = 0}^{N - 1} e_n(x)}^2                                                      & \text{(by Additional Corollary \ref{ac 5.4.2})} \\
               & = \frac{1}{N} \abs*{\frac{e^{\pi i (N - 1) x} \sin(\pi N x)}{\sin(\pi x)}}^2                            & \text{(from the proof above)}                   \\
               & = \frac{\abs*{e^{\pi i (N - 1) x}}^2 \abs*{\sin(\pi N x)}^2}{N \abs*{\sin(\pi x)}^2}                    & \text{(by Exercise \ref{ex 4.6.7})}             \\
               & = \frac{\abs*{e^{\pi i (N - 1) x}}^2 \big(\sin(\pi N x)\big)^2}{N \big(\sin(\pi x)\big)^2}              & (x \in \mathbf{R})                              \\
               & = \frac{e^{\pi i (N - 1) x} e^{- \pi i (N - 1) x} \big(\sin(\pi N x)\big)^2}{N \big(\sin(\pi x)\big)^2} & \text{(by Lemma \ref{4.6.11})}                  \\
               & = \frac{e^0 \big(\sin(\pi N x)\big)^2}{N \big(\sin(\pi x)\big)^2}                                       & \text{(by Exercise \ref{ex 4.6.16})}            \\
               & = \frac{\big(\sin(\pi N x)\big)^2}{N \big(\sin(\pi x)\big)^2}.                                          & \text{(by Theorem \ref{4.5.2}(e))}
    \end{align*}
    When \(x\) is an integer, the geometric series formula does not apply, but one has \(F_N(x) = N\) in that case, as one can see by direct computation.
    In either case we see that \(F_N(x) \geq 0\) for any \(x\).
    Also, we have
    \begin{align*}
         & \int_{[0, 1]} F_N(x) \; dx                                                                                                                                    \\
         & = \int_{[0, 1]} \sum_{n = -N}^N \bigg(1 - \frac{\abs*{n}}{N}\bigg) e_n(x) \; dx                                                                               \\
         & = \sum_{n = -N}^N \Bigg(\bigg(1 - \frac{\abs*{n}}{N}\bigg) \int_{[0, 1]} e_n(x) \; dx\Bigg)                            & \text{(by Remark \ref{5.2.2})}       \\
         & = \sum_{n = -N}^N \Bigg(\bigg(1 - \frac{\abs*{n}}{N}\bigg) \int_{[0, 1]} e_{n - 1}(x) e_1(x) \; dx\Bigg)               & \text{(by Exercise \ref{ex 4.6.16})} \\
         & = \sum_{n = -N}^N \Bigg(\bigg(1 - \frac{\abs*{n}}{N}\bigg) \int_{[0, 1]} e_{n - 1}(x) \overline{e_{-1}(x)} \; dx\Bigg) & \text{(by Definition \ref{4.6.15})}  \\
         & = \sum_{n = -N}^N \Bigg(\bigg(1 - \frac{\abs*{n}}{N}\bigg) \inner*{e_{n - 1}, e_{-1}}\Bigg)                            & \text{(by Definition \ref{5.2.1})}   \\
         & = \bigg(1 - \frac{\abs*{0}}{N}\bigg) 1                                                                                 & \text{(by Lemma \ref{5.3.5})}        \\
         & = 1.
    \end{align*}
    Finally, since \(\sin(\pi N x) \leq 1\), we have
    \[
        F_N(x) \leq \frac{1}{N \big(\sin(\pi x)\big)^2} \leq \frac{1}{N \big(\sin(\pi \delta)\big)^2}
    \]
    whenever \(\delta < \abs*{x} < 1 - \delta\)
    (this is because \(\sin\) is increasing on \([0, \pi / 2]\) and decreasing on \([\pi / 2, \pi]\)).
    Thus by choosing \(N\) large enough, we can make \(F_N (x) \leq \varepsilon\) for all \(\delta < \abs*{x} < 1 - \delta\).
    Note that since
    \begin{align*}
        \big(\sin(\pi \abs*{x})\big)^2 & = \begin{cases}
            \big(\sin(\pi x)\big)^2  & \text{if } x \geq 0 \\
            \big(\sin(\pi -x)\big)^2 & \text{if } x < 0
        \end{cases}                                      \\
                                       & = \begin{cases}
            \big(\sin(\pi x)\big)^2  & \text{if } x \geq 0 \\
            \big(-\sin(\pi x)\big)^2 & \text{if } x < 0
        \end{cases} & \text{(by Theorem \ref{4.7.2}(c))} \\
                                       & = \big(\sin(\pi x)\big)^2
    \end{align*}
    and
    \begin{align*}
                 & \begin{cases}
            \delta < \abs*{x} \leq \frac{1}{2}  & \text{if } \abs*{x} \leq \frac{1}{2} \\
            \frac{1}{2} < \abs*{x} < 1 - \delta & \text{if } \abs*{x} > \frac{1}{2}    \\
        \end{cases}                                                  \\
        \implies & \begin{cases}
            \pi \delta < \pi \abs*{x} \leq \frac{\pi}{2}    & \text{if } \abs*{x} \leq \frac{1}{2} \\
            \frac{\pi}{2} < \pi \abs*{x} < \pi - \pi \delta & \text{if } \abs*{x} > \frac{1}{2}    \\
        \end{cases}                                                  \\
        \implies & \begin{cases}
            \sin(\pi \delta) < \sin(\pi \abs*{x}) \leq \sin(\frac{\pi}{2})    & \text{if } \abs*{x} \leq \frac{1}{2} \\
            \sin(\frac{\pi}{2}) > \sin(\pi \abs*{x}) > \sin(\pi - \pi \delta) & \text{if } \abs*{x} > \frac{1}{2}    \\
        \end{cases}                                                  \\
        \implies & \begin{cases}
            \sin(\pi \delta) < \sin(\pi \abs*{x}) \leq \sin(\frac{\pi}{2}) & \text{if } \abs*{x} \leq \frac{1}{2} \\
            \sin(\frac{\pi}{2}) > \sin(\pi \abs*{x}) > -\sin(-\pi \delta)  & \text{if } \abs*{x} > \frac{1}{2}    \\
        \end{cases}             & \text{(by Theorem \ref{4.7.5}(a))} \\
        \implies & \begin{cases}
            \sin(\pi \delta) < \sin(\pi \abs*{x}) \leq \sin(\frac{\pi}{2}) & \text{if } \abs*{x} \leq \frac{1}{2} \\
            \sin(\frac{\pi}{2}) > \sin(\pi \abs*{x}) > \sin(\pi \delta)    & \text{if } \abs*{x} > \frac{1}{2}    \\
        \end{cases}             & \text{(by Theorem \ref{4.7.2}(c))} \\
        \implies & \sin(\pi \delta) < \sin(\pi \abs*{x}),
    \end{align*}
    we have
    \begin{align*}
                 & 0 < \big(\sin(\pi \delta)\big)^2 < \big(\sin(\pi \abs*{x})\big)^2               & \text{(by Additional Corollary \ref{ac 4.7.2}(d))} \\
        \implies & 0 < \big(\sin(\pi \delta)\big)^2 < \big(\sin(\pi x)\big)^2                      & \text{(from the proof above)}                      \\
        \implies & 0 < \frac{1}{\big(\sin(\pi x)\big)^2} < \frac{1}{\big(\sin(\pi \delta)\big)^2}.
    \end{align*}
\end{proof}

\exercisesection

\begin{exercise}\label{ex 5.4.1}
    Show that if \(f : \mathbf{R} \to \mathbf{C}\) is both compactly supported and \(\mathbf{Z}\)-periodic, then it is identically zero.
\end{exercise}

\begin{exercise}\label{ex 5.4.2}
    Prove Lemma \ref{5.4.4}.
\end{exercise}

\begin{proof}
    See Lemma \ref{5.4.4}.
\end{proof}

\begin{exercise}\label{ex 5.4.3}
    Fill in the gaps marked in Lemma \ref{5.4.6}.
\end{exercise}

\begin{proof}
    See Lemma \ref{5.4.6}.
\end{proof}
\section{The Fourier and Plancherel theorems}\label{sec 5.5}

\begin{theorem}[Fourier theorem]\label{5.5.1}
    For any \(f \in C(\mathbf{R} / \mathbf{Z} ; \mathbf{C})\), the series \(\sum_{n = -\infty}^\infty \hat{f}(n) e_n\) converges in \(L^2\) metric to \(f\).
    In other words, we have
    \[
        \lim_{N \to \infty} \norm*{f - \sum_{n = -N}^N \hat{f}(n) e_n}_2 = 0.
    \]
\end{theorem}

\begin{proof}
    Let \(\varepsilon > 0\).
    We have to show that there exists an \(N_0\) such that
    \[
        \norm*{f - \sum_{n = -N}^N \hat{f}(n) e_n}_2 \leq \varepsilon
    \]
    for all \(N \geq N_0\).

    By the Weierstrass approximation theorem (Theorem \ref{5.4.1}), we can find a trigonometric polynomial \(P = \sum_{n = -N_0}^{N_0} c_n e_n\) such that \(\norm*{f - P}_{\infty} \leq \varepsilon\), for some \(N_0 > 0\).
    In particular, we have \(\norm*{f - P}_2 \leq \varepsilon\) (Exercise \ref{ex 5.2.3}).

    Now let \(N > N_0\), and let \(F_N \coloneqq \sum_{n = -N}^N \hat{f}(n) e_n\).
    We claim that \(\norm*{f - F_N}_2 \leq \varepsilon\).
    First observe that for any \(\abs*{m} \leq N\), we have
    \[
        \inner*{f - F_N, e_m} = \inner*{f, e_m} - \sum_{n = -N}^N \hat{f}(n) \inner*{e_n, e_m} = \hat{f}(m) - \hat{f}(m) = 0,
    \]
    where we have used Lemma \ref{5.3.5} and Lemma \ref{5.2.5}.
    In particular we have
    \[
        \inner*{f - F_N, F_N - P} = 0
    \]
    since we can write \(F_N - P\) as a linear combination of the \(e_m\) for which \(\abs*{m} \leq N\).
    By Pythagoras' theorem (Lemma \ref{5.2.7}(d)) we therefore have
    \[
        \norm*{f - P}_2^2 = \norm*{f - F_N}_2^2 + \norm*{F_N - P}_2^2
    \]
    and in particular
    \[
        \norm*{f - F_N}_2 \leq \norm*{f - P}_2 \leq \varepsilon
    \]
    as desired.
\end{proof}

\begin{remark}\label{5.5.2}
    Note that we have only obtained convergence of the Fourier series \(\sum_{n = -\infty}^\infty \hat{f}(n) e_n\) to \(f\) in the \(L^2\) metric.
    One may ask whether one has convergence in the uniform or pointwise sense as well, but it turns out (perhaps somewhat surprisingly) that the answer is no to both of those questions.
    However, if one assumes that the function \(f\) is not only continuous, but is also differentiable, then one can recover pointwise convergence;
    if one assumes continuously differentiable, then one gets uniform convergence as well.
    These results are beyond the scope of this text and will not be proven here.
    However, we will prove one theorem about when one can improve the \(L^2\) convergence to uniform convergence.
\end{remark}

\begin{theorem}\label{5.5.3}
    Let \(f \in C(\mathbf{R} / \mathbf{Z} ; \mathbf{C})\), and suppose that the series \(\sum_{n = -\infty}^\infty \abs*{\hat{f}(n)}\) is absolutely convergent.
    Then the series \(\sum_{n = -\infty}^\infty \hat{f}(n) e_n\) converges uniformly to \(f\).
    In other words, we have
    \[
        \lim_{N \to \infty} \norm*{f - \sum_{n = -N}^N \hat{f}(n) e_n}_{\infty} = 0.
    \]
\end{theorem}

\begin{proof}
    By the Weierstrass \(M\)-test (Theorem \ref{3.5.7}), we see that \(\sum_{n = -\infty}^\infty \hat{f}(n) e_n\) converges to some function \(F\), which by Lemma \ref{5.1.5}(c) is also continuous and \(\mathbf{Z}\)-periodic.
    (Strictly speaking, the Weierstrass \(M\)-test was phrased for series from \(n = 1\) to \(n = +\infty\), but also works for series from \(n = -\infty\) to \(n = +\infty\);
    this can be seen by splitting the doubly infinite series into two pieces.)
    Thus
    \[
        \lim_{N \to \infty} \norm*{F - \sum_{n = -N}^N \hat{f}(n) e_n}_{\infty} = 0
    \]
    which implies that
    \[
        \lim_{N \to \infty} \norm*{F - \sum_{n = -N}^N \hat{f}(n) e_n}_2 = 0
    \]
    since the \(L^2\) norm is always less than or equal to the \(L^\infty\) norm (Exercise \ref{ex 5.2.3}).
    But the sequence \(\sum_{n = -N}^N \hat{f}(n) e_n\) is already converging in \(L^2\) metric to \(f\) by the Fourier theorem (Theorem \ref{5.5.1}), so can only converge in \(L^2\) metric to \(F\) if \(F = f\)
    (cf. Proposition \ref{1.1.20}).
    Thus \(F = f\), and so we have
    \[
        \lim_{N \to \infty} \norm*{f - \sum_{n = -N}^N \hat{f}(n) e_n}_{\infty} = 0
    \]
    as desired.
\end{proof}

\begin{theorem}[Plancherel theorem]\label{5.5.4}
    For any \(f \in C(\mathbf{R} / \mathbf{Z} ; \mathbf{C})\), the series
    \[
        \sum_{n = -\infty}^\infty \abs*{\hat{f}(n)}^2
    \]
    is absolutely convergent, and
    \[
        \norm*{f}_2^2 = \sum_{n = -\infty}^\infty \abs*{\hat{f}(n)}^2.
    \]
\end{theorem}

\begin{proof}
    Let \(\varepsilon > 0\).
    By the Fourier theorem (Theorem \ref{5.5.1}) we know that
    \[
        \norm*{f - \sum_{n = -N}^N \hat{f}(n) e_n}_2 \leq \varepsilon
    \]
    if \(N\) is large enough (depending on \(\varepsilon\)).
    In particular, by the triangle inequality (Lemma \ref{5.2.7}(c)(e)) this implies that
    \[
        \norm*{f}_2 - \varepsilon \leq \norm*{\sum_{n = -N}^N \hat{f}(n) e_n}_2 \leq \norm*{f}_2 + \varepsilon.
    \]
    On the other hand, by Corollary \ref{5.3.6} we have
    \[
        \norm*{\sum_{n = -N}^N \hat{f}(n) e_n}_2 = \bigg(\sum_{n = -N}^N \abs*{\hat{f}(n)}^2\bigg)^{1 / 2}
    \]
    and hence
    \[
        (\norm*{f}_2 - \varepsilon)^2 \leq \sum_{n = -N}^N \abs*{\hat{f}(n)}^2 \leq (\norm*{f}_2 + \varepsilon)^2.
    \]
    Taking \(\limsup\), we obtain
    \[
        (\norm*{f}_2 - \varepsilon)^2 \leq \limsup_{N \to \infty} \sum_{n = -N}^N \abs*{\hat{f}(n)}^2 \leq (\norm*{f}_2 + \varepsilon)^2.
    \]
    Since \(\varepsilon\) is arbitrary, we thus obtain by the squeeze test that
    \[
        \limsup_{N \to \infty} \sum_{n = -N}^N \abs*{\hat{f}(n)}^2 = \norm*{f}_2^2
    \]
    and the claim follows.
\end{proof}

\begin{note}
    Theorem \ref{5.5.4} is also known as \emph{Parseval's theorem}.
\end{note}

\exercisesection

\begin{exercise}\label{ex 5.5.1}
    Let \(f\) be a function in \(C(\mathbf{R} / \mathbf{Z} ; \mathbf{C})\), and define the \emph{trigonometric Fourier coefficients} \(a_n, b_n\) for \(n = 0, 1, 2, 3, \dots\) by
    \[
        a_n = 2 \int_{[0, 1]} f(x) \cos(2 \pi n x) \; dx; \quad b_n = 2 \int_{[0, 1]} f(x) \sin(2 \pi n x) \; dx.
    \]
    \begin{enumerate}
        \item Show that the series
              \[
                  \frac{1}{2} a_0 + \sum_{n = 1}^\infty \big(a_n \cos(2 \pi n x) + b_n \sin(2 \pi n x)\big)
              \]
              converges in \(L_2\) metric to \(f\).
        \item Show that if \(\sum_{n = 1}^\infty a_n\) and \(\sum_{n = 1}^\infty b_n\) are absolutely convergent, then the above series actually converges uniformly to \(f\), and not just in \(L_2\) metric.
    \end{enumerate}
\end{exercise}

\begin{proof}{(a)}
    Observe that for all \(n \in \mathbf{Z}\), we have
    \begin{align*}
         & \hat{f}(n)                                                                                                                                              \\
         & = \int_{[0, 1]} f(x) e^{- 2 \pi i n x} \; dx                                                                       & \text{(by Definition \ref{5.3.7})} \\
         & = \int_{[0, 1]} f(x) \big(\cos(2 \pi n x) - i \sin(2 \pi n x)\big) \; dx                                           & \text{(by Theorem \ref{4.7.2}(f))} \\
         & = \int_{[0, 1]} f(x) \cos(2 \pi n x) \; dx - i \int_{[0, 1]} f(x) \sin(2 \pi n x) \; dx                            & \text{(by Remark \ref{5.2.2})}     \\
         & = \frac{1}{2} \bigg(2 \int_{[0, 1]} f(x) \cos(2 \pi n x) \; dx - 2i \int_{[0, 1]} f(x) \sin(2 \pi n x) \; dx\bigg)                                      \\
         & = \frac{1}{2} (a_n - i b_n)
    \end{align*}
    and
    \begin{align*}
         & \hat{f}(-n)                                                                                                                                             \\
         & = \frac{1}{2} (a_{-n} - i b_{-n})                                                                                  & \text{(from the proof above)}      \\
         & = \frac{1}{2} \bigg(2 \int_{[0, 1]} f(x) \cos(- 2 \pi n x) \; dx                                                                                        \\
         & \quad - 2i \int_{[0, 1]} f(x) \sin(- 2 \pi n x) \; dx\bigg)                                                                                             \\
         & = \frac{1}{2} \bigg(2 \int_{[0, 1]} f(x) \cos(2 \pi n x) \; dx                                                                                          \\
         & \quad - 2i \int_{[0, 1]} -f(x) \sin(2 \pi n x) \; dx\bigg)                                                         & \text{(by Theorem \ref{4.7.2}(c))} \\
         & = \frac{1}{2} \bigg(2 \int_{[0, 1]} f(x) \cos(2 \pi n x) \; dx + 2i \int_{[0, 1]} f(x) \sin(2 \pi n x) \; dx\bigg) & \text{(by Remark \ref{5.2.2})}     \\
         & = \frac{1}{2} (a_n + i b_n).
    \end{align*}
    By Fourier theroem (Theorem \ref{5.5.1}) we know that
    \[
        \lim_{N \to \infty} \norm*{f - \sum_{n = -N}^N \hat{f}(n) e_n}_2 = 0.
    \]
    Since for all \(N \in \mathbf{Z}^+\), we have
    \begin{align*}
         & \sum_{n = -N}^N \hat{f}(n) e_n                                                                                                                                  \\
         & = \hat{f}(0) e_0 + \sum_{n = 1}^N \hat{f}(n) e_n + \sum_{n = -N}^{-1} \hat{f}(n) e_n                                                                            \\
         & = \hat{f}(0) e_0 + \sum_{n = 1}^N \hat{f}(n) e_n + \sum_{n = 1}^N \hat{f}(-n) e_{-n}                                                                            \\
         & = \frac{(a_0 - i b_0) e_0}{2} + \sum_{n = 1}^N \frac{(a_n - i b_n) e_n}{2} + \sum_{n = 1}^N \frac{(a_n + i b_n) e_{-n}}{2} & \text{(from the proof above)}      \\
         & = \frac{a_0 e_0}{2} + \sum_{n = 1}^N \frac{(a_n - i b_n) e_n}{2} + \sum_{n = 1}^N \frac{(a_n + i b_n) e_{-n}}{2}           & \text{(by Theorem \ref{4.7.2}(e))} \\
         & = \frac{a_0}{2} + \sum_{n = 1}^N \frac{(a_n - i b_n) e_n}{2} + \sum_{n = 1}^N \frac{(a_n + i b_n) e_{-n}}{2}               & \text{(by Theorem \ref{4.5.2}(e))} \\
         & = \frac{a_0}{2} + \sum_{n = 1}^N \frac{a_n (e_n + e_{-n}) - i b_n (e_n - e_{-n})}{2}                                       & \text{(by Lemma \ref{4.6.6})}      \\
         & = \frac{a_0}{2} + \sum_{n = 1}^N a_n \cos(2 \pi n x) + b_n \sin(2 \pi n x),                                                & \text{(by Definition \ref{4.7.1})}
    \end{align*}
    we know that
    \[
        \lim_{N \to \infty} \norm*{f - \bigg(\frac{a_0}{2} + \sum_{n = 1}^N a_n \cos(2 \pi n x) + b_n \sin(2 \pi n x)\bigg)}_2 = 0.
    \]
    Thus by Definition \ref{1.1.14} we have
    \begin{align*}
         & d_{L^2} - \lim_{N \to \infty} \bigg(\frac{a_0}{2} + \sum_{n = 1}^N a_n \cos(2 \pi n x) + b_n \sin(2 \pi n x)\bigg) \\
         & = \frac{a_0}{2} + \sum_{n = 1}^\infty a_n \cos(2 \pi n x) + b_n \sin(2 \pi n x)                                    \\
         & = f.
    \end{align*}
\end{proof}

\begin{proof}{(b)}
    Observe that
    \begin{align*}
         & \frac{a_0}{2} + \sum_{n = 1}^\infty \abs*{a_n} + \sum_{n = 1}^\infty \abs*{b_n}                                                             & \text{(by hypothesis)}                          \\
         & = \frac{a_0}{2} + \lim_{N \to \infty} \sum_{n = 1}^N \abs*{a_n} + \lim_{N \to \infty} \sum_{n = 1}^N \abs*{b_n}                             & \text{(by Additional Corollary \ref{ac 4.6.6})} \\
         & = \frac{a_0}{2} + \lim_{N \to \infty} \sum_{n = 1}^N \abs*{a_n} + \abs*{b_n}                                                                & \text{(by Lemma \ref{4.6.14})}                  \\
         & = \frac{a_0}{2} + 2 \bigg(\lim_{N \to \infty} \sum_{n = 1}^N \frac{\abs*{a_n} + \abs*{b_n}}{2}\bigg)                                        & \text{(by Lemma \ref{4.6.14})}                  \\
         & = \frac{a_0}{2} + 2 \sum_{n = 1}^\infty \frac{\abs*{a_n} + \abs*{b_n}}{2}                                                                   & \text{(by Additional Corollary \ref{ac 4.6.6})} \\
         & = \frac{a_0}{2} + \sum_{n = 1}^\infty \frac{\abs*{a_n} + \abs*{b_n}}{2} + \sum_{n = 1}^\infty \frac{\abs*{a_n} + \abs*{b_n}}{2}                                                               \\
         & = \frac{a_0}{2} + \sum_{n = 1}^\infty \frac{\abs*{a_n} + \abs*{i b_n}}{2} + \sum_{n = 1}^\infty \frac{\abs*{a_n} + \abs*{-i b_n}}{2}        & \text{(by Lemma \ref{4.6.11})}                  \\
         & \geq \frac{a_0}{2} + \sum_{n = 1}^\infty \frac{\abs*{a_n + i b_n}}{2} + \sum_{n = 1}^\infty \frac{\abs*{a_n - i b_n}}{2}                    & \text{(by Lemma \ref{4.6.11})}                  \\
         & = \frac{a_0}{2} + \sum_{n = 1}^\infty \frac{\abs*{a_{-n} - i b_{-n}}}{2} + \sum_{n = 1}^\infty \frac{\abs*{a_n - i b_n}}{2}                 & \text{(by Theorem \ref{4.7.2}(c))}              \\
         & = \frac{a_0}{2} + \sum_{n = 1}^\infty \frac{\abs*{a_{-n} - i b_{-n}}}{2} + \sum_{n = 1}^\infty \frac{\abs*{a_n - i b_n}}{2}                 & \text{(by Theorem \ref{4.7.2}(e))}              \\
         & = \lim_{N \to \infty} \frac{a_0}{2} + \sum_{n = 1}^N \frac{\abs*{a_{-n} - i b_{-n}}}{2} + \sum_{n = 1}^N \frac{\abs*{a_n - i b_n}}{2}       & \text{(by Lemma \ref{4.6.14})}                  \\
         & = \lim_{N \to \infty} \frac{a_0}{2} + \sum_{n = -N}^{-1} \frac{\abs*{a_n - i b_n}}{2} + \sum_{n = 1}^N \frac{\abs*{a_n - i b_n}}{2}                                                           \\
         & = \lim_{N \to \infty} \frac{a_0 - i b_0}{2} + \sum_{n = -N}^{-1} \frac{\abs*{a_n - i b_n}}{2} + \sum_{n = 1}^N \frac{\abs*{a_n - i b_n}}{2} & \text{(by Theorem \ref{4.7.2}(e))}              \\
         & = \lim_{N \to \infty} \sum_{n = -N}^N \frac{\abs*{a_{-n} - i b_{-n}}}{2}                                                                                                                      \\
         & = \sum_{n = -\infty}^\infty \frac{\abs*{a_{-n} - i b_{-n}}}{2}.
    \end{align*}
    Since
    \[
        \hat{f}(n) = \frac{1}{2} (a_n - i b_n)
    \]
    for all \(n \in \mathbf{Z}\) (cf. the proof of Exercise \ref{ex 5.5.1}(a)), we know that
    \begin{align*}
        \sum_{n = -\infty}^\infty \abs*{\hat{f}(n)} & = \sum_{n = -\infty}^\infty \abs*{\frac{a_n - i b_n}{2}}                                                                    \\
                                                    & = \sum_{n = -\infty}^\infty \frac{\abs*{a_n - i b_n}}{2}                              & \text{(by Exercise \ref{ex 4.6.7})} \\
                                                    & \leq \frac{a_0}{2} + \sum_{n = 1}^\infty \abs*{a_n} + \sum_{n = 1}^\infty \abs*{b_n}. & \text{(from the proof above)}
    \end{align*}
    Thus \(\sum_{n = -\infty}^\infty \abs*{\hat{f}(n)}\) is absolutely convergent.
    By Theorem \ref{5.5.3} we know that
    \[
        \lim_{N \to \infty} \norm*{f - \sum_{n = -N}^N \hat{f}(n) e_n}_{\infty} = 0.
    \]
    and \(\sum_{n = -\infty}^\infty \hat{f}(n) e_n\) converges uniformly to \(f\) on \(\mathbf{R}\) with respect to \(d_{l^1}|_{\mathbf{C} \times \mathbf{C}}\).
    In particular, we have (by Exercise \ref{ex 5.2.3} and squeeze test)
    \[
        \lim_{N \to \infty} \norm*{f - \sum_{n = -N}^N \hat{f}(n) e_n}_2 = 0.
    \]
    By Exercise \ref{ex 5.5.1}(a) we know that
    \[
        \lim_{N \to \infty} \norm*{f - \bigg(\frac{a_0}{2} + \sum_{n = 1}^N a_n \cos(2 \pi n x) + b_n \sin(2 \pi n x)\bigg)}_2 = 0.
    \]
    Thus by Proposition \ref{1.1.20} we have
    \[
        \lim_{N \to \infty} \norm*{f - \bigg(\frac{a_0}{2} + \sum_{n = 1}^N a_n \cos(2 \pi n x) + b_n \sin(2 \pi n x)\bigg)}_{\infty} = 0.
    \]
    and \(\frac{a_0}{2} + \sum_{n = 1}^\infty a_n \cos(2 \pi n x) + b_n \sin(2 \pi n x)\) converges uniformly to \(f\) on \(\mathbf{R}\) with respect to \(d_{l^1}|_{\mathbf{C} \times \mathbf{C}}\).
\end{proof}

\begin{exercise}\label{ex 5.5.2}
    Let \(f(x)\) be the function defined by \(f(x) = (1 - 2x)^2\) when \(x \in [0, 1)\), and extended to be \(\mathbf{Z}\)-periodic for the rest of the real line.
    \begin{enumerate}
        \item Using Exercise \ref{ex 5.5.1}, show that the series
              \[
                  \frac{1}{3} + \sum_{n = 1}^\infty \frac{4}{\pi^2 n^2} \cos(2 \pi n x)
              \]
              converges uniformly to \(f\) .
        \item Conclude that \(\sum_{n = 1}^\infty \frac{1}{n^2} = \frac{\pi^2}{6}\).
        \item Conclude that \(\sum_{n = 1}^\infty \frac{1}{n^4} = \frac{\pi^4}{90}\).
    \end{enumerate}
\end{exercise}

\begin{proof}{(a)}
    For each \(n \in \mathbf{N}\), we define \(a_n, b_n\) as in Exercise \ref{ex 5.5.1}.
    Observe that for all \(n \in \mathbf{Z}^+\), we have
    \begin{align*}
        a_n & = 2 \int_{[0, 1]} (1 - 2x)^2 \cos(2 \pi n x) \; dx                                         & \text{(by Exercise \ref{ex 5.5.1})} \\
            & = \frac{2}{2 \pi n} \int_{[0, 1]} (1 - 2x)^2 \sin'(2 \pi n x) \; dx                        & \text{(by Theorem \ref{4.7.2}(b))}  \\
            & = \frac{1}{\pi n} \bigg(\big((1 - 2x)^2 \sin(2 \pi n x)\big)|_{x = 0}^{x = 1}              & \text{(by Proposition 11.10.1)}     \\
            & \quad - \int_{[0, 1]} -4 (1 - 2x) \sin(2 \pi n x) \; dx\bigg)                                                                    \\
            & = \frac{4}{\pi n} \int_{[0, 1]} (1 - 2x) \sin(2 \pi n x) \; dx                             & \text{(by Theorem \ref{4.7.2}(e))}  \\
            & = \frac{-4}{2 \pi^2 n^2} \int_{[0, 1]} (1 - 2x) \cos'(2 \pi n x) \; dx                     & \text{(by Theorem \ref{4.7.2}(b))}  \\
            & = \frac{-2}{\pi^2 n^2} \bigg(\big((1 - 2x) \cos(2 \pi n x)\big)|_{x = 0}^{x = 1}                                                 \\
            & \quad - \int_{[0, 1]} -2 \cos(2 \pi n x) \; dx\bigg)                                       & \text{(by Proposition 11.10.1)}     \\
            & = \frac{-2}{\pi^2 n^2} \bigg(-2 + 2 \int_{[0, 1]} \cos(2 \pi n x) \; dx\bigg)              & \text{(by Theorem \ref{4.7.2}(e))}  \\
            & = \frac{-2}{\pi^2 n^2} \bigg(-2 + \frac{2 \sin(2 \pi n x)}{2 \pi n}|_{x = 0}^{x = 1}\bigg) & \text{(by Theorem \ref{4.7.2}(b))}  \\
            & = \frac{4}{\pi^2 n^2}                                                                      & \text{(by Theorem \ref{4.7.2}(e))}
    \end{align*}
    and
    \begin{align*}
        b_n & = 2 \int_{[0, 1]} (1 - 2x)^2 \sin(2 \pi n x) \; dx                               & \text{(by Exercise \ref{ex 5.5.1})} \\
            & = \frac{-2}{2 \pi n} \int_{[0, 1]} (1 - 2x)^2 \cos'(2 \pi n x) \; dx             & \text{(by Theorem \ref{4.7.2}(b))}  \\
            & = \frac{-1}{\pi n} \bigg(\big((1 - 2x)^2 \cos(2 \pi n x)\big)|_{x = 0}^{x = 1}   & \text{(by Proposition 11.10.1)}     \\
            & \quad - \int_{[0, 1]} -4 (1 - 2x) \cos(2 \pi n x) \; dx\bigg)                                                          \\
            & = \frac{-4}{\pi n} \int_{[0, 1]} (1 - 2x) \cos(2 \pi n x) \; dx                  & \text{(by Theorem \ref{4.7.2}(e))}  \\
            & = \frac{-4}{2 \pi^2 n^2} \int_{[0, 1]} (1 - 2x) \sin'(2 \pi n x) \; dx           & \text{(by Theorem \ref{4.7.2}(b))}  \\
            & = \frac{-2}{\pi^2 n^2} \bigg(\big((1 - 2x) \sin(2 \pi n x)\big)|_{x = 0}^{x = 1}                                       \\
            & \quad - \int_{[0, 1]} -2 \sin(2 \pi n x) \; dx\bigg)                             & \text{(by Proposition 11.10.1)}     \\
            & = \frac{-4}{\pi^2 n^2} \int_{[0, 1]} \sin(2 \pi n x) \; dx                       & \text{(by Theorem \ref{4.7.2}(e))}  \\
            & = \frac{-4}{\pi^2 n^2} \frac{-\cos(2 \pi n x)}{2 \pi n}|_{x = 0}^{x = 1}         & \text{(by Theorem \ref{4.7.2}(b))}  \\
            & = 0.                                                                             & \text{(by Theorem \ref{4.7.2}(e))}
    \end{align*}
    Since
    \begin{align*}
        \sum_{n = 1}^\infty \abs*{a_n} & = \sum_{n = 1}^\infty \frac{4}{\pi^2 n^2} \\
        \sum_{n = 1}^\infty \abs*{b_n} & = \sum_{n = 1}^\infty 0
    \end{align*}
    are absolutely convergent (by Corollary 7.3.7 in Analysis I), by Exercise \ref{ex 5.5.1}(b) we know that the series
    \[
        \frac{a_0}{2} + \sum_{n = 1}^\infty (a_n \cos(2 \pi n x) + b_n \sin(2 \pi n x))
    \]
    converges uniformly to \(f\) on \(\mathbf{R}\) with respect to \(d_{l^1}|_{\mathbf{C} \times \mathbf{C}}\), and
    \begin{align*}
         & \frac{a_0}{2} + \sum_{n = 1}^\infty (a_n \cos(2 \pi n x) + b_n \sin(2 \pi n x))                                                                                                  \\
         & = \int_{[0, 1]} (1 - 2x)^2 \cos(0) \; dx + \sum_{n = 1}^\infty \frac{4}{\pi^2 n^2} \cos(2 \pi n x)                                          & \text{(from the proof above)}      \\
         & = \int_{[0, 1]} (1 - 2x)^2 \; dx + \sum_{n = 1}^\infty \frac{4}{\pi^2 n^2} \cos(2 \pi n x)                                                  & \text{(by Theorem \ref{4.7.2}(e))} \\
         & = 1 - \big(2x^2|_{x = 0}^{x = 1}\big) + \big(\frac{4x^3}{3}|_{x = 0}^{x = 1}\big) + \sum_{n = 1}^\infty \frac{4}{\pi^2 n^2} \cos(2 \pi n x)                                      \\
         & = 1 - 2 + \frac{4}{3} + \sum_{n = 1}^\infty \frac{4}{\pi^2 n^2} \cos(2 \pi n x)                                                                                                  \\
         & = \frac{1}{3} + \sum_{n = 1}^\infty \frac{4}{\pi^2 n^2} \cos(2 \pi n x).
    \end{align*}
\end{proof}

\begin{proof}{(b)}
    We have
    \begin{align*}
                 & (1 - 2 \cdot 0)^2 = \frac{1}{3} + \sum_{n = 1}^\infty \frac{4}{\pi^2 n^2} \cos(2 \pi n \cdot 0) & \text{(by Exercise \ref{ex 5.5.2}(a))} \\
        \implies & 1 = \frac{1}{3} + \sum_{n = 1}^\infty \frac{4}{\pi^2 n^2}                                       & \text{(by Theorem \ref{4.7.2}(e))}     \\
        \implies & \frac{2}{3} = \sum_{n = 1}^\infty \frac{4}{\pi^2 n^2}                                                                                    \\
        \implies & \frac{\pi^2}{6} = \sum_{n = 1}^\infty \frac{1}{n^2}.
    \end{align*}
\end{proof}

\begin{proof}{(c)}
    By Exercise \ref{ex 5.5.2}(a) we know that
    \[
        f(x) = (1 - 2x)^2 = \frac{1}{3} + \sum_{n = 1}^\infty \frac{4}{\pi^2 n^2} \cos(2 \pi n x)
    \]
    and the series on the right hand side converges uniformly to \(f\).
    Observe that for each \(m \in \mathbf{Z}\), we have
    \begin{align*}
         & \hat{f}(m)                                                                                                                                                                                                     \\
         & = \int_{[0, 1]} \bigg(\frac{1}{3} + \sum_{n = 1}^\infty \frac{4}{\pi^2 n^2} \cos(2 \pi n x)\bigg) e^{-2 \pi i m x} \; dx                                                & \text{(by Definition \ref{5.3.7})}   \\
         & = \int_{[0, 1]} \frac{e^{- 2 \pi i m x}}{3} \; dx + \int_{[0, 1]} \sum_{n = 1}^\infty \frac{4 e^{- 2 \pi i m x}}{\pi^2 n^2} \cos(2 \pi n x) \; dx                       & \text{(by Remark \ref{5.2.2})}       \\
         & = \int_{[0, 1]} \frac{e^{- 2 \pi i m x}}{3} \; dx + \sum_{n = 1}^\infty \int_{[0, 1]} \frac{4 e^{- 2 \pi i m x}}{\pi^2 n^2} \cos(2 \pi n x) \; dx                       & \text{(by Theorem \ref{3.6.2})}      \\
         & = \int_{[0, 1]} \frac{e^{- 2 \pi i m x}}{3} \; dx + \sum_{n = 1}^\infty \int_{[0, 1]} \frac{2 e^{- 2 \pi i m x} (e^{2 \pi i n x} + e^{- 2 \pi i n x})}{\pi^2 n^2} \; dx & \text{(by Definition \ref{4.7.1})}   \\
         & = \int_{[0, 1]} \frac{e^{- 2 \pi i m x}}{3} \; dx + \sum_{n = 1}^\infty \int_{[0, 1]} \frac{2 (e^{2 \pi i (n - m) x} + e^{- 2 \pi i (n + m) x})}{\pi^2 n^2} \; dx.      & \text{(by Exercise \ref{ex 4.6.16})}
    \end{align*}
    Now we split into two cases:
    \begin{itemize}
        \item If \(m = 0\), then we have
              \begin{align*}
                  \hat{f}(0) & = \int_{[0, 1]} \frac{1}{3} \; dx + \sum_{n = 1}^\infty \int_{[0, 1]} \frac{2 (e^{2 \pi i n x} + e^{- 2 \pi i n x})}{\pi^2 n^2} \; dx & \text{(by Theorem \ref{4.5.2}(e))} \\
                             & = \frac{1}{3} + \sum_{n = 1}^\infty \bigg(\frac{2}{\pi^2 n^2} \int_{[0, 1]} e^{2 \pi i n x} + e^{- 2 \pi i n x} \; dx\bigg)                                                \\
                             & = \frac{1}{3} + \sum_{n = 1}^\infty \frac{2}{\pi^2 n^2} (0 + 0)                                                                       & \text{(by Lemma \ref{5.3.5})}      \\
                             & = \frac{1}{3}.
              \end{align*}
        \item If \(m \neq 0\), then we have
              \begin{align*}
                   & \hat{f}(m)                                                                                                                                                                 \\
                   & = 0 + \sum_{n = 1}^\infty \int_{[0, 1]} \frac{2 (e^{2 \pi i (n - m) x} + e^{- 2 \pi i (n + m) x})}{\pi^2 n^2} \; dx                   & \text{(by Lemma \ref{5.3.5})}      \\
                   & = \sum_{n = 1}^\infty \Bigg(\frac{2}{\pi^2 n^2} \bigg(\int_{[0, 1]} e^{2 \pi i (n - m) x} + e^{- 2 \pi i (n + m) x} \; dx\bigg)\Bigg)                                      \\
                   & = \sum_{n = 1}^\infty \bigg(\frac{2}{\pi^2 n^2} \big(\inner*{e_n, e_m} + \inner*{e_{-n}, e_m}\big)\bigg)                              & \text{(by Definition \ref{5.2.1})} \\
                   & = \frac{2}{\pi^2 m^2}.                                                                                                                & \text{(by Lemma \ref{5.3.5})}
              \end{align*}
    \end{itemize}
    From all cases above we have
    \begin{align*}
                 & \norm*{f}_2^2 = \sum_{n = -\infty}^\infty \abs*{\hat{f}(n)}^2                                                                      & \text{(by Theorem \ref{5.5.4})} \\
        \implies & \int_{[0, 1]} (1 - 2x)^4 \; dx = \sum_{n = -\infty}^\infty \abs*{\hat{f}(n)}^2                                                                                       \\
        \implies & \int_{[0, 1]} 1 - 8x + 24x^2 - 32x^3 + 16x^4 \; dx = \sum_{n = -\infty}^\infty \abs*{\hat{f}(n)}^2                                                                   \\
        \implies & 1 - 4 (x^2|_{x = 0}^{x = 1}) + 8 (x^3|_{x = 0}^{x = 1}) - 8(x^4|_{x = 0}^{x = 1}) + \frac{16}{5}(x^5|_{x = 0}^{x = 1})                                               \\
                 & = \sum_{n = -\infty}^\infty \abs*{\hat{f}(n)}^2                                                                                                                      \\
        \implies & \frac{1}{5} = \frac{1}{9} + \sum_{n = 1}^\infty \abs*{\frac{2}{\pi^2 n^2}}^2 + \sum_{n = 1}^\infty \abs*{\frac{2}{\pi^2 (-n)^2}}^2 & \text{(from the proof above)}   \\
        \implies & \frac{4}{45} = 2 \sum_{n = 1}^\infty \abs*{\frac{2}{\pi^2 n^2}}^2                                                                                                    \\
        \implies & \frac{4}{45} = 8 \sum_{n = 1}^\infty \frac{1}{\pi^4 n^4}                                                                                                             \\
        \implies & \frac{\pi^4}{90} = \sum_{n = 1}^\infty \frac{1}{n^4}.
    \end{align*}
\end{proof}

\begin{exercise}\label{ex 5.5.3}
    If \(f \in C(\mathbf{R} / \mathbf{Z} ; \mathbf{C})\) and \(P\) is a trigonometric polynomial, show that
    \[
        \widehat{f * P}(n) = \hat{f}(n) c_n = \hat{f}(n) \hat{P}(n)
    \]
    for all integers \(n\).
    More generally, if \(f, g \in C(\mathbf{R} / \mathbf{Z} ; \mathbf{C})\), show that
    \[
        \widehat{f * g}(n) = \hat{f}(n) \hat{g}(n)
    \]
    for all integers \(n\).
    (A fancy way of saying this is that the Fourier transform \emph{intertwines} convolution and multiplication.)
\end{exercise}

\begin{proof}
    Let \(P = \sum_{n = -N}^N c_n e_n\) for some \(N \in \mathbf{Z}^+\) and some \((c_n)_{n = -N}^N\) in \(\mathbf{C}\).
    By Additional Corollary \ref{ac 5.4.1} we know that
    \[
        f * P = \sum_{n = -N}^N \hat{f}(n) c_n e_n.
    \]
    Thus we have
    \begin{align*}
        \forall m \in \mathbf{Z}, \widehat{f * P}(m) & = \inner{f * P, e_m}                               & \text{(by Definition \ref{5.3.7})} \\
                                                     & = \sum_{n = -N}^N \hat{f}(n) c_n \inner{e_n, e_m}  & \text{(by Lemma \ref{5.2.5}(c))}   \\
                                                     & = \begin{cases}
                                                             \hat{f}(m) c_m & \text{if } n = m    \\
                                                             0              & \text{if } n \neq m
                                                         \end{cases}            & \text{(by Lemma \ref{5.3.5})}                                \\
                                                     & = \hat{f}(m) \sum_{n = -N}^N c_n \inner*{e_n, e_m} & \text{(by Lemma \ref{5.3.5})}      \\
                                                     & = \hat{f}(m) \inner*{\sum_{n = -N}^N c_n e_n, e_m} & \text{(by Lemma \ref{5.2.5}(c))}   \\
                                                     & = \hat{f}(m) \hat{P}(m).                           & \text{(by Definition \ref{5.3.7})}
    \end{align*}

    Now we show that \(\widehat{f * g}(n) = \hat{f}(n) \hat{g}(n)\) for all \(n \in \mathbf{Z}\).
    In particular, we want to show that
    \[
        \forall \varepsilon \in \mathbf{R}^+, \abs*{\widehat{f * g}(n) - \hat{f}(n) \hat{g}(n)} \leq \varepsilon.
    \]
    Let \(\varepsilon \in \mathbf{R}^+\).
    Since \(g \in C(\mathbf{R} / \mathbf{Z} ; \mathbf{C})\), by Theorem \ref{5.4.1} we know that there exists a trigonometric polynomial \(P\) such that
    \[
        \norm*{g - P}_{\infty} \leq \varepsilon.
    \]
    Fix such \(P\).
    Since \(f \in C(\mathbf{R} / \mathbf{Z} ; \mathbf{C})\), by Theorem \ref{5.5.4} we know that \(\sum_{n = -\infty}^\infty \abs*{\hat{f}(n)}^2 \in \mathbf{R}\), thus
    \[
        \exists\ M \in \mathbf{R}^+ : \forall n \in \mathbf{Z}, \abs*{\hat{f}(n)} \leq M.
    \]
    Fix such \(M\).
    Then for all \(n \in \mathbf{Z}\), we have
    \begin{align*}
        \abs*{\widehat{f * g}(n) - \widehat{f * P}(n)} & = \abs*{\inner*{f * g, e_n} - \inner*{f * P, e_n}} & \text{(by Definition \ref{5.3.7})} \\
                                                       & = \abs*{\inner*{f * g - f * P, e_n}}               & \text{(by Lemma \ref{5.2.5}(c))}   \\
                                                       & = \abs*{\inner*{f * (g - P), e_n}}                 & \text{(by Lemma \ref{5.4.4}(c))}   \\
                                                       & \leq \norm*{f * (g - P)}_2 \norm*{e_n}_2           & \text{(by Lemma \ref{5.2.7}(b))}   \\
                                                       & = \norm*{f * (g - P)}_2.                           & \text{(by Lemma \ref{5.3.5})}
    \end{align*}
    Need some helps.
\end{proof}

\begin{exercise}\label{ex 5.5.4}
    Let \(f \in C(\mathbf{R} / \mathbf{Z} ; \mathbf{C})\) be a function which is differentiable, and whose derivative \(f'\) is also continuous.
    Show that \(f'\) also lies in \(C(\mathbf{R} / \mathbf{Z} ; \mathbf{C})\), and that \(\hat{f}'(n) = 2 \pi i n \hat{f}(n)\) for all integers \(n\).
    Here the derivative of a complex-valued function is defined in exactly the same fashion as for real-valued functions.
\end{exercise}

\begin{exercise}\label{ex 5.5.5}
    Let \(f, g \in C(\mathbf{R} / \mathbf{Z} ; \mathbf{C})\).
    Prove the \emph{Parseval identity}
    \[
        \Re\bigg(\int_0^1 f(x) \overline{g(x)} \; dx\bigg) = \Re\bigg(\sum_{n \in \mathbf{Z}} \hat{f}(n) \overline{\hat(g)(n)}\bigg).
    \]
    Then conclude that the real parts can be removed, thus
    \[
        \int_0^1 f(x) \overline{g(x)} \; dx = \sum_{n \in \mathbf{Z}} \hat{f}(n) \overline{\hat(g)(n)}.
    \]
\end{exercise}

\begin{exercise}\label{ex 5.5.6}
    In this exercise we shall develop the theory of Fourier series for functions of any fixed period \(L\).

    Let \(L > 0\), and let \(f : \mathbf{R} \to \mathbf{C}\) be a complex-valued function which is continuous and \(L\)-periodic.
    Define the numbers \(c_n\) for every integer \(n\) by
    \[
        c_n \coloneqq \frac{1}{L} \int_{[0, L]} f(x) e^{- 2 \pi i n x / L} \; dx.
    \]
    \begin{enumerate}
        \item Show that the series
              \[
                  \sum_{n = -\infty}^\infty c_n e^{2 \pi i n x / L}
              \]
              converges in \(L_2\) metric to \(f\).
              More precisely, show that
              \[
                  \lim_{N \to \infty} \int_{[0, L]} \abs*{f(x) - \sum_{n = -N}^N c_n e^{2 \pi i n x / L}}^2 \; dx = 0.
              \]
        \item If the series \(\sum_{n = -\infty}^\infty \abs*{c_n}\) is absolutely convergent, show that
              \[
                  \sum_{n = -\infty}^\infty c_n e^{2 \pi i n x / L}
              \]
              converges uniformly to \(f\).
        \item Show that
              \[
                  \frac{1}{L} \int_{[0, L]} \abs*{f(x)}^2 \; dx = \sum_{n = -\infty}^\infty \abs*{c_n}^2.
              \]
    \end{enumerate}
\end{exercise}