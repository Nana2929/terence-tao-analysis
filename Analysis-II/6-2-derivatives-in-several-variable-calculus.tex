\section{Derivatives in several variable calculus}\label{sec 6.2}

\begin{note}
    In single variable calculus, when one wants to differentiate a function \(f : E \to \mathbf{R}\) at a point \(x_0\), where \(E\) is a subset of \(\mathbf{R}\) and \(x_0\) is a limit point of \(E\), this is given by
    \[
        f'(x_0) \coloneqq \lim_{x \to x_0 ; x \in E \setminus \{x_0\}} \frac{f(x) - f(x_0)}{x - x_0}.
    \]
    One could try to mimic this definition in the several variable case \(f : E \to \mathbf{R}^m\), where \(E\) is now a subset of \(\mathbf{R}^n\), however we encounter a difficulty in this case:
    the quantity \(f(x) - f(x_0)\) will live in \(\mathbf{R}^m\), and \(x - x_0\) lives in \(\mathbf{R}^n\), and we do not know how to divide an \(m\)-dimensional vector by an \(n\)-dimensional vector.

    To get around this problem, we first rewrite the concept of derivative (in one dimension) in a way which does not involve division of vectors.
    Instead, we view differentiability at a point \(x_0\) as an assertion that a function \(f\) is ``approximately linear'' near \(x_0\).
\end{note}

\begin{lemma}\label{6.2.1}
    Let \(E\) be a subset of \(\mathbf{R}\), \(f : E \to \mathbf{R}\) be a function, \(L \in \mathbf{R}\), and let \(x_0\) be a limit point of \(E\).
    Then the following two statements are equivalent.
    \begin{enumerate}
        \item \(f\) is differentiable at \(x_0\), and \(f'(x_0) = L\).
        \item We have
              \[
                  \lim_{x \to x_0 ; x \in E \setminus \{x_0\}} \frac{\abs*{f(x) - \big(f(x_0) + L (x - x_0)\big)}}{\abs*{x - x_0}} = 0.
              \]
    \end{enumerate}
\end{lemma}

\begin{proof}
    We have
    \begin{align*}
             & \lim_{x \to x_0 ; x \in E \setminus \{x_0\}} \frac{f(x) - f(x_0)}{x - x_0} = L                                                           \\
        \iff & \forall \varepsilon \in \mathbf{R}^+, \exists\ \delta \in \mathbf{R}^+ : \forall x \in E \setminus \{x_0\},                              \\
             & \bigg(\abs*{x - x_0} < \delta \implies \abs*{\frac{f(x) - f(x_0)}{x - x_0} - L} < \varepsilon\bigg)                                      \\
        \iff & \forall \varepsilon \in \mathbf{R}^+, \exists\ \delta \in \mathbf{R}^+ : \forall x \in E \setminus \{x_0\},                              \\
             & \bigg(\abs*{x - x_0} < \delta \implies \abs*{\frac{f(x) - f(x_0) - L(x - x_0)}{x - x_0}} < \varepsilon\bigg)                             \\
        \iff & \forall \varepsilon \in \mathbf{R}^+, \exists\ \delta \in \mathbf{R}^+ : \forall x \in E \setminus \{x_0\},                              \\
             & \bigg(\abs*{x - x_0} < \delta \implies \abs*{\frac{f(x) - \big(f(x_0) + L(x - x_0)\big)}{x - x_0}} < \varepsilon\bigg)                   \\
        \iff & \forall \varepsilon \in \mathbf{R}^+, \exists\ \delta \in \mathbf{R}^+ : \forall x \in E \setminus \{x_0\},                              \\
             & \bigg(\abs*{x - x_0} < \delta \implies \abs*{\frac{\abs*{f(x) - \big(f(x_0) + L(x - x_0)\big)}}{\abs*{x - x_0}}} < \varepsilon\bigg)     \\
        \iff & \forall \varepsilon \in \mathbf{R}^+, \exists\ \delta \in \mathbf{R}^+ : \forall x \in E \setminus \{x_0\},                              \\
             & \bigg(\abs*{x - x_0} < \delta \implies \abs*{\frac{\abs*{f(x) - \big(f(x_0) + L(x - x_0)\big)}}{\abs*{x - x_0}} - 0} < \varepsilon\bigg) \\
        \iff & \lim_{x \to x_0 ; x \in E \setminus \{x_0\}} \frac{\abs*{f(x) - \big(f(x_0) + L(x - x_0)\big)}}{\abs*{x - x_0}} = 0.
    \end{align*}
\end{proof}

\begin{note}
    In light of Lemma \ref{6.2.1}, we see that the derivative \(f'(x_0)\) can be interpreted as the number \(L\) for which \(\abs*{f(x) - \big(f(x_0) + L(x - x_0)\big)}\) is small, in the sense that it tends to zero as \(x\) tends to \(x_0\), even if we divide out by the very small number \(\abs*{x - x_0}\).
    More informally, the derivative is the quantity \(L\) such that we have the approximation \(f(x) - f(x_0) \approx L(x - x_0)\).

    This does not seem too different from the usual notion of differentiation, but the point is that we are no longer explicitly dividing by \(x - x_0\).
    (We are still dividing by \(\abs*{x - x_0}\), but this will turn out to be OK.)
    When we move to the several variable case \(f : E \to \mathbf{R}^m\), where \(E \subseteq \mathbf{R}^n\), we shall still want the derivative to be some quantity \(L\) such that \(f(x) - f(x_0) \approx L(x - x_0)\).
    However, since \(f(x) - f(x_0)\) is now an \(m\)-dimensional vector and \(x - x_0\) is an \(n\)-dimensional vector, we no longer want \(L\) to be a scalar;
    we want it to be a linear transformation.
\end{note}

\begin{definition}[Differentiability]\label{6.2.2}
    Let \(E\) be a subset of \(\mathbf{R}^n\), \(f : E \to \mathbf{R}^m\) be a function, \(x_0 \in E\) be a limit point, and let \(L : \mathbf{R}^n \to \mathbf{R}^m\) be a linear transformation.
    We say that \(f\) is \emph{differentiable at \(x_0\) with derivative \(L\)} if we have
    \[
        \lim_{x \to x_0 ; x \in E \setminus \{x_0\}} \frac{\norm*{f(x) - \big(f(x_0) + L(x - x_0)\big)}}{\norm*{x - x_0}} = 0.
    \]
    Here \(\norm*{x}\) is the length of \(x\) (as measured in the \(l^2\) metric)
    \[
        \norm*{(x_1, x_2, \dots, x_n)} = (x_1^2 + x_2^2 + \dots + x_n^2)^{1 / 2}.
    \]
\end{definition}

\setcounter{theorem}{3}
\begin{lemma}[Uniqueness of derivatives]\label{6.2.4}
    Let \(E\) be a subset of \(\mathbf{R}^n\), \(f : E \to \mathbf{R}^m\) be a function, \(x_0 \in E\) be an \emph{interior point} of \(E\), and let \(L_1 : \mathbf{R}^n \to \mathbf{R}^m\) and \(L_2 : \mathbf{R}^n \to \mathbf{R}^m\) be linear transformations.
    Suppose that \(f\) is differentiable at \(x_0\) with derivative \(L_1\), and also differentiable at \(x_0\) with derivative \(L_2\).
    Then \(L_1 = L_2\).
\end{lemma}

\begin{proof}
    Suppose for sake of contradiction that \(L_1 \neq L_2\).
    Then there exists a \(v \in \mathbf{R}^n\) such that \(L_1(v) \neq L_2(v)\).
    Fix such \(v\).
    We know that \(v \neq 0_{\mathbf{R}^n}\) since if \(v = 0_{\mathbf{R}^n}\), then we would have
    \[
        L_1(0_{\mathbf{R}^n}) = L_1(0 \cdot 0_{\mathbf{R}^n}) = 0 L_1(0_{\mathbf{R}^n}) = 0_{\mathbf{R}^m} = 0 L_2(0_{\mathbf{R}^n}) = L_2(0 \cdot 0_{\mathbf{R}^n}) = L_2(0_{\mathbf{R}^n}).
    \]
    Observe that
    \begin{align*}
        \forall t \in \mathbf{R}, \norm*{x_0 + tv - x_0} & = \sqrt{\sum_{i = 1}^n \big((x_0 + tv - x_0)_i\big)^2} & \text{(by Example \ref{1.1.6})}    \\
                                                         & = \sqrt{\sum_{i = 1}^n (t v_i)^2}                      & \text{(by Definition \ref{6.1.1})} \\
                                                         & = \abs*{t} \sqrt{\sum_{i = 1}^n (v_i)^2}.
    \end{align*}
    Since \(\lim_{t \to 0 ; t \neq 0} \abs*{t} = 0\), we have
    \[
        \lim_{t \to 0 ; t \neq 0} \norm*{x_0 + tv - x_0} = 0
    \]
    which means
    \[
        \forall \varepsilon_t \in \mathbf{R}^+, \exists\ \delta \in \mathbf{R}^+ : \forall t \in \mathbf{R} \setminus \{0\}, \abs*{t} < \delta \implies \norm*{x_0 + tv - x_0} < \varepsilon_t.
    \]
    Since \(x_0\) is an interior point, by Definition \ref{1.2.5} we know that
    \[
        \exists\ t \in \mathbf{R} \setminus \{0\} : x_0 + tv \in E.
    \]
    Thus we have
    \[
        \forall \varepsilon_t \in \mathbf{R}^+, \exists\ \delta \in \mathbf{R}^+ : \forall t \in \mathbf{R} \setminus \{0\}, \abs*{t} < \delta \implies \begin{cases}
            \norm*{x_0 + tv - x_0} < \varepsilon_t; \\
            x_0 + tv \in E.
        \end{cases}
    \]
    Since \(f\) is differentiable at \(x_0\) with derivative \(L_1\), by Definition \ref{6.2.2} we know that
    \begin{align*}
         & \forall \varepsilon \in \mathbf{R}^+, \exists\ \delta_1 \in \mathbf{R}^+ : \forall x \in E \setminus \{x_0\},             \\
         & \norm*{x - x_0} < \delta_1 \implies \frac{\norm*{f(x) - \big(f(x_0) + L_1(x - x_0)\big)}}{\norm*{x - x_0}} < \varepsilon.
    \end{align*}
    Similarly, we have
    \begin{align*}
         & \forall \varepsilon \in \mathbf{R}^+, \exists\ \delta_2 \in \mathbf{R}^+ : \forall x \in E \setminus \{x_0\},             \\
         & \norm*{x - x_0} < \delta_2 \implies \frac{\norm*{f(x) - \big(f(x_0) + L_2(x - x_0)\big)}}{\norm*{x - x_0}} < \varepsilon.
    \end{align*}
    Let \(\varepsilon_t = \min(\delta_1, \delta_2)\).
    Then we have
    \begin{align*}
                 & \forall \varepsilon \in \mathbf{R}^+, \exists\ \delta \in \mathbf{R}^+ : \forall t \in \mathbf{R} \setminus \{x_0\}, \abs*{t} < \delta \\
        \implies & \begin{cases}
                       \norm*{x_0 + tv - x_0} < \varepsilon_t \\
                       x_0 + tv \in E
                   \end{cases}                                                                                                 \\
        \implies & \begin{cases}
                       \frac{\norm*{f(x_0 + tv) - \big(f(x_0) + L_1(x_0 + tv - x_0)\big)}}{\norm*{x_0 + tv - x_0}} < \varepsilon \\
                       \frac{\norm*{f(x_0 + tv) - \big(f(x_0) + L_2(x_0 + tv - x_0)\big)}}{\norm*{x_0 + tv - x_0}} < \varepsilon
                   \end{cases}                                                                   \\
        \implies & \begin{cases}
                       \frac{\norm*{f(x_0 + tv) - \big(f(x_0) + L_1(tv)\big)}}{\norm*{tv}} < \varepsilon \\
                       \frac{\norm*{f(x_0 + tv) - \big(f(x_0) + L_2(tv)\big)}}{\norm*{tv}} < \varepsilon
                   \end{cases}
    \end{align*}
    and thus
    \begin{align*}
         & \lim_{t \to 0 ; t \neq 0, x_0 + tv \in E} \frac{\norm*{f(x_0 + tv) - \big(f(x_0) + L_1(tv)\big)}}{\norm*{tv}} = 0; \\
         & \lim_{t \to 0 ; t \neq 0, x_0 + tv \in E} \frac{\norm*{f(x_0 + tv) - \big(f(x_0) + L_2(tv)\big)}}{\norm*{tv}} = 0.
    \end{align*}
    By limit laws we have
    \begin{align*}
         & \lim_{t \to 0 ; t \neq 0, x_0 + tv \in E} \frac{\norm*{f(x_0 + tv) - \big(f(x_0) + L_1(tv)\big)} + \norm*{f(x_0 + tv) - \big(f(x_0) + L_2(tv)\big)}}{\norm*{tv}} \\
         & = 0.
    \end{align*}
    By Example \ref{1.1.6} and Definition \ref{1.1.2} we know that
    \begin{align*}
         & \norm*{f(x_0 + tv) - \big(f(x_0) + L_1(tv)\big)} + \norm*{f(x_0 + tv) - \big(f(x_0) + L_2(tv)\big)}   \\
         & = \norm*{f(x_0 + tv) - \big(f(x_0) + L_1(tv)\big)} + \norm*{\big(f(x_0) + L_2(tv)\big) - f(x_0 + tv)} \\
         & \geq \norm*{f(x_0 + tv) - \big(f(x_0) + L_1(tv)\big) + \big(f(x_0) + L_2(tv)\big) - f(x_0 + tv)}      \\
         & = \norm*{L_2(tv) - L_1(tv)}                                                                           \\
         & \geq 0.
    \end{align*}
    Thus by squeeze test we have
    \[
        \lim_{t \to 0 ; t \neq 0, x_0 + tv \in E} \frac{\norm*{L_2(tv) - L_1(tv)}}{\norm*{tv}} = 0
    \]
    which implies
    \begin{align*}
         & \lim_{t \to 0 ; t \neq 0, x_0 + tv \in E} \frac{\norm*{L_2(tv) - L_1(tv)}}{\norm*{tv}}                                                                   \\
         & = \lim_{t \to 0 ; t \neq 0, x_0 + tv \in E} \frac{\norm*{t L_2(v) - t L_1(v)}}{\norm*{tv}}                          & \text{(by Definition \ref{6.1.6})} \\
         & = \lim_{t \to 0 ; t \neq 0, x_0 + tv \in E} \frac{\abs*{t} \cdot \norm*{L_2(v) - L_1(v)}}{\abs*{t} \cdot \norm*{v}}                                      \\
         & = \lim_{t \to 0 ; t \neq 0, x_0 + tv \in E} \frac{\norm*{L_2(v) - L_1(v)}}{\norm*{v}}                                                                    \\
         & = \frac{\norm*{L_2(v) - L_1(v)}}{\norm*{v}}                                                                                                              \\
         & = 0.
    \end{align*}
    But \(v \neq 0\) implies \(\norm*{v} \neq 0\) (by Definition \ref{1.1.2}(b)), so we must have \(L_2(v) = L_1(v)\), a contradiction.
    Thus \(L_1 = L_2\).
\end{proof}

\begin{note}
    Because of Lemma \ref{6.2.4}, we can now talk about \emph{the} derivative of \(f\) at interior points \(x_0\), and we will denote this derivative by \(f'(x_0)\).
    Thus \(f'(x_0)\) is the unique linear transformation from \(\mathbf{R}^n\) to \(\mathbf{R}^m\) such that
    \[
        \lim_{x \to x_0 ; x \in E \setminus \{x_0\}} \frac{\norm*{f(x) - \big(f(x_0) + f'(x_0)(x - x_0)\big)}}{\norm*{x - x_0}} = 0.
    \]
    Informally, this means that the derivative \(f'(x_0))\) is the linear transformation such that we have
    \[
        f(x) - f(x_0) \approx f'(x_0)(x - x_0)
    \]
    or equivalently
    \[
        f(x) \approx f(x_0) + f'(x_0)(x - x_0)
    \]
    (this is known as Newton's approximation;
    compare with Proposition 10.1.7 in Analysis I).
\end{note}

\begin{note}
    Another consequence of Lemma \ref{6.2.4} is that if you know that \(f(x) = g(x)\) for all \(x \in E\), and \(f, g\) are differentiable at \(x_0\), then you also know that \(f'(x_0) = g'(x_0)\) at every \emph{interior} point of \(E\).
    However, this is not necessarily true if \(x_0\) is a boundary point of \(E\);
    for instance, if \(E\) is just a single point \(E = \{x_0\}\), merely knowing that \(f(x_0) = g(x_0)\) does not imply that \(f'(x_0) = g'(x_0)\).
    We will not deal with these boundary issues here, and only compute derivatives on the interior of the domain.
\end{note}

\begin{note}
    We will sometimes refer to \(f'\) as the \emph{total derivative} of \(f\), to distinguish this concept from that of partial and directional derivatives below.
    The total derivative \(f\) is also closely related to the \emph{derivative matrix} \(Df\),
    which we shall define in the next section.
\end{note}

\exercisesection

\begin{exercise}\label{ex 6.2.1}
    Prove Lemma \ref{6.2.1}.
\end{exercise}

\begin{proof}
    See Lemma \ref{6.2.1}.
\end{proof}

\begin{exercise}\label{ex 6.2.2}
    Prove Lemma \ref{6.2.4}.
\end{exercise}

\begin{proof}
    See Lemma \ref{6.2.4}.
\end{proof}