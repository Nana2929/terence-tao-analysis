\section{The several variable calculus chain rule}\label{sec 6.4}

\begin{theorem}[Several variable calculus chain rule]\label{6.4.1}
    Let \(E\) be a subset of \(\mathbf{R}^n\), and let \(F\) be a subset of \(\mathbf{R}^m\).
    Let \(f : E \to F\) be a function, and let \(g : F \to \mathbf{R}^p\) be another function.
    Let \(x_0\) be a point in the interior of \(E\).
    Suppose that \(f\) is differentiable at \(x_0\), and that \(f(x_0)\) is in the interior of \(F\).
    Suppose also that \(g\) is differentiable at \(f(x_0)\).
    Then \(g \circ f : E \to \mathbf{R}^p\) is also differentiable at \(x_0\), and we have the formula
    \[
        (g \circ f)'(x_0) = g'\big(f(x_0)\big) \circ f'(x_0).
    \]
\end{theorem}

\begin{proof}
    By Definition \ref{6.2.2} we want to show that
    \[
        \lim_{x \to x_0 ; x \in E \setminus \{x_0\}} \frac{\norm*{(g \circ f)(x) - (g \circ f)(x_0) - \Big(g'\big(f(x_0)\big) \circ f'(x_0)\Big)(x - x_0)}}{\norm*{x - x_0}} = 0.
    \]
    Equivalently, we want to show that
    \begin{align*}
                 & \forall \varepsilon \in \mathbf{R}^+, \exists\ \delta \in \mathbf{R}^+ : \forall x \in E \setminus \{x_0\}, \norm*{x - x_0} < \delta  \\
        \implies & \frac{\norm*{(g \circ f)(x) - (g \circ f)(x_0) - \Big(g'\big(f(x_0)\big) \circ f'(x_0)\Big)(x - x_0)}}{\norm*{x - x_0}} < \varepsilon \\
        \implies & \norm*{(g \circ f)(x) - (g \circ f)(x_0) - \Big(g'\big(f(x_0)\big) \circ f'(x_0)\Big)(x - x_0)} < \varepsilon \norm*{x - x_0}.
    \end{align*}
    Since \(f'(x_0)\) exists, we know that
    \[
        \lim_{x \to x_0 ; x \in E \setminus \{x_0\}} \frac{\norm*{f(x) - f(x_0) - f'(x_0)(x - x_0)}}{\norm*{x - x_0}} = 0.
    \]
    Equivalently, we know that
    \begin{align*}
                 & \forall \varepsilon_f \in \mathbf{R}^+, \exists\ \delta \in \mathbf{R}^+ : \forall x \in E \setminus \{x_0\}, \norm*{x - x_0} < \delta \\
        \implies & \frac{\norm*{f(x) - f(x_0) - f'(x_0)(x - x_0)}}{\norm*{x - x_0}} < \varepsilon_f                                                       \\
        \implies & \norm*{f(x) - f(x_0) - f'(x_0)(x - x_0)} < \varepsilon_f \norm*{x - x_0}                                                               \\
        \implies & \norm*{f(x) - f(x_0)} < \varepsilon_f \norm*{x - x_0} + \norm*{f'(x_0)(x - x_0)}.
    \end{align*}
    Since \(g'\big(f(x_0)\big)\) exists, we know that
    \[
        \lim_{y \to f(x_0) ; x \in F \setminus \{f(x_0)\}} \frac{\norm*{g(y) - g\big(f(x_0)\big) - g'\big(f(x_0)\big)\big(y - f(x_0)\big)}}{\norm*{y - f(x_0)}} = 0.
    \]
    Equivalently, we know that
    \begin{align*}
                 & \forall \varepsilon \in \mathbf{R}^+, \exists\ \delta_g \in \mathbf{R}^+ : \forall y \in F \setminus \{f(x_0)\}, \norm*{y - f(x_0)} < \delta_g \\
        \implies & \frac{\norm*{g(y) - g\big(f(x_0)\big) - g'\big(f(x_0)\big)\big(y - f(x_0)\big)}}{\norm*{y - f(x_0)}} < \varepsilon                             \\
        \implies & \norm*{g(y) - g\big(f(x_0)\big) - g'\big(f(x_0)\big)\big(y - f(x_0)\big)} < \varepsilon \norm*{y - f(x_0)}.
    \end{align*}
    Fix one pair of \(\varepsilon\) and \(\delta_g\).
    Then we have
    \begin{align*}
                 & \exists\ \delta \in \mathbf{R}^+ : \forall x \in E \setminus \{x_0\}, \norm*{x - x_0} < \delta                               \\
        \implies & \begin{cases}
                       \norm*{f(x) - f(x_0)} < \varepsilon \norm*{x - x_0} + \norm*{f'(x_0)(x - x_0)}; \\
                       \norm*{f(x) - f(x_0)} < \delta_g;
                   \end{cases}                                              \\
        \implies & \begin{cases}
                       \norm*{f(x) - f(x_0)} < \varepsilon \norm*{x - x_0} + \norm*{f'(x_0)(x - x_0)}; \\
                       \norm*{g\big(f(x)\big) - g\big(f(x_0)\big) - g'\big(f(x_0)\big) \big(f(x) - f(x_0)\big)} < \varepsilon \norm*{f(x) - f(x_0)};
                   \end{cases} \\
        \implies & \norm*{g\big(f(x)\big) - g\big(f(x_0)\big) - g'\big(f(x_0)\big) \big(f(x) - f(x_0)\big)}                                     \\
                 & < \varepsilon^2 \norm*{x - x_0} + \varepsilon \norm*{f'(x_0)(x - x_0)}                                                       \\
        \implies & \norm*{g\big(f(x)\big) - g\big(f(x_0)\big) - \Big(g'\big(f(x_0)\big) \circ f'(x_0)\Big) (x - x_0)}                           \\
                 & < \varepsilon^2 \norm*{x - x_0} + \varepsilon \norm*{f'(x_0)(x - x_0)}                                                       \\
                 & \quad + \norm*{g'\big(f(x_0)\big) \big(f(x) - f(x_0)\big) - \Big(g'\big(f(x_0)\big) \circ f'(x_0)\Big) (x - x_0)}.
    \end{align*}
    Since \(g'\big(f(x_0)\big)\) is a linear transformation, by Definition \ref{6.1.6} we know that
    \begin{align*}
         & \norm*{g'\big(f(x_0)\big) \big(f(x) - f(x_0)\big) - \Big(g'\big(f(x_0)\big) \circ f'(x_0)\Big) (x - x_0)} \\
         & = \norm*{g'\big(f(x_0)\big) \big(f(x) - f(x_0) - f'(x_0) (x - x_0)\big)}.
    \end{align*}
    By Exercise \ref{ex 6.1.4} we know that
    \begin{align*}
        \exists\ M \in \mathbf{R}^+ : & \norm*{g'\big(f(x_0)\big) \big(f(x) - f(x_0) - f'(x_0) (x - x_0)\big)} \\
                                      & \leq M \norm*{f(x) - f(x_0) - f'(x_0) (x - x_0)}                       \\
                                      & \leq M \varepsilon \norm*{x - x_0}.
    \end{align*}
    Fix such \(M\).
    Then we have
    \begin{align*}
                 & \exists\ \delta \in \mathbf{R}^+ : \forall x \in E \setminus \{x_0\}, \norm*{x - x_0} < \delta          \\
        \implies & \norm*{g\big(f(x)\big) - g\big(f(x_0)\big) - \Big(g'\big(f(x_0)\big) \circ f'(x_0)\Big) (x - x_0)}      \\
                 & < \varepsilon^2 \norm*{x - x_0} + \varepsilon \norm*{f'(x_0)(x - x_0)} + M \varepsilon \norm*{x - x_0}.
    \end{align*}
    Since \(\varepsilon\) is arbitrary, we conclude that
    \begin{align*}
                 & \forall \varepsilon \in \mathbf{R}^+, \exists\ \delta \in \mathbf{R}^+ : \forall x \in E \setminus \{x_0\}, \norm*{x - x_0} < \delta \\
        \implies & \norm*{g\big(f(x)\big) - g\big(f(x_0)\big) - \Big(g'\big(f(x_0)\big) \circ f'(x_0)\Big) (x - x_0)} < \varepsilon.
    \end{align*}
\end{proof}

\begin{note}
    As a corollary of the chain rule and Lemma \ref{6.1.16} (and Lemma \ref{6.1.13}), we see that
    \[
        D (g \circ f)(x_0) = D g\big(f(x_0)\big) \cdot D f(x_0);
    \]
    i.e., we can write the chain rule in terms of matrices and matrix multiplication, instead of in terms of linear transformations and composition.
\end{note}

\begin{example}\label{6.4.2}
    Let \(f : \mathbf{R}^n \to \mathbf{R}\) and \(g : \mathbf{R}^n \to \mathbf{R}\) be differentiable functions.
    We form the combined function \(h : \mathbf{R}^n \to \mathbf{R}^2\) by defining \(h(x) \coloneqq \big(f(x), g(x)\big)\).
    Now let \(k : \mathbf{R}^2 \to \mathbf{R}\) be the multiplication function \(k(a, b) \coloneqq ab\).

    We first show that
    \[
        D h(x_0) = \begin{pmatrix}
            \nabla f(x_0) \\
            \nabla g(x_0)
        \end{pmatrix}.
    \]
    Let \(x_0 \in \mathbf{R}^n\).
    Since
    \begin{align*}
         & \norm*{h(x) - h(x_0) - (x - x_0) D h(x_0)^\top}                                                                                  \\
         & = \norm*{\big(f(x), g(x)\big) - \big(f(x_0), g(x_0)\big) - \big((x - x_0) \nabla f(x_0)^\top, (x - x_0) \nabla g(x_0)^\top\big)} \\
         & = \norm*{\big(f(x) - f(x_0) - (x - x_0) \nabla f(x_0)^\top, g(x) - g(x_0) - (x - x_0) \nabla g(x_0)^\top\big)}                   \\
         & \leq \norm*{f(x) - f(x_0) - (x - x_0) \nabla f(x_0)^\top} + \norm*{g(x) - g(x_0) - (x - x_0) \nabla g(x_0)^\top}
    \end{align*}
    (note that the last line follow by Exercise \ref{ex 1.1.8}),
    by squeeze test we know that
    \begin{align*}
                 & \begin{cases}
                       \lim_{x \to x_0 ; x \in \mathbf{R}^n \setminus \{x_0\}} \frac{\norm*{f(x) - f(x_0) - f'(x_0)(x - x_0)}}{\norm*{x - x_0}} = 0 \\
                       \lim_{x \to x_0 ; x \in \mathbf{R}^n \setminus \{x_0\}} \frac{\norm*{g(x) - g(x_0) - g'(x_0)(x - x_0)}}{\norm*{x - x_0}} = 0
                   \end{cases}        \\
        \implies & \begin{cases}
                       \lim_{x \to x_0 ; x \in \mathbf{R}^n \setminus \{x_0\}} \frac{\norm*{f(x) - f(x_0) - (x - x_0) \nabla f(x_0)^\top}}{\norm*{x - x_0}} = 0 \\
                       \lim_{x \to x_0 ; x \in \mathbf{R}^n \setminus \{x_0\}} \frac{\norm*{g(x) - g(x_0) - (x - x_0) \nabla g(x_0)^\top}}{\norm*{x - x_0}} = 0
                   \end{cases}        \\
        \implies & \lim_{x \to x_0 ; x \in \mathbf{R}^n \setminus \{x_0\}} \frac{\norm*{h(x) - h(x_0) - (x - x_0) D h(x_0)^\top}}{\norm*{x - x_0}} = 0.
    \end{align*}
    Since \(x_0\) is arbitrary, we conclude that the identity is true.

    Now we show that
    \[
        D k(a, b) = (b, a).
    \]
    Let \((a, b) \in \mathbf{R}^2\).
    Observe that for any \((x, y) \in \mathbf{R}^2 \setminus \{(a, b)\}\), we have
    \begin{align*}
         & \frac{\norm*{k(x, y) - k(a, b) - \big((x, y) - (a, b)\big) (b, a)^\top}}{\norm*{(x, y) - (a, b)}} \\
         & = \frac{\norm*{xy - ab - xb - ay + ab + ba}}{\norm*{(x - a, y - b)}}                              \\
         & = \frac{\norm*{(x - a)(y - b)}}{\norm*{(x - a, y - b)}}                                           \\
         & = \sqrt{\frac{(x - a)^2 (y - b)^2}{(x - a)^2 + (y - b)^2}}                                        \\
         & \leq \sqrt{\frac{2 (x - a)^2 (y - b)^2}{(x - a)^2 + (y - b)^2}}                                   \\
         & \leq \sqrt{\frac{(x - a)^4 + 2 (x - a)^2 (y - b)^2 + (y - b)^4}{(x - a)^2 + (y - b)^2}}           \\
         & = \sqrt{\frac{\big((x - a)^2 + (y - b)^2\big)^2}{(x - a)^2 + (y - b)^2}}                          \\
         & = \sqrt{(x - a)^2 + (y - b)^2}.
    \end{align*}
    Since
    \begin{align*}
                 & \lim_{(x, y) \to (a, b) ; (x, y) \in \mathbf{R}^2 \setminus \{(a, b)\}} (x - a)^2 + (y - b)^2 = 0                                                                             \\
        \implies & \lim_{(x, y) \to (a, b) ; (x, y) \in \mathbf{R}^2 \setminus \{(a, b)\}} \sqrt{(x - a)^2 + (y - b)^2} = 0                                                                      \\
        \implies & \lim_{(x, y) \to (a, b) ; (x, y) \in \mathbf{R}^2 \setminus \{(a, b)\}} \frac{\norm*{k(x, y) - k(a, b) - \big((x, y) - (a, b)\big) (b, a)^\top}}{\norm*{(x, y) - (a, b)}} = 0
    \end{align*}
    (note that the last line follow by squeeze test),
    by Definition \ref{6.2.2} we know that the identity is true.

    By the chain rule, we thus see that
    \[
        D (k \circ h)(x_0) = \big(g(x_0), f(x_0)\big) \begin{pmatrix}
            \nabla f(x_0) \\
            \nabla g(x_0)
        \end{pmatrix} = g(x_0) \nabla f(x_0) + f(x_0) \nabla g(x_0).
    \]
    But \(k \circ h = fg\), and \(D (fg) = \nabla (fg)\).
    We have thus proven the \emph{product rule}
    \[
        \nabla (fg) = g \nabla f + f \nabla g.
    \]
    A similar argument gives the sum rule \(\nabla (f + g) = \nabla f + \nabla g\), or the difference rule \(\nabla (f - g) = \nabla f - \nabla g\), as well as the quotient rule (Exercise \ref{ex 6.4.4}).
\end{example}

\begin{note}
    We do record one further useful application of the chain rule.
    Let \(T : \mathbf{R}^n \to \mathbf{R}^m\) be a linear transformation.
    From Exercise \ref{ex 6.4.1} we observe that \(T\) is continuously differentiable at every point, and in fact \(T'(x) = T\) for every \(x\).
    (This equation may look a little strange, but perhaps it is easier to swallow if you view it in the form \(\frac{d}{dx} (Tx) = T\).)
    Thus, for any differentiable function \(f : E \to \mathbf{R}^n\), we see that \(T f : E \to \mathbf{R}^m\) is also differentiable, and hence by the chain rule
    \[
        (T f)'(x_0) = T\big(f'(x_0)\big).
    \]
    This is a generalization of the single-variable calculus rule \((cf)' = c(f')\) for constant scalars \(c\).
\end{note}

\begin{additional corollary}\label{ac 6.4.1}
If \(f : \mathbf{R}^n \to \mathbf{R}^m\) is some differentiable function, and \(x_j : \mathbf{R} \to \mathbf{R}\) are differentiable functions for each \(j = 1, \dots n\), then
\[
    \frac{d}{dt} f\big(x_1(t), x_2(t), \dots, x_n(t)\big) = \sum_{j = 1}^n x_j'(t) \frac{\partial f}{\partial x_j} \big(x_1(t), x_2(t), \dots, x_n(t)\big).
\]
\end{additional corollary}

\begin{proof}
    Let \(h : \mathbf{R} \to \mathbf{R}^n\) be the function
    \[
        \forall t \in \mathbf{R}, h(t) = \big(x_1(t), \dots, x_n(t)\big).
    \]
    We claim that \(h\) is differentiable on \(\mathbf{R}\), and
    \[
        \forall t \in \mathbf{R}, D h(t) = \big(x_1'(t), \dots, x_n'(t)\big)^\top.
    \]
    Let \(t_0 \in \mathbf{R}\).
    Since
    \begin{align*}
         & \frac{\norm*{h(t) - h(t_0) - (t - t_0) D_h(t_0)^\top}}{\norm*{t - t_0}}                                                                                   \\
         & = \frac{\norm*{\big(x_1(t), \dots, x_n(t)\big) - \big(x_1(t_0), \dots, x_n(t_0)\big) - (t - t_0) \big(x_1'(t), \dots, x_n'(t)\big)^\top}}{\abs*{t - t_0}} \\
         & = \frac{\norm*{\big(x_1(t) - x_1(t_0) - x_1'(t_0)(t - t_0), \dots, x_n(t) - x_n(t_0) - x_n'(t_0)(t - t_0)\big)}}{\abs*{t - t_0}}                          \\
         & \leq \frac{\sum_{i = 1}^n \abs*{\big(x_i(t) - x_i(t_0) - x_i'(t_0)(t - t_0)\big)}}{\abs*{t - t_0}}                                                        \\
         & = \sum_{i = 1}^n \frac{\abs*{\big(x_i(t) - x_i(t_0) - x_i'(t_0)(t - t_0)\big)}}{\abs*{t - t_0}},
    \end{align*}
    we know that
    \begin{align*}
                 & \forall 1 \leq i \leq n, \lim_{t \to t_0 ; t \in \mathbf{R} \setminus \{t_0\}} \frac{\abs*{x_i(t) - x_i(t_0) - x_i'(t_0)(x - x_0)}}{\abs*{t - t_0}} = 0                            \\
        \implies & \lim_{t \to t_0 ; t \in \mathbf{R} \setminus \{t_0\}} \sum_{i = 1}^n \frac{\abs*{x_i(t) - x_i(t_0) - x_i'(t_0)(x - x_0)}}{\abs*{t - t_0}} = 0           & \text{(by limit laws)}   \\
        \implies & \lim_{t \to t_0 ; t \in \mathbf{R} \setminus \{t_0\}} \frac{\norm*{h(t) - h(t_0) - (t - t_0) D_h(t_0)^\top}}{\norm*{t - t_0}}.                          & \text{(by squeeze test)}
    \end{align*}
    Since \(t_0\) is arbitrary, by Definition \ref{6.2.2} we know that
    \[
        \forall t \in \mathbf{R}, D h(t) = \big(x_1'(t), \dots, x_n'(t)\big)^\top.
    \]
    Using chain rule we have
    \begin{align*}
        \forall t \in \mathbf{R}, & D (f \circ h)(t)                                                                                                                           \\
                                  & = D f\big(h(t)\big) \cdot D h(t)                                                         & \text{(by Theorem \ref{6.4.1})}                 \\
                                  & = D f\big(x_1(t), \dots, x_n(t)\big) \cdot \big(x_1'(t), \dots, x_n'(t)\big)^\top                                                          \\
                                  & = D_{\big(x_1'(t), \dots, x_n'(t)\big)} f\big(x_1(t), \dots, x_n(t)\big)                 & \text{(by Lemma \ref{6.3.5})}                   \\
                                  & = \sum_{j = 1}^n x_j'(t) \frac{\partial f}{\partial x_j}\big(x_1(t), \dots, x_n(t)\big). & \text{(by Additional Corollary \ref{ac 6.3.2})}
    \end{align*}
\end{proof}

\exercisesection

\begin{exercise}\label{ex 6.4.1}
    Let \(T : \mathbf{R}^n \to \mathbf{R}^m\) be a linear transformation.
    Show that \(T\) is continuously differentiable at every point, and in fact \(T'(x) = T\) for every \(x\).
    What is \(D T\)?
\end{exercise}

\begin{proof}
    Let \(x_0 \in \mathbf{R}^n\).
    Since
    \begin{align*}
         & \lim_{x \to x_0 ; x \in \mathbf{R}^n \setminus \{x_0\}} \frac{\norm*{T(x) - T(x_0) - T(x - x_0)}}{\norm*{x - x_0}}                                           \\
         & = \lim_{x \to x_0 ; x \in \mathbf{R}^n \setminus \{x_0\}} \frac{\norm*{T(x) - T(x_0) - T(x) - T(x_0)}}{\norm*{x - x_0}} & \text{(by Definition \ref{6.1.6})} \\
         & = \lim_{x \to x_0 ; x \in \mathbf{R}^n \setminus \{x_0\}} \frac{\norm*{0_{\mathbf{R}^m}}}{\norm*{x - x_0}}                                                   \\
         & = 0
    \end{align*}
    and \(x_0\) is arbitrary, we conclude by Definition \ref{6.2.2} that
    \[
        \forall x \in \mathbf{R}^n, T'(x) = T.
    \]
    By Lemma \ref{6.1.13} we know that
    \[
        \forall x \in \mathbf{R}^n, D T(x) = \big(T(e_1)^\top, \dots, T(e_n)^\top\big).
    \]
\end{proof}

\begin{exercise}\label{ex 6.4.2}
    Let \(E\) be a subset of \(\mathbf{R}^n\).
    Prove that if a function \(f : E \to \mathbf{R}^m\) is differentiable at an interior point \(x_0\) of \(E\), then it is also continuous at \(x_0\).
\end{exercise}

\begin{proof}
    By Definition \ref{6.2.2} we have
    \begin{align*}
                 & \forall \varepsilon \in \mathbf{R}^+, \exists\ \delta_1 \in \mathbf{R}^+ : \forall x \in E \setminus \{x_0\},                             \\
                 & \bigg(\norm*{x - x_0} < \delta_1 \implies \frac{\norm*{f(x) - f(x_0) - f'(x_0)(x - x_0)}}{\norm*{x - x_0}} < \frac{\varepsilon}{2}\bigg)  \\
        \implies & \forall \varepsilon \in \mathbf{R}^+, \exists\ \delta_1 \in \mathbf{R}^+ : \forall x \in E \setminus \{x_0\},                             \\
                 & \bigg(\norm*{x - x_0} < \delta_1 \implies \norm*{f(x) - f(x_0) - f'(x_0)(x - x_0)} < \frac{\varepsilon}{2} \norm*{x - x_0}\bigg)          \\
        \implies & \forall \varepsilon \in \mathbf{R}^+, \exists\ \delta_1 \in \mathbf{R}^+ : \forall x \in E \setminus \{x_0\},                             \\
                 & \bigg(\norm*{x - x_0} < \delta_1 \implies \norm*{f(x) - f(x_0)} < \frac{\varepsilon}{2} \norm*{x - x_0} + \norm*{f'(x_0)(x - x_0)}\bigg)  \\
        \implies & \forall \varepsilon \in \mathbf{R}^+, \exists\ \delta_1 \in \mathbf{R}^+ : \forall x \in E,                                               \\
                 & \bigg(\norm*{x - x_0} < \delta_1 \implies \norm*{f(x) - f(x_0)} < \frac{\varepsilon}{2} \norm*{x - x_0} + \norm*{f'(x_0)(x - x_0)}\bigg).
    \end{align*}
    By Exercise \ref{ex 6.1.4} we know that every linear transformation is continuous.
    Thus we know that
    \begin{align*}
                 & \forall \varepsilon \in \mathbf{R}^+, \exists\ \delta_2 \in \mathbf{R}^+ : \forall x \in E,                                                    \\
                 & \bigg(\norm*{x - x_0} < \delta_2 \implies \norm*{f'(x_0)(x) - f'(x_0)(x_0)} < \frac{\varepsilon}{2}\bigg)                                      \\
        \implies & \forall \varepsilon \in \mathbf{R}^+, \exists\ \delta_2 \in \mathbf{R}^+ : \forall x \in E,                                                    \\
                 & \bigg(\norm*{x - x_0} < \delta_2 \implies \norm*{f'(x_0)(x - x_0)} < \frac{\varepsilon}{2}\bigg).         & \text{(by Definition \ref{6.1.6})}
    \end{align*}
    Let \(\delta = \min(\delta_1, \delta_2, 1)\).
    Then we have
    \begin{align*}
                 & \forall \varepsilon \in \mathbf{R}^+, \exists\ \delta \in \mathbf{R}^+ : \forall x \in E, \norm*{x - x_0} < \delta                                      \\
        \implies & \norm*{f(x) - f(x_0)} < \frac{\varepsilon}{2} \norm*{x - x_0} + \norm*{f'(x_0)(x - x_0)} < \frac{\varepsilon}{2} + \frac{\varepsilon}{2} = \varepsilon.
    \end{align*}
    Thus \(f\) is continuous at \(x_0\) from \((E, d_{l^2}|_{E \times E})\) to \((\mathbf{R}^m, d_{l^2}|_{\mathbf{R}^m \times \mathbf{R}^m})\).
\end{proof}

\begin{exercise}\label{ex 6.4.3}
    Prove Theorem \ref{6.4.1}.
\end{exercise}

\begin{proof}
    See Theorem \ref{6.4.1}.
\end{proof}

\begin{exercise}\label{ex 6.4.4}
    State and prove some version of the quotient rule for functions of several variables (i.e., functions of the form \(f : E \to \mathbf{R}\) for some subset \(E\) of \(\mathbf{R}^n\)).
    In other words, state a rule which gives a formula for the gradient of \(f / g\);
    compare your answer with Theorem 10.1.13(h) in Analysis I.
    Be sure to make clear what all your assumptions are.
\end{exercise}

\begin{proof}
    Let \(E \subseteq \mathbf{R}^n\) and let \(x_0\) be an interior point of \(E\).
    Let \(f : E \to \mathbf{R}\) and \(g : E \to \mathbf{R}\) be functions where \(g(x) \neq 0\) for all \(x \in E\).
    If \(f, g\) are differentiable at \(x_0\), then \(f / g\) is differentiable at \(x_0\), and
    \[
        \nabla \bigg(\frac{f}{g}\bigg)(x_0) = \frac{g(x_0) \nabla f(x_0) - f(x_0) \nabla g(x_0)}{\big(g(x_0)\big)^2}.
    \]
    If \(f, g\) are differentiable on \(E\), then \(f / g\) is differentiable on \(E\), and
    \[
        \nabla \bigg(\frac{f}{g}\bigg) = \frac{g \nabla f - f \nabla g}{g^2}.
    \]

    Let \(k : \mathbf{R} \times (\mathbf{R} \setminus \{0\}) \to \mathbf{R}\) be the function
    \[
        \forall (a, b) \in \mathbf{R} \times (\mathbf{R} \setminus \{0\}), k(a, b) = \frac{a}{b}.
    \]
    We claim that \(k\) is differentiable on \(\mathbf{R} \times (\mathbf{R} \setminus \{0\})\) and
    \[
        \forall (a, b) \in \mathbf{R} \times (\mathbf{R} \setminus \{0\}), D k(a, b) = \bigg(\frac{1}{b}, \frac{-a}{b^2}\bigg).
    \]
    Let \((a, b) \in \mathbf{R} \times (\mathbf{R} \setminus \{0\})\).
    Since for each \((x, y) \in \big(\mathbf{R} \times (\mathbf{R} \setminus \{0\})\big) \setminus \{(a, b)\}\), we have
    \begin{align*}
         & \frac{\norm*{k(x, y) - k(a, b) - \big((x, y) - (a, b)\big) \bigg(\frac{1}{b}, \frac{-a}{b^2}\bigg)^\top}}{\norm*{(x, y) - (a, b)}} \\
         & = \frac{\norm*{\frac{x}{y} - \frac{a}{b} - \frac{x - a}{b} + \frac{a(y - b)}{b^2}}}{\norm*{(x - a, y - b)}}                        \\
         & = \frac{\norm*{\frac{b^2 x - bxy + ay^2 - aby}{b^2y}}}{\norm*{(x - a, y - b)}}                                                     \\
         & = \frac{\norm*{\frac{(ay - bx)(y - b)}{b^2y}}}{\norm*{(x - a, y - b)}}                                                             \\
         & = \sqrt{\frac{(ay - bx)^2 (y - b)^2}{b^4 y^2 \big((x - a)^2 + (y - b)^2\big)}}                                                     \\
         & \leq \sqrt{\frac{(ay - bx)^2 \big((x - a)^2 + (y - b)^2\big)}{b^4 y^2 \big((x - a)^2 + (y - b)^2\big)}}                            \\
         & = \sqrt{\frac{(ay - bx)^2}{b^4 y^2}},
    \end{align*}
    we know that
    \begin{align*}
                 & \lim_{(x, y) \to (a, b) ; (x, y) \in \big(\mathbf{R} \times (\mathbf{R} \setminus \{0\})\big) \setminus \{(a, b)\}} (ay - bx)^2 = 0                                                                                                                  \\
        \implies & \lim_{(x, y) \to (a, b) ; (x, y) \in \big(\mathbf{R} \times (\mathbf{R} \setminus \{0\})\big) \setminus \{(a, b)\}} \frac{(ay - bx)^2}{b^4 y^2} = 0                                                                                                  \\
        \implies & \lim_{(x, y) \to (a, b) ; (x, y) \in \big(\mathbf{R} \times (\mathbf{R} \setminus \{0\})\big) \setminus \{(a, b)\}} \sqrt{\frac{(ay - bx)^2}{b^4 y^2}} = 0                                                                                           \\
        \implies & \lim_{(x, y) \to (a, b) ; (x, y) \in \big(\mathbf{R} \times (\mathbf{R} \setminus \{0\})\big) \setminus \{(a, b)\}} \frac{\norm*{k(x, y) - k(a, b) - \big((x, y) - (a, b)\big) \big(\frac{1}{b}, \frac{-a}{b^2}\big)^\top}}{\norm*{(x, y) - (a, b)}} \\
                 & = 0.
    \end{align*}
    (note that the last line was done by squeeze test)
    Since \((a, b)\) is arbitrary, by Definition \ref{6.2.2} we conclude that
    \[
        \forall (a, b) \in \mathbf{R} \times (\mathbf{R} \setminus \{0\}), D k(a, b) = \bigg(\frac{1}{b}, \frac{-a}{b^2}\bigg).
    \]

    Let \(h\) be the function in Example \ref{6.4.2}.
    Using chain rule we have
    \begin{align*}
        D (k \circ h)(x_0) & = D k\big(h(x_0)\big) \cdot D h(x_0)                                                     & \text{(by Theorem \ref{6.4.1})} \\
                           & = D k\big(f(x_0), g(x_0)\big) \cdot \begin{pmatrix}
                                                                     \nabla f(x_0) \\
                                                                     \nabla g(x_0)
                                                                 \end{pmatrix}                                      & \text{(by Example \ref{6.4.2})}   \\
                           & = \bigg(\frac{1}{g(x_0)}, \frac{-f(x_0)}{\big(g(x_0)\big)^2}\bigg) \cdot \begin{pmatrix}
                                                                                                          \nabla f(x_0) \\
                                                                                                          \nabla g(x_0)
                                                                                                      \end{pmatrix}                                    \\
                           & = \frac{\nabla f(x_0)}{g(x_0)} - \frac{f(x_0) \nabla g(x_0)}{\big(g(x_0)\big)^2}                                           \\
                           & = \frac{g(x_0) \nabla f(x_0) - f(x_0) \nabla g(x_0)}{\big(g(x_0)\big)^2}.
    \end{align*}
    But \(k \circ h = f / g\).
    Thus we have
    \[
        \nabla \bigg(\frac{f}{g}\bigg)(x_0) = \frac{g(x_0) \nabla f(x_0) - f(x_0) \nabla g(x_0)}{\big(g(x_0)\big)^2}.
    \]
\end{proof}

\begin{exercise}\label{ex 6.4.5}
    Let \(\vec{x} : \mathbf{R} \to \mathbf{R}^3\) be a differentiable function, and let \(r : \mathbf{R} \to \mathbf{R}\) be the function \(r(t) \coloneqq \norm*{\vec{x}(t)}\), where \(\norm*{\vec{x}}\) denotes the length of \(\vec{x}\) as measured in the usual \(l^2\) metric.
    Let \(t_0\) be a real number.
    Show that if \(r(t_0) \neq 0\), then \(r\) is differentiable at \(t_0\), and
    \[
        r'(t_0) = \frac{\vec{x}'(t_0) \cdot \vec{x}(t_0)}{r(t_0)}.
    \]
\end{exercise}

\begin{proof}
    Let \(f : \mathbf{R}^3 \to \mathbf{R}\) be the function
    \[
        \forall x \in \mathbf{R}^3, f(x) = \norm*{x}.
    \]
    Since
    \[
        \forall y \in \mathbf{R}^3 \setminus \{0_{\mathbf{R}^3}\}, \forall 1 \leq i \leq 3, \frac{\partial f}{\partial x_i}(y) = \frac{y_i}{\norm*{y}},
    \]
    and \(\frac{\partial f}{\partial x_i}\) is continuous on \(\mathbf{R}^3 \setminus \{0_{\mathbf{R}^3}\}\), by Theorem \ref{6.3.8} we know that
    \begin{align*}
        \forall y \in \mathbf{R}^3 \setminus \{0_{\mathbf{R}^3}\}, D f(y) & = \nabla f(y)                                                                                                            \\
                                                                          & = \bigg(\frac{\partial f}{\partial x_1}(y), \frac{\partial f}{\partial x_2}(y), \frac{\partial f}{\partial x_3}(y)\bigg) \\
                                                                          & = \bigg(\frac{y_1}{\norm*{y}}, \frac{y_2}{\norm*{y}}, \frac{y_3}{\norm*{y}}\bigg)                                        \\
                                                                          & = \frac{y}{\norm*{y}}.
    \end{align*}
    By chain rule (Theorem \ref{6.4.1}) we know that
    \begin{align*}
                 & \forall t \in \mathbf{R}, \vec{x}(t) \neq 0                                                                                         \\
        \implies & D (f \circ \vec{x})(t) = D f\big(\vec{x}(t)\big) \cdot D \vec{x}(t) = \frac{\vec{x}(t)}{\norm*{\vec{x}(t)}} \cdot \vec{x}'(t)^\top.
    \end{align*}
    Since \(r = f \circ \vec{x}\), we know that
    \begin{align*}
                 & \forall t \in \mathbf{R}, \vec{x}(t) \neq 0              \\
        \implies & \norm*{t} \neq 0                                         \\
        \implies & r(t) \neq 0                                              \\
        \implies & D r(t) = \frac{\vec{x}(t) \cdot \vec{x}'(t)^\top}{r(t)}.
    \end{align*}
\end{proof}