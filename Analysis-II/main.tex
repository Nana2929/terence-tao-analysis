% We use chapter structure.
\documentclass[12pt,oneside]{book}

%==============================================================================
% Preamble.
%==============================================================================

% Correctly showing characters outside ASCII.
\usepackage[T1]{fontenc}
% File is written and read with utf8 encoding.
\usepackage[utf8]{inputenc}
% Set paging layout.
\usepackage[margin=1.2in]{geometry}
% Including `amsfonts'.  Must be loaded before `mathtools'.
\usepackage{amssymb}
% Including `amsmath' and fixing bugs for `amsmath'.
\usepackage{mathtools}
% Must be loaded after `amsmath' and `mathtools'.
\usepackage{amsthm}
% Automatically adjust character spacing at margins.
\usepackage{microtype}
% Provide further utilities and fix bugs for `enumerate', `itemize' and
% `description'.
\usepackage{enumitem}
% Provide better quoting environment.
\usepackage{dirtytalk}
% Parsing list inside `\newcommand'.
\usepackage{listofitems}
% Nice looking if-then-else structure with comparison functionality.
\usepackage{ifthen}
% Automatically add hyperlinks to labels/refs.  Must be loaded after all
% packages above and before `cleveref'.  Recommend to use with `natbib' when
% you need bibtex.
\usepackage{hyperref}

\hypersetup{         % This macro come with `hyperref'.
	colorlinks=true, % Color hyperlinks.
	linkcolor=blue,  % Color local hyperlinks with blue.
	urlcolor=cyan,   % Color url links with cyan.
}

%------------------------------------------------------------------------------
% Define environments.
%------------------------------------------------------------------------------

% Text inside the body of theorem-like environments are set to Roman font.
% theorem-like environments share their counters, counters follow section and
% reset in every sections (except for axioms, axioms counters are reset in each
% chapter).  Exercises has their owned counter.  Notes do not use counter.
% See `amsthm' for details.
\theoremstyle{definition}
\newtheorem{axiom}{Axiom}[chapter]
\newtheorem{additional corollary}{Additional Corollary}[section]
\newtheorem{exercise}{Exercise}[section]
\newtheorem{theorem}{Theorem}[section]
\newtheorem{corollary}[theorem]{Corollary}
\newtheorem{definition}[theorem]{Definition}
\newtheorem{example}[theorem]{Example}
\newtheorem{lemma}[theorem]{Lemma}
\newtheorem{proposition}[theorem]{Proposition}
\newtheorem{remark}[theorem]{Remark}
\newtheorem*{note}{Note}

\theoremstyle{remark}
\newtheorem*{meta-proof}{Meta-proof}

% In `enumerate' enviroments, items' label are alphabets and surrounded by
% parentheses.  See `enumitem' for details.
\renewcommand{\labelenumi}{\textnormal{(}\alph{enumi}\textnormal{)}}

% Formatting exercises section.
\newcommand{\exercisesection}{
    \begin{center}
        --- Exercises ---
    \end{center}
}

%------------------------------------------------------------------------------
% Define operators and symbols.
%------------------------------------------------------------------------------

% Absolute value.
\DeclarePairedDelimiter{\absTmp}{\lvert}{\rvert}
\newcommand{\abs}[1]{\absTmp*{#1}}
% Ceiling.
\DeclarePairedDelimiter{\ceilTmp}{\lceil}{\rceil}
\newcommand{\ceil}[1]{\ceilTmp*{#1}}
% Floor.
\DeclarePairedDelimiter{\floorTmp}{\lfloor}{\rfloor}
\newcommand{\floor}[1]{\floorTmp*{#1}}
% Evaluate.
\DeclarePairedDelimiter{\evalTmp}{.}{\rvert}
\newcommand{\eval}[1]{\evalTmp*{#1}}
% Parenthese.
\DeclarePairedDelimiter{\paTmp}{\lparen}{\rparen}
\newcommand{\pa}[1]{\paTmp*{#1}}
% Bracket.
\DeclarePairedDelimiter{\brTmp}{\lbrack}{\rbrack}
\newcommand{\br}[1]{\brTmp*{#1}}
% Brace.
\DeclarePairedDelimiter{\BTmp}{\lbrace}{\rbrace}
\newcommand{\B}[1]{\BTmp*{#1}}
% Set.
\newcommand{\set}[1]{\B{#1}}
% Norm.
\DeclarePairedDelimiter\norm{\lVert}{\rVert}
% Inner product.
\DeclarePairedDelimiter\inner{\langle}{\rangle}

% Define common symbols.  See `amsmath' section 9.2 for details.

% Fields.
\newcommand{\field}[1]{\mathbf{#1}}
% Complex number field.
\newcommand{\C}{\field{C}}
% Natural number field.
\newcommand{\N}{\field{N}}
% Rational number field.
\newcommand{\Q}{\field{Q}}
% Real number field.
\newcommand{\R}{\field{R}}
% Integer number field.
\newcommand{\Z}{\field{Z}}

%==============================================================================
% Document.
%==============================================================================

\begin{document}

%------------------------------------------------------------------------------
% Front matters.
%------------------------------------------------------------------------------

\frontmatter

% Author informations.
\title{Analysis II}
\author{ProFatXuanAll}
\maketitle

% Table of contents.
\tableofcontents{}

%------------------------------------------------------------------------------
% Main matters.
%------------------------------------------------------------------------------

\mainmatter

% All chapters are in separated files.  We include them here.
\chapter{Metric spaces}\label{ch 1}

\section{Definitions and examples}\label{sec 1.1}

\begin{lemma}\label{1.1.1}
    Let \((x_n)_{n = m}^\infty\) be a sequence of real numbers, and let \(x\) be another real number.
    Then \((x_n)_{n = m}^\infty\) converges to \(x\) if and only if \(\lim_{n \to \infty} d(x_n, x) = 0\).
\end{lemma}

\begin{proof}
    We have \(\lim_{n \to \infty} x_n = x\) iff \(\forall\ \varepsilon > 0\), \(\exists\ N \geq m\) such that
    \[
        \forall\ n \geq N, \abs*{x_n - x} \leq \varepsilon.
    \]
    Since
    \[
        \abs*{x_n - x} \leq \varepsilon \iff \abs*{\abs*{x_n - x} - 0} \leq \varepsilon
    \]
    we have \(\lim_{n \to \infty} \abs*{x_n - x} = 0\).
    Thus \(\lim_{n \to \infty} x_n = x\) iff \(\lim_{n \to \infty} d(x_n, x) = 0\).
\end{proof}

\begin{note}
    One would now like to generalize this notion of convergence, so that one can take limits not just of sequences of real numbers, but also sequences of complex numbers, or sequences of vectors, or sequences of matrices, or sequences of functions, even sequences of sequences.
    One way to do this is to redefine the notion of convergence each time we deal with a new type of object.
    A more efficient way is to work \emph{abstractly}, defining a very general class of spaces - which includes such standard spaces as the real numbers, complex numbers, vectors, etc. - and define the notion of convergence on this entire class of spaces at once.
    (A \emph{space} is just the set of all objects of a certain type.
    Mathematically, there is not much distinction between a space and a set, except that spaces tend to have much more structure than what a random set would have.)
\end{note}

\begin{note}
    It turns out that there are two very useful classes of spaces which do the job.
    The first class is that of \emph{metric spaces}.
    There is a more general class of spaces, called \emph{topological spaces}.
\end{note}

\begin{definition}[Metric spaces]\label{1.1.2}
    A metric space \((X, d)\) is a space \(X\) of objects (called \emph{points}), together with a \emph{distance function} or \emph{metric} \(d : X \times X \to [0, +\infty)\), which associates to each pair \(x, y\) of points in \(X\) a non-negative real number \(d(x, y) \geq 0\).
    Furthermore, the metric must satisfy the following four axioms:
    \begin{enumerate}
        \item For any \(x \in X\), we have \(d(x, x) = 0\).
        \item (Positivity) For any distinct \(x, y \in X\), we have \(d(x, y) > 0\).
        \item (Symmetry) For any \(x, y \in X\), we have \(d(x, y) = d(y, x)\).
        \item (Triangle inequality) For any \(x, y, z \in X\), we have \(d(x, z) \leq d(x, y) + d(y, z)\).
    \end{enumerate}
\end{definition}

\begin{note}
    In many cases it will be clear what the metric \(d\) is, and we shall abbreviate \((X, d)\) as just \(X\).
\end{note}

\begin{remark}\label{1.1.3}
    The conditions (a) and (b) of Definition \ref{1.1.1} can be rephrased as follows:
    for any \(x, y \in X\) we have \(d(x, y) = 0\) if and only if \(x = y\).
\end{remark}

\begin{example}[The real line]\label{1.1.4}
    Let \(\mathbf{R}\) be the real numbers, and let \(d : \mathbf{R} \times \mathbf{R} \to [0, \infty)\) be the metric \(d(x, y) \coloneqq \abs*{x - y}\) mentioned earlier.
    Then \((\mathbf{R}, d)\) is a metric space.
    We refer to \(d\) as the \emph{standard metric} on \(\mathbf{R}\), and if we refer to \(\mathbf{R}\) as a metric space, we assume that the metric is given by the standard metric \(d\) unless otherwise specified.
\end{example}

\begin{example}[Induced metric spaces]\label{1.1.5}
    Let \((X, d)\) be any metric space, and let \(Y\) be a subset of \(X\).
    Then we can restrict the metric function \(d : X \times X \to [0, +\infty)\) to the subset \(Y \times Y\) of \(X \times X\) to create a restricted metric function \(d|_{Y \times Y} : Y \times Y \to [0, +\infty)\) of \(Y\);
    this is known as the metric on \(Y\) \emph{induced} by the metric \(d\) on \(X\).
    The pair \((Y, d|_{Y \times Y})\) is a metric space and is known the \emph{subspace} of \((X, d)\) induced by \(Y\).
    Thus for instance the metric on the real line in the Example \ref{1.1.4} induces a metric space structure on any subset of the reals, such as the integers \(Z\), or an interval \([a, b]\), etc.
\end{example}

\begin{example}[Euclidean spaces]\label{1.1.6}
    Let \(n \geq 1\) be a natural number, and let \(\mathbf{R}^n\) be the space of \(n\)-tuples of real numbers:
    \[
        \mathbf{R}^n = \{(x_1, x_2, \dots, x_n) : x_1, \dots, x_n \in \mathbf{R}\}.
    \]
    We define the \emph{Euclidean metric} (also called the \emph{\(l^2\) metric}) \(d_{l^2} : \mathbf{R}^n \times\mathbf{R}^n \to \mathbf{R}\) by
    \begin{align*}
        d_{l^2}((x_1, \dots, x_n), (y_1, \dots, y_n)) & \coloneqq \sqrt{(x_1 - y_1)^2 + \dots + (x_n - y_n)^2} \\
                                                      & = \bigg(\sum_{i = 1}^n (x_i - y_i)^2\bigg)^{1 / 2}.
    \end{align*}
\end{example}

\begin{note}
    Euclidean metric corresponds to the geometric distance between the two points \((x_1, x_2, \dots, x_n)\), \((y_1, y_2, \dots, y_n)\) as given by Pythagoras' theorem.
    While geometry does give some very important examples of metric spaces, it is possible to have metric spaces which have no obvious geometry whatsoever.
    The verification that \((\mathbf{R}^n, d)\) is indeed a metric space can be seen geometrically (for instance, the triangle inequality now asserts that the length of one side of a triangle is always less than or equal to the sum of the lengths of the other two sides), but can also be proven algebraically.
    We refer to \((\mathbf{R}^n , d_{l^2})\) as the \emph{Euclidean space} of \emph{dimension \(n\)}.
    Extending the convention from Example \ref{1.1.4}, if we refer to \(\mathbf{R}^n\) as a metric space, we assume that the metric is given by the Euclidean metric unless otherwise specified.
\end{note}

\begin{example}[Taxi-cab metric]\label{1.1.7}
    Again let \(n \geq 1\), and let \(\mathbf{R}^n\) be as before.
    But now we use a different metric \(d_{l^1}\), the so-called \emph{taxicab metric} (or \emph{\(l^1\) metric}), defined by
    \begin{align*}
        d_{l^1}((x_1, \dots, x_n), (y_1, \dots, y_n)) & \coloneqq \abs*{x_1 - y_1} + \dots + \abs*{x_n - y_n} \\
                                                      & = \sum_{i = 1}^n \abs*{x_i - y_i}.
    \end{align*}
\end{example}

\begin{note}
    This metric is called the taxi-cab metric, because it models the distance a taxi-cab would have to traverse to get from one point to another if the cab was only allowed to move in cardinal directions (north, south, east, west) and not diagonally.
    As such it is always at least as large as the Euclidean metric, which measures distance ``as the crow flies'', as it were.
    We claim that the space \((\mathbf{R}^n, d_{l^1})\) is also a metric space.
    The metrics are not quite the same, but we do have the inequalities
    \[
        d_{l^2}(x, y) \leq d_{l^1}(x, y) \leq \sqrt{n} d_{l^2}(x, y)
    \]
    for all \(x, y\).
\end{note}

\begin{remark}\label{1.1.8}
    The taxi-cab metric is useful in several places, for instance in the theory of error correcting codes.
    A string of \(n\) binary digits can be thought of as an element of \(\mathbf{R}^n\).
    The taxi-cab distance between two binary strings is then the number of bits in the two strings which do not match.
    The goal of error-correcting codes is to encode each piece of information (e.g., a letter of the alphabet) as a binary string in such a way that all the binary strings are as far away in the taxicab metric from each other as possible;
    this minimizes the chance that any distortion of the bits due to random noise can accidentally change one of the coded binary strings to another, and also maximizes the chance that any such distortion can be detected and correctly repaired.
\end{remark}

\begin{example}[Sup norm metric]\label{1.1.9}
    Again let \(n \geq 1\), and let \(\mathbf{R}^n\) be as before.
    But now we use a different metric \(d_{l^\infty}\), the so-called \emph{sup norm metric} (or \emph{\(l^\infty\) metric}), defined by
    \[
        d_{l^\infty} ((x_1, \dots, x_n), (y_1, \dots, y_n)) \coloneqq \sup\{\abs*{x_i - y_i} : 1 \leq i \leq n\}.
    \]
\end{example}

\begin{note}
    The space \((\mathbf{R}^n, d_{l^\infty})\) is also a metric space, and is related to the \(l^2\) metric by the inequalities
    \[
        \frac{1}{\sqrt{n}} d_{l^2}(x, y) \leq d_{l^\infty}(x, y) \leq d_{l^2}(x, y)
    \]
    for all \(x, y\).
\end{note}

\begin{remark}\label{1.1.10}
    The \(l^1\), \(l^2\), and \(l^\infty\) metrics are special cases of the more general \emph{\(l^p\) metrics}, where \(p \in [1, +\infty)\).
\end{remark}

\begin{example}[Discrete metric]\label{1.1.11}
    Let \(X\) be an arbitrary set (finite or infinite), and define the \emph{discrete metric} \(d_{\text{disc}}\) by setting \(d_{\text{disc}}(x, y) \coloneqq 0\) when \(x = y\), and \(d_{\text{disc}}(x, y) \coloneqq 1\) when \(x \neq y\).
    Thus, in this metric, all points are equally far apart.
    The space \((X, d_{\text{disc}})\) is a metric space.
    Thus every set \(X\) has at least one metric on it.
\end{example}

\setcounter{theorem}{13}
\begin{definition}[Convergence of sequences in metric spaces]\label{1.1.14}
    Let \(m\) be an integer, \((X, d)\) be a metric space and let \((x^{(n)})_{n = m}^\infty\) be a sequence of points in \(X\)
    (i.e., for every natural number \(n \geq m\), we assume that \(x^{(n)}\) is an element of \(X\)).
    Let \(x\) be a point in \(X\).
    We say that \emph{\((x^{(n)})_{n = m}^\infty\) converges to \(x\) with respect to the metric \(d\)}, if and only if the limit \(\lim_{n \to \infty} d(x^{(n)}, x)\) exists and is equal to \(0\).
    In other words, \((x^{(n)})_{n = m}^\infty\) converges to \(x\) with respect to \(d\) if and only if for every \(\varepsilon > 0\), there exists an \(N \geq m\) such that \(d(x^{(n)}, x) \leq \varepsilon\) for all \(n \geq N\).
\end{definition}

\begin{remark}\label{1.1.15}
    In view of Lemma \ref{1.1.1} we see that this definition generalizes our existing notion of convergence of sequences of real numbers.
    In many cases, it is obvious what the metric d is, and so we shall often just say ``\((x^{(n)})_{n = m}^\infty\) converges to \(x\)'' instead of ``\((x^{(n)})_{n = m}^\infty\) converges to \(x\) with respect to the metric \(d\)'' when there is no chance of confusion.
    We also sometimes write ``\(x^{(n)} \to x\) as \(n \to \infty\) instead.
\end{remark}

\begin{remark}\label{1.1.16}
    There is nothing special about the superscript \(n\) in the above definition;
    it is a dummy variable.
    Saying that \((x^{(n)})_{n = m}^\infty\) converges to \(x\) is exactly the same statement as saying that \((x^{(k)})_{k = m}^\infty\) converges to \(x\), for example;
    and sometimes it is convenient to change superscripts, for instance if the variable \(n\) is already being used for some other purpose.
    Similarly, it is not necessary for the sequence \(x^{(n)}\) to be denoted using the superscript \((n)\);
    the above definition is also valid for sequences \(x_n\), or functions \(f(n)\), or indeed of any expression which depends on \(n\) and takes values in \(X\).
    We see that the starting point \(m\) of the sequence is unimportant for the purposes of taking limits;
    if \((x^{(n)})_{n = m}^\infty\) converges to \(x\), then \((x^{(n)})_{n = m'}^\infty\) also converges to \(x\) for any \(m' \geq m\).
\end{remark}

\begin{note}
    The convergence of a sequence can depend on what metric one uses.
\end{note}

\setcounter{theorem}{17}
\begin{proposition}[Equivalence of \(l^1\), \(l^2\), \(l^\infty\)]\label{1.1.18}
    Let \(\mathbf{R}^n\) be a Euclidean space, and let \((x^{(k)})_{k = m}^\infty\) be a sequence of points in \(\mathbf{R}^n\).
    We write \(x^{(k)} = (x_1^{(k)}, x_2^{(k)}, \dots, x_n^{(k)})\), i.e., for \(j = 1, 2, \dots, n\), \(x_j \in \mathbf{R}\) is the \(j^{\text{th}}\) coordinate of \(x^{(k)} \in \mathbf{R}^n\).
    Let \(x = (x_1, \dots, x_n)\) be a point in \(\mathbf{R}^n\).
    Then the following four statements are equivalent:
    \begin{enumerate}
        \item \((x^{(k)})_{k = m}^\infty\) converges to \(x\) with respect to the Euclidean metric \(d_{l^2}\).
        \item \((x^{(k)})_{k = m}^\infty\) converges to \(x\) with respect to the taxi-cab metric \(d_{l^1}\).
        \item \((x^{(k)})_{k = m}^\infty\) converges to \(x\) with respect to the sup norm metric \(d_{l^\infty}\).
        \item For every \(1 \leq j \leq n\), the sequence \((x_j^{(k)})_{k = m}^\infty\) converges to \(x_j\).
              (Notice that this is a sequence of real numbers, not of points in \(\mathbf{R}^n\).)
    \end{enumerate}
\end{proposition}

\begin{proof}
    We have
    \begin{align*}
             & \lim_{k \to \infty} d_{l^2}(x^{(k)}, x) = 0                                                                     & \text{(by Definition \ref{1.1.14})} \\
        \iff & \lim_{k \to \infty} \sqrt{\sum_{j = 1}^n (x_j^{(k)} - x_j)^2} = 0                                               & \text{(by Example \ref{1.1.6})}     \\
        \iff & \lim_{k \to \infty} \bigg(\sum_{j = 1}^n (x_j^{(k)} - x_j)^2\bigg) = 0                                                                                \\
        \iff & \sum_{j = 1}^n \bigg(\lim_{k \to \infty} (x_j^{(k)} - x_j)^2\bigg) = 0                                                                                \\
        \iff & \forall\ j \in \{i \in \mathbf{N} : 1 \leq i \leq n\}, \lim_{k \to \infty} x_j^{(k)} - x_j = 0                                                        \\
        \iff & \forall\ j \in \{i \in \mathbf{N} : 1 \leq i \leq n\}, \lim_{k \to \infty} x_j^{(k)} = x_j                                                            \\
        \iff & \forall\ j \in \{i \in \mathbf{N} : 1 \leq i \leq n\}, \lim_{k \to \infty} \abs*{x_j^{(k)} - x_j} = 0           & \text{(by Lemma \ref{1.1.1})}       \\
        \iff & \sum_{j = 1}^n \bigg(\lim_{k \to \infty} \abs*{x_j^{(k)} - x_j}\bigg) = 0                                                                             \\
        \iff & \lim_{k \to \infty} \bigg(\sum_{j = 1}^n \abs*{x_j^{(k)} - x_j}\bigg) = 0                                                                             \\
        \iff & \lim_{k \to \infty} d_{l^1}(x^{(k)}, x) = 0                                                                     & \text{(by Example \ref{1.1.7})}     \\
        \iff & \lim_{k \to \infty} \sup\bigg\{\abs*{x_j^{(k)} - x_j} : j \in \{i \in \mathbf{N} : 1 \leq i \leq n\}\bigg\} = 0                                       \\
        \iff & \lim_{k \to \infty} d_{l^\infty}(x^{(k)}, x) = 0.                                                               & \text{(by Example \ref{1.1.9})}
    \end{align*}
\end{proof}

\begin{note}
    Because of the equivalence of Proposition \ref{1.1.18}(a), (b) and (c), we say that the Euclidean, taxicab, and sup norm metrics on \(\mathbf{R}^n\) are \emph{equivalent}.
    (There are infinite-dimensional analogues of the Euclidean, taxicab, and sup norm metrics which are \emph{not} equivalent.)
\end{note}

\begin{proposition}[Convergence in the discrete metric]\label{1.1.19}
    Let \(X\) be any set, and let \(d_{\text{disc}}\) be the discrete metric on \(X\).
    Let \((x^{(n)})_{n = m}^\infty\) be a sequence of points in \(X\), and let \(x\) be a point in \(X\).
    Then \((x^{(n)})_{n = m}^\infty\) converges to \(x\) with respect to the discrete metric \(d_{\text{disc}}\) if and only if there exists an \(N \geq m\) such that \(x^{(n)} = x\) for all \(n \geq N\).
\end{proposition}

\begin{proof}
    By Definition \ref{1.1.14} we know that \(\lim_{n \to \infty} d_{\text{disc}}(x^{(n)}, x) = 0\) iff \(\forall\ \varepsilon \in \mathbf{R}^+\), \(\exists\ N \in \mathbf{N}\) and \(N \geq m\) such that \(d(x^{(n)}, x) \leq \varepsilon\) for all \(n \geq N\).
    By Example \ref{1.1.11} we know that \(d_{\text{disc}}(x^{(n)}, x) \leq \varepsilon\) iff \(x^{(n)} = x\).
\end{proof}

\begin{proposition}[Uniqueness of limits]\label{1.1.20}
    Let \((X, d)\) be a metric space, and let \((x^{(n)})_{n = m}^\infty\) be a sequence in \(X\).
    Suppose that there are two points \(x, x' \in X\) such that \((x^{(n)})_{n = m}^\infty\) converges to \(x\) with respect to \(d\), and \((x^{(n)})_{n = m}^\infty\) also converges to \(x'\) with respect to \(d\).
    Then we have \(x = x'\).
\end{proposition}

\begin{proof}
    By Definition \ref{1.1.14} we have \(\lim_{n \to \infty} d(x^{(n)}, x) = 0\) and \(\lim_{n \to \infty} d(x^{(n)}, x') = 0\).
    So
    \begin{align*}
                 & \lim_{n \to \infty} d(x^{(n)}, x) + \lim_{n \to \infty} d(x^{(n)}, x') = 0                                            \\
        \implies & \lim_{n \to \infty} \bigg(d(x^{(n)}, x) + d(x^{(n)}, x')\bigg) = 0                                                    \\
        \implies & \lim_{n \to \infty} d(x, x') \leq 0                                        & \text{(by Definition \ref{1.1.2}(d))}    \\
        \implies & 0 \leq \lim_{n \to \infty} d(x, x') \leq 0                                 & \text{(by Definition \ref{1.1.2}(a)(b))} \\
        \implies & \lim_{n \to \infty} d(x, x') = 0                                                                                      \\
        \implies & d(x, x') = 0                                                                                                          \\
        \implies & x = x'.                                                                    & \text{(by Definition \ref{1.1.2}(a))}
    \end{align*}
\end{proof}

\begin{note}
    Because of Proposition \ref{1.1.20}, it is safe to introduce the following notation:
    if \((x^{(n)})_{n = m}^\infty\) converges to \(x\) in the metric \(d\), then we write \(d - \lim_{n \to \infty} x^{(n)} = x\), or simply \(\lim_{n \to \infty} x^{(n)} = x\) when there is no confusion as to what \(d\) is.
    The meaning of \(d - \lim_{n \to \infty} x^{(n)}\) can depend on what \(d\) is;
    however Proposition \ref{1.1.20} assures us that once \(d\) is fixed, there can be at most one value of \(d - \lim_{n \to \infty} x^{(n)}\).
    (Of course, it is still possible that this limit does not exist;
    some sequences are not convergent.)
\end{note}

\begin{remark}\label{1.1.21}
    It is possible for a sequence to converge to one point using one metric, and another point using a different metric, although such examples are usually quite artificial.
    Thus changing the metric on a space can greatly affect the nature of convergence (also called the \emph{topology}) on that space.
\end{remark}

\exercisesection

\begin{exercise}\label{ex 1.1.1}
    Prove Lemma \ref{1.1.1}.
\end{exercise}

\begin{proof}
    See Lemma \ref{1.1.1}.
\end{proof}

\begin{exercise}\label{ex 1.1.2}
    Show that the real line with the metric \(d(x, y) \coloneqq \abs*{x - y}\) is indeed a metric space.
\end{exercise}

\begin{proof}
    Let \(x, y, z \in \mathbf{R}\).
    For indentity:
    We have \(d(x, x) = \abs*{x - x} = 0\).
    For positivity:
    If \(x \neq y\), then \(d(x, y) = \abs*{x - y} > 0\).
    For symmetry:
    We have \(d(x, y) = \abs*{x - y} = \abs*{y - x} = d(y, x)\).
    For triangle inequality:
    We have \(d(x, z) = \abs*{x - z} = \abs*{x - y + y - z} \leq \abs*{x - y} + \abs*{y - z} = d(x, y) + d(y, z)\).
    Thus by Definition \ref{1.1.2} \((\mathbf{R}, d)\) is a metric space.
\end{proof}

\begin{exercise}\label{ex 1.1.3}
    Let \(X\) be a set, and let \(d : X \times X \to [0, \infty)\) be a function.
    \begin{enumerate}
        \item Give an example of a pair \((X, d)\) which obeys axioms (bcd) of Definition \ref{1.1.2}, but not (a).
        \item Give an example of a pair \((X, d)\) which obeys axioms (acd) of Definition \ref{1.1.2}, but not (b).
        \item Give an example of a pair \((X, d)\) which obeys axioms (abd) of Definition \ref{1.1.2}, but not (c).
        \item Give an example of a pair \((X, d)\) which obeys axioms (abc) of Definition \ref{1.1.2}, but not (d).
    \end{enumerate}
\end{exercise}

\begin{proof}{(a)}
    Let \(X = \mathbf{R}\), and let \(d(x, y) = 1\) for all \(x, y \in \mathbf{R}\).
    Then \((X, d)\) does not satisfy Definition \ref{1.1.2}(a) but (bcd).
\end{proof}

\begin{proof}{(b)}
    Let \(X = \mathbf{R}\) and let \(d(x, y) = 0\) for all \(x, y \in \mathbf{R}\).
    Then \((X, d)\) does not satisfy Definition \ref{1.1.2}(b) but (acd).
\end{proof}

\begin{proof}{(c)}
    Let \(X = \{1, 2\}\), let \(x, y \in \{1, 2\}\) and let \(d(x, y) = x^y\) if \(x \neq y\) and \(d(x, y) = 0\) if \(x = y\).
    Then \((X, d)\) does not satisfy Definition \ref{1.1.2}(c) but (abd).
\end{proof}

\begin{proof}{(d)}
    Let \(X = \mathbf{R}^+\), let \(x, y \in \mathbf{R}^+\) and let \(d(x, y) = \max(x, y)\) if \(x \neq y\) and \(d(x, y) = 0\) if \(x = y\).
    Then \((X, d)\) does not satisfy Definition \ref{1.1.2}(d) but (abc).
\end{proof}

\begin{exercise}\label{ex 1.1.4}
    Show that the pair \((Y, d|_{Y \times Y})\) defined in Example \ref{1.1.5} is indeed a metric space.
\end{exercise}

\begin{proof}
    Let \(x, y, z \in X\).
    Since \(Y \subseteq X\), we know that \(x, y, z \in X\).
    For indentity:
    We have \(d|_{Y \times Y}(x, x) = d(x, x) = 0\).
    For positivity:
    If \(x \neq y\), then \(d|_{Y \times Y}(x, y) = d(x, y) > 0\).
    For symmetry:
    We have \(d|_{Y \times Y}(x, y) = d(x, y) = d(y, x) = d|_{Y \times Y}(y, x)\).
    For triangle inequality:
    We have \(d|_{Y \times Y}(x, z) = d(x, z) \leq d(x, y) + d(y, z) = d|_{Y \times Y}(x, y) + d|_{Y \times Y}(y, z)\).
    Thus by Definition \ref{1.1.2} \((Y, d|_{Y \times Y})\) is a metric space.
\end{proof}

\begin{exercise}\label{ex 1.1.5}
    Let \(n \geq 1\), and let \(a_1, a_2, \dots, a_n\) and \(b_1, b_2, \dots, b_n\) be real numbers.
    Verify the identity
    \[
        \bigg(\sum_{i = 1}^n a_i b_i\bigg)^2 + \frac{1}{2} \sum_{i = 1}^n \sum_{j = 1}^n (a_i b_j - a_j b_i)^2 = \bigg(\sum_{i = 1}^n a_i^2\bigg) \bigg(\sum_{j = 1}^n b_j^2\bigg)
    \]
    and conclude the \emph{Cauchy-Schwarz inequality}
    \[
        \abs*{\sum_{i = 1}^n a_i b_i} \leq \bigg(\sum_{i = 1}^n a_i^2\bigg)^{1 / 2} \bigg(\sum_{j = 1}^n b_j^2\bigg)^{1 / 2}.
    \]
    Then use the Cauchy-Schwarz inequality to prove the \emph{triangle inequality}
    \[
        \bigg(\sum_{i = 1}^n (a_i + b_i)^2\bigg)^{1 / 2} \leq \bigg(\sum_{i = 1}^n a_i^2\bigg)^{1 / 2} + \bigg(\sum_{j = 1}^n b_j^2\bigg)^{1 / 2}.
    \]
\end{exercise}

\begin{proof}
    We first show the identity is true by induction on \(n\).
    For \(n = 0\), we have
    \[
        \bigg(\sum_{i = 1}^0 a_i b_i\bigg)^2 + \frac{1}{2} \sum_{i = 1}^0 \sum_{j = 1}^0 (a_i b_j - a_j b_i)^2 = 0
    \]
    and
    \[
        \bigg(\sum_{i = 1}^0 a_i^2\bigg) \bigg(\sum_{j = 1}^0 b_j^2\bigg) = 0
    \]
    so the base case holds.
    Suppose inductively that the identity is true for some \(n \geq 0\).
    Then for \(n + 1\), we have
    \begin{align*}
          & \bigg(\sum_{i = 1}^{n + 1} a_i b_i\bigg)^2 + \frac{1}{2} \sum_{i = 1}^{n + 1} \sum_{j = 1}^{n + 1} (a_i b_j - a_j b_i)^2                                                                                  \\
        = & \bigg(\sum_{i = 1}^n a_i b_i + a_{n + 1} b_{n + 1}\bigg)^2 + \frac{1}{2} \sum_{i = 1}^{n + 1} \sum_{j = 1}^{n + 1} (a_i b_j - a_j b_i)^2                                                                  \\
        = & \bigg(\sum_{i = 1}^n a_i b_i\bigg)^2 + 2 \bigg(\sum_{i = 1}^n a_i b_i\bigg) (a_{n + 1} b_{n + 1}) + (a_{n + 1} b_{n + 1})^2 + \frac{1}{2} \sum_{i = 1}^{n + 1} \sum_{j = 1}^{n + 1} (a_i b_j - a_j b_i)^2 \\
        = & \bigg(\sum_{i = 1}^n a_i b_i\bigg)^2 + 2 \bigg(\sum_{i = 1}^n a_i b_i\bigg) (a_{n + 1} b_{n + 1}) + (a_{n + 1} b_{n + 1})^2                                                                               \\
          & + \frac{1}{2} \sum_{i = 1}^{n + 1} \bigg(\sum_{j = 1}^n (a_i b_j - a_j b_i)^2 + (a_i b_{n + 1} - a_{n + 1} b_i)^2\bigg)                                                                                   \\
        = & \bigg(\sum_{i = 1}^n a_i b_i\bigg)^2 + 2 \bigg(\sum_{i = 1}^n a_i b_i\bigg) (a_{n + 1} b_{n + 1}) + (a_{n + 1} b_{n + 1})^2                                                                               \\
          & + \frac{1}{2} \sum_{i = 1}^{n + 1} \sum_{j = 1}^n (a_i b_j - a_j b_i)^2 + \frac{1}{2} \sum_{i = 1}^{n + 1} (a_i b_{n + 1} - a_{n + 1} b_i)^2                                                              \\
        = & \bigg(\sum_{i = 1}^n a_i b_i\bigg)^2 + 2 \bigg(\sum_{i = 1}^n a_i b_i\bigg) (a_{n + 1} b_{n + 1}) + (a_{n + 1} b_{n + 1})^2                                                                               \\
          & + \frac{1}{2} \sum_{i = 1}^n \sum_{j = 1}^n (a_i b_j - a_j b_i)^2 + \frac{1}{2} \sum_{j = 1}^n (a_{n + 1} b_j - a_j b_{n + 1})^2 + \frac{1}{2} \sum_{i = 1}^{n + 1} (a_i b_{n + 1} - a_{n + 1} b_i)^2     \\
        = & \bigg(\sum_{i = 1}^n a_i b_i\bigg)^2 + 2 \bigg(\sum_{i = 1}^n a_i b_i\bigg) (a_{n + 1} b_{n + 1}) + (a_{n + 1} b_{n + 1})^2                                                                               \\
          & + \frac{1}{2} \sum_{i = 1}^n \sum_{j = 1}^n (a_i b_j - a_j b_i)^2 + \frac{1}{2} \sum_{j = 1}^n (a_{n + 1} b_j - a_j b_{n + 1})^2 + \frac{1}{2} \sum_{i = 1}^n (a_i b_{n + 1} - a_{n + 1} b_i)^2           \\
        = & \bigg(\sum_{i = 1}^n a_i b_i\bigg)^2 + 2 \bigg(\sum_{i = 1}^n a_i b_i\bigg) (a_{n + 1} b_{n + 1}) + (a_{n + 1} b_{n + 1})^2                                                                               \\
          & + \frac{1}{2} \sum_{i = 1}^n \sum_{j = 1}^n (a_i b_j - a_j b_i)^2 + \sum_{i = 1}^n (a_{n + 1} b_i - a_i b_{n + 1})^2
    \end{align*}
    and
    \begin{align*}
          & \bigg(\sum_{i = 1}^{n + 1} a_i^2\bigg) \bigg(\sum_{j = 1}^{n + 1} b_j^2\bigg)                                                                                                             \\
        = & \bigg(\sum_{i = 1}^n a_i^2 + a_{n + 1}^2\bigg) \bigg(\sum_{j = 1}^n b_j^2 + b_{n + 1}^2\bigg)                                                                                             \\
        = & \bigg(\sum_{i = 1}^n a_i^2\bigg) \bigg(\sum_{j = 1}^n b_j^2\bigg) + a_{n + 1}^2 \bigg(\sum_{j = 1}^n b_j^2\bigg) + b_{n + 1}^2 \bigg(\sum_{i = 1}^n a_i^2\bigg) + (a_{n + 1} b_{n + 1})^2 \\
        = & \bigg(\sum_{i = 1}^n a_i^2\bigg) \bigg(\sum_{j = 1}^n b_j^2\bigg) + \bigg(\sum_{j = 1}^n a_{n + 1}^2 b_j^2\bigg) + \bigg(\sum_{i = 1}^n a_i^2 b_{n + 1}^2\bigg) + (a_{n + 1} b_{n + 1})^2 \\
        = & \bigg(\sum_{i = 1}^n a_i^2\bigg) \bigg(\sum_{j = 1}^n b_j^2\bigg) + \bigg(\sum_{j = 1}^n a_{n + 1}^2 b_j^2 - 2 a_{n + 1} b_j a_j b_{n + 1} + a_j^2 b_{n + 1}^2\bigg)                      \\
          & + \sum_{j = 1}^n 2 a_{n + 1} b_j a_j b_{n + 1} - \sum_{j = 1}^n a_j^2 b_{n + 1}^2 + \bigg(\sum_{i = 1}^n a_i^2 b_{n + 1}^2\bigg) + (a_{n + 1} b_{n + 1})^2                                \\
        = & \bigg(\sum_{i = 1}^n a_i^2\bigg) \bigg(\sum_{j = 1}^n b_j^2\bigg) + \bigg(\sum_{j = 1}^n (a_{n + 1} b_j - a_j b_{n + 1})^2\bigg)                                                          \\
          & + 2 \sum_{j = 1}^n \bigg(b_j a_j\bigg)(a_{n + 1} b_{n + 1}) + (a_{n + 1} b_{n + 1})^2.
    \end{align*}
    By induction hypothesis we thus have
    \[
        \bigg(\sum_{i = 1}^{n + 1} a_i b_i\bigg)^2 + \frac{1}{2} \sum_{i = 1}^{n + 1} \sum_{j = 1}^{n + 1} (a_i b_j - a_j b_i)^2 = \bigg(\sum_{i = 1}^{n + 1} a_i^2\bigg) \bigg(\sum_{j = 1}^{n + 1} b_j^2\bigg)
    \]
    and this close the induction.

    Next we show that Cauchy-Schwarz inequality is true.
    Since
    \[
        \bigg(\sum_{i = 1}^n a_i b_i\bigg)^2 + \frac{1}{2} \sum_{i = 1}^n \sum_{j = 1}^n (a_i b_j - a_j b_i)^2 = \bigg(\sum_{i = 1}^n a_i^2\bigg) \bigg(\sum_{j = 1}^n b_j^2\bigg),
    \]
    we have
    \[
        \bigg(\sum_{i = 1}^n a_i b_i\bigg)^2 \leq \bigg(\sum_{i = 1}^n a_i^2\bigg) \bigg(\sum_{j = 1}^n b_j^2\bigg)
    \]
    and thus
    \[
        \abs*{\sum_{i = 1}^n a_i b_i} \leq \bigg(\sum_{i = 1}^n a_i^2\bigg)^{1 / 2} \bigg(\sum_{j = 1}^n b_j^2\bigg)^{1 / 2}.
    \]

    Finally we show that \(d_{l^2}\) satisfy triangle inequality.
    We have
    \begin{align*}
         & \sum_{i = 1}^n (a_i + b_i)^2                                                                                                           \\
         & = \sum_{i = 1}^n a_i^2 + 2 a_i b_i + b_i^2                                                                                             \\
         & = \sum_{i = 1}^n a_i^2 + 2 \sum_{i = 1}^n a_i b_i + \sum_{i = 1}^n b_i^2                                                               \\
         & \leq \sum_{i = 1}^n a_i^2 + 2 \abs*{\sum_{i = 1}^n a_i b_i} + \sum_{i = 1}^n b_i^2                                                     \\
         & \leq \sum_{i = 1}^n a_i^2 + 2 \bigg(\sum_{i = 1}^n a_i^2\bigg)^{1 / 2} \bigg(\sum_{j = 1}^n b_j^2\bigg)^{1 / 2} + \sum_{i = 1}^n b_i^2 \\
         & = \bigg(\bigg(\sum_{i = 1}^n a_i^2\bigg)^{1 / 2} + \bigg(\sum_{j = 1}^n b_j^2\bigg)^{1 / 2}\bigg)^2
    \end{align*}
    and thus
    \[
        \bigg(\sum_{i = 1}^n (a_i + b_i)^2\bigg)^{1 / 2} \leq \bigg(\sum_{i = 1}^n a_i^2\bigg)^{1 / 2} + \bigg(\sum_{j = 1}^n b_j^2\bigg)^{1 / 2}.
    \]
\end{proof}

\begin{exercise}\label{ex 1.1.6}
    Show that \((\mathbf{R}^n, d_{l^2})\) in Example \ref{1.1.6} is indeed a metric space.
\end{exercise}

\begin{proof}
    Let \(n \in \mathbf{N}\), let \(x, y, z \in \mathbf{R}^n\), let \(i \in \mathbf{N}\) and \(1 \leq i \leq n\), and let \(x_i, y_i, z_i\) be the \(i^{\text{th}}\) coordinate of \(x, y, z\) respectively.
    For indentity:
    We have
    \[
        d_{l^2}(x, x) = \bigg(\sum_{i = 1}^n (x_i - x_i)^2\bigg)^{1 / 2} = 0.
    \]
    For positivity:
    If \(x \neq y\), then \(\exists\ j \in \mathbf{N}\) and \(1 \leq j \leq n\) such that \(x_j \neq y_j\).
    So
    \[
        d_{l^2}(x, y) = \bigg(\sum_{i = 1}^n (x_i - y_i)^2\bigg)^{1 / 2} \geq ((x_j - y_j)^2)^{1 / 2} > 0.
    \]
    For symmetry:
    We have
    \[
        d_{l^2}(x, y) = \bigg(\sum_{i = 1}^n (x_i - y_i)^2\bigg)^{1 / 2} = \bigg(\sum_{i = 1}^n (y_i - x_i)^2\bigg)^{1 / 2} = d_{l^2}(y, x).
    \]
    For triangle inequality:
    Let \(a_i = x_i - y_i\) and \(b_i = y_i - z_i\).
    Then we have
    \begin{align*}
        d_{l^2}(x, z) & = \bigg(\sum_{i = 1}^n (x_i - z_i)^2\bigg)^{1 / 2}                                                    \\
                      & = \bigg(\sum_{i = 1}^n (x_i - y_i + y_i - z_i)^2\bigg)^{1 / 2}                                        \\
                      & = \bigg(\sum_{i = 1}^n (a_i + b_i)^2\bigg)^{1 / 2}                                                    \\
                      & \leq \bigg(\sum_{i = 1}^n a_i^2\bigg)^{1 / 2} + \bigg(\sum_{i = 1}^n b_i^2\bigg)^{1 / 2}              \\
                      & = \bigg(\sum_{i = 1}^n (x_i - y_i)^2\bigg)^{1 / 2} + \bigg(\sum_{i = 1}^n (y_i - z_i)^2\bigg)^{1 / 2} \\
                      & = d_{l^2}(x, y) + d_{l^2}(y, z).
    \end{align*}
    Thus by Definition \ref{1.1.2} \((\mathbf{R}^n, d_{l^2})\) is a metric space.
\end{proof}

\begin{exercise}\label{ex 1.1.7}
    Show that \((\mathbf{R}^n, d_{l^1})\) in Example \ref{1.1.7} is indeed a metric space.
\end{exercise}

\begin{proof}
    Let \(n \in \mathbf{N}\), let \(x, y, z \in \mathbf{R}^n\), let \(i \in \mathbf{N}\) and \(1 \leq i \leq n\), and let \(x_i, y_i, z_i\) be the \(i^{\text{th}}\) coordinate of \(x, y, z\) respectively.
    For indentity:
    We have
    \[
        d_{l^1}(x, x) = \sum_{i = 1}^n \abs*{x_i - x_i} = 0.
    \]
    For positivity:
    If \(x \neq y\), then \(\exists\ j \in \mathbf{N}\) and \(1 \leq j \leq n\) such that \(x_j \neq y_j\).
    So
    \[
        d_{l^1}(x, y) = \sum_{i = 1}^n \abs*{x_i - y_i} \geq \abs*{x_j - y_j} > 0.
    \]
    For symmetry:
    We have
    \[
        d_{l^1}(x, y) = \sum_{i = 1}^n \abs*{x_i - y_i} = \sum_{i = 1}^n \abs*{y_i - x_i} = d_{l^1}(y, x).
    \]
    For triangle inequality:
    We have
    \begin{align*}
        d_{l^1}(x, z) & = \sum_{i = 1}^n \abs*{x_i - z_i}                         \\
                      & = \sum_{i = 1}^n \abs*{x_i - y_i + y_i - z_i}             \\
                      & \leq \sum_{i = 1}^n (\abs*{x_i - y_i} + \abs*{y_i - z_i}) \\
                      & = d_{l^1}(x, y) + d_{l^1}(y, z).
    \end{align*}
    Thus by Definition \ref{1.1.2} \((\mathbf{R}^n, d_{l^1})\) is a metric space.
\end{proof}

\begin{exercise}\label{ex 1.1.8}
    Prove the two inequalities
    \[
        d_{l^2}(x, y) \leq d_{l^1}(x, y) \leq \sqrt{n} d_{l^2}(x, y)
    \]
    for all \(x, y \in \mathbf{R}^n\).
\end{exercise}

\begin{proof}
    Let \(n \in \mathbf{N}\), let \(x, y \in \mathbf{R}^n\), let \(i \in \mathbf{N}\) and \(1 \leq i \leq n\), and let \(x_i, y_i\) be the \(i^{\text{th}}\) coordinate of \(x, y\) respectively.
    We have
    \begin{align*}
        (d_{l^2}(x, y))^2 & = \sum_{i = 1}^n (x_i - y_i)^2                                                                                                                                               \\
                          & = \sum_{i = 1}^n \abs*{x_i - y_i}^2                                                                                                                                          \\
                          & \leq \sum_{i = 1}^n \abs*{x_i - y_i}^2 + \sum_{i = 1}^n \Bigg(\abs*{x_i - y_i} \bigg(\sum_{j = 1}^{i - 1} \abs*{x_j - y_j} + \sum_{j = i + 1}^n \abs*{x_j - y_j}\bigg)\Bigg) \\
                          & = \sum_{i = 1}^n \abs*{x_i - y_i} \bigg(\sum_{j = 1}^{i - 1} \abs*{x_j - y_j} + \abs*{x_i - y_i} + \sum_{j = i + 1}^n \abs*{x_j - y_j}\bigg)                                 \\
                          & = \sum_{i = 1}^n \abs*{x_i - y_i} \bigg(\sum_{j = 1}^n \abs*{x_j - y_j}\bigg)                                                                                                \\
                          & = \bigg(\sum_{i = 1}^n \abs*{x_i - y_i}\bigg)^2                                                                                                                              \\
                          & = (d_{l^1}(x, y))^2
    \end{align*}
    and thus \(d_{l^2}(x, y) \leq d_{l^1}(x, y)\).
    Now let \(a_i = \abs*{x_i - y_i}\) and \(b_i = 1\).
    Then we have
    \begin{align*}
        d_{l^1}(x, y) & = \sum_{i = 1}^n \abs*{x_i - y_i}                                                                                                 \\
                      & = \abs*{\sum_{i = 1}^n \abs*{x_i - y_i}}                                                                                          \\
                      & = \abs*{\sum_{i = 1}^n a_i b_i}                                                                                                   \\
                      & \leq \bigg(\sum_{i = 1}^n a_i^2\bigg)^{1 / 2} \bigg(\sum_{i = 1}^n b_i^2\bigg)^{1 / 2}      & \text{(by Exercise \ref{ex 1.1.5})} \\
                      & = \bigg(\sum_{i = 1}^n \abs{x_i - y_i}^2\bigg)^{1 / 2} \bigg(\sum_{i = 1}^n 1\bigg)^{1 / 2}                                       \\
                      & = \sqrt{n} d_{l^2}(x, y).
    \end{align*}
    Combining the results we have
    \[
        d_{l^2}(x, y) \leq d_{l^1}(x, y) \leq \sqrt{n} d_{l^2}(x, y).
    \]
\end{proof}

\begin{exercise}\label{ex 1.1.9}
    Show that \((\mathbf{R}^n, d_{l^\infty})\) in Example \ref{1.1.9} is indeed a metric space.
\end{exercise}

\begin{proof}
    Let \(n \in \mathbf{N}\), let \(x, y, z \in \mathbf{R}^n\), let \(i \in \mathbf{N}\) and \(1 \leq i \leq n\), and let \(x_i, y_i, z_i\) be the \(i^{\text{th}}\) coordinate of \(x, y, z\) respectively.
    For indentity:
    We have
    \[
        d_{l^\infty}(x, x) = \sup_{i \in \mathbf{N} : 1 \leq i \leq n} \{\abs*{x_i - x_i}\} = 0.
    \]
    For positivity:
    If \(x \neq y\), then \(\exists\ j \in \mathbf{N}\) and \(1 \leq j \leq n\) such that \(x_j \neq y_j\).
    So
    \[
        d_{l^\infty}(x, y) = \sup_{i \in \mathbf{N} : 1 \leq i \leq n} \{\abs*{x_i - y_i}\} \geq \abs*{x_j - y_j} > 0.
    \]
    For symmetry:
    We have
    \[
        d_{l^\infty}(x, y) = \sup_{i \in \mathbf{N} : 1 \leq i \leq n} \{\abs*{x_i - y_i}\} = \sup_{i \in \mathbf{N} : 1 \leq i \leq n} \{\abs*{y_i - x_i}\} = d_{l^\infty}(y, x).
    \]
    For triangle inequality:
    We have
    \begin{align*}
        d_{l^\infty}(x, z) & = \sup_{i \in \mathbf{N} : 1 \leq i \leq n} \{\abs*{x_i - z_i}\}                                                                     \\
                           & = \sup_{i \in \mathbf{N} : 1 \leq i \leq n} \{\abs*{x_i - y_i + y_i - z_i}\}                                                         \\
                           & \leq \sup_{i \in \mathbf{N} : 1 \leq i \leq n} \{\abs*{x_i - y_i} + \abs*{y_i - z_i}\}                                               \\
                           & \leq \sup_{i \in \mathbf{N} : 1 \leq i \leq n} \{\abs*{x_i - y_i}\} + \sup_{i \in \mathbf{N} : 1 \leq i \leq n} \{\abs*{y_i - z_i}\} \\
                           & = d_{l^\infty}(x, y) + d_{l^\infty}(y, z).
    \end{align*}
    Thus by Definition \ref{1.1.2} \((\mathbf{R}^n, d_{l^\infty})\) is a metric space.
\end{proof}

\begin{exercise}\label{ex 1.1.10}
    Prove the two inequalities
    \[
        \frac{1}{\sqrt{n}} d_{l^2}(x, y) \leq d_{l^\infty}(x, y) \leq d_{l^2}(x, y)
    \]
    for all \(x, y \in \mathbf{R}^n\).
\end{exercise}

\begin{proof}
    Let \(n \in \mathbf{N}\), let \(x, y \in \mathbf{R}^n\), let \(i \in \mathbf{N}\) and \(1 \leq i \leq n\), and let \(x_i, y_i\) be the \(i^{\text{th}}\) coordinate of \(x, y\) respectively.
    Since
    \begin{align*}
        \frac{1}{\sqrt{n}} d_{l^2}(x, y) & = \frac{1}{\sqrt{n}} \bigg(\sum_{i = 1}^n (x_i - y_i)^2\bigg)^{1 / 2}                                                         \\
                                         & \leq \frac{1}{\sqrt{n}} \bigg(\sum_{i = 1}^n (\sup_{i \in \mathbf{N} : 1 \leq i \leq n} \{\abs*{x_i - y_i}\})^2\bigg)^{1 / 2} \\
                                         & = \frac{1}{\sqrt{n}} \bigg(n (\sup_{i \in \mathbf{N} : 1 \leq i \leq n} \{\abs*{x_i - y_i}\})^2\bigg)^{1 / 2}                 \\
                                         & = \sup_{i \in \mathbf{N} : 1 \leq i \leq n} \{\abs*{x_i - y_i}\}                                                              \\
                                         & = d_{l^\infty}(x, y)                                                                                                          \\
                                         & = ((\sup_{i \in \mathbf{N} : 1 \leq i \leq n} \{\abs*{x_i - y_i}\})^2)^{1 / 2}                                                \\
                                         & \leq \bigg(\sum_{i = 1}^n (x_i - y_i)^2\bigg)^{1 / 2}                                                                         \\
                                         & = d_{l^2}(x, y),
    \end{align*}
    we have
    \[
        \frac{1}{\sqrt{n}} d_{l^2}(x, y) \leq d_{l^\infty}(x, y) \leq d_{l^2}(x, y).
    \]
\end{proof}

\begin{exercise}\label{ex 1.1.11}
    Show that \((X, d_{\text{disc}})\) in Example \ref{1.1.11} is indeed a metric space.
\end{exercise}

\begin{proof}
    Let \(x, y, z \in X\).
    For indentity:
    We have
    \[
        d_{\text{disc}}(x, x) = 0.
    \]
    For positivity:
    If \(x \neq y\), then \(\exists\ j \in \mathbf{N}\) and \(1 \leq j \leq n\) such that \(x_j \neq y_j\).
    So
    \[
        d_{\text{disc}}(x, y) = 1 > 0.
    \]
    For symmetry:
    We have
    \[
        d_{\text{disc}}(x, y) = 0 = d_{\text{disc}}(y, x).
    \]
    if \(x = y\) and
    \[
        d_{\text{disc}}(x, y) = 1 = d_{\text{disc}}(y, x).
    \]
    if \(x \neq y\).
    For triangle inequality:
    We have
    \[
        d_{\text{disc}}(x, z) = 0 \leq d_{\text{disc}}(x, y) + d_{\text{disc}}(y, z)
    \]
    if \(x = z\) and
    \[
        d_{\text{disc}}(x, z) = 1 \leq d_{\text{disc}}(x, y) + d_{\text{disc}}(y, z)
    \]
    if \(x \neq z\).
    Thus by Definition \ref{1.1.2} \((X, d_{\text{disc}})\) is a metric space.
\end{proof}

\begin{exercise}\label{ex 1.1.12}
    Prove Proposition \ref{1.1.18}.
\end{exercise}

\begin{proof}
    See Proposition \ref{1.1.18}.
\end{proof}

\begin{exercise}\label{ex 1.1.13}
    Prove Proposition \ref{1.1.19}.
\end{exercise}

\begin{proof}
    See Proposition \ref{1.1.19}.
\end{proof}

\begin{exercise}\label{ex 1.1.14}
    Prove Proposition \ref{1.1.20}.
\end{exercise}

\begin{proof}
    See Proposition \ref{1.1.20}.
\end{proof}

\begin{exercise}\label{ex 1.1.15}
    Let
    \[
        X \coloneqq \bigg\{(a_n)_{n = 0}^\infty : \sum_{n = 0}^\infty \abs*{a_n} < \infty\bigg\}
    \]
    be the space of absolutely convergent sequences. Define the \(l^1\) and \(l^\infty\) metrics
    on this space by
    \begin{align*}
        d_{l^1}((a_n)_{n = 0}^\infty, (b_n)_{n = 0}^\infty)      & \coloneqq \sum_{n = 0}^\infty \abs*{a_n - b_n};     \\
        d_{l^\infty}((a_n)_{n = 0}^\infty, (b_n)_{n = 0}^\infty) & \coloneqq \sup_{n \in \mathbf{N}} \abs*{a_n - b_n}.
    \end{align*}
    Show that these are both metrics on \(X\), but show that there exist sequences \(x^{(1)}, x^{(2)}, \dots\) of elements of \(X\) (i.e., sequences of sequences) which are convergent with respect to the \(d_{l^\infty}\) metric but not with respect to the \(d_{l^1}\) metric.
    Conversely, show that any sequence which converges in the \(d_{l^1}\) metric automatically converges in the \(d_{l^\infty}\) metric.
\end{exercise}

\begin{proof}
    Let \((a_n)_{n = 0}^\infty, (b_n)_{n = 0}^\infty, (c_n)_{n = 0}^\infty \in X\) and let \(A = \sum_{n = 0}^\infty \abs*{a_n}, B = \sum_{n = 0}^\infty \abs*{b_n}, C = \sum_{n = 0}^\infty \abs*{c_n}\).
    We first show that \((X, d_{l^1})\) and \((X, d_{l^\infty})\) are metric spaces.
    For indentity:
    We have
    \[
        d_{l^1}((a_n)_{n = 0}^\infty, (a_n)_{n = 0}^\infty) = \sum_{n = 0}^\infty \abs*{a_n - a_n} = 0
    \]
    and
    \[
        d_{l^1}((a_n)_{n = 0}^\infty, (a_n)_{n = 0}^\infty) = \sup_{n \in \mathbf{N}} \abs*{a_n - a_n} = 0.
    \]
    For positivity:
    If \((a_n)_{n = 0}^\infty \neq (b_n)_{n = 0}^\infty\), then \(\exists\ N \in \mathbf{N}\) such that \(a_N \neq b_N\).
    So we have
    \[
        d_{l^1}((a_n)_{n = 0}^\infty, (b_n)_{n = 0}^\infty) = \sum_{n = 0}^\infty \abs*{a_n - b_n} \geq \abs*{a_N - b_N} > 0
    \]
    and
    \[
        d_{l^\infty}((a_n)_{n = 0}^\infty, (b_n)_{n = 0}^\infty) = \sup_{n \in \mathbf{N}} \abs*{a_n - b_n} \geq \abs*{a_N - b_N} > 0.
    \]
    For symmetry:
    We have
    \[
        d_{l^1}((a_n)_{n = 0}^\infty, (b_n)_{n = 0}^\infty) = \sum_{n = 0}^\infty \abs*{a_n - b_n} = \sum_{n = 0}^\infty \abs*{b_n - a_n} = d_{l^1}((b_n)_{n = 0}^\infty, (a_n)_{n = 0}^\infty)
    \]
    and
    \[
        d_{l^\infty}((a_n)_{n = 0}^\infty, (b_n)_{n = 0}^\infty) = \sup_{n \in \mathbf{N}} \abs*{a_n - b_n} = \sup_{n \in \mathbf{N}} \abs*{b_n - a_n} = d_{l^\infty}((b_n)_{n = 0}^\infty, (a_n)_{n = 0}^\infty).
    \]
    For triangle inequality:
    We have
    \begin{align*}
        d_{l^1}((a_n)_{n = 0}^\infty, (c_n)_{n = 0}^\infty) & = \sum_{n = 0}^\infty \abs*{a_n - c_n}                                                                      \\
                                                            & = \sum_{n = 0}^\infty \abs*{a_n - b_n + b_n - c_n}                                                          \\
                                                            & \leq \sum_{n = 0}^\infty \abs*{a_n - b_n} + \sum_{n = 0}^\infty \abs*{b_n - c_n}                            \\
                                                            & = d_{l^1}((a_n)_{n = 0}^\infty, (b_n)_{n = 0}^\infty) + d_{l^1}((b_n)_{n = 0}^\infty, (c_n)_{n = 0}^\infty)
    \end{align*}
    and
    \begin{align*}
        d_{l^\infty}((a_n)_{n = 0}^\infty, (c_n)_{n = 0}^\infty) & = \sup_{n \in \mathbf{N}} \abs*{a_n - c_n}                                                                             \\
                                                                 & = \sup_{n \in \mathbf{N}} \abs*{a_n - b_n + b_n - c_n}                                                                 \\
                                                                 & \leq \sup_{n \in \mathbf{N}} (\abs*{a_n - b_n} + \abs*{b_n - c_n})                                                     \\
                                                                 & \leq \sup_{n \in \mathbf{N}} \abs*{a_n - b_n} + \sup_{n \in \mathbf{N}} \abs*{b_n - c_n}                               \\
                                                                 & = d_{l^\infty}((a_n)_{n = 0}^\infty, (b_n)_{n = 0}^\infty) + d_{l^\infty}((b_n)_{n = 0}^\infty, (c_n)_{n = 0}^\infty).
    \end{align*}
    Thus by Definition \ref{1.1.2} \((X, d_{l^1})\) and \((X, d_{l^\infty})\) are metric spaces.

    Next we show that there exist sequences of elements of \(X\) which are convergent with respect to the \(d_{l^\infty}\) metric but not with respect to the \(d_{l^1}\) metric.
    Let \((x^{(k)})_{k = 1}^\infty\) be the sequence of sequence \((x_n^{(k)})_{n = 0}^\infty\) where
    \[
        x_n^{(k)} = \begin{cases}
            0                           & \text{if } n = 0    \\
            \frac{1}{n^2} + \frac{1}{k} & \text{if } n \leq k \\
            \frac{1}{n^2}               & \text{if } n > k.
        \end{cases}
    \]
    Then we know that \(\sum_{n = 0}^\infty \abs*{x_n^{(k)}}\) is absolutely convergent for all \(k \in \mathbf{N}\) and \(k \geq 1\).
    Let \((y_n)_{n = 0}^\infty\) be a sequence where
    \[
        y_n = \begin{cases}
            0             & \text{if } n = 0  \\
            \frac{1}{n^2} & \text{if } n > 0.
        \end{cases}
    \]
    Then \(\sum_{n = 0}^\infty \abs*{y_n}\) is also absolutely convergent.
    Now we show that \((x^{(k)})_{k = 1}^\infty\) converges to \((y_n)_{n = 0}^\infty\) with respect to \(d_{l^\infty}\).
    By Definition \ref{1.1.14} we need to show that
    \[
        \lim_{k \to \infty} d_{l^\infty}((x_n^{(k)})_{n = 0}^\infty, (y_n)_{n = 0}^\infty) = 0.
    \]
    We have
    \begin{align*}
        d_{l^\infty}((x_n^{(k)})_{n = 0}^\infty, (y_n)_{n = 0}^\infty) & = \sup_{n \in \mathbf{N}} \abs*{x_n^{(k)} - y_n} \\
                                                                       & = \sup \{0, \frac{1}{k}\}                        \\
                                                                       & = \frac{1}{k}
    \end{align*}
    and thus
    \[
        \lim_{k \to \infty} d_{l^\infty}((x_n^{(k)})_{n = 0}^\infty, (y_n)_{n = 0}^\infty) = \lim_{k \to \infty} \frac{1}{k} = 0.
    \]
    But we also have
    \begin{align*}
        d_{l^1}((x_n^{(k)})_{n = 0}^\infty, (y_n)_{n = 0}^\infty) & = \sum_{n = 0}^\infty \abs*{x_n^{(k)} - y_n} \\
                                                                  & = \sum_{n = 1}^k \frac{1}{k}                 \\
                                                                  & = 1
    \end{align*}
    and thus
    \[
        \lim_{k \to \infty} d_{l^1}((x_n^{(k)})_{n = 0}^\infty, (y_n)_{n = 0}^\infty) = \lim_{k \to \infty} 1 = 1 \neq 0
    \]
    and we conclude that \((x^{(k)})_{k = 1}^\infty\) converges to \((y_n)_{n = 0}^\infty\) with respect to \(d_{l^\infty}\) but not \(d_{l^1}\).

    Finally we show that any sequence which converges in the \(d_{l^1}\) metric automatically converges in the \(d_{l^\infty}\) metric.
    Suppose that \((x^{(k)})_{k = 1}^\infty\) converges to \((y_n)_{n = 0}^\infty\) with respect to \(d_{l^1}\).
    Since
    \begin{align*}
        d_{l^1}((x_n^{(k)})_{n = 0}^\infty, (y_n)_{n = 0}^\infty) & = \sum_{n = 0}^\infty \abs*{x_n^{(k)} - y_n}                      \\
                                                                  & \geq \sup_{n \in N} \abs*{x_n^{(k)} - y_n}                        \\
                                                                  & = d_{l^\infty}((x_n^{(k)})_{n = 0}^\infty, (y_n)_{n = 0}^\infty),
    \end{align*}
    by squeeze test we have
    \begin{align*}
                 & 0 \leq d_{l^\infty}((x_n^{(k)})_{n = 0}^\infty, (y_n)_{n = 0}^\infty) \leq d_{l^1}((x_n^{(k)})_{n = 0}^\infty, (y_n)_{n = 0}^\infty)                                                                     \\
        \implies & 0 = \lim_{k \to \infty} 0 \leq \lim_{k \to \infty} d_{l^\infty}((x_n^{(k)})_{n = 0}^\infty, (y_n)_{n = 0}^\infty) \leq \lim_{k \to \infty} d_{l^1}((x_n^{(k)})_{n = 0}^\infty, (y_n)_{n = 0}^\infty) = 0 \\
        \implies & \lim_{k \to \infty} d_{l^\infty}((x_n^{(k)})_{n = 0}^\infty, (y_n)_{n = 0}^\infty) = 0.
    \end{align*}
    Thus any sequence which converges in the \(d_{l^1}\) metric automatically converges in the \(d_{l^\infty}\) metric.
\end{proof}

\begin{exercise}\label{ex 1.1.16}
    Let \((x_n)_{n = 1}^\infty\) and \((y_n)_{n = 1}^\infty\) be two sequences in a metric space \((X, d)\).
    Suppose that \((x_n)_{n = 1}^\infty\) converges to a point \(x \in X\), and \((y_n)_{n = 1}^\infty\) converges to a point \(y \in X\).
    Show that \(\lim_{n \to \infty} d(x_n, y_n) = d(x, y)\).
\end{exercise}
\chapter{Continuous functions on metric spaces}\label{ch 2}

\section{Continuous functions}\label{sec 2.1}

\begin{definition}[Continuous functions]\label{2.1.1}
    Let \((X, d_X)\) be a metric space, and let \((Y, d_Y)\) be another metric space, and let \(f : X \to Y\) be a function.
    If \(x_0 \in X\), we say that \(f\) is \emph{continuous at \(x_0\)} iff for every \(\varepsilon > 0\), there exists a \(\delta > 0\) such that \(d_Y(f(x), f(x_0 )) < \varepsilon\) whenever \(d_X(x, x_0) < \delta\).
    We say that \(f\) is \emph{continuous} iff it is continuous at every point \(x \in X\).
\end{definition}

\begin{remark}\label{2.1.2}
    Continuous functions are also sometimes called \emph{continuous maps}.
    Mathematically, there is no distinction between the two terminologies.
\end{remark}

\begin{remark}\label{2.1.3}
    If \(f : X \to Y\) is continuous, and \(K\) is any subset of \(X\), then the restriction \(f|_K : K \to Y\) of \(f\) to \(K\) is also continuous.
\end{remark}

\begin{proof}
    Let \(x_0 \in K\).
    Suppose that \(f : X \to Y\) is continuous at \(x_0\) in \((X, d_X)\).
    Then we have
    \begin{align*}
                 & f : X \to Y \text{ is continuous at } x_0 \text{ in } (X, d_X)                                                                         \\
        \implies & \forall\ \varepsilon \in \mathbf{R}^+, \exists\ \delta \in \mathbf{R}^+ :                                                              \\
                 & \Big(\forall\ x \in X, d_X(x, x_0) < \delta \implies d_Y\big(f(x), f(x_0) < \varepsilon\big)\Big) & \text{(by definition \ref{2.1.1})} \\
        \implies & \forall\ \varepsilon \in \mathbf{R}^+, \exists\ \delta \in \mathbf{R}^+ :                                                              \\
                 & \Big(\forall\ x \in K, d_X(x, x_0) < \delta \implies d_Y\big(f(x), f(x_0) < \varepsilon\big)\Big) & (K \subseteq X)                    \\
        \implies & f|_K : K \to Y \text{ is continuous at } x_0 \text{ in } (K, d_X|_{K \times K}).                  & \text{(by definition \ref{2.1.1})}
    \end{align*}

    Now suppose that \(f : X \to Y\) is continuous on \(X\) in \((X, d_X)\).
    Then we have
    \begin{align*}
                 & f : X \to Y \text{ is continuous in } (X, d_X)                                                                              \\
        \implies & \forall\ x_0 \in X, f \text{ is continuous at } x_0 \text{ in } (X, d_X)               & \text{(by Definition \ref{2.1.1})} \\
        \implies & \forall\ x_0 \in K, f \text{ is continuous at } x_0 \text{ in } (K, d_X|_{K \times K}) & (K \subseteq X)                    \\
        \implies & f|_K : K \to Y \text{ is continuous in } (K, d_X|_{K \times K}).                       & \text{(by Definition \ref{2.1.1})}
    \end{align*}
\end{proof}

\begin{theorem}[Continuity preserves convergence]\label{2.1.4}
    Suppose that \((X, d_X)\) and \((Y, d_Y)\) are metric spaces.
    Let \(f : X \to Y\) be a function, and let \(x_0 \in X\) be a point in \(X\).
    Then the following three statements are logically equivalent:
    \begin{enumerate}
        \item \(f\) is continuous at \(x_0\).
        \item Whenever \((x^{(n)})_{n = 1}^\infty\) is a sequence in \(X\) which converges to \(x_0\) with respect to the metric \(d_X\), the sequence \(\big(f(x^{(n)})\big)_{n = 1}^\infty\) converges to \(f(x_0)\) with respect to the metric \(d_Y\).
        \item For every open set \(V \subseteq Y\) that contains \(f(x_0)\), there exists an open set \(U \subseteq X\) containing \(x_0\) such that \(f(U) \subseteq V\).
    \end{enumerate}
\end{theorem}

\begin{proof}
    We first show that statement (a) implies statement (b).
    Suppose that \(f : X \to Y\) is continuous at \(x_0\) in \((X, d_X)\).
    Then by Definition \ref{2.1.1} we have
    \[
        \forall\ \varepsilon \in \mathbf{R}^+, \exists\ d \in \mathbf{R}^+ : \Big(\forall\ x \in X, d_X(x, x_0) < \delta \implies d_Y\big(f(x), f(x_0)\big) < \varepsilon\Big).
    \]
    Now we choose \(\delta\) for each \(\varepsilon \in \mathbf{R}^+\) and denoted it as \(\delta_\varepsilon\).
    Let \((x^{(n)})_{n = 1}^\infty\) be a sequence in \(X\) such that \(\lim_{n \to \infty} d_X(x^{(n)}, x_0) = 0\).
    Then we have
    \begin{align*}
                 & \lim_{n \to \infty} d_X(x^{(n)}, x_0) = 0                                                                                                                                   \\
        \implies & \forall\ \delta \in \mathbf{R}^+, \exists\ N \in \mathbf{Z}^+ : \forall\ n \geq N, d_X(x^{(n)}, x_0) \leq \delta                      & \text{(by Definition \ref{1.1.14})} \\
        \implies & \forall\ \varepsilon \in \mathbf{R}^+, \exists\ \delta_\varepsilon \in \mathbf{R}^+ :                                                                                       \\
                 & \bigg(\exists\ N \in \mathbf{Z}^+ : \forall\ n \geq N, d_X(x^{(n)}, x_0) \leq \frac{\delta_\varepsilon}{2} < \delta_\varepsilon\bigg)                                       \\
        \implies & \forall\ \varepsilon \in \mathbf{R}^+, \exists\ \delta_\varepsilon \in \mathbf{R}^+ :                                                                                       \\
                 & \bigg(\exists\ N \in \mathbf{Z}^+ : \forall\ n \geq N, d_Y\big(f(x^{(n)}), f(x_0)\big) < \varepsilon\bigg)                            & \text{(by hypothesis)}              \\
        \implies & \lim_{n \to \infty} d_Y\big(f(x^{(n)}), f(x_0)\big) = 0.                                                                              & \text{(by Definition \ref{1.1.14})}
    \end{align*}
    Since \((x^{(n)})_{n = 1}^\infty\) is arbitrary, we know that statement (a) implies statement (b).

    Next we show that statement (b) implies statement (c).
    Suppose that
    \[
        \forall\ (x^{(n)})_{n = 1}^\infty \text{ in } X, \lim_{n \to \infty} d_X(x^{(n)}, x_0) = 0 \implies \lim_{n \to \infty} d_Y\big(f(x^{(n)}), f(x_0)\big) = 0.
    \]
    Let \(V\) be an open set in \((Y, d_Y)\) such that \(f(x_0) \in V\).
    Then we have
    \begin{align*}
                 & V \text{ is open in } (Y, d_Y)                                                                                                           \\
        \implies & V = \text{int}_{(Y, d_Y)}(V)                                                                   & \text{(by Proposition \ref{1.2.15}(a))} \\
        \implies & \exists\ \varepsilon \in \mathbf{R}^+ : B_{(Y, d_Y)}\big(f(x_0), \varepsilon\big) \subseteq V. & \text{(by Definition \ref{1.2.5})}
    \end{align*}
    Now we choose one \(\varepsilon\) and define \(V_\varepsilon = B_{(Y, d_Y)}\big(f(x_0), \varepsilon\big)\).
    By Remark \ref{1.2.4} we know that \(f(x_0) \in V_\varepsilon\), thus we have \(x_0 \in f^{-1}(V_\varepsilon)\) and \(f^{-1}(V_\varepsilon) \neq \emptyset\).
    Now we claim that
    \[
        \exists\ \delta \in \mathbf{R}^+ : B_{(X, d_X)}(x_0, \delta) \subseteq f^{-1}(V_\varepsilon).
    \]
    Suppose the claim is false.
    Then we have
    \begin{align*}
                 & \forall\ \delta \in \mathbf{R}^+, B_{(X, d_X)}(x_0, \delta) \not\subseteq f^{-1}(V_\varepsilon)                                                 \\
        \implies & \forall\ \delta \in \mathbf{R}^+, B_{(X, d_X)}(x_0, \delta) \setminus f^{-1}(V_\varepsilon) \neq \emptyset                                      \\
        \implies & \forall\ \delta \in \mathbf{R}^+, \exists\ x \in X :                                                                                            \\
                 & (d_X(x, x_0) < \delta) \land \Big(d_Y\big(f(x), f(x_0)\big) \geq \varepsilon\Big)                          & \text{(by Definition \ref{1.2.1})} \\
        \implies & \forall\ n \in \mathbf{Z}^+, \exists\ x \in X :                                                                                                 \\
                 & (d_X(x, x_0) < \frac{1}{n}) \land \Big(d_Y\big(f(x), f(x_0)\big) \geq \varepsilon\Big).
    \end{align*}
    For each \(n \in \mathbf{Z}^+\), we define \(X_n = B_{(X, d_X)}(x_0, \frac{1}{n}) \setminus f^{-1}(V_\varepsilon)\).
    We choose one sequence \((x^{(n)})_{n = 1}^\infty \in \prod_{n \in \mathbf{Z}^+} X_n\).
    Then we have
    \begin{align*}
                 & \forall\ n \in \mathbf{Z}^+, d_X(x^{(n)}, x_0) < \frac{1}{n}                                                                                       \\
        \implies & \lim_{n \to \infty} d_X(x^{(n)}, x_0) = 0                                                                                                          \\
        \implies & \lim_{n \to \infty} d_Y\big(f(x^{(n)}), f(x_0)\big) = 0                                                                                            \\
        \implies & \exists\ N \in \mathbf{Z}^+ : \forall\ n \geq N, d_Y\big(f(x^{(n)}), f(x_0)\big) \leq \frac{\varepsilon}{2}. & \text{(by Definition \ref{1.1.14})}
    \end{align*}
    But by the definition of \((x^{(n)})_{n = 1}^\infty\) we know that
    \[
        \forall\ n \in \mathbf{Z}^+, d_Y\big(f(x^{(n)}), f(x_0)\big) \geq \varepsilon,
    \]
    a contradiction.
    Thus the claim is true.
    Using the claim we choose one \(\delta\) and define \(U = B_{(X, d_X)}(x_0, \delta)\).
    By Proposition \ref{1.2.15}(c) we know that \(U\) is open in \((X, d_X)\).
    By Remark \ref{1.2.4} we know that \(x_0 \in U\).
    Since \(U \subseteq f^{-1}(V_\varepsilon) \subseteq X\), we know that \(f(U) \subseteq V\).

    Finally we show that statement (c) implies statement (a).
    Suppose that
    \begin{align*}
                 & \forall\ V \subseteq Y, \big(V \text{ is open in } (Y, d_Y)\big) \land \big(f(x_0) \in V\big)                         \\
        \implies & \exists\ U \subseteq X : \big(U \text{ is open in } (X, d_X)\big) \land (x_0 \in U) \land \big(f(U) \subseteq V\big).
    \end{align*}
    Let \(\varepsilon \in \mathbf{R}^+\).
    By Proposition \ref{1.2.15}(c) we know that \(B_{(Y, d_Y)}\big(f(x_0), \varepsilon\big)\) is open in \((Y, d_Y)\).
    By hypothesis we know that
    \[
        \exists\ U \subseteq X : \big(U \text{ is open in } (X, d_X)\big) \land (x_0 \in U) \land \Big(f(U) \subseteq B_{(Y, d_Y)}\big(f(x_0), \varepsilon\big)\Big).
    \]
    Now we choose one such \(U\).
    Since \(U\) is open in \((X, d_X)\) and \(x_0 \in U\), we have
    \begin{align*}
                 & x_0 \in \text{int}_{(X, d_X)}(U)                                                                                            & \text{(by Proposition \ref{1.2.15}(a))} \\
        \implies & \exists\ \delta \in \mathbf{R}^+ : B_{(X, d_X)}(x_0, \delta) \subseteq U                                                    & \text{(by Definition \ref{1.2.5})}      \\
        \implies & \exists\ \delta \in \mathbf{R}^+ : f\big(B_{(X, d_X)}(x_0, \delta)\big) \subseteq f(U)                                                                                \\
        \implies & \exists\ \delta \in \mathbf{R}^+ : f\big(B_{(X, d_X)}(x_0, \delta)\big) \subseteq B_{(Y, d_Y)}\big(f(x_0), \varepsilon\big)                                           \\
        \implies & \exists\ \delta \in \mathbf{R}^+ :                                                                                                                                    \\
                 & \Big(\forall\ x \in X, d_X(x, x_0) < \delta \implies d_Y\big(f(x), f(x_0)\big) < \varepsilon\Big).                          & \text{(by Definition \ref{1.2.1})}
    \end{align*}
    Since \(\varepsilon\) is arbitrary, we have
    \[
        \forall\ \varepsilon \in \mathbf{R}^+, \exists\ \delta \in \mathbf{R}^+ : \Big(\forall\ x \in X, d_X(x, x_0) < \delta \implies d_Y\big(f(x), f(x_0)\big) < \varepsilon\Big).
    \]
    Thus by Definition \ref{2.1.1} \(f\) is continuous at \(x_0\) in \((X, d_X)\).
    We conclude that statements (a)(b)(c) are equivalent.
\end{proof}

\begin{theorem}\label{2.1.5}
    Let \((X, d_X)\) be a metric space, and let \((Y, d_Y)\) be another metric space.
    Let \(f : X \to Y\) be a function.
    Then the following four statements are equivalent:
    \begin{enumerate}
        \item \(f\) is continuous.
        \item Whenever \((x^{(n)})_{n = 1}^\infty\) is a sequence in \(X\) which converges to some point \(x_0 \in X\) with respect to the metric \(d_X\), the sequence \(\big(f(x^{(n)})\big)_{n = 1}^\infty\) converges to \(f(x_0)\) with respect to the metric \(d_Y\).
        \item Whenever \(V\) is an open set in \(Y\), the set \(f^{-1}(V) \coloneqq \{x \in X : f(x) \in V\}\) is an open set in \(X\).
        \item Whenever \(F\) is a closed set in \(Y\), the set \(f^{-1}(F) \coloneqq \{x \in X : f(x) \in F\}\) is a closed set in \(X\).
    \end{enumerate}
\end{theorem}

\begin{proof}
    We first show that statements (a)(b) are equivalent.
    \begin{align*}
             & f \text{ is continuous in } (X, d_X)                                                                             \\
        \iff & \forall\ x_0 \in X, f \text{ is continuous at } x_0 \text{ in } (X, d_X) & \text{(by Definition \ref{2.1.1})}    \\
        \iff & \forall\ x_0 \in X, \Big(\lim_{n \to \infty} d_X(x^{(n)}, x_0) = 0                                               \\
             & \implies \lim_{n \to \infty} d_Y\big(f(x^{(n)}), f(x_0)\big) = 0\Big).   & \text{(by Theorem \ref{2.1.4}(a)(b))}
    \end{align*}

    Next we show that statements (a) implies statement (c).
    Suppose that \(f\) is continuous in \((X, d_X)\).
    Let \(V\) be an open set in \((Y, d_Y)\).
    Then we have
    \begin{align*}
                 & f \text{ is continuous in } (X, d_X)                                                                                                       \\
        \implies & f \text{ is continuous in } (f^{-1}(V), d_X)                                                     & \text{(by Remark \ref{2.1.3})}          \\
        \implies & \forall\ x_0 \in f^{-1}(V), f \text{ is continuous at } x_0 \text{ in } (f^{-1}(V), d_X)         & \text{(by Definition \ref{2.1.1})}      \\
        \implies & \forall\ x_0 \in f^{-1}(V), \exists\ U \subseteq X :                                                                                       \\
                 & \big(U \text{ is open in } (X, d_X)\big) \land (x_0 \in U) \land \big(f(U) \subseteq V\big)      & \text{(by Theorem \ref{2.1.4}(a)(c))}   \\
        \implies & \forall\ x_0 \in f^{-1}(V), \exists\ U \subseteq X :                                                                                       \\
                 & \big(U \text{ is open in } (X, d_X)\big) \land (x_0 \in U) \land \big(U \subseteq f^{-1}(V)\big)                                           \\
        \implies & \forall\ x_0 \in f^{-1}(V), \exists\ U \subseteq X :                                                                                       \\
                 & \big(\exists\ r \in \mathbf{R}^+ : B_{(X, d_X)}(x_0, r) \subseteq U \subseteq f^{-1}(V)\big)     & \text{(by Proposition \ref{1.2.15}(a))} \\
        \implies & f^{(-1)}(V) \text{ is open in } (X, d_X).                                                        & \text{(by Proposition \ref{1.2.15}(a))}
    \end{align*}
    Since \(V\) is arbitrary, we know that statement (a) implies statement (c).

    Next we show that statements (c) implies statement (a).
    Suppose that
    \[
        \forall\ V \subseteq Y, V \text{ is open in } (Y, d_Y) \implies f^{-1}(V) \text{ is open in } (X, d_X).
    \]
    Let \(x_0 \in X\).
    Then we have
    \begin{align*}
                 & \forall\ V \subseteq Y, \big(V \text{ is open in } (Y, d_Y)\big) \land \big(f(x_0) \in V\big)                          \\
        \implies & \big(f^{-1}(V) \text{ is open in } (X, d_X)\big) \land \big(x_0 \in f^{-1}(V)\big)            & \text{(by hypothesis)}
    \end{align*}
    and by Theorem \ref{2.1.4}(a)(c) we know that \(f\) is continuous at \(x_0\) in \((X, d_X)\).
    Since \(x_0\) is arbitrary, we know that \(f\) is continuous in \((X, d_X)\).
    Thus statements (c) implies statement (a) and from the proof above we conclude that statements (a)(c) are equivalent.

    Next we show that statements (c) implies statement (d).
    Suppose that
    \[
        \forall\ V \subseteq Y, V \text{ is open in } (Y, d_Y) \implies f^{-1}(V) \text{ is open in } (X, d_X).
    \]
    Let \(F\) be an closed set in \((Y, d_Y)\).
    Then we have
    \begin{align*}
                 & F \text{ is closed in } (Y, d_Y)                                                                                          \\
        \implies & Y \setminus F \text{ is open in } (Y, d_Y)                                      & \text{(by Proposition \ref{1.2.15}(e))} \\
        \implies & f^{-1}(Y \setminus F) \text{ is open in } (X, d_X)                              & \text{(by hypothesis)}                  \\
        \implies & X \setminus f^{-1}(Y \setminus F) \text{ is closed in } (X, d_X)                & \text{(by Proposition \ref{1.2.15}(e))} \\
        \implies & X \setminus \{x \in X : f(x) \in Y \setminus F\} \text{ is closed in } (X, d_X)                                           \\
        \implies & \{x \in X : f(x) \in F\} \text{ is closed in } (X, d_X)                                                                   \\
        \implies & f^{-1}(F) \text{ is closed in } (X, d_X).
    \end{align*}
    Since \(F\) is arbitrary, we know that statement (c) implies statement (d).

    Finally we show that statements (d) implies statement (c).
    Suppose that
    \[
        \forall\ F \subseteq Y, F \text{ is closed in } (Y, d_Y) \implies f^{-1}(F) \text{ is closed in } (X, d_X).
    \]
    Let \(V\) be an open set in \((Y, d_Y)\).
    Then we have
    \begin{align*}
                 & V \text{ is open in } (Y, d_Y)                                                                                          \\
        \implies & Y \setminus V \text{ is closed in } (Y, d_Y)                                  & \text{(by Proposition \ref{1.2.15}(e))} \\
        \implies & f^{-1}(Y \setminus V) \text{ is closed in } (X, d_X)                          & \text{(by hypothesis)}                  \\
        \implies & X \setminus f^{-1}(Y \setminus V) \text{ is open in } (X, d_X)                & \text{(by Proposition \ref{1.2.15}(e))} \\
        \implies & X \setminus \{x \in X : f(x) \in Y \setminus V\} \text{ is open in } (X, d_X)                                           \\
        \implies & \{x \in X : f(x) \in V\} \text{ is open in } (X, d_X)                                                                   \\
        \implies & f^{-1}(V) \text{ is open in } (X, d_X).
    \end{align*}
    Since \(V\) is arbitrary, we know that statement (d) implies statement (c).
    We conclude that statements (a)(b)(c)(d) are all equivalent.
\end{proof}

\begin{remark}\label{2.1.6}
    It may seem strange that continuity ensures that the \emph{inverse} image of an open set is open.
    One may guess instead that the reverse should be true, that the \emph{forward} image of an open set is open;
    but this is not true;
    see Exercises \ref{ex 1.5.4}, \ref{ex 1.5.5}.
\end{remark}

\begin{corollary}[Continuity preserved by composition]\label{2.1.7}
    Let \((X, d_X)\), \((Y, d_Y)\), and \((Z, d_Z)\) be metric spaces.
    \begin{enumerate}
        \item If \(f : X \to Y\) is continuous at a point \(x_0 \in X\), and \(g : Y \to Z\) is continuous at \(f(x_0)\), then the composition \(g \circ f : X \to Z\), defined by \(g \circ f(x) \coloneqq g(f(x))\), is continuous at \(x_0\).
        \item If \(f : X \to Y\) is continuous, and \(g : Y \to Z\) is continuous, then \(g \circ f : X \to Z\) is also continuous.
    \end{enumerate}
\end{corollary}

\begin{proof}{(a)}
    Since \(f\) is continuous at \(x_0\) in \((X, d_X)\), by Theorem \ref{2.1.4}(a)(c) we know that
    \begin{align*}
                 & \forall\ V \subseteq Y, \big(V \text{ is open in } (Y, d_Y)\big) \land \big(f(x_0) \in V\big)                        \\
        \implies & \exists\ U \subseteq X : \big(U \text{ is open in } (X, d_X)\big) \land (x_0 \in U) \land \big(f(U) \subseteq V\big)
    \end{align*}
    Now we choose such \(U\) for each open set \(V\) in \((Y, d_Y)\) and denote it as \(U_V\).
    Since \(g\) is continuous at \(f(x_0)\) in \((Y, d_Y)\), by Theorem \ref{2.1.4}(a)(c) we know that
    \begin{align*}
                 & \forall\ W \subseteq Z, \big(W \text{ is open in } (Z, d_Z)\big) \land \Big(g\big(f(x_0)\big) \in W\Big)                                                \\
        \implies & \exists\ V \subseteq Y : \big(V \text{ is open in } (Y, d_Y)\big) \land (y_0 \in V) \land \big(g(V) \subseteq W\big)                                    \\
        \implies & \exists\ U_V \subseteq X : \big(U_V \text{ is open in } (X, d_X)\big) \land (x_0 \in U_V) \land \big(f(U_V) \subseteq V\big)                            \\
        \implies & \exists\ U_V \subseteq X : \big(U_V \text{ is open in } (X, d_X)\big) \land (x_0 \in U_V) \land \Big(g\big(f(U_V)\big) \subseteq g(V) \subseteq W\Big).
    \end{align*}
    Thus by Theorem \ref{2.1.4}(a)(c) we know that \(g \circ f\) is continuous at \(x_0\) in \((X, d_X)\).
\end{proof}

\begin{proof}{(b)}
    Let \(x_0 \in X\).
    Then we have
    \begin{align*}
                 & f \text{ is continuous in } (X, d_X)                                                       \\
        \implies & f \text{ is continuous at } x_0 \text{ in } (X, d_X). & \text{(by Definition \ref{2.1.1})}
    \end{align*}
    Since \(f(x_0) \in Y\), we have
    \begin{align*}
                 & g \text{ is continuous in } (Y, d_Y)                                                                 \\
        \implies & g \text{ is continuous at } f(x_0) \text{ in } (Y, d_Y)       & \text{(by Definition \ref{2.1.1})}   \\
        \implies & g \circ f \text{ is continuous at } x_0 \text{ in } (X, d_X). & \text{(by Corollary \ref{2.1.7}(a))}
    \end{align*}
    Since \(x_0\) is arbitrary, by Definition \ref{2.1.1} we know that \(g \circ f\) is continuous in \((X, d_X)\).
\end{proof}

\exercisesection

\begin{exercise}\label{ex 2.1.1}
    Prove Theorem \ref{2.1.4}.
\end{exercise}

\begin{proof}
    See Theorem \ref{2.1.4}.
\end{proof}

\begin{exercise}\label{ex 2.1.2}
    Prove Theorem \ref{2.1.5}.
\end{exercise}

\begin{proof}
    See Theorem \ref{2.1.5}.
\end{proof}

\begin{exercise}\label{ex 2.1.3}
    Use Theorem \ref{2.1.4} and Theorem \ref{2.1.5} to prove Corollary \ref{2.1.7}.
\end{exercise}

\begin{proof}
    See Corollary \ref{2.1.7}.
\end{proof}

\begin{exercise}\label{ex 2.1.4}
    Give an example of functions \(f : \mathbf{R} \to \mathbf{R}\) and \(g : \mathbf{R} \to \mathbf{R}\) such that
    \begin{enumerate}
        \item \(f\) is not continuous, but \(g\) and \(g \circ f\) are continuous;
        \item \(g\) is not continuous, but \(f\) and \(g \circ f\) are continuous;
        \item \(f\) and \(g\) are not continuous, but \(g \circ f\) is continuous.
    \end{enumerate}
    Explain briefly why these examples do not contradict Corollary \ref{2.1.7}.
\end{exercise}

\begin{proof}{(a)}
    Let \(f : \mathbf{R} \to \mathbf{R}\) be the function
    \[
        \forall\ x \in \mathbf{R}, f(x) = \begin{cases}
            1 & \text{if } x = 0    \\
            0 & \text{if } x \neq 0
        \end{cases}
    \]
    and let \(g : \mathbf{R} \to \mathbf{R}\) be the function \(g(x) = 0\) for all \(x \in \mathbf{R}^+\).
    Then we know that \(f\) is not continuous at \(0\) in \((\mathbf{R}, d_{l^1}|_{\mathbf{R} \times \mathbf{R}})\) and thus \(f\) is not continuous in \((\mathbf{R}, d_{l^1}|_{\mathbf{R} \times \mathbf{R}})\).
    Since \(g\) is constant function, we know that \(g\) is continuous in \((\mathbf{R}, d_{l^1}|_{\mathbf{R} \times \mathbf{R}})\).
    Since \(g \circ f\) is also a constant function, we know that \(g \circ f\) is continuous in \((\mathbf{R}, d_{l^1}|_{\mathbf{R} \times \mathbf{R}})\).
    This does not contradict to Corollary \ref{2.1.7} since \(f\) is not continuous in \((\mathbf{R}, d_{l^1}|_{\mathbf{R} \times \mathbf{R}})\).
\end{proof}

\begin{proof}{(b)}
    Let \(f : \mathbf{R} \to \mathbf{R}\) be the function \(f(x) = 0\) for all \(x \in \mathbf{R}^+\).
    Let \(g : \mathbf{R} \to \mathbf{R}\) be the function
    \[
        \forall\ x \in \mathbf{R}, g(x) = \begin{cases}
            1 & \text{if } x = 0    \\
            0 & \text{if } x \neq 0
        \end{cases}
    \]
    Since \(f\) is constant function, we know that \(f\) is continuous in \((\mathbf{R}, d_{l^1}|_{\mathbf{R} \times \mathbf{R}})\).
    Since \(g\) is not continuous at \(0\) in \((\mathbf{R}, d_{l^1}|_{\mathbf{R} \times \mathbf{R}})\), we know that \(g\) is not continuous in \((\mathbf{R}, d_{l^1}|_{\mathbf{R} \times \mathbf{R}})\).
    Since \(g \circ f\) is a constant function, we know that \(g \circ f\) is continuous in \((\mathbf{R}, d_{l^1}|_{\mathbf{R} \times \mathbf{R}})\).
    This does not contradict to Corollary \ref{2.1.7} since \(g\) is not continuous in \((\mathbf{R}, d_{l^1}|_{\mathbf{R} \times \mathbf{R}})\).
\end{proof}

\begin{proof}{(c)}
    Let \(f : \mathbf{R} \to \mathbf{R}\) be the function
    \[
        \forall\ x \in \mathbf{R}, f(x) = \begin{cases}
            1 & \text{if } x = 0    \\
            0 & \text{if } x \neq 0
        \end{cases}
    \]
    and let \(g : \mathbf{R} \to \mathbf{R}\) be the function
    \[
        \forall\ x \in \mathbf{R}, g(x) = \begin{cases}
            1 & \text{if } x = 2    \\
            0 & \text{if } x \neq 2
        \end{cases}
    \]
    Since \(f\) is not continuous at \(0\) in \((\mathbf{R}, d_{l^1}|_{\mathbf{R} \times \mathbf{R}})\), we know that \(f\) is not continuous in \((\mathbf{R}, d_{l^1}|_{\mathbf{R} \times \mathbf{R}})\).
    Similarly \(g\) is not continuous in \((\mathbf{R}, d_{l^1}|_{\mathbf{R} \times \mathbf{R}})\).
    Since \(g \circ f\) is a constant function, we know that \(g \circ f\) is continuous in \((\mathbf{R}, d_{l^1}|_{\mathbf{R} \times \mathbf{R}})\).
    This does not contradict to Corollary \ref{2.1.7} since \(f, g\) are not continuous in \((\mathbf{R}, d_{l^1}|_{\mathbf{R} \times \mathbf{R}})\).
\end{proof}

\begin{exercise}\label{ex 2.1.5}
    Let \((X, d)\) be a metric space, and let \((E, d|_{E \times E})\) be a subspace of \((X, d)\).
    Let \(\iota_{E \to X} : E \to X\) be the inclusion map, defined by setting \(\iota_{E \to X}(x) \coloneqq x\) for all \(x \in E\).
    Show that \(\iota_{E \to X}\) is continuous.
\end{exercise}

\begin{proof}
    Let \(x_0 \in E\).
    Since
    \begin{align*}
                 & \forall\ \varepsilon \in \mathbf{R}^+, \forall\ x \in E, d|_{E \times E}(x, x_0) < \varepsilon                                                                 \\
        \implies & d|_{E \times E}\big(\iota_{E \to X}(x), \iota_{E \to X}(x_0)\big) = d|_{E \times E}(x, x_0) < \varepsilon, & \text{(by the definition of \(\iota_{E \to X}\))}
    \end{align*}
    by setting \(\delta = \varepsilon\) we have
    \begin{align*}
         & \forall\ \varepsilon \in \mathbf{R}^+, \exists\ \delta \in \mathbf{R}^+ :                                                                             \\
         & \Big(\forall\ x \in E, d|_{E \times E}(x, x_0) < \delta \implies d|_{E \times E}\big(\iota_{E \to X}(x), \iota_{E \to X}(x_0)\big) < \varepsilon\Big)
    \end{align*}
    and thus by Definition \ref{2.1.1} \(\iota_{E \to X}\) is continuous at \(x_0\) in \((E, d|_{E \times E})\).
    Since \(x_0\) is arbitrary, by Definition \ref{2.1.1} \(\iota_{E \to X}\) is continuous in \((E, d|_{E \times E})\).
\end{proof}

\begin{exercise}\label{ex 2.1.6}
    Let \(f : X \to Y\) be a function from one metric space \((X, d_X)\) to another \((Y, d_Y)\).
    Let \(E\) be a subset of \(X\) (which we give the induced metric \(d_X|_{E \times E}\)), and let \(f|_E : E \to Y\) be the restriction of \(f\) to \(E\), thus \(f|_E(x) \coloneqq f(x)\) when \(x \in E\).
    If \(x_0 \in E\) and \(f\) is continuous at \(x_0\), show that \(f|_E\) is also continuous at \(x_0\).
    (Is the converse of this statement true? Explain.)
    Conclude that if \(f\) is continuous, then \(f|_E\) is continuous.
    Thus restriction of the domain of a function does not destroy continuity.
\end{exercise}

\begin{proof}
    See Remark \ref{2.1.3}.
    The converse is not true since the statement
    \[
        \forall\ \varepsilon \in \mathbf{R}^+, \exists\ \delta \in \mathbf{R}^+ : \Big(\forall\ x \in E, d_X|_{E \times E}(x, x_0) < \delta \implies d_Y\big(f(x), f(x_0)\big) < \varepsilon\Big)
    \]
    does not imply the statement
    \[
        \forall\ \varepsilon \in \mathbf{R}^+, \exists\ \delta \in \mathbf{R}^+ : \Big(\forall\ x \in X, d_X(x, x_0) < \delta \implies d_Y\big(f(x), f(x_0)\big) < \varepsilon\Big).
    \]
\end{proof}

\begin{exercise}\label{ex 2.1.7}
    Let \(f : X \to Y\) be a function from one metric space \((X, d_X)\) to another \((Y, d_Y)\).
    Suppose that the image \(f(X)\) of \(X\) is contained in some subset \(E \subseteq Y\) of \(Y\).
    Let \(g : X \to E\) be the function which is the same as \(f\) but with the range restricted from \(Y\) to \(E\), thus \(g(x) = f(x)\) for all \(x \in X\).
    We give \(E\) the metric \(d_Y|_{E \times E}\) induced from \(Y\).
    Show that for any \(x_0 \in X\), that \(f\) is continuous at \(x_0\) if and only if \(g\) is continuous at \(x_0\).
    Conclude that \(f\) is continuous if and only if \(g\) is continuous.
    (Thus the notion of continuity is not affected if one restricts the range of the function.)
\end{exercise}
\section{Continuity and product spaces}\label{sec 2.2}

\begin{note}
    Given two functions \(f : X \to Y\) and \(g : X \to Z\), one can define their \emph{direct sum} \(f \oplus g : X \to Y \times Z\) defined by \(f \oplus g(x) \coloneqq \big(f(x), g(x)\big)\), i.e., this is the function taking values in the Cartesian product \(Y \times Z\) whose first co-ordinate is \(f(x)\) and whose second co-ordinate is \(g(x)\)
    (cf. Exercise 3.5.7 in Analysis I).
\end{note}

\begin{lemma}\label{2.2.1}
    Let \(f : X \to \mathbf{R}\) and \(g : X \to \mathbf{R}\) be functions, and let \(f \oplus g : X \to \mathbf{R}^2\) be their direct sum.
    We give \(\mathbf{R}^2\) the Euclidean metric.
    \begin{enumerate}
        \item If \(x_0 \in X\), then \(f\) and \(g\) are both continuous at \(x_0\) if and only if \(f \oplus g\) is continuous at \(x_0\).
        \item \(f\) and \(g\) are both continuous if and only if \(f \oplus g\) is continuous.
    \end{enumerate}
\end{lemma}

\begin{proof}{(a)}
    Let \((X, d)\) be a metric space.
    Then we have
    \begin{align*}
             & f, g \text{ are continuous at } x_0                                                                                                                                             \\
             & \text{from } (X, d) \text{ to } (\mathbf{R}, d_{l^2}|_{\mathbf{R} \times \mathbf{R}})                                                                                           \\
        \iff & \text{every sequence } (x^{(n)})_{n = 1}^\infty \text{ in } X \text{ satisfies the following:}                                                                                  \\
             & \lim_{n \to \infty} d\big(x^{(n)}, x_0\big) = 0 \text{ implies }                                                                                                                \\
             & \begin{cases}
            \lim_{n \to \infty} d_{l^2}|_{\mathbf{R} \times \mathbf{R}}\big(f(x^{(n)}), f(x_0)\big) = 0 \\
            \lim_{n \to \infty} d_{l^2}|_{\mathbf{R} \times \mathbf{R}}\big(g(x^{(n)}), g(x_0)\big) = 0
        \end{cases}                                                                                                               & \text{(by Theorem \ref{2.1.4}(a)(b))} \\
        \iff & \text{every sequence } (x^{(n)})_{n = 1}^\infty \text{ in } X \text{ satisfies the following:}                                                                                  \\
             & \lim_{n \to \infty} d\big(x^{(n)}, x_0\big) = 0 \text{ implies }                                                                                                                \\
             & \lim_{n \to \infty} d_{l^2}|_{\mathbf{R}^2 \times \mathbf{R}^2}\Big(\big(f(x^{(n)}), g(x^{(n)})\big), \big(f(x_0), g(x_0)\big)\Big) = 0 & \text{(by Proposition \ref{1.1.18})}  \\
        \iff & f \oplus g \text{ is continuous at } x_0                                                                                                                                        \\
             & \text{from } (X, d) \text{ to } (\mathbf{R}^2, d_{l^2}|_{\mathbf{R}^2 \times \mathbf{R}^2}).                                            & \text{(by Theorem \ref{2.1.4}(a)(b))}
    \end{align*}
\end{proof}

\begin{proof}{(b)}
    Let \((X, d)\) be a metric space.
    Then we have
    \begin{align*}
             & f, g \text{ are continuous from } (X, d) \text{ to } (\mathbf{R}, d_{l^2}|_{\mathbf{R} \times \mathbf{R}})                                                  \\
        \iff & \forall\ x_0 \in X, f, g \text{ are continuous at } x_0                                                                                                     \\
             & \text{from } (X, d) \text{ to } (\mathbf{R}, d_{l^2}|_{\mathbf{R} \times \mathbf{R}})                                  & \text{(by Definition \ref{2.1.1})} \\
        \iff & \forall\ x_0 \in X, f \oplus g \text{ is continuous at } x_0                                                           & \text{(by Lemma \ref{2.2.1}(a))}   \\
             & \text{from } (X, d) \text{ to } (\mathbf{R}^2, d_{l^2}|_{\mathbf{R}^2 \times \mathbf{R}^2})                                                                 \\
        \iff & f \oplus g \text{ is continuous from } (X, d) \text{ to } (\mathbf{R}^2, d_{l^2}|_{\mathbf{R}^2 \times \mathbf{R}^2}). & \text{(by Definition \ref{2.1.1})}
    \end{align*}
\end{proof}

\begin{additional corollary}\label{ac 2.2.1}
Let \((X, d)\) be a metric space.
Let \((\mathbf{R}, d_{\mathbf{R}})\) be a metric space where \(d_{\mathbf{R}}\) can be \(d_{l^1}|_{\mathbf{R} \times \mathbf{R}}\), \(d_{l^2}|_{\mathbf{R} \times \mathbf{R}}\) or \(d_{l^\infty}|_{\mathbf{R} \times \mathbf{R}}\).
Let \((\mathbf{R}^2, d_{\mathbf{R}^2})\) be a metric space where \(d_{\mathbf{R}^2}\) can be \(d_{l^1}|_{\mathbf{R}^2 \times \mathbf{R}^2}\), \(d_{l^2}|_{\mathbf{R}^2 \times \mathbf{R}^2}\) or \(d_{l^\infty}|_{\mathbf{R}^2 \times \mathbf{R}^2}\).
Let \(f : X \to \mathbf{R}\) and \(g : X \to \mathbf{R}\) be functions, and let \(f \oplus g : X \to \mathbf{R}^2\) be their direct sum.
\begin{enumerate}
    \item If \(x_0 \in X\), then \(f\) and \(g\) are both continuous at \(x_0\) from \((X, d)\) to \((\mathbf{R}, d_{\mathbf{R}})\) if and only if \(f \oplus g\) is continuous at \(x_0\) from \((X, d)\) to \((\mathbf{R}^2, d_{\mathbf{R}^2})\).
    \item \(f\) and \(g\) are both continuous from \((X, d)\) to \((\mathbf{R}, d_{\mathbf{R}})\) if and only if \(f \oplus g\) is continuous from \((X, d)\) to \((\mathbf{R}^2, d_{\mathbf{R}^2})\).
\end{enumerate}
\end{additional corollary}

\begin{proof}
    By Proposition \ref{1.1.18} and Lemma \ref{2.2.1} we are done.
\end{proof}

\begin{lemma}\label{2.2.2}
    The addition function \((x, y) \mapsto x + y\), the subtraction function \((x, y) \mapsto x - y\), the multiplication function \((x, y) \mapsto xy\), the maximum function \((x, y) \mapsto \max(x, y)\), and the minimum function \((x, y) \mapsto \min(x, y)\), are all continuous functions from \(\mathbf{R}^2\) to \(\mathbf{R}\).
    The division function \((x, y) \mapsto x / y\) is a continuous function from \(\mathbf{R} \times (\mathbf{R} \setminus \{0\}) = \{(x, y) \in \mathbf{R}^2 : y \neq 0\}\) to \(\mathbf{R}\).
    For any real number \(c\), the function \(x \mapsto cx\) is a continuous function from \(\mathbf{R}\) to \(\mathbf{R}\).
\end{lemma}

\begin{proof}
    First we show that the addition, subtraction, multiplication, maximum and minimum functions from \(\mathbf{R}^2\) to \(\mathbf{R}\) are continuous from \((\mathbf{R}^2, d_{l^1}|_{\mathbf{R}^2 \times \mathbf{R}^2})\) to \((\mathbf{R}, d_{l^1}|_{\mathbf{R} \times \mathbf{R}})\).
    Let \((x, y) \in \mathbf{R}^2\) and let \((x^{(n)}, y^{(n)})_{n = 1}^\infty\) be a sequence in \(\mathbf{R}^2\) such that
    \[
        \lim_{n \to \infty} d_{l^1}|_{\mathbf{R}^2 \times \mathbf{R}^2}\big((x^{(n)}, y^{(n)}), (x, y)\big) = 0
    \]
    By limit laws we know that
    \begin{align*}
         & \lim_{n \to \infty} d_{l^1}|_{\mathbf{R} \times \mathbf{R}}(x^{(n)} + y^{(n)}, x + y) = 0                   \\
         & \lim_{n \to \infty} d_{l^1}|_{\mathbf{R} \times \mathbf{R}}(x^{(n)} - y^{(n)}, x - y) = 0                   \\
         & \lim_{n \to \infty} d_{l^1}|_{\mathbf{R} \times \mathbf{R}}(x^{(n)} y^{(n)}, xy) = 0                        \\
         & \lim_{n \to \infty} d_{l^1}|_{\mathbf{R} \times \mathbf{R}}\big(\max(x^{(n)}, y^{(n)}), \max(x, y)\big) = 0 \\
         & \lim_{n \to \infty} d_{l^1}|_{\mathbf{R} \times \mathbf{R}}\big(\min(x^{(n)}, y^{(n)}), \min(x, y)\big) = 0 \\
    \end{align*}
    Since \((x^{(n)}, y^{(n)})_{n = 1}^\infty\) is arbitrary, by Theorem \ref{2.1.4}(a)(b) we know that the addition, subtraction, multiplication, maximum and minimum functions from \(\mathbf{R}^2\) to \(\mathbf{R}\) are continuous at \((x, y)\) from \((\mathbf{R}^2, d_{l^1}|_{\mathbf{R}^2 \times \mathbf{R}^2})\) to \((\mathbf{R}, d_{l^1}|_{\mathbf{R} \times \mathbf{R}})\).
    Since \((x, y)\) is arbitrary, by Theorem \ref{2.1.5}(a)(b) we know that the addition, subtraction, multiplication, maximum and minimum functions from \(\mathbf{R}^2\) to \(\mathbf{R}\) are continuous from \((\mathbf{R}^2, d_{l^1}|_{\mathbf{R}^2 \times \mathbf{R}^2})\) to \((\mathbf{R}, d_{l^1}|_{\mathbf{R} \times \mathbf{R}})\).

    Next we show that the division function from \(E = \mathbf{R} \times (\mathbf{R} \setminus \{0\})\) to \(\mathbf{R}\) is continuous from \((E, d_{l^1}|_{E \times E})\) to \((\mathbf{R}, d_{l^1}|_{\mathbf{R} \times \mathbf{R}})\).
    Let \((x, y) \in E\) and let \((x^{(n)}, y^{(n)})_{n = 1}^\infty\) be a sequence in \(E\) such that
    \[
        \lim_{n \to \infty} d_{l^1}|_{E \times E}\big((x^{(n)}, y^{(n)}), (x, y)\big) = 0
    \]
    By limit laws we know that
    \[
        \lim_{n \to \infty} d_{l^1}|_{\mathbf{R} \times \mathbf{R}}(x^{(n)} / y^{(n)}, x / y) = 0.
    \]
    Thus using similar arguments as above we know that the division function from \(E\) to \(\mathbf{R}\) is continuous from \((E, d_{l^1}|_{E \times E})\) to \((\mathbf{R}, d_{l^1}|_{\mathbf{R} \times \mathbf{R}})\).

    Finally we show that the constant multiplication function from \(\mathbf{R}\) to \(\mathbf{R}\) is continuous from \((\mathbf{R}, d_{l^1}|_{\mathbf{R} \times \mathbf{R}})\) to \((\mathbf{R}, d_{l^1}|_{\mathbf{R} \times \mathbf{R}})\).
    Let \(c, x \in \mathbf{R}\) and let \((x^{(n)})_{n = 1}^\infty\) be a sequence in \(\mathbf{R}\) such that
    \[
        \lim_{n \to \infty} d_{l^1}|_{\mathbf{R} \times \mathbf{R}}(x^{(n)}, x) = 0.
    \]
    By limit laws we know that
    \[
        \lim_{n \to \infty} d_{l^1}|_{\mathbf{R} \times \mathbf{R}}(cx^{(n)}, cx) = 0.
    \]
    Thus using similar arguments as above we know that the constant function from \(\mathbf{R}\) to \(\mathbf{R}\) is continuous from \((\mathbf{R}, d_{l^1}|_{\mathbf{R} \times \mathbf{R}})\) to \((\mathbf{R}, d_{l^1}|_{\mathbf{R} \times \mathbf{R}})\).
\end{proof}

\begin{corollary}\label{2.2.3}
    Let \((X, d)\) be a metric space, let \(f : X \to \mathbf{R}\) and \(g : X \to \mathbf{R}\) be functions.
    Let \(c\) be a real number.
    \begin{enumerate}
        \item If \(x_0 \in X\) and \(f\) and \(g\) are continuous at \(x_0\), then the functions \(f + g : X \to \mathbf{R}\), \(f - g : X \to \mathbf{R}\), \(fg : X \to \mathbf{R}\), \(\max(f, g) : X \to \mathbf{R}\), \(\min(f, g) : X \to \mathbf{R}\), and \(cf : X \to \mathbf{R}\) (see Definition 9.2.1 in Analysis I for definitions) are also continuous at \(x_0\).
              If \(g(x) \neq 0\) for all \(x \in X\), then \(f / g : X \to \mathbf{R}\) is also continuous at \(x_0\).
        \item If \(f\) and \(g\) are continuous, then the functions \(f + g : X \to \mathbf{R}\), \(f - g : X \to \mathbf{R}\), \(fg : X \to \mathbf{R}\), \(\max(f, g) : X \to \mathbf{R}\), \(\min(f, g) : X \to \mathbf{R}\), and \(cf : X \to \mathbf{R}\) are also continuous.
              If \(g(x) \neq 0\) for all \(x \in X\), then \(f / g : X \to \mathbf{R}\) is also continuous.
    \end{enumerate}
\end{corollary}

\begin{proof}
    We first prove (a). Since \(f, g\) are continuous at \(x_0\) from \((X, d)\) to \((\mathbf{R}, d_{l^1}|_{\mathbf{R} \times \mathbf{R}})\), then by Additional Corollary \ref{ac 2.2.1}(a) \(f \oplus g : X \to \mathbf{R}^2\) is also continuous at \(x_0\) from \((X, d)\) to \((\mathbf{R}^2, d_{l^1}|_{\mathbf{R}^2 \times \mathbf{R}^2})\).
    On the other hand, from Lemma \ref{2.2.2} the function \((x, y) \mapsto x + y\) is continuous from \((\mathbf{R}^2, d_{l^1}|_{\mathbf{R}^2 \times \mathbf{R}^2})\) to \((\mathbf{R}, d_{l^1}|_{\mathbf{R} \times \mathbf{R}})\), and in particular is continuous at \(f \oplus g(x_0)\) from \((\mathbf{R}^2, d_{l^1}|_{\mathbf{R}^2 \times \mathbf{R}^2})\) to \((\mathbf{R}, d_{l^1}|_{\mathbf{R} \times \mathbf{R}})\).
    If we then compose these two functions using Corollary \ref{2.1.7} we conclude that \(f + g : X \to \mathbf{R}\) is continuous from \((X, d)\) to \((\mathbf{R}, d_{l^1}|_{\mathbf{R} \times \mathbf{R}})\).
    A similar argument gives the continuity of \(f - g\), \(fg\), \(\max(f, g)\), \(\min(f, g)\) and \(cf\).
    To prove the claim for \(f / g\), we first use Exercise \ref{ex 2.1.7} to restrict the range of \(g\) from \(\mathbf{R}\) to \(\mathbf{R} \setminus \{0\}\), and then one can argue as before.
    The claim (b) follows immediately from (a).
\end{proof}

\exercisesection

\begin{exercise}\label{ex 2.2.1}
    Prove Lemma \ref{2.2.1}.
\end{exercise}

\begin{proof}
    See Lemma \ref{2.2.1}.
\end{proof}

\begin{exercise}\label{ex 2.2.2}
    Prove Lemma \ref{2.2.2}.
\end{exercise}

\begin{proof}
    See Lemma \ref{2.2.2}.
\end{proof}

\begin{exercise}\label{ex 2.2.3}
    Show that if \(f : X \to \mathbf{R}\) is a continuous function, so is the function \(\abs*{f} : X \to \mathbf{R}\) defined by \(\abs*{f}(x) \coloneqq \abs*{f(x)}\).
\end{exercise}

\begin{proof}
    Let \((X, d_X)\) be a metric space and let \(f : X \to \mathbf{R}\) be a function which is continuous from \((X, d_X)\) to \(\).
    Since
    \begin{align*}
                 & \forall\ x_0 \in X, \abs*{f}(x) = \abs*{f(x)} = \max\big(-f(x), f(x)\big) \\
        \implies & \abs*{f} = \max(-f, f),
    \end{align*}
    we have
    \begin{align*}
                 & f \text{ is continuous from } (X, d_X) \text{ to } (\mathbf{R}, d_{l^1}|_{\mathbf{R} \times \mathbf{R}})                                                  \\
        \implies & -f \text{ is continuous from } (X, d_X) \text{ to } (\mathbf{R}, d_{l^1}|_{\mathbf{R} \times \mathbf{R}})          & \text{(by Corollary \ref{2.2.3}(b))} \\
        \implies & \max(f, -f) \text{ is continuous from } (X, d_X) \text{ to } (\mathbf{R}, d_{l^1}|_{\mathbf{R} \times \mathbf{R}}) & \text{(by Corollary \ref{2.2.3}(b))} \\
        \implies & \abs*{f} \text{ is continuous from } (X, d_X) \text{ to } (\mathbf{R}, d_{l^1}|_{\mathbf{R} \times \mathbf{R}}).
    \end{align*}
\end{proof}

\begin{exercise}\label{ex 2.2.4}
    Let \(\pi_1 : \mathbf{R}^2 \to \mathbf{R}\) and \(\pi_2 : \mathbf{R}^2 \to \mathbf{R}\) be the functions \(\pi_1(x, y) \coloneqq x\) and \(\pi_2(x, y) \coloneqq y\) (these two functions are sometimes called the \emph{co-ordinate functions} on \(\mathbf{R}^2\)).
    Show that \(\pi_1\) and \(\pi_2\) are continuous.
    Conclude that if \(f : \mathbf{R} \to X\) is any continuous function into a metric space \((X, d)\), then the functions \(g_1 : \mathbf{R}^2 \to X\) and \(g_2 : \mathbf{R}^2 \to X\) defined by \(g_1(x, y) \coloneqq f(x)\) and \(g_2(x, y) \coloneqq f(y)\) are also continuous.
\end{exercise}

\begin{proof}
    Let \((x, y) \in \mathbf{R}^2\).
    We know that
    \begin{align*}
                 & \forall\ \varepsilon \in \mathbf{R}^+, \forall\ (x', y') \in \mathbf{R}^2,                                                            \\
                 & d_{l^1}|_{\mathbf{R}^2 \times \mathbf{R}^2}\big((x, y), (x', y')\big) < \varepsilon                                                   \\
        \implies & \abs*{x - x'} + \abs*{y - y'} < \varepsilon                                                & \text{(by Example \ref{1.1.7})}          \\
        \implies & \abs*{x - x'} < \varepsilon                                                                & \text{(by Definition \ref{1.1.2}(a)(b))} \\
        \implies & d_{l^1}|_{\mathbf{R} \times \mathbf{R}}(x, x') < \varepsilon                               & \text{(by Example \ref{1.1.7})}          \\
        \implies & d_{l^1}|_{\mathbf{R} \times \mathbf{R}}\big(\pi_1(x, y), \pi_1(x', y')\big) < \varepsilon. & \text{(by the definition of \(\pi_1\))}
    \end{align*}
    Thus by setting \(\delta = \varepsilon\) we have
    \begin{align*}
         & \forall\ \varepsilon \in \mathbf{R}^+, \exists\ \delta \in \mathbf{R}^+ :                                                                                                                                                        \\
         & \Big(\forall\ (x', y') \in \mathbf{R}^2, d_{l^1}|_{\mathbf{R}^2 \times \mathbf{R}^2}\big((x, y), (x', y')\big) < \delta \implies d_{l^1}|_{\mathbf{R} \times \mathbf{R}}\big(\pi_1(x, y), \pi_1(x', y')\big) < \varepsilon\Big).
    \end{align*}
    Since \((x, y)\) is arbitrary, by Definition \ref{2.1.1} \(\pi_1\) is continuous from \((\mathbf{R}^2, d_{l^1}|_{\mathbf{R}^2 \times \mathbf{R}^2})\) to \((\mathbf{R}, d_{l^1}|_{\mathbf{R} \times \mathbf{R}})\).
    Using similar arguments we can show that \(\pi_2\) is continuous from \((\mathbf{R}^2, d_{l^1}|_{\mathbf{R}^2 \times \mathbf{R}^2})\) to \((\mathbf{R}, d_{l^1}|_{\mathbf{R} \times \mathbf{R}})\).

    Let \(f : \mathbf{R} \to X\) be a function which is continuous from \((\mathbf{R}, d_{l^1}|_{\mathbf{R} \times \mathbf{R}})\) to \((X, d)\).
    Let \(g_1 : \mathbf{R}^2 \to X\) and \(g_2 : \mathbf{R}^2 \to X\) be functions where
    \[
        \forall\ (x, y) \in \mathbf{R}^2, \begin{cases}
            g_1(x, y) = f(x) \\
            g_2(x, y) = f(y)
        \end{cases}
    \]
    Since
    \begin{align*}
         & \forall\ (x, y) \in \mathbf{R}^2,                                \\
         & f \circ \pi_1(x, y) = f\big(\pi_1(x, y)\big) = f(x) = g_1(x, y); \\
         & f \circ \pi_2(x, y) = f\big(\pi_2(x, y)\big) = f(y) = g_2(x, y),
    \end{align*}
    we know that \(g_1 = f \circ \pi_1\) and \(g_2 = f \circ \pi_2\).
    Thus by Corollary \ref{2.1.7}(b) \(g_1, g_2\) are continuous from \((\mathbf{R}^2, d_{l^1}|_{\mathbf{R}^2 \times \mathbf{R}^2})\) to \((X, d)\).
\end{proof}

\begin{exercise}\label{ex 2.2.5}
    Let \(n, m \geq 0\) be integers.
    Suppose that for every \(0 \leq i \leq n\) and \(0 \leq j \leq m\) we have a real number \(c_{ij}\).
    Form the function \(P : \mathbf{R}^2 \to \mathbf{R}\) defined by
    \[
        P(x, y) \coloneqq \sum_{i = 0}^n \sum_{j = 0}^m c_{ij} x^i y^j.
    \]
    (Such a function is known as a \emph{polynomial of two variables})
    Show that \(P\) is continuous.
    Conclude that if \(f : X \to \mathbf{R}\) and \(g : X \to \mathbf{R}\) are continuous functions, then the function \(P(f, g) : X \to \mathbf{R}\) defined by \(P(f, g)(x) \coloneqq P\big(f(x), g(x)\big)\) is also continuous.
\end{exercise}

\begin{proof}
    First we show that \(P\) is continuous from \((\mathbf{R}^2, d_{l^1}|_{\mathbf{R}^2 \times \mathbf{R}^2})\) to \((\mathbf{R}, d_{l^1}|_{\mathbf{R} \times \mathbf{R}})\).
    Let \((x, y) \in \mathbf{R}^2\).
    Let \(\pi_1, \pi_2\) be the functions defined in Exercise \ref{ex 2.2.4}.
    Since \(\pi_1\) is continuous from \((\mathbf{R}^2, d_{l^1}|_{\mathbf{R}^2 \times \mathbf{R}^2})\) to \((\mathbf{R}, d_{l^1}|_{\mathbf{R} \times \mathbf{R}})\), by Corollary \ref{2.2.3}(b) we know that
    \[
        x^i = \prod_{k = 1}^i x = \prod_{k = 1}^i \pi_1(x, y)
    \]
    is continuous at \((x, y)\) from \((\mathbf{R}^2, d_{l^1}|_{\mathbf{R}^2 \times \mathbf{R}^2})\) to \((\mathbf{R}, d_{l^1}|_{\mathbf{R} \times \mathbf{R}})\) for every \(0 \leq i \leq n\).
    Similarly
    \[
        y^j = \prod_{k = 1}^j y = \prod_{k = 1}^j \pi_2(x, y)
    \]
    is continuous at \((x, y)\) from \((\mathbf{R}^2, d_{l^1}|_{\mathbf{R}^2 \times \mathbf{R}^2})\) to \((\mathbf{R}, d_{l^1}|_{\mathbf{R} \times \mathbf{R}})\) for every \(0 \leq j \leq m\).
    Thus by Corollary \ref{2.2.3}(b) we know that \(c_{ij} x^i y^j\) is continuous at \((x, y)\) from \((\mathbf{R}^2, d_{l^1}|_{\mathbf{R}^2 \times \mathbf{R}^2})\) to \((\mathbf{R}, d_{l^1}|_{\mathbf{R} \times \mathbf{R}})\) for every \(0 \leq i \leq n\) and \(0 \leq j \leq m\), and
    \[
        \sum_{i = 0}^n \sum_{j = 0}^m c_{ij} x^i y^j = P(x, y)
    \]
    is continuous at \((x, y)\) from \((\mathbf{R}^2, d_{l^1}|_{\mathbf{R}^2 \times \mathbf{R}^2})\) to \((\mathbf{R}, d_{l^1}|_{\mathbf{R} \times \mathbf{R}})\).
    Since \((x, y)\) is arbitrary, by Definition \ref{2.1.1} we know that \(P\) is continuous from \((\mathbf{R}^2, d_{l^1}|_{\mathbf{R}^2 \times \mathbf{R}^2})\) to \((\mathbf{R}, d_{l^1}|_{\mathbf{R} \times \mathbf{R}})\).

    Now suppose that \(f : X \to \mathbf{R}\) and \(g : X \to \mathbf{R}\) are to continuous functions from \((X, d)\) to \((\mathbf{R}, d_{l^1}|_{\mathbf{R} \times \mathbf{R}})\).
    Then we have
    \begin{align*}
                 & f \oplus g \text{ is continuous }                                                                                                                \\
                 & \text{from } (X, d) \text{ to } (\mathbf{R}^2, d_{l^1}|_{\mathbf{R}^2 \times \mathbf{R}^2}) & \text{(by Additional Corollary \ref{ac 2.2.1}(b))} \\
        \implies & P \circ (f \oplus g) \text{ is continuous }                                                                                                      \\
                 & \text{from } (X, d) \text{ to } (\mathbf{R}, d_{l^1}|_{\mathbf{R} \times \mathbf{R}})       & \text{(by Corollary \ref{2.1.7}(b))}               \\
        \implies & P(f, g) \text{ is continuous }                                                                                                                   \\
                 & \text{from } (X, d) \text{ to } (\mathbf{R}, d_{l^1}|_{\mathbf{R} \times \mathbf{R}}).      & \text{(by the definition of \(P\))}
    \end{align*}
\end{proof}

\begin{exercise}\label{ex 2.2.6}
    Let \(\mathbf{R}^m\) and \(\mathbf{R}^n\) be Euclidean spaces.
    If \(f : X \to \mathbf{R}^m\) and \(g : X \to \mathbf{R}^n\) are continuous functions, show that \(f \oplus g : X \to \mathbf{R}^{m + n}\) is also continuous, where we have identified \(\mathbf{R}^m \times \mathbf{R}^n\) with \(\mathbf{R}^{m + n}\) in the obvious manner.
    Is the converse statement true?
\end{exercise}

\begin{proof}
    Let \((X, d)\) be a metric space.
    For each \(k \in \mathbf{Z}^+\), let \(d_k\) be one of the metric functions \(d_{l^1}|_{\mathbf{R}^k \times \mathbf{R}^k}\), \(d_{l^2}|_{\mathbf{R}^k \times \mathbf{R}^k}\) or \(d_{l^\infty}|_{\mathbf{R}^k \times \mathbf{R}^k}\).
    Let \(f : X \to \mathbf{R}^m\) be a continuous function from \((X, d)\) to \((\mathbf{R}^m, d_m)\), and let \(g : X \to \mathbf{R}^n\) be a continuous function from \((X, d)\) to \((\mathbf{R}^n, d_n)\).
    Let \(x_0 \in X\).
    Then we have
    \begin{align*}
             & \begin{cases}
            f \text{ is continuous from } (X, d) \text{ to } (\mathbf{R}^m, d_m) \\
            g \text{ is continuous from } (X, d) \text{ to } (\mathbf{R}^n, d_n)
        \end{cases}                                                                                                                                \\
        \iff & \text{every sequence } (x^{(k)})_{k = 1}^\infty \text{ in } X \text{ satisfies the following:}                                                            \\
             & \lim_{k \to \infty} d\big(x^{(k)}, x_0\big) = 0 \text{ implies }                                                                                          \\
             & \begin{cases}
            \lim_{k \to \infty} d_m\big(f(x^{(k)}), f(x_0)\big) = 0 \\
            \lim_{k \to \infty} d_n\big(g(x^{(k)}), g(x_0)\big) = 0
        \end{cases}                                                                                        & \text{(by Theorem \ref{2.1.4}(a)(b))} \\
        \iff & \text{every sequence } (x^{(k)})_{k = 1}^\infty \text{ in } X \text{ satisfies the following:}                                                            \\
             & \lim_{k \to \infty} d\big(x^{(k)}, x_0\big) = 0 \text{ implies }                                                                                          \\
             & \lim_{k \to \infty} d_{m + n}\Big(\big(f(x^{(k)}) \oplus g(x^{(k)})\big), \big(f(x_0) \oplus g(x_0)\big)\Big) = 0 & \text{(by Proposition \ref{1.1.18})}  \\
        \iff & f \oplus g \text{ is continuous at } x_0                                                                                                                  \\
             & \text{from } (X, d) \text{ to } (\mathbf{R}^{m + n}, d_{m + n}).                                                  & \text{(by Theorem \ref{2.1.4}(a)(b))}
    \end{align*}
    Thus the statment is true and the converse is also true.
\end{proof}

\begin{exercise}\label{ex 2.2.7}
    Let \(k \geq 1\), let \(I\) be a finite subset of \(\mathbf{N}^k\), and let \(c : I \to \mathbf{R}\) be a function.
    Form the function \(P : \mathbf{R}^k \to \mathbf{R}\) defined by
    \[
        P(x_1, \dots, x_k) \coloneqq \sum_{(i_1, \dots, i_k) \in I} c(i_1, \dots, i_k) x_1^{i_1} \dots x_k^{i_k}.
    \]
    (Such a function is known as a \emph{polynomial of \(k\) variables};
    Show that \(P\) is continuous.
\end{exercise}

\begin{proof}
    For each \(k \in \mathbf{Z}^+\), let \(I_k\) be the finite subset of \(\mathbf{N}^k\), let \(d_k = d_{l^1}|_{\mathbf{R}^k \times \mathbf{R}^k}\), and let \(P_k\) be a polynomial of \(k\) variables, i.e.,
    \[
        P_k(x_1, \dots, x_k) = \sum_{(i_1, \dots, i_k) \in I_k} \big(c(i_1, \dots, i_k) x_1^{i_1} \cdots x_k^{i_k}\big).
    \]
    We use induction on \(k\) to show that \(P_k\) is continuous from \((\mathbf{R}^k, d_k)\) to \((\mathbf{R}, d_1)\) for every \(k \in \mathbf{Z}^+\).
    We start with \(k = 1\).
    For \(k = 1\), we have
    \begin{align*}
                 & \forall\ i_1 \in I_1, c(i_1) x_1^{i_1} \text{ is continuous from } (\mathbf{R}, d_1) \text{ to } (\mathbf{R}, d_1) & \text{(by Lemma \ref{2.2.2}(b))}     \\
        \implies & P_1(x_1) = \sum_{i_1 \in I_1} \big(c(i_1) x_1^{i_1}\big) \text{ is continuous}                                     & \text{(note that \(I_1\) is finite)} \\
                 & \text{from } (\mathbf{R}, d_1) \text{ to } (\mathbf{R}, d_1)                                                       & \text{(by Lemma \ref{2.2.2}(b))}
    \end{align*}
    and thus the base case holds.
    Suppose inductively that for some \(k \geq 1\), \(P_k\) is continuous from \((\mathbf{R}^k, d_k)\) to \((\mathbf{R}, d_1)\).
    Then for \(k + 1\), we need to show that \(P_{k + 1}\) is continuous from \((\mathbf{R}^{k + 1}, d_{k + 1})\) to \((\mathbf{R}, d_1)\).
    Let \(F : \mathbf{R}^k \to 2^\mathbf{R}\) be the function
    \[
        \forall\ (x_1, \dots, x_k, x_{k + 1}) \in \mathbf{R}^{k + 1}, F(x_1, \dots, x_k) = \big\{x_{k + 1} \in \mathbf{R} : (x_1, \dots, x_k, x_{k + 1}) \in I_{k + 1}\big\},
    \]
    and let \(A = \big\{(i_1, \dots, i_k) \in \mathbf{R}^k : (i_1, \dots, i_k, i_{k + 1}) \in I_{k + 1}\big\}\).
    Then we have
    \begin{align*}
         & P_{k + 1}(x_1, \dots, x_k, x_{k + 1})                                                                                                                                                                                \\
         & = \sum_{(i_1 \dots, i_k, i_{k + 1}) \in I_{k + 1}} c(i_1, \dots, i_k, i_{k + 1}) x_1^{i_1} \cdots x_k^{i_k} x_{k + 1}^{i_{k + 1}}                                                                                    \\
         & = \sum_{(i_1 \dots, i_k) \in A} \bigg(\sum_{i_{k + 1} \in \mathbf{R} : (i_1, \dots, i_k, i_{k + 1}) \in I_{k + 1}} c(i_1, \dots, i_k, i_{k + 1}) x_1^{i_1} \cdots x_k^{i_k} x_{k + 1}^{i_{k + 1}}\bigg)              \\
         & = \sum_{(i_1 \dots, i_k) \in A} \Bigg(x_1^{i_1} \cdots x_k^{i_k} \bigg(\sum_{i_{k + 1} \in \mathbf{R} : (i_1, \dots, i_k, i_{k + 1}) \in I_{k + 1}} c(i_1, \dots, i_k, i_{k + 1}) x_{k + 1}^{i_{k + 1}}\bigg)\Bigg).
    \end{align*}
    By induction hypothesis we know that
    \[
        \sum_{i_{k + 1} \in \mathbf{R} : (i_1, \dots, i_k, i_{k + 1}) \in I_{k + 1}} c(i_1, \dots, i_k, i_{k + 1}) x_{k + 1}^{i_{k + 1}}
    \]
    is continuous from \((\mathbf{R}, d_1)\) to \((\mathbf{R}, d_1)\) and
    \[
        x_1^{i_1} \cdots x_k^{i_k}
    \]
    is continuous from \((\mathbf{R}^k, d_k)\) to \((\mathbf{R}, d_1)\).
    Thus by Lemma \ref{2.2.2}(b) we know that
    \begin{align*}
                 & \forall\ (i_1, \dots, i_k, i_{k + 1}) \in I_{k + 1},                                                                                                                                                              \\
                 & (x_1^{i_1} \cdots x_k^{i_k}) \bigg(\sum_{i_{k + 1} \in \mathbf{R} : (i_1, \dots, i_k, i_{k + 1}) \in I_{k + 1}} c(i_1, \dots, i_k, i_{k + 1}) x_{k + 1}^{i_{k + 1}}\bigg)                                         \\
                 & \text{is continuous from } (\mathbf{R}^{k + 1}, d_{k + 1}) \text{ to } (\mathbf{R}, d_1)                                                                                                                          \\
        \implies & \sum_{(i_1 \dots, i_k) \in A} \Bigg(x_1^{i_1} \cdots x_k^{i_k} \bigg(\sum_{i_{k + 1} \in \mathbf{R} : (i_1, \dots, i_k, i_{k + 1}) \in I_{k + 1}} c(i_1, \dots, i_k, i_{k + 1}) x_{k + 1}^{i_{k + 1}}\bigg)\Bigg) \\
                 & \text{is continuous from } (\mathbf{R}^{k + 1}, d_{k + 1}) \text{ to } (\mathbf{R}, d_1)                                                                                                                          \\
        \implies & P_{k + 1} \text{ is continuous from } (\mathbf{R}^{k + 1}, d_{k + 1}) \text{ to } (\mathbf{R}, d_1)
    \end{align*}
    and this close the induction.
\end{proof}

\begin{exercise}\label{ex 2.2.8}
    Let \((X, d_X)\) and \((Y, d_Y)\) be metric spaces.
    Define the metric \(d_{X \times Y} : (X \times Y) \times (X \times Y) \to [0, \infty)\) by the formula
    \[
        d_{X \times Y}\big((x, y), (x', y')\big) \coloneqq d_X(x, x') + d_Y(y, y').
    \]
    Show that \((X \times Y, d_{X \times Y})\) is a metric space, and deduce an analogue of Proposition \ref{1.1.18} and Lemma \ref{2.2.1}.
\end{exercise}

\begin{proof}
    We first show that \((X \times Y, d_{X \times Y})\) is a metric space.
    For any \((x, y) \in X \times Y\), we have
    \begin{align*}
        d_{X \times Y}\big((x, y), (x, y)\big) & = d_X(x, x) + d_Y(y, y)                                         \\
                                               & = 0 + 0 = 0             & \text{(by Definition \ref{1.1.2}(a))}
    \end{align*}
    and thus \((X \times Y, d_{X \times Y})\) satisfies Definition \ref{1.1.2}(a).
    For any \((x_1, y_1), (x_2, y_2) \in X \times Y\), we have
    \begin{align*}
                 & (x_1, y_1) \neq (x_2, y_2)                                                                                                    \\
        \implies & (x_1 \neq x_2) \lor (y_1 \neq y_2)                                                                                            \\
        \implies & d_{X \times Y}\big((x_1, y_1), (x_2, y_2)\big) = d_X(x_1, x_2) + d_Y(y_1, y_2) \neq 0 & \text{(by Definition \ref{1.1.2}(b))}
    \end{align*}
    and thus \((X \times Y, d_{X \times Y})\) satisfies Definition \ref{1.1.2}(b).
    For any \((x_1, y_1), (x_2, y_2) \in X \times Y\), we have
    \begin{align*}
         & d_{X \times Y}\big((x_1, y_1), (x_2, y_2)\big)                                           \\
         & = d_X(x_1, x_2) + d_Y(y_1, y_2)                                                          \\
         & = d_X(x_2, x_1) + d_Y(y_2, y_1)                  & \text{(by Definition \ref{1.1.2}(c))} \\
         & = d_{X \times Y}\big((x_2, y_2), (x_1, y_1)\big)
    \end{align*}
    and thus \((X \times Y, d_{X \times Y})\) satisfies Definition \ref{1.1.2}(c).
    For any \((x_1, y_1), (x_2, y_2), (x_3, y_3) \in X \times Y\), we have
    \begin{align*}
         & d_{X \times Y}\big((x_1, y_1), (x_2, y_2)\big) + d_{X \times Y}\big((x_2, y_2), (x_3, y_3)\big)                                         \\
         & = d_X(x_1, x_2) + d_Y(y_1, y_2) + d_X(x_2, x_3) + d_Y(y_2, y_3)                                                                         \\
         & \geq d_X(x_1, x_3) + d_Y(y_1, y_3)                                                              & \text{(by Definition \ref{1.1.2}(d))} \\
         & = d_{X \times Y}\big((x_1, y_1), (x_3, y_3)\big)
    \end{align*}
    and thus \((X \times Y, d_{X \times Y})\) satisfies Definition \ref{1.1.2}(d).
    By Definition \ref{1.1.2} we conclude that \((X \times Y, d_{X \times Y})\) is a metric space.

    Next we propose an analogue of Proposition \ref{1.1.18} and proof it.
    Let \((X, d_X)\), \((Y, d_Y)\) be two metric spaces, let \((x, y) \in X \times Y\) and let \((x^{(n)}, y^{(n)})_{n = 1}^\infty\) be a sequence in \(X \times Y\).
    We claim that the follow two statements are equivalent:
    \begin{itemize}
        \item \(\lim_{n \to \infty} d_{X \times Y}\big((x^{(n)}, y^{(n)}), (x, y)\big) = 0\).
        \item \(\lim_{n \to \infty} d_X(x^{(n)}, x) = 0\) and \(\lim_{n \to \infty} d_Y(y^{(n)}, y) = 0\).
    \end{itemize}
    The claim is true since
    \begin{align*}
             & \lim_{n \to \infty} d_{X \times Y}\big((x^{(n)}, y^{(n)}), (x, y)\big) = 0 \\
        \iff & \lim_{n \to \infty} \big(d_X(x^{(n)}, x) + d_Y(y^{(n)}, y)\big) = 0        \\
        \iff & \begin{cases}
            0 \leq \lim_{n \to \infty} d_X(x^{(n)}, x) \leq \lim_{n \to \infty} \big(d_X(x^{(n)}, x) + d_Y(y^{(n)}, y)\big) \leq 0 \\
            0 \leq \lim_{n \to \infty} d_Y(y^{(n)}, y) \leq \lim_{n \to \infty} \big(d_X(x^{(n)}, x) + d_Y(y^{(n)}, y)\big) \leq 0
        \end{cases}                                                 \\
        \iff & \begin{cases}
            \lim_{n \to \infty} d_X(x^{(n)}, x) = 0 \\
            \lim_{n \to \infty} d_Y(y^{(n)}, y) = 0
        \end{cases}
    \end{align*}

    Finally we propose an analogue of Lemma \ref{2.2.1} and proof it.
    Let \((X, d_X)\), \((Y_1, d_{Y_1})\), \((Y_2, d_{Y_2})\) be metric spaces, let \(f_1 : X \to Y_1\) and \(f_2 : X \to Y_2\) be two functions, let \(f_1 \oplus f_2 : X \to (Y_1 \times Y_2)\) and let \(x_0 \in X\).
    We claim that the follow two statements are equivalent:
    \begin{itemize}
        \item \(f_1\) is continuous at \(x_0\) from \((X, d_X)\) to \((Y_1, d_{Y_1})\) and \(f_2\) is continuous at \(x_0\) from \((X, d_X)\) to \((Y_2, d_{Y_2})\).
        \item \(f_1 \oplus f_2\) is continuous at \(x_0\) from \((X, d_X)\) to \((Y_1 \times Y_2, d_{Y_1 \times Y_2})\).
    \end{itemize}
    The claim is true since by Definition \ref{2.1.1} we have
    \begin{align*}
             & \begin{cases}
            f_1 \text{ is continuous at } x_0 \text{ from } (X, d_X) \text{ to } (Y_1, d_{Y_1}) \\
            f_2 \text{ is continuous at } x_0 \text{ from } (X, d_X) \text{ to } (Y_2, d_{Y_2})
        \end{cases}                                                                                                                             \\
        \iff & \forall\ \varepsilon \in \mathbf{R}^+, \exists\ \delta_1, \delta_2 \in \mathbf{R}^+ :                                                                  \\
             & \begin{cases}
            \big(\forall\ x \in X, d_X(x, x_0) < \delta_1 \implies d_{Y_1}\big(f_1(x), f_1(x_0)\big) < \frac{\varepsilon}{2}) \\
            \big(\forall\ x \in X, d_X(x, x_0) < \delta_2 \implies d_{Y_2}\big(f_2(x), f_2(x_0)\big) < \frac{\varepsilon}{2})
        \end{cases}                                                                                                                             \\
        \iff & \forall\ \varepsilon \in \mathbf{R}^+, \exists\ \delta = \min(\delta_1, \delta_2) \in \mathbf{R}^+ :                                                   \\
             & \big(\forall\ x \in X, d_X(x, x_0) < \delta \implies d_{Y_1 \times Y_2}\Big(\big(f_1(x), f_2(x)\big), \big(f_1(x_0), f_2(x_0)\big)\Big) < \varepsilon) \\
        \iff & f_1 \oplus f_2 \text{ is continuous at } x_0 \text{ from } (X, d_X) \text{ to } (Y_1 \times Y_2, d_{Y_1 \times Y_2}).
    \end{align*}
\end{proof}

\begin{exercise}\label{ex 2.2.9}
    Let \(f : \mathbf{R}^2 \to \mathbf{R}\) be a function from \(\mathbf{R}^2\) to \(\mathbf{R}\).
    Let \((x_0, y_0)\) be a point in \(\mathbf{R}^2\).
    If \(f\) is continuous at \((x_0, y_0)\), show that
    \[
        \lim_{x \to x_0} \limsup_{y \to y_0} f(x, y) = \lim_{y \to y_0} \limsup_{x \to x_0} f(x, y) = f(x_0, y_0)
    \]
    and
    \[
        \lim_{x \to x_0} \liminf_{y \to y_0} f(x, y) = \lim_{y \to y_0} \liminf_{x \to x_0} f(x, y) = f(x_0, y_0).
    \]
    Recall that
    \begin{align*}
         & \limsup_{x \to x_0} f(x) \coloneqq \inf_{r > 0} \sup_{\abs*{x - x_0} < r} f(x) \\
         & \liminf_{x \to x_0} f(x) \coloneqq \sup_{r > 0} \inf_{\abs*{x - x_0} < r} f(x)
    \end{align*}
    In particular, we have
    \[
        \lim_{x \to x_0} \lim_{y \to y_0} f(x, y) = \lim_{y \to y_0} \lim_{x \to x_0} f(x, y)
    \]
    whenever the limits on both sides exist.
    (Note that the limits do not necessarily exist in general.)
    Discuss the comparison between this result and Example 1.2.7.
\end{exercise}

\begin{exercise}\label{ex 2.2.10}
    Let \(f : \mathbf{R}^2 \to \mathbf{R}\) be a continuous function.
    Show that for each \(x \in \mathbf{R}\), the function \(y \mapsto f(x, y)\) is continuous on \(\mathbf{R}\), and for each \(y \in \mathbf{R}\), the function \(x \mapsto f(x, y)\) is continuous on \(\mathbf{R}\).
    Thus a function \(f(x, y)\) which is jointly continuous in \((x, y)\) is also continuous in each variable \(x, y\) separately.
\end{exercise}

\begin{exercise}\label{ex 2.2.11}
    Let \(f : \mathbf{R}^2 \to \mathbf{R}\) be the function defined by \(f(x, y) = \frac{xy}{x^2 + y^2}\) when \((x, y) \neq (0, 0)\), and \(f(x, y) = 0\) otherwise.
    Show that for each fixed \(x \in \mathbf{R}\), the function \(y \mapsto f(x, y)\) is continuous on \(\mathbf{R}\), and that for each fixed \(y \in \mathbf{R}\), the function \(x \mapsto f(x, y)\) is continuous on \(\mathbf{R}\), but that the function \(f : \mathbf{R}^2 \to \mathbf{R}\) is not continuous on \(\mathbf{R}^2\).
    This shows that the converse to Exercise \ref{ex 2.2.10} fails;
    it is possible to be continuous in each variable separately without being jointly continuous.
\end{exercise}

\begin{exercise}\label{ex 2.2.12}
    Let \(f: \mathbf{R}^2 \to \mathbf{R}\) be the function defined by \(f(x, y) \coloneqq x^2 / y\) when \(y \neq 0\), and \(f(x, y) \coloneqq 0\) when \(y = 0\).
    Show that \(\lim_{t \to 0} f(tx, ty) = f(0, 0)\) for every \((x, y) \in \mathbf{R}^2\), but that \(f\) is not continuous at the origin.
    Thus being continuous on every line through the origin is not enough to guarantee continuity at the origin!
\end{exercise}
\section{Continuity and compactness}\label{sec 2.3}
\section{Continuity and connectedness}\label{sec 2.4}
\chapter{Uniform convergence}\label{ch 3}

\begin{note}
    It turns out that there are several different concepts of convergence of functions;
    here we describe the two most important ones, \emph{pointwise convergence} and \emph{uniform convergence}.
    (There are other types of convergence for functions, such as \(L^1\) convergence, \(L^2\) convergence, convergence in measure, almost everywhere convergence, and so forth, but these are beyond the scope of this text.)
    The two notions are related, but not identical;
    the relationship between the two is somewhat analogous to the relationship between continuity and uniform continuity.
\end{note}

\section{Limiting values of functions}\label{sec 3.1}

\begin{definition}[Limiting value of a function]\label{3.1.1}
    Let \((X, d_X)\) and \((Y, d_Y)\) be metric spaces, let \(E\) be a subset of \(X\), and let \(f : E \to Y\) be a function.
    If \(x_0 \in X\) is an adherent point of \(E\), and \(L \in Y\), we say that \emph{\(f(x)\) converges to \(L\) in \(Y\) as \(x\) converges to \(x_0\) in \(E\)}, or write \(\lim_{x \to x_0 ; x \in E} f(x) = L\), if for every \(\varepsilon > 0\) there exists a \(\delta > 0\) such that \(d_Y\big(f(x), L\big) < \varepsilon\) for all \(x \in E\) such that \(d_X(x, x_0) < \delta\).
\end{definition}

\begin{remark}\label{3.1.2}
    Some authors exclude the case \(x = x_0\) from the above definition, thus requiring \(0 < d_X(x, x_0) < \delta\).
    In our current notation, this would correspond to removing \(x_0\) from \(E\), thus one would consider
    \[
        \lim_{x \to x_0 ; x \in E \setminus \{x_0\}} f(x)
    \]
    instead of
    \[
        \lim_{x \to x_0 ; x \in E} f(x).
    \]
\end{remark}

\begin{note}
    Comparing this with Definition \ref{2.1.1}, we see that \(f\) is continuous at \(x_0\) if and only if
    \[
        \lim_{x \to x_0 ; x \in X} f(x) = f(x_0).
    \]
    Thus \(f\) is continuous on \(X\) iff we have
    \[
        \lim_{x \to x_0 ; x \in X} f(x) = f(x_0) \text{ for all } x_0 \in X.
    \]
\end{note}

\setcounter{theorem}{3}
\begin{remark}\label{3.1.4}
    Often we shall omit the condition \(x \in X\), and abbreviate
    \[
        \lim_{x \to x_0 ; x \in X} f(x)
    \]
    as simply
    \[
        \lim_{x \to x_0} f(x)
    \]
    when it is clear what space \(x\) will range in.
\end{remark}

\begin{proposition}\label{3.1.5}
    Let \((X, d_X)\) and \((Y, d_Y)\) be metric spaces, let \(E\) be a subset of \(X\), and let \(f : E \to Y\) be a function.
    Let \(x_0 \in X\) be an adherent point of \(E\) and \(L \in Y\).
    Then the following four statements are logically equivalent:
    \begin{enumerate}
        \item \(\lim_{x \to x_0 ; x \in E} f(x) = L\).
        \item For every sequence \((x^{(n)})_{n = 1}^\infty\) in \(E\) which converges to \(x_0\) with respect to the metric \(d_X\), the sequence \(\big(f(x^{(n)})\big)_{n = 1}^\infty\) converges to \(L\) with respect to the metric \(d_Y\).
        \item For every open set \(V \subseteq Y\) which contains \(L\), there exists an open set \(U \subseteq X\) containing \(x_0\) such that \(f(U \cap E) \subseteq V\).
        \item If one defines the function \(g : E \cup \{x_0\} \to Y\) by defining \(g(x_0) \coloneqq L\), and \(g(x) \coloneqq f(x)\) for \(x \in E \setminus \{x_0\}\), then \(g\) is continuous at \(x_0\).
              Furthermore, if \(x_0 \in E\), then \(f(x_0) = L\).
    \end{enumerate}
\end{proposition}

\begin{proof}
    We first show that statement (a) implies statement (b).
    Suppose that
    \[
        d_Y - \lim_{x \to x_0 ; x \in E} f(x) = L.
    \]
    By Definition \ref{3.1.1} we have
    \[
        \forall\ \varepsilon \in \mathbf{R}^+, \exists\ \delta \in \mathbf{R}^+ : \Big(\forall\ x \in E, d_X(x, x_0) < \delta \implies d_Y\big(f(x), L\big) < \varepsilon\Big).
    \]
    Let \((x^{(n)})_{n = 1}^\infty\) be a sequence in \(E\) such that \(\lim_{n \to \infty} d_X(x^{(n)}, x_0) = 0\).
    By Definition \ref{1.1.14} we have
    \[
        \forall\ \delta \in \mathbf{R}^+, \exists\ N \in \mathbf{Z}^+ : \forall\ n \geq N, d_X(x^{(n)}, x_0) < \delta.
    \]
    Since \((x^{(n)})_{n = 1}^\infty\) is in \(E\), we have
    \[
        \forall\ \varepsilon \in \mathbf{R}^+, \exists\ \delta \in \mathbf{R}^+ : \begin{cases}
            \exists\ N \in \mathbf{Z}^+ : \forall\ n \geq N, d_X(x^{(n)}, x_0) < \delta \\
            d_X(x^{(n)}, x_0) < \delta \implies d_Y\big(f(x^{(n)}), L\big) < \varepsilon
        \end{cases}
    \]
    and
    \[
        \forall\ \varepsilon \in \mathbf{R}^+, \exists\ N \in \mathbf{Z}^+ : \forall\ n \geq N, d_Y\big(f(x^{(n)}, L)\big) < \varepsilon.
    \]
    By Definition \ref{1.1.14} we have \(\lim_{n \to \infty} d_Y\big(f(x^{(n)}), L\big) = 0\).
    Since \((x^{(n)})_{n = 1}^\infty\) is arbitrary, we conclude that (a) implies (b).

    Next we show that statement (b) implies statement (a).
    Suppose that if \((x^{(n)})_{n = 1}^\infty\) is a sequence in \(X\) such that \(\lim_{n \to \infty} d_X(x^{(n)}, x_0) = 0\), then \(\lim_{n \to \infty} d_Y\big(f(x), L\big) = 0\).
    Suppose for sake of contradiction that
    \[
        d_Y - \lim_{x \to x_0 ; x \in X} f(x) \neq L.
    \]
    Then by Definition \ref{3.1.1} we have
    \[
        \exists\ \varepsilon \in \mathbf{R}^+ : \forall\ \delta \in \mathbf{R}^+, \exists\ x \in X : \begin{cases}
            d_X(x, x_0) < \delta \\
            d_Y\big(f(x), L\big) \geq \varepsilon
        \end{cases}
    \]
    Thus we can choose one sequence \((x^{(n)})_{n = 1}^\infty\) which satsifies
    \[
        \forall\ n \in \mathbf{Z}^+, \begin{cases}
            d_X(x^{(n)}, x_0) < \frac{1}{n} \\
            d_Y\big(f(x^{(n)}), L\big) \geq \varepsilon
        \end{cases}
    \]
    By squeeze test we have \(\lim_{n \to \infty} d_X(x^{(n)}, x_0) = 0\).
    But by hypothesis we know that \(\lim_{n \to \infty} d_Y\big(f(x^{(n)}), L\big) = 0\), which means
    \[
        \exists\ N \in \mathbf{Z}^+ : \forall\ n \geq N, d_Y\big(f(x^{(n)}), L\big) < \varepsilon,
    \]
    a contradiction.
    Thus we have
    \[
        d_Y - \lim_{x \to x_0 ; x \in X} f(x) = L
    \]
    and we conclude that statements (a)(b) are equivalent.

    Next we show that statement (a) implies statement (c).
    Suppose that
    \[
        d_Y - \lim_{x \to x_0 ; x \in E} f(x) = L.
    \]
    By Definition \ref{3.1.1} we have
    \begin{align*}
                 & \forall\ \varepsilon \in \mathbf{R}^+, \exists\ \delta \in \mathbf{R}^+ : \Big(\forall\ x \in E, d_X(x, x_0) < \delta \implies d_Y\big(f(x), L\big) < \varepsilon\Big)  \\
        \implies & \forall\ \varepsilon \in \mathbf{R}^+, \exists\ \delta \in \mathbf{R}^+ : \Big(x \in B_{(X, d_X)}(x_0, \delta) \cap E \implies d_Y\big(f(x), L\big) < \varepsilon\Big).
    \end{align*}
    Let \(V\) be an open set in \((Y, d_Y)\) such that \(L \in V\).
    Then we have
    \begin{align*}
                 & V = \text{int}_{(Y, d_Y)}(V)                                                     & \text{(by Proposition \ref{1.2.15}(a))} \\
        \implies & \exists\ \varepsilon \in \mathbf{R}^+ : B_{(Y, d_Y)}(L, \varepsilon) \subseteq V & \text{(by Definition \ref{1.2.5})}      \\
        \implies & \exists\ \delta \in \mathbf{R}^+ :                                                                                         \\
                 & \begin{cases}
            x \in B_{(X, d_X)}(x_0, \delta) \cap E \implies d_Y\big(f(x), L\big) < \varepsilon \\
            f\big(B_{(X, d_X)}(x_0, \delta) \cap E\big) \subseteq B_{(Y, d_Y)}\big(L, \varepsilon\big) \subseteq V
        \end{cases}
    \end{align*}
    and by Proposition \ref{1.2.15}(c) we know that \(B_{(X, d_X)}(x_0, \delta)\) is open in \((X, d_X)\).
    Since \(V\) is arbitrary, we conclude that statement (a) implies statement (c).

    Next we show that statement (c) implies statement (a).
    Suppose that
    \[
        \forall\ V \subseteq Y, \begin{cases}
            L \in V \\
            V \text{ is open in } (Y, d_Y)
        \end{cases} \implies \exists\ U \subseteq X : \begin{cases}
            x_0 \in U                      \\
            U \text{ is open in } (X, d_X) \\
            f(U \cap E) \subseteq V
        \end{cases}
    \]
    Let \(\varepsilon \in \mathbf{R}^+\).
    By Proposition \ref{1.2.15}(c) we know that \(B_{(Y, d_Y)}(L, \varepsilon)\) is open in \((Y, d_Y)\).
    By hypothesis we know that there exists some \(U \subseteq X\) such that
    \[
        \begin{cases}
            x_0 \in U                      \\
            U \text{ is open in } (X, d_X) \\
            f(U \cap E) \subseteq B_{(Y, d_Y)}(L, \varepsilon)
        \end{cases}
    \]
    Then we have
    \begin{align*}
                 & \begin{cases}
            x_0 \in U \\
            U = \text{int}_{(X, d_X)}(U)
        \end{cases}                                                                               & \text{(by Proposition \ref{1.2.15}(a))} \\
        \implies & \exists\ \delta \in \mathbf{R}^+ : B_{(X, d_X)}(x_0, \delta) \subseteq U                                 & \text{(by Definition \ref{1.2.5})}      \\
        \implies & \exists\ \delta \in \mathbf{R}^+ : B_{(X, d_X)}(x_0, \delta) \cap E \subseteq U \cap E                                                             \\
        \implies & \exists\ \delta \in \mathbf{R}^+ :                                                                                                                 \\
                 & f\big(B_{(X, d_X)}(x_0, \delta) \cap E\big) \subseteq f(U \cap E) \subseteq B_{(Y, d_Y)}(L, \varepsilon)                                           \\
        \implies & \exists\ \delta \in \mathbf{R}^+ :                                                                                                                 \\
                 & \Big(\forall\ x \in E, d_X(x, x_0) < \delta \implies d_Y\big(f(x), L\big) < \varepsilon\Big).
    \end{align*}
    Since \(\varepsilon\) is arbitrary, by Definition \ref{3.1.1} we have
    \[
        d_Y - \lim_{x \to x_0 ; x \in E} f(x) = L
    \]
    and we conclude that statements (a)(c) are equivalent.

    Next we show that statement (a) implies statement (d).
    Suppose that
    \[
        d_Y - \lim_{x \to x_0 ; x \in E} f(x) = L.
    \]
    Then by Definition \ref{3.1.1} we have
    \[
        \forall\ \varepsilon \in \mathbf{R}^+, \exists\ \delta \in \mathbf{R}^+ : \Big(\forall\ x \in E, d_X(x, x_0) < \delta \implies d_Y\big(f(x), L\big) < \varepsilon\Big).
    \]
    Let \(g : E \cup \{x_0\} \to Y\) be a function where
    \[
        \forall\ x \in E \cup \{x_0\}, g(x) = \begin{cases}
            L    & \text{if } x = x_0    \\
            f(x) & \text{if } x \neq x_0
        \end{cases}
    \]
    Then we have
    \begin{align*}
                 & \forall\ \varepsilon \in \mathbf{R}^+, \exists\ \delta \in \mathbf{R}^+ :                                       \\
                 & \Big(\forall\ x \in E, d_X(x, x_0) < \delta \implies d_Y\big(f(x), L\big) < \varepsilon\Big)                    \\
        \implies & \forall\ \varepsilon \in \mathbf{R}^+, \exists\ \delta \in \mathbf{R}^+ :                                       \\
                 & \Big(\forall\ x \in E \cup \{x_0\}, d_X(x, x_0) < \delta \implies d_Y\big(g(x), g(x_0)\big) < \varepsilon\Big).
    \end{align*}
    Thus by Definition \ref{2.1.1} \(g\) is continuous at \(x_0\) from \((E \cup \{x_0\}, d_X|_{(E \cup \{x_0\}) \times (E \cup \{x_0\})})\) to \((Y, d_Y)\).

    Now suppose that \(x_0 \in E\).
    We claim that \(f(x_0) = L\).
    Suppose for sake of contradiction that \(f(x_0) \neq L\).
    Then by Definition \ref{1.1.2}(b) we have \(d_Y\big(f(x_0), L\big) > 0\).
    Let \(r = d_Y\big(f(x_0), L\big)\).
    By Definition \ref{3.1.1} we have
    \[
        \exists\ \delta \in \mathbf{R}^+ : \forall\ x \in E, d_X(x, x_0) < \delta \implies d_Y\big(f(x), L\big) < r.
    \]
    Since \(x_0 \in E\), we have \(d_X(x_0, x_0) = 0 < \delta\).
    But then we have \(d_Y\big(f(x_0), L\big) < r = d_Y\big(f(x_0), L\big)\), a contradiction.
    Thus we have \(f(x_0) = L\).

    Finally we show that statement (d) implies statement (a).
    Suppose that \(g : E \cup \{x_0\} \to Y\) is a function where
    \[
        \forall\ x \in E \cup \{x_0\}, g(x) = \begin{cases}
            L    & \text{if } x = x_0    \\
            f(x) & \text{if } x \neq x_0
        \end{cases}
    \]
    and \(g\) is continuous from \((E \cup \{x_0\}, d_X|_{(E \cup \{x_0\}) \times (E \cup \{x_0\})})\) to \((Y, d_Y)\).
    Then by Definition \ref{2.1.1} we have
    \begin{align*}
                 & \forall\ \varepsilon \in \mathbf{R}^+, \exists\ \delta \in \mathbf{R}^+ :                                      \\
                 & \Big(\forall\ x \in E \cup \{x_0\}, d_X(x, x_0) < \delta \implies d_Y\big(g(x), g(x_0)\big) < \varepsilon\Big) \\
        \implies & \forall\ \varepsilon \in \mathbf{R}^+, \exists\ \delta \in \mathbf{R}^+ :                                      \\
                 & \Big(\forall\ x \in E, d_X(x, x_0) < \delta \implies d_Y\big(f(x), L\big) < \varepsilon\Big).
    \end{align*}
    By Definition \ref{3.1.1} this means
    \[
        d_Y - \lim_{x \to x_0 ; x \in E} f(x) = L.
    \]
    We conclude that statements (a)(b)(c)(d) are all equivalent.
\end{proof}

\begin{remark}\label{3.1.6}
    Observe from Proposition \ref{3.1.5}(b) and Proposition \ref{1.1.20} that a function \(f(x)\) can converge to at most one limit \(L\) as \(x\) converges to \(x_0\).
    In other words, if the limit
    \[
        \lim_{x \to x_0 ; x \in E} f(x)
    \]
    exists at all, then it can only take at most one value.
\end{remark}

\begin{remark}\label{3.1.7}
    The requirement that \(x_0\) be an adherent point of \(E\) is necessary for the concept of limiting value to be useful, otherwise \(x_0\) will lie in the exterior of \(E\), the notion that \(f(x)\) converges to \(L\) as \(x\) converges to \(x_0\) in \(E\) is vacuous
    (for \(\delta\) sufficiently small, there are no points \(x \in E\) so that \(d(x, x_0) < \delta\)).
\end{remark}

\begin{remark}\label{3.1.8}
    Strictly speaking, we should write
    \[
        d_Y - \lim_{x \to x_0 ; x \in E} f(x) \text{ instead of } \lim_{x \to x_0 ; x \in E} f(x),
    \]
    since the convergence depends on the metric \(d_Y\).
    However in practice it will be obvious what the metric \(d_Y\) is and so we will omit the \(d_Y -\) prefix from the notation.
\end{remark}

\exercisesection

\begin{exercise}\label{ex 3.1.1}
    Let \((X, d_X)\) and \((Y, d_Y)\) be metric spaces, let \(E\) be a subset of \(X\), let \(f : E \to Y\) be a function, and let \(x_0\) be an element of \(E\).
    Assume that \(x_0\) is an adherent point of \(E \setminus \{x_0\}\)
    (or equivalently, that \(x_0\) is not an \emph{isolated point} of \(E\)).
    Show that the limit \(\lim_{x \to x_0 ; x \in E} f(x)\) exists if and only if the limit \(\lim_{x \to x_0 ; x \in E \setminus \{x_0\}} f(x)\) exists and is equal to \(f(x_0)\).
    Also, show that if the limit \(\lim_{x \to x_0 ; x \in E} f(x)\) exists at all, then it must equal \(f(x_0)\).
\end{exercise}

\begin{proof}
    Let \(L \in Y\).
    By Definition \ref{1.1.2}(a) we know that
    \[
        \forall\ \varepsilon \in \mathbf{R}^+, d_Y\big(f(x_0), L\big) < \varepsilon \iff L = f(x_0).
    \]
    Thus we have
    \begin{align*}
             & d_Y - \lim_{x \to x_0 ; x \in E \setminus \{x_0\}} f(x) = f(x_0)                                                                                         \\
        \iff & \forall\ \varepsilon \in \mathbf{R}^+, \exists\ \delta \in \mathbf{R}^+ :                                                                                \\
             & \Big(\forall\ x \in E \setminus \{x_0\}, d_X(x, x_0) < \delta \implies d_Y\big(f(x), f(x_0)\big) < \varepsilon\Big) & \text{(by Definition \ref{3.1.1})} \\
        \iff & \forall\ \varepsilon \in \mathbf{R}^+, \exists\ \delta \in \mathbf{R}^+ :                                                                                \\
             & \Big(\forall\ x \in E, d_X(x, x_0) < \delta \implies d_Y\big(f(x), f(x_0)\big) < \varepsilon\Big)                   & (E \setminus \{x_0\} \subseteq E)  \\
        \iff & d_Y - \lim_{x \to x_0 ; x \in E} f(x) = f(x_0).                                                                     & \text{(by Definition \ref{3.1.1})}
    \end{align*}
\end{proof}

\begin{exercise}\label{ex 3.1.2}
    Prove Proposition \ref{3.1.5}.
\end{exercise}

\begin{proof}
    See Proposition \ref{3.1.5}.
\end{proof}

\begin{exercise}\label{ex 3.1.3}
    Use Proposition \ref{3.1.5}(c) to define a notion of a limiting value of a function \(f : E \to Y\) from one topological space \((X, \mathcal{F}_X)\) to another \((Y, \mathcal{F}_Y)\) where \(E \subseteq X\).
    If \(X\) is a topological space and \(Y\) is a Hausdorff topological space (see Exercise \ref{ex 2.5.4}), prove the equivalence of Proposition \ref{3.1.5}(c) and \ref{3.1.5}(d) in this setting, as well as an analogue of Remark \ref{3.1.6}.
    What happens to these statements of \(Y\) is not Hausdorff?
\end{exercise}

\begin{proof}
    Let \((X, \mathcal{F}_X)\), \((Y, \mathcal{F}_Y)\) be topological spaces, let \(E \subseteq X\), let \(f : E \to Y\) be a function, let \(x_0 \in \overline{E}_{(X, \mathcal{F}_X)}\), and let \(L \in Y\).
    We say that \(f(x)\) converges to \(L\) in \(Y\) as \(x\) converges to \(x_0\) in \(E\) iff
    \[
        \forall\ V \in \mathcal{F}_Y, L \in V \implies \exists\ U \in \mathcal{F}_X : \begin{cases}
            x_0 \in U \\
            f(U \cap E) \subseteq V
        \end{cases}
    \]
    We want to show that if \((Y, \mathcal{F}_Y)\) is Hausdorff, then the definition above is equivalent to the follow:
    If \(g : E \cup \{x_0\} \to Y\) is a function such that
    \[
        \forall\ x \in E \cup \{x_0\}, g(x) = \begin{cases}
            L    & \text{if } x = x_0    \\
            f(x) & \text{if } x \neq x_0
        \end{cases}
    \]
    and \((E \cup \{x_0\}, \mathcal{F}_{E \cup \{x_0\}})\) is a topological subspace induced by \((X, \mathcal{F}_X)\), then \(g\) is continuous at \(x_0\) from \((E \cup \{x_0\}, \mathcal{F}_{E \cup \{x_0\}})\) to \((Y, \mathcal{F}_Y)\).

    First suppose that \(f(x)\) converges to \(L\) in \(Y\) as \(x\) converges to \(x_0\) in \(E\).
    Let \(g\) be the function in the definition and let \(V \in \mathcal{F}_Y\) such that \(L \in V\).
    By hypothesis we know that
    \[
        \exists\ U \in \mathcal{F}_X : \begin{cases}
            x_0 \in U \\
            f(U \cap E) \subseteq V
        \end{cases}
    \]
    Then by Definition \ref{2.5.7} we have \(U \cap (E \cup \{x_0\}) \in \mathcal{F}_{E \cup \{x_0\}}\) and
    \begin{align*}
        g\big(U \cap (E \cup \{x_0\})\big) & = g\big((U \cap E) \cup \{x_0\}\big)                      \\
                                           & = g\big((U \cap E) \setminus \{x_0\}\big) \cup g(\{x_0\}) \\
                                           & = f\big((U \cap E) \setminus \{x_0\}\big) \cup \{L\}      \\
                                           & \subseteq f(U \cap E) \cup \{L\}                          \\
                                           & \subseteq V.
    \end{align*}
    Since \(V\) is arbitrary, by Definition \ref{2.5.8} we know that \(g\) is continuous at \(x_0\) from \((E \cup \{x_0\}, \mathcal{F}_{E \cup \{x_0\}})\) to \((Y, \mathcal{F}_Y)\).

    Next suppose that \(x_0 \in E\) and \(f(x)\) converges to \(L\) in \(Y\) as \(x\) converges to \(x_0\) in \(E\).
    We want to show that \(f(x_0) = L\).
    Suppose for sake of contradiction that \(f(x_0) \neq L\).
    Since \((Y, \mathcal{F}_Y)\) is Hausdorff, by Exercise \ref{ex 2.5.4} we know that
    \[
        \exists\ V, W \in \mathcal{F}_Y : \begin{cases}
            L \in V      \\
            f(x_0) \in W \\
            V \cap W = \emptyset
        \end{cases}
    \]
    By definition we have
    \[
        \exists\ U_V, U_W \in \mathcal{F}_X : \begin{cases}
            x_0 \in U_V               \\
            x_0 \in U_W               \\
            f(U_V \cap E) \subseteq V \\
            f(U_W \cap E) \subseteq W
        \end{cases}
    \]
    By Definition \ref{2.5.1} we know that \(U_V \cap U_W \in \mathcal{F}_X\).
    But then we have
    \[
        \begin{cases}
            x_0 \in U_V \cap U_W                                       \\
            f(U_V \cap U_W \cap E) \subseteq f(U_V \cap E) \subseteq V \\
            f(U_V \cap U_W \cap E) \subseteq f(U_W \cap E) \subseteq W
        \end{cases}
    \]
    which means \(V \cap W \neq \emptyset\), a contradiction.
    Thus we have \(f(x_0) = L\).

    Now suppose that \(g\) is the function in the definition such that \(g\) is continuous at \(x_0\) from \((E \cup \{x_0\}, \mathcal{F}_{E \cup \{x_0\}})\) to \((Y, \mathcal{F}_Y)\).
    Also suppose that if \(x_0 \in E\), then \(f(x_0) = L\).
    Let \(V \in \mathcal{F}_Y\) such that \(g(x_0) = L \in V\).
    By Definition \ref{2.5.8} we know that
    \[
        \exists\ U \in \mathcal{F}_{E \cup \{x_0\}} : \begin{cases}
            x_0 \in U \\
            g(U) \subseteq V
        \end{cases}
    \]
    By Definition \ref{2.5.7} we know that
    \[
        \exists\ U_X \in \mathcal{F}_X : U_X \cap (E \cup \{x_0\}) = U.
    \]
    Since \(x_0 \in U\), we know that \(x_0 \in U_X\).
    Thus we have
    \begin{align*}
        f(U_X \cap E) & = f\big((U_X \cap E) \setminus \{x_0\}\big) \cup f(E \cap \{x_0\})  & (f(E \cap \{x_0\}) = \emptyset \iff x_0 \notin E) \\
                      & \subseteq g\big((U_X \cap E) \setminus \{x_0\}\big) \cup g(\{x_0\}) & (x_0 \in E \iff f(x_0) = L = g(x_0))              \\
                      & = g\big(U_X \cap (E \cup \{x_0\})\big)                                                                                  \\
                      & = g(U)                                                                                                                  \\
                      & \subseteq V.
    \end{align*}
    Since \(V\) is arbitrary, we conclude that \(f(x)\) converges to \(L\) in \(Y\) as \(x\) converges to \(x_0\) in \(E\).

    If \((Y, \mathcal{F}_Y)\) is not Hausdorff, then \(x_0 \in E\) may not implies \(f(x_0) = L\).
\end{proof}

\begin{exercise}\label{ex 3.1.4}
    Recall from Exercise \ref{ex 2.5.5} that the extended real line \(\mathbf{R}^*\) comes with a standard topology (the order topology).
    We view the natural numbers \(\mathbf{N}\) as a subspace of this topological space, and \(+\infty\) as an adherent point of \(\mathbf{N}\) in \(\mathbf{R}^*\).
    Let \((a_n)_{n = 0}^\infty\) be a sequence taking values in a topological space \((Y, \mathcal{F}_Y)\), and let \(L \in Y\).
    Show that \(\lim_{n \to +\infty ; n \in \mathbf{N}} a_n = L\) (in the sense of Exercise \ref{ex 3.1.3}) if and only if \(\lim_{n \to \infty} a_n = L\) (in the sense of Definition \ref{2.5.4}).
    This shows that the notions of limiting values of a sequence, and limiting values of a function, are compatible.
\end{exercise}

\begin{exercise}\label{ex 3.1.5}
    Let \((X, d_X)\), \((Y, d_Y)\), \((Z, d Z)\) be metric spaces, and let \(x_0 \in X\), \(y_0 \in Y\), \(z_0 \in Z\).
    Let \(f : E \to Y\) and \(g : Y \to Z\) be functions, and let \(E\) be a set.
    If we have \(\lim_{x \to x_0 ; x \in E} f(x) = y_0\) and \(\lim_{y \to y_0 ; y \in f(E)} g(y) = z_0\), conclude that \(\lim_{x \to x_0 ; x \in E} g \circ f(x) = z_0\).
\end{exercise}

\begin{exercise}\label{ex 3.1.6}
    State and prove an analogue of the limit laws in Proposition 9.3.14 in Analysis I when \(X\) is now a metric space rather than a subset of \(\mathbf{R}\).
\end{exercise}
\section{Pointwise and uniform convergence}\label{sec 3.2}

\begin{definition}[Pointwise convergence]\label{3.2.1}
    Let \((f^{(n)})_{n = 1}^\infty\) be a sequence of functions from one metric space \((X, d_X)\) to another \((Y, d_Y)\), and let \(f : X \to Y\) be another function.
    We say that \emph{\((f^{(n)})_{n = 1}^\infty\) converges pointwise to \(f\) on \(X\)} if we have
    \[
        \lim_{n \to \infty} f^{(n)}(x) = f(x)
    \]
    for all \(x \in X\), i.e.,
    \[
        \lim_{n \to \infty} d_Y\big(f^{(n)}(x), f(x)\big) = 0.
    \]
\end{definition}
\section{Uniform convergence and continuity}\label{sec 3.3}

\begin{theorem}[Uniform limits preserve continuity I]\label{3.3.1}
    Suppose \((f^{(n)})_{n = 1}^\infty\) is a sequence of functions from one metric space \((X, d_X)\) to another \((Y, d_Y)\), and suppose that this sequence converges uniformly to another function \(f : X \to Y\).
    Let \(x_0\) be a point in \(X\).
    If the functions \(f^{(n)}\) are continuous at \(x_0\) for each \(n\), then the limiting function \(f\) is also continuous at \(x_0\).
\end{theorem}

\begin{proof}
    We have
    \begin{align*}
                 & (f^{(n)})_{n = 1}^\infty \text{ converges uniformly to } f \text{ on } X                                                         \\
                 & \text{with respect to } d_Y                                                                                                      \\
        \implies & \forall\ \varepsilon \in \mathbf{R}^+, \exists\ N \in \mathbf{Z}^+ :                                                             \\
                 & \forall\ n \geq N, \forall\ x \in X, d_Y\big(f^{(n)}(x), f(x)\big) < \frac{\varepsilon}{3}. & \text{(by Definition \ref{3.2.7})}
    \end{align*}
    We choose one pair of \(\varepsilon\) and \(N\).
    For each \(n \in \mathbf{Z}^+\), since \(f^{(n)}\) is continuous at \(x_0\) from \((X, d_X)\) to \((Y, d_Y)\), by Definition \ref{2.1.1} we have
    \begin{align*}
                 & \forall\ n \geq N, f^{(n)} \text{ is continuous at } x_0 \text{ from } (X, d_X) \text{ to } (Y, d_Y)                                                                                        \\
        \implies & \forall\ n \geq N, \exists\ \delta \in \mathbf{R}^+ :                                                                                                                                       \\
                 & \Big(\forall\ x \in X, d_X(x, x_0) < \delta \implies d_Y\big(f^{(n)}(x), f^{(n)}(x_0)\big) < \frac{\varepsilon}{3}\Big)                                                                     \\
        \implies & \forall\ n \geq N, \exists\ \delta \in \mathbf{R}^+ :                                                                                                                                       \\
                 & \Big(\forall\ x \in X, d_X(x, x_0) < \delta \implies d_Y\big(f(x), f(x_0)\big)                                                                                                              \\
                 & \leq d_Y\big(f(x), f^{(n)}(x)\big) + d_Y\big(f^{(n)}(x), f^{(n)}(x_0)\big) + d_Y\big(f^{(n)}(x_0), f(x_0)\big) < \frac{\varepsilon}{3} + \frac{\varepsilon}{3} + \frac{\varepsilon}{3}\Big) \\
        \implies & \forall\ n \geq N, \exists\ \delta \in \mathbf{R}^+ :                                                                                                                                       \\
                 & \Big(\forall\ x \in X, d_X(x, x_0) < \delta \implies d_Y\big(f(x), f(x_0)\big) < \varepsilon.
    \end{align*}
    Since \(\varepsilon\) is arbitrary, by Definition \ref{2.1.1} we know that \(f\) is continuous at \(x_0\) from \((X, d_X)\) to \((Y, d_Y)\).
\end{proof}

\begin{corollary}[Uniform limits preserve continuity II]\label{3.3.2}
    Let \((f^{(n)})_{n = 1}^\infty\) be a sequence of functions from one metric space \((X, d_X)\) to another \((Y, d_Y)\), and suppose that this sequence converges uniformly to another function \(f : X \to Y\).
    If the functions \(f^{(n)}\) are continuous on \(X\) for each \(n\), then the limiting function \(f\) is also continuous on \(X\).
\end{corollary}

\begin{proof}
    By applying Theorem \ref{3.3.1} to each \(x \in X\) we conclude that \(f\) is continuous on \(X\) from \((X, d_X)\) to \((Y, d_Y)\).
\end{proof}

\begin{proposition}[Interchange of limits and uniform limits]\label{3.3.3}
    Let
    \((X, d_X)\) and \((Y, d_Y)\) be metric spaces, with \(Y\) complete, and let \(E\) be a subset of \(X\).
    Let \((f^{(n)})_{n = 1}^\infty\) be a sequence of functions from \(E\) to \(Y\), and suppose that this sequence converges uniformly in \(E\) to some function \(f : E \to Y\).
    Let \(x_0 \in X\) be an adherent point of \(E\), and suppose that for each \(n\) the limit \(\lim_{x \to x_0 ; x \in E} f^{(n)}(x)\) exists.
    Then the limit \(\lim_{x \to x_0 ; x \in E} f(x)\) also exists, and is equal to the limit of the sequence \(\big(\lim_{x \to x_0 ; x \in E} f^{(n)}(x)\big)_{n = 1}^\infty\);
    in other words we have the interchange of limits
    \[
        \lim_{n \to \infty} \lim_{x \to x_0 ; x \in E} f^{(n)}(x) = \lim_{x \to x_0 ; x \in E} \lim_{n \to \infty} f^{(n)}(x).
    \]
\end{proposition}

\begin{proof}
    For each \(n \in \mathbf{Z}^+\), we define \(d_Y - \lim_{x \to x_0 ; x \in E} f^{(n)}(x) = L^{(n)}\).
    We claim that the sequence \((L^{(n)})_{n = 1}^\infty\) converges in \(Y\) with respect to \(d_Y\).
    Since \((Y, d_Y)\) is complete, by Definition \ref{1.4.10} it suffices to show that \((L^{(n)})_{n = 1}^\infty\) is a Cauchy sequence in \((Y, d_Y)\).
    Let \(n_1, n_2 \in \mathbf{Z}^+\).
    Then by Definition \ref{3.2.7} we have
    \begin{align*}
                 & (f^{(n)})_{n = 1}^\infty \text{ converges uniformly to } f \text{ on } X \text{ with respect to } d_Y                                                           \\
        \implies & \forall\ \varepsilon \in \mathbf{R}^+, \exists\ N \in \mathbf{Z}^+ : \forall\ n \geq N, \forall\ x \in X, d_Y\big(f^{(n)}(x), f(x)\big) < \frac{\varepsilon}{4}
    \end{align*}
    Now fix one pair of \(\varepsilon\) and \(N\).
    Since \(L^{(n)}\) exists for each \(n \in \mathbf{N}\), by Definition \ref{3.1.1} we have
    \begin{align*}
                 & \forall\ n \geq N, d_Y - \lim_{x \to x_0 ; x \in E} f^{(n)}(x) = L^{(n)}                                                                                                 \\
        \implies & \forall\ n \geq N, \exists\ \delta \in \mathbf{R}^+ : \Big(\forall\ x \in X, d_X(x, x_0) < \delta \implies d_Y\big(f^{(n)}(x), L^{(n)}\big) < \frac{\varepsilon}{4}\Big) \\
        \implies & \forall\ n_1, n_2 \geq N, \exists\ \delta \in \mathbf{R}^+ :                                                                                                             \\
                 & \Big(\forall\ x \in X, d_X(x, x_0) < \delta \implies d_Y\big(L^{(n_1)}, L^{(n_2)}\big)                                                                                   \\
                 & \leq d_Y\big(L^{(n_1)}, f^{(n_1)}(x)\big) + d_Y\big(f^{(n_1)}(x), f(x)\big)                                                                                              \\
                 & \quad + d_Y\big(f(x), f^{(n_2)}(x)\big) + d_Y\big(f^{(n_2)}(x), L^{(n_2)}\big)                                                                                           \\
                 & < \frac{\varepsilon}{4} + \frac{\varepsilon}{4} + \frac{\varepsilon}{4} + \frac{\varepsilon}{4}\Big)                                                                     \\
        \implies & \forall\ n_1, n_2 \geq N, d_Y\big(L^{(n_1)}, L^{(n_2)}\big) < \varepsilon
    \end{align*}
    Since \(\varepsilon\) is arbitrary, we have
    \[
        \forall\ \varepsilon \in \mathbf{R}^+, \exists\ N \in \mathbf{Z}^+ : \forall\ n_1, n_2 \geq N, d_Y\big(L^{(n_1)}, L^{(n_2)}\big) < \varepsilon
    \]
    and by Definition \ref{1.4.6} \((L^{(n)})_{n = 1}^\infty\) is a Cauchy sequence in \((Y, d_Y)\).

    Let \(L \in Y\) such that \(d_Y - \lim_{n \to \infty} L^{(n)} = L\).
    Again by Definition \ref{3.2.7} we have
    \begin{align*}
                 & (f^{(n)})_{n = 1}^\infty \text{ converges uniformly to } f \text{ on } X \text{with respect to } d_Y                                                                 \\
        \implies & \forall\ \varepsilon \in \mathbf{R}^+, \exists\ N_1 \in \mathbf{Z}^+ : \forall\ n \geq N_1, \forall\ x \in X, d_Y\big(f^{(n)}(x), f(x)\big) < \frac{\varepsilon}{3}.
    \end{align*}
    Again we choose one pair of \(\varepsilon\) and \(N_1\).
    Since \(L\) exists, by Definition \ref{3.1.1} we have
    \begin{align*}
                 & \lim_{n \to \infty} d_Y\big(L^{(n)}, L\big) = 0                                                                                                                              \\
        \implies & \exists\ N_2 \in \mathbf{Z}^+ : \forall\ n \geq N_2, d_Y(L^{(n)}, L) < \frac{\varepsilon}{3}                                                                                 \\
        \implies & \exists\ N = \max(N_1, N_2) : \forall\ n \geq N,                                                                                                                             \\
                 & \begin{cases}
            \exists\ \delta \in \mathbf{R}^+ : \forall\ x \in X, d_X(x, x_0) < \delta \implies d_Y\big(f^{(n)}(x), L^{(n)}\big) < \frac{\varepsilon}{3} \\
            d_Y(L^{(n)}, L) < \frac{\varepsilon}{3}                                                                                                     \\
            \forall\ x \in X, d_Y\big(f^{(n)}(x), f(x)\big) < \frac{\varepsilon}{3}
        \end{cases}                                                                                                                                                   \\
        \implies & \exists\ N = \max(N_1, N_2) : \forall\ n \geq N, \exists\ \delta \in \mathbf{R}^+ :                                                                                          \\
                 & \Big(\forall\ x \in X, d_X(x, x_0) < \delta \implies d_Y\big(f(x), L\big)                                                                                                    \\
                 & \leq d_Y\big(f(x), f^{(n)}(x)\big) + d_Y\big(f^{(n)}(x), L^{(n)}\big) + d_Y\big(L^{(n)}, L\big) < \frac{\varepsilon}{3} + \frac{\varepsilon}{3} + \frac{\varepsilon}{3}\Big) \\
        \implies & \exists\ \delta \in \mathbf{R}^+ : \Big(\forall\ x \in X, d_X(x, x_0) < \delta \implies d_Y\big(f(x), L\big) < \varepsilon\Big).
    \end{align*}
    Since \(\varepsilon\) is arbitrary, by Definition \ref{3.1.1} we know that \(d_Y - \lim_{x \to x_0 ; x \in E} f(x) = L\).
\end{proof}

\begin{proposition}\label{3.3.4}
    Let \((f^{(n)})_{n = 1}^\infty\) be a sequence of continuous functions from one metric space \((X, d_X)\) to another \((Y, d_Y)\), and suppose that this sequence converges uniformly to another function \(f : X \to Y\).
    Let \(x^{(n)}\) be a sequence of points in \(X\) which converge to some limit \(x\).
    Then \(f^{(n)}(x^{(n)})\) converges (in \(Y\)) to \(f(x)\).
\end{proposition}

\begin{proof}
    Let \(x_0 \in X\).
    Suppose that \((x^{(n)})_{n = 1}^\infty\) is a sequence in \(X\) such that
    \[
        \lim_{n \to \infty} d_X(x^{(n)}, x_0) = 0.
    \]
    By Theorem \ref{3.3.1} we know that \(f\) is continuous at \(x_0\) from \((X, d_X)\) to \((Y, d_Y)\).
    Thus by Theorem \ref{2.1.4}(a)(b) we have
    \[
        \lim_{n \to \infty} d_Y\big(f(x^{(n)}), f(x_0)\big) = 0
    \]
    and by Definition \ref{1.1.14} we have
    \[
        \forall\ \varepsilon \in \mathbf{R}^+, \exists\ N_1 \in \mathbf{Z}^+ : \forall\ n \geq N_1, d_Y\big(f(x^{(n)}), f(x_0)\big) < \frac{\varepsilon}{2}.
    \]
    Now we choose one pair of \(\varepsilon\) and \(N_1\).
    Since \((f^{(n)})_{n = 1}^\infty\) converges uniformly to \(f\) on \(X\) with respect to \(d_Y\), by Definition \ref{3.2.7} we have
    \begin{align*}
                 & \exists\ N_2 \in \mathbf{Z}^+ : \forall\ n \geq N_2, \forall\ x \in X, d_Y\big(f^{(n)}(x), f(x)\big) < \frac{\varepsilon}{2}                                           \\
        \implies & \exists\ N_2 \in \mathbf{Z}^+ : \forall\ n \geq N_2, d_Y\big(f^{(n)}(x^{(n)}), f(x^{(n)})\big) < \frac{\varepsilon}{2}                                                 \\
        \implies & \exists\ N = \max(N_1, N_2) : \forall\ n \geq N,                                                                                                                       \\
                 & d_Y\big(f^{(n)}(x^{(n)}), f(x_0)\big) \leq d_Y\big(f^{(n)}(x^{(n)}), f(x^{(n)})\big) + d_Y\big(f(x^{(n)}), f(x_0)\big) < \frac{\varepsilon}{2} + \frac{\varepsilon}{2} \\
        \implies & \exists\ N = \max(N_1, N_2) : \forall\ n \geq N, d_Y\big(f^{(n)}(x^{(n)}), f(x_0)\big) < \varepsilon.
    \end{align*}
    Since \(\varepsilon\) is arbitrary, by Definition \ref{1.1.14} we know that
    \[
        \lim_{n \to \infty} d_Y\big(f^{(n)}(x^{(n)}), f(x_0)\big) = 0.
    \]
    Since \(x_0\) is arbitrary, we conclude that for any \(x_0 \in X\), if \((x^{(n)})_{n = 1}^\infty\) is a sequence in \(X\) such that
    \[
        \lim_{n \to \infty} d_X(x^{(n)}, x_0) = 0,
    \]
    then we have
    \[
        \lim_{n \to \infty} d_Y\big(f^{(n)}(x^{(n)}), f(x_0)\big) = 0.
    \]
\end{proof}

\begin{definition}[Bounded functions]\label{3.3.5}
    A function \(f : X \to Y\) from one metric space \((X, d_X)\) to another \((Y, d_Y)\) is \emph{bounded} if \(f(X)\) is a bounded set, i.e., there exists a ball \(B_{(Y, d_Y)}(y_0, R)\) in \(Y\) such that \(f(x) \in B_{(Y, d_Y)}(y_0, R)\) for all \(x \in X\).
\end{definition}

\begin{proposition}[Uniform limits preserve boundedness]\label{3.3.6}
    Let \((f^{(n)})_{n = 1}^\infty\) be a sequence of functions from one metric space \((X, d_X)\) to another \((Y, d_Y)\), and suppose that this sequence converges uniformly to another function \(f : X \to Y\).
    If the functions \(f^{(n)}\) are bounded on \(X\) for each \(n\), then the limiting function \(f\) is also bounded on \(X\).
\end{proposition}

\begin{proof}
    Since \(f^{(n)}\) is bounded in \((Y, d_Y)\) for each \(n \in \mathbf{Z}^+\), by Definition \ref{3.3.5} and Definition \ref{1.5.3} we have
    \begin{align*}
                 & \forall\ n \in \mathbf{Z}^+, \forall\ y \in Y, \exists\ \varepsilon \in \mathbf{R}^+ : f^{(n)}(X) \subseteq B_{(Y, d_Y)}(y, \varepsilon)           \\
        \implies & \forall\ n \in \mathbf{Z}^+, \forall\ y \in Y, \exists\ \varepsilon \in \mathbf{R}^+ : \forall\ x \in X, d_Y\big(f^{(n)}(x), y\big) < \varepsilon.
    \end{align*}
    Now we choose \(y\) and \(\varepsilon\) for each \(n \in \mathbf{Z}^+\) and we denote them as \(y^{(n)}\) and \(\varepsilon^{(n)}\).
    Since \((f^{(n)})_{n = 1}^\infty\) converges uniformly to \(f\) on \(X\) with respect to \(d_Y\), by Definition \ref{3.2.7} we have
    \begin{align*}
                 & \exists\ N \in \mathbf{Z}^+ : \forall\ n \geq N, \forall\ x \in X, d_Y\big(f^{(n)}(x), f(x)\big) < 1                     \\
                 & \exists\ N \in \mathbf{Z}^+ : \forall\ x \in X, d_Y\big(f^{(N)}(x), f(x)\big) < 1                                        \\
        \implies & \exists\ N \in \mathbf{Z}^+ : \forall\ x \in X,                                                                          \\
                 & d_Y\big(f(x), y^{(N)}\big) \leq d_Y\big(f(x), f^{(N)}(x)\big) + d_Y\big(f^{(N)}(x), y^{(N)}\big) < \varepsilon^{(N)} + 1 \\
        \implies & \exists\ N \in \mathbf{Z}^+ : \forall\ x \in X, d_Y\big(f(x), y^{(N)}\big) < \varepsilon^{(N)} + 1                       \\
        \implies & \exists\ N \in \mathbf{Z}^+ : f(X) \subseteq B_{(Y, d_Y)}(y^{(N)}, \varepsilon^{(N)} + 1).
    \end{align*}
    Since \(y^{(N)}\) is arbitrary, we have
    \[
        \forall\ y \in Y, \exists\ \varepsilon \in \mathbf{R}^+ : f(X) \subseteq B_{(Y, d_Y)}(y, \varepsilon)
    \]
    and by Definition \ref{1.5.3} and Definition \ref{3.3.5} \(f\) is bounded in \((Y, d_Y)\).
\end{proof}

\begin{remark}\label{3.3.7}
    The above propositions sound very reasonable, but one should caution that it only works if one assumes uniform convergence;
    pointwise convergence is not enough.
\end{remark}

\exercisesection

\begin{exercise}\label{ex 3.3.1}
    Prove Theorem \ref{3.3.1}.
    Explain briefly why your proof requires uniform convergence, and why pointwise convergence would not suffice.
\end{exercise}

\begin{proof}
    See Theorem \ref{3.3.1}.
    Without uniform convergence we cannot make \(f^{(n)}(x)\) and \(f(x)\) arbitrary close.
\end{proof}

\begin{exercise}\label{ex 3.3.2}
    Prove Proposition \ref{3.3.3}.
\end{exercise}

\begin{proof}
    See Proposition \ref{3.3.3}.
\end{proof}

\begin{exercise}\label{ex 3.3.3}
    Compare Proposition \ref{3.3.3} with Example 1.2.8 in Analysis I.
    Can you now explain why the interchange of limits in Example 1.2.8 in Analysis I led to a false statement, whereas the interchange of limits in Proposition \ref{3.3.3} is justified?
\end{exercise}

\begin{exercise}\label{ex 3.3.4}
    Prove Proposition \ref{3.3.4}.
\end{exercise}

\begin{proof}
    See Proposition \ref{3.3.4}.
\end{proof}

\begin{exercise}\label{ex 3.3.5}
    Give an example to show that Proposition \ref{3.3.4} fails if the phrase ``converges uniformly'' is replaced by ``converges pointwise''.
\end{exercise}

\begin{exercise}\label{ex 3.3.6}
    Prove Proposition \ref{3.3.6}.
\end{exercise}

\begin{proof}
    See Proposition \ref{3.3.6}.
\end{proof}

\begin{exercise}\label{ex 3.3.7}
    Give an example to show that Proposition \ref{3.3.6} fails if the phrase ``converges uniformly'' is replaced by ``converges pointwise''.
\end{exercise}

\begin{exercise}\label{ex 3.3.8}
    Let \((X, d)\) be a metric space, and for every positive integer \(n\), let \(f_n : X \to \mathbf{R}\) and \(g_n : X \to \mathbf{R}\) be functions.
    Suppose that \((f_n)_{n = 1}^\infty\) converges uniformly to another function \(f : X \to \mathbf{R}\), and that \((g_n)_{n = 1}^\infty\) converges uniformly to another function \(g : X \to \mathbf{R}\).
    Suppose also that the functions \((f_n)_{n = 1}^\infty\) and \((g_n)_{n = 1}^\infty\) are uniformly bounded, i.e., there exists an \(M > 0\) such that \(\abs*{f_n(x)} \leq M\) and \(\abs*{g_n(x)} \leq M\) for all \(n \geq 1\) and \(x \in X\).
    Prove that the functions \(f_n g_n : X \to \mathbf{R}\) converge uniformly to \(fg : X \to \mathbf{R}\).
\end{exercise}
\chapter{Power series}\label{ch 4}

\section{Formal power series}\label{sec 4.1}
\section{Real analytic functions}\label{sec 4.2}

\begin{definition}[Real analytic functions]\label{4.2.1}
    Let \(E\) be a subset of \(\mathbf{R}\), and let \(f : E \to \mathbf{R}\) be a function.
    If \(a\) is an interior point of \(E\), we say that \(f\) is \emph{real analytic at \(a\)} if there exists an open interval \((a - r, a + r)\) in \(E\) for some \(r > 0\) such that there exists a power series \(\sum_{n = 0}^\infty c_n (x - a)^n\) centered at \(a\) which has a radius of convergence greater than or equal to \(r\), and which converges to \(f\) on \((a - r, a + r)\).
    If \(E\) is an open set, and \(f\) is real analytic at every point \(a\) of \(E\), we say that \(f\) is \emph{real analytic on \(E\)}.
\end{definition}

\begin{example}\label{4.2.2}
    Consider the function \(f : \mathbf{R} \setminus \{1\} \to \mathbf{R}\) defined by \(f(x) \coloneqq \frac{1}{1 - x}\).
    This function is real analytic at \(0\) because we have a power series \(\sum_{n = 0}^\infty x^n\) centred at \(0\) which converges to \(\frac{1}{1 - x} = f(x)\) on the interval \((-1, 1)\).
    This function is also real analytic at \(2\) because we have a power series \(\sum_{n = 0}^\infty (-1)^{n + 1} (x - 2)^n\) which converges to \(\frac{-1}{1 - \big(-(x - 2)\big)} = \frac{1}{1 - x} = f(x)\) on the interval \((1, 3)\)
    (why? use Lemma 7.3.3 in Analysis I).
    In fact this function is real analytic on all of \(\mathbf{R} \setminus \{1\}\);
    see Exercise \ref{ex 4.2.2}.
\end{example}

\begin{remark}\label{4.2.3}
    The notion of being real analytic is closely related to another notion, that of being \emph{complex analytic}, but this is a topic for complex analysis, and will not be discussed here.
\end{remark}

\begin{note}
    From Theorem \ref{4.1.6}(c) and (d) we see that if \(f\) is real analytic at a point \(a\), then \(f\) is both continuous and differentiable on \((a - r, a + r)\) for some \(r > 0\).
\end{note}

\begin{definition}[\(k\)-times differentiability]\label{4.2.4}
    Let \(E\) be a subset of \(\mathbf{R}\) with the property that every element of \(E\) is a limit point of \(E\).
    We say a function \(f : E \to \mathbf{R}\) is \emph{once differentiable on \(E\)} iff it is differentiable, in particular \(f': E \to \mathbf{R}\) is also a function on \(E\).
    More generally, for any \(k \geq 2\) we say that \(f : E \to \mathbf{R}\) is \emph{\(k\) times differentiable on \(E\)}, or just \emph{\(k\) times differentiable}, iff \(f\) is differentiable, and \(f'\) is \(k - 1\) times differentiable.
    If \(f\) is \(k\) times differentiable, we define the \(k^{\text{th}}\) derivative \(f^{(k)} : E \to \mathbf{R}\) by the recursive rule \(f^{(1)} \coloneqq f'\), and \(f^{(k)} = (f^{(k - 1)})'\) for all \(k \geq 2\).
    We also define \(f^{(0)} \coloneqq f\) (this is \(f\) differentiated \(0\) times), and we allow every function to be zero times differentiable (since clearly \(f^{(0)}\) exists for every \(f\)).
    A function is said to be \emph{infinitely differentiable} (or \emph{smooth}) iff it is \(k\) times differentiable for every \(k \geq 0\).
\end{definition}

\setcounter{theorem}{5}
\begin{proposition}[Real analytic functions are \(k\)-times differentiable]\label{4.2.6}
    Let \(E\) be a subset of \(\mathbf{R}\), let \(a\) be an interior point of \(E\), and and let \(f\) be a function which is real analytic at \(a\), thus there is an \(r > 0\) for which we have the power series expansion
    \[
        f(x) = \sum_{n = 0}^\infty c_n (x - a)^n
    \]
    for all \(x \in (a - r, a + r)\).
    Then for every \(k \geq 0\), the function \(f\) is \(k\)-times differentiable on \((a - r, a + r)\), and for each \(k \geq 0\) the \(k^{\text{th}}\) derivative is given by
    \begin{align*}
        f^{(k)}(x) & = \sum_{n = 0}^\infty c_{n + k} (n + 1) (n + 2) \dots (n + k) (x - a)^n \\
                   & = \sum_{n = 0}^\infty c_{n + k} \frac{(n + k)!}{n!} (x - a)^n
    \end{align*}
    for all \(x \in (a - r, a + r)\).
\end{proposition}

\begin{corollary}[Real analytic functions are infinitely differentiable]\label{4.2.7}
    Let \(E\) be an open subset of \(\mathbf{R}\), and let \(f : E \to \mathbf{R}\) be a real analytic function on \(E\).
    Then \(f\) is infinitely differentiable on \(E\).
    Also, all derivatives of \(f\) are also real analytic on \(E\).
\end{corollary}

\begin{proof}
    For every point \(a \in E\) and \(k \geq 0\), we know from Proposition \ref{4.2.6} that \(f\) is \(k\)-times differentiable at \(a\)
    (we will have to apply Exercise 10.1.1 in Analysis I \(k\) times here, why?).
    Thus \(f\) is \(k\)-times differentiable on \(E\) for every \(k \geq 0\) and is hence infinitely differentiable.
    Also, from Proposition \ref{4.2.6} we see that each derivative \(f^{(k)}\) of \(f\) has a convergent power series expansion at every \(x \in E\) and thus \(f^{(k)}\) is real analytic.
\end{proof}

\setcounter{theorem}{8}
\begin{remark}\label{4.2.9}
    The converse statement to Corollary \ref{4.2.7} is not true;
    there are infinitely differentiable functions which are not real analytic.
\end{remark}

\begin{note}
    Proposition \ref{4.2.6} has an important corollary, due to Brook Taylor (1685 -- 1731).
\end{note}

\begin{corollary}[Taylor's formula]\label{4.2.10}
    Let \(E\) be a subset of \(\mathbf{R}\), let \(a\) be an interior point of \(E\), and let \(f : E \to \mathbf{R}\) be a function which is real analytic at \(a\) and has the power series expansion
    \[
        f(x) = \sum_{n = 0}^\infty c_n (x - a)^n
    \]
    for all \(x \in (a - r, a + r)\) and some \(r > 0\).
    Then for any integer \(k \geq 0\), we have
    \[
        f^{(k)}(a) = k! c_k,
    \]
    where \(k! \coloneqq 1 \times 2 \times \dots \times k\)
    (and we adopt the convention that \(0! = 1\)).
    In particular, we have Taylor's formula
    \[
        f(x) = \sum_{n = 0}^\infty \frac{f^{(n)}(a)}{n!} (x - a)^n
    \]
    for all \(x\) in \((a - r, a + r)\).
\end{corollary}

\begin{note}
    The power series \(\sum_{n = 0}^\infty \frac{f^{(n)}(a)}{n!} (x - a)^n\) is sometimes called the \emph{Taylor series} of \(f\) around \(a\).
    Taylor's formula thus asserts that if a function is real analytic, then it is equal to its Taylor series.
\end{note}

\begin{remark}\label{4.2.11}
    Note that Taylor's formula only works for functions which are real analytic;
    there are examples of functions which are infinitely differentiable but for which Taylor's theorem fails.
\end{remark}

\begin{corollary}[Uniqueness of power series]\label{4.2.12}
    Let \(E\) be a subset of \(\mathbf{R}\), let \(a\) be an interior point of \(E\), and let \(f : E \to \mathbf{R}\) be a function which is real analytic at \(a\).
    Suppose that \(f\) has two power series expansions
    \[
        f(x) = \sum_{n = 0}^\infty c_n (x - a)^n
    \]
    and
    \[
        f(x) = \sum_{n = 0}^\infty d_n (x - a)^n
    \]
    centered at \(a\), each with a non-zero radius of convergence.
    Then \(c_n = d_n\) for all \(n \geq 0\).
\end{corollary}

\begin{proof}
    By Corollary \ref{4.2.10}, we have \(f^{(k)}(a) = k! c_k\) for all \(k \geq 0\).
    But we also have \(f^{(k)}(a) = k! d_k\), by similar reasoning.
    Since \(k!\) is never zero, we can cancel it and obtain \(c_k = d_k\) for all \(k \geq 0\), as desired.
\end{proof}

\begin{remark}\label{4.2.13}
    While a real analytic function has a unique power series around any given point, it can certainly have different power series at different points.
    For instance, the function \(f(x) \coloneqq \frac{1}{1 - x}\), defined on \(\mathbf{R} \setminus \{1\}\), has the power series
    \[
        f(x) \coloneqq \sum_{n = 0}^\infty x^n
    \]
    around \(0\), on the interval \((-1, 1)\), but also has the power series
    \begin{align*}
        f(x) & = \frac{1}{1 - x}                                                  \\
             & = \frac{2}{1 - 2(x - \frac{1}{2})}                                 \\
             & = \sum_{n = 0}^\infty 2 \bigg(2\bigg(x - \frac{1}{2}\bigg)\bigg)^n \\
             & = \sum_{n = 0}^\infty 2^{n + 1} \bigg(x - \frac{1}{2}\bigg)^n
    \end{align*}
    around \(1 / 2\), on the interval \((0, 1)\)
    (note that the above power series has a radius of convergence of \(1 / 2\), thanks to the root test).
\end{remark}

\exercisesection

\begin{exercise}\label{ex 4.2.1}
    Let \(n \geq 0\) be an integer, let \(c, a\) be real numbers, and let \(f\) be the function \(f(x) \coloneqq c (x - a)^n\).
    Show that \(f\) is infinitely differentiable, and that \(f^{(k)}(x) = c \frac{n!}{(n - k)!} (x - a)^{n - k}\) for all integers \(0 \leq k \leq n\).
    What happens when \(k > n\)?
\end{exercise}

\begin{proof}
    For each \(n \in \mathbf{N}\), let \(P(n)\) be the statement ``If \(f(x) = c (x - a)^n\), then \(f^{(k)}(x) = c \frac{n!}{(n - k)!} (x - a)^{n - k}\) for all \(0 \leq k \leq n\)''.
    We use induction on \(n\) to show that \(P(n)\) is true for all \(n \in \mathbf{N}\).
    For \(n = 0\), we have \(0 \leq k \leq 0 \implies k = 0\).
    By Definition \ref{4.2.4} we have
    \[
        f^{(0)} = f = c (x - a)^0 = c = c \frac{0!}{(0 - 0)!} (x - a)^{0 - 0}
    \]
    and thus the base case holds.
    Suppose inductive that \(P(n)\) is true for some \(n \geq 0\).
    Then we want to show that \(P(n + 1)\) is true.
    Let \(f(x) = c (x - a)^{n + 1}\).
    Then we have
    \[
        f'(x) = c (n + 1) (x - a)^n.
    \]
    By induction hypothesis we know that
    \begin{align*}
                 & \frac{f'(x)}{n + 1} = c (x - a)^n                                                                                                            \\
        \implies & \forall\ 0 \leq k \leq n, \bigg(\frac{f'(x)}{n + 1}\bigg)^{(k)} = c \frac{n!}{(n - k)!} (x - a)^{n - k}                                      \\
        \implies & \forall\ 0 \leq k \leq n, \big(f'(x)\big)^{(k)} = c \frac{(n + 1)!}{(n - k)!} (x - a)^{n - k}                                                \\
        \implies & \forall\ 0 \leq k \leq n, f(x)^{(k + 1)} = c \frac{(n + 1)!}{(n - k)!} (x - a)^{n - k}                  & \text{(by Definition \ref{4.2.4})} \\
        \implies & \forall\ 1 \leq k \leq n + 1,                                                                                                                \\
                 & f(x)^{\big((k - 1) + 1\big)} = c \frac{(n + 1)!}{\big(n - (k - 1)\big)!} (x - a)^{n - (k - 1)}                                               \\
        \implies & \forall\ 1 \leq k \leq n + 1, f(x)^{(k)} = c \frac{(n + 1)!}{(n + 1 - k)!} (x - a)^{n + 1 - k}
    \end{align*}
    and we know that
    \[
        f(x)^{(0)} = f(x) = c (x - a)^{n + 1} = c \frac{(n + 1)!}{(n + 1 - 0)!} (x - a)^{n + 1 - 0}.
    \]
    Thus we have
    \[
        \forall\ 0 \leq k \leq n + 1, f(x)^{(k)} = c \frac{(n + 1)!}{(n + 1 - k)!} (x - a)^{n + 1 - k}
    \]
    and this closes the induction.

    Now let \(n \in \mathbf{N}\) and let \(f(x) = c (x - a)^n\).
    From the proof above we know that
    \[
        f^{(n)}(x) = c \frac{n!}{(n - n)!} (x - a)^{n - n} = c n!
    \]
    is a constant function.
    Thus we have
    \[
        \forall\ k > n, f^{(k)}(x) = 0
    \]
    and by Definition \ref{4.2.4} \(f\) is infinitely differentiable.
\end{proof}

\begin{exercise}\label{ex 4.2.2}
    Show that the function \(f\) defined in Example \ref{4.2.2} is real analytic on all of \(\mathbf{R} \setminus \{1\}\).
\end{exercise}

\begin{proof}
    Let \(a \in \mathbf{R} \setminus \{1\}\), let \(r = \abs*{1 - a}\), let \(x \in (a - r, a + r)\) and let \(c_n = (\frac{1}{1 - a})^{n + 1}\) for all \(n \in \mathbf{N}\).
    Then we have
    \begin{align*}
                 & a - r < x < a + r              \\
        \implies & \abs*{x - a} < r               \\
        \implies & \abs*{\frac{x - a}{1 - a}} < 1
    \end{align*}
    and by Lemma 7.3.3 in Analysis I we know that
    \begin{align*}
        \sum_{n = 0}^\infty c_n (x - a)^n & = \sum_{n = 0}^\infty \bigg(\frac{1}{1 - a}\bigg)^{n + 1} (x - a)^n     \\
                                          & = \frac{1}{1 - a} \sum_{n = 0}^\infty \bigg(\frac{x - a}{1 - a}\bigg)^n \\
                                          & = \frac{1}{1 - a} \frac{1}{1 - \frac{x - a}{1 - a}}                     \\
                                          & = \frac{1}{1 - x}.
    \end{align*}
    Since \(x\) is arbitrary, we know that \(\sum_{n = 0}^\infty c_n (x - a)^n\) converges to \(f\) on \((a - r, a + r)\).
    By Definition \ref{4.2.1} we know that \(f\) is real analytic at \(a\).
    Since \(a\) is arbitrary, by Definition \ref{4.2.1} we know that \(f\) is real analysis on \(\mathbf{R} \setminus \{1\}\).
\end{proof}

\begin{exercise}\label{ex 4.2.3}
    Prove Proposition \ref{4.2.6}.
\end{exercise}

\begin{proof}
    See Proposition \ref{4.2.6}.
\end{proof}

\begin{exercise}\label{ex 4.2.4}
    Use Proposition \ref{4.2.6} and Exercise \ref{ex 4.2.1} to prove Corollary \ref{4.2.10}.
\end{exercise}

\begin{proof}
    See Corollary \ref{4.2.10}.
\end{proof}

\begin{exercise}\label{ex 4.2.5}
    Let \(a, b\) be real numbers, and let \(n \geq 0\) be an integer.
    Prove the identity
    \[
        (x - a)^n = \sum_{m = 0}^n \frac{n!}{m! (n - m)!} (b - a)^{n - m} (x - b)^m
    \]
    or any real number \(x\).
    Explain why this identity is consistent with Taylor's theorem and Exercise \ref{ex 4.2.1}.
    (Note however that Taylor's theorem cannot be rigorously applied until one verifies Exercise \ref{ex 4.2.6} below.)
\end{exercise}

\begin{exercise}\label{ex 4.2.6}
    Using Exercise \ref{ex 4.2.5}, show that every polynomial \(P(x)\) of one variable is real analytic on \(\mathbf{R}\).
\end{exercise}

\begin{exercise}\label{ex 4.2.7}
    Let \(m \geq 0\) be a positive integer, and let \(0 < x < r\) be real numbers.
    Use Lemma 7.3.3 in Analysis I to establish the identity
    \[
        \frac{r}{r - x} = \sum_{n = 0}^\infty x^n r^{-n}
    \]
    for all \(x \in (-r, r)\).
    Using Proposition \ref{4.2.6}, conclude the identity
    \[
        \frac{r}{(r - x)^{m + 1}} = \sum_{n = m}^\infty \frac{n!}{m! (n - m)!} x^{n - m} r^{-n}
    \]
    for all integers \(m \geq 0\) and \(x \in (-r, r)\).
    Also explain why the series on the right-hand side is absolutely convergent.
\end{exercise}

\begin{exercise}\label{ex 4.2.8}
    Let \(E\) be a subset of \(\mathbf{R}\), let \(a\) be an interior point of \(E\), and let \(f : E \to \mathbf{R}\) be a function which is real analytic at \(a\), and has a power series expansion
    \[
        f(x) = \sum_{n = 0}^\infty c_n (x - a)^n
    \]
    at \(a\) which converges on the interval \((a - r, a + r)\).
    Let \((b - s, b + s)\) be any sub-interval of \((a - r, a + r)\) for some \(s > 0\).
    \begin{enumerate}
        \item Prove that \(\abs*{a - b} \leq r - s\), so in particular \(\abs*{a - b} < r\).
        \item Show that for every \(0 < \varepsilon < r\), there exists a \(C > 0\) such that \(\abs*{c_n} \leq C(r - \varepsilon)^{-n}\) for all integers \(n \geq 0\).
        \item Show that the numbers \(d_0, d_1, \dots\) given by the formula
              \[
                  d_m \coloneqq \sum_{n = m}^\infty \frac{n!}{m! (n - m)!} (b - a)^{n - m} c_n \text{ for all integers } m \geq 0
              \]
              are well-defined, in the sense that the above series is absolutely convergent.
        \item Show that for every \(0 < \varepsilon < s\) there exists a \(C > 0\) such that
              \[
                  \abs*{d_m} \leq C(s - \varepsilon)^{-m}
              \]
              for all integers \(m \geq 0\).
        \item Show that the power series \(\sum_{m = 0}^\infty d_m (x - b)^m\) is absolutely convergent for \(x \in (b - s, b + s)\) and converges to \(f(x)\).
        \item Conclude that \(f\) is real analytic at every point in \((a - r, a + r)\).
    \end{enumerate}
\end{exercise}
\chapter{Fourier series}\label{ch 5}

\begin{note}
    Power series are already immensely useful, especially when dealing with special functions such as the exponential and trigonometric functions discussed earlier.
    However, there are some circumstances where power series are not so useful, because one has to deal with functions (e.g., \(\sqrt{x}\)) which are not real analytic, and so do not have power series.
\end{note}

\begin{note}
    Fortunately, there is another type of series expansion, known as \emph{Fourier series}, which is also a very powerful tool in analysis
    (though used for slightly different purposes).
    Instead of analyzing compactly supported functions, it instead analyzes \emph{periodic functions};
    instead of decomposing into polynomials, it decomposes into \emph{trigonometric polynomials}.
    Roughly speaking, the theory of Fourier series asserts that just about every periodic function can be decomposed as an (infinite) sum of sines and cosines.
\end{note}

\begin{remark}\label{5.0.1}
    Jean-Baptiste Fourier (1768 -- 1830) was, among other things, an administrator accompanying Napoleon on his invasion of Egypt, and then a Prefect in France during Napoleons reign.
    After the Napoleonic wars, he returned to mathematics.
    He introduced Fourier series in an important 1807 paper in which he used them to solve what is now known as the heat equation.
    At the time, the claim that every periodic function could be expressed as a sum of sines and cosines was extremely controversial, even such leading mathematicians as Euler declared that it was impossible.
    Nevertheless, Fourier managed to show that this was indeed the case, although the proof was not completely rigorous and was not totally accepted for almost another hundred years.
\end{remark}

\begin{note}
    For instance, the convergence of Fourier series is usually not uniform (i.e., not in the \(L^\infty\) metric), but instead we have convergence in a different metric, the \(L^2\)-metric.
    We will need to use complex numbers heavily in our theory, while they played only a tangential rôle in power series.
\end{note}

\begin{note}
    The theory of Fourier series (and of related topics such as Fourier integrals and the Laplace transform) is vast, and deserves an entire course in itself.
    It has many, many applications, most directly to differential equations, signal processing, electrical engineering, physics, and analysis, but also to algebra and number theory.
\end{note}

\section{Periodic functions}\label{sec 5.1}
\section{Inner products on periodic functions}\label{sec 5.2}
\chapter{Several variable differential calculus}\label{ch 6}

%------------------------------------------------------------------------------
% Back matters.
%------------------------------------------------------------------------------

\backmatter

\end{document}
